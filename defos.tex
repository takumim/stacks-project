\IfFileExists{stacks-project.cls}{%
\documentclass{stacks-project}
}{%
\documentclass{amsart}
}

% For dealing with references we use the comment environment
\usepackage{verbatim}
\newenvironment{reference}{\comment}{\endcomment}
%\newenvironment{reference}{}{}
\newenvironment{slogan}{\comment}{\endcomment}
\newenvironment{history}{\comment}{\endcomment}

% For commutative diagrams we use Xy-pic
\usepackage[all]{xy}

% We use 2cell for 2-commutative diagrams.
\xyoption{2cell}
\UseAllTwocells

% We use multicol for the list of chapters between chapters
\usepackage{multicol}

% This is generall recommended for better output
\usepackage[T1]{fontenc}

% For cross-file-references
\usepackage{xr-hyper}

% Package for hypertext links:
\usepackage{hyperref}

% For any local file, say "hello.tex" you want to link to please
% use \externaldocument[hello-]{hello}
\externaldocument[introduction-]{introduction}
\externaldocument[conventions-]{conventions}
\externaldocument[sets-]{sets}
\externaldocument[categories-]{categories}
\externaldocument[topology-]{topology}
\externaldocument[sheaves-]{sheaves}
\externaldocument[sites-]{sites}
\externaldocument[stacks-]{stacks}
\externaldocument[fields-]{fields}
\externaldocument[algebra-]{algebra}
\externaldocument[brauer-]{brauer}
\externaldocument[homology-]{homology}
\externaldocument[derived-]{derived}
\externaldocument[simplicial-]{simplicial}
\externaldocument[more-algebra-]{more-algebra}
\externaldocument[smoothing-]{smoothing}
\externaldocument[modules-]{modules}
\externaldocument[sites-modules-]{sites-modules}
\externaldocument[injectives-]{injectives}
\externaldocument[cohomology-]{cohomology}
\externaldocument[sites-cohomology-]{sites-cohomology}
\externaldocument[dga-]{dga}
\externaldocument[dpa-]{dpa}
\externaldocument[hypercovering-]{hypercovering}
\externaldocument[schemes-]{schemes}
\externaldocument[constructions-]{constructions}
\externaldocument[properties-]{properties}
\externaldocument[morphisms-]{morphisms}
\externaldocument[coherent-]{coherent}
\externaldocument[divisors-]{divisors}
\externaldocument[limits-]{limits}
\externaldocument[varieties-]{varieties}
\externaldocument[topologies-]{topologies}
\externaldocument[descent-]{descent}
\externaldocument[perfect-]{perfect}
\externaldocument[more-morphisms-]{more-morphisms}
\externaldocument[flat-]{flat}
\externaldocument[groupoids-]{groupoids}
\externaldocument[more-groupoids-]{more-groupoids}
\externaldocument[etale-]{etale}
\externaldocument[chow-]{chow}
\externaldocument[intersection-]{intersection}
\externaldocument[pic-]{pic}
\externaldocument[adequate-]{adequate}
\externaldocument[dualizing-]{dualizing}
\externaldocument[duality-]{duality}
\externaldocument[discriminant-]{discriminant}
\externaldocument[local-cohomology-]{local-cohomology}
\externaldocument[curves-]{curves}
\externaldocument[resolve-]{resolve}
\externaldocument[models-]{models}
\externaldocument[pione-]{pione}
\externaldocument[etale-cohomology-]{etale-cohomology}
\externaldocument[proetale-]{proetale}
\externaldocument[crystalline-]{crystalline}
\externaldocument[spaces-]{spaces}
\externaldocument[spaces-properties-]{spaces-properties}
\externaldocument[spaces-morphisms-]{spaces-morphisms}
\externaldocument[decent-spaces-]{decent-spaces}
\externaldocument[spaces-cohomology-]{spaces-cohomology}
\externaldocument[spaces-limits-]{spaces-limits}
\externaldocument[spaces-divisors-]{spaces-divisors}
\externaldocument[spaces-over-fields-]{spaces-over-fields}
\externaldocument[spaces-topologies-]{spaces-topologies}
\externaldocument[spaces-descent-]{spaces-descent}
\externaldocument[spaces-perfect-]{spaces-perfect}
\externaldocument[spaces-more-morphisms-]{spaces-more-morphisms}
\externaldocument[spaces-flat-]{spaces-flat}
\externaldocument[spaces-groupoids-]{spaces-groupoids}
\externaldocument[spaces-more-groupoids-]{spaces-more-groupoids}
\externaldocument[bootstrap-]{bootstrap}
\externaldocument[spaces-pushouts-]{spaces-pushouts}
\externaldocument[spaces-chow-]{spaces-chow}
\externaldocument[groupoids-quotients-]{groupoids-quotients}
\externaldocument[spaces-more-cohomology-]{spaces-more-cohomology}
\externaldocument[spaces-simplicial-]{spaces-simplicial}
\externaldocument[spaces-duality-]{spaces-duality}
\externaldocument[formal-spaces-]{formal-spaces}
\externaldocument[restricted-]{restricted}
\externaldocument[spaces-resolve-]{spaces-resolve}
\externaldocument[formal-defos-]{formal-defos}
\externaldocument[defos-]{defos}
\externaldocument[cotangent-]{cotangent}
\externaldocument[examples-defos-]{examples-defos}
\externaldocument[algebraic-]{algebraic}
\externaldocument[examples-stacks-]{examples-stacks}
\externaldocument[stacks-sheaves-]{stacks-sheaves}
\externaldocument[criteria-]{criteria}
\externaldocument[artin-]{artin}
\externaldocument[quot-]{quot}
\externaldocument[stacks-properties-]{stacks-properties}
\externaldocument[stacks-morphisms-]{stacks-morphisms}
\externaldocument[stacks-limits-]{stacks-limits}
\externaldocument[stacks-cohomology-]{stacks-cohomology}
\externaldocument[stacks-perfect-]{stacks-perfect}
\externaldocument[stacks-introduction-]{stacks-introduction}
\externaldocument[stacks-more-morphisms-]{stacks-more-morphisms}
\externaldocument[stacks-geometry-]{stacks-geometry}
\externaldocument[moduli-]{moduli}
\externaldocument[moduli-curves-]{moduli-curves}
\externaldocument[examples-]{examples}
\externaldocument[exercises-]{exercises}
\externaldocument[guide-]{guide}
\externaldocument[desirables-]{desirables}
\externaldocument[coding-]{coding}
\externaldocument[obsolete-]{obsolete}
\externaldocument[fdl-]{fdl}
\externaldocument[index-]{index}

% Theorem environments.
%
\theoremstyle{plain}
\newtheorem{theorem}[subsection]{Theorem}
\newtheorem{proposition}[subsection]{Proposition}
\newtheorem{lemma}[subsection]{Lemma}

\theoremstyle{definition}
\newtheorem{definition}[subsection]{Definition}
\newtheorem{example}[subsection]{Example}
\newtheorem{exercise}[subsection]{Exercise}
\newtheorem{situation}[subsection]{Situation}

\theoremstyle{remark}
\newtheorem{remark}[subsection]{Remark}
\newtheorem{remarks}[subsection]{Remarks}

\numberwithin{equation}{subsection}

% Macros
%
\def\lim{\mathop{\mathrm{lim}}\nolimits}
\def\colim{\mathop{\mathrm{colim}}\nolimits}
\def\Spec{\mathop{\mathrm{Spec}}}
\def\Hom{\mathop{\mathrm{Hom}}\nolimits}
\def\Ext{\mathop{\mathrm{Ext}}\nolimits}
\def\SheafHom{\mathop{\mathcal{H}\!\mathit{om}}\nolimits}
\def\SheafExt{\mathop{\mathcal{E}\!\mathit{xt}}\nolimits}
\def\Sch{\mathit{Sch}}
\def\Mor{\mathop{Mor}\nolimits}
\def\Ob{\mathop{\mathrm{Ob}}\nolimits}
\def\Sh{\mathop{\mathit{Sh}}\nolimits}
\def\NL{\mathop{N\!L}\nolimits}
\def\proetale{{pro\text{-}\acute{e}tale}}
\def\etale{{\acute{e}tale}}
\def\QCoh{\mathit{QCoh}}
\def\Ker{\mathop{\mathrm{Ker}}}
\def\Im{\mathop{\mathrm{Im}}}
\def\Coker{\mathop{\mathrm{Coker}}}
\def\Coim{\mathop{\mathrm{Coim}}}

%
% Macros for moduli stacks/spaces
%
\def\QCohstack{\mathcal{QC}\!\mathit{oh}}
\def\Cohstack{\mathcal{C}\!\mathit{oh}}
\def\Spacesstack{\mathcal{S}\!\mathit{paces}}
\def\Quotfunctor{\mathrm{Quot}}
\def\Hilbfunctor{\mathrm{Hilb}}
\def\Curvesstack{\mathcal{C}\!\mathit{urves}}
\def\Polarizedstack{\mathcal{P}\!\mathit{olarized}}
\def\Complexesstack{\mathcal{C}\!\mathit{omplexes}}
% \Pic is the operator that assigns to X its picard group, usage \Pic(X)
% \Picardstack_{X/B} denotes the Picard stack of X over B
% \Picardfunctor_{X/B} denotes the Picard functor of X over B
\def\Pic{\mathop{\mathrm{Pic}}\nolimits}
\def\Picardstack{\mathcal{P}\!\mathit{ic}}
\def\Picardfunctor{\mathrm{Pic}}
\def\Deformationcategory{\mathcal{D}\!\mathit{ef}}


% OK, start here.
%
\begin{document}

\title{Deformation Theory}


\maketitle

\phantomsection
\label{section-phantom}

\tableofcontents

\section{Introduction}
\label{section-introduction}

\noindent
The goal of this chapter is to give a (relatively) gentle introduction to
deformation theory of modules, morphisms, etc. In this chapter we deal with
those results that can be proven using the naive cotangent complex. In
the chapter on the cotangent complex we will extend these results a little
bit. The advanced reader may wish to consult the treatise by Illusie on this
subject, see \cite{cotangent}.





\section{Deformations of rings and the naive cotangent complex}
\label{section-deformations}

\noindent
In this section we use the naive cotangent complex to do a little bit
of deformation theory. We start with a surjective ring map $A' \to A$
whose kernel is an ideal $I$ of square zero. Moreover we assume
given a ring map $A \to B$, a $B$-module $N$, and an $A$-module map
$c : I \to N$. In this section we ask ourselves whether we can find
the question mark fitting into the following diagram
\begin{equation}
\label{equation-to-solve}
\vcenter{
\xymatrix{
0 \ar[r] & N \ar[r] & {?} \ar[r] & B \ar[r] & 0 \\
0 \ar[r] & I \ar[u]^c \ar[r] & A' \ar[u] \ar[r] & A \ar[u] \ar[r] & 0
}
}
\end{equation}
and moreover how unique the solution is (if it exists). More precisely,
we look for a surjection of $A'$-algebras $B' \to B$ whose kernel is
identified with $N$ such that $A' \to B'$ induces the given map $c$.
We will say $B'$ is a {\it solution} to (\ref{equation-to-solve}).

\begin{lemma}
\label{lemma-huge-diagram}
Given a commutative diagram
$$
\xymatrix{
& 0 \ar[r] & N_2 \ar[r] & B'_2 \ar[r] & B_2 \ar[r] & 0 \\
& 0 \ar[r]|\hole & I_2 \ar[u]_{c_2} \ar[r] &
A'_2 \ar[u] \ar[r]|\hole & A_2 \ar[u] \ar[r] & 0 \\
0 \ar[r] & N_1 \ar[ruu] \ar[r] & B'_1 \ar[r] & B_1 \ar[ruu] \ar[r] & 0 \\
0 \ar[r] & I_1 \ar[ruu]|\hole \ar[u]^{c_1} \ar[r] &
A'_1 \ar[ruu]|\hole \ar[u] \ar[r] & A_1 \ar[ruu]|\hole \ar[u] \ar[r] & 0
}
$$
with front and back solutions to (\ref{equation-to-solve}) we have
\begin{enumerate}
\item There exist a canonical element in
$\Ext^1_{B_1}(\NL_{B_1/A_1}, N_2)$
whose vanishing is a necessary and sufficient condition for the existence
of a ring map $B'_1 \to B'_2$ fitting into the diagram.
\item If there exists a map $B'_1 \to B'_2$ fitting into the diagram
the set of all such maps is a principal homogeneous space under
$\Hom_{B_1}(\Omega_{B_1/A_1}, N_2)$.
\end{enumerate}
\end{lemma}

\begin{proof}
Let $E = B_1$ viewed as a set.
Consider the surjection $A_1[E] \to B_1$ with kernel $J$ used
to define the naive cotangent complex by the formula
$$
\NL_{B_1/A_1} = (J/J^2 \to \Omega_{A_1[E]/A_1} \otimes_{A_1[E]} B_1)
$$
in
Algebra, Section \ref{algebra-section-netherlander}.
Since $\Omega_{A_1[E]/A_1} \otimes B_1$ is a free
$B_1$-module we have
$$
\Ext^1_{B_1}(\NL_{B_1/A_1}, N_2) =
\frac{\Hom_{B_1}(J/J^2, N_2)}
{\Hom_{B_1}(\Omega_{A_1[E]/A_1} \otimes B_1, N_2)}
$$
We will construct an obstruction in the module on the right.
Let $J' = \Ker(A'_1[E] \to B_1)$. Note that there is a surjection
$J' \to J$ whose kernel is $I_1A_1[E]$.
For every $e \in E$ denote $x_e \in A_1[E]$ the corresponding variable.
Choose a lift $y_e \in B'_1$ of the image of $x_e$ in $B_1$ and
a lift $z_e \in B'_2$ of the image of $x_e$ in $B_2$.
These choices determine $A'_1$-algebra maps
$$
A'_1[E] \to B'_1 \quad\text{and}\quad A'_1[E] \to B'_2
$$
The first of these gives a map $J' \to N_1$, $f' \mapsto f'(y_e)$
and the second gives a map $J' \to N_2$, $f' \mapsto f'(z_e)$.
A calculation shows that these maps annihilate $(J')^2$.
Because the left square of the diagram (involving $c_1$ and $c_2$)
commutes we see that these maps agree on $I_1A_1[E]$ as maps into $N_2$.
Observe that $B'_1$ is the pushout of $J' \to A'_1[B_1]$ and $J' \to N_1$. 
Thus, if the maps $J' \to N_1 \to N_2$ and $J' \to N_2$ agree, then we
obtain a map $B'_1 \to B'_2$ fitting into the diagram.
Thus we let the obstruction be the class of the map
$$
J/J^2 \to N_2,\quad f \mapsto f'(z_e) - \nu(f'(y_e))
$$
where $\nu : N_1 \to N_2$ is the given map and where $f' \in J'$
is a lift of $f$. This is well defined by our remarks above.
Note that we have the freedom
to modify our choices of $z_e$ into $z_e + \delta_{2, e}$
and $y_e$ into $y_e + \delta_{1, e}$ for some $\delta_{i, e} \in N_i$.
This will modify the map above into
$$
f \mapsto f'(z_e + \delta_{2, e}) - \nu(f'(y_e + \delta_{1, e})) =
f'(z_e) - \nu(f'(z_e)) +
\sum (\delta_{2, e} - \nu(\delta_{1, e}))\frac{\partial f}{\partial x_e}
$$
This means exactly that we are modifying the map $J/J^2 \to N_2$
by the composition $J/J^2 \to \Omega_{A_1[E]/A_1} \otimes B_1 \to N_2$
where the second map sends $\text{d}x_e$ to
$\delta_{2, e} - \nu(\delta_{1, e})$. Thus our obstruction is well defined
and is zero if and only if a lift exists.

\medskip\noindent
Part (2) comes from the observation that given two maps
$\varphi, \psi : B'_1 \to B'_2$ fitting into the diagram, then
$\varphi - \psi$ factors through a map $D : B_1 \to N_2$ which
is an $A_1$-derivation:
\begin{align*}
D(fg) & = \varphi(f'g') - \psi(f'g') \\
& =
\varphi(f')\varphi(g') - \psi(f')\psi(g') \\
& =
(\varphi(f') - \psi(f'))\varphi(g') + \psi(f')(\varphi(g') - \psi(g')) \\
& =
gD(f) + fD(g)
\end{align*}
Thus $D$ corresponds to a unique $B_1$-linear map
$\Omega_{B_1/A_1} \to N_2$. Conversely, given such a linear map
we get a derivation $D$ and given a ring map $\psi : B'_1 \to B'_2$
fitting into the diagram
the map $\psi + D$ is another ring map fitting into the diagram.
\end{proof}

\noindent
The naive cotangent complex isn't good enough to contain all information
regarding obstructions to finding solutions to (\ref{equation-to-solve}).
However, if the ring map is a local complete intersection, then the
obstruction vanishes. This is a kind of lifting result; observe that
for syntomic ring maps we have proved a rather strong lifting result in
Smoothing Ring Maps, Proposition \ref{smoothing-proposition-lift-smooth}.

\begin{lemma}
\label{lemma-existence-lci}
If $A \to B$ is a local complete intersection ring map, then
there exists a solution to (\ref{equation-to-solve}).
\end{lemma}

\begin{proof}
Write $B = A[x_1, \ldots, x_n]/J$. Let $J' \subset A'[x_1, \ldots, x_n]$
be the inverse image of $J$. Denote $I[x_1, \ldots, x_n]$ the
kernel of $A'[x_1, \ldots, x_n] \to A[x_1, \ldots, x_n]$.
By More on Algebra, Lemma
\ref{more-algebra-lemma-conormal-sequence-H1-regular-ideal} we have
$I[x_1, \ldots, x_n] \cap (J')^2 = J'I[x_1, \ldots, x_n] =
JI[x_1, \ldots, x_n]$. Hence we obtain a short exact sequence
$$
0 \to I \otimes_A B \to J'/(J')^2 \to J/J^2 \to 0
$$
Since $J/J^2$ is projective (More on Algebra, Lemma
\ref{more-algebra-lemma-quasi-regular-ideal-finite-projective})
we can choose a splitting of this sequence
$$
J'/(J')^2 = I \otimes_A B \oplus J/J^2 
$$
Let $(J')^2 \subset J'' \subset J'$ be the elements which map to the
second summand in the decomposition above. Then
$$
0 \to I \otimes_A B \to A'[x_1, \ldots, x_n]/J'' \to B \to 0
$$
is a solution to (\ref{equation-to-solve}) with $N = I \otimes_A B$.
The general case is obtained by doing a pushout along the given
map $I \otimes_A B \to N$.
\end{proof}

\begin{lemma}
\label{lemma-choices}
If there exists a solution to (\ref{equation-to-solve}), then the set of
isomorphism classes of solutions is principal homogeneous under
$\Ext^1_B(\NL_{B/A}, N)$.
\end{lemma}

\begin{proof}
We observe right away that given two solutions $B'_1$ and $B'_2$
to (\ref{equation-to-solve}) we obtain by Lemma \ref{lemma-huge-diagram} an
obstruction element $o(B'_1, B'_2) \in \Ext^1_B(\NL_{B/A}, N)$
to the existence of a map $B'_1 \to B'_2$. Clearly, this element
is the obstruction to the existence of an isomorphism, hence separates
the isomorphism classes. To finish the proof it therefore suffices to
show that given a solution $B'$ and an element
$\xi \in \Ext^1_B(\NL_{B/A}, N)$
we can find a second solution $B'_\xi$ such that
$o(B', B'_\xi) = \xi$.

\medskip\noindent
Let $E = B$ viewed as a set. Consider the surjection $A[E] \to B$ with kernel
$J$ used to define the naive cotangent complex by the formula
$$
\NL_{B/A} = (J/J^2 \to \Omega_{A[E]/A} \otimes_{A[E]} B)
$$
in Algebra, Section \ref{algebra-section-netherlander}.
Since $\Omega_{A[E]/A} \otimes B$ is a free $B$-module we have
$$
\Ext^1_B(\NL_{B/A}, N) =
\frac{\Hom_B(J/J^2, N)}
{\Hom_B(\Omega_{A[E]/A} \otimes B, N)}
$$
Thus we may represent $\xi$ as the class of a morphism $\delta : J/J^2 \to N$.

\medskip\noindent
For every $e \in E$ denote $x_e \in A[E]$ the corresponding variable.
Choose a lift $y_e \in B'$ of the image of $x_e$ in $B$.
These choices determine an $A'$-algebra map $\varphi : A'[E] \to B'$.
Let $J' = \Ker(A'[E] \to B)$. Observe that $\varphi$ induces a map
$\varphi|_{J'} : J' \to N$ and that $B'$ is the pushout, as in the following
diagram
$$
\xymatrix{
0 \ar[r] & N \ar[r] & B' \ar[r] & B \ar[r] & 0 \\
0 \ar[r] & J' \ar[u]^{\varphi|_{J'}} \ar[r] & A'[E] \ar[u] \ar[r] &
B \ar[u]_{=} \ar[r] & 0
}
$$
Let $\psi : J' \to N$ be the sum of the map $\varphi|_{J'}$ and the
composition
$$
J' \to J'/(J')^2 \to J/J^2 \xrightarrow{\delta} N.
$$
Then the pushout along $\psi$ is an other ring extension $B'_\xi$
fitting into a diagram as above. A calculation shows that
$o(B', B'_\xi) = \xi$ as desired.
\end{proof}

\begin{lemma}
\label{lemma-extensions-of-rings}
Let $A$ be a ring and let $I$ be an $A$-module.
\begin{enumerate}
\item The set of extensions of rings $0 \to I \to A' \to A \to 0$
where $I$ is an ideal of square zero is canonically bijective to
$\Ext^1_A(\NL_{A/\mathbf{Z}}, I)$.
\item Given a ring map $A \to B$, a $B$-module $N$, an $A$-module
map $c : I \to N$, and given extensions of rings with square zero kernels:
\begin{enumerate}
\item[(a)] $0 \to I \to A' \to A \to 0$ corresponding to
$\alpha \in \Ext^1_A(\NL_{A/\mathbf{Z}}, I)$, and
\item[(b)] $0 \to N \to B' \to B \to 0$ corresponding to
$\beta \in \Ext^1_B(\NL_{B/\mathbf{Z}}, N)$
\end{enumerate}
then there is a map $A' \to B'$ fitting into a diagram
(\ref{equation-to-solve}) if and only if $\beta$ and $\alpha$
map to the same element of
$\Ext^1_A(\NL_{A/\mathbf{Z}}, N)$.
\end{enumerate}
\end{lemma}

\begin{proof}
To prove this we apply the previous results where we work over
$0 \to 0 \to \mathbf{Z} \to \mathbf{Z} \to 0$, in order words,
we work over the extension of $\mathbf{Z}$ by $0$.
Part (1) follows from Lemma \ref{lemma-choices}
and the fact that there exists a solution, namely $I \oplus A$.
Part (2) follows from Lemma \ref{lemma-huge-diagram}
and a compatibility between the constructions in the proofs
of Lemmas \ref{lemma-choices} and \ref{lemma-huge-diagram}
whose statement and proof we omit.
\end{proof}








\section{Thickenings of ringed spaces}
\label{section-thickenings-spaces}

\noindent
In the following few sections we will use the following notions:
\begin{enumerate}
\item A sheaf of ideals $\mathcal{I} \subset \mathcal{O}_{X'}$ on
a ringed space $(X', \mathcal{O}_{X'})$ is {\it locally nilpotent}
if any local section of $\mathcal{I}$ is locally nilpotent.
Compare with Algebra, Item \ref{algebra-item-ideal-locally-nilpotent}.
\item A {\it thickening} of ringed spaces is a morphism
$i : (X, \mathcal{O}_X) \to (X', \mathcal{O}_{X'})$ of ringed spaces
such that
\begin{enumerate}
\item $i$ induces a homeomorphism $X \to X'$,
\item the map $i^\sharp : \mathcal{O}_{X'} \to i_*\mathcal{O}_X$
is surjective, and
\item the kernel of $i^\sharp$ is a locally nilpotent sheaf of ideals.
\end{enumerate}
\item A {\it first order thickening} of ringed spaces is a thickening
$i : (X, \mathcal{O}_X) \to (X', \mathcal{O}_{X'})$ of ringed spaces
such that $\Ker(i^\sharp)$ has square zero.
\item It is clear how to define {\it morphisms of thickenings},
{\it morphisms of thickenings over a base ringed space}, etc.
\end{enumerate}
If $i : (X, \mathcal{O}_X) \to (X', \mathcal{O}_{X'})$ is a thickening
of ringed spaces then we identify the underlying topological spaces
and think of $\mathcal{O}_X$, $\mathcal{O}_{X'}$, and
$\mathcal{I} = \Ker(i^\sharp)$ as sheaves on $X = X'$. We obtain
a short exact sequence
$$
0 \to \mathcal{I} \to \mathcal{O}_{X'} \to \mathcal{O}_X \to 0
$$
of $\mathcal{O}_{X'}$-modules. By
Modules, Lemma \ref{modules-lemma-i-star-equivalence}
the category of $\mathcal{O}_X$-modules is equivalent to the category
of $\mathcal{O}_{X'}$-modules annihilated by $\mathcal{I}$. In particular,
if $i$ is a first order thickening, then
$\mathcal{I}$ is a $\mathcal{O}_X$-module.

\begin{situation}
\label{situation-morphism-thickenings}
A morphism of thickenings $(f, f')$ is given by a commutative diagram
\begin{equation}
\label{equation-morphism-thickenings}
\vcenter{
\xymatrix{
(X, \mathcal{O}_X) \ar[r]_i \ar[d]_f & (X', \mathcal{O}_{X'}) \ar[d]^{f'} \\
(S, \mathcal{O}_S) \ar[r]^t & (S', \mathcal{O}_{S'})
}
}
\end{equation}
of ringed spaces whose horizontal arrows are thickenings. In this
situation we set
$\mathcal{I} = \Ker(i^\sharp) \subset \mathcal{O}_{X'}$ and
$\mathcal{J} = \Ker(t^\sharp) \subset \mathcal{O}_{S'}$.
As $f = f'$ on underlying topological spaces we will identify
the (topological) pullback functors $f^{-1}$ and $(f')^{-1}$.
Observe that $(f')^\sharp : f^{-1}\mathcal{O}_{S'} \to \mathcal{O}_{X'}$
induces in particular a map $f^{-1}\mathcal{J} \to \mathcal{I}$
and therefore a map of $\mathcal{O}_{X'}$-modules
$$
(f')^*\mathcal{J} \longrightarrow \mathcal{I}
$$
If $i$ and $t$ are first order thickenings, then
$(f')^*\mathcal{J} = f^*\mathcal{J}$ and the map above becomes a
map $f^*\mathcal{J} \to \mathcal{I}$.
\end{situation}

\begin{definition}
\label{definition-strict-morphism-thickenings}
In Situation \ref{situation-morphism-thickenings} we say that $(f, f')$ is a
{\it strict morphism of thickenings}
if the map $(f')^*\mathcal{J} \longrightarrow \mathcal{I}$ is surjective.
\end{definition}

\noindent
The following lemma in particular shows that a morphism
$(f, f') : (X \subset X') \to (S \subset S')$ of
thickenings of schemes is strict if and only if $X = S \times_{S'} X'$.

\begin{lemma}
\label{lemma-strict-morphism-thickenings}
In Situation \ref{situation-morphism-thickenings} the morphism $(f, f')$
is a strict morphism of thickenings if and only if
(\ref{equation-morphism-thickenings}) is cartesian in the category
of ringed spaces.
\end{lemma}

\begin{proof}
Omitted.
\end{proof}




\section{Modules on first order thickenings of ringed spaces}
\label{section-modules-thickenings}

\noindent
In this section we discuss some preliminaries to the deformation theory
of modules. Let $i : (X, \mathcal{O}_X) \to (X', \mathcal{O}_{X'})$
be a first order thickening of ringed spaces. We will freely use the notation
introduced in Section \ref{section-thickenings-spaces}, in particular we will
identify the underlying topological spaces.
In this section we consider short exact sequences
\begin{equation}
\label{equation-extension}
0 \to \mathcal{K} \to \mathcal{F}' \to \mathcal{F} \to 0
\end{equation}
of $\mathcal{O}_{X'}$-modules, where $\mathcal{F}$, $\mathcal{K}$ are
$\mathcal{O}_X$-modules and $\mathcal{F}'$ is an $\mathcal{O}_{X'}$-module.
In this situation we have a canonical $\mathcal{O}_X$-module map
$$
c_{\mathcal{F}'} :
\mathcal{I} \otimes_{\mathcal{O}_X} \mathcal{F}
\longrightarrow
\mathcal{K}
$$
where $\mathcal{I} = \Ker(i^\sharp)$.
Namely, given local sections $f$ of $\mathcal{I}$ and $s$
of $\mathcal{F}$ we set $c_{\mathcal{F}'}(f \otimes s) = fs'$
where $s'$ is a local section of $\mathcal{F}'$ lifting $s$.

\begin{lemma}
\label{lemma-inf-map}
Let $i : (X, \mathcal{O}_X) \to (X', \mathcal{O}_{X'})$
be a first order thickening of ringed spaces. Assume given
extensions
$$
0 \to \mathcal{K} \to \mathcal{F}' \to \mathcal{F} \to 0
\quad\text{and}\quad
0 \to \mathcal{L} \to \mathcal{G}' \to \mathcal{G} \to 0
$$
as in (\ref{equation-extension})
and maps $\varphi : \mathcal{F} \to \mathcal{G}$ and
$\psi : \mathcal{K} \to \mathcal{L}$.
\begin{enumerate}
\item If there exists an $\mathcal{O}_{X'}$-module
map $\varphi' : \mathcal{F}' \to \mathcal{G}'$ compatible with $\varphi$
and $\psi$, then the diagram
$$
\xymatrix{
\mathcal{I} \otimes_{\mathcal{O}_X} \mathcal{F}
\ar[r]_-{c_{\mathcal{F}'}} \ar[d]_{1 \otimes \varphi} &
\mathcal{K} \ar[d]^\psi \\
\mathcal{I} \otimes_{\mathcal{O}_X} \mathcal{G}
\ar[r]^-{c_{\mathcal{G}'}} &
\mathcal{L}
}
$$
is commutative.
\item The set of $\mathcal{O}_{X'}$-module
maps $\varphi' : \mathcal{F}' \to \mathcal{G}'$ compatible with $\varphi$
and $\psi$ is, if nonempty, a principal homogeneous space under
$\Hom_{\mathcal{O}_X}(\mathcal{F}, \mathcal{L})$.
\end{enumerate}
\end{lemma}

\begin{proof}
Part (1) is immediate from the description of the maps.
For (2), if $\varphi'$ and $\varphi''$ are two maps
$\mathcal{F}' \to \mathcal{G}'$ compatible with $\varphi$
and $\psi$, then $\varphi' - \varphi''$ factors as
$$
\mathcal{F}' \to \mathcal{F} \to \mathcal{L} \to \mathcal{G}'
$$
The map in the middle comes from a unique element of
$\Hom_{\mathcal{O}_X}(\mathcal{F}, \mathcal{L})$ by
Modules, Lemma \ref{modules-lemma-i-star-equivalence}.
Conversely, given an element $\alpha$ of this group we can add the
composition (as displayed above with $\alpha$ in the middle)
to $\varphi'$. Some details omitted.
\end{proof}

\begin{lemma}
\label{lemma-inf-obs-map}
Let $i : (X, \mathcal{O}_X) \to (X', \mathcal{O}_{X'})$
be a first order thickening of ringed spaces. Assume given
extensions
$$
0 \to \mathcal{K} \to \mathcal{F}' \to \mathcal{F} \to 0
\quad\text{and}\quad
0 \to \mathcal{L} \to \mathcal{G}' \to \mathcal{G} \to 0
$$
as in (\ref{equation-extension})
and maps $\varphi : \mathcal{F} \to \mathcal{G}$ and
$\psi : \mathcal{K} \to \mathcal{L}$. Assume the diagram
$$
\xymatrix{
\mathcal{I} \otimes_{\mathcal{O}_X} \mathcal{F}
\ar[r]_-{c_{\mathcal{F}'}} \ar[d]_{1 \otimes \varphi} &
\mathcal{K} \ar[d]^\psi \\
\mathcal{I} \otimes_{\mathcal{O}_X} \mathcal{G}
\ar[r]^-{c_{\mathcal{G}'}} &
\mathcal{L}
}
$$
is commutative. Then there exists an element
$$
o(\varphi, \psi) \in
\Ext^1_{\mathcal{O}_X}(\mathcal{F}, \mathcal{L})
$$
whose vanishing is a necessary and sufficient condition for the existence
of a map $\varphi' : \mathcal{F}' \to \mathcal{G}'$ compatible with
$\varphi$ and $\psi$.
\end{lemma}

\begin{proof}
We can construct explicitly an extension
$$
0 \to \mathcal{L} \to \mathcal{H} \to \mathcal{F} \to 0
$$
by taking $\mathcal{H}$ to be the cohomology of the complex
$$
\mathcal{K}
\xrightarrow{1, - \psi}
\mathcal{F}' \oplus \mathcal{G}' \xrightarrow{\varphi, 1}
\mathcal{G}
$$
in the middle (with obvious notation). A calculation with local sections
using the assumption that the diagram of the lemma commutes
shows that $\mathcal{H}$ is annihilated by $\mathcal{I}$. Hence
$\mathcal{H}$ defines a class in
$$
\Ext^1_{\mathcal{O}_X}(\mathcal{F}, \mathcal{L})
\subset
\Ext^1_{\mathcal{O}_{X'}}(\mathcal{F}, \mathcal{L})
$$
Finally, the class of $\mathcal{H}$ is the difference of the pushout
of the extension $\mathcal{F}'$ via $\psi$ and the pullback
of the extension $\mathcal{G}'$ via $\varphi$ (calculations omitted).
Thus the vanishing of the class of $\mathcal{H}$ is equivalent to the
existence of a commutative diagram
$$
\xymatrix{
0 \ar[r] &
\mathcal{K} \ar[r] \ar[d]_{\psi} &
\mathcal{F}' \ar[r] \ar[d]_{\varphi'} &
\mathcal{F} \ar[r] \ar[d]_\varphi & 0\\
0 \ar[r] &
\mathcal{L} \ar[r] &
\mathcal{G}' \ar[r] &
\mathcal{G} \ar[r] & 0
}
$$
as desired.
\end{proof}

\begin{lemma}
\label{lemma-inf-ext}
Let $i : (X, \mathcal{O}_X) \to (X', \mathcal{O}_{X'})$ be a first order
thickening of ringed spaces.
Assume given $\mathcal{O}_X$-modules $\mathcal{F}$, $\mathcal{K}$
and an $\mathcal{O}_X$-linear map
$c : \mathcal{I} \otimes_{\mathcal{O}_X} \mathcal{F} \to \mathcal{K}$.
If there exists a sequence (\ref{equation-extension}) with
$c_{\mathcal{F}'} = c$ then the set of isomorphism classes of these
extensions is principal homogeneous under
$\Ext^1_{\mathcal{O}_X}(\mathcal{F}, \mathcal{K})$.
\end{lemma}

\begin{proof}
Assume given extensions
$$
0 \to \mathcal{K} \to \mathcal{F}'_1 \to \mathcal{F} \to 0
\quad\text{and}\quad
0 \to \mathcal{K} \to \mathcal{F}'_2 \to \mathcal{F} \to 0
$$
with $c_{\mathcal{F}'_1} = c_{\mathcal{F}'_2} = c$. Then the difference
(in the extension group, see
Homology, Section \ref{homology-section-extensions})
is an extension
$$
0 \to \mathcal{K} \to \mathcal{E} \to \mathcal{F} \to 0
$$
where $\mathcal{E}$ is annihilated by $\mathcal{I}$ (local computation
omitted). Hence the sequence is an extension of $\mathcal{O}_X$-modules,
see Modules, Lemma \ref{modules-lemma-i-star-equivalence}.
Conversely, given such an extension $\mathcal{E}$ we can add the extension
$\mathcal{E}$ to the $\mathcal{O}_{X'}$-extension $\mathcal{F}'$ without
affecting the map $c_{\mathcal{F}'}$. Some details omitted.
\end{proof}

\begin{lemma}
\label{lemma-inf-obs-ext}
Let $i : (X, \mathcal{O}_X) \to (X', \mathcal{O}_{X'})$
be a first order thickening of ringed spaces. Assume given
$\mathcal{O}_X$-modules $\mathcal{F}$, $\mathcal{K}$
and an $\mathcal{O}_X$-linear map
$c : \mathcal{I} \otimes_{\mathcal{O}_X} \mathcal{F} \to \mathcal{K}$.
Then there exists an element
$$
o(\mathcal{F}, \mathcal{K}, c) \in
\Ext^2_{\mathcal{O}_X}(\mathcal{F}, \mathcal{K})
$$
whose vanishing is a necessary and sufficient condition for the existence
of a sequence (\ref{equation-extension}) with $c_{\mathcal{F}'} = c$.
\end{lemma}

\begin{proof}
We first show that if $\mathcal{K}$ is an injective $\mathcal{O}_X$-module,
then there does exist a sequence (\ref{equation-extension}) with
$c_{\mathcal{F}'} = c$. To do this, choose a flat
$\mathcal{O}_{X'}$-module $\mathcal{H}'$ and a surjection
$\mathcal{H}' \to \mathcal{F}$
(Modules, Lemma \ref{modules-lemma-module-quotient-flat}).
Let $\mathcal{J} \subset \mathcal{H}'$ be the kernel. Since $\mathcal{H}'$
is flat we have
$$
\mathcal{I} \otimes_{\mathcal{O}_{X'}} \mathcal{H}' =
\mathcal{I}\mathcal{H}'
\subset \mathcal{J} \subset \mathcal{H}'
$$
Observe that the map
$$
\mathcal{I}\mathcal{H}' =
\mathcal{I} \otimes_{\mathcal{O}_{X'}} \mathcal{H}'
\longrightarrow
\mathcal{I} \otimes_{\mathcal{O}_{X'}} \mathcal{F} =
\mathcal{I} \otimes_{\mathcal{O}_X} \mathcal{F}
$$
annihilates $\mathcal{I}\mathcal{J}$. Namely, if $f$ is a local section
of $\mathcal{I}$ and $s$ is a local section of $\mathcal{H}$, then
$fs$ is mapped to $f \otimes \overline{s}$ where $\overline{s}$ is
the image of $s$ in $\mathcal{F}$. Thus we obtain
$$
\xymatrix{
\mathcal{I}\mathcal{H}'/\mathcal{I}\mathcal{J}
\ar@{^{(}->}[r] \ar[d] &
\mathcal{J}/\mathcal{I}\mathcal{J} \ar@{..>}[d]_\gamma \\
\mathcal{I} \otimes_{\mathcal{O}_X} \mathcal{F} \ar[r]^-c &
\mathcal{K}
}
$$
a diagram of $\mathcal{O}_X$-modules. If $\mathcal{K}$ is injective
as an $\mathcal{O}_X$-module, then we obtain the dotted arrow.
Denote $\gamma' : \mathcal{J} \to \mathcal{K}$ the composition
of $\gamma$ with $\mathcal{J} \to \mathcal{J}/\mathcal{I}\mathcal{J}$.
A local calculation shows the pushout
$$
\xymatrix{
0 \ar[r] &
\mathcal{J} \ar[r] \ar[d]_{\gamma'} &
\mathcal{H}' \ar[r] \ar[d] &
\mathcal{F} \ar[r] \ar@{=}[d] &
0 \\
0 \ar[r] &
\mathcal{K} \ar[r] &
\mathcal{F}' \ar[r] &
\mathcal{F} \ar[r] &
0
}
$$
is a solution to the problem posed by the lemma.

\medskip\noindent
General case. Choose an embedding $\mathcal{K} \subset \mathcal{K}'$
with $\mathcal{K}'$ an injective $\mathcal{O}_X$-module. Let $\mathcal{Q}$
be the quotient, so that we have an exact sequence
$$
0 \to \mathcal{K} \to \mathcal{K}' \to \mathcal{Q} \to 0
$$
Denote
$c' : \mathcal{I} \otimes_{\mathcal{O}_X} \mathcal{F} \to \mathcal{K}'$
be the composition. By the paragraph above there exists a sequence
$$
0 \to \mathcal{K}' \to \mathcal{E}' \to \mathcal{F} \to 0
$$
as in (\ref{equation-extension}) with $c_{\mathcal{E}'} = c'$.
Note that $c'$ composed with the map $\mathcal{K}' \to \mathcal{Q}$
is zero, hence the pushout of $\mathcal{E}'$ by
$\mathcal{K}' \to \mathcal{Q}$ is an extension
$$
0 \to \mathcal{Q} \to \mathcal{D}' \to \mathcal{F} \to 0
$$
as in (\ref{equation-extension}) with $c_{\mathcal{D}'} = 0$.
This means exactly that $\mathcal{D}'$ is annihilated by
$\mathcal{I}$, in other words, the $\mathcal{D}'$ is an extension
of $\mathcal{O}_X$-modules, i.e., defines an element
$$
o(\mathcal{F}, \mathcal{K}, c) \in
\Ext^1_{\mathcal{O}_X}(\mathcal{F}, \mathcal{Q}) =
\Ext^2_{\mathcal{O}_X}(\mathcal{F}, \mathcal{K})
$$
(the equality holds by the long exact cohomology sequence associated
to the exact sequence above and the vanishing of higher ext groups
into the injective module $\mathcal{K}'$). If
$o(\mathcal{F}, \mathcal{K}, c) = 0$, then we can choose a splitting
$s : \mathcal{F} \to \mathcal{D}'$ and we can set
$$
\mathcal{F}' = \Ker(\mathcal{E}' \to \mathcal{D}'/s(\mathcal{F}))
$$
so that we obtain the following diagram
$$
\xymatrix{
0 \ar[r] &
\mathcal{K} \ar[r] \ar[d] &
\mathcal{F}' \ar[r] \ar[d] &
\mathcal{F} \ar[r] \ar@{=}[d] &
0 \\
0 \ar[r] &
\mathcal{K}' \ar[r] &
\mathcal{E}' \ar[r] &
\mathcal{F} \ar[r] & 0
}
$$
with exact rows which shows that $c_{\mathcal{F}'} = c$. Conversely, if
$\mathcal{F}'$ exists, then the pushout of $\mathcal{F}'$ by the map
$\mathcal{K} \to \mathcal{K}'$ is isomorphic to $\mathcal{E}'$ by
Lemma \ref{lemma-inf-ext} and the vanishing of higher ext groups
into the injective module $\mathcal{K}'$. This gives a diagram
as above, which implies that $\mathcal{D}'$ is split as an extension, i.e.,
the class $o(\mathcal{F}, \mathcal{K}, c)$ is zero.
\end{proof}

\begin{remark}
\label{remark-trivial-thickening}
Let $(X, \mathcal{O}_X)$ be a ringed space. A first order thickening
$i : (X, \mathcal{O}_X) \to (X', \mathcal{O}_{X'})$ is said
to be {\it trivial} if there exists a morphism of ringed spaces
$\pi : (X', \mathcal{O}_{X'}) \to (X, \mathcal{O}_X)$ which is a
left inverse to $i$. The choice of such a morphism
$\pi$ is called a {\it trivialization} of the first order thickening.
Given $\pi$ we obtain a splitting
\begin{equation}
\label{equation-splitting}
\mathcal{O}_{X'} = \mathcal{O}_X \oplus \mathcal{I}
\end{equation}
as sheaves of algebras on $X$ by using $\pi^\sharp$ to split the surjection
$\mathcal{O}_{X'} \to \mathcal{O}_X$. Conversely, such a splitting determines
a morphism $\pi$. The category of trivialized first order thickenings of
$(X, \mathcal{O}_X)$ is equivalent to the category of 
$\mathcal{O}_X$-modules.
\end{remark}

\begin{remark}
\label{remark-trivial-extension}
Let $i : (X, \mathcal{O}_X) \to (X', \mathcal{O}_{X'})$
be a trivial first order thickening of ringed spaces
and let $\pi : (X', \mathcal{O}_{X'}) \to (X, \mathcal{O}_X)$
be a trivialization. Then given any triple
$(\mathcal{F}, \mathcal{K}, c)$ consisting of a pair of
$\mathcal{O}_X$-modules and a map
$c : \mathcal{I} \otimes_{\mathcal{O}_X} \mathcal{F} \to \mathcal{K}$
we may set
$$
\mathcal{F}'_{c, triv} = \mathcal{F} \oplus \mathcal{K}
$$
and use the splitting (\ref{equation-splitting}) associated to $\pi$
and the map $c$ to define the $\mathcal{O}_{X'}$-module structure
and obtain an extension (\ref{equation-extension}). We will call
$\mathcal{F}'_{c, triv}$ the {\it trivial extension} of $\mathcal{F}$
by $\mathcal{K}$ corresponding
to $c$ and the trivialization $\pi$. Given any extension
$\mathcal{F}'$ as in (\ref{equation-extension}) we can use
$\pi^\sharp : \mathcal{O}_X \to \mathcal{O}_{X'}$ to think of $\mathcal{F}'$
as an $\mathcal{O}_X$-module extension, hence a class $\xi_{\mathcal{F}'}$
in $\Ext^1_{\mathcal{O}_X}(\mathcal{F}, \mathcal{K})$.
Lemma \ref{lemma-inf-ext} assures that
$\mathcal{F}' \mapsto \xi_{\mathcal{F}'}$
induces a bijection
$$
\left\{
\begin{matrix}
\text{isomorphism classes of extensions}\\
\mathcal{F}'\text{ as in (\ref{equation-extension}) with }c = c_{\mathcal{F}'}
\end{matrix}
\right\}
\longrightarrow
\Ext^1_{\mathcal{O}_X}(\mathcal{F}, \mathcal{K})
$$
Moreover, the trivial extension $\mathcal{F}'_{c, triv}$ maps to the zero class.
\end{remark}

\begin{remark}
\label{remark-extension-functorial}
Let $(X, \mathcal{O}_X)$ be a ringed space. Let
$(X, \mathcal{O}_X) \to (X'_i, \mathcal{O}_{X'_i})$, $i = 1, 2$
be first order thickenings with ideal sheaves $\mathcal{I}_i$.
Let $h : (X'_1, \mathcal{O}_{X'_1}) \to (X'_2, \mathcal{O}_{X'_2})$
be a morphism of first order thickenings of $(X, \mathcal{O}_X)$.
Picture
$$
\xymatrix{
& (X, \mathcal{O}_X) \ar[ld] \ar[rd] & \\
(X'_1, \mathcal{O}_{X'_1}) \ar[rr]^h & & 
(X'_2, \mathcal{O}_{X'_2})
}
$$
Observe that $h^\sharp : \mathcal{O}_{X'_2} \to \mathcal{O}_{X'_1}$
in particular induces an $\mathcal{O}_X$-module map
$\mathcal{I}_2 \to \mathcal{I}_1$.
Let $\mathcal{F}$ be an
$\mathcal{O}_X$-module. Let $(\mathcal{K}_i, c_i)$, $i = 1, 2$ be a pair
consisting of an $\mathcal{O}_X$-module $\mathcal{K}_i$ and a map
$c_i : \mathcal{I}_i \otimes_{\mathcal{O}_X} \mathcal{F} \to
\mathcal{K}_i$. Assume furthermore given a map
of $\mathcal{O}_X$-modules $\mathcal{K}_2 \to \mathcal{K}_1$
such that
$$
\xymatrix{
\mathcal{I}_2 \otimes_{\mathcal{O}_X} \mathcal{F}
\ar[r]_-{c_2} \ar[d] &
\mathcal{K}_2 \ar[d] \\
\mathcal{I}_1 \otimes_{\mathcal{O}_X} \mathcal{F}
\ar[r]^-{c_1} &
\mathcal{K}_1
}
$$
is commutative. Then there is a canonical functoriality
$$
\left\{
\begin{matrix}
\mathcal{F}'_2\text{ as in (\ref{equation-extension}) with }\\
c_2 = c_{\mathcal{F}'_2}\text{ and }\mathcal{K} = \mathcal{K}_2
\end{matrix}
\right\}
\longrightarrow
\left\{
\begin{matrix}
\mathcal{F}'_1\text{ as in (\ref{equation-extension}) with }\\
c_1 = c_{\mathcal{F}'_1}\text{ and }\mathcal{K} = \mathcal{K}_1
\end{matrix}
\right\}
$$
Namely, thinking of all sheaves $\mathcal{O}_X$, $\mathcal{O}_{X'_i}$,
$\mathcal{F}$, $\mathcal{K}_i$, etc as sheaves on $X$, we set
given $\mathcal{F}'_2$ the sheaf $\mathcal{F}'_1$ equal to the
pushout, i.e., fitting into the following diagram of extensions
$$
\xymatrix{
0 \ar[r] &
\mathcal{K}_2 \ar[r] \ar[d] &
\mathcal{F}'_2 \ar[r] \ar[d] &
\mathcal{F} \ar@{=}[d] \ar[r] & 0 \\
0 \ar[r] &
\mathcal{K}_1 \ar[r] &
\mathcal{F}'_1 \ar[r] &
\mathcal{F} \ar[r] & 0
}
$$
We omit the construction of the $\mathcal{O}_{X'_1}$-module structure
on the pushout (this uses the commutativity of the diagram
involving $c_1$ and $c_2$).
\end{remark}

\begin{remark}
\label{remark-trivial-extension-functorial}
Let $(X, \mathcal{O}_X)$, $(X, \mathcal{O}_X) \to (X'_i, \mathcal{O}_{X'_i})$,
$\mathcal{I}_i$, and
$h : (X'_1, \mathcal{O}_{X'_1}) \to (X'_2, \mathcal{O}_{X'_2})$
be as in Remark \ref{remark-extension-functorial}. Assume that we are
given trivializations $\pi_i : X'_i \to X$ such that
$\pi_1 = h \circ \pi_2$. In other words, assume $h$ is a morphism
of trivialized first order thickening of $(X, \mathcal{O}_X)$. Let
$(\mathcal{K}_i, c_i)$, $i = 1, 2$ be a pair consisting of an
$\mathcal{O}_X$-module $\mathcal{K}_i$ and a map
$c_i : \mathcal{I}_i \otimes_{\mathcal{O}_X} \mathcal{F} \to
\mathcal{K}_i$. Assume furthermore given a map
of $\mathcal{O}_X$-modules $\mathcal{K}_2 \to \mathcal{K}_1$
such that
$$
\xymatrix{
\mathcal{I}_2 \otimes_{\mathcal{O}_X} \mathcal{F}
\ar[r]_-{c_2} \ar[d] &
\mathcal{K}_2 \ar[d] \\
\mathcal{I}_1 \otimes_{\mathcal{O}_X} \mathcal{F}
\ar[r]^-{c_1} &
\mathcal{K}_1
}
$$
is commutative. In this situation the construction of
Remark \ref{remark-trivial-extension} induces
a commutative diagram
$$
\xymatrix{
\{\mathcal{F}'_2\text{ as in (\ref{equation-extension}) with }
c_2 = c_{\mathcal{F}'_2}\text{ and }\mathcal{K} = \mathcal{K}_2\}
\ar[d] \ar[rr] & &
\Ext^1_{\mathcal{O}_X}(\mathcal{F}, \mathcal{K}_2) \ar[d] \\
\{\mathcal{F}'_1\text{ as in (\ref{equation-extension}) with }
c_1 = c_{\mathcal{F}'_1}\text{ and }\mathcal{K} = \mathcal{K}_1\}
\ar[rr] & &
\Ext^1_{\mathcal{O}_X}(\mathcal{F}, \mathcal{K}_1)
}
$$
where the vertical map on the right is given by functoriality of $\Ext$
and the map $\mathcal{K}_2 \to \mathcal{K}_1$ and the vertical map on the left
is the one from Remark \ref{remark-extension-functorial}.
\end{remark}

\begin{remark}
\label{remark-short-exact-sequence-thickenings}
Let $(X, \mathcal{O}_X)$ be a ringed space. We define a sequence of morphisms
of first order thickenings
$$
(X'_1, \mathcal{O}_{X'_1}) \to
(X'_2, \mathcal{O}_{X'_2}) \to
(X'_3, \mathcal{O}_{X'_3})
$$
of $(X, \mathcal{O}_X)$ to be a {\it complex}
if the corresponding maps between
the ideal sheaves $\mathcal{I}_i$
give a complex of $\mathcal{O}_X$-modules
$\mathcal{I}_3 \to \mathcal{I}_2 \to \mathcal{I}_1$
(i.e., the composition is zero). In this case the composition
$(X'_1, \mathcal{O}_{X'_1}) \to (X_3', \mathcal{O}_{X'_3})$ factors through
$(X, \mathcal{O}_X) \to (X'_3, \mathcal{O}_{X'_3})$, i.e.,
the first order thickening $(X'_1, \mathcal{O}_{X'_1})$ of
$(X, \mathcal{O}_X)$ is trivial and comes with
a canonical trivialization
$\pi : (X'_1, \mathcal{O}_{X'_1}) \to (X, \mathcal{O}_X)$.

\medskip\noindent
We say a sequence of morphisms of first order thickenings
$$
(X'_1, \mathcal{O}_{X'_1}) \to
(X'_2, \mathcal{O}_{X'_2}) \to
(X'_3, \mathcal{O}_{X'_3})
$$
of $(X, \mathcal{O}_X)$ is {\it a short exact sequence} if the
corresponding maps between ideal sheaves is a short exact sequence
$$
0 \to \mathcal{I}_3 \to \mathcal{I}_2 \to \mathcal{I}_1 \to 0
$$
of $\mathcal{O}_X$-modules.
\end{remark}

\begin{remark}
\label{remark-complex-thickenings-and-ses-modules}
Let $(X, \mathcal{O}_X)$ be a ringed space. Let $\mathcal{F}$ be an
$\mathcal{O}_X$-module. Let
$$
(X'_1, \mathcal{O}_{X'_1}) \to
(X'_2, \mathcal{O}_{X'_2}) \to
(X'_3, \mathcal{O}_{X'_3})
$$
be a complex first order thickenings of $(X, \mathcal{O}_X)$, see
Remark \ref{remark-short-exact-sequence-thickenings}.
Let $(\mathcal{K}_i, c_i)$, $i = 1, 2, 3$ be pairs consisting of
an $\mathcal{O}_X$-module $\mathcal{K}_i$ and a map
$c_i : \mathcal{I}_i \otimes_{\mathcal{O}_X} \mathcal{F} \to
\mathcal{K}_i$. Assume given a short exact sequence
of $\mathcal{O}_X$-modules
$$
0 \to \mathcal{K}_3 \to \mathcal{K}_2 \to \mathcal{K}_1 \to 0
$$
such that
$$
\vcenter{
\xymatrix{
\mathcal{I}_2 \otimes_{\mathcal{O}_X} \mathcal{F}
\ar[r]_-{c_2} \ar[d] &
\mathcal{K}_2 \ar[d] \\
\mathcal{I}_1 \otimes_{\mathcal{O}_X} \mathcal{F}
\ar[r]^-{c_1} &
\mathcal{K}_1
}
}
\quad\text{and}\quad
\vcenter{
\xymatrix{
\mathcal{I}_3 \otimes_{\mathcal{O}_X} \mathcal{F}
\ar[r]_-{c_3} \ar[d] &
\mathcal{K}_3 \ar[d] \\
\mathcal{I}_2 \otimes_{\mathcal{O}_X} \mathcal{F}
\ar[r]^-{c_2} &
\mathcal{K}_2
}
}
$$
are commutative. Finally, assume given an extension
$$
0 \to \mathcal{K}_2 \to \mathcal{F}'_2 \to \mathcal{F} \to 0
$$
as in (\ref{equation-extension}) with $\mathcal{K} = \mathcal{K}_2$
of $\mathcal{O}_{X'_2}$-modules with $c_{\mathcal{F}'_2} = c_2$.
In this situation we can apply the functoriality of
Remark \ref{remark-extension-functorial} to obtain an extension
$\mathcal{F}'_1$ on $X'_1$ (we'll describe $\mathcal{F}'_1$
in this special case below). By
Remark \ref{remark-trivial-extension}
using the canonical splitting
$\pi : (X'_1, \mathcal{O}_{X'_1}) \to (X, \mathcal{O}_X)$ of
Remark \ref{remark-short-exact-sequence-thickenings}
we obtain
$\xi_{\mathcal{F}'_1} \in
\Ext^1_{\mathcal{O}_X}(\mathcal{F}, \mathcal{K}_1)$.
Finally, we have the obstruction
$$
o(\mathcal{F}, \mathcal{K}_3, c_3) \in
\Ext^2_{\mathcal{O}_X}(\mathcal{F}, \mathcal{K}_3)
$$
see Lemma \ref{lemma-inf-obs-ext}.
In this situation we {\bf claim} that the canonical map
$$
\partial :
\Ext^1_{\mathcal{O}_X}(\mathcal{F}, \mathcal{K}_1)
\longrightarrow
\Ext^2_{\mathcal{O}_X}(\mathcal{F}, \mathcal{K}_3)
$$
coming from the short exact sequence
$0 \to \mathcal{K}_3 \to \mathcal{K}_2 \to \mathcal{K}_1 \to 0$
sends $\xi_{\mathcal{F}'_1}$
to the obstruction class $o(\mathcal{F}, \mathcal{K}_3, c_3)$.

\medskip\noindent
To prove this claim choose an embedding $j : \mathcal{K}_3 \to \mathcal{K}$
where $\mathcal{K}$ is an injective $\mathcal{O}_X$-module.
We can lift $j$ to a map $j' : \mathcal{K}_2 \to \mathcal{K}$.
Set $\mathcal{E}'_2 = j'_*\mathcal{F}'_2$ equal to the pushout
of $\mathcal{F}'_2$ by $j'$ so that $c_{\mathcal{E}'_2} = j' \circ c_2$.
Picture:
$$
\xymatrix{
0 \ar[r] &
\mathcal{K}_2 \ar[r] \ar[d]_{j'} &
\mathcal{F}'_2 \ar[r] \ar[d] &
\mathcal{F} \ar[r] \ar[d] & 0 \\
0 \ar[r] &
\mathcal{K} \ar[r] &
\mathcal{E}'_2 \ar[r] &
\mathcal{F} \ar[r] & 0
}
$$
Set $\mathcal{E}'_3 = \mathcal{E}'_2$ but viewed as an
$\mathcal{O}_{X'_3}$-module via $\mathcal{O}_{X'_3} \to \mathcal{O}_{X'_2}$.
Then $c_{\mathcal{E}'_3} = j \circ c_3$.
The proof of Lemma \ref{lemma-inf-obs-ext} constructs
$o(\mathcal{F}, \mathcal{K}_3, c_3)$
as the boundary of the class of the extension of $\mathcal{O}_X$-modules
$$
0 \to
\mathcal{K}/\mathcal{K}_3 \to
\mathcal{E}'_3/\mathcal{K}_3 \to
\mathcal{F} \to 0
$$
On the other hand, note that $\mathcal{F}'_1 = \mathcal{F}'_2/\mathcal{K}_3$
hence the class $\xi_{\mathcal{F}'_1}$ is the class
of the extension
$$
0 \to \mathcal{K}_2/\mathcal{K}_3 \to \mathcal{F}'_2/\mathcal{K}_3
\to \mathcal{F} \to 0
$$
seen as a sequence of $\mathcal{O}_X$-modules using $\pi^\sharp$
where $\pi : (X'_1, \mathcal{O}_{X'_1}) \to (X, \mathcal{O}_X)$
is the canonical splitting.
Thus finally, the claim follows from the fact that we have
a commutative diagram
$$
\xymatrix{
0 \ar[r] &
\mathcal{K}_2/\mathcal{K}_3 \ar[r] \ar[d] &
\mathcal{F}'_2/\mathcal{K}_3 \ar[r] \ar[d] &
\mathcal{F} \ar[r] \ar[d] & 0 \\
0 \ar[r] &
\mathcal{K}/\mathcal{K}_3 \ar[r] &
\mathcal{E}'_3/\mathcal{K}_3 \ar[r] &
\mathcal{F} \ar[r] & 0
}
$$
which is $\mathcal{O}_X$-linear (with the $\mathcal{O}_X$-module
structures given above).
\end{remark}








\section{Infinitesimal deformations of modules on ringed spaces}
\label{section-deformation-modules}

\noindent
Let $i : (X, \mathcal{O}_X) \to (X', \mathcal{O}_{X'})$ be a first
order thickening of ringed spaces. We freely use the notation introduced in
Section \ref{section-thickenings-spaces}.
Let $\mathcal{F}'$ be an $\mathcal{O}_{X'}$-module
and set $\mathcal{F} = i^*\mathcal{F}'$.
In this situation we have a short exact sequence
$$
0 \to \mathcal{I}\mathcal{F}' \to \mathcal{F}' \to \mathcal{F} \to 0
$$
of $\mathcal{O}_{X'}$-modules. Since $\mathcal{I}^2 = 0$ the
$\mathcal{O}_{X'}$-module structure on $\mathcal{I}\mathcal{F}'$
comes from a unique $\mathcal{O}_X$-module structure.
Thus the sequence above is an extension as in (\ref{equation-extension}).
As a special case, if $\mathcal{F}' = \mathcal{O}_{X'}$ we have
$i^*\mathcal{O}_{X'} = \mathcal{O}_X$ and
$\mathcal{I}\mathcal{O}_{X'} = \mathcal{I}$ and we recover the
sequence of structure sheaves
$$
0 \to \mathcal{I} \to \mathcal{O}_{X'} \to \mathcal{O}_X \to 0
$$

\begin{lemma}
\label{lemma-inf-map-special}
Let $i : (X, \mathcal{O}_X) \to (X', \mathcal{O}_{X'})$
be a first order thickening of ringed spaces.
Let $\mathcal{F}'$, $\mathcal{G}'$ be $\mathcal{O}_{X'}$-modules.
Set $\mathcal{F} = i^*\mathcal{F}'$ and $\mathcal{G} = i^*\mathcal{G}'$.
Let $\varphi : \mathcal{F} \to \mathcal{G}$ be an $\mathcal{O}_X$-linear map.
The set of lifts of $\varphi$ to an $\mathcal{O}_{X'}$-linear map
$\varphi' : \mathcal{F}' \to \mathcal{G}'$ is, if nonempty, a principal
homogeneous space under
$\Hom_{\mathcal{O}_X}(\mathcal{F}, \mathcal{I}\mathcal{G}')$.
\end{lemma}

\begin{proof}
This is a special case of Lemma \ref{lemma-inf-map} but we also
give a direct proof. We have short exact sequences of modules
$$
0 \to \mathcal{I} \to \mathcal{O}_{X'} \to \mathcal{O}_X \to 0
\quad\text{and}\quad
0 \to \mathcal{I}\mathcal{G}' \to \mathcal{G}' \to \mathcal{G} \to 0
$$
and similarly for $\mathcal{F}'$.
Since $\mathcal{I}$ has square zero the $\mathcal{O}_{X'}$-module
structure on $\mathcal{I}$ and $\mathcal{I}\mathcal{G}'$ comes from
a unique $\mathcal{O}_X$-module structure. It follows that
$$
\Hom_{\mathcal{O}_{X'}}(\mathcal{F}', \mathcal{I}\mathcal{G}') =
\Hom_{\mathcal{O}_X}(\mathcal{F}, \mathcal{I}\mathcal{G}')
\quad\text{and}\quad
\Hom_{\mathcal{O}_{X'}}(\mathcal{F}', \mathcal{G}) =
\Hom_{\mathcal{O}_X}(\mathcal{F}, \mathcal{G})
$$
The lemma now follows from the exact sequence
$$
0 \to \Hom_{\mathcal{O}_{X'}}(\mathcal{F}', \mathcal{I}\mathcal{G}') \to
\Hom_{\mathcal{O}_{X'}}(\mathcal{F}', \mathcal{G}') \to
\Hom_{\mathcal{O}_{X'}}(\mathcal{F}', \mathcal{G})
$$
see Homology, Lemma \ref{homology-lemma-check-exactness}.
\end{proof}

\begin{lemma}
\label{lemma-deform-module}
Let $(f, f')$ be a morphism of first order thickenings of ringed spaces
as in Situation \ref{situation-morphism-thickenings}.
Let $\mathcal{F}'$ be an $\mathcal{O}_{X'}$-module
and set $\mathcal{F} = i^*\mathcal{F}'$.
Assume that $\mathcal{F}$ is flat over $S$
and that $(f, f')$ is a strict morphism of thickenings
(Definition \ref{definition-strict-morphism-thickenings}).
Then the following are equivalent
\begin{enumerate}
\item $\mathcal{F}'$ is flat over $S'$, and
\item the canonical map
$f^*\mathcal{J} \otimes_{\mathcal{O}_X} \mathcal{F} \to
\mathcal{I}\mathcal{F}'$
is an isomorphism.
\end{enumerate}
Moreover, in this case the maps
$$
f^*\mathcal{J} \otimes_{\mathcal{O}_X} \mathcal{F} \to
\mathcal{I} \otimes_{\mathcal{O}_X} \mathcal{F} \to
\mathcal{I}\mathcal{F}'
$$
are isomorphisms.
\end{lemma}

\begin{proof}
The map $f^*\mathcal{J} \to \mathcal{I}$ is surjective
as $(f, f')$ is a strict morphism of thickenings.
Hence the final statement is a consequence of (2).

\medskip\noindent
Proof of the equivalence of (1) and (2). We may check these conditions
at stalks. Let $x \in X \subset X'$
be a point with image $s = f(x) \in S \subset S'$.
Set $A' = \mathcal{O}_{S', s}$, $B' = \mathcal{O}_{X', x}$,
$A = \mathcal{O}_{S, s}$, and $B = \mathcal{O}_{X, x}$.
Then $A = A'/J$ and $B = B'/I$ for some square zero ideals.
Since $(f, f')$ is a strict morphism of thickenings we have $I = JB'$.
Let $M' = \mathcal{F}'_x$ and $M = \mathcal{F}_x$.
Then $M'$ is a $B'$-module and $M$ is a $B$-module.
Since $\mathcal{F} = i^*\mathcal{F}'$ we see that the kernel of the
surjection $M' \to M$ is $IM' = JM'$. Thus we have a short exact
sequence
$$
0 \to JM' \to M' \to M \to 0
$$
Using
Sheaves, Lemma \ref{sheaves-lemma-stalk-pullback-modules}
and
Modules, Lemma \ref{modules-lemma-stalk-tensor-product}
to identify stalks of pullbacks and tensor products we see
that the stalk at $x$ of the canonical map of the lemma is the map
$$
(J \otimes_A B) \otimes_B M = J \otimes_A M = J \otimes_{A'} M'
\longrightarrow JM'
$$
The assumption that $\mathcal{F}$ is flat over $S$ signifies that
$M$ is a flat $A$-module.

\medskip\noindent
Assume (1). Flatness implies $\text{Tor}_1^{A'}(M', A) = 0$ by
Algebra, Lemma \ref{algebra-lemma-characterize-flat}.
This means $J \otimes_{A'} M' \to M'$ is injective by
Algebra, Remark \ref{algebra-remark-Tor-ring-mod-ideal}.
Hence $J \otimes_A M \to JM'$ is an isomorphism.

\medskip\noindent
Assume (2). Then $J \otimes_{A'} M' \to M'$ is injective. Hence
$\text{Tor}_1^{A'}(M', A) = 0$ by
Algebra, Remark \ref{algebra-remark-Tor-ring-mod-ideal}.
Hence $M'$ is flat over $A'$ by
Algebra, Lemma \ref{algebra-lemma-what-does-it-mean}.
\end{proof}

\begin{lemma}
\label{lemma-inf-map-rel}
Let $(f, f')$ be a morphism of first order thickenings as in
Situation \ref{situation-morphism-thickenings}.
Let $\mathcal{F}'$, $\mathcal{G}'$ be $\mathcal{O}_{X'}$-modules and set
$\mathcal{F} = i^*\mathcal{F}'$ and $\mathcal{G} = i^*\mathcal{G}'$.
Let $\varphi : \mathcal{F} \to \mathcal{G}$ be an $\mathcal{O}_X$-linear map.
Assume that $\mathcal{G}'$ is flat over $S'$ and that
$(f, f')$ is a strict morphism of thickenings.
The set of lifts of $\varphi$ to an $\mathcal{O}_{X'}$-linear map
$\varphi' : \mathcal{F}' \to \mathcal{G}'$ is, if nonempty, a principal
homogeneous space under
$$
\Hom_{\mathcal{O}_X}(\mathcal{F},
\mathcal{G} \otimes_{\mathcal{O}_X} f^*\mathcal{J})
$$
\end{lemma}

\begin{proof}
Combine Lemmas \ref{lemma-inf-map-special} and \ref{lemma-deform-module}.
\end{proof}

\begin{lemma}
\label{lemma-inf-obs-map-special}
Let $i : (X, \mathcal{O}_X) \to (X', \mathcal{O}_{X'})$
be a first order thickening of ringed spaces.
Let $\mathcal{F}'$, $\mathcal{G}'$ be $\mathcal{O}_{X'}$-modules and set
$\mathcal{F} = i^*\mathcal{F}'$ and $\mathcal{G} = i^*\mathcal{G}'$.
Let $\varphi : \mathcal{F} \to \mathcal{G}$ be an $\mathcal{O}_X$-linear map.
There exists an element
$$
o(\varphi) \in
\Ext^1_{\mathcal{O}_X}(Li^*\mathcal{F}',
\mathcal{I}\mathcal{G}')
$$
whose vanishing is a necessary and sufficient condition for the
existence of a lift of $\varphi$ to an $\mathcal{O}_{X'}$-linear map
$\varphi' : \mathcal{F}' \to \mathcal{G}'$.
\end{lemma}

\begin{proof}
It is clear from the proof of Lemma \ref{lemma-inf-map-special} that the
vanishing of the boundary of $\varphi$ via the map
$$
\Hom_{\mathcal{O}_X}(\mathcal{F}, \mathcal{G}) =
\Hom_{\mathcal{O}_{X'}}(\mathcal{F}', \mathcal{G}) \longrightarrow
\Ext^1_{\mathcal{O}_{X'}}(\mathcal{F}', \mathcal{I}\mathcal{G}')
$$
is a necessary and sufficient condition for the existence of a lift. We
conclude as
$$
\Ext^1_{\mathcal{O}_{X'}}(\mathcal{F}', \mathcal{I}\mathcal{G}') =
\Ext^1_{\mathcal{O}_X}(Li^*\mathcal{F}', \mathcal{I}\mathcal{G}')
$$
the adjointness of $i_* = Ri_*$ and $Li^*$ on the derived category
(Cohomology, Lemma \ref{cohomology-lemma-adjoint}).
\end{proof}

\begin{lemma}
\label{lemma-inf-obs-map-rel}
Let $(f, f')$ be a morphism of first
order thickenings as in Situation \ref{situation-morphism-thickenings}.
Let $\mathcal{F}'$, $\mathcal{G}'$ be $\mathcal{O}_{X'}$-modules and set
$\mathcal{F} = i^*\mathcal{F}'$ and $\mathcal{G} = i^*\mathcal{G}'$.
Let $\varphi : \mathcal{F} \to \mathcal{G}$ be an $\mathcal{O}_X$-linear map.
Assume that $\mathcal{F}'$ and $\mathcal{G}'$ are flat over $S'$ and
that $(f, f')$ is a strict morphism of thickenings. There exists an element
$$
o(\varphi) \in  \Ext^1_{\mathcal{O}_X}(\mathcal{F},
\mathcal{G} \otimes_{\mathcal{O}_X} f^*\mathcal{J})
$$
whose vanishing is a necessary and sufficient condition for the
existence of a lift of $\varphi$ to an $\mathcal{O}_{X'}$-linear map
$\varphi' : \mathcal{F}' \to \mathcal{G}'$.
\end{lemma}

\begin{proof}[First proof]
This follows from Lemma \ref{lemma-inf-obs-map-special}
as we claim that under the assumptions of the lemma we have
$$
\Ext^1_{\mathcal{O}_X}(Li^*\mathcal{F}',
\mathcal{I}\mathcal{G}') =
\Ext^1_{\mathcal{O}_X}(\mathcal{F},
\mathcal{G} \otimes_{\mathcal{O}_X} f^*\mathcal{J})
$$
Namely, we have
$\mathcal{I}\mathcal{G}' =
\mathcal{G} \otimes_{\mathcal{O}_X} f^*\mathcal{J}$
by Lemma \ref{lemma-deform-module}.
On the other hand, observe that
$$
H^{-1}(Li^*\mathcal{F}') =
\text{Tor}_1^{\mathcal{O}_{X'}}(\mathcal{F}', \mathcal{O}_X)
$$
(local computation omitted). Using the short exact sequence
$$
0 \to \mathcal{I} \to \mathcal{O}_{X'} \to \mathcal{O}_X \to 0
$$
we see that this $\text{Tor}_1$ is computed by the kernel of the map
$\mathcal{I} \otimes_{\mathcal{O}_X} \mathcal{F} \to \mathcal{I}\mathcal{F}'$
which is zero by the final assertion of Lemma \ref{lemma-deform-module}.
Thus $\tau_{\geq -1}Li^*\mathcal{F}' = \mathcal{F}$.
On the other hand, we have
$$
\Ext^1_{\mathcal{O}_X}(Li^*\mathcal{F}',
\mathcal{I}\mathcal{G}') =
\Ext^1_{\mathcal{O}_X}(\tau_{\geq -1}Li^*\mathcal{F}',
\mathcal{I}\mathcal{G}')
$$
by the dual of
Derived Categories, Lemma \ref{derived-lemma-negative-vanishing}.
\end{proof}

\begin{proof}[Second proof]
We can apply Lemma \ref{lemma-inf-obs-map} as follows. Note that
$\mathcal{K} = \mathcal{I} \otimes_{\mathcal{O}_X} \mathcal{F}$ and
$\mathcal{L} = \mathcal{I} \otimes_{\mathcal{O}_X} \mathcal{G}$
by Lemma \ref{lemma-deform-module}, that
$c_{\mathcal{F}'} = 1 \otimes 1$ and $c_{\mathcal{G}'} = 1 \otimes 1$
and taking $\psi = 1 \otimes \varphi$ the diagram of the lemma
commutes. Thus $o(\varphi) = o(\varphi, 1 \otimes \varphi)$
works.
\end{proof}

\begin{lemma}
\label{lemma-inf-ext-rel}
Let $(f, f')$ be a morphism of first order thickenings as in
Situation \ref{situation-morphism-thickenings}.
Let $\mathcal{F}$ be an $\mathcal{O}_X$-module.
Assume $(f, f')$ is a strict morphism of thickenings and
$\mathcal{F}$ flat over $S$. If there exists a pair
$(\mathcal{F}', \alpha)$ consisting of an
$\mathcal{O}_{X'}$-module $\mathcal{F}'$ flat over $S'$ and an isomorphism
$\alpha : i^*\mathcal{F}' \to \mathcal{F}$, then the set of
isomorphism classes of such pairs is principal homogeneous
under
$\Ext^1_{\mathcal{O}_X}(
\mathcal{F}, \mathcal{I} \otimes_{\mathcal{O}_X} \mathcal{F})$.
\end{lemma}

\begin{proof}
If we assume there exists one such module, then the canonical map
$$
f^*\mathcal{J} \otimes_{\mathcal{O}_X} \mathcal{F} \to
\mathcal{I} \otimes_{\mathcal{O}_X} \mathcal{F}
$$
is an isomorphism by Lemma \ref{lemma-deform-module}. Apply
Lemma \ref{lemma-inf-ext} with $\mathcal{K} = 
\mathcal{I} \otimes_{\mathcal{O}_X} \mathcal{F}$
and $c = 1$. By Lemma \ref{lemma-deform-module} the corresponding extensions
$\mathcal{F}'$ are all flat over $S'$.
\end{proof}

\begin{lemma}
\label{lemma-inf-obs-ext-rel}
Let $(f, f')$ be a morphism of first order thickenings as in
Situation \ref{situation-morphism-thickenings}.
Let $\mathcal{F}$ be an $\mathcal{O}_X$-module. Assume
$(f, f')$ is a strict morphism of thickenings
and $\mathcal{F}$ flat over $S$. There exists an
$\mathcal{O}_{X'}$-module $\mathcal{F}'$ flat over $S'$ with
$i^*\mathcal{F}' \cong \mathcal{F}$, if and only if
\begin{enumerate}
\item the canonical map $
f^*\mathcal{J} \otimes_{\mathcal{O}_X} \mathcal{F} \to
\mathcal{I} \otimes_{\mathcal{O}_X} \mathcal{F}$
is an isomorphism, and
\item the class
$o(\mathcal{F}, \mathcal{I} \otimes_{\mathcal{O}_X} \mathcal{F}, 1)
\in \Ext^2_{\mathcal{O}_X}(
\mathcal{F}, \mathcal{I} \otimes_{\mathcal{O}_X} \mathcal{F})$
of Lemma \ref{lemma-inf-obs-ext} is zero.
\end{enumerate}
\end{lemma}

\begin{proof}
This follows immediately from the characterization of
$\mathcal{O}_{X'}$-modules flat over $S'$ of 
Lemma \ref{lemma-deform-module} and
Lemma \ref{lemma-inf-obs-ext}.
\end{proof}






\section{Application to flat modules on flat thickenings of ringed spaces}
\label{section-flat}

\noindent
Consider a commutative diagram
$$
\xymatrix{
(X, \mathcal{O}_X) \ar[r]_i \ar[d]_f & (X', \mathcal{O}_{X'}) \ar[d]^{f'} \\
(S, \mathcal{O}_S) \ar[r]^t & (S', \mathcal{O}_{S'})
}
$$
of ringed spaces whose horizontal arrows are first order thickenings as in
Situation \ref{situation-morphism-thickenings}. Set
$\mathcal{I} = \Ker(i^\sharp) \subset \mathcal{O}_{X'}$ and
$\mathcal{J} = \Ker(t^\sharp) \subset \mathcal{O}_{S'}$.
Let $\mathcal{F}$ be an $\mathcal{O}_X$-module. Assume that
\begin{enumerate}
\item $(f, f')$ is a strict morphism of thickenings,
\item $f'$ is flat, and
\item $\mathcal{F}$ is flat over $S$.
\end{enumerate}
Note that (1) $+$ (2) imply that $\mathcal{I} = f^*\mathcal{J}$
(apply Lemma \ref{lemma-deform-module} to $\mathcal{O}_{X'}$).
The theory of the preceding section is especially nice
under these assumptions. We summarize the results already obtained
in the following lemma.

\begin{lemma}
\label{lemma-flat}
In the situation above.
\begin{enumerate}
\item There exists an $\mathcal{O}_{X'}$-module $\mathcal{F}'$ flat over
$S'$ with $i^*\mathcal{F}' \cong \mathcal{F}$, if and only if
the class
$o(\mathcal{F}, f^*\mathcal{J} \otimes_{\mathcal{O}_X} \mathcal{F}, 1)
\in \Ext^2_{\mathcal{O}_X}(
\mathcal{F}, f^*\mathcal{J} \otimes_{\mathcal{O}_X} \mathcal{F})$
of Lemma \ref{lemma-inf-obs-ext} is zero.
\item If such a module exists, then the set of isomorphism classes
of lifts is principal homogeneous under
$\Ext^1_{\mathcal{O}_X}(
\mathcal{F}, f^*\mathcal{J} \otimes_{\mathcal{O}_X} \mathcal{F})$.
\item Given a lift $\mathcal{F}'$, the set of automorphisms of
$\mathcal{F}'$ which pull back to $\text{id}_\mathcal{F}$ is canonically
isomorphic to $\Ext^0_{\mathcal{O}_X}(
\mathcal{F}, f^*\mathcal{J} \otimes_{\mathcal{O}_X} \mathcal{F})$.
\end{enumerate}
\end{lemma}

\begin{proof}
Part (1) follows from Lemma \ref{lemma-inf-obs-ext-rel}
as we have seen above that $\mathcal{I} = f^*\mathcal{J}$.
Part (2) follows from Lemma \ref{lemma-inf-ext-rel}.
Part (3) follows from Lemma \ref{lemma-inf-map-rel}.
\end{proof}

\begin{situation}
\label{situation-ses-flat-thickenings}
Let $f : (X, \mathcal{O}_X) \to (S, \mathcal{O}_S)$ be a morphism of
ringed spaces. Consider a commutative diagram
$$
\xymatrix{
(X'_1, \mathcal{O}'_1) \ar[r]_h \ar[d]_{f'_1} &
(X'_2, \mathcal{O}'_2) \ar[r] \ar[d]_{f'_2} &
(X'_3, \mathcal{O}'_3) \ar[d]_{f'_3} \\
(S'_1, \mathcal{O}_{S'_1}) \ar[r] &
(S'_2, \mathcal{O}_{S'_2}) \ar[r] &
(S'_3, \mathcal{O}_{S'_3})
}
$$
where (a) the top row is a short exact sequence of first order thickenings
of $X$, (b) the lower row is a short exact sequence of first order
thickenings of $S$, (c) each $f'_i$ restricts to $f$, (d) each pair
$(f, f_i')$ is a strict morphism of thickenings, and (e) each $f'_i$ is flat.
Finally, let $\mathcal{F}'_2$ be an $\mathcal{O}'_2$-module flat over
$S'_2$ and set $\mathcal{F} = \mathcal{F}'_2|_X$. Let $\pi : X'_1 \to X$
be the canonical splitting
(Remark \ref{remark-short-exact-sequence-thickenings}).
\end{situation}

\begin{lemma}
\label{lemma-verify-iv}
In Situation \ref{situation-ses-flat-thickenings} the modules
$\pi^*\mathcal{F}$ and $h^*\mathcal{F}'_2$ are $\mathcal{O}'_1$-modules
flat over $S'_1$ restricting to $\mathcal{F}$ on $X$.
Their difference (Lemma \ref{lemma-flat}) is an element
$\theta$ of $\Ext^1_{\mathcal{O}_X}(
\mathcal{F}, f^*\mathcal{J}_1 \otimes_{\mathcal{O}_X} \mathcal{F})$
whose boundary in
$\Ext^2_{\mathcal{O}_X}(
\mathcal{F}, f^*\mathcal{J}_3 \otimes_{\mathcal{O}_X} \mathcal{F})$
equals the obstruction (Lemma \ref{lemma-flat})
to lifting $\mathcal{F}$ to an $\mathcal{O}'_3$-module flat over $S'_3$.
\end{lemma}

\begin{proof}
Note that both $\pi^*\mathcal{F}$ and $h^*\mathcal{F}'_2$
restrict to $\mathcal{F}$ on $X$ and that the kernels of
$\pi^*\mathcal{F} \to \mathcal{F}$ and $h^*\mathcal{F}'_2 \to \mathcal{F}$
are given by $f^*\mathcal{J}_1 \otimes_{\mathcal{O}_X} \mathcal{F}$.
Hence flatness by Lemma \ref{lemma-deform-module}.
Taking the boundary makes sense as the sequence of modules
$$
0 \to f^*\mathcal{J}_3 \otimes_{\mathcal{O}_X} \mathcal{F} \to
f^*\mathcal{J}_2 \otimes_{\mathcal{O}_X} \mathcal{F} \to
f^*\mathcal{J}_1 \otimes_{\mathcal{O}_X} \mathcal{F} \to 0
$$
is short exact due to the assumptions in
Situation \ref{situation-ses-flat-thickenings}
and the fact that $\mathcal{F}$ is flat over $S$.
The statement on the obstruction class is a direct translation
of the result of
Remark \ref{remark-complex-thickenings-and-ses-modules}
to this particular situation.
\end{proof}






\section{Deformations of ringed spaces and the naive cotangent complex}
\label{section-deformations-ringed-spaces}

\noindent
In this section we use the naive cotangent complex to do a little bit
of deformation theory. We start with a first order thickening
$t : (S, \mathcal{O}_S) \to (S', \mathcal{O}_{S'})$ of ringed spaces.
We denote $\mathcal{J} = \Ker(t^\sharp)$ and we
identify the underlying topological spaces of $S$ and $S'$.
Moreover we assume given a morphism of ringed spaces
$f : (X, \mathcal{O}_X) \to (S, \mathcal{O}_S)$, an $\mathcal{O}_X$-module
$\mathcal{G}$, and an $f$-map $c : \mathcal{J} \to \mathcal{G}$
of sheaves of modules (Sheaves, Definition \ref{sheaves-definition-f-map}
and Section \ref{sheaves-section-ringed-spaces-functoriality-modules}).
In this section we ask ourselves whether we can find
the question mark fitting into the following diagram
\begin{equation}
\label{equation-to-solve-ringed-spaces}
\vcenter{
\xymatrix{
0 \ar[r] & \mathcal{G} \ar[r] & {?} \ar[r] & \mathcal{O}_X \ar[r] & 0 \\
0 \ar[r] & \mathcal{J} \ar[u]^c \ar[r] & \mathcal{O}_{S'} \ar[u] \ar[r] &
\mathcal{O}_S \ar[u] \ar[r] & 0
}
}
\end{equation}
(where the vertical arrows are $f$-maps)
and moreover how unique the solution is (if it exists). More precisely,
we look for a first order thickening
$i : (X, \mathcal{O}_X) \to (X', \mathcal{O}_{X'})$
and a morphism of thickenings $(f, f')$ as in
(\ref{equation-morphism-thickenings})
where $\Ker(i^\sharp)$ is identified with $\mathcal{G}$
such that $(f')^\sharp$ induces the given map $c$.
We will say $X'$ is a {\it solution} to
(\ref{equation-to-solve-ringed-spaces}).

\begin{lemma}
\label{lemma-huge-diagram-ringed-spaces}
Assume given a commutative diagram of morphisms ringed spaces
\begin{equation}
\label{equation-huge-1}
\vcenter{
\xymatrix{
& (X_2, \mathcal{O}_{X_2}) \ar[r]_{i_2} \ar[d]_{f_2} \ar[ddl]_g &
(X'_2, \mathcal{O}_{X'_2}) \ar[d]^{f'_2} \\
& (S_2, \mathcal{O}_{S_2}) \ar[r]^{t_2} \ar[ddl]|\hole &
(S'_2, \mathcal{O}_{S'_2}) \ar[ddl] \\
(X_1, \mathcal{O}_{X_1}) \ar[r]_{i_1} \ar[d]_{f_1} &
(X'_1, \mathcal{O}_{X'_1}) \ar[d]^{f'_1} \\
(S_1, \mathcal{O}_{S_1}) \ar[r]^{t_1} &
(S'_1, \mathcal{O}_{S'_1})
}
}
\end{equation}
whose horizontal arrows are first order thickenings. Set
$\mathcal{G}_j = \Ker(i_j^\sharp)$ and assume given
a $g$-map $\nu : \mathcal{G}_1 \to \mathcal{G}_2$ of modules
giving rise to the commutative diagram
\begin{equation}
\label{equation-huge-2}
\vcenter{
\xymatrix{
& 0 \ar[r] & \mathcal{G}_2 \ar[r] &
\mathcal{O}_{X'_2} \ar[r] &
\mathcal{O}_{X_2} \ar[r] & 0 \\
& 0 \ar[r]|\hole &
\mathcal{J}_2 \ar[u]_{c_2} \ar[r] &
\mathcal{O}_{S'_2} \ar[u] \ar[r]|\hole &
\mathcal{O}_{S_2} \ar[u] \ar[r] & 0 \\
0 \ar[r] & \mathcal{G}_1 \ar[ruu] \ar[r] &
\mathcal{O}_{X'_1} \ar[r] &
\mathcal{O}_{X_1} \ar[ruu] \ar[r] & 0 \\
0 \ar[r] & \mathcal{J}_1 \ar[ruu]|\hole \ar[u]^{c_1} \ar[r] &
\mathcal{O}_{S'_1} \ar[ruu]|\hole \ar[u] \ar[r] &
\mathcal{O}_{S_1} \ar[ruu]|\hole \ar[u] \ar[r] & 0
}
}
\end{equation}
with front and back solutions to (\ref{equation-to-solve-ringed-spaces}).
\begin{enumerate}
\item There exist a canonical element in
$\Ext^1_{\mathcal{O}_{X_2}}(Lg^*\NL_{X_1/S_1}, \mathcal{G}_2)$
whose vanishing is a necessary and sufficient condition for the existence
of a morphism of ringed spaces $X'_2 \to X'_1$ fitting into
(\ref{equation-huge-1}) compatibly with $\nu$.
\item If there exists a morphism $X'_2 \to X'_1$ fitting into
(\ref{equation-huge-1}) compatibly with $\nu$ the set of all such morphisms
is a principal homogeneous space under
$$
\Hom_{\mathcal{O}_{X_1}}(\Omega_{X_1/S_1}, g_*\mathcal{G}_2) =
\Hom_{\mathcal{O}_{X_2}}(g^*\Omega_{X_1/S_1}, \mathcal{G}_2) =
\Ext^0_{\mathcal{O}_{X_2}}(Lg^*\NL_{X_1/S_1}, \mathcal{G}_2).
$$
\end{enumerate}
\end{lemma}

\begin{proof}
The naive cotangent complex $\NL_{X_1/S_1}$ is defined in Modules, Definition
\ref{modules-definition-cotangent-complex-morphism-ringed-topoi}.
The equalities in the last statement of the lemma follow from
the fact that $g^*$ is adjoint to $g_*$, the fact that
$H^0(\NL_{X_1/S_1}) = \Omega_{X_1/S_1}$ (by construction of the
naive cotangent complex) and the fact that $Lg^*$ is the left
derived functor of $g^*$. Thus we will work with the groups
$\Ext^k_{\mathcal{O}_{X_2}}(Lg^*\NL_{X_1/S_1}, \mathcal{G}_2)$,
$k = 0, 1$ in the rest of the proof. We first argue that we can reduce
to the case where the underlying topological spaces of all ringed
spaces in the lemma is the same.

\medskip\noindent
To do this, observe that $g^{-1}\NL_{X_1/S_1}$ is equal to the naive
cotangent complex of the homomorphism of sheaves of rings
$g^{-1}f_1^{-1}\mathcal{O}_{S_1} \to g^{-1}\mathcal{O}_{X_1}$, see
Modules, Lemma \ref{modules-lemma-pullback-NL}.
Moreover, the degree $0$ term of $\NL_{X_1/S_1}$ is a flat
$\mathcal{O}_{X_1}$-module, hence the canonical map
$$
Lg^*\NL_{X_1/S_1}
\longrightarrow
g^{-1}\NL_{X_1/S_1} \otimes_{g^{-1}\mathcal{O}_{X_1}} \mathcal{O}_{X_2}
$$
induces an isomorphism on cohomology sheaves in degrees $0$ and $-1$.
Thus we may replace the Ext groups of the lemma with
$$
\Ext^k_{g^{-1}\mathcal{O}_{X_1}}(g^{-1}\NL_{X_1/S_1}, \mathcal{G}_2) =
\Ext^k_{g^{-1}\mathcal{O}_{X_1}}(
\NL_{g^{-1}\mathcal{O}_{X_1}/g^{-1}f_1^{-1}\mathcal{O}_{S_1}}, \mathcal{G}_2)
$$
The set of morphism of ringed spaces $X'_2 \to X'_1$ fitting into
(\ref{equation-huge-1}) compatibly with $\nu$
is in one-to-one bijection with
the set of homomorphisms of $g^{-1}f_1^{-1}\mathcal{O}_{S'_1}$-algebras
$g^{-1}\mathcal{O}_{X'_1} \to \mathcal{O}_{X'_2}$ which are compatible with
$f^\sharp$ and $\nu$. In this way we see that we may assume we have a
diagram (\ref{equation-huge-2}) of sheaves on $X$ and we are looking to
find a homomorphism of sheaves of rings
$\mathcal{O}_{X'_1} \to \mathcal{O}_{X'_2}$ fitting into it.

\medskip\noindent
In the rest of the proof of the lemma we assume
all underlying topological spaces are the
same, i.e., we have a diagram (\ref{equation-huge-2}) of sheaves on
a space $X$ and we are looking for homomorphisms of sheaves of rings
$\mathcal{O}_{X'_1} \to \mathcal{O}_{X'_2}$ fitting into it.
As ext groups we will use
$\Ext^k_{\mathcal{O}_{X_1}}(
\NL_{\mathcal{O}_{X_1}/\mathcal{O}_{S_1}}, \mathcal{G}_2)$, $k = 0, 1$.

\medskip\noindent
Step 1. Construction of the obstruction class. Consider the sheaf
of sets
$$
\mathcal{E} = \mathcal{O}_{X'_1} \times_{\mathcal{O}_{X_2}} \mathcal{O}_{X'_2}
$$
This comes with a surjective map $\alpha : \mathcal{E} \to \mathcal{O}_{X_1}$
and hence we can use $\NL(\alpha)$ instead of
$\NL_{\mathcal{O}_{X_1}/\mathcal{O}_{S_1}}$, see
Modules, Lemma \ref{modules-lemma-NL-up-to-qis}.
Set
$$
\mathcal{I}' =
\Ker(\mathcal{O}_{S'_1}[\mathcal{E}] \to \mathcal{O}_{X_1})
\quad\text{and}\quad
\mathcal{I} =
\Ker(\mathcal{O}_{S_1}[\mathcal{E}] \to \mathcal{O}_{X_1})
$$
There is a surjection $\mathcal{I}' \to \mathcal{I}$ whose kernel
is $\mathcal{J}_1\mathcal{O}_{S'_1}[\mathcal{E}]$.
We obtain two homomorphisms of $\mathcal{O}_{S'_2}$-algebras
$$
a : \mathcal{O}_{S'_1}[\mathcal{E}] \to \mathcal{O}_{X'_1}
\quad\text{and}\quad
b : \mathcal{O}_{S'_1}[\mathcal{E}] \to \mathcal{O}_{X'_2}
$$
which induce maps $a|_{\mathcal{I}'} : \mathcal{I}' \to \mathcal{G}_1$ and
$b|_{\mathcal{I}'} : \mathcal{I}' \to \mathcal{G}_2$. Both $a$ and $b$
annihilate $(\mathcal{I}')^2$. Moreover $a$ and $b$ agree on
$\mathcal{J}_1\mathcal{O}_{S'_1}[\mathcal{E}]$ as maps into $\mathcal{G}_2$
because the left hand square of (\ref{equation-huge-2}) is commutative.
Thus the difference
$b|_{\mathcal{I}'} - \nu \circ a|_{\mathcal{I}'}$
induces a well defined $\mathcal{O}_{X_1}$-linear map
$$
\xi : \mathcal{I}/\mathcal{I}^2 \longrightarrow \mathcal{G}_2
$$
which sends the class of a local section $f$ of $\mathcal{I}$ to
$a(f') - \nu(b(f'))$ where $f'$ is a lift of $f$ to a local
section of $\mathcal{I}'$. We let
$[\xi] \in \Ext^1_{\mathcal{O}_{X_1}}(\NL(\alpha), \mathcal{G}_2)$
be the image (see below).

\medskip\noindent
Step 2. Vanishing of $[\xi]$ is necessary. Let us write
$\Omega = \Omega_{\mathcal{O}_{S_1}[\mathcal{E}]/\mathcal{O}_{S_1}}
\otimes_{\mathcal{O}_{S_1}[\mathcal{E}]} \mathcal{O}_{X_1}$.
Observe that $\NL(\alpha) = (\mathcal{I}/\mathcal{I}^2 \to \Omega)$
fits into a distinguished triangle
$$
\Omega[0] \to
\NL(\alpha) \to
\mathcal{I}/\mathcal{I}^2[1] \to
\Omega[1]
$$
Thus we see that $[\xi]$ is zero if and only if $\xi$
is a composition $\mathcal{I}/\mathcal{I}^2 \to \Omega \to \mathcal{G}_2$
for some map $\Omega \to \mathcal{G}_2$. Suppose there exists a
homomorphisms of sheaves of rings
$\varphi : \mathcal{O}_{X'_1} \to \mathcal{O}_{X'_2}$ fitting into
(\ref{equation-huge-2}). In this case consider the map
$\mathcal{O}_{S'_1}[\mathcal{E}] \to \mathcal{G}_2$,
$f' \mapsto b(f') - \varphi(a(f'))$. A calculation
shows this annihilates $\mathcal{J}_1\mathcal{O}_{S'_1}[\mathcal{E}]$
and induces a derivation $\mathcal{O}_{S_1}[\mathcal{E}] \to \mathcal{G}_2$.
The resulting linear map $\Omega \to \mathcal{G}_2$ witnesses the
fact that $[\xi] = 0$ in this case.

\medskip\noindent
Step 3. Vanishing of $[\xi]$ is sufficient. Let
$\theta : \Omega \to \mathcal{G}_2$ be a $\mathcal{O}_{X_1}$-linear map
such that $\xi$ is equal to
$\theta \circ (\mathcal{I}/\mathcal{I}^2 \to \Omega)$.
Then a calculation shows that
$$
b + \theta \circ d : \mathcal{O}_{S'_1}[\mathcal{E}] \to \mathcal{O}_{X'_2}
$$
annihilates $\mathcal{I}'$ and hence defines a map
$\mathcal{O}_{X'_1} \to \mathcal{O}_{X'_2}$ fitting into
(\ref{equation-huge-2}).

\medskip\noindent
Proof of (2) in the special case above. Omitted. Hint:
This is exactly the same as the proof of (2) of Lemma \ref{lemma-huge-diagram}.
\end{proof}

\begin{lemma}
\label{lemma-NL-represent-ext-class}
Let $X$ be a topological space. Let $\mathcal{A} \to \mathcal{B}$ be a
homomorphism of sheaves of rings. Let $\mathcal{G}$ be a $\mathcal{B}$-module.
Let
$\xi \in \Ext^1_\mathcal{B}(\NL_{\mathcal{B}/\mathcal{A}}, \mathcal{G})$. 
There exists a map of sheaves of sets $\alpha : \mathcal{E} \to \mathcal{B}$
such that $\xi \in \Ext^1_\mathcal{B}(\NL(\alpha), \mathcal{G})$
is the class of a map $\mathcal{I}/\mathcal{I}^2 \to \mathcal{G}$
(see proof for notation).
\end{lemma}

\begin{proof}
Recall that given $\alpha : \mathcal{E} \to \mathcal{B}$
such that $\mathcal{A}[\mathcal{E}] \to \mathcal{B}$ is surjective
with kernel $\mathcal{I}$ the complex
$\NL(\alpha) = (\mathcal{I}/\mathcal{I}^2 \to 
\Omega_{\mathcal{A}[\mathcal{E}]/\mathcal{A}}
\otimes_{\mathcal{A}[\mathcal{E}]} \mathcal{B})$ is canonically
isomorphic to $\NL_{\mathcal{B}/\mathcal{A}}$, see
Modules, Lemma \ref{modules-lemma-NL-up-to-qis}.
Observe moreover, that
$\Omega = \Omega_{\mathcal{A}[\mathcal{E}]/\mathcal{A}}
\otimes_{\mathcal{A}[\mathcal{E}]} \mathcal{B}$ is the sheaf
associated to the presheaf
$U \mapsto \bigoplus_{e \in \mathcal{E}(U)} \mathcal{B}(U)$.
In other words, $\Omega$ is the free $\mathcal{B}$-module on the
sheaf of sets $\mathcal{E}$ and in particular there is a canonical
map $\mathcal{E} \to \Omega$.

\medskip\noindent
Having said this, pick some $\mathcal{E}$ (for example
$\mathcal{E} = \mathcal{B}$ as in the definition of the naive cotangent
complex). The obstruction to writing $\xi$ as the class of a map
$\mathcal{I}/\mathcal{I}^2 \to \mathcal{G}$ is an element in
$\Ext^1_\mathcal{B}(\Omega, \mathcal{G})$. Say this is represented
by the extension $0 \to \mathcal{G} \to \mathcal{H} \to \Omega \to 0$
of $\mathcal{B}$-modules. Consider the sheaf of sets
$\mathcal{E}' = \mathcal{E} \times_\Omega \mathcal{H}$
which comes with an induced map $\alpha' : \mathcal{E}' \to \mathcal{B}$.
Let $\mathcal{I}' = \Ker(\mathcal{A}[\mathcal{E}'] \to \mathcal{B})$
and $\Omega' = \Omega_{\mathcal{A}[\mathcal{E}']/\mathcal{A}}
\otimes_{\mathcal{A}[\mathcal{E}']} \mathcal{B}$.
The pullback of $\xi$ under the quasi-isomorphism
$\NL(\alpha') \to \NL(\alpha)$ maps to zero in
$\Ext^1_\mathcal{B}(\Omega', \mathcal{G})$
because the pullback of the extension $\mathcal{H}$
by the map $\Omega' \to \Omega$ is split as $\Omega'$ is the free
$\mathcal{B}$-module on the sheaf of sets $\mathcal{E}'$ and since
by construction there is a commutative diagram
$$
\xymatrix{
\mathcal{E}' \ar[r] \ar[d] & \mathcal{E} \ar[d] \\
\mathcal{H} \ar[r] & \Omega
}
$$
This finishes the proof.
\end{proof}

\begin{lemma}
\label{lemma-choices-ringed-spaces}
If there exists a solution to (\ref{equation-to-solve-ringed-spaces}),
then the set of isomorphism classes of solutions is principal homogeneous
under $\Ext^1_{\mathcal{O}_X}(\NL_{X/S}, \mathcal{G})$.
\end{lemma}

\begin{proof}
We observe right away that given two solutions $X'_1$ and $X'_2$
to (\ref{equation-to-solve-ringed-spaces}) we obtain by
Lemma \ref{lemma-huge-diagram-ringed-spaces} an obstruction element
$o(X'_1, X'_2) \in \Ext^1_{\mathcal{O}_X}(\NL_{X/S}, \mathcal{G})$
to the existence of a map $X'_1 \to X'_2$. Clearly, this element
is the obstruction to the existence of an isomorphism, hence separates
the isomorphism classes. To finish the proof it therefore suffices to
show that given a solution $X'$ and an element
$\xi \in \Ext^1_{\mathcal{O}_X}(\NL_{X/S}, \mathcal{G})$
we can find a second solution $X'_\xi$ such that
$o(X', X'_\xi) = \xi$.

\medskip\noindent
Pick $\alpha : \mathcal{E} \to \mathcal{O}_X$ as in
Lemma \ref{lemma-NL-represent-ext-class}
for the class $\xi$. Consider the surjection
$f^{-1}\mathcal{O}_S[\mathcal{E}] \to \mathcal{O}_X$
with kernel $\mathcal{I}$ and corresponding naive cotangent complex
$\NL(\alpha) = (\mathcal{I}/\mathcal{I}^2 \to
\Omega_{f^{-1}\mathcal{O}_S[\mathcal{E}]/f^{-1}\mathcal{O}_S}
\otimes_{f^{-1}\mathcal{O}_S[\mathcal{E}]} \mathcal{O}_X)$.
By the lemma $\xi$ is the class of a morphism
$\delta : \mathcal{I}/\mathcal{I}^2 \to \mathcal{G}$.
After replacing $\mathcal{E}$ by
$\mathcal{E} \times_{\mathcal{O}_X} \mathcal{O}_{X'}$ we may also assume
that $\alpha$ factors through a map
$\alpha' : \mathcal{E} \to \mathcal{O}_{X'}$.

\medskip\noindent
These choices determine an $f^{-1}\mathcal{O}_{S'}$-algebra map
$\varphi : \mathcal{O}_{S'}[\mathcal{E}] \to \mathcal{O}_{X'}$.
Let $\mathcal{I}' = \Ker(\varphi)$.
Observe that $\varphi$ induces a map
$\varphi|_{\mathcal{I}'} : \mathcal{I}' \to \mathcal{G}$
and that $\mathcal{O}_{X'}$ is the pushout, as in the following
diagram
$$
\xymatrix{
0 \ar[r] & \mathcal{G} \ar[r] & \mathcal{O}_{X'} \ar[r] &
\mathcal{O}_X \ar[r] & 0 \\
0 \ar[r] & \mathcal{I}' \ar[u]^{\varphi|_{\mathcal{I}'}} \ar[r] &
f^{-1}\mathcal{O}_{S'}[\mathcal{E}] \ar[u] \ar[r] &
\mathcal{O}_X \ar[u]_{=} \ar[r] & 0
}
$$
Let $\psi : \mathcal{I}' \to \mathcal{G}$ be the sum of the map
$\varphi|_{\mathcal{I}'}$ and the composition
$$
\mathcal{I}' \to \mathcal{I}'/(\mathcal{I}')^2 \to
\mathcal{I}/\mathcal{I}^2 \xrightarrow{\delta} \mathcal{G}.
$$
Then the pushout along $\psi$ is an other ring extension
$\mathcal{O}_{X'_\xi}$ fitting into a diagram as above.
A calculation (omitted) shows that $o(X', X'_\xi) = \xi$ as desired.
\end{proof}

\begin{lemma}
\label{lemma-extensions-of-ringed-spaces}
Let $(S, \mathcal{O}_S)$ be a ringed space and let $\mathcal{J}$
be an $\mathcal{O}_S$-module.
\begin{enumerate}
\item The set of extensions of sheaves of rings
$0 \to \mathcal{J} \to \mathcal{O}_{S'} \to \mathcal{O}_S \to 0$
where $\mathcal{J}$ is an ideal of square zero is canonically bijective to
$\Ext^1_{\mathcal{O}_S}(\NL_{S/\mathbf{Z}}, \mathcal{J})$.
\item Given a morphism of ringed spaces
$f : (X, \mathcal{O}_X) \to (S, \mathcal{O}_S)$, an $\mathcal{O}_X$-module
$\mathcal{G}$, an $f$-map $c : \mathcal{J} \to \mathcal{G}$, and
given extensions of sheaves of rings with square zero kernels:
\begin{enumerate}
\item[(a)] $0 \to \mathcal{J} \to \mathcal{O}_{S'} \to \mathcal{O}_S \to 0$
corresponding to
$\alpha \in \Ext^1_{\mathcal{O}_S}(\NL_{S/\mathbf{Z}}, \mathcal{J})$,
\item[(b)] $0 \to \mathcal{G} \to \mathcal{O}_{X'} \to \mathcal{O}_X \to 0$
corresponding to
$\beta \in \Ext^1_{\mathcal{O}_X}(\NL_{X/\mathbf{Z}}, \mathcal{G})$
\end{enumerate}
then there is a morphism $X' \to S'$ fitting into a diagram
(\ref{equation-to-solve-ringed-spaces}) if and only if $\beta$ and $\alpha$
map to the same element of
$\Ext^1_{\mathcal{O}_X}(Lf^*\NL_{S/\mathbf{Z}}, \mathcal{G})$.
\end{enumerate}
\end{lemma}

\begin{proof}
To prove this we apply the previous results where we work over
the base ringed space $(*, \mathbf{Z})$ with trivial thickening.
Part (1) follows from Lemma \ref{lemma-choices-ringed-spaces}
and the fact that there exists a solution, namely
$\mathcal{J} \oplus \mathcal{O}_S$.
Part (2) follows from Lemma \ref{lemma-huge-diagram-ringed-spaces}
and a compatibility between the constructions in the proofs
of Lemmas \ref{lemma-choices-ringed-spaces} and
\ref{lemma-huge-diagram-ringed-spaces}
whose statement and proof we omit.
\end{proof}








\section{Deformations of schemes}
\label{section-deformations-schemes}

\noindent
In this section we spell out what the results in
Section \ref{section-deformations-ringed-spaces}
mean for deformations of schemes.

\begin{lemma}
\label{lemma-deform}
Let $S \subset S'$ be a first order thickening of schemes.
Let $f : X \to S$ be a flat morphism of schemes.
If there exists a flat morphism $f' : X' \to S'$ of schemes
and an isomorphsm $a : X \to X' \times_{S'} S$ over $S$, then
\begin{enumerate}
\item the set of isomorphism classes of pairs $(f' : X' \to S', a)$ is
principal homogeneous under
$\Ext^1_{\mathcal{O}_X}(\NL_{X/S}, f^*\mathcal{C}_{S/S'})$, and
\item the set of automorphisms of $\varphi : X' \to X'$
over $S'$ which reduce to the identity on $X' \times_{S'} S$
is $\Ext^0_{\mathcal{O}_X}(\NL_{X/S}, f^*\mathcal{C}_{S/S'})$.
\end{enumerate}
\end{lemma}

\begin{proof}
First we observe that thickenings of schemes as defined in
More on Morphisms, Section \ref{more-morphisms-section-thickenings}
are the same things as morphisms of schemes which
are thickenings in the sense of
Section \ref{section-thickenings-spaces}.
We may think of $X$ as a closed subscheme of $X'$
so that $(f, f') : (X \subset X') \to (S \subset S')$
is a morphism of first order thickenings. Then we see
from More on Morphisms, Lemma \ref{more-morphisms-lemma-deform}
(or from the more general Lemma \ref{lemma-deform-module})
that the ideal sheaf of $X$ in $X'$ is equal to $f^*\mathcal{C}_{S'/S}$.
Hence we have a commutative diagram
$$
\xymatrix{
0 \ar[r] & f^*\mathcal{C}_{S/S'} \ar[r] &
\mathcal{O}_{X'} \ar[r] &
\mathcal{O}_X \ar[r] & 0 \\
0 \ar[r] & \mathcal{C}_{S/S'} \ar[u] \ar[r] &
\mathcal{O}_{S'} \ar[u] \ar[r] &
\mathcal{O}_S \ar[u] \ar[r] & 0
}
$$
where the vertical arrows are $f$-maps; please compare with
(\ref{equation-to-solve-ringed-spaces}).
Thus part (1) follows from
Lemma \ref{lemma-choices-ringed-spaces}
and part (2) from part (2) of
Lemma \ref{lemma-huge-diagram-ringed-spaces}.
(Note that $\NL_{X/S}$ as defined for a morphism of schemes in
More on Morphisms, Section \ref{more-morphisms-section-netherlander}
agrees with $\NL_{X/S}$ as used in
Section \ref{section-deformations-ringed-spaces}.)
\end{proof}










\section{Thickenings of ringed topoi}
\label{section-thickenings-ringed-topoi}

\noindent
This section is the analogue of Section \ref{section-thickenings-spaces}
for ringed topoi.
In the following few sections we will use the following notions:
\begin{enumerate}
\item A sheaf of ideals $\mathcal{I} \subset \mathcal{O}'$ on
a ringed topos $(\Sh(\mathcal{D}), \mathcal{O}')$ is {\it locally nilpotent}
if any local section of $\mathcal{I}$ is locally nilpotent.
\item A {\it thickening} of ringed topoi is a morphism
$i : (\Sh(\mathcal{C}), \mathcal{O}) \to (\Sh(\mathcal{D}), \mathcal{O}')$
of ringed topoi such that
\begin{enumerate}
\item $i_*$ is an equivalence $\Sh(\mathcal{C}) \to \Sh(\mathcal{D})$,
\item the map $i^\sharp : \mathcal{O}' \to i_*\mathcal{O}$
is surjective, and
\item the kernel of $i^\sharp$ is a locally nilpotent sheaf of ideals.
\end{enumerate}
\item A {\it first order thickening} of ringed topoi is a thickening
$i : (\Sh(\mathcal{C}), \mathcal{O}) \to (\Sh(\mathcal{D}), \mathcal{O}')$
of ringed topoi such that $\Ker(i^\sharp)$ has square zero.
\item It is clear how to define
{\it morphisms of thickenings of ringed topoi},
{\it morphisms of thickenings of ringed topoi over a base ringed topos}, etc.
\end{enumerate}
If
$i : (\Sh(\mathcal{C}), \mathcal{O}) \to (\Sh(\mathcal{D}), \mathcal{O}')$
is a thickening of ringed topoi then we identify the underlying topoi
and think of $\mathcal{O}$, $\mathcal{O}'$, and
$\mathcal{I} = \Ker(i^\sharp)$ as sheaves on $\mathcal{C}$.
We obtain a short exact sequence
$$
0 \to \mathcal{I} \to \mathcal{O}' \to \mathcal{O} \to 0
$$
of $\mathcal{O}'$-modules. By
Modules on Sites, Lemma \ref{sites-modules-lemma-i-star-equivalence}
the category of $\mathcal{O}$-modules is equivalent to the category
of $\mathcal{O}'$-modules annihilated by $\mathcal{I}$. In particular,
if $i$ is a first order thickening, then
$\mathcal{I}$ is a $\mathcal{O}$-module.

\begin{situation}
\label{situation-morphism-thickenings-ringed-topoi}
A morphism of thickenings of ringed topoi $(f, f')$
is given by a commutative diagram
\begin{equation}
\label{equation-morphism-thickenings-ringed-topoi}
\vcenter{
\xymatrix{
(\Sh(\mathcal{C}), \mathcal{O}) \ar[r]_i \ar[d]_f &
(\Sh(\mathcal{D}), \mathcal{O}') \ar[d]^{f'} \\
(\Sh(\mathcal{B}), \mathcal{O}_\mathcal{B}) \ar[r]^t &
(\Sh(\mathcal{B}'), \mathcal{O}_{\mathcal{B}'})
}
}
\end{equation}
of ringed topoi whose horizontal arrows are thickenings. In this
situation we set
$\mathcal{I} = \Ker(i^\sharp) \subset \mathcal{O}'$ and
$\mathcal{J} = \Ker(t^\sharp) \subset \mathcal{O}_{\mathcal{B}'}$.
As $f = f'$ on underlying topoi we will identify
the pullback functors $f^{-1}$ and $(f')^{-1}$.
Observe that
$(f')^\sharp : f^{-1}\mathcal{O}_{\mathcal{B}'} \to \mathcal{O}'$
induces in particular a map $f^{-1}\mathcal{J} \to \mathcal{I}$
and therefore a map of $\mathcal{O}'$-modules
$$
(f')^*\mathcal{J} \longrightarrow \mathcal{I}
$$
If $i$ and $t$ are first order thickenings, then
$(f')^*\mathcal{J} = f^*\mathcal{J}$ and the map above becomes a
map $f^*\mathcal{J} \to \mathcal{I}$.
\end{situation}

\begin{definition}
\label{definition-strict-morphism-thickenings-ringed-topoi}
In Situation \ref{situation-morphism-thickenings-ringed-topoi}
we say that $(f, f')$ is a {\it strict morphism of thickenings}
if the map $(f')^*\mathcal{J} \longrightarrow \mathcal{I}$ is surjective.
\end{definition}








\section{Modules on first order thickenings of ringed topoi}
\label{section-modules-thickenings-ringed-topoi}

\noindent
In this section we discuss some preliminaries to the deformation theory
of modules. Let
$i : (\Sh(\mathcal{C}, \mathcal{O}) \to (\Sh(\mathcal{D}), \mathcal{O}')$
be a first order thickening of ringed topoi. We will freely use the notation
introduced in Section \ref{section-thickenings-ringed-topoi},
in particular we will identify the underlying topological topoi.
In this section we consider short exact sequences
\begin{equation}
\label{equation-extension-ringed-topoi}
0 \to \mathcal{K} \to \mathcal{F}' \to \mathcal{F} \to 0
\end{equation}
of $\mathcal{O}'$-modules, where $\mathcal{F}$, $\mathcal{K}$ are
$\mathcal{O}$-modules and $\mathcal{F}'$ is an $\mathcal{O}'$-module.
In this situation we have a canonical $\mathcal{O}$-module map
$$
c_{\mathcal{F}'} :
\mathcal{I} \otimes_\mathcal{O} \mathcal{F}
\longrightarrow
\mathcal{K}
$$
where $\mathcal{I} = \Ker(i^\sharp)$.
Namely, given local sections $f$ of $\mathcal{I}$ and $s$
of $\mathcal{F}$ we set $c_{\mathcal{F}'}(f \otimes s) = fs'$
where $s'$ is a local section of $\mathcal{F}'$ lifting $s$.

\begin{lemma}
\label{lemma-inf-map-ringed-topoi}
Let $i : (\Sh(\mathcal{C}), \mathcal{O}) \to (\Sh(\mathcal{D}), \mathcal{O}')$
be a first order thickening of ringed topoi. Assume given
extensions
$$
0 \to \mathcal{K} \to \mathcal{F}' \to \mathcal{F} \to 0
\quad\text{and}\quad
0 \to \mathcal{L} \to \mathcal{G}' \to \mathcal{G} \to 0
$$
as in (\ref{equation-extension-ringed-topoi})
and maps $\varphi : \mathcal{F} \to \mathcal{G}$ and
$\psi : \mathcal{K} \to \mathcal{L}$.
\begin{enumerate}
\item If there exists an $\mathcal{O}'$-module
map $\varphi' : \mathcal{F}' \to \mathcal{G}'$ compatible with $\varphi$
and $\psi$, then the diagram
$$
\xymatrix{
\mathcal{I} \otimes_\mathcal{O} \mathcal{F}
\ar[r]_-{c_{\mathcal{F}'}} \ar[d]_{1 \otimes \varphi} &
\mathcal{K} \ar[d]^\psi \\
\mathcal{I} \otimes_\mathcal{O} \mathcal{G}
\ar[r]^-{c_{\mathcal{G}'}} &
\mathcal{L}
}
$$
is commutative.
\item The set of $\mathcal{O}'$-module
maps $\varphi' : \mathcal{F}' \to \mathcal{G}'$ compatible with $\varphi$
and $\psi$ is, if nonempty, a principal homogeneous space under
$\Hom_\mathcal{O}(\mathcal{F}, \mathcal{L})$.
\end{enumerate}
\end{lemma}

\begin{proof}
Part (1) is immediate from the description of the maps.
For (2), if $\varphi'$ and $\varphi''$ are two maps
$\mathcal{F}' \to \mathcal{G}'$ compatible with $\varphi$
and $\psi$, then $\varphi' - \varphi''$ factors as
$$
\mathcal{F}' \to \mathcal{F} \to \mathcal{L} \to \mathcal{G}'
$$
The map in the middle comes from a unique element of
$\Hom_\mathcal{O}(\mathcal{F}, \mathcal{L})$ by
Modules on Sites, Lemma \ref{sites-modules-lemma-i-star-equivalence}.
Conversely, given an element $\alpha$ of this group we can add the
composition (as displayed above with $\alpha$ in the middle)
to $\varphi'$. Some details omitted.
\end{proof}

\begin{lemma}
\label{lemma-inf-obs-map-ringed-topoi}
Let $i : (\Sh(\mathcal{C}), \mathcal{O}) \to (\Sh(\mathcal{D}), \mathcal{O}')$
be a first order thickening of ringed topoi. Assume given extensions
$$
0 \to \mathcal{K} \to \mathcal{F}' \to \mathcal{F} \to 0
\quad\text{and}\quad
0 \to \mathcal{L} \to \mathcal{G}' \to \mathcal{G} \to 0
$$
as in (\ref{equation-extension-ringed-topoi})
and maps $\varphi : \mathcal{F} \to \mathcal{G}$ and
$\psi : \mathcal{K} \to \mathcal{L}$. Assume the diagram
$$
\xymatrix{
\mathcal{I} \otimes_\mathcal{O} \mathcal{F}
\ar[r]_-{c_{\mathcal{F}'}} \ar[d]_{1 \otimes \varphi} &
\mathcal{K} \ar[d]^\psi \\
\mathcal{I} \otimes_\mathcal{O} \mathcal{G}
\ar[r]^-{c_{\mathcal{G}'}} &
\mathcal{L}
}
$$
is commutative. Then there exists an element
$$
o(\varphi, \psi) \in
\Ext^1_\mathcal{O}(\mathcal{F}, \mathcal{L})
$$
whose vanishing is a necessary and sufficient condition for the existence
of a map $\varphi' : \mathcal{F}' \to \mathcal{G}'$ compatible with
$\varphi$ and $\psi$.
\end{lemma}

\begin{proof}
We can construct explicitly an extension
$$
0 \to \mathcal{L} \to \mathcal{H} \to \mathcal{F} \to 0
$$
by taking $\mathcal{H}$ to be the cohomology of the complex
$$
\mathcal{K}
\xrightarrow{1, - \psi}
\mathcal{F}' \oplus \mathcal{G}' \xrightarrow{\varphi, 1}
\mathcal{G}
$$
in the middle (with obvious notation). A calculation with local sections
using the assumption that the diagram of the lemma commutes
shows that $\mathcal{H}$ is annihilated by $\mathcal{I}$. Hence
$\mathcal{H}$ defines a class in
$$
\Ext^1_\mathcal{O}(\mathcal{F}, \mathcal{L})
\subset
\Ext^1_{\mathcal{O}'}(\mathcal{F}, \mathcal{L})
$$
Finally, the class of $\mathcal{H}$ is the difference of the pushout
of the extension $\mathcal{F}'$ via $\psi$ and the pullback
of the extension $\mathcal{G}'$ via $\varphi$ (calculations omitted).
Thus the vanishing of the class of $\mathcal{H}$ is equivalent to the
existence of a commutative diagram
$$
\xymatrix{
0 \ar[r] &
\mathcal{K} \ar[r] \ar[d]_{\psi} &
\mathcal{F}' \ar[r] \ar[d]_{\varphi'} &
\mathcal{F} \ar[r] \ar[d]_\varphi & 0\\
0 \ar[r] &
\mathcal{L} \ar[r] &
\mathcal{G}' \ar[r] &
\mathcal{G} \ar[r] & 0
}
$$
as desired.
\end{proof}

\begin{lemma}
\label{lemma-inf-ext-ringed-topoi}
Let $i : (\Sh(\mathcal{C}), \mathcal{O}) \to (\Sh(\mathcal{D}), \mathcal{O}')$
be a first order thickening of ringed topoi. Assume given
$\mathcal{O}$-modules $\mathcal{F}$, $\mathcal{K}$
and an $\mathcal{O}$-linear map
$c : \mathcal{I} \otimes_\mathcal{O} \mathcal{F} \to \mathcal{K}$.
If there exists a sequence (\ref{equation-extension-ringed-topoi}) with
$c_{\mathcal{F}'} = c$ then the set of isomorphism classes of these
extensions is principal homogeneous under
$\Ext^1_\mathcal{O}(\mathcal{F}, \mathcal{K})$.
\end{lemma}

\begin{proof}
Assume given extensions
$$
0 \to \mathcal{K} \to \mathcal{F}'_1 \to \mathcal{F} \to 0
\quad\text{and}\quad
0 \to \mathcal{K} \to \mathcal{F}'_2 \to \mathcal{F} \to 0
$$
with $c_{\mathcal{F}'_1} = c_{\mathcal{F}'_2} = c$. Then the difference
(in the extension group, see
Homology, Section \ref{homology-section-extensions})
is an extension
$$
0 \to \mathcal{K} \to \mathcal{E} \to \mathcal{F} \to 0
$$
where $\mathcal{E}$ is annihilated by $\mathcal{I}$ (local computation
omitted). Hence the sequence is an extension of $\mathcal{O}$-modules,
see Modules on Sites, Lemma \ref{sites-modules-lemma-i-star-equivalence}.
Conversely, given such an extension $\mathcal{E}$ we can add the extension
$\mathcal{E}$ to the $\mathcal{O}'$-extension $\mathcal{F}'$ without
affecting the map $c_{\mathcal{F}'}$. Some details omitted.
\end{proof}

\begin{lemma}
\label{lemma-inf-obs-ext-ringed-topoi}
Let $i : (\Sh(\mathcal{C}), \mathcal{O}) \to (\Sh(\mathcal{D}), \mathcal{O}')$
be a first order thickening of ringed topoi. Assume given
$\mathcal{O}$-modules $\mathcal{F}$, $\mathcal{K}$
and an $\mathcal{O}$-linear map
$c : \mathcal{I} \otimes_\mathcal{O} \mathcal{F} \to \mathcal{K}$.
Then there exists an element
$$
o(\mathcal{F}, \mathcal{K}, c) \in
\Ext^2_\mathcal{O}(\mathcal{F}, \mathcal{K})
$$
whose vanishing is a necessary and sufficient condition for the existence
of a sequence (\ref{equation-extension-ringed-topoi})
with $c_{\mathcal{F}'} = c$.
\end{lemma}

\begin{proof}
We first show that if $\mathcal{K}$ is an injective $\mathcal{O}$-module,
then there does exist a sequence (\ref{equation-extension-ringed-topoi}) with
$c_{\mathcal{F}'} = c$. To do this, choose a flat
$\mathcal{O}'$-module $\mathcal{H}'$ and a surjection
$\mathcal{H}' \to \mathcal{F}$
(Modules on Sites, Lemma \ref{sites-modules-lemma-module-quotient-flat}).
Let $\mathcal{J} \subset \mathcal{H}'$ be the kernel. Since $\mathcal{H}'$
is flat we have
$$
\mathcal{I} \otimes_{\mathcal{O}'} \mathcal{H}' =
\mathcal{I}\mathcal{H}'
\subset \mathcal{J} \subset \mathcal{H}'
$$
Observe that the map
$$
\mathcal{I}\mathcal{H}' =
\mathcal{I} \otimes_{\mathcal{O}'} \mathcal{H}'
\longrightarrow
\mathcal{I} \otimes_{\mathcal{O}'} \mathcal{F} =
\mathcal{I} \otimes_\mathcal{O} \mathcal{F}
$$
annihilates $\mathcal{I}\mathcal{J}$. Namely, if $f$ is a local section
of $\mathcal{I}$ and $s$ is a local section of $\mathcal{H}$, then
$fs$ is mapped to $f \otimes \overline{s}$ where $\overline{s}$ is
the image of $s$ in $\mathcal{F}$. Thus we obtain
$$
\xymatrix{
\mathcal{I}\mathcal{H}'/\mathcal{I}\mathcal{J}
\ar@{^{(}->}[r] \ar[d] &
\mathcal{J}/\mathcal{I}\mathcal{J} \ar@{..>}[d]_\gamma \\
\mathcal{I} \otimes_\mathcal{O} \mathcal{F} \ar[r]^-c &
\mathcal{K}
}
$$
a diagram of $\mathcal{O}$-modules. If $\mathcal{K}$ is injective
as an $\mathcal{O}$-module, then we obtain the dotted arrow.
Denote $\gamma' : \mathcal{J} \to \mathcal{K}$ the composition
of $\gamma$ with $\mathcal{J} \to \mathcal{J}/\mathcal{I}\mathcal{J}$.
A local calculation shows the pushout
$$
\xymatrix{
0 \ar[r] &
\mathcal{J} \ar[r] \ar[d]_{\gamma'} &
\mathcal{H}' \ar[r] \ar[d] &
\mathcal{F} \ar[r] \ar@{=}[d] &
0 \\
0 \ar[r] &
\mathcal{K} \ar[r] &
\mathcal{F}' \ar[r] &
\mathcal{F} \ar[r] &
0
}
$$
is a solution to the problem posed by the lemma.

\medskip\noindent
General case. Choose an embedding $\mathcal{K} \subset \mathcal{K}'$
with $\mathcal{K}'$ an injective $\mathcal{O}$-module. Let $\mathcal{Q}$
be the quotient, so that we have an exact sequence
$$
0 \to \mathcal{K} \to \mathcal{K}' \to \mathcal{Q} \to 0
$$
Denote
$c' : \mathcal{I} \otimes_\mathcal{O} \mathcal{F} \to \mathcal{K}'$
be the composition. By the paragraph above there exists a sequence
$$
0 \to \mathcal{K}' \to \mathcal{E}' \to \mathcal{F} \to 0
$$
as in (\ref{equation-extension-ringed-topoi}) with $c_{\mathcal{E}'} = c'$.
Note that $c'$ composed with the map $\mathcal{K}' \to \mathcal{Q}$
is zero, hence the pushout of $\mathcal{E}'$ by
$\mathcal{K}' \to \mathcal{Q}$ is an extension
$$
0 \to \mathcal{Q} \to \mathcal{D}' \to \mathcal{F} \to 0
$$
as in (\ref{equation-extension-ringed-topoi}) with $c_{\mathcal{D}'} = 0$.
This means exactly that $\mathcal{D}'$ is annihilated by
$\mathcal{I}$, in other words, the $\mathcal{D}'$ is an extension
of $\mathcal{O}$-modules, i.e., defines an element
$$
o(\mathcal{F}, \mathcal{K}, c) \in
\Ext^1_\mathcal{O}(\mathcal{F}, \mathcal{Q}) =
\Ext^2_\mathcal{O}(\mathcal{F}, \mathcal{K})
$$
(the equality holds by the long exact cohomology sequence associated
to the exact sequence above and the vanishing of higher ext groups
into the injective module $\mathcal{K}'$). If
$o(\mathcal{F}, \mathcal{K}, c) = 0$, then we can choose a splitting
$s : \mathcal{F} \to \mathcal{D}'$ and we can set
$$
\mathcal{F}' = \Ker(\mathcal{E}' \to \mathcal{D}'/s(\mathcal{F}))
$$
so that we obtain the following diagram
$$
\xymatrix{
0 \ar[r] &
\mathcal{K} \ar[r] \ar[d] &
\mathcal{F}' \ar[r] \ar[d] &
\mathcal{F} \ar[r] \ar@{=}[d] &
0 \\
0 \ar[r] &
\mathcal{K}' \ar[r] &
\mathcal{E}' \ar[r] &
\mathcal{F} \ar[r] & 0
}
$$
with exact rows which shows that $c_{\mathcal{F}'} = c$. Conversely, if
$\mathcal{F}'$ exists, then the pushout of $\mathcal{F}'$ by the map
$\mathcal{K} \to \mathcal{K}'$ is isomorphic to $\mathcal{E}'$ by
Lemma \ref{lemma-inf-ext-ringed-topoi} and the vanishing of higher ext groups
into the injective module $\mathcal{K}'$. This gives a diagram
as above, which implies that $\mathcal{D}'$ is split as an extension, i.e.,
the class $o(\mathcal{F}, \mathcal{K}, c)$ is zero.
\end{proof}

\begin{remark}
\label{remark-trivial-thickening-ringed-topoi}
Let $(\Sh(\mathcal{C}), \mathcal{O})$ be a ringed topos. A first order
thickening $i : (\Sh(\mathcal{C}), \mathcal{O}) \to
(\Sh(\mathcal{D}), \mathcal{O}')$ is said
to be {\it trivial} if there exists a morphism of ringed topoi
$\pi : (\Sh(\mathcal{D}), \mathcal{O}') \to (\Sh(\mathcal{C}), \mathcal{O})$
which is a left inverse to $i$. The choice of such a morphism
$\pi$ is called a {\it trivialization} of the first order thickening.
Given $\pi$ we obtain a splitting
\begin{equation}
\label{equation-splitting-ringed-topoi}
\mathcal{O}' = \mathcal{O} \oplus \mathcal{I}
\end{equation}
as sheaves of algebras on $\mathcal{C}$ by using $\pi^\sharp$
to split the surjection $\mathcal{O}' \to \mathcal{O}$.
Conversely, such a splitting determines
a morphism $\pi$. The category of trivialized first order thickenings of
$(\Sh(\mathcal{C}), \mathcal{O})$ is equivalent to the category of 
$\mathcal{O}$-modules.
\end{remark}

\begin{remark}
\label{remark-trivial-extension-ringed-topoi}
Let $i : (\Sh(\mathcal{C}), \mathcal{O}) \to (\Sh(\mathcal{D}), \mathcal{O}')$
be a trivial first order thickening of ringed topoi
and let $\pi : (\Sh(\mathcal{D}), \mathcal{O}') \to
(\Sh(\mathcal{C}), \mathcal{O})$ be a trivialization. Then given any triple
$(\mathcal{F}, \mathcal{K}, c)$ consisting of a pair of
$\mathcal{O}$-modules and a map
$c : \mathcal{I} \otimes_\mathcal{O} \mathcal{F} \to \mathcal{K}$
we may set
$$
\mathcal{F}'_{c, triv} = \mathcal{F} \oplus \mathcal{K}
$$
and use the splitting (\ref{equation-splitting-ringed-topoi})
associated to $\pi$ and the map $c$ to define the $\mathcal{O}'$-module
structure and obtain an extension (\ref{equation-extension-ringed-topoi}).
We will call $\mathcal{F}'_{c, triv}$ the {\it trivial extension} of
$\mathcal{F}$ by $\mathcal{K}$ corresponding
to $c$ and the trivialization $\pi$. Given any extension
$\mathcal{F}'$ as in (\ref{equation-extension-ringed-topoi}) we can use
$\pi^\sharp : \mathcal{O} \to \mathcal{O}'$ to think of $\mathcal{F}'$
as an $\mathcal{O}$-module extension, hence a class $\xi_{\mathcal{F}'}$
in $\Ext^1_\mathcal{O}(\mathcal{F}, \mathcal{K})$.
Lemma \ref{lemma-inf-ext-ringed-topoi} assures that
$\mathcal{F}' \mapsto \xi_{\mathcal{F}'}$
induces a bijection
$$
\left\{
\begin{matrix}
\text{isomorphism classes of extensions}\\
\mathcal{F}'\text{ as in (\ref{equation-extension-ringed-topoi}) with }
c = c_{\mathcal{F}'}
\end{matrix}
\right\}
\longrightarrow
\Ext^1_\mathcal{O}(\mathcal{F}, \mathcal{K})
$$
Moreover, the trivial extension $\mathcal{F}'_{c, triv}$ maps to the zero class.
\end{remark}

\begin{remark}
\label{remark-extension-functorial-ringed-topoi}
Let $(\Sh(\mathcal{C}), \mathcal{O})$ be a ringed topos. Let
$(\Sh(\mathcal{C}), \mathcal{O}) \to (\Sh(\mathcal{D}_i), \mathcal{O}'_i)$,
$i = 1, 2$ be first order thickenings with ideal sheaves $\mathcal{I}_i$.
Let $h : (\Sh(\mathcal{D}_1), \mathcal{O}'_1) \to
(\Sh(\mathcal{D}_2), \mathcal{O}'_2)$
be a morphism of first order thickenings of $(\Sh(\mathcal{C}), \mathcal{O})$.
Picture
$$
\xymatrix{
& (\Sh(\mathcal{C}), \mathcal{O}) \ar[ld] \ar[rd] & \\
(\Sh(\mathcal{D}_1), \mathcal{O}'_1) \ar[rr]^h & & 
(\Sh(\mathcal{D}_2), \mathcal{O}'_2)
}
$$
Observe that $h^\sharp : \mathcal{O}'_2 \to \mathcal{O}'_1$
in particular induces an $\mathcal{O}$-module map
$\mathcal{I}_2 \to \mathcal{I}_1$.
Let $\mathcal{F}$ be an $\mathcal{O}$-module.
Let $(\mathcal{K}_i, c_i)$, $i = 1, 2$ be a pair
consisting of an $\mathcal{O}$-module $\mathcal{K}_i$ and a map
$c_i : \mathcal{I}_i \otimes_\mathcal{O} \mathcal{F} \to
\mathcal{K}_i$. Assume furthermore given a map
of $\mathcal{O}$-modules $\mathcal{K}_2 \to \mathcal{K}_1$
such that
$$
\xymatrix{
\mathcal{I}_2 \otimes_\mathcal{O} \mathcal{F}
\ar[r]_-{c_2} \ar[d] &
\mathcal{K}_2 \ar[d] \\
\mathcal{I}_1 \otimes_\mathcal{O} \mathcal{F}
\ar[r]^-{c_1} &
\mathcal{K}_1
}
$$
is commutative. Then there is a canonical functoriality
$$
\left\{
\begin{matrix}
\mathcal{F}'_2\text{ as in (\ref{equation-extension-ringed-topoi}) with }\\
c_2 = c_{\mathcal{F}'_2}\text{ and }\mathcal{K} = \mathcal{K}_2
\end{matrix}
\right\}
\longrightarrow
\left\{
\begin{matrix}
\mathcal{F}'_1\text{ as in (\ref{equation-extension-ringed-topoi}) with }\\
c_1 = c_{\mathcal{F}'_1}\text{ and }\mathcal{K} = \mathcal{K}_1
\end{matrix}
\right\}
$$
Namely, thinking of all sheaves $\mathcal{O}$, $\mathcal{O}'_i$,
$\mathcal{F}$, $\mathcal{K}_i$, etc as sheaves on $\mathcal{C}$, we set
given $\mathcal{F}'_2$ the sheaf $\mathcal{F}'_1$ equal to the
pushout, i.e., fitting into the following diagram of extensions
$$
\xymatrix{
0 \ar[r] &
\mathcal{K}_2 \ar[r] \ar[d] &
\mathcal{F}'_2 \ar[r] \ar[d] &
\mathcal{F} \ar@{=}[d] \ar[r] & 0 \\
0 \ar[r] &
\mathcal{K}_1 \ar[r] &
\mathcal{F}'_1 \ar[r] &
\mathcal{F} \ar[r] & 0
}
$$
We omit the construction of the $\mathcal{O}'_1$-module structure
on the pushout (this uses the commutativity of the diagram
involving $c_1$ and $c_2$).
\end{remark}

\begin{remark}
\label{remark-trivial-extension-functorial-ringed-topoi}
Let $(\Sh(\mathcal{C}), \mathcal{O})$,
$(\Sh(\mathcal{C}), \mathcal{O}) \to (\Sh(\mathcal{D}_i), \mathcal{O}'_i)$,
$\mathcal{I}_i$, and $h : (\Sh(\mathcal{D}_1), \mathcal{O}'_1) \to
(\Sh(\mathcal{D}_2), \mathcal{O}'_2)$ be as in
Remark \ref{remark-extension-functorial-ringed-topoi}.
Assume that we are given trivializations
$\pi_i : (\Sh(\mathcal{D}_i), \mathcal{O}'_i) \to
(\Sh(\mathcal{C}), \mathcal{O})$ such that
$\pi_1 = h \circ \pi_2$. In other words, assume $h$ is a morphism
of trivialized first order thickenings of $(\Sh(\mathcal{C}), \mathcal{O})$.
Let $(\mathcal{K}_i, c_i)$, $i = 1, 2$ be a pair consisting of an
$\mathcal{O}$-module $\mathcal{K}_i$ and a map
$c_i : \mathcal{I}_i \otimes_\mathcal{O} \mathcal{F} \to
\mathcal{K}_i$. Assume furthermore given a map
of $\mathcal{O}$-modules $\mathcal{K}_2 \to \mathcal{K}_1$
such that
$$
\xymatrix{
\mathcal{I}_2 \otimes_\mathcal{O} \mathcal{F}
\ar[r]_-{c_2} \ar[d] &
\mathcal{K}_2 \ar[d] \\
\mathcal{I}_1 \otimes_\mathcal{O} \mathcal{F}
\ar[r]^-{c_1} &
\mathcal{K}_1
}
$$
is commutative. In this situation the construction of
Remark \ref{remark-trivial-extension-ringed-topoi} induces
a commutative diagram
$$
\xymatrix{
\{\mathcal{F}'_2\text{ as in (\ref{equation-extension-ringed-topoi}) with }
c_2 = c_{\mathcal{F}'_2}\text{ and }\mathcal{K} = \mathcal{K}_2\}
\ar[d] \ar[rr] & &
\Ext^1_\mathcal{O}(\mathcal{F}, \mathcal{K}_2) \ar[d] \\
\{\mathcal{F}'_1\text{ as in (\ref{equation-extension-ringed-topoi}) with }
c_1 = c_{\mathcal{F}'_1}\text{ and }\mathcal{K} = \mathcal{K}_1\}
\ar[rr] & &
\Ext^1_\mathcal{O}(\mathcal{F}, \mathcal{K}_1)
}
$$
where the vertical map on the right is given by functoriality of $\Ext$
and the map $\mathcal{K}_2 \to \mathcal{K}_1$ and the vertical map on the left
is the one from Remark \ref{remark-extension-functorial-ringed-topoi}.
\end{remark}

\begin{remark}
\label{remark-obstruction-extension-functorial-ringed-topoi}
Let $(\Sh(\mathcal{C}), \mathcal{O})$,
$(\Sh(\mathcal{C}), \mathcal{O}) \to (\Sh(\mathcal{D}_i), \mathcal{O}'_i)$,
$\mathcal{I}_i$, and $h : (\Sh(\mathcal{D}_1), \mathcal{O}'_1) \to
(\Sh(\mathcal{D}_2), \mathcal{O}'_2)$ be as in
Remark \ref{remark-extension-functorial-ringed-topoi}.
Observe that $h^\sharp : \mathcal{O}'_2 \to \mathcal{O}'_1$
in particular induces an $\mathcal{O}$-module map
$\mathcal{I}_2 \to \mathcal{I}_1$.
Let $\mathcal{F}$ be an $\mathcal{O}$-module.
Let $(\mathcal{K}_i, c_i)$, $i = 1, 2$ be a pair
consisting of an $\mathcal{O}$-module $\mathcal{K}_i$ and a map
$c_i : \mathcal{I}_i \otimes_\mathcal{O} \mathcal{F} \to
\mathcal{K}_i$. Assume furthermore given a map
of $\mathcal{O}$-modules $\mathcal{K}_2 \to \mathcal{K}_1$
such that
$$
\xymatrix{
\mathcal{I}_2 \otimes_\mathcal{O} \mathcal{F}
\ar[r]_-{c_2} \ar[d] &
\mathcal{K}_2 \ar[d] \\
\mathcal{I}_1 \otimes_\mathcal{O} \mathcal{F}
\ar[r]^-{c_1} &
\mathcal{K}_1
}
$$
is commutative. Then we {\bf claim} the map
$$
\Ext^2_\mathcal{O}(\mathcal{F}, \mathcal{K}_2)
\longrightarrow
\Ext^2_\mathcal{O}(\mathcal{F}, \mathcal{K}_1)
$$
sends $o(\mathcal{F}, \mathcal{K}_2, c_2)$ to
$o(\mathcal{F}, \mathcal{K}_1, c_1)$.

\medskip\noindent
To prove this claim choose an embedding
$j_2 : \mathcal{K}_2 \to \mathcal{K}_2'$
where $\mathcal{K}_2'$ is an injective $\mathcal{O}$-module.
As in the proof of Lemma \ref{lemma-inf-obs-ext-ringed-topoi}
we can choose an extension of $\mathcal{O}_2$-modules
$$
0 \to \mathcal{K}_2' \to \mathcal{E}_2 \to \mathcal{F} \to 0
$$
such that $c_{\mathcal{E}_2} = j_2 \circ c_2$.
The proof of Lemma \ref{lemma-inf-obs-ext-ringed-topoi} constructs
$o(\mathcal{F}, \mathcal{K}_2, c_2)$
as the Yoneda extension class (in the sense of
Derived Categories, Section \ref{derived-section-ext})
of the exact sequence of $\mathcal{O}$-modules
$$
0 \to
\mathcal{K}_2 \to \mathcal{K}_2' \to
\mathcal{E}_2/\mathcal{K}_2 \to
\mathcal{F} \to 0
$$
Let $\mathcal{K}_1'$ be the cokernel of
$\mathcal{K}_2 \to \mathcal{K}_1 \oplus \mathcal{K}_2'$.
There is an injection $j_1 : \mathcal{K}_1 \to \mathcal{K}_1'$
and a map $\mathcal{K}_2' \to \mathcal{K}_1'$ forming
a commutative square. We form the pushout:
$$
\xymatrix{
0 \ar[r] &
\mathcal{K}_2' \ar[r] \ar[d] &
\mathcal{E}_2 \ar[r] \ar[d] &
\mathcal{F} \ar[r] \ar[d] & 0 \\
0 \ar[r] &
\mathcal{K}_1' \ar[r] &
\mathcal{E}_1 \ar[r] &
\mathcal{F} \ar[r] & 0
}
$$
There is a canonical $\mathcal{O}_1$-module structure on
$\mathcal{E}_1$ and for this structure we have
$c_{\mathcal{E}_1} = j_1 \circ c_1$ (this uses the commutativity
of the diagram involving $c_1$ and $c_2$ above).
The procedure of Lemma \ref{lemma-inf-obs-ext-ringed-topoi}
tells us that $o(\mathcal{F}, \mathcal{K}_1, c_1)$
is the Yoneda extension class of the exact sequence
of $\mathcal{O}$-modules
$$
0 \to
\mathcal{K}_1 \to
\mathcal{K}_1' \to
\mathcal{E}_1/\mathcal{K}_1 \to
\mathcal{F} \to 0
$$
Since we have maps of exact sequences
$$
\xymatrix{
0 \ar[r] &
\mathcal{K}_2 \ar[d] \ar[r] &
\mathcal{K}_2' \ar[d] \ar[r] &
\mathcal{E}_2/\mathcal{K}_2 \ar[r] \ar[d] &
\mathcal{F} \ar[r] \ar@{=}[d] &
0 \\
0 \ar[r] &
\mathcal{K}_2 \ar[r] &
\mathcal{K}_2' \ar[r] &
\mathcal{E}_2/\mathcal{K}_2 \ar[r] &
\mathcal{F} \ar[r] &
0
}
$$
we conclude that the claim is true.
\end{remark}

\begin{remark}
\label{remark-short-exact-sequence-thickenings-ringed-topoi}
Let $(\Sh(\mathcal{C}), \mathcal{O})$ be a ringed topos.
We define a sequence of morphisms of first order thickenings
$$
(\Sh(\mathcal{D}_1), \mathcal{O}'_1) \to
(\Sh(\mathcal{D}_2), \mathcal{O}'_2) \to
(\Sh(\mathcal{D}_3), \mathcal{O}'_3)
$$
of $(\Sh(\mathcal{C}), \mathcal{O})$ to be a {\it complex}
if the corresponding maps between
the ideal sheaves $\mathcal{I}_i$
give a complex of $\mathcal{O}$-modules
$\mathcal{I}_3 \to \mathcal{I}_2 \to \mathcal{I}_1$
(i.e., the composition is zero). In this case the composition
$(\Sh(\mathcal{D}_1), \mathcal{O}'_1) \to
(\Sh(\mathcal{D}_3), \mathcal{O}'_3)$ factors through
$(\Sh(\mathcal{C}), \mathcal{O}) \to
(\Sh(\mathcal{D}_3), \mathcal{O}'_3)$, i.e.,
the first order thickening
$(\Sh(\mathcal{D}_1), \mathcal{O}'_1)$ of
$(\Sh(\mathcal{C}), \mathcal{O})$ is trivial and comes with
a canonical trivialization
$\pi : (\Sh(\mathcal{D}_1), \mathcal{O}'_1) \to
(\Sh(\mathcal{C}), \mathcal{O})$.

\medskip\noindent
We say a sequence of morphisms of first order thickenings
$$
(\Sh(\mathcal{D}_1), \mathcal{O}'_1) \to
(\Sh(\mathcal{D}_2), \mathcal{O}'_2) \to
(\Sh(\mathcal{D}_3), \mathcal{O}'_3)
$$
of $(\Sh(\mathcal{C}), \mathcal{O})$ is {\it a short exact sequence} if the
corresponding maps between ideal sheaves is a short exact sequence
$$
0 \to \mathcal{I}_3 \to \mathcal{I}_2 \to \mathcal{I}_1 \to 0
$$
of $\mathcal{O}$-modules.
\end{remark}

\begin{remark}
\label{remark-complex-thickenings-and-ses-modules-ringed-topoi}
Let $(\Sh(\mathcal{C}), \mathcal{O})$ be a ringed topos.
Let $\mathcal{F}$ be an $\mathcal{O}$-module. Let
$$
(\Sh(\mathcal{D}_1), \mathcal{O}'_1) \to
(\Sh(\mathcal{D}_2), \mathcal{O}'_2) \to
(\Sh(\mathcal{D}_3), \mathcal{O}'_3)
$$
be a complex first order thickenings of $(\Sh(\mathcal{C}), \mathcal{O})$, see
Remark \ref{remark-short-exact-sequence-thickenings-ringed-topoi}.
Let $(\mathcal{K}_i, c_i)$, $i = 1, 2, 3$ be pairs consisting of
an $\mathcal{O}$-module $\mathcal{K}_i$ and a map
$c_i : \mathcal{I}_i \otimes_\mathcal{O} \mathcal{F} \to
\mathcal{K}_i$. Assume given a short exact sequence
of $\mathcal{O}$-modules
$$
0 \to \mathcal{K}_3 \to \mathcal{K}_2 \to \mathcal{K}_1 \to 0
$$
such that
$$
\vcenter{
\xymatrix{
\mathcal{I}_2 \otimes_\mathcal{O} \mathcal{F}
\ar[r]_-{c_2} \ar[d] &
\mathcal{K}_2 \ar[d] \\
\mathcal{I}_1 \otimes_\mathcal{O} \mathcal{F}
\ar[r]^-{c_1} &
\mathcal{K}_1
}
}
\quad\text{and}\quad
\vcenter{
\xymatrix{
\mathcal{I}_3 \otimes_\mathcal{O} \mathcal{F}
\ar[r]_-{c_3} \ar[d] &
\mathcal{K}_3 \ar[d] \\
\mathcal{I}_2 \otimes_\mathcal{O} \mathcal{F}
\ar[r]^-{c_2} &
\mathcal{K}_2
}
}
$$
are commutative. Finally, assume given an extension
$$
0 \to \mathcal{K}_2 \to \mathcal{F}'_2 \to \mathcal{F} \to 0
$$
as in (\ref{equation-extension-ringed-topoi})
with $\mathcal{K} = \mathcal{K}_2$
of $\mathcal{O}'_2$-modules with $c_{\mathcal{F}'_2} = c_2$.
In this situation we can apply the functoriality of
Remark \ref{remark-extension-functorial-ringed-topoi}
to obtain an extension $\mathcal{F}'_1$ of $\mathcal{O}'_1$-modules
(we'll describe $\mathcal{F}'_1$ in this special case below). By
Remark \ref{remark-trivial-extension-ringed-topoi}
using the canonical splitting
$\pi : (\Sh(\mathcal{D}_1), \mathcal{O}'_1) \to
(\Sh(\mathcal{C}), \mathcal{O})$ of
Remark \ref{remark-short-exact-sequence-thickenings-ringed-topoi}
we obtain
$\xi_{\mathcal{F}'_1} \in
\Ext^1_\mathcal{O}(\mathcal{F}, \mathcal{K}_1)$.
Finally, we have the obstruction
$$
o(\mathcal{F}, \mathcal{K}_3, c_3) \in
\Ext^2_\mathcal{O}(\mathcal{F}, \mathcal{K}_3)
$$
see Lemma \ref{lemma-inf-obs-ext-ringed-topoi}.
In this situation we {\bf claim} that the canonical map
$$
\partial :
\Ext^1_\mathcal{O}(\mathcal{F}, \mathcal{K}_1)
\longrightarrow
\Ext^2_\mathcal{O}(\mathcal{F}, \mathcal{K}_3)
$$
coming from the short exact sequence
$0 \to \mathcal{K}_3 \to \mathcal{K}_2 \to \mathcal{K}_1 \to 0$
sends $\xi_{\mathcal{F}'_1}$
to the obstruction class $o(\mathcal{F}, \mathcal{K}_3, c_3)$.

\medskip\noindent
To prove this claim choose an embedding $j : \mathcal{K}_3 \to \mathcal{K}$
where $\mathcal{K}$ is an injective $\mathcal{O}$-module.
We can lift $j$ to a map $j' : \mathcal{K}_2 \to \mathcal{K}$.
Set $\mathcal{E}'_2 = j'_*\mathcal{F}'_2$ equal to the pushout
of $\mathcal{F}'_2$ by $j'$ so that $c_{\mathcal{E}'_2} = j' \circ c_2$.
Picture:
$$
\xymatrix{
0 \ar[r] &
\mathcal{K}_2 \ar[r] \ar[d]_{j'} &
\mathcal{F}'_2 \ar[r] \ar[d] &
\mathcal{F} \ar[r] \ar[d] & 0 \\
0 \ar[r] &
\mathcal{K} \ar[r] &
\mathcal{E}'_2 \ar[r] &
\mathcal{F} \ar[r] & 0
}
$$
Set $\mathcal{E}'_3 = \mathcal{E}'_2$ but viewed as an
$\mathcal{O}'_3$-module via $\mathcal{O}'_3 \to \mathcal{O}'_2$.
Then $c_{\mathcal{E}'_3} = j \circ c_3$.
The proof of Lemma \ref{lemma-inf-obs-ext-ringed-topoi} constructs
$o(\mathcal{F}, \mathcal{K}_3, c_3)$
as the boundary of the class of the extension of $\mathcal{O}$-modules
$$
0 \to
\mathcal{K}/\mathcal{K}_3 \to
\mathcal{E}'_3/\mathcal{K}_3 \to
\mathcal{F} \to 0
$$
On the other hand, note that $\mathcal{F}'_1 = \mathcal{F}'_2/\mathcal{K}_3$
hence the class $\xi_{\mathcal{F}'_1}$ is the class
of the extension
$$
0 \to \mathcal{K}_2/\mathcal{K}_3 \to \mathcal{F}'_2/\mathcal{K}_3
\to \mathcal{F} \to 0
$$
seen as a sequence of $\mathcal{O}$-modules using $\pi^\sharp$
where $\pi : (\Sh(\mathcal{D}_1), \mathcal{O}'_1) \to
(\Sh(\mathcal{C}), \mathcal{O})$ is the canonical splitting.
Thus finally, the claim follows from the fact that we have
a commutative diagram
$$
\xymatrix{
0 \ar[r] &
\mathcal{K}_2/\mathcal{K}_3 \ar[r] \ar[d] &
\mathcal{F}'_2/\mathcal{K}_3 \ar[r] \ar[d] &
\mathcal{F} \ar[r] \ar[d] & 0 \\
0 \ar[r] &
\mathcal{K}/\mathcal{K}_3 \ar[r] &
\mathcal{E}'_3/\mathcal{K}_3 \ar[r] &
\mathcal{F} \ar[r] & 0
}
$$
which is $\mathcal{O}$-linear (with the $\mathcal{O}$-module
structures given above).
\end{remark}







\section{Infinitesimal deformations of modules on ringed topoi}
\label{section-deformation-modules-ringed-topoi}

\noindent
Let $i : (\Sh(\mathcal{C}), \mathcal{O}) \to (\Sh(\mathcal{D}), \mathcal{O}')$
be a first order thickening of ringed topoi. We freely use the notation
introduced in Section \ref{section-thickenings-ringed-topoi}.
Let $\mathcal{F}'$ be an $\mathcal{O}'$-module
and set $\mathcal{F} = i^*\mathcal{F}'$.
In this situation we have a short exact sequence
$$
0 \to \mathcal{I}\mathcal{F}' \to \mathcal{F}' \to \mathcal{F} \to 0
$$
of $\mathcal{O}'$-modules. Since $\mathcal{I}^2 = 0$ the
$\mathcal{O}'$-module structure on $\mathcal{I}\mathcal{F}'$
comes from a unique $\mathcal{O}$-module structure.
Thus the sequence above is an extension as in
(\ref{equation-extension-ringed-topoi}).
As a special case, if $\mathcal{F}' = \mathcal{O}'$ we have
$i^*\mathcal{O}' = \mathcal{O}$ and
$\mathcal{I}\mathcal{O}' = \mathcal{I}$ and we recover the
sequence of structure sheaves
$$
0 \to \mathcal{I} \to \mathcal{O}' \to \mathcal{O} \to 0
$$

\begin{lemma}
\label{lemma-inf-map-special-ringed-topoi}
Let $i : (\Sh(\mathcal{C}), \mathcal{O}) \to (\Sh(\mathcal{D}), \mathcal{O}')$
be a first order thickening of ringed topoi.
Let $\mathcal{F}'$, $\mathcal{G}'$ be $\mathcal{O}'$-modules.
Set $\mathcal{F} = i^*\mathcal{F}'$ and $\mathcal{G} = i^*\mathcal{G}'$.
Let $\varphi : \mathcal{F} \to \mathcal{G}$ be an $\mathcal{O}$-linear map.
The set of lifts of $\varphi$ to an $\mathcal{O}'$-linear map
$\varphi' : \mathcal{F}' \to \mathcal{G}'$ is, if nonempty, a principal
homogeneous space under
$\Hom_\mathcal{O}(\mathcal{F}, \mathcal{I}\mathcal{G}')$.
\end{lemma}

\begin{proof}
This is a special case of Lemma \ref{lemma-inf-map-ringed-topoi} but we also
give a direct proof. We have short exact sequences of modules
$$
0 \to \mathcal{I} \to \mathcal{O}' \to \mathcal{O} \to 0
\quad\text{and}\quad
0 \to \mathcal{I}\mathcal{G}' \to \mathcal{G}' \to \mathcal{G} \to 0
$$
and similarly for $\mathcal{F}'$.
Since $\mathcal{I}$ has square zero the $\mathcal{O}'$-module
structure on $\mathcal{I}$ and $\mathcal{I}\mathcal{G}'$ comes from
a unique $\mathcal{O}$-module structure. It follows that
$$
\Hom_{\mathcal{O}'}(\mathcal{F}', \mathcal{I}\mathcal{G}') =
\Hom_\mathcal{O}(\mathcal{F}, \mathcal{I}\mathcal{G}')
\quad\text{and}\quad
\Hom_{\mathcal{O}'}(\mathcal{F}', \mathcal{G}) =
\Hom_\mathcal{O}(\mathcal{F}, \mathcal{G})
$$
The lemma now follows from the exact sequence
$$
0 \to \Hom_{\mathcal{O}'}(\mathcal{F}', \mathcal{I}\mathcal{G}') \to
\Hom_{\mathcal{O}'}(\mathcal{F}', \mathcal{G}') \to
\Hom_{\mathcal{O}'}(\mathcal{F}', \mathcal{G})
$$
see Homology, Lemma \ref{homology-lemma-check-exactness}.
\end{proof}

\begin{lemma}
\label{lemma-deform-module-ringed-topoi}
Let $(f, f')$ be a morphism of first order thickenings of ringed topoi
as in Situation \ref{situation-morphism-thickenings-ringed-topoi}.
Let $\mathcal{F}'$ be an $\mathcal{O}'$-module
and set $\mathcal{F} = i^*\mathcal{F}'$.
Assume that $\mathcal{F}$ is flat over $\mathcal{O}_\mathcal{B}$
and that $(f, f')$ is a strict morphism of thickenings
(Definition \ref{definition-strict-morphism-thickenings-ringed-topoi}).
Then the following are equivalent
\begin{enumerate}
\item $\mathcal{F}'$ is flat over $\mathcal{O}_{\mathcal{B}'}$, and
\item the canonical map
$f^*\mathcal{J} \otimes_\mathcal{O} \mathcal{F} \to
\mathcal{I}\mathcal{F}'$
is an isomorphism.
\end{enumerate}
Moreover, in this case the maps
$$
f^*\mathcal{J} \otimes_\mathcal{O} \mathcal{F} \to
\mathcal{I} \otimes_\mathcal{O} \mathcal{F} \to
\mathcal{I}\mathcal{F}'
$$
are isomorphisms.
\end{lemma}

\begin{proof}
The map $f^*\mathcal{J} \to \mathcal{I}$ is surjective
as $(f, f')$ is a strict morphism of thickenings.
Hence the final statement is a consequence of (2).

\medskip\noindent
Proof of the equivalence of (1) and (2). By definition flatness over
$\mathcal{O}_\mathcal{B}$ means flatness over $f^{-1}\mathcal{O}_\mathcal{B}$.
Similarly for flatness over $f^{-1}\mathcal{O}_{\mathcal{B}'}$.
Note that the strictness of $(f, f')$ and the assumption that
$\mathcal{F} = i^*\mathcal{F}'$ imply that
$$
\mathcal{F} = \mathcal{F}'/(f^{-1}\mathcal{J})\mathcal{F}'
$$
as sheaves on $\mathcal{C}$. Moreover, observe that
$f^*\mathcal{J} \otimes_\mathcal{O} \mathcal{F} =
f^{-1}\mathcal{J} \otimes_{f^{-1}\mathcal{O}_\mathcal{B}} \mathcal{F}$.
Hence the equivalence of (1) and (2) follows from
Modules on Sites, Lemma \ref{sites-modules-lemma-flat-over-thickening}.
\end{proof}

\begin{lemma}
\label{lemma-deform-fp-module-ringed-topoi}
Let $(f, f')$ be a morphism of first order thickenings of ringed topoi
as in Situation \ref{situation-morphism-thickenings-ringed-topoi}.
Let $\mathcal{F}'$ be an $\mathcal{O}'$-module
and set $\mathcal{F} = i^*\mathcal{F}'$.
Assume that $\mathcal{F}'$ is flat over $\mathcal{O}_{\mathcal{B}'}$
and that $(f, f')$ is a strict morphism of thickenings.
Then the following are equivalent
\begin{enumerate}
\item $\mathcal{F}'$ is an $\mathcal{O}'$-module of finite presentation, and
\item $\mathcal{F}$ is an $\mathcal{O}$-module of finite presentation.
\end{enumerate}
\end{lemma}

\begin{proof}
The implication (1) $\Rightarrow$ (2) follows from
Modules on Sites, Lemma \ref{sites-modules-lemma-local-pullback}.
For the converse, assume $\mathcal{F}$ of finite presentation.
We may and do assume that $\mathcal{C} = \mathcal{C}'$.
By Lemma \ref{lemma-deform-module-ringed-topoi} we have a short exact sequence
$$
0 \to \mathcal{I} \otimes_{\mathcal{O}_X} \mathcal{F} \to
\mathcal{F}' \to \mathcal{F} \to 0
$$
Let $U$ be an object of $\mathcal{C}$ such that $\mathcal{F}|_U$ has a
presentation
$$
\mathcal{O}_U^{\oplus m} \to \mathcal{O}_U^{\oplus n} \to \mathcal{F}|_U \to 0
$$
After replacing $U$ by the members of a covering we may assume the
map $\mathcal{O}_U^{\oplus n} \to \mathcal{F}|_U$ lifts to a map
$(\mathcal{O}'_U)^{\oplus n} \to \mathcal{F}'|_U$. The induced map
$\mathcal{I}^{\oplus n} \to \mathcal{I} \otimes \mathcal{F}$ is
surjective by right exactness of $\otimes$. Thus after replacing $U$
by the members of a covering we can find a lift
$(\mathcal{O}'|_U)^{\oplus m} \to (\mathcal{O}'|_U)^{\oplus n}$
of the given map $\mathcal{O}_U^{\oplus m} \to \mathcal{O}_U^{\oplus n}$
such that
$$
(\mathcal{O}'_U)^{\oplus m} \to (\mathcal{O}'_U)^{\oplus n} \to
\mathcal{F}'|_U \to 0
$$
is a complex. Using right exactness of $\otimes$ once more it is seen
that this complex is exact.
\end{proof}

\begin{lemma}
\label{lemma-inf-map-rel-ringed-topoi}
Let $(f, f')$ be a morphism of first order thickenings as in
Situation \ref{situation-morphism-thickenings-ringed-topoi}.
Let $\mathcal{F}'$, $\mathcal{G}'$ be $\mathcal{O}'$-modules and set
$\mathcal{F} = i^*\mathcal{F}'$ and $\mathcal{G} = i^*\mathcal{G}'$.
Let $\varphi : \mathcal{F} \to \mathcal{G}$ be an $\mathcal{O}$-linear map.
Assume that $\mathcal{G}'$ is flat over $\mathcal{O}_{\mathcal{B}'}$ and that
$(f, f')$ is a strict morphism of thickenings.
The set of lifts of $\varphi$ to an $\mathcal{O}'$-linear map
$\varphi' : \mathcal{F}' \to \mathcal{G}'$ is, if nonempty, a principal
homogeneous space under
$$
\Hom_\mathcal{O}(\mathcal{F},
\mathcal{G} \otimes_\mathcal{O} f^*\mathcal{J})
$$
\end{lemma}

\begin{proof}
Combine Lemmas \ref{lemma-inf-map-special-ringed-topoi} and
\ref{lemma-deform-module-ringed-topoi}.
\end{proof}

\begin{lemma}
\label{lemma-inf-obs-map-special-ringed-topoi}
Let $i : (\Sh(\mathcal{C}), \mathcal{O}) \to (\Sh(\mathcal{D}), \mathcal{O}')$
be a first order thickening of ringed topoi.
Let $\mathcal{F}'$, $\mathcal{G}'$ be $\mathcal{O}'$-modules and set
$\mathcal{F} = i^*\mathcal{F}'$ and $\mathcal{G} = i^*\mathcal{G}'$.
Let $\varphi : \mathcal{F} \to \mathcal{G}$ be an $\mathcal{O}$-linear map.
There exists an element
$$
o(\varphi) \in
\Ext^1_\mathcal{O}(Li^*\mathcal{F}', \mathcal{I}\mathcal{G}')
$$
whose vanishing is a necessary and sufficient condition for the
existence of a lift of $\varphi$ to an $\mathcal{O}'$-linear map
$\varphi' : \mathcal{F}' \to \mathcal{G}'$.
\end{lemma}

\begin{proof}
It is clear from the proof of Lemma \ref{lemma-inf-map-special-ringed-topoi}
that the vanishing of the boundary of $\varphi$ via the map
$$
\Hom_\mathcal{O}(\mathcal{F}, \mathcal{G}) =
\Hom_{\mathcal{O}'}(\mathcal{F}', \mathcal{G}) \longrightarrow
\Ext^1_{\mathcal{O}'}(\mathcal{F}', \mathcal{I}\mathcal{G}')
$$
is a necessary and sufficient condition for the existence of a lift. We
conclude as
$$
\Ext^1_{\mathcal{O}'}(\mathcal{F}', \mathcal{I}\mathcal{G}') =
\Ext^1_\mathcal{O}(Li^*\mathcal{F}', \mathcal{I}\mathcal{G}')
$$
the adjointness of $i_* = Ri_*$ and $Li^*$ on the derived category
(Cohomology on Sites, Lemma \ref{sites-cohomology-lemma-adjoint}).
\end{proof}

\begin{lemma}
\label{lemma-inf-obs-map-rel-ringed-topoi}
Let $(f, f')$ be a morphism of first order thickenings as in
Situation \ref{situation-morphism-thickenings-ringed-topoi}.
Let $\mathcal{F}'$, $\mathcal{G}'$ be $\mathcal{O}'$-modules and set
$\mathcal{F} = i^*\mathcal{F}'$ and $\mathcal{G} = i^*\mathcal{G}'$.
Let $\varphi : \mathcal{F} \to \mathcal{G}$ be an $\mathcal{O}$-linear map.
Assume that $\mathcal{F}'$ and $\mathcal{G}'$ are flat over
$\mathcal{O}_{\mathcal{B}'}$ and
that $(f, f')$ is a strict morphism of thickenings. There exists an element
$$
o(\varphi) \in
\Ext^1_\mathcal{O}(\mathcal{F},
\mathcal{G} \otimes_\mathcal{O} f^*\mathcal{J})
$$
whose vanishing is a necessary and sufficient condition for the
existence of a lift of $\varphi$ to an $\mathcal{O}'$-linear map
$\varphi' : \mathcal{F}' \to \mathcal{G}'$.
\end{lemma}

\begin{proof}[First proof]
This follows from Lemma \ref{lemma-inf-obs-map-special-ringed-topoi}
as we claim that under the assumptions of the lemma we have
$$
\Ext^1_\mathcal{O}(Li^*\mathcal{F}', \mathcal{I}\mathcal{G}') =
\Ext^1_\mathcal{O}(\mathcal{F},
\mathcal{G} \otimes_\mathcal{O} f^*\mathcal{J})
$$
Namely, we have
$\mathcal{I}\mathcal{G}' =
\mathcal{G} \otimes_\mathcal{O} f^*\mathcal{J}$
by Lemma \ref{lemma-deform-module-ringed-topoi}.
On the other hand, observe that
$$
H^{-1}(Li^*\mathcal{F}') =
\text{Tor}_1^{\mathcal{O}'}(\mathcal{F}', \mathcal{O})
$$
(local computation omitted). Using the short exact sequence
$$
0 \to \mathcal{I} \to \mathcal{O}' \to \mathcal{O} \to 0
$$
we see that this $\text{Tor}_1$ is computed by the kernel of the map
$\mathcal{I} \otimes_\mathcal{O} \mathcal{F} \to \mathcal{I}\mathcal{F}'$
which is zero by the final assertion of
Lemma \ref{lemma-deform-module-ringed-topoi}.
Thus $\tau_{\geq -1}Li^*\mathcal{F}' = \mathcal{F}$.
On the other hand, we have
$$
\Ext^1_\mathcal{O}(Li^*\mathcal{F}',
\mathcal{I}\mathcal{G}') =
\Ext^1_\mathcal{O}(\tau_{\geq -1}Li^*\mathcal{F}',
\mathcal{I}\mathcal{G}')
$$
by the dual of
Derived Categories, Lemma \ref{derived-lemma-negative-vanishing}.
\end{proof}

\begin{proof}[Second proof]
We can apply Lemma \ref{lemma-inf-obs-map-ringed-topoi} as follows. Note that
$\mathcal{K} = \mathcal{I} \otimes_\mathcal{O} \mathcal{F}$ and
$\mathcal{L} = \mathcal{I} \otimes_\mathcal{O} \mathcal{G}$
by Lemma \ref{lemma-deform-module-ringed-topoi}, that
$c_{\mathcal{F}'} = 1 \otimes 1$ and $c_{\mathcal{G}'} = 1 \otimes 1$
and taking $\psi = 1 \otimes \varphi$ the diagram of the lemma
commutes. Thus $o(\varphi) = o(\varphi, 1 \otimes \varphi)$
works.
\end{proof}

\begin{lemma}
\label{lemma-inf-ext-rel-ringed-topoi}
Let $(f, f')$ be a morphism of first order thickenings as in
Situation \ref{situation-morphism-thickenings-ringed-topoi}.
Let $\mathcal{F}$ be an $\mathcal{O}$-module.
Assume $(f, f')$ is a strict morphism of thickenings and
$\mathcal{F}$ flat over $\mathcal{O}_\mathcal{B}$. If there exists a pair
$(\mathcal{F}', \alpha)$ consisting of an
$\mathcal{O}'$-module $\mathcal{F}'$ flat over $\mathcal{O}_{\mathcal{B}'}$
and an isomorphism
$\alpha : i^*\mathcal{F}' \to \mathcal{F}$, then the set of
isomorphism classes of such pairs is principal homogeneous
under
$\Ext^1_\mathcal{O}(
\mathcal{F}, \mathcal{I} \otimes_\mathcal{O} \mathcal{F})$.
\end{lemma}

\begin{proof}
If we assume there exists one such module, then the canonical map
$$
f^*\mathcal{J} \otimes_\mathcal{O} \mathcal{F} \to
\mathcal{I} \otimes_\mathcal{O} \mathcal{F}
$$
is an isomorphism by Lemma \ref{lemma-deform-module-ringed-topoi}. Apply
Lemma \ref{lemma-inf-ext-ringed-topoi} with $\mathcal{K} = 
\mathcal{I} \otimes_\mathcal{O} \mathcal{F}$
and $c = 1$. By Lemma \ref{lemma-deform-module-ringed-topoi}
the corresponding extensions
$\mathcal{F}'$ are all flat over $\mathcal{O}_{\mathcal{B}'}$.
\end{proof}

\begin{lemma}
\label{lemma-inf-obs-ext-rel-ringed-topoi}
Let $(f, f')$ be a morphism of first order thickenings as in
Situation \ref{situation-morphism-thickenings-ringed-topoi}.
Let $\mathcal{F}$ be an $\mathcal{O}$-module. Assume
$(f, f')$ is a strict morphism of thickenings
and $\mathcal{F}$ flat over $\mathcal{O}_\mathcal{B}$. There exists an
$\mathcal{O}'$-module $\mathcal{F}'$ flat over $\mathcal{O}_{\mathcal{B}'}$
with $i^*\mathcal{F}' \cong \mathcal{F}$, if and only if
\begin{enumerate}
\item the canonical map
$f^*\mathcal{J} \otimes_\mathcal{O} \mathcal{F} \to
\mathcal{I} \otimes_\mathcal{O} \mathcal{F}$
is an isomorphism, and
\item the class
$o(\mathcal{F}, \mathcal{I} \otimes_\mathcal{O} \mathcal{F}, 1)
\in \Ext^2_\mathcal{O}(
\mathcal{F}, \mathcal{I} \otimes_\mathcal{O} \mathcal{F})$
of Lemma \ref{lemma-inf-obs-ext-ringed-topoi} is zero.
\end{enumerate}
\end{lemma}

\begin{proof}
This follows immediately from the characterization of
$\mathcal{O}'$-modules flat over $\mathcal{O}_{\mathcal{B}'}$ of 
Lemma \ref{lemma-deform-module-ringed-topoi} and
Lemma \ref{lemma-inf-obs-ext-ringed-topoi}.
\end{proof}






\section{Application to flat modules on flat thickenings of ringed topoi}
\label{section-flat-ringed-topoi}

\noindent
Consider a commutative diagram
$$
\xymatrix{
(\Sh(\mathcal{C}), \mathcal{O}) \ar[r]_i \ar[d]_f &
(\Sh(\mathcal{D}), \mathcal{O}') \ar[d]^{f'} \\
(\Sh(\mathcal{B}), \mathcal{O}_\mathcal{B}) \ar[r]^t &
(\Sh(\mathcal{B}'), \mathcal{O}_{\mathcal{B}'})
}
$$
of ringed topoi whose horizontal arrows are first order thickenings
as in Situation \ref{situation-morphism-thickenings-ringed-topoi}. Set
$\mathcal{I} = \Ker(i^\sharp) \subset \mathcal{O}'$ and
$\mathcal{J} = \Ker(t^\sharp) \subset \mathcal{O}_{\mathcal{B}'}$.
Let $\mathcal{F}$ be an $\mathcal{O}$-module. Assume that
\begin{enumerate}
\item $(f, f')$ is a strict morphism of thickenings,
\item $f'$ is flat, and
\item $\mathcal{F}$ is flat over $\mathcal{O}_\mathcal{B}$.
\end{enumerate}
Note that (1) $+$ (2) imply that $\mathcal{I} = f^*\mathcal{J}$
(apply Lemma \ref{lemma-deform-module-ringed-topoi} to $\mathcal{O}'$).
The theory of the preceding section is especially nice
under these assumptions. We summarize the results already obtained
in the following lemma.

\begin{lemma}
\label{lemma-flat-ringed-topoi}
In the situation above.
\begin{enumerate}
\item There exists an $\mathcal{O}'$-module $\mathcal{F}'$ flat over
$\mathcal{O}_{\mathcal{B}'}$ with $i^*\mathcal{F}' \cong \mathcal{F}$,
if and only if
the class $o(\mathcal{F}, f^*\mathcal{J} \otimes_\mathcal{O} \mathcal{F}, 1)
\in \Ext^2_\mathcal{O}(
\mathcal{F}, f^*\mathcal{J} \otimes_\mathcal{O} \mathcal{F})$
of Lemma \ref{lemma-inf-obs-ext-ringed-topoi} is zero.
\item If such a module exists, then the set of isomorphism classes
of lifts is principal homogeneous under
$\Ext^1_\mathcal{O}(
\mathcal{F}, f^*\mathcal{J} \otimes_\mathcal{O} \mathcal{F})$.
\item Given a lift $\mathcal{F}'$, the set of automorphisms of
$\mathcal{F}'$ which pull back to $\text{id}_\mathcal{F}$ is canonically
isomorphic to $\Ext^0_\mathcal{O}(
\mathcal{F}, f^*\mathcal{J} \otimes_\mathcal{O} \mathcal{F})$.
\end{enumerate}
\end{lemma}

\begin{proof}
Part (1) follows from Lemma \ref{lemma-inf-obs-ext-rel-ringed-topoi}
as we have seen above that $\mathcal{I} = f^*\mathcal{J}$.
Part (2) follows from Lemma \ref{lemma-inf-ext-rel-ringed-topoi}.
Part (3) follows from Lemma \ref{lemma-inf-map-rel-ringed-topoi}.
\end{proof}

\begin{situation}
\label{situation-morphism-flat-thickenings-ringed-topoi}
Let $f : (\Sh(\mathcal{C}), \mathcal{O}) \to
(\Sh(\mathcal{B}), \mathcal{O}_\mathcal{B})$ be a morphism of
ringed topoi. Consider a commutative diagram
$$
\xymatrix{
(\Sh(\mathcal{C}'_1), \mathcal{O}'_1) \ar[r]_h \ar[d]_{f'_1} &
(\Sh(\mathcal{C}'_2), \mathcal{O}'_2) \ar[d]_{f'_2} \\
(\Sh(\mathcal{B}'_1), \mathcal{O}_{\mathcal{B}'_1}) \ar[r] &
(\Sh(\mathcal{B}'_2), \mathcal{O}_{\mathcal{B}'_2})
}
$$
where $h$ is a morphism of first order thickenings
of $(\Sh(\mathcal{C}), \mathcal{O})$, the lower horizontal arrow
is a morphism of first order thickenings of
$(\Sh(\mathcal{B}), \mathcal{O}_\mathcal{B})$, each $f'_i$ restricts
to $f$, both pairs $(f, f_i')$ are strict morphisms of thickenings, and
both $f'_i$ are flat. Finally, let $\mathcal{F}$ be an
$\mathcal{O}$-module flat over $\mathcal{O}_\mathcal{B}$.
\end{situation}

\begin{lemma}
\label{lemma-functorial-ringed-topoi}
In Situation \ref{situation-morphism-flat-thickenings-ringed-topoi}
the obstruction class
$o(\mathcal{F}, f^*\mathcal{J}_2 \otimes_\mathcal{O} \mathcal{F}, 1)$
maps to the obstruction class
$o(\mathcal{F}, f^*\mathcal{J}_1 \otimes_\mathcal{O} \mathcal{F}, 1)$
under the canonical map
$$
\Ext^2_\mathcal{O}(
\mathcal{F}, f^*\mathcal{J}_2 \otimes_\mathcal{O} \mathcal{F})
\to \Ext^2_\mathcal{O}(
\mathcal{F}, f^*\mathcal{J}_1 \otimes_\mathcal{O} \mathcal{F})
$$
\end{lemma}

\begin{proof}
Follows from Remark \ref{remark-obstruction-extension-functorial-ringed-topoi}.
\end{proof}

\begin{situation}
\label{situation-ses-flat-thickenings-ringed-topoi}
Let $f : (\Sh(\mathcal{C}), \mathcal{O}) \to
(\Sh(\mathcal{B}), \mathcal{O}_\mathcal{B})$ be a morphism of
ringed topoi. Consider a commutative diagram
$$
\xymatrix{
(\Sh(\mathcal{C}'_1), \mathcal{O}'_1) \ar[r]_h \ar[d]_{f'_1} &
(\Sh(\mathcal{C}'_2), \mathcal{O}'_2) \ar[r] \ar[d]_{f'_2} &
(\Sh(\mathcal{C}'_3), \mathcal{O}'_3) \ar[d]_{f'_3} \\
(\Sh(\mathcal{B}'_1), \mathcal{O}_{\mathcal{B}'_1}) \ar[r] &
(\Sh(\mathcal{B}'_2), \mathcal{O}_{\mathcal{B}'_2}) \ar[r] &
(\Sh(\mathcal{B}'_3), \mathcal{O}_{\mathcal{B}'_3})
}
$$
where (a) the top row is a short exact sequence of first order thickenings
of $(\Sh(\mathcal{C}), \mathcal{O})$, (b) the lower row is a short exact
sequence of first order thickenings of
$(\Sh(\mathcal{B}), \mathcal{O}_\mathcal{B})$, (c) each $f'_i$ restricts
to $f$, (d) each pair $(f, f_i')$ is a strict morphism of thickenings, and
(e) each $f'_i$ is flat. Finally, let $\mathcal{F}'_2$ be an
$\mathcal{O}'_2$-module flat over $\mathcal{O}_{\mathcal{B}'_2}$ and
set $\mathcal{F} = \mathcal{F}'_2 \otimes \mathcal{O}$. Let
$\pi : (\Sh(\mathcal{C}'_1), \mathcal{O}'_1) \to
(\Sh(\mathcal{C}), \mathcal{O})$ be the canonical splitting
(Remark \ref{remark-short-exact-sequence-thickenings-ringed-topoi}).
\end{situation}

\begin{lemma}
\label{lemma-verify-iv-ringed-topoi}
In Situation \ref{situation-ses-flat-thickenings-ringed-topoi} the modules
$\pi^*\mathcal{F}$ and $h^*\mathcal{F}'_2$ are $\mathcal{O}'_1$-modules
flat over $\mathcal{O}_{\mathcal{B}'_1}$ restricting to $\mathcal{F}$ on
$(\Sh(\mathcal{C}), \mathcal{O})$.
Their difference (Lemma \ref{lemma-flat-ringed-topoi}) is an element
$\theta$ of
$\Ext^1_\mathcal{O}(\mathcal{F},
f^*\mathcal{J}_1 \otimes_\mathcal{O} \mathcal{F})$
whose boundary in
$\Ext^2_\mathcal{O}(\mathcal{F},
f^*\mathcal{J}_3 \otimes_\mathcal{O} \mathcal{F})$
equals the obstruction (Lemma \ref{lemma-flat-ringed-topoi})
to lifting $\mathcal{F}$ to an $\mathcal{O}'_3$-module flat over
$\mathcal{O}_{\mathcal{B}'_3}$.
\end{lemma}

\begin{proof}
Note that both $\pi^*\mathcal{F}$ and $h^*\mathcal{F}'_2$
restrict to $\mathcal{F}$ on $(\Sh(\mathcal{C}), \mathcal{O})$
and that the kernels of
$\pi^*\mathcal{F} \to \mathcal{F}$ and $h^*\mathcal{F}'_2 \to \mathcal{F}$
are given by $f^*\mathcal{J}_1 \otimes_\mathcal{O} \mathcal{F}$.
Hence flatness by Lemma \ref{lemma-deform-module-ringed-topoi}.
Taking the boundary makes sense as the sequence of modules
$$
0 \to f^*\mathcal{J}_3 \otimes_\mathcal{O} \mathcal{F} \to
f^*\mathcal{J}_2 \otimes_\mathcal{O} \mathcal{F} \to
f^*\mathcal{J}_1 \otimes_\mathcal{O} \mathcal{F} \to 0
$$
is short exact due to the assumptions in
Situation \ref{situation-ses-flat-thickenings-ringed-topoi}
and the fact that $\mathcal{F}$ is flat over $\mathcal{O}_\mathcal{B}$.
The statement on the obstruction class is a direct translation
of the result of
Remark \ref{remark-complex-thickenings-and-ses-modules-ringed-topoi}
to this particular situation.
\end{proof}







\section{Deformations of ringed topoi and the naive cotangent complex}
\label{section-deformations-ringed-topoi}

\noindent
In this section we use the naive cotangent complex to do a little bit
of deformation theory. We start with a first order thickening
$t : (\Sh(\mathcal{B}), \mathcal{O}_\mathcal{B}) \to
(\Sh(\mathcal{B}'), \mathcal{O}_{\mathcal{B}'})$ of ringed topoi.
We denote $\mathcal{J} = \Ker(t^\sharp)$ and we
identify the underlying topoi of $\mathcal{B}$ and $\mathcal{B}'$.
Moreover we assume given a morphism of ringed topoi
$f : (\Sh(\mathcal{C}), \mathcal{O}) \to
(\Sh(\mathcal{B}), \mathcal{O}_\mathcal{B})$, an $\mathcal{O}$-module
$\mathcal{G}$, and a map $f^{-1}\mathcal{J} \to \mathcal{G}$
of sheaves of $f^{-1}\mathcal{O}_\mathcal{B}$-modules.
In this section we ask ourselves whether we can find
the question mark fitting into the following diagram
\begin{equation}
\label{equation-to-solve-ringed-topoi}
\vcenter{
\xymatrix{
0 \ar[r] & \mathcal{G} \ar[r] & {?} \ar[r] & \mathcal{O} \ar[r] & 0 \\
0 \ar[r] & f^{-1}\mathcal{J} \ar[u]^c \ar[r] &
f^{-1}\mathcal{O}_{\mathcal{B}'} \ar[u] \ar[r] &
f^{-1}\mathcal{O}_\mathcal{B} \ar[u] \ar[r] & 0
}
}
\end{equation}
and moreover how unique the solution is (if it exists). More precisely,
we look for a first order thickening
$i : (\Sh(\mathcal{C}), \mathcal{O}) \to (\Sh(\mathcal{C}'), \mathcal{O}')$
and a morphism of thickenings $(f, f')$ as in
(\ref{equation-morphism-thickenings-ringed-topoi})
where $\Ker(i^\sharp)$ is identified with $\mathcal{G}$
such that $(f')^\sharp$ induces the given map $c$.
We will say $(\Sh(\mathcal{C}'), \mathcal{O}')$ is a {\it solution} to
(\ref{equation-to-solve-ringed-topoi}).


\begin{lemma}
\label{lemma-huge-diagram-ringed-topoi}
Assume given a commutative diagram of morphisms ringed topoi
\begin{equation}
\label{equation-huge-1-ringed-topoi}
\vcenter{
\xymatrix{
& (\Sh(\mathcal{C}_2), \mathcal{O}_2) \ar[r]_{i_2} \ar[d]_{f_2} \ar[ddl]_g &
(\Sh(\mathcal{C}'_2), \mathcal{O}'_2) \ar[d]^{f'_2} \\
&
(\Sh(\mathcal{B}_2), \mathcal{O}_{\mathcal{B}_2}) \ar[r]^{t_2} \ar[ddl]|\hole &
(\Sh(\mathcal{B}'_2), \mathcal{O}_{\mathcal{B}'_2}) \ar[ddl] \\
(\Sh(\mathcal{C}_1), \mathcal{O}_1) \ar[r]_{i_1} \ar[d]_{f_1} &
(\Sh(\mathcal{C}'_1), \mathcal{O}'_1) \ar[d]^{f'_1} \\
(\Sh(\mathcal{B}_1), \mathcal{O}_{\mathcal{B}_1}) \ar[r]^{t_1} &
(\Sh(\mathcal{B}'_1), \mathcal{O}_{\mathcal{B}'_1})
}
}
\end{equation}
whose horizontal arrows are first order thickenings. Set
$\mathcal{G}_j = \Ker(i_j^\sharp)$ and assume given a
map of $g^{-1}\mathcal{O}_1$-modules
$\nu : g^{-1}\mathcal{G}_1 \to \mathcal{G}_2$
giving rise to the commutative diagram
\begin{equation}
\label{equation-huge-2-ringed-topoi}
\vcenter{
\xymatrix{
& 0 \ar[r] & \mathcal{G}_2 \ar[r] &
\mathcal{O}'_2 \ar[r] &
\mathcal{O}_2 \ar[r] & 0 \\
& 0 \ar[r]|\hole &
f_2^{-1}\mathcal{J}_2 \ar[u]_{c_2} \ar[r] &
f_2^{-1}\mathcal{O}_{\mathcal{B}'_2} \ar[u] \ar[r]|\hole &
f_2^{-1}\mathcal{O}_{\mathcal{B}_2} \ar[u] \ar[r] & 0 \\
0 \ar[r] &
\mathcal{G}_1 \ar[ruu] \ar[r] &
\mathcal{O}'_1 \ar[r] &
\mathcal{O}_1 \ar[ruu] \ar[r] & 0 \\
0 \ar[r] &
f_1^{-1}\mathcal{J}_1 \ar[ruu]|\hole \ar[u]^{c_1} \ar[r] &
f_1^{-1}\mathcal{O}_{\mathcal{B}'_1} \ar[ruu]|\hole \ar[u] \ar[r] &
f_1^{-1}\mathcal{O}_{\mathcal{B}_1} \ar[ruu]|\hole \ar[u] \ar[r] & 0
}
}
\end{equation}
with front and back solutions to (\ref{equation-to-solve-ringed-topoi}).
(The north-north-west arrows are maps on $\mathcal{C}_2$ after applying
$g^{-1}$ to the source.)
\begin{enumerate}
\item There exist a canonical element in
$\Ext^1_{\mathcal{O}_2}(
Lg^*\NL_{\mathcal{O}_1/\mathcal{O}_{\mathcal{B}_1}}, \mathcal{G}_2)$
whose vanishing is a necessary and sufficient condition for the existence
of a morphism of ringed topoi
$(\Sh(\mathcal{C}'_2), \mathcal{O}'_2) \to
(\Sh(\mathcal{C}'_1), \mathcal{O}'_1)$ fitting into
(\ref{equation-huge-1-ringed-topoi}) compatibly with $\nu$.
\item If there exists a morphism
$(\Sh(\mathcal{C}'_2), \mathcal{O}'_2) \to
(\Sh(\mathcal{C}'_1), \mathcal{O}'_1)$
fitting into
(\ref{equation-huge-1-ringed-topoi}) compatibly with $\nu$ the set
of all such morphisms is a principal homogeneous space under
$$
\Hom_{\mathcal{O}_1}(
\Omega_{\mathcal{O}_1/\mathcal{O}_{\mathcal{B}_1}}, g_*\mathcal{G}_2) =
\Hom_{\mathcal{O}_2}(
g^*\Omega_{\mathcal{O}_1/\mathcal{O}_{\mathcal{B}_1}}, \mathcal{G}_2) =
\Ext^0_{\mathcal{O}_2}(
Lg^*\NL_{\mathcal{O}_1/\mathcal{O}_{\mathcal{B}_1}}, \mathcal{G}_2).
$$
\end{enumerate}
\end{lemma}

\begin{proof}
The proof of this lemma is identical to the proof of
Lemma \ref{lemma-huge-diagram-ringed-spaces}.
We urge the reader to read that proof instead of this one.
We will identify the underlying topoi for every
thickening in sight (we have already used this convention
in the statement). The equalities in the last statement of the
lemma are immediate from the definitions. Thus we will work with the groups
$\Ext^k_{\mathcal{O}_2}(
Lg^*\NL_{\mathcal{O}_1/\mathcal{O}_{\mathcal{B}_1}}, \mathcal{G}_2)$,
$k = 0, 1$ in the rest of the proof. We first argue that we can reduce
to the case where the underlying topos of all ringed topoi in the lemma
is the same.

\medskip\noindent
To do this, observe that
$g^{-1}\NL_{\mathcal{O}_1/\mathcal{O}_{\mathcal{B}_1}}$ is equal to the naive
cotangent complex of the homomorphism of sheaves of rings
$g^{-1}f_1^{-1}\mathcal{O}_{\mathcal{B}_1} \to g^{-1}\mathcal{O}_1$, see
Modules on Sites, Lemma \ref{sites-modules-lemma-pullback-differentials}.
Moreover, the degree $0$ term of
$\NL_{\mathcal{O}_1/\mathcal{O}_{\mathcal{B}_1}}$ is a flat
$\mathcal{O}_1$-module, hence the canonical map
$$
Lg^*\NL_{\mathcal{O}_1/\mathcal{O}_{\mathcal{B}_1}}
\longrightarrow
g^{-1}\NL_{\mathcal{O}_1/\mathcal{O}_{\mathcal{B}_1}}
\otimes_{g^{-1}\mathcal{O}_1} \mathcal{O}_2
$$
induces an isomorphism on cohomology sheaves in degrees $0$ and $-1$.
Thus we may replace the Ext groups of the lemma with
$$
\Ext^k_{g^{-1}\mathcal{O}_1}(
g^{-1}\NL_{\mathcal{O}_1/\mathcal{O}_{\mathcal{B}_1}}, \mathcal{G}_2) =
\Ext^k_{g^{-1}\mathcal{O}_1}(
\NL_{g^{-1}\mathcal{O}_1/g^{-1}f_1^{-1}\mathcal{O}_{\mathcal{B}_1}},
\mathcal{G}_2)
$$
The set of morphism of ringed topoi
$(\Sh(\mathcal{C}'_2), \mathcal{O}'_2) \to
(\Sh(\mathcal{C}'_1), \mathcal{O}'_1)$ fitting into
(\ref{equation-huge-1-ringed-topoi}) compatibly with $\nu$ is in
one-to-one bijection with the set of homomorphisms of
$g^{-1}f_1^{-1}\mathcal{O}_{\mathcal{B}'_1}$-algebras
$g^{-1}\mathcal{O}'_1 \to \mathcal{O}'_2$ which are compatible with
$f^\sharp$ and $\nu$. In this way we see that we may assume we have a
diagram (\ref{equation-huge-2-ringed-topoi}) of sheaves on a site
$\mathcal{C}$ (with $f_1 = f_2 = \text{id}$ on underlying topoi)
and we are looking to find a homomorphism of sheaves of rings
$\mathcal{O}'_1 \to \mathcal{O}'_2$ fitting into it.

\medskip\noindent
In the rest of the proof of the lemma we assume
all underlying topological spaces are the
same, i.e., we have a diagram (\ref{equation-huge-2-ringed-topoi})
of sheaves on a site $\mathcal{C}$ (with $f_1 = f_2 = \text{id}$
on underlying topoi) and we are looking for
homomorphisms of sheaves of rings
$\mathcal{O}'_1 \to \mathcal{O}'_2$ fitting into it.
As ext groups we will use
$\Ext^k_{\mathcal{O}_1}(
\NL_{\mathcal{O}_1/\mathcal{O}_{\mathcal{B}_1}}, \mathcal{G}_2)$, $k = 0, 1$.

\medskip\noindent
Step 1. Construction of the obstruction class. Consider the sheaf
of sets
$$
\mathcal{E} = \mathcal{O}'_1 \times_{\mathcal{O}_2} \mathcal{O}'_2
$$
This comes with a surjective map $\alpha : \mathcal{E} \to \mathcal{O}_1$
and hence we can use $\NL(\alpha)$ instead of
$\NL_{\mathcal{O}_1/\mathcal{O}_{\mathcal{B}_1}}$, see
Modules on Sites, Lemma \ref{sites-modules-lemma-NL-up-to-qis}.
Set
$$
\mathcal{I}' =
\Ker(\mathcal{O}_{\mathcal{B}'_1}[\mathcal{E}] \to \mathcal{O}_1)
\quad\text{and}\quad
\mathcal{I} =
\Ker(\mathcal{O}_{\mathcal{B}_1}[\mathcal{E}] \to \mathcal{O}_1)
$$
There is a surjection $\mathcal{I}' \to \mathcal{I}$ whose kernel
is $\mathcal{J}_1\mathcal{O}_{\mathcal{B}'_1}[\mathcal{E}]$.
We obtain two homomorphisms of $\mathcal{O}_{\mathcal{B}'_2}$-algebras
$$
a : \mathcal{O}_{\mathcal{B}'_1}[\mathcal{E}] \to \mathcal{O}'_1
\quad\text{and}\quad
b : \mathcal{O}_{\mathcal{B}'_1}[\mathcal{E}] \to \mathcal{O}'_2
$$
which induce maps $a|_{\mathcal{I}'} : \mathcal{I}' \to \mathcal{G}_1$ and
$b|_{\mathcal{I}'} : \mathcal{I}' \to \mathcal{G}_2$. Both $a$ and $b$
annihilate $(\mathcal{I}')^2$. Moreover $a$ and $b$ agree on
$\mathcal{J}_1\mathcal{O}_{\mathcal{B}'_1}[\mathcal{E}]$
as maps into $\mathcal{G}_2$
because the left hand square of (\ref{equation-huge-2-ringed-topoi})
is commutative. Thus the difference
$b|_{\mathcal{I}'} - \nu \circ a|_{\mathcal{I}'}$
induces a well defined $\mathcal{O}_1$-linear map
$$
\xi : \mathcal{I}/\mathcal{I}^2 \longrightarrow \mathcal{G}_2
$$
which sends the class of a local section $f$ of $\mathcal{I}$ to
$a(f') - \nu(b(f'))$ where $f'$ is a lift of $f$ to a local
section of $\mathcal{I}'$. We let
$[\xi] \in \Ext^1_{\mathcal{O}_1}(\NL(\alpha), \mathcal{G}_2)$
be the image (see below).

\medskip\noindent
Step 2. Vanishing of $[\xi]$ is necessary. Let us write $\Omega =
\Omega_{\mathcal{O}_{\mathcal{B}_1}[\mathcal{E}]/\mathcal{O}_{\mathcal{B}_1}}
\otimes_{\mathcal{O}_{\mathcal{B}_1}[\mathcal{E}]} \mathcal{O}_1$.
Observe that $\NL(\alpha) = (\mathcal{I}/\mathcal{I}^2 \to \Omega)$
fits into a distinguished triangle
$$
\Omega[0] \to
\NL(\alpha) \to
\mathcal{I}/\mathcal{I}^2[1] \to
\Omega[1]
$$
Thus we see that $[\xi]$ is zero if and only if $\xi$
is a composition $\mathcal{I}/\mathcal{I}^2 \to \Omega \to \mathcal{G}_2$
for some map $\Omega \to \mathcal{G}_2$. Suppose there exists a
homomorphisms of sheaves of rings
$\varphi : \mathcal{O}'_1 \to \mathcal{O}'_2$ fitting into
(\ref{equation-huge-2-ringed-topoi}). In this case consider the map
$\mathcal{O}'_1[\mathcal{E}] \to \mathcal{G}_2$,
$f' \mapsto b(f') - \varphi(a(f'))$. A calculation
shows this annihilates $\mathcal{J}_1\mathcal{O}_{\mathcal{B}'_1}[\mathcal{E}]$
and induces a derivation
$\mathcal{O}_{\mathcal{B}_1}[\mathcal{E}] \to \mathcal{G}_2$.
The resulting linear map $\Omega \to \mathcal{G}_2$ witnesses the
fact that $[\xi] = 0$ in this case.

\medskip\noindent
Step 3. Vanishing of $[\xi]$ is sufficient. Let
$\theta : \Omega \to \mathcal{G}_2$ be a $\mathcal{O}_1$-linear map
such that $\xi$ is equal to
$\theta \circ (\mathcal{I}/\mathcal{I}^2 \to \Omega)$.
Then a calculation shows that
$$
b + \theta \circ d :
\mathcal{O}_{\mathcal{B}'_1}[\mathcal{E}]
\longrightarrow
\mathcal{O}'_2
$$
annihilates $\mathcal{I}'$ and hence defines a map
$\mathcal{O}'_1 \to \mathcal{O}'_2$ fitting into
(\ref{equation-huge-2-ringed-topoi}).

\medskip\noindent
Proof of (2) in the special case above. Omitted. Hint:
This is exactly the same as the proof of (2) of
Lemma \ref{lemma-huge-diagram}.
\end{proof}

\begin{lemma}
\label{lemma-NL-represent-ext-class-ringed-topoi}
Let $\mathcal{C}$ be a site. Let $\mathcal{A} \to \mathcal{B}$ be a
homomorphism of sheaves of rings on $\mathcal{C}$.
Let $\mathcal{G}$ be a $\mathcal{B}$-module.
Let
$\xi \in \Ext^1_\mathcal{B}(\NL_{\mathcal{B}/\mathcal{A}}, \mathcal{G})$. 
There exists a map of sheaves of sets $\alpha : \mathcal{E} \to \mathcal{B}$
such that $\xi \in \Ext^1_\mathcal{B}(\NL(\alpha), \mathcal{G})$
is the class of a map $\mathcal{I}/\mathcal{I}^2 \to \mathcal{G}$
(see proof for notation).
\end{lemma}

\begin{proof}
Recall that given $\alpha : \mathcal{E} \to \mathcal{B}$
such that $\mathcal{A}[\mathcal{E}] \to \mathcal{B}$ is surjective
with kernel $\mathcal{I}$ the complex
$\NL(\alpha) = (\mathcal{I}/\mathcal{I}^2 \to 
\Omega_{\mathcal{A}[\mathcal{E}]/\mathcal{A}}
\otimes_{\mathcal{A}[\mathcal{E}]} \mathcal{B})$ is canonically
isomorphic to $\NL_{\mathcal{B}/\mathcal{A}}$, see
Modules on Sites, Lemma \ref{sites-modules-lemma-NL-up-to-qis}.
Observe moreover, that
$\Omega = \Omega_{\mathcal{A}[\mathcal{E}]/\mathcal{A}}
\otimes_{\mathcal{A}[\mathcal{E}]} \mathcal{B}$ is the sheaf
associated to the presheaf
$U \mapsto \bigoplus_{e \in \mathcal{E}(U)} \mathcal{B}(U)$.
In other words, $\Omega$ is the free $\mathcal{B}$-module on the
sheaf of sets $\mathcal{E}$ and in particular there is a canonical
map $\mathcal{E} \to \Omega$.

\medskip\noindent
Having said this, pick some $\mathcal{E}$ (for example
$\mathcal{E} = \mathcal{B}$ as in the definition of the naive cotangent
complex). The obstruction to writing $\xi$ as the class of a map
$\mathcal{I}/\mathcal{I}^2 \to \mathcal{G}$ is an element in
$\Ext^1_\mathcal{B}(\Omega, \mathcal{G})$. Say this is represented
by the extension $0 \to \mathcal{G} \to \mathcal{H} \to \Omega \to 0$
of $\mathcal{B}$-modules. Consider the sheaf of sets
$\mathcal{E}' = \mathcal{E} \times_\Omega \mathcal{H}$
which comes with an induced map $\alpha' : \mathcal{E}' \to \mathcal{B}$.
Let $\mathcal{I}' = \Ker(\mathcal{A}[\mathcal{E}'] \to \mathcal{B})$
and $\Omega' = \Omega_{\mathcal{A}[\mathcal{E}']/\mathcal{A}}
\otimes_{\mathcal{A}[\mathcal{E}']} \mathcal{B}$.
The pullback of $\xi$ under the quasi-isomorphism
$\NL(\alpha') \to \NL(\alpha)$ maps to zero in
$\Ext^1_\mathcal{B}(\Omega', \mathcal{G})$
because the pullback of the extension $\mathcal{H}$
by the map $\Omega' \to \Omega$ is split as $\Omega'$ is the free
$\mathcal{B}$-module on the sheaf of sets $\mathcal{E}'$ and since
by construction there is a commutative diagram
$$
\xymatrix{
\mathcal{E}' \ar[r] \ar[d] & \mathcal{E} \ar[d] \\
\mathcal{H} \ar[r] & \Omega
}
$$
This finishes the proof.
\end{proof}

\begin{lemma}
\label{lemma-choices-ringed-topoi}
If there exists a solution to (\ref{equation-to-solve-ringed-topoi}),
then the set of isomorphism classes of solutions is principal homogeneous
under $\Ext^1_\mathcal{O}(
\NL_{\mathcal{O}/\mathcal{O}_\mathcal{B}}, \mathcal{G})$.
\end{lemma}

\begin{proof}
We observe right away that given two solutions $\mathcal{O}'_1$ and
$\mathcal{O}'_2$ to (\ref{equation-to-solve-ringed-topoi}) we obtain by
Lemma \ref{lemma-huge-diagram-ringed-topoi} an obstruction element
$o(\mathcal{O}'_1, \mathcal{O}'_2) \in \Ext^1_\mathcal{O}(
\NL_{\mathcal{O}/\mathcal{O}_\mathcal{B}}, \mathcal{G})$
to the existence of a map $\mathcal{O}'_1 \to \mathcal{O}'_2$.
Clearly, this element
is the obstruction to the existence of an isomorphism, hence separates
the isomorphism classes. To finish the proof it therefore suffices to
show that given a solution $\mathcal{O}'$ and an element
$\xi \in \Ext^1_\mathcal{O}(
\NL_{\mathcal{O}/\mathcal{O}_\mathcal{B}}, \mathcal{G})$
we can find a second solution $\mathcal{O}'_\xi$ such that
$o(\mathcal{O}', \mathcal{O}'_\xi) = \xi$.

\medskip\noindent
Pick $\alpha : \mathcal{E} \to \mathcal{O}$ as in
Lemma \ref{lemma-NL-represent-ext-class-ringed-topoi}
for the class $\xi$. Consider the surjection
$f^{-1}\mathcal{O}_\mathcal{B}[\mathcal{E}] \to \mathcal{O}$
with kernel $\mathcal{I}$ and corresponding naive cotangent complex
$\NL(\alpha) = (\mathcal{I}/\mathcal{I}^2 \to
\Omega_{f^{-1}\mathcal{O}_\mathcal{B}[\mathcal{E}]/
f^{-1}\mathcal{O}_\mathcal{B}}
\otimes_{f^{-1}\mathcal{O}_\mathcal{B}[\mathcal{E}]} \mathcal{O})$.
By the lemma $\xi$ is the class of a morphism
$\delta : \mathcal{I}/\mathcal{I}^2 \to \mathcal{G}$.
After replacing $\mathcal{E}$ by
$\mathcal{E} \times_\mathcal{O} \mathcal{O}'$ we may also assume
that $\alpha$ factors through a map
$\alpha' : \mathcal{E} \to \mathcal{O}'$.

\medskip\noindent
These choices determine an $f^{-1}\mathcal{O}_{\mathcal{B}'}$-algebra map
$\varphi : \mathcal{O}_{\mathcal{B}'}[\mathcal{E}] \to \mathcal{O}'$.
Let $\mathcal{I}' = \Ker(\varphi)$.
Observe that $\varphi$ induces a map
$\varphi|_{\mathcal{I}'} : \mathcal{I}' \to \mathcal{G}$
and that $\mathcal{O}'$ is the pushout, as in the following
diagram
$$
\xymatrix{
0 \ar[r] & \mathcal{G} \ar[r] & \mathcal{O}' \ar[r] &
\mathcal{O} \ar[r] & 0 \\
0 \ar[r] & \mathcal{I}' \ar[u]^{\varphi|_{\mathcal{I}'}} \ar[r] &
f^{-1}\mathcal{O}_{\mathcal{B}'}[\mathcal{E}] \ar[u] \ar[r] &
\mathcal{O} \ar[u]_{=} \ar[r] & 0
}
$$
Let $\psi : \mathcal{I}' \to \mathcal{G}$ be the sum of the map
$\varphi|_{\mathcal{I}'}$ and the composition
$$
\mathcal{I}' \to \mathcal{I}'/(\mathcal{I}')^2 \to
\mathcal{I}/\mathcal{I}^2 \xrightarrow{\delta} \mathcal{G}.
$$
Then the pushout along $\psi$ is an other ring extension
$\mathcal{O}'_\xi$ fitting into a diagram as above.
A calculation (omitted) shows that $o(\mathcal{O}', \mathcal{O}'_\xi) = \xi$
as desired.
\end{proof}

\begin{lemma}
\label{lemma-extensions-of-ringed-topoi}
Let $(\Sh(\mathcal{B}), \mathcal{O}_\mathcal{B})$ be a ringed topos
and let $\mathcal{J}$ be an $\mathcal{O}_\mathcal{B}$-module.
\begin{enumerate}
\item The set of extensions of sheaves of rings
$0 \to \mathcal{J} \to \mathcal{O}_{\mathcal{B}'} \to
\mathcal{O}_\mathcal{B} \to 0$
where $\mathcal{J}$ is an ideal of square zero is canonically bijective to
$\Ext^1_{\mathcal{O}_\mathcal{B}}(
\NL_{\mathcal{O}_\mathcal{B}/\mathbf{Z}}, \mathcal{J})$.
\item Given a morphism of ringed topoi
$f : (\Sh(\mathcal{C}), \mathcal{O}) \to
(\Sh(\mathcal{B}), \mathcal{O}_\mathcal{B})$, an $\mathcal{O}$-module
$\mathcal{G}$, an $f^{-1}\mathcal{O}_\mathcal{B}$-module map
$c : f^{-1}\mathcal{J} \to \mathcal{G}$, and
given extensions of sheaves of rings with square zero kernels:
\begin{enumerate}
\item[(a)] $0 \to \mathcal{J} \to \mathcal{O}_{\mathcal{B}'} \to
\mathcal{O}_\mathcal{B} \to 0$ corresponding to
$\alpha \in \Ext^1_{\mathcal{O}_\mathcal{B}}(
\NL_{\mathcal{O}_\mathcal{B}/\mathbf{Z}}, \mathcal{J})$,
\item[(b)] $0 \to \mathcal{G} \to \mathcal{O}' \to \mathcal{O} \to 0$
corresponding to
$\beta \in \Ext^1_\mathcal{O}(\NL_{\mathcal{O}/\mathbf{Z}}, \mathcal{G})$
\end{enumerate}
then there is a morphism $(\Sh(\mathcal{C}), \mathcal{O}') \to
(\Sh(\mathcal{B}, \mathcal{O}_{\mathcal{B}'})$ fitting into a diagram
(\ref{equation-to-solve-ringed-topoi}) if and only if $\beta$ and $\alpha$
map to the same element of
$\Ext^1_\mathcal{O}(
Lf^*\NL_{\mathcal{O}_\mathcal{B}/\mathbf{Z}}, \mathcal{G})$.
\end{enumerate}
\end{lemma}

\begin{proof}
To prove this we apply the previous results where we work over
the base ringed topos $(\Sh(*), \mathbf{Z})$ with trivial thickening.
Part (1) follows from Lemma \ref{lemma-choices-ringed-topoi}
and the fact that there exists a solution, namely
$\mathcal{J} \oplus \mathcal{O}_\mathcal{B}$.
Part (2) follows from Lemma \ref{lemma-huge-diagram-ringed-topoi}
and a compatibility between the constructions in the proofs
of Lemmas \ref{lemma-choices-ringed-topoi} and
\ref{lemma-huge-diagram-ringed-topoi}
whose statement and proof we omit.
\end{proof}








\section{Deformations of algebraic spaces}
\label{section-deformations-spaces}

\noindent
In this section we spell out what the results in
Section \ref{section-deformations-ringed-topoi}
mean for deformations of algebraic spaces.

\begin{lemma}
\label{lemma-match-thickenings}
Let $S$ be a scheme. Let $i : Z \to Z'$ be a morphism of algebraic spaces
over $S$. The following are equivalent
\begin{enumerate}
\item $i$ is a thickening of algebraic spaces as defined
in More on Morphisms of Spaces, Section
\ref{spaces-more-morphisms-section-thickenings}, and
\item the associated morphism
$i_{small} : (\Sh(Z_\etale), \mathcal{O}_Z) \to
(\Sh(Z'_\etale), \mathcal{O}_{Z'})$
of ringed topoi (Properties of Spaces, Lemma
\ref{spaces-properties-lemma-morphism-ringed-topoi})
is a thickening in the sense of
Section \ref{section-thickenings-ringed-topoi}.
\end{enumerate}
\end{lemma}

\begin{proof}
We stress that this is not a triviality.

\medskip\noindent
Assume (1). By More on Morphisms of Spaces, Lemma
\ref{spaces-more-morphisms-lemma-thickening-equivalence}
the morphism $i$ induces an equivalence of small \'etale
sites and in particular of topoi. Of course $i^\sharp$
is surjective with locally nilpotent kernel by definition
of thickenings.

\medskip\noindent
Assume (2). (This direction is less important and more of
a curiosity.) For any \'etale morphism $Y' \to Z'$ we see
that $Y = Z \times_{Z'} Y'$ has the same \'etale topos
as $Y'$. In particular, $Y'$ is quasi-compact if and only if
$Y$ is quasi-compact because being quasi-compact
is a topos theoretic notion (Sites, Lemma \ref{sites-lemma-quasi-compact}).
Having said this we see that $Y'$ is quasi-compact and quasi-separated
if and only if $Y$ is quasi-compact and quasi-separated
(because you can characterize $Y'$ being quasi-separated by saying
that for all $Y'_1, Y'_2$ quasi-compact algebraic spaces \'etale over $Y'$
we have that $Y'_1 \times_{Y'} Y'_2$ is quasi-compact).
Take $Y'$ affine. Then the algebraic space $Y$ is
quasi-compact and quasi-separated. For any
quasi-coherent $\mathcal{O}_Y$-module $\mathcal{F}$ we have
$H^q(Y, \mathcal{F}) = H^q(Y', (Y \to Y')_*\mathcal{F})$
because the \'etale topoi are the same.
Then $H^q(Y', (Y \to Y')_*\mathcal{F}) = 0$
because the pushforward is quasi-coherent
(Morphisms of Spaces, Lemma \ref{spaces-morphisms-lemma-pushforward})
and $Y$ is affine. It follows that $Y'$ is affine by
Cohomology of Spaces, Proposition
\ref{spaces-cohomology-proposition-vanishing-affine}
(there surely is a proof of this direction of the lemma
avoiding this reference).
Hence $i$ is an affine morphism. In the affine case it
follows easily from the conditions in
Section \ref{section-thickenings-ringed-topoi}
that $i$ is a thickening of algebraic spaces.
\end{proof}

\begin{lemma}
\label{lemma-deform-spaces}
Let $S$ be a scheme.
Let $Y \subset Y'$ be a first order thickening of algebraic spaces
over $S$.
Let $f : X \to Y$ be a flat morphism of algebraic spaces over $S$.
If there exists a flat morphism $f' : X' \to Y'$ of algebraic spaces over $S$
and an isomorphsm $a : X \to X' \times_{Y'} Y$ over $Y$, then
\begin{enumerate}
\item the set of isomorphism classes of pairs $(f' : X' \to Y', a)$ is
principal homogeneous under
$\Ext^1_{\mathcal{O}_X}(\NL_{X/Y}, f^*\mathcal{C}_{Y/Y'})$, and
\item the set of automorphisms of $\varphi : X' \to X'$
over $Y'$ which reduce to the identity on $X' \times_{Y'} Y$
is $\Ext^0_{\mathcal{O}_X}(\NL_{X/Y}, f^*\mathcal{C}_{Y/Y'})$.
\end{enumerate}
\end{lemma}

\begin{proof}
We will apply the material on deformations of ringed topoi
to the small \'etale topoi of the algebraic spaces in the lemma.
We may think of $X$ as a closed subspace of $X'$
so that $(f, f') : (X \subset X') \to (Y \subset Y')$
is a morphism of first order thickenings.
By Lemma \ref{lemma-match-thickenings}
this translates into a morphism of thickenings of ringed topoi.
Then we see from More on Morphisms of Spaces, Lemma
\ref{spaces-more-morphisms-lemma-deform}
(or from the more general Lemma \ref{lemma-deform-module-ringed-topoi})
that the ideal sheaf of $X$ in $X'$ is equal to $f^*\mathcal{C}_{Y'/Y}$
and this is in fact equivalent to flatness of $X'$ over $Y'$.
Hence we have a commutative diagram
$$
\xymatrix{
0 \ar[r] & f^*\mathcal{C}_{Y/Y'} \ar[r] &
\mathcal{O}_{X'} \ar[r] &
\mathcal{O}_X \ar[r] & 0 \\
0 \ar[r] &
f_{small}^{-1}\mathcal{C}_{Y/Y'} \ar[u] \ar[r] &
f_{small}^{-1}\mathcal{O}_{Y'} \ar[u] \ar[r] &
f_{small}^{-1}\mathcal{O}_Y \ar[u] \ar[r] & 0
}
$$
Please compare with (\ref{equation-to-solve-ringed-topoi}).
Observe that automorphisms $\varphi$ as in (2)
give automorphisms $\varphi^\sharp : \mathcal{O}_{X'} \to \mathcal{O}_{X'}$
fitting in the diagram above. Conversely, an automorphism
$\alpha : \mathcal{O}_{X'} \to \mathcal{O}_{X'}$
fitting into the diagram of sheaves above is equal to $\varphi^\sharp$
for some automorphism $\varphi$ as in (2)
by More on Morphisms of Spaces, Lemma
\ref{spaces-more-morphisms-lemma-first-order-thickening-maps}.
Finally, by More on Morphisms of Spaces, Lemma
\ref{spaces-more-morphisms-lemma-first-order-thickening}
if we find another sheaf of rings $\mathcal{A}$ on $X_\etale$
fitting into the diagram
$$
\xymatrix{
0 \ar[r] & f^*\mathcal{C}_{Y/Y'} \ar[r] &
\mathcal{A} \ar[r] &
\mathcal{O}_X \ar[r] & 0 \\
0 \ar[r] &
f_{small}^{-1}\mathcal{C}_{Y/Y'} \ar[u] \ar[r] &
f_{small}^{-1}\mathcal{O}_{Y'} \ar[u] \ar[r] &
f_{small}^{-1}\mathcal{O}_Y \ar[u] \ar[r] & 0
}
$$
then there exists a first order thickening $X \subset X''$
with $\mathcal{O}_{X''} = \mathcal{A}$ and applying 
More on Morphisms of Spaces, Lemma
\ref{spaces-more-morphisms-lemma-first-order-thickening-maps}
once more, we obtain a morphism
$(f, f'') : (X \subset X'') \to (Y \subset Y')$ with all the
desired properties.
Thus part (1) follows from
Lemma \ref{lemma-choices-ringed-topoi}
and part (2) from part (2) of
Lemma \ref{lemma-huge-diagram-ringed-topoi}.
(Note that $\NL_{X/Y}$ as defined for a morphism of algebraic spaces in
More on Morphisms of Spaces, Section
\ref{spaces-more-morphisms-section-netherlander}
agrees with $\NL_{X/Y}$ as used in
Section \ref{section-deformations-ringed-topoi}.)
\end{proof}

\noindent
Let $S$ be a scheme. Let $f : X \to B$
be a morphism of algebraic spaces over $S$.
Let $\mathcal{F} \to \mathcal{G}$ be a homomorphism of $\mathcal{O}_X$-modules
(not necessarily quasi-coherent).
Consider the functor
$$
F :
\left\{
\begin{matrix}
\text{extensions of }f^{-1}\mathcal{O}_B\text{ algebras}\\
0 \to \mathcal{F} \to \mathcal{O}' \to \mathcal{O}_X \to 0\\
\text{where }\mathcal{F}\text{ is an ideal of square zero}
\end{matrix}
\right\}
\longrightarrow
\left\{
\begin{matrix}
\text{extensions of }f^{-1}\mathcal{O}_B\text{ algebras}\\
0 \to \mathcal{G} \to \mathcal{O}' \to \mathcal{O}_X \to 0\\
\text{where }\mathcal{G}\text{ is an ideal of square zero}
\end{matrix}
\right\}
$$
given by pushout.

\begin{lemma}
\label{lemma-thickening-space-quasi-coherent}
In the situation above assume that $X$ is quasi-compact and quasi-separated
and that $DQ_X(\mathcal{F}) \to DQ_X(\mathcal{G})$
(Derived Categories of Spaces, Section
\ref{spaces-perfect-section-better-coherator})
is an isomorphism. Then the functor $F$ is an equivalence of categories.
\end{lemma}

\begin{proof}
Recall that $\NL_{X/B}$ is an object of $D_\QCoh(\mathcal{O}_X)$, see
More on Morphisms of Spaces, Lemma
\ref{spaces-more-morphisms-lemma-netherlander-quasi-coherent}.
Hence our assumption implies the maps
$$
\Ext^i_X(\NL_{X/B}, \mathcal{F}) \longrightarrow
\Ext^i_X(\NL_{X/B}, \mathcal{G})
$$
are isomorphisms for all $i$. This implies our functor is fully
faithful by Lemma \ref{lemma-huge-diagram-ringed-topoi}.
On the other hand, the functor is essentially surjective by
Lemma \ref{lemma-choices-ringed-topoi} because
we have the solutions $\mathcal{O}_X \oplus \mathcal{F}$
and $\mathcal{O}_X \oplus \mathcal{G}$ in both categories.
\end{proof}

\noindent
Let $S$ be a scheme. Let $B \subset B'$ be a first order thickening of
algebraic spaces over $S$ with ideal sheaf $\mathcal{J}$
which we view either as a quasi-coherent $\mathcal{O}_B$-module
or as a quasi-coherent sheaf of ideals on $B'$, see
More on Morphisms of Spaces, Section
\ref{spaces-more-morphisms-section-thickenings}.
Let $f : X \to B$ be a morphism of algebraic spaces over $S$.
Let $\mathcal{F} \to \mathcal{G}$ be a homomorphism of
$\mathcal{O}_X$-modules (not necessarily quasi-coherent).
Let $c : f^{-1}\mathcal{J} \to \mathcal{F}$ be a map
of $f^{-1}\mathcal{O}_B$-modules and denote
$c' : f^{-1}\mathcal{J} \to \mathcal{G}$ the composition.
Consider the functor
$$
FT :
\{\text{solutions to }(\ref{equation-to-solve-ringed-topoi})
\text{ for }\mathcal{F}\text{ and }c\}
\longrightarrow
\{\text{solutions to }(\ref{equation-to-solve-ringed-topoi})
\text{ for }\mathcal{G}\text{ and }c'\}
$$
given by pushout.

\begin{lemma}
\label{lemma-thickening-over-thickening-space-quasi-coherent}
In the situation above assume that $X$ is quasi-compact and quasi-separated
and that $DQ_X(\mathcal{F}) \to DQ_X(\mathcal{G})$
(Derived Categories of Spaces, Section
\ref{spaces-perfect-section-better-coherator})
is an isomorphism. Then the functor $FT$ is an equivalence of categories.
\end{lemma}

\begin{proof}
A solution of (\ref{equation-to-solve-ringed-topoi}) for $\mathcal{F}$
in particular gives an extension of $f^{-1}\mathcal{O}_{B'}$-algebras
$$
0 \to \mathcal{F} \to \mathcal{O}' \to \mathcal{O}_X \to 0
$$
where $\mathcal{F}$ is an ideal of square zero. Similarly for $\mathcal{G}$.
Moreover, given such an extension, we obtain a map
$c_{\mathcal{O}'} : f^{-1}\mathcal{J} \to \mathcal{F}$.
Thus we are looking at the full subcategory of such extensions
of $f^{-1}\mathcal{O}_{B'}$-algebras with $c = c_{\mathcal{O}'}$.
Clearly, if $\mathcal{O}'' = F(\mathcal{O}')$ where
$F$ is the equivalence of Lemma \ref{lemma-thickening-space-quasi-coherent}
(applied to $X \to B'$ this time),
then $c_{\mathcal{O}''}$ is the composition of
$c_{\mathcal{O}'}$ and the map $\mathcal{F} \to \mathcal{G}$.
This proves the lemma.
\end{proof}





\section{Deformations of complexes}
\label{section-deformations-complexes}

\noindent
This section is a warmup for the next one.
We will use as much as possible the material
in the chapters on commutative algebra.

\begin{lemma}
\label{lemma-canonical-class-algebra}
Let $R' \to R$ be a surjection of rings whose kernel is an ideal
$I$ of square zero. For every $K \in D^-(R)$ there is a canonical
map
$$
\omega(K) : K \longrightarrow K \otimes_R^\mathbf{L} I[2]
$$
in $D(R)$ with the following properties
\begin{enumerate}
\item $\omega(K) = 0$ if and only if there exists
$K' \in D(R')$ with $K' \otimes_{R'}^\mathbf{L} R = K$,
\item given $K \to L$ in $D^-(R)$ the diagram
$$
\xymatrix{
K \ar[d] \ar[rr]_-{\omega(K)} & &
K \otimes^\mathbf{L}_R I[2] \ar[d] \\
L \ar[rr]^-{\omega(L)} & &
L \otimes^\mathbf{L}_R I[2]
}
$$
commutes, and
\item formation of $\omega(K)$ is compatible with ring maps $R' \to S'$
(see proof for a precise statement).
\end{enumerate}
\end{lemma}

\begin{proof}
Choose a bounded above complex $K^\bullet$ of free $R$-modules representing
$K$. Then we can choose free $R'$-modules $(K')^n$ lifting $K^n$.
We can choose $R'$-module maps $(d')^n_K : (K')^n \to (K')^{n + 1}$
lifting the differentials $d^n_K : K^n \to K^{n + 1}$ of $K^\bullet$.
Although the compositions
$$
(d')^{n + 1}_K \circ (d')^n_K : (K')^n \to (K')^{n + 2}
$$
may not be zero, they do factor as
$$
(K')^n \to K^n \xrightarrow{\omega^n_K}
K^{n + 2} \otimes_R I = I(K')^{n + 2} \to (K')^{n + 2}
$$
because $d^{n + 1} \circ d^n = 0$.
A calculation shows that $\omega^n_K$ defines a map of complexes.
This map of complexes defines $\omega(K)$.

\medskip\noindent
Let us prove this construction is compatible with a map of complexes
$\alpha^\bullet : K^\bullet \to L^\bullet$ of bounded above free $R$-modules
and given choices of lifts $(K')^n, (L')^n, (d')^n_K, (d')^n_L$.
Namely, choose $(\alpha')^n : (K')^n \to (L')^n$ lifting the
components $\alpha^n : K^n \to L^n$. As before we get a
factorization
$$
(K')^n \to K^n \xrightarrow{h^n}
L^{n + 1} \otimes_R I = I(L')^{n + 1} \to (L')^{n + 2}
$$
of $(d')^n_L \circ (\alpha')^n - (\alpha')^{n + 1} \circ (d')_K^n$.
Then it is an pleasant calculation to show that
$$
\omega^n_L \circ \alpha^n =
(d_L^{n + 1} \otimes \text{id}_I) \circ h^n + h^{n + 1} \circ d_K^n +
(\alpha^{n + 2} \otimes \text{id}_I) \circ \omega^n_K
$$
This proves the commutativity of the diagram in (2) of the lemma
in this particular case. Using this for two different choices
of bounded above free complexes representing $K$, we find that
$\omega(K)$ is well defined! And of course (2) holds in general as well.

\medskip\noindent
If $K$ lifts to $K'$ in $D^-(R')$, then we can represent
$K'$ by a bounded above complex of free $R'$-modules
and we see immediately that $\omega(K) = 0$.
Conversely, going back to our choices $K^\bullet$,
$(K')^n$, $(d')^n_K$, if $\omega(K) = 0$, then we can find
$g^n : K^n \to K^{n + 1} \otimes_R I$ with
$$
\omega^n = (d_K^{n + 1} \otimes \text{id}_I) \circ g^n +
g^{n + 1} \circ d_K^n
$$
This means that with differentials
$(d')^n_K + g^n : (K')^n \to (K')^{n + 1}$
we obtain a complex of free $R'$-modules lifting $K^\bullet$.
This proves (1).

\medskip\noindent
Finally, part (3) means the following: Let $R' \to S'$ be a map of
rings. Set $S = S' \otimes_{R'} R$ and denote $J = IS' \subset S'$
the square zero kernel of $S' \to S$. Then given $K \in D^-(R)$
the statement is that we get a commutative diagram
$$
\xymatrix{
K \otimes_R^\mathbf{L} S \ar[d] \ar[rr]_-{\omega(K) \otimes \text{id}} & &
(K \otimes^\mathbf{L}_R I[2]) \otimes_R^\mathbf{L} S \ar[d] \\
K \otimes_R^\mathbf{L} S \ar[rr]^-{\omega(K \otimes_R^\mathbf{L} S)} & &
(K \otimes_R^\mathbf{L} S) \otimes^\mathbf{L}_S J[2]
}
$$
Here the right vertical arrow comes from
$$
(K \otimes^\mathbf{L}_R I[2]) \otimes_R^\mathbf{L} S =
(K \otimes_R^\mathbf{L} S) \otimes_S^\mathbf{L}
(I \otimes_R^\mathbf{L} S)[2] \longrightarrow
(K \otimes_R^\mathbf{L} S) \otimes_S^\mathbf{L} J[2]
$$
Choose $K^\bullet$, $(K')^n$, and $(d')^n_K$ as above.
Then we can use $K^\bullet \otimes_R S$, $(K')^n \otimes_{R'} S'$, and
$(d')^n_K \otimes \text{id}_{S'}$ for the construction of
$\omega(K \otimes_R^\mathbf{L} S)$.
With these choices commutativity
is immediately verified on the level of maps of complexes.
\end{proof}







\section{Deformations of complexes on ringed topoi}
\label{section-thickenings-complexes}

\noindent
This material is taken from \cite{lieblich-complexes}.

\medskip\noindent
The material in this section works in the setting of a first order thickening
of ringed topoi as defined in Section \ref{section-thickenings-ringed-topoi}.
However, in order to simplify the notation we will assume the underlying
sites $\mathcal{C}$ and $\mathcal{D}$ are the same.
Moreover, the surjective homomorphism $\mathcal{O}' \to \mathcal{O}$
of sheaves of rings will be denoted $\mathcal{O} \to \mathcal{O}_0$
as is perhaps more customary in the literature.

\begin{lemma}
\label{lemma-lift-complex}
Let $\mathcal{C}$ be a site. Let $\mathcal{O} \to \mathcal{O}_0$
be a surjection of sheaves of rings. Assume given the following data
\begin{enumerate}
\item flat $\mathcal{O}$-modules $\mathcal{G}^n$,
\item maps of $\mathcal{O}$-modules $\mathcal{G}^n \to \mathcal{G}^{n + 1}$,
\item a complex $\mathcal{K}_0^\bullet$ of $\mathcal{O}_0$-modules,
\item maps of $\mathcal{O}$-modules $\mathcal{G}^n \to \mathcal{K}_0^n$
\end{enumerate}
such that
\begin{enumerate}
\item[(a)] $H^n(\mathcal{K}_0^\bullet) = 0$ for $n \gg 0$,
\item[(b)] $\mathcal{G}^n = 0$ for $n \gg 0$,
\item[(c)] with
$\mathcal{G}^n_0 = \mathcal{G}^n \otimes_\mathcal{O} \mathcal{O}_0$
the induced maps determine a complex $\mathcal{G}_0^\bullet$ and a map
of complexes $\mathcal{G}_0^\bullet \to \mathcal{K}_0^\bullet$.
\end{enumerate}
Then there exist
\begin{enumerate}
\item[(\romannumeral1)]
flat $\mathcal{O}$-modules $\mathcal{F}^n$,
\item[(\romannumeral2)]
maps of $\mathcal{O}$-modules $\mathcal{F}^n \to \mathcal{F}^{n + 1}$,
\item[(\romannumeral3)]
maps of $\mathcal{O}$-modules $\mathcal{F}^n \to \mathcal{K}_0^n$,
\item[(\romannumeral4)]
maps of $\mathcal{O}$-modules $\mathcal{G}^n \to \mathcal{F}^n$,
\end{enumerate}
such that $\mathcal{F}^n = 0$ for $n \gg 0$, such that the diagrams
$$
\xymatrix{
\mathcal{G}^n \ar[r] \ar[d] & \mathcal{G}^{n + 1} \ar[d] \\
\mathcal{F}^n \ar[r] & \mathcal{F}^{n + 1}
}
$$
commute for all $n$, such that the composition
$\mathcal{G}^n \to \mathcal{F}^n \to \mathcal{K}_0^n$
is the given map $\mathcal{G}^n \to \mathcal{K}_0^n$, and such that with
$\mathcal{F}^n_0 = \mathcal{F}^n \otimes_\mathcal{O} \mathcal{O}_0$
we obtain a complex $\mathcal{F}_0^\bullet$ and map of complexes
$\mathcal{F}_0^\bullet \to \mathcal{K}_0^\bullet$ which is a
quasi-isomorphism.
\end{lemma}

\begin{proof}
We will prove by descending induction on $e$ that we can find $\mathcal{F}^n$,
$\mathcal{G}^n \to \mathcal{F}^n$, and
$\mathcal{F}^n \to \mathcal{F}^{n + 1}$ for $n \geq e$
fitting into a commutative diagram
$$
\xymatrix{
\ldots \ar[r] &
\mathcal{G}^{e - 1} \ar[r] \ar@/_2pc/[dd] &
\mathcal{G}^e \ar[d] \ar[r] \ar@/_2pc/[dd] &
\mathcal{G}^{e + 1} \ar[d] \ar[r] \ar@/_2pc/[dd]|\hole &
\ldots \\
& &
\mathcal{F}^e \ar[d] \ar[r] &
\mathcal{F}^{e + 1} \ar[d] \ar[r] & \ldots \\
\ldots  \ar[r] &
\mathcal{K}_0^{e - 1} \ar[r] &
\mathcal{K}_0^e \ar[r] &
\mathcal{K}_0^{e + 1} \ar[r] & \ldots
}
$$
such that $\mathcal{F}_0^\bullet$ is a complex,
the induced map $\mathcal{F}_0^\bullet \to \mathcal{K}_0^\bullet$
induces an isomorphism on $H^n$ for $n > e$ and a surjection
for $n = e$. For $e \gg 0$ this is true because we can take
$\mathcal{F}^n = 0$ for $n \geq e$ in that case by assumptions
(a) and (b).

\medskip\noindent
Induction step. We have to construct $\mathcal{F}^{e - 1}$
and the maps $\mathcal{G}^{e - 1} \to \mathcal{F}^{e - 1}$,
$\mathcal{F}^{e - 1} \to \mathcal{F}^e$, and
$\mathcal{F}^{e - 1} \to \mathcal{K}_0^{e - 1}$.
We will choose $\mathcal{F}^{e - 1} = A \oplus B \oplus C$
as a direct sum of three pieces.

\medskip\noindent
For the first we take $A = \mathcal{G}^{e - 1}$ and we choose our map
$\mathcal{G}^{e - 1} \to \mathcal{F}^{e - 1}$ to be the inclusion of
the first summand. The maps $A \to \mathcal{K}^{e - 1}_0$
and $A \to \mathcal{F}^e$ will be the obvious ones.

\medskip\noindent
To choose $B$ we consider the surjection (by induction hypothesis)
$$
\gamma :
\Ker(\mathcal{F}^e_0 \to \mathcal{F}^{e + 1}_0)
\longrightarrow
\Ker(\mathcal{K}^e_0 \to \mathcal{K}^{e + 1}_0)/
\Im(\mathcal{K}^{e - 1}_0 \to \mathcal{K}^e_0)
$$
We can choose a set $I$, for each $i \in I$
an object $U_i$ of $\mathcal{C}$, and sections
$s_i \in \mathcal{F}^e(U_i)$, $t_i \in \mathcal{K}^{e - 1}_0(U_i)$
such that
\begin{enumerate}
\item $s_i$ maps to a section of $\Ker(\gamma) \subset
\Ker(\mathcal{F}^e_0 \to \mathcal{F}^{e + 1}_0)$,
\item $s_i$ and $t_i$ map to the same section of
$\mathcal{K}^e_0$,
\item the sections $s_i$ generate $\Ker(\gamma)$ as an $\mathcal{O}_0$-module.
\end{enumerate}
We omit giving the full justification for this;
one uses that $\mathcal{F}^e \to \mathcal{F}^e_0$
is a surjective maps of sheaves of sets. Then we set
to put
$$
B = \bigoplus\nolimits_{i \in I} j_{U_i!}\mathcal{O}_{U_i}
$$
and define the maps $B \to \mathcal{F}^e$ and $B \to \mathcal{K}_0^{e - 1}$
by using $s_i$ and $t_i$ to determine where to send the summand
$j_{U_i!}\mathcal{O}_{U_i}$.

\medskip\noindent
With $\mathcal{F}^{e - 1} = A \oplus B$ and maps as above,
this produces a diagram as above for $e - 1$ such that
$\mathcal{F}_0^\bullet \to \mathcal{K}_0^\bullet$
induces an isomorphism on $H^n$ for $n \geq e$.
To get the map to be surjective on $H^{e - 1}$ we choose
the summand $C$ as follows.
Choose a set $J$, for each $j \in J$ an object $U_j$ of $\mathcal{C}$
and a section $t_j$ of $\Ker(\mathcal{K}^{e - 1}_0 \to \mathcal{K}^e_0)$
over $U_j$ such that these sections generate this kernel over
$\mathcal{O}_0$. Then we put
$$
C = \bigoplus\nolimits_{j \in J} j_{U_j!}\mathcal{O}_{U_j}
$$
and the zero map $C \to \mathcal{F}^e$ and the map
$C \to \mathcal{K}_0^{e - 1}$ by using $s_j$ to determine where to the summand
$j_{U_j!}\mathcal{O}_{U_j}$. This finishes the induction step
by taking $\mathcal{F}^{e - 1} = A \oplus B \oplus C$ and
maps as indicated.
\end{proof}

\begin{lemma}
\label{lemma-canonical-class}
Let $\mathcal{C}$ be a site. Let $\mathcal{O} \to \mathcal{O}_0$
be a surjection of sheaves of rings whose kernel is an ideal sheaf
$\mathcal{I}$ of square zero. For every object
$K_0$ in $D^-(\mathcal{O}_0)$ there is a canonical map
$$
\omega(K_0) :
K_0 \longrightarrow
K_0 \otimes_{\mathcal{O}_0}^\mathbf{L} \mathcal{I}[2]
$$
in $D(\mathcal{O}_0)$ such that for any map
$K_0 \to L_0$ in $D^-(\mathcal{O}_0)$ the diagram
$$
\xymatrix{
K_0 \ar[d] \ar[rr]_-{\omega(K_0)} & &
(K_0 \otimes^\mathbf{L}_{\mathcal{O}_0} \mathcal{I})[2] \ar[d] \\
L_0 \ar[rr]^-{\omega(L_0)} & &
(L_0 \otimes^\mathbf{L}_{\mathcal{O}_0} \mathcal{I})[2]
}
$$
commutes.
\end{lemma}

\begin{proof}
Represent $K_0$ by any complex
$\mathcal{K}_0^\bullet$ of $\mathcal{O}_0$-modules.
Apply Lemma \ref{lemma-lift-complex}
with $\mathcal{G}^n = 0$ for all $n$.
Denote $d : \mathcal{F}^n \to \mathcal{F}^{n + 1}$
the maps produced by the lemma. Then we see that
$d \circ d : \mathcal{F}^n \to \mathcal{F}^{n + 2}$
is zero modulo $\mathcal{I}$. Since $\mathcal{F}^n$ is flat,
we see that
$\mathcal{I}\mathcal{F}^n =
\mathcal{F}^n \otimes_{\mathcal{O}} \mathcal{I} =
\mathcal{F}^n_0 \otimes_{\mathcal{O}_0} \mathcal{I}$.
Hence we obtain a canonical map of complexes
$$
d \circ d : \mathcal{F}_0^\bullet
\longrightarrow
(\mathcal{F}_0^\bullet \otimes_{\mathcal{O}_0} \mathcal{I})[2]
$$
Since $\mathcal{F}_0^\bullet$ is a bounded above complex
of flat $\mathcal{O}_0$-modules, it is K-flat and may be used
to compute derived tensor product. Moreover, the map of
complexes $\mathcal{F}_0^\bullet \to \mathcal{K}_0^\bullet$
is a quasi-isomorphism by construction. Therefore the source and
target of the map just constructed represent $K_0$ and
$K_0 \otimes_{\mathcal{O}_0}^\mathbf{L} \mathcal{I}[2]$
and we obtain our map $\omega(K_0)$.

\medskip\noindent
Let us show that this procedure is compatible with maps of complexes.
Namely, let $\mathcal{L}_0^\bullet$ represent another object of
$D^-(\mathcal{O}_0)$ and suppose that
$$
\mathcal{K}_0^\bullet \longrightarrow \mathcal{L}_0^\bullet
$$
is a map of complexes. Apply Lemma \ref{lemma-lift-complex}
for the complex $\mathcal{L}_0^\bullet$, the flat modules
$\mathcal{F}^n$, the maps $\mathcal{F}^n \to \mathcal{F}^{n + 1}$,
and the compositions
$\mathcal{F}^n \to \mathcal{K}_0^n \to \mathcal{L}_0^n$
(we apologize for the reversal of letters used).
We obtain flat modules $\mathcal{G}^n$, maps
$\mathcal{F}^n \to \mathcal{G}^n$, maps
$\mathcal{G}^n \to \mathcal{G}^{n + 1}$, and maps
$\mathcal{G}^n \to \mathcal{L}_0^n$ with all properties
as in the lemma. Then it is clear that
$$
\xymatrix{
\mathcal{F}_0^\bullet \ar[d] \ar[r] &
(\mathcal{F}_0^\bullet \otimes_{\mathcal{O}_0} \mathcal{I})[2] \ar[d] \\
\mathcal{G}_0^\bullet \ar[r] &
(\mathcal{G}_0^\bullet \otimes_{\mathcal{O}_0} \mathcal{I})[2]
}
$$
is a commutative diagram of complexes.

\medskip\noindent
To see that $\omega(K_0)$ is well defined, suppose that we have two complexes
$\mathcal{K}_0^\bullet$ and $(\mathcal{K}'_0)^\bullet$
of $\mathcal{O}_0$-modules representing $K_0$ and two systems
$(\mathcal{F}^n, d : \mathcal{F}^n \to \mathcal{F}^{n + 1},
\mathcal{F}^n \to \mathcal{K}_0^n)$
and
$((\mathcal{F}')^n, d : (\mathcal{F}')^n \to (\mathcal{F}')^{n + 1},
(\mathcal{F}')^n \to \mathcal{K}_0^n)$
as above. Then we can choose a complex $(\mathcal{K}''_0)^\bullet$
and quasi-isomorphisms
$\mathcal{K}_0^\bullet  \to (\mathcal{K}''_0)^\bullet$
and
$(\mathcal{K}'_0)^\bullet  \to (\mathcal{K}''_0)^\bullet$
realizing the fact that both complexes represent $K_0$ in the
derived category. Next, we apply the result of the previous paragraph
to
$$
(\mathcal{K}_0)^\bullet \oplus (\mathcal{K}'_0)^\bullet
\longrightarrow
(\mathcal{K}''_0)^\bullet
$$
This produces a commutative diagram
$$
\xymatrix{
\mathcal{F}_0^\bullet \oplus (\mathcal{F}'_0)^\bullet
\ar[d] \ar[r] &
(\mathcal{F}_0^\bullet \otimes_{\mathcal{O}_0} \mathcal{I})[2] \oplus
((\mathcal{F}'_0)^\bullet \otimes_{\mathcal{O}_0} \mathcal{I})[2] \ar[d] \\
\mathcal{G}_0^\bullet \ar[r] &
(\mathcal{G}_0^\bullet \otimes_{\mathcal{O}_0} \mathcal{I})[2]
}
$$
Since the vertical arrows give quasi-isomorphisms on the summands
we conclude the desired commutativity in $D(\mathcal{O}_0)$.

\medskip\noindent
Having established well-definedness, the statement on compatibility
with maps is a consequence of the result in the second
paragraph.
\end{proof}

\begin{lemma}
\label{lemma-induced-map}
Let $(\mathcal{C}, \mathcal{O})$ be a ringed site.
Let $\alpha : K \to L$ be a map of $D^-(\mathcal{O})$.
Let $\mathcal{F}$ be a sheaf of $\mathcal{O}$-modules.
Let $n \in \mathbf{Z}$.
\begin{enumerate}
\item If $H^i(\alpha)$ is an isomorphism for $i \geq n$,
then $H^i(\alpha \otimes_\mathcal{O}^\mathbf{L} \text{id}_\mathcal{F})$
is an isomorphism for $i \geq n$.
\item If $H^i(\alpha)$ is an isomorphism for $i > n$ 
and surjective for $i = n$,
then $H^i(\alpha \otimes_\mathcal{O}^\mathbf{L} \text{id}_\mathcal{F})$
is an isomorphism for $i > n$ and surjective for $i = n$.
\end{enumerate}
\end{lemma}

\begin{proof}
Choose a distinguished triangle
$$
K \to L \to C \to K[1]
$$
In case (2) we see that $H^i(C) = 0$ for $i \geq n$.
Hence $H^i(C \otimes_\mathcal{O}^\mathbf{L} \mathcal{F}) = 0$
for $i \geq n$ by (the dual of)
Derived Categories, Lemma \ref{derived-lemma-negative-vanishing}.
This in turn shows that
$H^i(\alpha \otimes_\mathcal{O}^\mathbf{L} \text{id}_\mathcal{F})$
is an isomorphism for $i > n$ and surjective for $i = n$.
In case (1) we moreover see that $H^{n - 1}(L) \to H^{n - 1}(C)$
is surjective. Considering the diagram
$$
\xymatrix{
H^{n - 1}(L) \otimes_\mathcal{O} \mathcal{F} \ar[r] \ar[d] &
H^{n - 1}(C) \otimes_\mathcal{O} \mathcal{F} \ar@{=}[d] \\
H^{n - 1}(L \otimes_\mathcal{O}^\mathbf{L} \mathcal{F}) \ar[r] &
H^{n - 1}(C \otimes_\mathcal{O}^\mathbf{L} \mathcal{F})
}
$$
we conclude the lower horizontal arrow is surjective. Combined with what
was said before this implies that
$H^n(\alpha \otimes_\mathcal{O}^\mathbf{L} \text{id}_\mathcal{F})$
is an isomorphism.
\end{proof}

\begin{lemma}
\label{lemma-canonical-class-obstruction}
Let $\mathcal{C}$ be a site. Let $\mathcal{O} \to \mathcal{O}_0$
be a surjection of sheaves of rings whose kernel is an ideal sheaf
$\mathcal{I}$ of square zero. For every object
$K_0$ in $D^-(\mathcal{O}_0)$ the following are equivalent
\begin{enumerate}
\item  the class
$\omega(K_0) \in
\Ext^2_{\mathcal{O}_0}(K_0, K_0 \otimes_{\mathcal{O}_0} \mathcal{I})$
constructed in Lemma \ref{lemma-canonical-class} is zero,
\item there exists $K \in D^-(\mathcal{O})$ with
$K \otimes_\mathcal{O}^\mathbf{L} \mathcal{O}_0 = K_0$
in $D(\mathcal{O}_0)$.
\end{enumerate}
\end{lemma}

\begin{proof}
Let $K$ be as in (2). Then we can represent $K$ by a bounded above
complex $\mathcal{F}^\bullet$ of flat $\mathcal{O}$-modules.
Then $\mathcal{F}_0^\bullet =
\mathcal{F}^\bullet \otimes_{\mathcal{O}} \mathcal{O}_0$
represents $K_0$ in $D(\mathcal{O}_0)$.
Since $d_{\mathcal{F}^\bullet} \circ d_{\mathcal{F}^\bullet} = 0$
as $\mathcal{F}^\bullet$ is a complex, we see from the very construction
of $\omega(K_0)$ that it is zero.

\medskip\noindent
Assume (1). Let $\mathcal{F}^n$, $d : \mathcal{F}^n \to \mathcal{F}^{n + 1}$
be as in the construction of $\omega(K_0)$. The nullity of
$\omega(K_0)$ implies that the map
$$
\omega = d \circ d : \mathcal{F}_0^\bullet
\longrightarrow
(\mathcal{F}_0^\bullet \otimes_{\mathcal{O}_0} \mathcal{I})[2]
$$
is zero in $D(\mathcal{O}_0)$. By definition of the derived category
as the localization of the homotopy category of complexes
of $\mathcal{O}_0$-modules, there exists a quasi-isomorphism
$\alpha : \mathcal{G}_0^\bullet \to \mathcal{F}_0^\bullet$
such that there exist $\mathcal{O}_0$-modules maps
$h^n : \mathcal{G}_0^n \to
\mathcal{F}_0^{n + 1} \otimes_\mathcal{O} \mathcal{I}$
with
$$
\omega \circ \alpha =
d_{\mathcal{F}_0^\bullet \otimes \mathcal{I}} \circ h +
h \circ d_{\mathcal{G}_0^\bullet}
$$
We set
$$
\mathcal{H}^n = \mathcal{F}^n \times_{\mathcal{F}^n_0} \mathcal{G}_0^n
$$
and we define
$$
d' : \mathcal{H}^n \longrightarrow \mathcal{H}^{n + 1},\quad
(f^n, g_0^n) \longmapsto (d(f^n) - h^n(g_0^n), d(g_0^n))
$$
with obvious notation using that
$\mathcal{F}_0^{n + 1} \otimes_{\mathcal{O}_0} \mathcal{I} =
\mathcal{F}^{n + 1} \otimes_\mathcal{O} \mathcal{I} =
\mathcal{I}\mathcal{F}^{n + 1} \subset \mathcal{F}^{n + 1}$.
Then one checks $d' \circ d' = 0$ by our choice of $h^n$
and definition of $\omega$.
Hence $\mathcal{H}^\bullet$ defines an object in $D(\mathcal{O})$.
On the other hand, there is a short exact sequence of complexes
of $\mathcal{O}$-modules
$$
0 \to \mathcal{F}_0^\bullet \otimes_{\mathcal{O}_0} \mathcal{I} \to
\mathcal{H}^\bullet \to \mathcal{G}_0^\bullet \to 0
$$
We still have to show that
$\mathcal{H}^\bullet \otimes_\mathcal{O}^\mathbf{L} \mathcal{O}_0$
is isomorphic to $K_0$.
Choose a quasi-isomorphism
$\mathcal{E}^\bullet \to \mathcal{H}^\bullet$
where $\mathcal{E}^\bullet$ is a bounded above complex of flat
$\mathcal{O}$-modules. We obtain a commutative diagram
$$
\xymatrix{
0 \ar[r] &
\mathcal{E}^\bullet \otimes_\mathcal{O} \mathcal{I} \ar[d]^\beta \ar[r] &
\mathcal{E}^\bullet \ar[d]^\gamma \ar[r] &
\mathcal{E}_0^\bullet \ar[d]^\delta \ar[r] &
0 \\
0 \ar[r] &
\mathcal{F}_0^\bullet \otimes_{\mathcal{O}_0} \mathcal{I} \ar[r] &
\mathcal{H}^\bullet \ar[r] &
\mathcal{G}_0^\bullet \ar[r] &
0
}
$$
We claim that $\delta$ is a quasi-isomorphism. Since $H^i(\delta)$
is an isomorphism for $i \gg 0$, we can use descending induction
on $n$ such that $H^i(\delta)$ is an isomorphism for $i \geq n$.
Observe that
$\mathcal{E}^\bullet \otimes_\mathcal{O} \mathcal{I}$
represents
$\mathcal{E}_0^\bullet \otimes_{\mathcal{O}_0}^\mathbf{L} \mathcal{I}$,
that
$\mathcal{F}_0^\bullet \otimes_{\mathcal{O}_0} \mathcal{I}$
represents
$\mathcal{G}_0^\bullet \otimes_{\mathcal{O}_0}^\mathbf{L} \mathcal{I}$,
and that
$\beta = \delta \otimes_{\mathcal{O}_0}^\mathbf{L} \text{id}_\mathcal{I}$
as maps in $D(\mathcal{O}_0)$. This is true because
$\beta =
(\alpha \otimes \text{id}_\mathcal{I})
\circ
(\delta \otimes \text{id}_\mathcal{I})$.
Suppose that $H^i(\delta)$ is an isomorphism in degrees $\geq n$.
Then the same is true for $\beta$ by what we just said
and Lemma \ref{lemma-induced-map}.
Then we can look at the diagram
$$
\xymatrix{
H^{n - 1}(\mathcal{E}^\bullet \otimes_\mathcal{O} \mathcal{I})
\ar[r] \ar[d]^{H^{n - 1}(\beta)} &
H^{n - 1}(\mathcal{E}^\bullet) \ar[r] \ar[d] &
H^{n - 1}(\mathcal{E}_0^\bullet) \ar[r] \ar[d]^{H^{n - 1}(\delta)} &
H^n(\mathcal{E}^\bullet \otimes_\mathcal{O} \mathcal{I})
\ar[r] \ar[d]^{H^n(\beta)} &
H^n(\mathcal{E}^\bullet) \ar[d] \\
H^{n - 1}(\mathcal{F}_0^\bullet \otimes_\mathcal{O} \mathcal{I}) \ar[r] &
H^{n - 1}(\mathcal{H}^\bullet) \ar[r] &
H^{n - 1}(\mathcal{G}_0^\bullet) \ar[r] &
H^n(\mathcal{F}_0^\bullet \otimes_\mathcal{O} \mathcal{I}) \ar[r] &
H^n(\mathcal{H}^\bullet)
}
$$
Using Homology, Lemma \ref{homology-lemma-four-lemma}
we see that $H^{n - 1}(\delta)$ is surjective.
This in turn implies that $H^{n - 1}(\beta)$ is surjective
by Lemma \ref{lemma-induced-map}.
Using Homology, Lemma \ref{homology-lemma-four-lemma}
again we see that $H^{n - 1}(\delta)$ is an isomorphism.
The claim holds by induction, so $\delta$ is a quasi-isomorphism
which is what we wanted to show.
\end{proof}

\begin{lemma}
\label{lemma-lift-map-complexes}
Let $\mathcal{C}$ be a site. Let $\mathcal{O} \to \mathcal{O}_0$
be a surjection of sheaves of rings. Assume given the following data
\begin{enumerate}
\item a complex of $\mathcal{O}$-modules $\mathcal{F}^\bullet$,
\item a complex $\mathcal{K}_0^\bullet$ of $\mathcal{O}_0$-modules,
\item a quasi-isomorphism $\mathcal{K}_0^\bullet \to
\mathcal{F}^\bullet \otimes_\mathcal{O} \mathcal{O}_0$,
\end{enumerate}
Then there exist a quasi-isomorphism
$\mathcal{G}^\bullet \to \mathcal{F}^\bullet$ such that the map
of complexes
$\mathcal{G}^\bullet  \otimes_\mathcal{O} \mathcal{O}_0 \to
\mathcal{F}^\bullet \otimes_\mathcal{O} \mathcal{O}_0$ factors
through $\mathcal{K}_0^\bullet$ in the homotopy category
of complexes of $\mathcal{O}_0$-modules.
\end{lemma}

\begin{proof}
Set $\mathcal{F}_0^\bullet =
\mathcal{F}^\bullet \otimes_\mathcal{O} \mathcal{O}_0$.
By Derived Categories, Lemma \ref{derived-lemma-make-surjective}
there exists a factorization
$$
\mathcal{K}_0^\bullet \to \mathcal{L}_0^\bullet \to \mathcal{F}_0^\bullet
$$
of the given map such that the first arrow has an inverse up
to homotopy and the second arrow is termwise split surjective.
Hence we may assume that $\mathcal{K}_0^\bullet \to \mathcal{F}_0^\bullet$
is termwise surjective.
In that case we take
$$
\mathcal{G}^n = \mathcal{F}^n \times_{\mathcal{F}^n_0} \mathcal{K}_0^n
$$
and everything is clear.
\end{proof}

\begin{lemma}
\label{lemma-inf-obs-map-defo-complex}
Let $\mathcal{C}$ be a site. Let $\mathcal{O} \to \mathcal{O}_0$
be a surjection of sheaves of rings whose kernel is an ideal sheaf
$\mathcal{I}$ of square zero. Let $K, L \in D^-(\mathcal{O})$.
Set $K_0 = K \otimes_\mathcal{O}^\mathbf{L} \mathcal{O}_0$
and $L_0 = L \otimes_\mathcal{O}^\mathbf{L} \mathcal{O}_0$
in $D^-(\mathcal{O}_0)$. Given $\alpha_0 : K_0 \to L_0$ in $D(\mathcal{O}_0)$
there is a canonical element
$$
o(\alpha_0) \in \Ext^1_{\mathcal{O}_0}(K_0,
L_0 \otimes_{\mathcal{O}_0}^\mathbf{L} \mathcal{I})
$$
whose vanishing is necessary and sufficient for the
existence of a map $\alpha : K \to L$ in $D(\mathcal{O})$
with $\alpha_0 = \alpha \otimes_\mathcal{O}^\mathbf{L} \text{id}$.
\end{lemma}

\begin{proof}
Finding $\alpha : K \to L$ lifing $\alpha_0$ is the same as finding
$\alpha : K \to L$ such that the composition $K \xrightarrow{\alpha} L \to L_0$
is equal to the composition $K \to K_0 \xrightarrow{\alpha_0} L_0$.
The short exact sequence
$0 \to \mathcal{I} \to \mathcal{O} \to \mathcal{O}_0 \to 0$
gives rise to a canonical distinguished triangle
$$
L \otimes_\mathcal{O}^\mathbf{L} \mathcal{I} \to
L \to
L_0 \to
(L \otimes_\mathcal{O}^\mathbf{L} \mathcal{I})[1]
$$
in $D(\mathcal{O})$.
By Derived Categories, Lemma \ref{derived-lemma-representable-homological}
the composition
$$
K \to K_0 \xrightarrow{\alpha_0} L_0 \to
(L \otimes_\mathcal{O}^\mathbf{L} \mathcal{I})[1]
$$
is zero if and only if we can find $\alpha : K \to L$
lifting $\alpha_0$. The composition is an element in
$$
\Hom_{D(\mathcal{O})}(K, (L \otimes_\mathcal{O}^\mathbf{L} \mathcal{I})[1]) =
\Hom_{D(\mathcal{O}_0)}(K_0,
(L \otimes_\mathcal{O}^\mathbf{L} \mathcal{I})[1]) =
\Ext^1_{\mathcal{O}_0}(K_0,
L_0 \otimes_{\mathcal{O}_0}^\mathbf{L} \mathcal{I})
$$
by adjunction.
\end{proof}

\begin{lemma}
\label{lemma-first-order-defos-complex}
Let $\mathcal{C}$ be a site. Let $\mathcal{O} \to \mathcal{O}_0$
be a surjection of sheaves of rings whose kernel is an ideal sheaf
$\mathcal{I}$ of square zero. Let $K_0 \in D^-(\mathcal{O})$.
A lift of $K_0$ is a pair $(K, \alpha_0)$ consisting of an object
$K$ in $D^-(\mathcal{O})$ and an isomorphism
$\alpha_0 : K \otimes_\mathcal{O}^\mathbf{L} \mathcal{O}_0 \to K_0$
in $D(\mathcal{O}_0)$.
\begin{enumerate}
\item Given a lift $(K, \alpha)$ the group of automorphism of the pair
is canonically the cokernel of a map
$$
\Ext^{-1}_{\mathcal{O}_0}(K_0, K_0)
\longrightarrow
\Hom_{\mathcal{O}_0}(K_0, K_0 \otimes_{\mathcal{O}_0}^\mathbf{L} \mathcal{I})
$$
\item If there is a lift, then the set of isomorphism classes of lifts
is principal homogenenous under
$\Ext^1_{\mathcal{O}_0}(K_0,
K_0 \otimes_{\mathcal{O}_0}^\mathbf{L} \mathcal{I})$.
\end{enumerate}
\end{lemma}

\begin{proof}
An automorphism of $(K, \alpha)$ is a map $\varphi : K \to K$
in $D(\mathcal{O})$ with
$\varphi \otimes_\mathcal{O} \text{id}_{\mathcal{O}_0} = \text{id}$.
This is the same thing as saying that
$$
K \xrightarrow{\varphi - \text{id}} K \to
K \otimes_\mathcal{O}^\mathbf{L} \mathcal{O}_0
$$
is zero. We conclude the group of automorphisms is
the cokernel of a map
$$
\Hom_\mathcal{O}(K, K_0[-1])
\longrightarrow
\Hom_\mathcal{O}(K, K_0 \otimes_{\mathcal{O}_0}^\mathbf{L} \mathcal{I})
$$
by the distinguished triangle
$$
K \otimes_\mathcal{O}^\mathbf{L} \mathcal{I} \to
K \to
K \otimes_\mathcal{O}^\mathbf{L} \mathcal{O}_0 \to
(K \otimes_\mathcal{O}^\mathbf{L} \mathcal{I})[1]
$$
in $D(\mathcal{O})$ and 
Derived Categories, Lemma \ref{derived-lemma-representable-homological}.
To translate into the groups in the lemma use adjunction
of the restriction functor $D(\mathcal{O}_0) \to D(\mathcal{O})$ and
$- \otimes_\mathcal{O} \mathcal{O}_0 : D(\mathcal{O}) \to D(\mathcal{O}_0)$.
This proves (1).

\medskip\noindent
Proof of (2).
Assume that $K_0 = K \otimes_\mathcal{O}^\mathbf{L} \mathcal{O}_0$
in $D(\mathcal{O})$. By Lemma \ref{lemma-inf-obs-map-defo-complex}
the map sending a lift $(K', \alpha_0)$ to the obstruction $o(\alpha_0)$
to lifting $\alpha_0$ defines a canonical injective map
from the set of isomomorphism classes of pairs
to $\Ext^1_{\mathcal{O}_0}(K_0,
K_0 \otimes_{\mathcal{O}_0}^\mathbf{L} \mathcal{I})$.
To finish the proof we show that it is surjective.
Pick $\xi : K_0 \to (K_0 \otimes_{\mathcal{O}_0}^\mathbf{L} \mathcal{I})[1]$
in the $\Ext^1$ of the lemma.
Choose a bounded above complex $\mathcal{F}^\bullet$
of flat $\mathcal{O}$-modules representing $K$.
The map $\xi$ can be represented as $t \circ s^{-1}$
where $s : \mathcal{K}_0^\bullet \to \mathcal{F}_0^\bullet$
is a quasi-isomorphism and
$t : \mathcal{K}_0^\bullet \to
\mathcal{F}_0^\bullet \otimes_{\mathcal{O}_0} \mathcal{I}[1]$
is a map of complexes.
By Lemma \ref{lemma-lift-map-complexes}
we can assume there exists a quasi-isomorphism
$\mathcal{G}^\bullet \to \mathcal{F}^\bullet$
of complexes of $\mathcal{O}$-modules
such that $\mathcal{G}_0^\bullet \to \mathcal{F}_0^\bullet$
factors through $s$ up to homotopy.
We may and do replace $\mathcal{G}^\bullet$ by a bounded
above complex of flat $\mathcal{O}$-modules (by picking a qis
from such to $\mathcal{G}^\bullet$ and replacing).
Then we see that $\xi$ is represented by
a map of complexes
$t : \mathcal{G}_0^\bullet \to
\mathcal{F}_0^\bullet \otimes_{\mathcal{O}_0} \mathcal{I}[1]$
and the quasi-isomorphism $\mathcal{G}_0^\bullet \to \mathcal{F}_0^\bullet$.
Set
$$
\mathcal{H}^n = \mathcal{F}^n \times_{\mathcal{F}_0^n} \mathcal{G}_0^n
$$
with differentials
$$
\mathcal{H}^n \to \mathcal{H}^{n + 1},\quad
(f^n, g_0^n) \mapsto (d(f^n) + t(g_0^n), d(g_0^n))
$$
This makes sense as
$\mathcal{F}_0^{n + 1} \otimes_{\mathcal{O}_0} \mathcal{I} =
\mathcal{F}^{n + 1} \otimes_\mathcal{O} \mathcal{I} =
\mathcal{I}\mathcal{F}^{n + 1} \subset \mathcal{F}^{n + 1}$.
We omit the computation that shows that $\mathcal{H}^\bullet$
is a complex of $\mathcal{O}$-modules. By construction there is
a short exact sequence
$$
0 \to \mathcal{F}_0^\bullet \otimes_{\mathcal{O}_0} \mathcal{I} \to
\mathcal{H}^\bullet \to \mathcal{G}_0^\bullet \to 0
$$
of complexes of $\mathcal{O}$-modules.
Exactly as in the proof of Lemma \ref{lemma-canonical-class-obstruction}
one shows that this sequence induces an isomorphism
$\alpha_0 :
\mathcal{H}^\bullet \otimes_\mathcal{O}^\mathbf{L} \mathcal{O}_0 \to
\mathcal{G}_0^\bullet$ in $D(\mathcal{O}_0)$.
In other words, we have produced a pair $(\mathcal{H}^\bullet, \alpha_0)$.
We omit the verification that $o(\alpha_0) = \xi$; hint: $o(\alpha_0)$
can be computed explitly in this case as we have maps
$\mathcal{H}^n \to \mathcal{F}^n$ (not compatible with differentials)
lifting the components of $\alpha_0$. This finishes the proof.
\end{proof}












\begin{multicols}{2}[\section{Other chapters}]
\noindent
Preliminaries
\begin{enumerate}
\item \hyperref[introduction-section-phantom]{Introduction}
\item \hyperref[conventions-section-phantom]{Conventions}
\item \hyperref[sets-section-phantom]{Set Theory}
\item \hyperref[categories-section-phantom]{Categories}
\item \hyperref[topology-section-phantom]{Topology}
\item \hyperref[sheaves-section-phantom]{Sheaves on Spaces}
\item \hyperref[sites-section-phantom]{Sites and Sheaves}
\item \hyperref[stacks-section-phantom]{Stacks}
\item \hyperref[fields-section-phantom]{Fields}
\item \hyperref[algebra-section-phantom]{Commutative Algebra}
\item \hyperref[brauer-section-phantom]{Brauer Groups}
\item \hyperref[homology-section-phantom]{Homological Algebra}
\item \hyperref[derived-section-phantom]{Derived Categories}
\item \hyperref[simplicial-section-phantom]{Simplicial Methods}
\item \hyperref[more-algebra-section-phantom]{More on Algebra}
\item \hyperref[smoothing-section-phantom]{Smoothing Ring Maps}
\item \hyperref[modules-section-phantom]{Sheaves of Modules}
\item \hyperref[sites-modules-section-phantom]{Modules on Sites}
\item \hyperref[injectives-section-phantom]{Injectives}
\item \hyperref[cohomology-section-phantom]{Cohomology of Sheaves}
\item \hyperref[sites-cohomology-section-phantom]{Cohomology on Sites}
\item \hyperref[dga-section-phantom]{Differential Graded Algebra}
\item \hyperref[dpa-section-phantom]{Divided Power Algebra}
\item \hyperref[sdga-section-phantom]{Differential Graded Sheaves}
\item \hyperref[hypercovering-section-phantom]{Hypercoverings}
\end{enumerate}
Schemes
\begin{enumerate}
\setcounter{enumi}{25}
\item \hyperref[schemes-section-phantom]{Schemes}
\item \hyperref[constructions-section-phantom]{Constructions of Schemes}
\item \hyperref[properties-section-phantom]{Properties of Schemes}
\item \hyperref[morphisms-section-phantom]{Morphisms of Schemes}
\item \hyperref[coherent-section-phantom]{Cohomology of Schemes}
\item \hyperref[divisors-section-phantom]{Divisors}
\item \hyperref[limits-section-phantom]{Limits of Schemes}
\item \hyperref[varieties-section-phantom]{Varieties}
\item \hyperref[topologies-section-phantom]{Topologies on Schemes}
\item \hyperref[descent-section-phantom]{Descent}
\item \hyperref[perfect-section-phantom]{Derived Categories of Schemes}
\item \hyperref[more-morphisms-section-phantom]{More on Morphisms}
\item \hyperref[flat-section-phantom]{More on Flatness}
\item \hyperref[groupoids-section-phantom]{Groupoid Schemes}
\item \hyperref[more-groupoids-section-phantom]{More on Groupoid Schemes}
\item \hyperref[etale-section-phantom]{\'Etale Morphisms of Schemes}
\end{enumerate}
Topics in Scheme Theory
\begin{enumerate}
\setcounter{enumi}{41}
\item \hyperref[chow-section-phantom]{Chow Homology}
\item \hyperref[intersection-section-phantom]{Intersection Theory}
\item \hyperref[pic-section-phantom]{Picard Schemes of Curves}
\item \hyperref[weil-section-phantom]{Weil Cohomology Theories}
\item \hyperref[adequate-section-phantom]{Adequate Modules}
\item \hyperref[dualizing-section-phantom]{Dualizing Complexes}
\item \hyperref[duality-section-phantom]{Duality for Schemes}
\item \hyperref[discriminant-section-phantom]{Discriminants and Differents}
\item \hyperref[derham-section-phantom]{de Rham Cohomology}
\item \hyperref[local-cohomology-section-phantom]{Local Cohomology}
\item \hyperref[algebraization-section-phantom]{Algebraic and Formal Geometry}
\item \hyperref[curves-section-phantom]{Algebraic Curves}
\item \hyperref[resolve-section-phantom]{Resolution of Surfaces}
\item \hyperref[models-section-phantom]{Semistable Reduction}
\item \hyperref[functors-section-phantom]{Functors and Morphisms}
\item \hyperref[equiv-section-phantom]{Derived Categories of Varieties}
\item \hyperref[pione-section-phantom]{Fundamental Groups of Schemes}
\item \hyperref[etale-cohomology-section-phantom]{\'Etale Cohomology}
\item \hyperref[crystalline-section-phantom]{Crystalline Cohomology}
\item \hyperref[proetale-section-phantom]{Pro-\'etale Cohomology}
\item \hyperref[relative-cycles-section-phantom]{Relative Cycles}
\item \hyperref[more-etale-section-phantom]{More \'Etale Cohomology}
\item \hyperref[trace-section-phantom]{The Trace Formula}
\end{enumerate}
Algebraic Spaces
\begin{enumerate}
\setcounter{enumi}{64}
\item \hyperref[spaces-section-phantom]{Algebraic Spaces}
\item \hyperref[spaces-properties-section-phantom]{Properties of Algebraic Spaces}
\item \hyperref[spaces-morphisms-section-phantom]{Morphisms of Algebraic Spaces}
\item \hyperref[decent-spaces-section-phantom]{Decent Algebraic Spaces}
\item \hyperref[spaces-cohomology-section-phantom]{Cohomology of Algebraic Spaces}
\item \hyperref[spaces-limits-section-phantom]{Limits of Algebraic Spaces}
\item \hyperref[spaces-divisors-section-phantom]{Divisors on Algebraic Spaces}
\item \hyperref[spaces-over-fields-section-phantom]{Algebraic Spaces over Fields}
\item \hyperref[spaces-topologies-section-phantom]{Topologies on Algebraic Spaces}
\item \hyperref[spaces-descent-section-phantom]{Descent and Algebraic Spaces}
\item \hyperref[spaces-perfect-section-phantom]{Derived Categories of Spaces}
\item \hyperref[spaces-more-morphisms-section-phantom]{More on Morphisms of Spaces}
\item \hyperref[spaces-flat-section-phantom]{Flatness on Algebraic Spaces}
\item \hyperref[spaces-groupoids-section-phantom]{Groupoids in Algebraic Spaces}
\item \hyperref[spaces-more-groupoids-section-phantom]{More on Groupoids in Spaces}
\item \hyperref[bootstrap-section-phantom]{Bootstrap}
\item \hyperref[spaces-pushouts-section-phantom]{Pushouts of Algebraic Spaces}
\end{enumerate}
Topics in Geometry
\begin{enumerate}
\setcounter{enumi}{81}
\item \hyperref[spaces-chow-section-phantom]{Chow Groups of Spaces}
\item \hyperref[groupoids-quotients-section-phantom]{Quotients of Groupoids}
\item \hyperref[spaces-more-cohomology-section-phantom]{More on Cohomology of Spaces}
\item \hyperref[spaces-simplicial-section-phantom]{Simplicial Spaces}
\item \hyperref[spaces-duality-section-phantom]{Duality for Spaces}
\item \hyperref[formal-spaces-section-phantom]{Formal Algebraic Spaces}
\item \hyperref[restricted-section-phantom]{Algebraization of Formal Spaces}
\item \hyperref[spaces-resolve-section-phantom]{Resolution of Surfaces Revisited}
\end{enumerate}
Deformation Theory
\begin{enumerate}
\setcounter{enumi}{89}
\item \hyperref[formal-defos-section-phantom]{Formal Deformation Theory}
\item \hyperref[defos-section-phantom]{Deformation Theory}
\item \hyperref[cotangent-section-phantom]{The Cotangent Complex}
\item \hyperref[examples-defos-section-phantom]{Deformation Problems}
\end{enumerate}
Algebraic Stacks
\begin{enumerate}
\setcounter{enumi}{93}
\item \hyperref[algebraic-section-phantom]{Algebraic Stacks}
\item \hyperref[examples-stacks-section-phantom]{Examples of Stacks}
\item \hyperref[stacks-sheaves-section-phantom]{Sheaves on Algebraic Stacks}
\item \hyperref[criteria-section-phantom]{Criteria for Representability}
\item \hyperref[artin-section-phantom]{Artin's Axioms}
\item \hyperref[quot-section-phantom]{Quot and Hilbert Spaces}
\item \hyperref[stacks-properties-section-phantom]{Properties of Algebraic Stacks}
\item \hyperref[stacks-morphisms-section-phantom]{Morphisms of Algebraic Stacks}
\item \hyperref[stacks-limits-section-phantom]{Limits of Algebraic Stacks}
\item \hyperref[stacks-cohomology-section-phantom]{Cohomology of Algebraic Stacks}
\item \hyperref[stacks-perfect-section-phantom]{Derived Categories of Stacks}
\item \hyperref[stacks-introduction-section-phantom]{Introducing Algebraic Stacks}
\item \hyperref[stacks-more-morphisms-section-phantom]{More on Morphisms of Stacks}
\item \hyperref[stacks-geometry-section-phantom]{The Geometry of Stacks}
\end{enumerate}
Topics in Moduli Theory
\begin{enumerate}
\setcounter{enumi}{107}
\item \hyperref[moduli-section-phantom]{Moduli Stacks}
\item \hyperref[moduli-curves-section-phantom]{Moduli of Curves}
\end{enumerate}
Miscellany
\begin{enumerate}
\setcounter{enumi}{109}
\item \hyperref[examples-section-phantom]{Examples}
\item \hyperref[exercises-section-phantom]{Exercises}
\item \hyperref[guide-section-phantom]{Guide to Literature}
\item \hyperref[desirables-section-phantom]{Desirables}
\item \hyperref[coding-section-phantom]{Coding Style}
\item \hyperref[obsolete-section-phantom]{Obsolete}
\item \hyperref[fdl-section-phantom]{GNU Free Documentation License}
\item \hyperref[index-section-phantom]{Auto Generated Index}
\end{enumerate}
\end{multicols}


\bibliography{my}
\bibliographystyle{amsalpha}

\end{document}
