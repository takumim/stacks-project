\IfFileExists{stacks-project.cls}{%
\documentclass{stacks-project}
}{%
\documentclass{amsart}
}

% For dealing with references we use the comment environment
\usepackage{verbatim}
\newenvironment{reference}{\comment}{\endcomment}
%\newenvironment{reference}{}{}
\newenvironment{slogan}{\comment}{\endcomment}
\newenvironment{history}{\comment}{\endcomment}

% For commutative diagrams we use Xy-pic
\usepackage[all]{xy}

% We use 2cell for 2-commutative diagrams.
\xyoption{2cell}
\UseAllTwocells

% We use multicol for the list of chapters between chapters
\usepackage{multicol}

% This is generall recommended for better output
\usepackage[T1]{fontenc}

% For cross-file-references
\usepackage{xr-hyper}

% Package for hypertext links:
\usepackage{hyperref}

% For any local file, say "hello.tex" you want to link to please
% use \externaldocument[hello-]{hello}
\externaldocument[introduction-]{introduction}
\externaldocument[conventions-]{conventions}
\externaldocument[sets-]{sets}
\externaldocument[categories-]{categories}
\externaldocument[topology-]{topology}
\externaldocument[sheaves-]{sheaves}
\externaldocument[sites-]{sites}
\externaldocument[stacks-]{stacks}
\externaldocument[fields-]{fields}
\externaldocument[algebra-]{algebra}
\externaldocument[brauer-]{brauer}
\externaldocument[homology-]{homology}
\externaldocument[derived-]{derived}
\externaldocument[simplicial-]{simplicial}
\externaldocument[more-algebra-]{more-algebra}
\externaldocument[smoothing-]{smoothing}
\externaldocument[modules-]{modules}
\externaldocument[sites-modules-]{sites-modules}
\externaldocument[injectives-]{injectives}
\externaldocument[cohomology-]{cohomology}
\externaldocument[sites-cohomology-]{sites-cohomology}
\externaldocument[dga-]{dga}
\externaldocument[dpa-]{dpa}
\externaldocument[hypercovering-]{hypercovering}
\externaldocument[schemes-]{schemes}
\externaldocument[constructions-]{constructions}
\externaldocument[properties-]{properties}
\externaldocument[morphisms-]{morphisms}
\externaldocument[coherent-]{coherent}
\externaldocument[divisors-]{divisors}
\externaldocument[limits-]{limits}
\externaldocument[varieties-]{varieties}
\externaldocument[topologies-]{topologies}
\externaldocument[descent-]{descent}
\externaldocument[perfect-]{perfect}
\externaldocument[more-morphisms-]{more-morphisms}
\externaldocument[flat-]{flat}
\externaldocument[groupoids-]{groupoids}
\externaldocument[more-groupoids-]{more-groupoids}
\externaldocument[etale-]{etale}
\externaldocument[chow-]{chow}
\externaldocument[intersection-]{intersection}
\externaldocument[pic-]{pic}
\externaldocument[adequate-]{adequate}
\externaldocument[dualizing-]{dualizing}
\externaldocument[duality-]{duality}
\externaldocument[discriminant-]{discriminant}
\externaldocument[local-cohomology-]{local-cohomology}
\externaldocument[curves-]{curves}
\externaldocument[resolve-]{resolve}
\externaldocument[models-]{models}
\externaldocument[pione-]{pione}
\externaldocument[etale-cohomology-]{etale-cohomology}
\externaldocument[proetale-]{proetale}
\externaldocument[crystalline-]{crystalline}
\externaldocument[spaces-]{spaces}
\externaldocument[spaces-properties-]{spaces-properties}
\externaldocument[spaces-morphisms-]{spaces-morphisms}
\externaldocument[decent-spaces-]{decent-spaces}
\externaldocument[spaces-cohomology-]{spaces-cohomology}
\externaldocument[spaces-limits-]{spaces-limits}
\externaldocument[spaces-divisors-]{spaces-divisors}
\externaldocument[spaces-over-fields-]{spaces-over-fields}
\externaldocument[spaces-topologies-]{spaces-topologies}
\externaldocument[spaces-descent-]{spaces-descent}
\externaldocument[spaces-perfect-]{spaces-perfect}
\externaldocument[spaces-more-morphisms-]{spaces-more-morphisms}
\externaldocument[spaces-flat-]{spaces-flat}
\externaldocument[spaces-groupoids-]{spaces-groupoids}
\externaldocument[spaces-more-groupoids-]{spaces-more-groupoids}
\externaldocument[bootstrap-]{bootstrap}
\externaldocument[spaces-pushouts-]{spaces-pushouts}
\externaldocument[spaces-chow-]{spaces-chow}
\externaldocument[groupoids-quotients-]{groupoids-quotients}
\externaldocument[spaces-more-cohomology-]{spaces-more-cohomology}
\externaldocument[spaces-simplicial-]{spaces-simplicial}
\externaldocument[spaces-duality-]{spaces-duality}
\externaldocument[formal-spaces-]{formal-spaces}
\externaldocument[restricted-]{restricted}
\externaldocument[spaces-resolve-]{spaces-resolve}
\externaldocument[formal-defos-]{formal-defos}
\externaldocument[defos-]{defos}
\externaldocument[cotangent-]{cotangent}
\externaldocument[examples-defos-]{examples-defos}
\externaldocument[algebraic-]{algebraic}
\externaldocument[examples-stacks-]{examples-stacks}
\externaldocument[stacks-sheaves-]{stacks-sheaves}
\externaldocument[criteria-]{criteria}
\externaldocument[artin-]{artin}
\externaldocument[quot-]{quot}
\externaldocument[stacks-properties-]{stacks-properties}
\externaldocument[stacks-morphisms-]{stacks-morphisms}
\externaldocument[stacks-limits-]{stacks-limits}
\externaldocument[stacks-cohomology-]{stacks-cohomology}
\externaldocument[stacks-perfect-]{stacks-perfect}
\externaldocument[stacks-introduction-]{stacks-introduction}
\externaldocument[stacks-more-morphisms-]{stacks-more-morphisms}
\externaldocument[stacks-geometry-]{stacks-geometry}
\externaldocument[moduli-]{moduli}
\externaldocument[moduli-curves-]{moduli-curves}
\externaldocument[examples-]{examples}
\externaldocument[exercises-]{exercises}
\externaldocument[guide-]{guide}
\externaldocument[desirables-]{desirables}
\externaldocument[coding-]{coding}
\externaldocument[obsolete-]{obsolete}
\externaldocument[fdl-]{fdl}
\externaldocument[index-]{index}

% Theorem environments.
%
\theoremstyle{plain}
\newtheorem{theorem}[subsection]{Theorem}
\newtheorem{proposition}[subsection]{Proposition}
\newtheorem{lemma}[subsection]{Lemma}

\theoremstyle{definition}
\newtheorem{definition}[subsection]{Definition}
\newtheorem{example}[subsection]{Example}
\newtheorem{exercise}[subsection]{Exercise}
\newtheorem{situation}[subsection]{Situation}

\theoremstyle{remark}
\newtheorem{remark}[subsection]{Remark}
\newtheorem{remarks}[subsection]{Remarks}

\numberwithin{equation}{subsection}

% Macros
%
\def\lim{\mathop{\mathrm{lim}}\nolimits}
\def\colim{\mathop{\mathrm{colim}}\nolimits}
\def\Spec{\mathop{\mathrm{Spec}}}
\def\Hom{\mathop{\mathrm{Hom}}\nolimits}
\def\Ext{\mathop{\mathrm{Ext}}\nolimits}
\def\SheafHom{\mathop{\mathcal{H}\!\mathit{om}}\nolimits}
\def\SheafExt{\mathop{\mathcal{E}\!\mathit{xt}}\nolimits}
\def\Sch{\mathit{Sch}}
\def\Mor{\mathop{Mor}\nolimits}
\def\Ob{\mathop{\mathrm{Ob}}\nolimits}
\def\Sh{\mathop{\mathit{Sh}}\nolimits}
\def\NL{\mathop{N\!L}\nolimits}
\def\proetale{{pro\text{-}\acute{e}tale}}
\def\etale{{\acute{e}tale}}
\def\QCoh{\mathit{QCoh}}
\def\Ker{\mathop{\mathrm{Ker}}}
\def\Im{\mathop{\mathrm{Im}}}
\def\Coker{\mathop{\mathrm{Coker}}}
\def\Coim{\mathop{\mathrm{Coim}}}

%
% Macros for moduli stacks/spaces
%
\def\QCohstack{\mathcal{QC}\!\mathit{oh}}
\def\Cohstack{\mathcal{C}\!\mathit{oh}}
\def\Spacesstack{\mathcal{S}\!\mathit{paces}}
\def\Quotfunctor{\mathrm{Quot}}
\def\Hilbfunctor{\mathrm{Hilb}}
\def\Curvesstack{\mathcal{C}\!\mathit{urves}}
\def\Polarizedstack{\mathcal{P}\!\mathit{olarized}}
\def\Complexesstack{\mathcal{C}\!\mathit{omplexes}}
% \Pic is the operator that assigns to X its picard group, usage \Pic(X)
% \Picardstack_{X/B} denotes the Picard stack of X over B
% \Picardfunctor_{X/B} denotes the Picard functor of X over B
\def\Pic{\mathop{\mathrm{Pic}}\nolimits}
\def\Picardstack{\mathcal{P}\!\mathit{ic}}
\def\Picardfunctor{\mathrm{Pic}}
\def\Deformationcategory{\mathcal{D}\!\mathit{ef}}


% OK, start here.
%
\begin{document}

\title{Groupoid Schemes}


\maketitle

\phantomsection
\label{section-phantom}

\tableofcontents

\section{Introduction}
\label{section-introduction}

\noindent
This chapter is devoted to generalities concerning groupoid schemes.
See for example the beautiful paper \cite{K-M} by Keel and Mori.





\section{Notation}
\label{section-notation}

\noindent
Let $S$ be a scheme. If $U$, $T$ are schemes over $S$ we denote
$U(T)$ for the set of $T$-valued points of $U$ {\it over} $S$. In a formula:
$U(T) = \Mor_S(T, U)$. We try to reserve the letter $T$ to denote
a ``test scheme'' over $S$, as in the discussion that follows.
Suppose we are given schemes $X$, $Y$ over
$S$ and a morphism of schemes $f : X \to Y$ over $S$.
For any scheme $T$ over $S$ we get an induced map of sets
$$
f : X(T) \longrightarrow Y(T)
$$
which as indicated we denote by $f$ also. In fact this construction
is functorial in the scheme $T/S$. Yoneda's Lemma, see Categories,
Lemma \ref{categories-lemma-yoneda}, says that $f$ determines and is
determined by this transformation of functors $f : h_X \to h_Y$.
More generally, we use the same notation for maps between fibre
products. For example, if
$X$, $Y$, $Z$ are schemes over $S$, and if
$m : X \times_S Y \to Z \times_S Z$ is
a morphism of schemes over $S$, then we think of $m$ as corresponding
to a collection of maps between $T$-valued points
$$
X(T) \times Y(T) \longrightarrow Z(T) \times Z(T).
$$
And so on and so forth.

\medskip\noindent
We continue our convention to label projection maps starting with
index $0$, so we have $\text{pr}_0 : X \times_S Y \to X$ and
$\text{pr}_1 : X \times_S Y \to Y$.






\section{Equivalence relations}
\label{section-equivalence-relations}

\noindent
Recall that a {\it relation} $R$ on a set $A$ is just a subset
of $R \subset A \times A$. We usually write $a R b$ to indicate
$(a, b) \in R$. We say the relation is {\it transitive} if
$a R b, b R c \Rightarrow a R c$. We say the relation is
{\it reflexive} if $a R a$ for all $a \in A$. We say the relation is
{\it symmetric} if $a R b \Rightarrow b R a$.
A relation is called an {\it equivalence relation} if
it is transitive, reflexive and symmetric.

\medskip\noindent
In the setting of schemes we are going to relax the notion of a
relation a little bit and just require $R \to A \times A$ to
be a map. Here is the definition.

\begin{definition}
\label{definition-equivalence-relation}
Let $S$ be a scheme. Let $U$ be a scheme over $S$.
\begin{enumerate}
\item A {\it pre-relation} on $U$ over $S$ is any morphism
of schemes $j : R \to U \times_S U$. In this case we set
$t = \text{pr}_0 \circ j$ and $s = \text{pr}_1 \circ j$, so
that $j = (t, s)$.
\item A {\it relation} on $U$ over $S$ is a monomorphism
of schemes $j : R \to U \times_S U$.
\item A {\it pre-equivalence relation} is a pre-relation
$j : R \to U \times_S U$ such that the image of
$j : R(T) \to U(T) \times U(T)$ is an equivalence relation for
all $T/S$.
\item We say a morphism $R \to U \times_S U$ of schemes is
an {\it equivalence relation on $U$ over $S$}
if and only if for every scheme $T$ over $S$ the $T$-valued
points of $R$ define an equivalence relation
on the set of $T$-valued points of $U$.
\end{enumerate}
\end{definition}

\noindent
In other words, an equivalence relation is a pre-equivalence relation
such that $j$ is a relation.

\begin{lemma}
\label{lemma-restrict-relation}
Let $S$ be a scheme.
Let $U$ be a scheme over $S$.
Let $j : R \to U \times_S U$ be a pre-relation.
Let $g : U' \to U$ be a morphism of schemes.
Finally, set
$$
R' = (U' \times_S U')\times_{U \times_S U} R
\xrightarrow{j'}
U' \times_S U'
$$
Then $j'$ is a pre-relation on $U'$ over $S$.
If $j$ is a relation, then $j'$ is a relation.
If $j$ is a pre-equivalence relation, then $j'$ is a pre-equivalence relation.
If $j$ is an equivalence relation, then $j'$ is an equivalence relation.
\end{lemma}

\begin{proof}
Omitted.
\end{proof}

\begin{definition}
\label{definition-restrict-relation}
Let $S$ be a scheme.
Let $U$ be a scheme over $S$.
Let $j : R \to U \times_S U$ be a pre-relation.
Let $g : U' \to U$ be a morphism of schemes.
The pre-relation $j' : R' \to U' \times_S U'$ is called
the {\it restriction}, or {\it pullback} of the pre-relation $j$ to $U'$.
In this situation we sometimes write $R' = R|_{U'}$.
\end{definition}

\begin{lemma}
\label{lemma-pre-equivalence-equivalence-relation-points}
Let $j : R \to U \times_S U$ be a pre-relation.
Consider the relation on points of the scheme $U$ defined by
the rule
$$
x \sim y
\Leftrightarrow
\exists\ r \in R :
t(r) = x,
s(r) = y.
$$
If $j$ is a pre-equivalence relation then this is an
equivalence relation.
\end{lemma}

\begin{proof}
Suppose that $x \sim y$ and $y \sim z$.
Pick $r \in R$ with $t(r) = x$, $s(r) = y$ and
pick $r' \in R$ with $t(r') = y$, $s(r') = z$.
Pick a field $K$ fitting into the following commutative
diagram
$$
\xymatrix{
\kappa(r) \ar[r] & K \\
\kappa(y) \ar[u] \ar[r] & \kappa(r') \ar[u]
}
$$
Denote $x_K, y_K, z_K : \Spec(K) \to U$
the morphisms
$$
\begin{matrix}
\Spec(K) \to \Spec(\kappa(r))
\to
\Spec(\kappa(x)) \to U \\
\Spec(K) \to \Spec(\kappa(r))
\to
\Spec(\kappa(y)) \to U \\
\Spec(K) \to \Spec(\kappa(r'))
\to
\Spec(\kappa(z)) \to U
\end{matrix}
$$
By construction $(x_K, y_K) \in j(R(K))$ and
$(y_K, z_K) \in j(R(K))$. Since $j$ is a pre-equivalence relation
we see that also $(x_K, z_K) \in j(R(K))$.
This clearly implies that $x \sim z$.

\medskip\noindent
The proof that $\sim$ is reflexive and symmetric is omitted.
\end{proof}

\begin{lemma}
\label{lemma-etale-equivalence-relation}
Let $j : R \to U \times_S U$ be a pre-relation. Assume
\begin{enumerate}
\item $s, t$ are unramified,
\item for any algebraically closed field $k$ over $S$
the map $R(k) \to U(k) \times U(k)$ is an equivalence relation,
\item there are morphisms $e : U \to R$, $i : R \to R$,
$c : R \times_{s, U, t} R \to R$ such that
$$
\xymatrix{
U \ar[r]_e \ar[d]_\Delta &
R \ar[d]_j &
R \ar[d]^j \ar[r]_i &
R \ar[d]^j &
R \times_{s, U, t} R \ar[d]^{j \times j} \ar[r]_c &
R \ar[d]^j \\
U \times_S U \ar[r] &
U \times_S U &
U \times_S U \ar[r]^{flip} &
U \times_S U &
U \times_S U \times_S U \ar[r]^{\text{pr}_{02}} &
U \times_S U
}
$$
are commutative.
\end{enumerate}
Then $j$ is an equivalence relation.
\end{lemma}

\begin{proof}
By condition (1) and
Morphisms, Lemma \ref{morphisms-lemma-unramified-permanence}
we see that $j$ is a unramified. Then
$\Delta_j : R \to R \times_{U \times_S U} R$ is an open immersion by
Morphisms, Lemma \ref{morphisms-lemma-diagonal-unramified-morphism}.
However, then condition (2) says $\Delta_j$ is bijective on
$k$-valued points, hence $\Delta_j$ is an isomorphism, hence $j$
is a monomorphism. Then it easily follows from the commutative
diagrams that $R(T) \subset U(T) \times U(T)$ is an equivalence
relation for all schemes $T$ over $S$.
\end{proof}













\section{Group schemes}
\label{section-group-schemes}

\noindent
Let us recall that a {\it group} is a pair
$(G, m)$ where $G$ is a set, and $m : G \times G \to G$ is
a map of sets with the following properties:
\begin{enumerate}
\item (associativity) $m(g, m(g', g'')) = m(m(g, g'), g'')$
for all $g, g', g'' \in G$,
\item (identity) there exists a unique element $e \in G$
(called the {\it identity}, {\it unit}, or $1$ of $G$) such that
$m(g, e) = m(e, g) = g$ for all $g \in G$, and
\item (inverse) for all $g \in G$ there exists a $i(g) \in G$
such that $m(g, i(g)) = m(i(g), g) = e$, where $e$ is the
identity.
\end{enumerate}
Thus we obtain a map $e : \{*\} \to G$ and a map
$i : G \to G$ so that the quadruple $(G, m, e, i)$
satisfies the axioms listed above.

\medskip\noindent
A {\it homomorphism of groups} $\psi : (G, m) \to (G', m')$
is a map of sets $\psi : G \to G'$ such that
$m'(\psi(g), \psi(g')) = \psi(m(g, g'))$. This automatically
insures that $\psi(e) = e'$ and $i'(\psi(g)) = \psi(i(g))$.
(Obvious notation.) We will use this below.

\begin{definition}
\label{definition-group-scheme}
Let $S$ be a scheme.
\begin{enumerate}
\item A {\it group scheme over $S$} is a pair $(G, m)$, where
$G$ is a scheme over $S$ and $m : G \times_S G \to G$ is
a morphism of schemes over $S$ with the following property:
For every scheme $T$ over $S$ the pair $(G(T), m)$
is a group.
\item A {\it morphism $\psi : (G, m) \to (G', m')$ of group schemes over $S$}
is a morphism $\psi : G \to G'$ of schemes over $S$ such that for
every $T/S$ the induced map $\psi : G(T) \to G'(T)$ is a homomorphism
of groups.
\end{enumerate}
\end{definition}

\noindent
Let $(G, m)$ be a group scheme over the scheme $S$.
By the discussion above (and the discussion in Section \ref{section-notation})
we obtain morphisms of schemes over $S$:
(identity) $e : S \to G$ and (inverse) $i : G \to G$ such that
for every $T$ the quadruple $(G(T), m, e, i)$ satisfies the
axioms of a group listed above.

\medskip\noindent
Let $(G, m)$, $(G', m')$ be group schemes over $S$.
Let $f : G \to G'$ be a morphism of schemes over $S$.
It follows from the definition that $f$ is a morphism
of group schemes over $S$ if and only if the following diagram
is commutative:
$$
\xymatrix{
G \times_S G \ar[r]_-{f \times f} \ar[d]_m &
G' \times_S G' \ar[d]^m \\
G \ar[r]^f & G'
}
$$

\begin{lemma}
\label{lemma-base-change-group-scheme}
Let $(G, m)$ be a group scheme over $S$.
Let $S' \to S$ be a morphism of schemes.
The pullback $(G_{S'}, m_{S'})$ is a group scheme over $S'$.
\end{lemma}

\begin{proof}
Omitted.
\end{proof}

\begin{definition}
\label{definition-closed-subgroup-scheme}
Let $S$ be a scheme. Let $(G, m)$ be a group scheme over $S$.
\begin{enumerate}
\item A {\it closed subgroup scheme} of $G$ is a closed subscheme
$H \subset G$ such that $m|_{H \times_S H}$ factors through $H$ and induces a
group scheme structure on $H$ over $S$.
\item An {\it open subgroup scheme} of $G$ is an open subscheme
$G' \subset G$ such that $m|_{G' \times_S G'}$ factors through $G'$
and induces a group scheme structure on $G'$ over $S$.
\end{enumerate}
\end{definition}

\noindent
Alternatively, we could say that $H$ is a closed subgroup scheme of $G$
if it is a group scheme over $S$ endowed with a morphism of group schemes
$i : H \to G$ over $S$ which identifies $H$ with a closed subscheme of $G$.

\begin{definition}
\label{definition-smooth-group-scheme}
Let $S$ be a scheme. Let $(G, m)$ be a group scheme over $S$.
\begin{enumerate}
\item We say $G$ is a {\it smooth group scheme} if the structure
morphism $G \to S$ is smooth.
\item We say $G$ is a {\it flat group scheme} if the structure
morphism $G \to S$ is flat.
\item We say $G$ is a {\it separated group scheme} if the structure
morphism $G \to S$ is separated.
\end{enumerate}
Add more as needed.
\end{definition}






\section{Examples of group schemes}
\label{section-examples-group-schemes}

\begin{example}[Multiplicative group scheme]
\label{example-multiplicative-group}
Consider the functor which associates
to any scheme $T$ the group $\Gamma(T, \mathcal{O}_T^*)$
of units in the global sections of the structure sheaf.
This is representable by the scheme
$$
\mathbf{G}_m = \Spec(\mathbf{Z}[x, x^{-1}])
$$
The morphism giving the group structure is the morphism
\begin{eqnarray*}
\mathbf{G}_m \times \mathbf{G}_m & \to & \mathbf{G}_m \\
\Spec(\mathbf{Z}[x, x^{-1}] \otimes_{\mathbf{Z}} \mathbf{Z}[x, x^{-1}])
& \to &
\Spec(\mathbf{Z}[x, x^{-1}]) \\
\mathbf{Z}[x, x^{-1}] \otimes_{\mathbf{Z}} \mathbf{Z}[x, x^{-1}]
& \leftarrow &
\mathbf{Z}[x, x^{-1}] \\
x \otimes x & \leftarrow & x
\end{eqnarray*}
Hence we see that $\mathbf{G}_m$ is a group scheme over $\mathbf{Z}$.
For any scheme $S$ the base change $\mathbf{G}_{m, S}$ is a
group scheme over $S$ whose functor of points is
$$
T/S
\longmapsto
\mathbf{G}_{m, S}(T) = \mathbf{G}_m(T) = \Gamma(T, \mathcal{O}_T^*)
$$
as before.
\end{example}

\begin{example}[Roots of unity]
\label{example-roots-of-unity}
Let $n \in \mathbf{N}$.
Consider the functor which associates
to any scheme $T$ the subgroup of $\Gamma(T, \mathcal{O}_T^*)$
consisting of $n$th roots of unity.
This is representable by the scheme
$$
\mu_n = \Spec(\mathbf{Z}[x]/(x^n - 1)).
$$
The morphism giving the group structure is the morphism
\begin{eqnarray*}
\mu_n \times \mu_n & \to & \mu_n \\
\Spec(
\mathbf{Z}[x]/(x^n - 1)
\otimes_{\mathbf{Z}}
\mathbf{Z}[x]/(x^n - 1))
& \to &
\Spec(\mathbf{Z}[x]/(x^n - 1)) \\
\mathbf{Z}[x]/(x^n - 1) \otimes_{\mathbf{Z}} \mathbf{Z}[x]/(x^n - 1)
& \leftarrow &
\mathbf{Z}[x]/(x^n - 1) \\
x \otimes x & \leftarrow & x
\end{eqnarray*}
Hence we see that $\mu_n$ is a group scheme over $\mathbf{Z}$.
For any scheme $S$ the base change $\mu_{n, S}$ is a
group scheme over $S$ whose functor of points is
$$
T/S
\longmapsto
\mu_{n, S}(T) = \mu_n(T) = \{f \in \Gamma(T, \mathcal{O}_T^*) \mid f^n = 1\}
$$
as before.
\end{example}


\begin{example}[Additive group scheme]
\label{example-additive-group}
Consider the functor which associates
to any scheme $T$ the group $\Gamma(T, \mathcal{O}_T)$
of global sections of the structure sheaf.
This is representable by the scheme
$$
\mathbf{G}_a = \Spec(\mathbf{Z}[x])
$$
The morphism giving the group structure is the morphism
\begin{eqnarray*}
\mathbf{G}_a \times \mathbf{G}_a & \to & \mathbf{G}_a \\
\Spec(\mathbf{Z}[x] \otimes_{\mathbf{Z}} \mathbf{Z}[x])
& \to &
\Spec(\mathbf{Z}[x]) \\
\mathbf{Z}[x] \otimes_{\mathbf{Z}} \mathbf{Z}[x]
& \leftarrow &
\mathbf{Z}[x] \\
x \otimes 1 + 1 \otimes x & \leftarrow & x
\end{eqnarray*}
Hence we see that $\mathbf{G}_a$ is a group scheme over $\mathbf{Z}$.
For any scheme $S$ the base change $\mathbf{G}_{a, S}$ is a
group scheme over $S$ whose functor of points is
$$
T/S
\longmapsto
\mathbf{G}_{a, S}(T) = \mathbf{G}_a(T) = \Gamma(T, \mathcal{O}_T)
$$
as before.
\end{example}

\begin{example}[General linear group scheme]
\label{example-general-linear-group}
Let $n \geq 1$.
Consider the functor which associates
to any scheme $T$ the group
$$
\text{GL}_n(\Gamma(T, \mathcal{O}_T))
$$
of invertible $n \times n$ matrices over
the global sections of the structure sheaf.
This is representable by the scheme
$$
\text{GL}_n = \Spec(\mathbf{Z}[\{x_{ij}\}_{1 \leq i, j \leq n}][1/d])
$$
where $d = \det((x_{ij}))$ with $(x_{ij})$ the $n \times n$ matrix
with entry $x_{ij}$ in the $(i, j)$-spot.
The morphism giving the group structure is the morphism
\begin{eqnarray*}
\text{GL}_n \times \text{GL}_n & \to & \text{GL}_n \\
\Spec(\mathbf{Z}[x_{ij}, 1/d] \otimes_{\mathbf{Z}}
\mathbf{Z}[x_{ij}, 1/d])
& \to &
\Spec(\mathbf{Z}[x_{ij}, 1/d]) \\
\mathbf{Z}[x_{ij}, 1/d] \otimes_{\mathbf{Z}} \mathbf{Z}[x_{ij}, 1/d]
& \leftarrow &
\mathbf{Z}[x_{ij}, 1/d] \\
\sum x_{ik} \otimes x_{kj} & \leftarrow & x_{ij}
\end{eqnarray*}
Hence we see that $\text{GL}_n$ is a group scheme over $\mathbf{Z}$.
For any scheme $S$ the base change $\text{GL}_{n, S}$ is a
group scheme over $S$ whose functor of points is
$$
T/S
\longmapsto
\text{GL}_{n, S}(T) = \text{GL}_n(T) =\text{GL}_n(\Gamma(T, \mathcal{O}_T))
$$
as before.
\end{example}

\begin{example}
\label{example-determinant}
The determinant defines a morphism of group schemes
$$
\det : \text{GL}_n \longrightarrow \mathbf{G}_m
$$
over $\mathbf{Z}$. By base change it gives a morphism
of group schemes $\text{GL}_{n, S} \to \mathbf{G}_{m, S}$
over any base scheme $S$.
\end{example}

\begin{example}[Constant group]
\label{example-constant-group}
Let $G$ be an abstract group. Consider the functor
which associates to any scheme $T$ the group
of locally constant maps $T \to G$ (where $T$ has the Zariski topology
and $G$ the discrete topology). This is representable by the scheme
$$
G_{\Spec(\mathbf{Z})} =
\coprod\nolimits_{g \in G} \Spec(\mathbf{Z}).
$$
The morphism giving the group structure is the morphism
$$
G_{\Spec(\mathbf{Z})}
\times_{\Spec(\mathbf{Z})}
G_{\Spec(\mathbf{Z})}
\longrightarrow
G_{\Spec(\mathbf{Z})}
$$
which maps the component corresponding to the pair $(g, g')$ to the
component corresponding to $gg'$. For any scheme $S$ the base change
$G_S$ is a group scheme over $S$ whose functor of points is
$$
T/S
\longmapsto
G_S(T) = \{f : T \to G \text{ locally constant}\}
$$
as before.
\end{example}





\section{Properties of group schemes}
\label{section-properties-group-schemes}

\noindent
In this section we collect some simple properties of group schemes which
hold over any base.

\begin{lemma}
\label{lemma-group-scheme-separated}
Let $S$ be a scheme.
Let $G$ be a group scheme over $S$.
Then $G \to S$ is separated (resp.\ quasi-separated) if and only if
the identity morphism $e : S \to G$ is a closed immersion
(resp.\ quasi-compact).
\end{lemma}

\begin{proof}
We recall that by
Schemes, Lemma \ref{schemes-lemma-section-immersion}
we have that $e$ is an immersion which is a closed immersion
(resp.\ quasi-compact) if $G \to S$ is separated (resp.\ quasi-separated).
For the converse, consider the diagram
$$
\xymatrix{
G \ar[r]_-{\Delta_{G/S}} \ar[d] &
G \times_S G \ar[d]^{(g, g') \mapsto m(i(g), g')} \\
S \ar[r]^e & G
}
$$
It is an exercise in the functorial point of view in algebraic geometry
to show that this diagram is cartesian. In other words, we see that
$\Delta_{G/S}$ is a base change of $e$. Hence if $e$ is a
closed immersion (resp.\ quasi-compact) so is $\Delta_{G/S}$, see
Schemes, Lemma \ref{schemes-lemma-base-change-immersion}
(resp.\ Schemes, Lemma
\ref{schemes-lemma-quasi-compact-preserved-base-change}).
\end{proof}

\begin{lemma}
\label{lemma-flat-action-on-group-scheme}
Let $S$ be a scheme.
Let $G$ be a group scheme over $S$.
Let $T$ be a scheme over $S$ and let $\psi : T \to G$ be a morphism over $S$.
If $T$ is flat over $S$, then the morphism
$$
T \times_S G \longrightarrow G, \quad
(t, g) \longmapsto m(\psi(t), g)
$$
is flat. In particular, if $G$ is flat over $S$, then
$m : G \times_S G \to G$ is flat.
\end{lemma}

\begin{proof}
Consider the diagram
$$
\xymatrix{
T \times_S G \ar[rrr]_{(t, g) \mapsto (t, m(\psi(t), g))} & & &
T \times_S G \ar[r]_{\text{pr}} \ar[d] &
G \ar[d] \\
& & &
T \ar[r] &
S
}
$$
The left top horizontal arrow is an isomorphism and the
square is cartesian. Hence the lemma follows from
Morphisms, Lemma \ref{morphisms-lemma-base-change-flat}.
\end{proof}

\begin{lemma}
\label{lemma-group-scheme-module-differentials}
Let $(G, m, e, i)$ be a group scheme over the scheme $S$.
Denote $f : G \to S$ the structure morphism. Assume $f$ is flat.
Then there exist canonical isomorphisms
$$
\Omega_{G/S} \cong f^*\mathcal{C}_{S/G} \cong f^*e^*\Omega_{G/S}
$$
where $\mathcal{C}_{S/G}$ denotes the conormal sheaf of the
immersion $e$. In particular, if $S$ is the spectrum of a field, then
$\Omega_{G/S}$ is a free $\mathcal{O}_G$-module.
\end{lemma}

\begin{proof}
In
Morphisms, Lemma \ref{morphisms-lemma-differentials-diagonal}
we identified $\Omega_{G/S}$ with the conormal sheaf of the
diagonal morphism $\Delta_{G/S}$. In the proof of
Lemma \ref{lemma-group-scheme-separated}
we showed that $\Delta_{G/S}$ is a base change of the immersion $e$
by the morphism $(g, g') \mapsto m(i(g), g')$. This morphism
is isomorphic to the morphism $(g, g') \mapsto m(g, g')$
hence is flat by
Lemma \ref{lemma-flat-action-on-group-scheme}.
Hence we get the first isomorphism by
Morphisms, Lemma \ref{morphisms-lemma-conormal-functorial-flat}.
By
Morphisms, Lemma \ref{morphisms-lemma-differentials-relative-immersion-section}
we have $\mathcal{C}_{S/G} \cong e^*\Omega_{G/S}$.

\medskip\noindent
If $S$ is the spectrum of a field, then $G \to S$ is flat, and
any $\mathcal{O}_S$-module on $S$ is free.
\end{proof}

\begin{lemma}
\label{lemma-group-scheme-addition-tangent-vectors}
Let $S$ be a scheme. Let $G$ be a group scheme over $S$.
Let $s \in S$. Then the composition
$$
T_{G/S, e(s)} \oplus T_{G/S, e(s)} = T_{G \times_S G/S, (e(s), e(s))}
\rightarrow T_{G/S, e(s)}
$$
is addition of tangent vectors. Here the $=$ comes from
Varieties, Lemma \ref{varieties-lemma-tangent-space-product}
and the right arrow is induced from $m : G \times_S G \to G$ via
Varieties, Lemma \ref{varieties-lemma-map-tangent-spaces}.
\end{lemma}

\begin{proof}
We will use Varieties, Equation (\ref{varieties-equation-tangent-space-fibre})
and work with tangent vectors in fibres.
An element $\theta$ in the first factor $T_{G_s/s, e(s)}$
is the image of $\theta$ via the map
$T_{G_s/s, e(s)} \to T_{G_s \times G_s/s, (e(s), e(s))}$
coming from $(1, e) : G_s \to G_s \times G_s$.
Since $m \circ (1, e) = 1$ we see that $\theta$ maps to $\theta$
by functoriality. Since the map is linear we see that
$(\theta_1, \theta_2)$ maps to $\theta_1 + \theta_2$.
\end{proof}





\section{Properties of group schemes over a field}
\label{section-properties-group-schemes-field}

\noindent
In this section we collect some properties of group schemes over a
field. In the case of group schemes which are (locally) algebraic
over a field we can say a lot more, see
Section \ref{section-algebraic-group-schemes}.

\begin{lemma}
\label{lemma-group-scheme-over-field-open-multiplication}
If $(G, m)$ is a group scheme over a field $k$, then the
multiplication map $m : G \times_k G \to G$ is open.
\end{lemma}

\begin{proof}
The multiplication map is isomorphic to the projection map
$\text{pr}_0 : G \times_k G \to G$
because the diagram
$$
\xymatrix{
G \times_k G \ar[d]^m \ar[rrr]_{(g, g') \mapsto (m(g, g'), g')} & & &
G \times_k G \ar[d]^{(g, g') \mapsto g} \\
G \ar[rrr]^{\text{id}} & & & G
}
$$
is commutative with isomorphisms as horizontal arrows. The projection
is open by
Morphisms, Lemma \ref{morphisms-lemma-scheme-over-field-universally-open}.
\end{proof}

\begin{lemma}
\label{lemma-group-scheme-over-field-translate-open}
If $(G, m)$ is a group scheme over a field $k$. Let $U \subset G$
open and $T \to G$ a morphism of schemes. Then the image of the
composition $T \times_k U \to G \times_k G \to G$ is open.
\end{lemma}

\begin{proof}
For any field extension $k \subset K$ the morphism $G_K \to G$ is open
(Morphisms, Lemma \ref{morphisms-lemma-scheme-over-field-universally-open}).
Every point $\xi$ of $T \times_k U$ is the image of a morphism
$(t, u) : \Spec(K) \to T \times_k U$ for some $K$. Then the image of
$T_K \times_K U_K = (T \times_k U)_K \to G_K$ contains the translate
$t \cdot U_K$ which is open. Combining these facts we see that the
image of $T \times_k U \to G$ contains an open neighbourhood of
the image of $\xi$. Since $\xi$ was arbitrary we win.
\end{proof}

\begin{lemma}
\label{lemma-group-scheme-over-field-separated}
Let $G$ be a group scheme over a field.
Then $G$ is a separated scheme.
\end{lemma}

\begin{proof}
Say $S = \Spec(k)$ with $k$ a field, and let $G$ be a group scheme
over $S$. By
Lemma \ref{lemma-group-scheme-separated}
we have to show that $e : S \to G$ is a closed immersion. By
Morphisms, Lemma
\ref{morphisms-lemma-algebraic-residue-field-extension-closed-point-fibre}
the image of $e : S \to G$ is a closed point of $G$.
It is clear that $\mathcal{O}_G \to e_*\mathcal{O}_S$ is surjective,
since $e_*\mathcal{O}_S$ is a skyscraper sheaf supported at the neutral
element of $G$ with value $k$. We conclude that $e$ is a closed immersion by
Schemes, Lemma \ref{schemes-lemma-characterize-closed-immersions}.
\end{proof}

\begin{lemma}
\label{lemma-group-scheme-field-geometrically-irreducible}
Let $G$ be a group scheme over a field $k$.
Then
\begin{enumerate}
\item every local ring $\mathcal{O}_{G, g}$ of $G$ has a unique
minimal prime ideal,
\item there is exactly one irreducible component $Z$ of $G$
passing through $e$, and
\item $Z$ is geometrically irreducible over $k$.
\end{enumerate}
\end{lemma}

\begin{proof}
For any point $g \in G$ there exists a field extension
$k \subset K$ and a $K$-valued point $g' \in G(K)$ mapping to
$g$. If we think of $g'$ as a $K$-rational point of the
group scheme $G_K$, then we see that
$\mathcal{O}_{G, g} \to \mathcal{O}_{G_K, g'}$ is a faithfully flat
local ring map (as $G_K \to G$ is flat, and a local flat ring map
is faithfully flat, see
Algebra, Lemma \ref{algebra-lemma-local-flat-ff}).
The result for $\mathcal{O}_{G_K, g'}$ implies the
result for $\mathcal{O}_{G, g}$, see
Algebra, Lemma \ref{algebra-lemma-injective-minimal-primes-in-image}.
Hence in order to prove (1) it suffices to
prove it for $k$-rational points $g$ of $G$. In this case
translation by $g$ defines an automorphism $G \to G$
which maps $e$ to $g$. Hence $\mathcal{O}_{G, g} \cong \mathcal{O}_{G, e}$.
In this way we see that (2) implies (1), since irreducible components
passing through $e$ correspond one to one with minimal prime ideals
of $\mathcal{O}_{G, e}$.

\medskip\noindent
In order to prove (2) and (3) it suffices to prove (2) when $k$
is algebraically closed. In this case, let $Z_1$, $Z_2$ be two
irreducible components of $G$ passing through $e$.
Since $k$ is algebraically closed the closed subscheme
$Z_1 \times_k Z_2 \subset G \times_k G$ is irreducible too, see
Varieties, Lemma \ref{varieties-lemma-bijection-irreducible-components}.
Hence $m(Z_1 \times_k Z_2)$ is contained in an irreducible
component of $G$. On the other hand it contains
$Z_1$ and $Z_2$ since $m|_{e \times G} = \text{id}_G$ and
$m|_{G \times e} = \text{id}_G$. We conclude $Z_1 = Z_2$ as desired.
\end{proof}

\begin{remark}
\label{remark-warning-group-scheme-geometrically-irreducible}
Warning: The result of
Lemma \ref{lemma-group-scheme-field-geometrically-irreducible}
does not mean that every irreducible component of $G/k$ is
geometrically irreducible. For example the group scheme
$\mu_{3, \mathbf{Q}} = \Spec(\mathbf{Q}[x]/(x^3 - 1))$
over $\mathbf{Q}$ has two irreducible components corresponding
to the factorization $x^3 - 1 = (x - 1)(x^2 + x + 1)$.
The first factor corresponds to the irreducible component
passing through the identity, and the second irreducible component
is not geometrically irreducible over $\Spec(\mathbf{Q})$.
\end{remark}

\begin{lemma}
\label{lemma-reduced-subgroup-scheme-perfect}
Let $G$ be a group scheme over a perfect field $k$.
Then the reduction $G_{red}$ of $G$ is a closed subgroup scheme of $G$.
\end{lemma}

\begin{proof}
Omitted. Hint: Use that $G_{red} \times_k G_{red}$ is reduced by
Varieties, Lemmas \ref{varieties-lemma-perfect-reduced} and
\ref{varieties-lemma-geometrically-reduced-any-base-change}.
\end{proof}

\begin{lemma}
\label{lemma-open-subgroup-closed-over-field}
Let $k$ be a field. Let $\psi : G' \to G$ be a morphism of group schemes
over $k$. If $\psi(G')$ is open in $G$, then $\psi(G')$ is closed in $G$.
\end{lemma}

\begin{proof}
Let $U = \psi(G') \subset G$. Let $Z = G \setminus \psi(G') = G \setminus U$
with the reduced induced closed subscheme structure. By
Lemma \ref{lemma-group-scheme-over-field-translate-open}
the image of
$$
Z \times_k G' \longrightarrow
Z \times_k U \longrightarrow G
$$
is open (the first arrow is surjective). On the other hand, since $\psi$
is a homomorphism of group schemes, the image of $Z \times_k G' \to G$
is contained in $Z$ (because translation by $\psi(g')$ preserves
$U$ for all points $g'$ of $G'$; small detail omitted).
Hence $Z \subset G$ is an open subset (although not
necessarily an open subscheme). Thus $U = \psi(G')$ is closed.
\end{proof}

\begin{lemma}
\label{lemma-immersion-group-schemes-closed-over-field}
Let $i : G' \to G$ be an immersion of group schemes over a field $k$.
Then $i$ is a closed immersion, i.e., $i(G')$ is a closed subgroup scheme
of $G$.
\end{lemma}

\begin{proof}
To show that $i$ is a closed immersion it suffices to show that
$i(G')$ is a closed subset of $G$. Let $k \subset k'$ be a perfect
extension of $k$. If $i(G'_{k'}) \subset G_{k'}$ is closed, then
$i(G') \subset G$ is closed by
Morphisms, Lemma \ref{morphisms-lemma-fpqc-quotient-topology}
(as $G_{k'} \to G$ is flat, quasi-compact and surjective).
Hence we may and do assume $k$ is perfect. We will use without further
mention that products of reduced schemes over $k$ are reduced.
We may replace $G'$ and $G$ by their reductions, see
Lemma \ref{lemma-reduced-subgroup-scheme-perfect}.
Let $\overline{G'} \subset G$ be the closure of $i(G')$ viewed
as a reduced closed subscheme. By
Varieties, Lemma \ref{varieties-lemma-closure-of-product}
we conclude that $\overline{G'} \times_k \overline{G'}$
is the closure of the image of $G' \times_k G' \to G \times_k G$. Hence
$$
m\Big(\overline{G'} \times_k \overline{G'}\Big)
\subset \overline{G'}
$$
as $m$ is continuous. It follows that $\overline{G'} \subset G$
is a (reduced) closed subgroup scheme. By
Lemma \ref{lemma-open-subgroup-closed-over-field}
we see that $i(G') \subset \overline{G'}$ is also closed
which implies that $i(G') = \overline{G'}$ as desired.
\end{proof}

\begin{lemma}
\label{lemma-irreducible-group-scheme-over-field-qc}
Let $G$ be a group scheme over a field $k$. If $G$ is irreducible,
then $G$ is quasi-compact.
\end{lemma}

\begin{proof}
Suppose that $k \subset K$ is a field extension. If $G_K$
is quasi-compact, then $G$ is too as $G_K \to G$ is surjective.
By Lemma \ref{lemma-group-scheme-field-geometrically-irreducible}
we see that $G_K$ is irreducible. Hence it suffices to prove the lemma
after replacing $k$ by some extension. Choose $K$ to be an algebraically
closed field extension of very large cardinality. Then by
Varieties, Lemma \ref{varieties-lemma-make-Jacobson},
we see that $G_K$ is a Jacobson scheme all of whose closed points have residue
field equal to $K$. In other words we may assume $G$ is a Jacobson
scheme all of whose closed points have residue field $k$.

\medskip\noindent
Let $U \subset G$ be a nonempty affine open. Let $g \in G(k)$. Then
$gU \cap U \not = \emptyset$. Hence we see that $g$ is in the image
of the morphism
$$
U \times_{\Spec(k)} U \longrightarrow G, \quad
(u_1, u_2) \longmapsto u_1u_2^{-1}
$$
Since the image of this morphism is open
(Lemma \ref{lemma-group-scheme-over-field-open-multiplication})
we see that the image is all of $G$ (because $G$ is Jacobson
and closed points are $k$-rational).
Since $U$ is affine, so is $U \times_{\Spec(k)} U$. Hence $G$ is the
image of a quasi-compact scheme, hence quasi-compact.
\end{proof}

\begin{lemma}
\label{lemma-connected-group-scheme-over-field-irreducible}
Let $G$ be a group scheme over a field $k$. If $G$ is connected,
then $G$ is irreducible.
\end{lemma}

\begin{proof}
By Varieties, Lemma \ref{varieties-lemma-geometrically-connected-criterion}
we see that $G$ is geometrically connected. If we show that $G_K$
is irreducible for some field extension $k \subset K$, then
the lemma follows. Hence we may apply
Varieties, Lemma \ref{varieties-lemma-make-Jacobson}
to reduce to the case where $k$ is algebraically closed,
$G$ is a Jacobson scheme, and all the closed points are $k$-rational.

\medskip\noindent
Let $Z \subset G$ be the unique irreducible component of $G$ passing
through the neutral element, see
Lemma \ref{lemma-group-scheme-field-geometrically-irreducible}.
Endowing $Z$ with the reduced induced closed subscheme structure,
we see that $Z \times_k Z$ is reduced and irreducible
(Varieties, Lemmas
\ref{varieties-lemma-geometrically-reduced-any-base-change} and
\ref{varieties-lemma-bijection-irreducible-components}).
We conclude that $m|_{Z \times_k Z} : Z \times_k Z \to G$ factors
through $Z$. Hence $Z$ becomes a closed subgroup scheme of $G$.

\medskip\noindent
To get a contradiction, assume there exists another irreducible
component $Z' \subset G$. Then $Z \cap Z' = \emptyset$ by
Lemma \ref{lemma-group-scheme-field-geometrically-irreducible}.
By Lemma \ref{lemma-irreducible-group-scheme-over-field-qc}
we see that $Z$ is quasi-compact. Thus we may choose a quasi-compact open
$U \subset G$ with $Z \subset U$ and $U \cap Z' = \emptyset$.
The image $W$ of $Z \times_k U \to G$ is open in $G$ by
Lemma \ref{lemma-group-scheme-over-field-translate-open}.
On the other hand, $W$ is quasi-compact as the image of a
quasi-compact space. We claim that $W$ is closed.
If the claim is true, then $W \subset G \setminus Z'$ is a proper open
and closed subset of $G$, which contradicts the assumption that
$G$ is connected.

\medskip\noindent
Proof of the claim. Since $W$ is quasi-compact, we see that
points in the closure of $W$ are specializations of points of $W$
(Morphisms, Lemma \ref{morphisms-lemma-reach-points-scheme-theoretic-image}).
Thus we have to show that any irreducible
component $Z'' \subset G$ of $G$ which meets $W$ is contained in $W$.
As $G$ is Jacobson and closed points are rational,
$Z'' \cap W$ has a rational point
$g \in Z''(k) \cap W(k)$ and hence $Z'' = Zg$. But $W = m(Z \times_k W)$
by construction, so $Z'' \cap W \not = \emptyset$ implies
$Z'' \subset W$.
\end{proof}

\begin{proposition}
\label{proposition-connected-component}
Let $G$ be a group scheme over a field $k$. There exists a canonical closed
subgroup scheme $G^0 \subset G$ with the following properties
\begin{enumerate}
\item $G^0 \to G$ is a flat closed immersion,
\item $G^0 \subset G$ is the connected component of the identity,
\item $G^0$ is geometrically irreducible, and
\item $G^0$ is quasi-compact.
\end{enumerate}
\end{proposition}

\begin{proof}
Let $G^0$ be the connected component of the identity with its canonical
scheme structure (Morphisms, Definition
\ref{morphisms-definition-scheme-structure-connected-component}).
By Varieties, Lemma \ref{varieties-lemma-geometrically-connected-criterion}
we see that $G^0$ is geometrically connected. Thus
$G^0 \times_k G^0$ is connected
(Varieties, Lemma \ref{varieties-lemma-bijection-connected-components}).
Thus $m(G^0 \times_k G^0) \subset G^0$ set theoretically.
To see that this holds scheme theoretically, note that
$G^0 \times_k G^0 \to G \times_k G$ is a flat closed immersion.
By Morphisms, Lemma \ref{morphisms-lemma-characterize-flat-closed-immersions}
it follows that $G^0 \times_k G^0$ is a closed subscheme of
$(G \times_k G) \times_{m, G} G^0$. Thus we see that
$m|_{G^0 \times_k G^0} : G^0 \times_k G^0 \to G$ factors through
$G^0$. Hence $G^0$ becomes a closed subgroup scheme of $G$.
By Lemma \ref{lemma-connected-group-scheme-over-field-irreducible}
we see that $G^0$ is irreducible. By
Lemma \ref{lemma-group-scheme-field-geometrically-irreducible}
we see that $G^0$ is geometrically irreducible. By
Lemma \ref{lemma-irreducible-group-scheme-over-field-qc}
we see that $G^0$ is quasi-compact.
\end{proof}

\begin{lemma}
\label{lemma-profinite-product-over-field}
Let $k$ be a field. Let $T = \Spec(A)$ where $A$ is a directed colimit of
algebras which are finite products of copies of $k$. For any scheme $X$
over $k$ we have $|T \times_k X| = |T| \times |X|$ as topological spaces.
\end{lemma}

\begin{proof}
By taking an affine open covering we reduce to the case of an affine $X$.
Say $X = \Spec(B)$.
Write $A = \colim A_i$ with $A_i = \prod_{t \in T_i} k$ and $T_i$ finite.
Then $T_i = |\Spec(A_i)|$ with the discrete topology and the transition
morphisms $A_i \to A_{i'}$ are given by set maps $T_{i'} \to T_i$. Thus
$|T| = \lim T_i$ as a topological space, see
Limits, Lemma \ref{limits-lemma-topology-limit}. Similarly we have
\begin{align*}
|T \times_k X| & =
|\Spec(A \otimes_k B)| \\
& =
|\Spec(\colim A_i \otimes_k B)| \\
& =
\lim |\Spec(A_i \otimes_k B)| \\
& =
\lim |\Spec(\prod\nolimits_{t \in T_i} B)| \\
& =
\lim T_i \times |X| \\
& =
(\lim T_i) \times |X| \\
& =
|T| \times |X|
\end{align*}
by the lemma above and the fact that limits commute with limits.
\end{proof}

\noindent
The following lemma says that in fact we can put a
``algebraic profinite family of points'' in an affine open.
We urge the reader to read Lemma \ref{lemma-points-in-affine} first.

\begin{lemma}
\label{lemma-compact-set-in-affine}
Let $k$ be an algebraically closed field. Let $G$ be a group scheme over $k$.
Assume that $G$ is Jacobson and that all closed points are $k$-rational.
Let $T = \Spec(A)$ where $A$ is a directed colimit of algebras which
are finite products of copies of $k$. For any morphism $f : T \to G$
there exists an affine open $U \subset G$ containing $f(T)$.
\end{lemma}

\begin{proof}
Let $G^0 \subset G$ be the closed subgroup scheme found in
Proposition \ref{proposition-connected-component}. The first two paragraphs
serve to reduce to the case $G = G^0$.

\medskip\noindent
Observe that $T$ is a directed inverse limit of finite topological spaces
(Limits, Lemma \ref{limits-lemma-topology-limit}), hence profinite as a
topological space (Topology, Definition \ref{topology-definition-profinite}).
Let $W \subset G$ be a quasi-compact open containing the image of $T \to G$.
After replacing $W$ by the image of $G^0 \times W \to G \times G \to G$ we may
assume that $W$ is invariant under the action of left translation by $G^0$, see
Lemma \ref{lemma-group-scheme-over-field-translate-open}.
Consider the composition
$$
\psi = \pi \circ f : T \xrightarrow{f} W \xrightarrow{\pi} \pi_0(W)
$$
The space $\pi_0(W)$ is profinite
(Topology, Lemma \ref{topology-lemma-spectral-pi0} and
Properties, Lemma
\ref{properties-lemma-quasi-compact-quasi-separated-spectral}).
Let $F_\xi \subset T$ be the fibre of $T \to \pi_0(W)$ over $\xi \in \pi_0(W)$.
Assume that for all $\xi$ we can find an affine open $U_\xi \subset W$ with
$F \subset U$. Since $\psi : T \to \pi_0(W)$ is proper as a map of
topological spaces (Topology, Lemma \ref{topology-lemma-closed-map}),
we can find a quasi-compact open $V_\xi \subset \pi_0(W)$ such that
$\psi^{-1}(V_\xi) \subset f^{-1}(U_\xi)$ (easy topological argument omitted).
After replacing $U_\xi$ by $U_\xi \cap \pi^{-1}(V_\xi)$, which is open and
closed in $U_\xi$ hence affine, we see that $U_\xi \subset \pi^{-1}(V_\xi)$
and $U_\xi \cap T = \psi^{-1}(V_\xi)$.
By Topology, Lemma \ref{topology-lemma-profinite-refine-open-covering}
we can find a finite disjoint union decomposition
$\pi_0(W) = \bigcup_{i = 1, \ldots, n} V_i$ by quasi-compact opens such that
$V_i \subset V_{\xi_i}$ for some $i$. Then we see that
$$
f(T) \subset \bigcup\nolimits_{i = 1, \ldots, n} U_{\xi_i} \cap \pi^{-1}(V_i)
$$
the right hand side of which is a finite disjoint union of affines, therefore
affine.

\medskip\noindent
Let $Z$ be a connected component of $G$ which meets $f(T)$. Then $Z$
has a $k$-rational point $z$ (because all residue fields of the scheme $T$
are isomorphic to $k$). Hence $Z = G^0 z$. By our choice of $W$, we see
that $Z \subset W$. The argument in the preceding paragraph reduces us to
the problem of finding an affine open neighbourhood of $f(T) \cap Z$ in $W$.
After translation by a rational point we may assume that $Z = G^0$
(details omitted). Observe that the scheme theoretic inverse image
$T' = f^{-1}(G^0) \subset T$ is a closed subscheme, which has the same type.
After replacing $T$ by $T'$ we may assume that $f(T) \subset G^0$.
Choose an affine open neighbourhood $U \subset G$
of $e \in G$, so that in particular $U \cap G^0$ is nonempty. We will show
there exists a $g \in G^0(k)$ such that $f(T) \subset g^{-1}U$.
This will finish the proof as $g^{-1}U \subset W$ by the left
$G^0$-invariance of $W$.

\medskip\noindent
The arguments in the preceding two paragraphs allow us to pass to $G^0$
and reduce the problem to the following:
Assume $G$ is irreducible and $U \subset G$ an affine
open neighbourhood of $e$. Show that $f(T) \subset g^{-1}U$
for some $g \in G(k)$. Consider the morphism
$$
U \times_k T \longrightarrow G \times_k T,\quad
(t, u) \longrightarrow (uf(t)^{-1}, t)
$$
which is an open immersion (because the extension of this morphism to
$G \times_k T \to G \times_k T$ is an isomorphism).
By our assumption on $T$ we see that we have $|U \times_k T| = |U| \times |T|$
and similarly for $G \times_k T$, see
Lemma \ref{lemma-profinite-product-over-field}.
Hence the image of the displayed open immersion is a finite union
of boxes $\bigcup_{i = 1, \ldots, n} U_i \times V_i$ with
$V_i \subset T$ and $U_i \subset G$ quasi-compact open. This means that
the possible opens $Uf(t)^{-1}$, $t \in T$ are finite in number, say
$Uf(t_1)^{-1}, \ldots, Uf(t_r)^{-1}$. Since $G$ is irreducible the
intersection
$$
Uf(t_1)^{-1} \cap \ldots \cap Uf(t_r)^{-1}
$$
is nonempty and since $G$ is Jacobson with closed points $k$-rational,
we can choose a $k$-valued point $g \in G(k)$ of this intersection.
Then we see that $g \in Uf(t)^{-1}$ for all $t \in T$ which
means that $f(t) \in g^{-1}U$ as desired.
\end{proof}

\begin{remark}
\label{remark-easy}
If $G$ is a group scheme over a field, is there always a quasi-compact
open and closed subgroup scheme? By
Proposition \ref{proposition-connected-component}
this question is only interesting if $G$ has infinitely many connected
components (geometrically).
\end{remark}

\begin{lemma}
\label{lemma-group-scheme-field-countable-affine}
Let $G$ be a group scheme over a field.
There exists an open and closed subscheme $G' \subset G$
which is a countable union of affines.
\end{lemma}

\begin{proof}
Let $e \in U(k)$ be a quasi-compact open neighbourhood of the identity
element. By replacing $U$ by $U \cap i(U)$ we may assume that
$U$ is invariant under the inverse map. As $G$ is separated this is
still a quasi-compact set. Set
$$
G' = \bigcup\nolimits_{n \geq 1} m_n(U \times_k \ldots \times_k U)
$$
where $m_n : G \times_k \ldots \times_k G \to G$ is the $n$-slot
multiplication map
$(g_1, \ldots, g_n) \mapsto m(m(\ldots (m(g_1, g_2), g_3), \ldots ), g_n)$.
Each of these maps are open (see
Lemma \ref{lemma-group-scheme-over-field-open-multiplication})
hence $G'$ is an open subgroup scheme. By
Lemma \ref{lemma-open-subgroup-closed-over-field}
it is also a closed subgroup scheme.
\end{proof}






\section{Properties of algebraic group schemes}
\label{section-algebraic-group-schemes}

\noindent
Recall that a scheme over a field $k$ is (locally) algebraic if it is
(locally) of finite type over $\Spec(k)$, see
Varieties, Definition \ref{varieties-definition-algebraic-scheme}.
This is the sense of algebraic we are using in the title of this section.

\begin{lemma}
\label{lemma-group-scheme-finite-type-field}
Let $k$ be a field. Let $G$ be a locally algebraic group scheme over $k$.
Then $G$ is equidimensional and $\dim(G) = \dim_g(G)$ for all $g \in G$.
For any closed point $g \in G$ we have $\dim(G) = \dim(\mathcal{O}_{G, g})$.
\end{lemma}

\begin{proof}
Let us first prove that $\dim_g(G) = \dim_{g'}(G)$ for any
pair of points $g, g' \in G$. By
Morphisms, Lemma \ref{morphisms-lemma-dimension-fibre-after-base-change}
we may extend the ground field at will. Hence we may assume that
both $g$ and $g'$ are defined over $k$. Hence there exists an
automorphism of $G$ mapping $g$ to $g'$, whence the equality.
By
Morphisms, Lemma \ref{morphisms-lemma-dimension-fibre-at-a-point}
we have
$\dim_g(G) = \dim(\mathcal{O}_{G, g}) +
\text{trdeg}_k(\kappa(g))$.
On the other hand, the dimension of $G$ (or any open subset of $G$)
is the supremum of the dimensions of the local rings of $G$, see
Properties, Lemma \ref{properties-lemma-codimension-local-ring}.
Clearly this is maximal for closed points $g$ in which case
$\text{trdeg}_k(\kappa(g)) = 0$ (by the Hilbert Nullstellensatz, see
Morphisms, Section \ref{morphisms-section-points-finite-type}).
Hence the lemma follows.
\end{proof}

\noindent
The following result is sometimes referred to as Cartier's theorem.

\begin{lemma}
\label{lemma-group-scheme-characteristic-zero-smooth}
Let $k$ be a field of characteristic $0$. Let $G$ be a
locally algebraic group scheme over $k$. Then the structure
morphism $G \to \Spec(k)$ is smooth, i.e., $G$ is a smooth
group scheme.
\end{lemma}

\begin{proof}
By
Lemma \ref{lemma-group-scheme-module-differentials}
the module of differentials of $G$ over $k$ is free.
Hence smoothness follows from
Varieties, Lemma \ref{varieties-lemma-char-zero-differentials-free-smooth}.
\end{proof}

\begin{remark}
\label{remark-when-reduced}
Any group scheme over a field of characteristic $0$ is reduced, see
\cite[I, Theorem 1.1 and I, Corollary 3.9, and II, Theorem 2.4]{Perrin-thesis}
and also
\cite[Proposition 4.2.8]{Perrin}.
This was a question raised in
\cite[page 80]{Oort}.
We have seen in
Lemma \ref{lemma-group-scheme-characteristic-zero-smooth}
that this holds when the group scheme is locally of finite type.
\end{remark}

\begin{lemma}
\label{lemma-reduced-group-scheme-prefect-field-characteristic-p-smooth}
Let $k$ be a perfect field of characteristic $p > 0$ (see
Lemma \ref{lemma-group-scheme-characteristic-zero-smooth}
for the characteristic zero case).
Let $G$ be a locally algebraic group scheme over $k$.
If $G$ is reduced then the structure
morphism $G \to \Spec(k)$ is smooth, i.e., $G$ is a smooth
group scheme.
\end{lemma}

\begin{proof}
By
Lemma \ref{lemma-group-scheme-module-differentials}
the sheaf $\Omega_{G/k}$ is free. Hence the lemma follows from
Varieties, Lemma \ref{varieties-lemma-char-p-differentials-free-smooth}.
\end{proof}

\begin{remark}
\label{remark-reduced-smooth-not-true-general}
Let $k$ be a field of characteristic $p > 0$.
Let $\alpha \in k$ be an element which is not a $p$th power.
The closed subgroup scheme
$$
G = V(x^p + \alpha y^p) \subset \mathbf{G}_{a, k}^2
$$
is reduced and irreducible but not smooth (not even normal).
\end{remark}

\noindent
The following lemma is a special case of
Lemma \ref{lemma-compact-set-in-affine}
with a somewhat easier proof.

\begin{lemma}
\label{lemma-points-in-affine}
Let $k$ be an algebraically closed field.
Let $G$ be a locally algebraic group scheme over $k$.
Let $g_1, \ldots, g_n \in G(k)$ be $k$-rational points.
Then there exists an affine open $U \subset G$ containing $g_1, \ldots, g_n$.
\end{lemma}

\begin{proof}
We first argue by induction on $n$ that we may assume all $g_i$ are
on the same connected component of $G$. Namely, if not, then we can
find a decomposition $G = W_1 \amalg W_2$ with $W_i$ open in $G$ and
(after possibly renumbering) $g_1, \ldots, g_r \in W_1$ and
$g_{r + 1}, \ldots, g_n \in W_2$ for some $0 < r < n$. By
induction we can find affine opens $U_1$ and $U_2$ of $G$ with
$g_1, \ldots, g_r \in U_1$ and $g_{r + 1}, \ldots, g_n \in U_2$.
Then
$$
g_1, \ldots, g_n \in (U_1 \cap W_1) \cup (U_2 \cap W_2)
$$
is a solution to the problem. Thus we may assume $g_1, \ldots, g_n$
are all on the same connected component of $G$. Translating by $g_1^{-1}$
we may assume $g_1, \ldots, g_n \in G^0$ where $G^0 \subset G$ is as in
Proposition \ref{proposition-connected-component}. Choose an affine
open neighbourhood $U$ of $e$, in particular $U \cap G^0$ is nonempty.
Since $G^0$ is irreducible we see that
$$
G^0 \cap (Ug_1^{-1} \cap \ldots \cap Ug_n^{-1})
$$
is nonempty. Since $G \to \Spec(k)$ is locally of finite type, also
$G^0 \to \Spec(k)$ is locally of finite type, hence any nonempty
open has a $k$-rational point. Thus we can pick $g \in G^0(k)$ with
$g \in Ug_i^{-1}$ for all $i$. Then $g_i \in g^{-1}U$ for all $i$
and $g^{-1}U$ is the affine open we were looking for.
\end{proof}

\begin{lemma}
\label{lemma-algebraic-quasi-projective}
Let $k$ be a field. Let $G$ be an algebraic group scheme over $k$.
Then $G$ is quasi-projective over $k$.
\end{lemma}

\begin{proof}
By Varieties, Lemma \ref{varieties-lemma-ample-after-field-extension}
we may assume that $k$ is algebraically closed. Let $G^0 \subset G$
be the connected component of $G$ as in
Proposition \ref{proposition-connected-component}.
Then every other connected component of $G$ has a $k$-rational
point and hence is isomorphic to $G^0$ as a scheme.
Since $G$ is quasi-compact and Noetherian, there are finitely many of these
connected components. Thus we reduce to the case discussed in
the next paragraph.

\medskip\noindent
Let $G$ be a connected algebraic group scheme over an algebraically closed
field $k$. If the characteristic of $k$ is zero, then $G$ is smooth over
$k$ by Lemma \ref{lemma-group-scheme-characteristic-zero-smooth}.
If the characteristic of $k$ is $p > 0$, then we let $H = G_{red}$
be the reduction of $G$. By
Divisors, Proposition \ref{divisors-proposition-push-down-ample}
it suffices to show that $H$ has an ample invertible sheaf.
(For an algebraic scheme over $k$ having an ample invertible
sheaf is equivalent to being quasi-projective over $k$, see
for example the very general
More on Morphisms, Lemma \ref{more-morphisms-lemma-quasi-projective}.)
By Lemma \ref{lemma-reduced-subgroup-scheme-perfect}
we see that $H$ is a group scheme over $k$.
By Lemma \ref{lemma-reduced-group-scheme-prefect-field-characteristic-p-smooth}
we see that $H$ is smooth over $k$.
This reduces us to the situation discussed in the next
paragraph.

\medskip\noindent
Let $G$ be a quasi-compact irreducible smooth group scheme over an
algebraically closed field $k$. Observe that the local rings of $G$
are regular and hence UFDs
(Varieties, Lemma \ref{varieties-lemma-smooth-regular} and
More on Algebra, Lemma \ref{more-algebra-lemma-regular-local-UFD}).
The complement of a nonempty affine open of $G$
is the support of an effective Cartier divisor $D$.
This follows from Divisors, Lemma
\ref{divisors-lemma-complement-open-affine-effective-cartier-divisor}.
(Observe that $G$ is separated by
Lemma \ref{lemma-group-scheme-over-field-separated}.)
We conclude there exists an effective Cartier divisor $D \subset G$
such that $G \setminus D$ is affine. We will use below that
for any $n \geq 1$ and $g_1, \ldots, g_n \in G(k)$ the complement
$G \setminus \bigcup D g_i$ is affine. Namely, it is the intersection
of the affine opens $G \setminus Dg_i \cong G \setminus D$
in the separated scheme $G$.

\medskip\noindent
We may choose the top row of the diagram
$$
\xymatrix{
G & U \ar[l]_j \ar[r]^\pi & \mathbf{A}^d_k \\
& W \ar[r]^{\pi'} \ar[u] & V \ar[u]
}
$$
such that $U \not = \emptyset$, $j : U \to G$ is an open immersion, and
$\pi$ is \'etale, see
Morphisms, Lemma \ref{morphisms-lemma-smooth-etale-over-affine-space}.
There is a nonempty affine open $V \subset \mathbf{A}^d_k$ such that
with $W = \pi^{-1}(V)$ the morphism $\pi' = \pi|_W : W \to V$ is finite \'etale.
In particular $\pi'$ is finite locally free, say of degree $n$.
Consider the effective Cartier divisor
$$
\mathcal{D} = \{(g, w) \mid m(g, j(w)) \in D\} \subset G \times W
$$
(This is the restriction to $G \times W$ of the pullback of $D \subset G$
under the flat morphism $m : G \times G \to G$.)
Consider the closed subset\footnote{Using the material
in Divisors, Section \ref{divisors-section-norms}
we could take as effective Cartier
divisor $E$ the norm of the effective Cartier divisor $\mathcal{D}$
along the finite locally free morphism $1 \times \pi'$ bypassing
some of the arguments.}
$T = (1 \times \pi')(\mathcal{D}) \subset G \times V$.
Since $\pi'$ is finite locally free, every irreducible component
of $T$ has codimension $1$ in $G \times V$. Since $G \times V$
is smooth over $k$ we conclude these components are effective Cartier
divisors (Divisors, Lemma \ref{divisors-lemma-weil-divisor-is-cartier-UFD}
and lemmas cited above)
and hence $T$ is the support of an effective Cartier divisor
$E$ in $G \times V$. If $v \in V(k)$, then
$(\pi')^{-1}(v) = \{w_1, \ldots, w_n\} \subset W(k)$ and we see that
$$
E_v = \bigcup\nolimits_{i = 1, \ldots, n} D j(w_i)^{-1}
$$
in $G$ set theoretically. In particular we see that $G \setminus E_v$
is affine open (see above).
Moreover, if $g \in G(k)$, then there exists a $v \in V$
such that $g \not \in E_v$. Namely, the set $W'$ of $w \in W$ such that
$g \not \in Dj(w)^{-1}$ is nonempty open and it suffices to pick $v$
such that the fibre of $W' \to V$ over $v$ has $n$ elements.

\medskip\noindent
Consider the invertible sheaf
$\mathcal{M} = \mathcal{O}_{G \times V}(E)$ on $G \times V$.
By Varieties, Lemma \ref{varieties-lemma-rational-equivalence-for-Pic}
the isomorphism class $\mathcal{L}$ of the restriction
$\mathcal{M}_v = \mathcal{O}_G(E_v)$ is independent of $v \in V(k)$.
On the other hand, for every $g \in G(k)$ we can find a $v$
such that $g \not \in E_v$ and such that $G \setminus E_v$
is affine. Thus the canonical section
(Divisors, Definition
\ref{divisors-definition-invertible-sheaf-effective-Cartier-divisor})
of $\mathcal{O}_G(E_v)$
corresponds to a section $s_v$ of $\mathcal{L}$ which does not
vanish at $g$ and such that $G_{s_v}$ is affine.
This means that $\mathcal{L}$ is ample by definition
(Properties, Definition \ref{properties-definition-ample}).
\end{proof}

\begin{lemma}
\label{lemma-algebraic-center}
Let $k$ be a field. Let $G$ be a locally algebraic group scheme over $k$.
Then the center of $G$ is a closed subgroup scheme of $G$.
\end{lemma}

\begin{proof}
Let $\text{Aut}(G)$ denote the contravariant functor on the category of
schemes over $k$ which associates to $S/k$ the set of automorphisms
of the base change $G_S$ as a group scheme over $S$. There is a natural
transformation
$$
G \longrightarrow \text{Aut}(G),\quad
g \longmapsto \text{inn}_g
$$
sending an $S$-valued point $g$ of $G$ to the inner automorphism of $G$
determined by $g$. The center $C$ of $G$ is by definition the kernel of
this transformation, i.e., the functor which to $S$ associates those
$g \in G(S)$ whose associated inner automorphism is trivial. The statement
of the lemma is that this functor is representable by a closed subgroup
scheme of $G$.

\medskip\noindent
Choose an integer $n \geq 1$. Let $G_n \subset G$ be the $n$th infinitesimal
neighbourhood of the identity element $e$ of $G$. For every scheme $S/k$
the base change $G_{n, S}$ is the $n$th infinitesimal neighbourhood of
$e_S : S \to G_S$. Thus we see that there is a natural transformation
$\text{Aut}(G) \to \text{Aut}(G_n)$ where the right hand side is the
functor of automorphisms of $G_n$ as a scheme ($G_n$ isn't in general
a group scheme). Observe that $G_n$ is the spectrum of an artinian
local ring $A_n$ with residue field $k$ which has finite dimension
as a $k$-vector space
(Varieties, Lemma \ref{varieties-lemma-algebraic-scheme-dim-0}).
Since every automorphism of $G_n$ induces in particular an invertible
linear map $A_n \to A_n$, we obtain transformations of functors
$$
G \to \text{Aut}(G) \to \text{Aut}(G_n) \to \text{GL}(A_n)
$$
The final group valued functor is representable, see
Example \ref{example-general-linear-group}, and the
last arrow is visibly injective.
Thus for every $n$ we obtain a closed subgroup scheme
$$
H_n = \Ker(G \to \text{Aut}(G_n)) = \Ker(G \to \text{GL}(A_n)).
$$
As a first approximation we set $H = \bigcap_{n \geq 1} H_n$
(scheme theoretic intersection). This is a closed subgroup scheme
which contains the center $C$.

\medskip\noindent
Let $h$ be an $S$-valued point of $H$ with $S$ locally Noetherian.
Then the automorphism $\text{inn}_h$ induces the identity on all
the closed subschemes $G_{n, S}$. Consider the kernel
$K = \Ker(\text{inn}_h : G_S \to G_S)$.
This is a closed subgroup scheme of $G_S$ over $S$ containing the
closed subschemes $G_{n, S}$ for $n \geq 1$.
This implies that $K$ contains an open neighbourhood of
$e(S) \subset G_S$, see
Algebra, Remark \ref{algebra-remark-intersection-powers-ideal}.
Let $G^0 \subset G$ be as in Proposition \ref{proposition-connected-component}.
Since $G^0$ is geometrically irreducible, we conclude that $K$ contains
$G^0_S$ (for any nonempty open $U \subset G^0_{k'}$ and any field extension
$k'/k$ we have $U \cdot U^{-1} = G^0_{k'}$, see proof of
Lemma \ref{lemma-irreducible-group-scheme-over-field-qc}).
Applying this with $S = H$ we find that $G^0$ and $H$
are subgroup schemes of $G$ whose points commute: for any scheme $S$
and any $S$-valued points $g \in G^0(S)$, $h \in H(S)$ we have
$gh = hg$ in $G(S)$.

\medskip\noindent
Assume that $k$ is algebraically closed. Then we can pick a $k$-valued
point $g_i$ in each irreducible component $G_i$ of $G$. Observe that in
this case the connected components of $G$ are the irreducible components
of $G$ are the translates of $G^0$ by our $g_i$. We claim that
$$
C = H \cap \bigcap\nolimits_i \Ker(\text{inn}_{g_i} : G \to G)
\quad (\text{scheme theoretic intersection})
$$
Namely, $C$ is contained in the right hand side. On the other hand, every
$S$-valued point $h$ of the right hand side commutes with $G^0$
and with $g_i$ hence with everything in $G = \bigcup G^0g_i$.

\medskip\noindent
The case of a general base field $k$ follows from the result for the
algebraic closure $\overline{k}$ by descent. Namely, let
$A \subset G_{\overline{k}}$ the closed subgroup scheme representing
the center of $G_{\overline{k}}$. Then we have
$$
A \times_{\Spec(k)} \Spec(\overline{k}) =
\Spec(\overline{k}) \times_{\Spec(k)} A
$$
as closed subschemes of $G_{\overline{k} \otimes_k \overline{k}}$ by
the functorial nature of the center. Hence we see that $A$ descends
to a closed subgroup scheme $Z \subset G$ by
Descent, Lemma \ref{descent-lemma-closed-immersion}
(and Descent, Lemma \ref{descent-lemma-descending-property-closed-immersion}).
Then $Z$ represents $C$ (small argument omitted) and the proof is complete.
\end{proof}







\section{Abelian varieties}
\label{section-abelian-varieties}

\noindent
An excellent reference for this material is Mumford's book on
abelian varieties, see \cite{AVar}. We encourage the reader to
look there. There are many equivalent definitions; here is one.

\begin{definition}
\label{definition-abelian-variety}
Let $k$ be a field. An {\it abelian variety} is a group scheme over
$k$ which is also a proper, geometrically integral variety over $k$.
\end{definition}

\noindent
We prove a few lemmas about this notion and then we collect
all the results together in
Proposition \ref{proposition-review-abelian-varieties}.

\begin{lemma}
\label{lemma-abelian-variety-projective}
Let $k$ be a field. Let $A$ be an abelian variety over $k$.
Then $A$ is projective.
\end{lemma}

\begin{proof}
This follows from
Lemma \ref{lemma-algebraic-quasi-projective} and
More on Morphisms, Lemma \ref{more-morphisms-lemma-projective}.
\end{proof}

\begin{lemma}
\label{lemma-abelian-variety-change-field}
Let $k$ be a field. Let $A$ be an abelian variety over $k$.
For any field extension $K/k$ the base change $A_K$ is an
abelian variety over $K$.
\end{lemma}

\begin{proof}
Omitted. Note that this is why we insisted on $A$ being
geometrically integral; without that condition this lemma
(and many others below) would be wrong.
\end{proof}

\begin{lemma}
\label{lemma-abelian-variety-smooth}
Let $k$ be a field. Let $A$ be an abelian variety over $k$.
Then $A$ is smooth over $k$.
\end{lemma}

\begin{proof}
If $k$ is perfect then this follows from
Lemma \ref{lemma-group-scheme-characteristic-zero-smooth}
(characteristic zero) and
Lemma \ref{lemma-reduced-group-scheme-prefect-field-characteristic-p-smooth}
(positive characteristic).
We can reduce the general case to this case by descent for smoothness
(Descent, Lemma \ref{descent-lemma-descending-property-smooth})
and going to the perfect closure using
Lemma \ref{lemma-abelian-variety-change-field}.
\end{proof}

\begin{lemma}
\label{lemma-abelian-variety-abelian}
An abelian variety is an abelian group scheme, i.e., the group
law is commutative.
\end{lemma}

\begin{proof}
Let $k$ be a field. Let $A$ be an abelian variety over $k$.
By Lemma \ref{lemma-abelian-variety-change-field} we may replace
$k$ by its algebraic closure. Consider the morphism
$$
h : A \times_k A \longrightarrow A \times_k A,\quad
(x, y) \longmapsto (x, xyx^{-1}y^{-1})
$$
This is a morphism over $A$ via the first projection on either side.
Let $e \in A(k)$ be the unit. Then we see that $h|_{e \times A}$ is
constant with value $(e, e)$. By More on Morphisms, Lemma
\ref{more-morphisms-lemma-flat-proper-family-cannot-collapse-fibre}
there exists an open neighbourhood $U \subset A$ of $e$
such that $h|_{U \times A}$ factors through some $Z \subset U \times A$
finite over $U$. This means that for $x \in U(k)$ the morphism
$A \to A$, $y \mapsto xyx^{-1}y^{-1}$ takes finitely many values.
Of course this means it is constant with value $e$. Thus
$(x, y) \mapsto xyx^{-1}y^{-1}$ is
constant with value $e$ on $U \times A$ which implies
that the group law on $A$ is abelian.
\end{proof}

\begin{lemma}
\label{lemma-apply-cube}
Let $k$ be a field. Let $A$ be an abelian variety over $k$.
Let $\mathcal{L}$ be an invertible $\mathcal{O}_A$-module.
Then there is an isomorphism
$$
m_{1, 2, 3}^*\mathcal{L} \otimes
m_1^*\mathcal{L} \otimes
m_2^*\mathcal{L} \otimes
m_3^*\mathcal{L} \cong
m_{1, 2}^*\mathcal{L} \otimes
m_{1, 3}^*\mathcal{L} \otimes
m_{2, 3}^*\mathcal{L}
$$
of invertible modules on $A \times_k A \times_k A$
where $m_{i_1, \ldots, i_t} : A \times_k A \times_k A \to A$
is the morphism $(x_1, x_2, x_3) \mapsto \sum x_{i_j}$.
\end{lemma}

\begin{proof}
Apply the theorem of the cube
(More on Morphisms, Theorem \ref{more-morphisms-theorem-of-the-cube})
to the difference
$$
\mathcal{M} =
m_{1, 2, 3}^*\mathcal{L} \otimes
m_1^*\mathcal{L} \otimes
m_2^*\mathcal{L} \otimes
m_3^*\mathcal{L} \otimes
m_{1, 2}^*\mathcal{L}^{\otimes -1} \otimes
m_{1, 3}^*\mathcal{L}^{\otimes -1} \otimes
m_{2, 3}^*\mathcal{L}^{\otimes -1}
$$
This works because the restriction of $\mathcal{M}$
to $A \times A \times e = A \times A$ is equal to
$$
n_{1, 2}^*\mathcal{L} \otimes
n_1^*\mathcal{L} \otimes
n_2^*\mathcal{L} \otimes
n_{1, 2}^*\mathcal{L}^{\otimes -1} \otimes
n_1^*\mathcal{L}^{\otimes -1} \otimes
n_2^*\mathcal{L}^{\otimes -1} \cong \mathcal{O}_{A \times_k A}
$$
where $n_{i_1, \ldots, i_t} : A \times_k A \to A$
is the morphism $(x_1, x_2) \mapsto \sum x_{i_j}$.
Similarly for $A \times e \times A$ and $e \times A \times A$.
\end{proof}

\begin{lemma}
\label{lemma-pullbacks-by-n}
Let $k$ be a field. Let $A$ be an abelian variety over $k$.
Let $\mathcal{L}$ be an invertible $\mathcal{O}_A$-module.
Then
$$
[n]^*\mathcal{L} \cong
\mathcal{L}^{\otimes n(n + 1)/2} \otimes
([-1]^*\mathcal{L})^{\otimes n(n - 1)/2}
$$
where $[n] : A \to A$ sends $x$ to $x + x + \ldots + x$ with $n$ summands
and where $[-1] : A \to A$ is the inverse of $A$.
\end{lemma}

\begin{proof}
Consider the morphism $A \to A \times_k A \times_k A$,
$x \mapsto (x, x, -x)$ where $-x = [-1](x)$. Pulling back
the relation of Lemma \ref{lemma-apply-cube} we obtain
$$
\mathcal{L} \otimes
\mathcal{L} \otimes
\mathcal{L} \otimes
[-1]^*\mathcal{L} \cong
[2]^*\mathcal{L}
$$
which proves the result for $n = 2$. By induction assume the result holds
for $1, 2, \ldots, n$. Then consider the morphism
$A \to A \times_k A \times_k A$, $x \mapsto (x, x, [n - 1]x)$.
Pulling back
the relation of Lemma \ref{lemma-apply-cube} we obtain
$$
[n + 1]^*\mathcal{L} \otimes
\mathcal{L} \otimes
\mathcal{L} \otimes
[n - 1]^*\mathcal{L} \cong
[2]^*\mathcal{L} \otimes
[n]^*\mathcal{L} \otimes
[n]^*\mathcal{L}
$$
and the result follows by elementary arithmetic.
\end{proof}

\begin{lemma}
\label{lemma-degree-multiplication-by-d}
Let $k$ be a field. Let $A$ be an abelian variety over $k$.
Let $[d] : A \to A$ be the multiplication by $d$.
Then $[d]$ is finite locally free of degree $d^{2\dim(A)}$.
\end{lemma}

\begin{proof}
By Lemma \ref{lemma-abelian-variety-projective}
(and More on Morphisms, Lemma \ref{more-morphisms-lemma-projective})
we see that $A$ has an ample invertible module $\mathcal{L}$.
Since $[-1] : A \to A$ is an automorphism, we see that
$[-1]^*\mathcal{L}$ is an ample invertible $\mathcal{O}_X$-module
as well. Thus $\mathcal{N} = \mathcal{L} \otimes [-1]^*\mathcal{L}$
is ample, see
Properties, Lemma \ref{properties-lemma-ample-tensor-globally-generated}.
Since $\mathcal{N} \cong [-1]^*\mathcal{N}$ we see that
$[d]^*\mathcal{N} \cong \mathcal{N}^{\otimes n^2}$ by
Lemma \ref{lemma-pullbacks-by-n}.

\medskip\noindent
To get a contradiction $C \subset X$ be a proper curve contained in a
fibre of $[d]$. Then $\mathcal{N}^{\otimes d^2}|_C \cong \mathcal{O}_C$
is an ample invertible $\mathcal{O}_C$-module of degree $0$ which
contradicts Varieties, Lemma \ref{varieties-lemma-ample-curve} for example.
(You can also use Varieties, Lemma \ref{varieties-lemma-ample-positive}.)
Thus every fibre of $[d]$ has dimension $0$ and hence $[d]$ is finite
for example by Cohomology of Schemes, Lemma
\ref{coherent-lemma-characterize-finite}.
Moreover, since $A$ is smooth over $k$ by
Lemma \ref{lemma-abelian-variety-smooth}
we see that $[d] : A \to A$ is flat by
Algebra, Lemma \ref{algebra-lemma-CM-over-regular-flat}
(we also use that schemes smooth over fields are regular and that
regular rings are Cohen-Macaulay, see
Varieties, Lemma \ref{varieties-lemma-smooth-regular} and
Algebra, Lemma \ref{algebra-lemma-regular-ring-CM}).
Thus $[d]$ is finite flat hence finite locally free by
Morphisms, Lemma \ref{morphisms-lemma-finite-flat}.

\medskip\noindent
Finally, we come to the formula for the degree. By
Varieties, Lemma \ref{varieties-lemma-degree-finite-morphism-in-terms-degrees}
we see that
$$
\deg_{\mathcal{N}^{\otimes d^2}}(A) = \deg([d]) \deg_\mathcal{N}(A)
$$
Since the degree of $A$ with respect to
$\mathcal{N}^{\otimes d^2}$, respectively $\mathcal{N}$
is the coefficient of $n^{\dim(A)}$ in the polynomial
$$
n \longmapsto \chi(A, \mathcal{N}^{\otimes nd^2}),\quad
\text{respectively}\quad n \longmapsto \chi(A, \mathcal{N}^{\otimes n})
$$
we see that $\deg([d]) = d^{2 \dim(A)}$.
\end{proof}

\begin{lemma}
\label{lemma-abelian-variety-multiplication-by-d-etale}
\begin{slogan}
Multiplication by an integer on an abelian variety is an etale morphism
if and only if the integer is invertible in the base field.
\end{slogan}
Let $k$ be a field. Let $A$ be an abelian variety over $k$.
Then $[d] : A \to A$ is \'etale if and only if $d$ is invertible in $k$.
\end{lemma}

\begin{proof}
Observe that $[d](x + y) = [d](x) + [d](y)$. Since translation by a
point is an automorphism of $A$, we see that the set of points where
$[d] : A \to A$ is \'etale is either empty or equal to $A$ (some details
omitted). Thus it suffices to check whether $[d]$ is \'etale at
the unit $e \in A(k)$. Since we know that $[d]$ is finite locally free
(Lemma \ref{lemma-degree-multiplication-by-d})
to see that it is \'etale at $e$ is equivalent to
proving that $\text{d}[d] : T_{A/k, e} \to T_{A/k, e}$ is injective. See
Varieties, Lemma \ref{varieties-lemma-injective-tangent-spaces-unramified} and
Morphisms, Lemma \ref{morphisms-lemma-flat-unramified-etale}.
By Lemma \ref{lemma-group-scheme-addition-tangent-vectors} we see that
$\text{d}[d]$ is given by multiplication by $d$ on $T_{A/k, e}$.
\end{proof}

\begin{lemma}
\label{lemma-abelian-variety-multiplication-by-p}
Let $k$ be a field of characteristic $p > 0$. Let $A$ be an abelian variety
over $k$. The fibre of $[p] : A \to A$ over $0$ has at most
$p^g$ distinct points.
\end{lemma}

\begin{proof}
To prove this, we may and do replace $k$ by the algebraic closure.
By Lemma \ref{lemma-group-scheme-addition-tangent-vectors}
the derivative of $[p]$ is multiplication by $p$ as a map
$T_{A/k, e} \to T_{A/k, e}$ and hence is zero (compare
with proof of Lemma \ref{lemma-abelian-variety-multiplication-by-d-etale}).
Since $[p]$ commutes with translation we conclude that the derivative of $[p]$
is everywhere zero, i.e., that the induced map
$[p]^*\Omega_{A/k} \to \Omega_{A/k}$ is zero.
Looking at generic points, we find that
the corresponding map $[p]^* : k(A) \to k(A)$
of function fields induces the zero map on $\Omega_{k(A)/k}$.
Let $t_1, \ldots, t_g$ be a p-basis of $k(A)$ over $k$
(More on Algebra, Definition \ref{more-algebra-definition-p-basis} and
Lemma \ref{more-algebra-lemma-p-basis}). Then $[p]^*(t_i)$
has a $p$th root by
Algebra, Lemma \ref{algebra-lemma-derivative-zero-pth-power}.
We conclude that
$k(A)[x_1, \ldots, x_g]/(x_1^p - t_1, \ldots, x_g^p - t_g)$ is a subextension
of $[p]^* : k(A) \to k(A)$.
Thus we can find an affine open $U \subset A$ such that
$t_i \in \mathcal{O}_A(U)$ and $x_i \in \mathcal{O}_A([p]^{-1}(U))$.
We obtain a factorization
$$
[p]^{-1}(U)
\xrightarrow{\pi_1}
\Spec(\mathcal{O}(U)[x_1, \ldots, x_g]/(x_1^p - t_1, \ldots, x_g^p - t_g))
\xrightarrow{\pi_2}
U
$$
of $[p]$ over $U$. After shrinking $U$ we may assume that $\pi_1$
is finite locally free (for example by generic flatness -- actually it is
already finite locally free in our case).
By Lemma \ref{lemma-degree-multiplication-by-d} we see that
$[p]$ has degree $p^{2g}$. Since $\pi_2$
has degree $p^g$ we see that $\pi_1$ has degree $p^g$ as well.
The morphism $\pi_2$ is a universal homeomorphism hence the fibres are
singletons. We conclude that the (set theoretic) fibres of $[p]^{-1}(U) \to U$
are the fibres of $\pi_1$. Hence they
have at most $p^g$ elements. Since $[p]$ is a homomorphism of group
schemes over $k$, the fibre of $[p] : A(k) \to A(k)$ has the
same cardinality for every $a \in A(k)$ and the proof is complete.
\end{proof}

\begin{proposition}
\label{proposition-review-abelian-varieties}
\begin{reference}
Wonderfully explained in \cite{AVar}.
\end{reference}
Let $A$ be an abelian variety over a field $k$. Then
\begin{enumerate}
\item $A$ is projective over $k$,
\item $A$ is a commutative group scheme,
\item the morphism $[n] : A \to A$ is surjective for all $n \geq 1$,
\item if $k$ is algebraically closed, then $A(k)$ is a divisible abelian group,
\item $A[n] = \Ker([n] : A \to A)$ is a finite group scheme of degree
$n^{2\dim A}$ over $k$,
\item $A[n]$ is \'etale over $k$ if and only if $n \in k^*$,
\item if $n \in k^*$ and $k$ is algebraically closed,
then $A(k)[n] \cong (\mathbf{Z}/n\mathbf{Z})^{\oplus 2\dim(A)}$,
\item if $k$ is algebraically closed of characteristic $p > 0$, then
there exists an integer $0 \leq f \leq \dim(A)$ such that
$A(k)[p^m] \cong (\mathbf{Z}/p^m\mathbf{Z})^{\oplus f}$
for all $m \geq 1$.
\end{enumerate}
\end{proposition}

\begin{proof}
Part (1) follows from Lemma \ref{lemma-abelian-variety-projective}.
Part (2) follows from Lemma \ref{lemma-abelian-variety-abelian}.
Part (3) follows from Lemma \ref{lemma-degree-multiplication-by-d}.
If $k$ is algebraically closed then surjective morphisms of varieties
over $k$ induce surjective maps on $k$-rational points, hence
(4) follows from (3).
Part (5) follows from Lemma \ref{lemma-degree-multiplication-by-d}
and the fact that a base change of a finite locally free morphism
of degree $N$ is a finite locally free morphism of degree $N$.
Part (6) follows from 
Lemma \ref{lemma-abelian-variety-multiplication-by-d-etale}.
Namely, if $n$ is invertible in $k$, then $[n]$ is \'etale
and hence $A[n]$ is \'etale over $k$.
On the other hand, if $n$ is not invertible in $k$, then
$[n]$ is not \'etale at $e$ and it follows that $A[n]$
is not \'etale over $k$ at $e$ (use
Morphisms, Lemmas \ref{morphisms-lemma-flat-unramified-etale} and
\ref{morphisms-lemma-set-points-where-fibres-unramified}).

\medskip\noindent
Assume $k$ is algebraically closed. Set $g = \dim(A)$. Proof of (7).
Let $\ell$ be a prime number which is invertible in $k$. Then we see that
$$
A[\ell](k) = A(k)[\ell]
$$
is a finite abelian group, annihilated by $\ell$, of order $\ell^{2g}$.
It follows that it is isomorphic to $(\mathbf{Z}/\ell\mathbf{Z})^{2g}$
by the structure theory for finite abelian groups. Next, we consider
the short exact sequence
$$
0 \to A(k)[\ell] \to A(k)[\ell^2] \xrightarrow{\ell} A(k)[\ell] \to 0
$$
Arguing similarly as above we conclude that 
$A(k)[\ell^2] \cong (\mathbf{Z}/\ell^2\mathbf{Z})^{2g}$.
By induction on the exponent we find that
$A(k)[\ell^m] \cong (\mathbf{Z}/\ell^m\mathbf{Z})^{2g}$.
For composite integers $n$ prime to the characteristic of $k$
we take primary parts and we find the correct shape of the
$n$-torsion in $A(k)$.
The proof of (8) proceeds in exactly the same way, using that
Lemma \ref{lemma-abelian-variety-multiplication-by-p} gives
$A(k)[p] \cong (\mathbf{Z}/p\mathbf{Z})^{\oplus f}$ for some $0 \leq f \leq g$.
\end{proof}







\section{Actions of group schemes}
\label{section-action-group-scheme}

\noindent
Let $(G, m)$ be a group and let $V$ be a set.
Recall that a {\it (left) action} of $G$ on $V$ is given
by a map $a : G \times V \to V$ such that
\begin{enumerate}
\item (associativity) $a(m(g, g'), v) = a(g, a(g', v))$ for all
$g, g' \in G$ and $v \in V$, and
\item (identity) $a(e, v) = v$ for all $v \in V$.
\end{enumerate}
We also say that $V$ is a {\it $G$-set} (this usually means we
drop the $a$ from the notation -- which is abuse of notation).
A {\it map of $G$-sets} $\psi : V \to V'$ is any set map
such that $\psi(a(g, v)) = a(g, \psi(v))$ for all $v \in V$.

\begin{definition}
\label{definition-action-group-scheme}
Let $S$ be a scheme. Let $(G, m)$ be a group scheme over $S$.
\begin{enumerate}
\item An {\it action of $G$ on the scheme $X/S$} is
a morphism $a : G \times_S X \to X$ over $S$ such that
for every $T/S$ the map $a : G(T) \times X(T) \to X(T)$
defines the structure of a $G(T)$-set on $X(T)$.
\item Suppose that $X$, $Y$ are schemes over $S$ each endowed
with an action of $G$. An {\it equivariant} or more precisely
a {\it $G$-equivariant} morphism $\psi : X \to Y$
is a morphism of schemes over $S$ such
that for every $T/S$ the map $\psi : X(T) \to Y(T)$ is
a morphism of $G(T)$-sets.
\end{enumerate}
\end{definition}

\noindent
In situation (1) this means that the diagrams
\begin{equation}
\label{equation-action}
\xymatrix{
G \times_S G \times_S X \ar[r]_-{1_G \times a} \ar[d]_{m \times 1_X} &
G \times_S X \ar[d]^a \\
G \times_S X \ar[r]^a & X
}
\quad
\xymatrix{
G \times_S X \ar[r]_-a & X \\
X\ar[u]^{e \times 1_X} \ar[ru]_{1_X}
}
\end{equation}
are commutative. In situation (2) this just means that the diagram
$$
\xymatrix{
G \times_S X \ar[r]_-{\text{id} \times f} \ar[d]_a &
G \times_S Y \ar[d]^a \\
X \ar[r]^f & Y
}
$$
commutes.

\begin{definition}
\label{definition-free-action}
Let $S$, $G \to S$, and $X \to S$ as in
Definition \ref{definition-action-group-scheme}.
Let $a : G \times_S X \to X$ be an action of $G$ on $X/S$.
We say the action is {\it free} if for every scheme $T$ over $S$
the action $a : G(T) \times X(T) \to X(T)$ is a free action of
the group $G(T)$ on the set $X(T)$.
\end{definition}

\begin{lemma}
\label{lemma-free-action}
Situation as in Definition \ref{definition-free-action},
The action $a$ is free if and only if
$$
G \times_S X \to X \times_S X, \quad (g, x) \mapsto (a(g, x), x)
$$
is a monomorphism.
\end{lemma}

\begin{proof}
Immediate from the definitions.
\end{proof}






\section{Principal homogeneous spaces}
\label{section-principal-homogeneous}

\noindent
In
Cohomology on Sites, Definition \ref{sites-cohomology-definition-torsor}
we have defined a torsor for a sheaf of groups on a site.
Suppose $\tau \in \{Zariski, \etale, smooth, syntomic, fppf\}$ is a
topology and $(G, m)$ is a group scheme over $S$. Since $\tau$ is stronger than
the canonical topology (see
Descent, Lemma \ref{descent-lemma-fpqc-universal-effective-epimorphisms})
we see that $\underline{G}$ (see
Sites, Definition \ref{sites-definition-representable-sheaf})
is a sheaf of groups on $(\Sch/S)_\tau$.
Hence we already know what it means to have a
torsor for $\underline{G}$ on $(\Sch/S)_\tau$. A special situation
arises if this sheaf is representable. In the following definitions
we define directly what it means for the representing scheme to be a
$G$-torsor.

\begin{definition}
\label{definition-pseudo-torsor}
Let $S$ be a scheme.
Let $(G, m)$ be a group scheme over $S$.
Let $X$ be a scheme over $S$, and let
$a : G \times_S X \to X$ be an action of $G$ on $X$.
\begin{enumerate}
\item We say $X$ is a {\it pseudo $G$-torsor} or that $X$ is
{\it formally principally homogeneous under $G$} if the induced
morphism of schemes $G \times_S X \to X \times_S X$,
$(g, x) \mapsto (a(g, x), x)$ is an isomorphism of schemes over $S$.
\item A pseudo $G$-torsor $X$ is called {\it trivial} if there exists
an $G$-equivariant isomorphism $G \to X$ over $S$ where $G$ acts on
$G$ by left multiplication.
\end{enumerate}
\end{definition}

\noindent
It is clear that if $S' \to S$ is a morphism of schemes then
the pullback $X_{S'}$ of a pseudo $G$-torsor over $S$ is a
pseudo $G_{S'}$-torsor over $S'$.

\begin{lemma}
\label{lemma-characterize-trivial-pseudo-torsors}
In the situation of
Definition \ref{definition-pseudo-torsor}.
\begin{enumerate}
\item The scheme $X$ is a pseudo $G$-torsor if and only if for every scheme
$T$ over $S$ the set $X(T)$ is either empty or the action of the group $G(T)$
on $X(T)$ is simply transitive.
\item A pseudo $G$-torsor $X$ is trivial if and only if the morphism
$X \to S$ has a section.
\end{enumerate}
\end{lemma}

\begin{proof}
Omitted.
\end{proof}

\begin{definition}
\label{definition-principal-homogeneous-space}
Let $S$ be a scheme.
Let $(G, m)$ be a group scheme over $S$.
Let $X$ be a pseudo $G$-torsor over $S$.
\begin{enumerate}
\item We say $X$ is a {\it principal homogeneous space}
or a {\it $G$-torsor} if there exists a fpqc covering\footnote{This means
that the default type of torsor is a pseudo torsor which is trivial on an
fpqc covering. This is the definition in \cite[Expos\'e IV, 6.5]{SGA3}.
It is a little bit inconvenient for us as we most often work in the fppf
topology.}
$\{S_i \to S\}_{i \in I}$ such that each
$X_{S_i} \to S_i$ has a section (i.e., is a trivial pseudo $G_{S_i}$-torsor).
\item Let $\tau \in \{Zariski, \etale, smooth, syntomic, fppf\}$.
We say $X$ is a {\it $G$-torsor in the $\tau$ topology}, or a
{\it $\tau$ $G$-torsor}, or simply a {\it $\tau$ torsor}
if there exists a $\tau$ covering $\{S_i \to S\}_{i \in I}$
such that each $X_{S_i} \to S_i$ has a section.
\item If $X$ is a $G$-torsor, then we say that it is
{\it quasi-isotrivial} if it is a torsor for the \'etale topology.
\item If $X$ is a $G$-torsor, then we say that it is
{\it locally trivial} if it is a torsor for the Zariski topology.
\end{enumerate}
\end{definition}

\noindent
We sometimes say ``let $X$ be a $G$-torsor over $S$'' to indicate that
$X$ is a scheme over $S$ equipped with an action of $G$ which turns it
into a principal homogeneous space over $S$.
Next we show that this agrees with the notation introduced earlier
when both apply.

\begin{lemma}
\label{lemma-torsor}
Let $S$ be a scheme.
Let $(G, m)$ be a group scheme over $S$.
Let $X$ be a scheme over $S$, and let
$a : G \times_S X \to X$ be an action of $G$ on $X$.
Let $\tau \in \{Zariski, \etale, smooth, syntomic, fppf\}$.
Then $X$ is a $G$-torsor in the $\tau$-topology if and only if
$\underline{X}$ is a $\underline{G}$-torsor on $(\Sch/S)_\tau$.
\end{lemma}

\begin{proof}
Omitted.
\end{proof}

\begin{remark}
\label{remark-fun-with-torsors}
Let $(G, m)$ be a group scheme over the scheme $S$.
In this situation we have the following natural types of questions:
\begin{enumerate}
\item If $X \to S$ is a pseudo $G$-torsor and $X \to S$ is surjective,
then is $X$ necessarily a $G$-torsor?
\item Is every $\underline{G}$-torsor on $(\Sch/S)_{fppf}$
representable? In other words, does every $\underline{G}$-torsor
come from a fppf $G$-torsor?
\item Is every $G$-torsor an
fppf (resp.\ smooth, resp.\ \'etale, resp.\ Zariski) torsor?
\end{enumerate}
In general the answers to these questions is no. To get a positive answer
we need to impose additional conditions on $G \to S$.
For example:
If $S$ is the spectrum of a field, then the answer to (1) is yes
because then $\{X \to S\}$ is a fpqc covering trivializing $X$.
If $G \to S$ is affine, then the answer to (2) is yes
(insert future reference here).
If $G = \text{GL}_{n, S}$ then the answer to (3) is yes
and in fact any $\text{GL}_{n, S}$-torsor is locally trivial
(insert future reference here).
\end{remark}



\section{Equivariant quasi-coherent sheaves}
\label{section-equivariant}

\noindent
We think of ``functions'' as dual to ``space''. Thus for a morphism of spaces
the map on functions goes the other way. Moreover, we think of the
sections of a sheaf of modules as ``functions''. This leads us naturally
to the direction of the arrows chosen in the following definition.

\begin{definition}
\label{definition-equivariant-module}
Let $S$ be a scheme, let $(G, m)$ be a group scheme over $S$, and
let $a : G \times_S X \to X$ be an action of the group scheme $G$
on $X/S$. An {\it $G$-equivariant quasi-coherent $\mathcal{O}_X$-module},
or simply a {\it equivariant quasi-coherent $\mathcal{O}_X$-module},
is a pair $(\mathcal{F}, \alpha)$, where $\mathcal{F}$ is a quasi-coherent
$\mathcal{O}_X$-module, and $\alpha$ is a $\mathcal{O}_{G \times_S X}$-module
map
$$
\alpha : a^*\mathcal{F} \longrightarrow \text{pr}_1^*\mathcal{F}
$$
where $\text{pr}_1 : G \times_S X \to X$ is the projection
such that
\begin{enumerate}
\item the diagram
$$
\xymatrix{
(1_G \times a)^*\text{pr}_1^*\mathcal{F} \ar[r]_-{\text{pr}_{12}^*\alpha} &
\text{pr}_2^*\mathcal{F} \\
(1_G \times a)^*a^*\mathcal{F} \ar[u]^{(1_G \times a)^*\alpha} \ar@{=}[r] &
(m \times 1_X)^*a^*\mathcal{F} \ar[u]_{(m \times 1_X)^*\alpha}
}
$$
is a commutative in the category of
$\mathcal{O}_{G \times_S G \times_S X}$-modules, and
\item the pullback
$$
(e \times 1_X)^*\alpha : \mathcal{F} \longrightarrow \mathcal{F}
$$
is the identity map.
\end{enumerate}
For explanation compare with the relevant diagrams of
Equation (\ref{equation-action}).
\end{definition}

\noindent
Note that the commutativity of the first diagram guarantees that
$(e \times 1_X)^*\alpha$ is an idempotent operator on $\mathcal{F}$,
and hence condition (2) is just the condition that it is an isomorphism.

\begin{lemma}
\label{lemma-pullback-equivariant}
Let $S$ be a scheme. Let $G$ be a group scheme over $S$.
Let $f : X \to Y$ be a $G$-equivariant morphism between $S$-schemes
endowed with $G$-actions. Then pullback $f^*$ given by
$(\mathcal{F}, \alpha) \mapsto (f^*\mathcal{F}, (1_G \times f)^*\alpha)$
defines a functor from the category of $G$-equivariant sheaves on
$X$ to the category of quasi-coherent $G$-equivariant sheaves on $Y$.
\end{lemma}

\begin{proof}
Omitted.
\end{proof}





\section{Groupoids}
\label{section-groupoids}

\noindent
Recall that a groupoid is a category in which every morphism
is an isomorphism, see
Categories, Definition \ref{categories-definition-groupoid}.
Hence a groupoid has a set of objects $\text{Ob}$,
a set of arrows $\text{Arrows}$, a {\it source} and {\it target}
map $s, t : \text{Arrows} \to \text{Ob}$, and a {\it composition law}
$c : \text{Arrows} \times_{s, \text{Ob}, t} \text{Arrows}
\to \text{Arrows}$.
These maps satisfy exactly the following axioms
\begin{enumerate}
\item (associativity) $c \circ (1, c) = c \circ (c, 1)$ as maps
$\text{Arrows} \times_{s, \text{Ob}, t}
\text{Arrows} \times_{s, \text{Ob}, t}
\text{Arrows} \to \text{Arrows}$,
\item (identity) there exists a map $e : \text{Ob} \to \text{Arrows}$
such that
\begin{enumerate}
\item $s \circ e = t \circ e = \text{id}$ as maps $\text{Ob} \to \text{Ob}$,
\item $c \circ (1, e \circ s) = c \circ (e \circ t, 1) = 1$
as maps $\text{Arrows} \to \text{Arrows}$,
\end{enumerate}
\item (inverse) there exists a map $i : \text{Arrows} \to \text{Arrows}$
such that
\begin{enumerate}
\item $s \circ i = t$, $t \circ i = s$ as maps $\text{Arrows} \to \text{Ob}$,
and
\item $c \circ (1, i) = e \circ t$ and $c \circ (i, 1) = e \circ s$
as maps $\text{Arrows} \to \text{Arrows}$.
\end{enumerate}
\end{enumerate}
If this is the case the maps $e$ and $i$ are uniquely determined and
$i$ is a bijection. Note that if $(\text{Ob}', \text{Arrows}', s', t', c')$
is a second groupoid category, then a functor
$f : (\text{Ob}, \text{Arrows}, s, t, c) \to
(\text{Ob}', \text{Arrows}', s', t', c')$
is given by a pair of set maps $f : \text{Ob} \to \text{Ob}'$ and
$f : \text{Arrows} \to \text{Arrows}'$ such that
$s' \circ f = f \circ s$, $t' \circ f = f \circ t$, and
$c' \circ (f, f) = f \circ c$. The compatibility with identity and
inverse is automatic. We will use this below.
(Warning: The compatibility with identity
has to be imposed in the case of general categories.)

\begin{definition}
\label{definition-groupoid}
Let $S$ be a scheme.
\begin{enumerate}
\item A {\it groupoid scheme over $S$}, or simply a
{\it groupoid over $S$} is a
quintuple $(U, R, s, t, c)$ where
$U$ and $R$ are schemes over $S$, and
$s, t : R \to U$ and $c : R \times_{s, U, t} R \to R$
are morphisms of schemes over $S$ with the
following property: For any scheme
$T$ over $S$ the quintuple
$$
(U(T), R(T), s, t, c)
$$
is a groupoid category in the sense described above.
\item A {\it morphism
$f : (U, R, s, t, c) \to (U', R', s', t', c')$
of groupoid schemes over $S$} is given by morphisms
of schemes $f : U \to U'$ and $f : R \to R'$ with the
following property:  For any scheme
$T$ over $S$ the maps $f$ define a functor from the
groupoid category $(U(T), R(T), s, t, c)$ to the
groupoid category $(U'(T), R'(T), s', t', c')$.
\end{enumerate}
\end{definition}

\noindent
Let $(U, R, s, t, c)$ be a groupoid over $S$.
Note that, by the remarks preceding the definition and the Yoneda lemma,
there are unique morphisms of schemes
$e : U \to R$ and
$i : R \to R$ over $S$ such that for every scheme $T$ over $S$
the induced map $e : U(T) \to R(T)$ is the identity, and
$i : R(T) \to R(T)$ is the inverse of
the groupoid category. The septuple $(U, R, s, t, c, e, i)$
satisfies commutative diagrams corresponding to each of the
axioms (1), (2)(a), (2)(b), (3)(a) and (3)(b) above, and conversely
given a septuple with this property the quintuple $(U, R, s, t, c)$
is a groupoid scheme. Note that $i$ is an isomorphism,
and $e$ is a section of both $s$ and $t$.
Moreover, given a groupoid scheme over $S$ we denote
$$
j = (t, s) : R \longrightarrow U \times_S U
$$
which is compatible with our conventions in Section
\ref{section-equivalence-relations} above.
We sometimes say ``let $(U, R, s, t, c, e, i)$ be a
groupoid over $S$'' to stress the existence of identity and
inverse.

\begin{lemma}
\label{lemma-groupoid-pre-equivalence}
Given a groupoid scheme $(U, R, s, t, c)$ over $S$
the morphism $j : R \to U \times_S U$ is a pre-equivalence
relation.
\end{lemma}

\begin{proof}
Omitted.
This is a nice exercise in the definitions.
\end{proof}

\begin{lemma}
\label{lemma-equivalence-groupoid}
Given an equivalence relation $j : R \to U$ over $S$
there is a unique way to extend it to a groupoid
$(U, R, s, t, c)$ over $S$.
\end{lemma}

\begin{proof}
Omitted.
This is a nice exercise in the definitions.
\end{proof}

\begin{lemma}
\label{lemma-diagram}
Let $S$ be a scheme.
Let $(U, R, s, t, c)$ be a groupoid over $S$.
In the commutative diagram
$$
\xymatrix{
& U & \\
R \ar[d]_s \ar[ru]^t &
R \times_{s, U, t} R
\ar[l]^-{\text{pr}_0} \ar[d]^{\text{pr}_1} \ar[r]_-c &
R \ar[d]^s \ar[lu]_t \\
U & R \ar[l]_t \ar[r]^s & U
}
$$
the two lower squares are fibre product squares.
Moreover, the triangle on top (which is really a square)
is also cartesian.
\end{lemma}

\begin{proof}
Omitted.
Exercise in the definitions and the functorial point of
view in algebraic geometry.
\end{proof}

\begin{lemma}
\label{lemma-diagram-pull}
Let $S$ be a scheme.
Let $(U, R, s, t, c, e, i)$ be a groupoid over $S$.
The diagram
\begin{equation}
\label{equation-pull}
\xymatrix{
R \times_{t, U, t} R
\ar@<1ex>[r]^-{\text{pr}_1} \ar@<-1ex>[r]_-{\text{pr}_0}
\ar[d]_{(\text{pr}_0, c \circ (i, 1))} &
R \ar[r]^t \ar[d]^{\text{id}_R} &
U \ar[d]^{\text{id}_U} \\
R \times_{s, U, t} R
\ar@<1ex>[r]^-c \ar@<-1ex>[r]_-{\text{pr}_0} \ar[d]_{\text{pr}_1} &
R \ar[r]^t \ar[d]^s &
U \\
R \ar@<1ex>[r]^s \ar@<-1ex>[r]_t &
U
}
\end{equation}
is commutative. The two top rows are isomorphic via the vertical maps given.
The two lower left squares are cartesian.
\end{lemma}

\begin{proof}
The commutativity of the diagram follows from the axioms of a groupoid.
Note that, in terms of groupoids, the top left vertical arrow assigns to
a pair of morphisms $(\alpha, \beta)$ with the same target, the pair
of morphisms $(\alpha, \alpha^{-1} \circ \beta)$. In any groupoid
this defines a bijection between
$\text{Arrows} \times_{t, \text{Ob}, t} \text{Arrows}$
and
$\text{Arrows} \times_{s, \text{Ob}, t} \text{Arrows}$. Hence the second
assertion of the lemma.
The last assertion follows from Lemma \ref{lemma-diagram}.
\end{proof}

\begin{lemma}
\label{lemma-base-change-groupoid}
Let $(U, R, s, t, c)$ be a groupoid over a scheme $S$.
Let $S' \to S$ be a morphism. Then the base changes $U' = S' \times_S U$,
$R' = S' \times_S R$ endowed with the base changes $s'$, $t'$, $c'$
of the morphisms $s, t, c$ form a groupoid scheme
$(U', R', s', t', c')$ over $S'$ and the projections
determine a morphism
$(U', R', s', t', c') \to (U, R, s, t, c)$
of groupoid schemes over $S$.
\end{lemma}

\begin{proof}
Omitted. Hint:
$R' \times_{s', U', t'} R' = S' \times_S (R \times_{s, U, t} R)$.
\end{proof}









\section{Quasi-coherent sheaves on groupoids}
\label{section-groupoids-quasi-coherent}

\noindent
See the introduction of Section \ref{section-equivariant} for our
choices in direction of arrows.

\begin{definition}
\label{definition-groupoid-module}
Let $S$ be a scheme, let $(U, R, s, t, c)$ be a groupoid scheme over $S$.
A {\it quasi-coherent module on $(U, R, s, t, c)$}
is a pair $(\mathcal{F}, \alpha)$, where $\mathcal{F}$ is a quasi-coherent
$\mathcal{O}_U$-module, and $\alpha$ is a $\mathcal{O}_R$-module
map
$$
\alpha : t^*\mathcal{F} \longrightarrow s^*\mathcal{F}
$$
such that
\begin{enumerate}
\item the diagram
$$
\xymatrix{
& \text{pr}_1^*t^*\mathcal{F} \ar[r]_-{\text{pr}_1^*\alpha} &
\text{pr}_1^*s^*\mathcal{F} \ar@{=}[rd] & \\
\text{pr}_0^*s^*\mathcal{F} \ar@{=}[ru] & & & c^*s^*\mathcal{F} \\
& \text{pr}_0^*t^*\mathcal{F} \ar[lu]^{\text{pr}_0^*\alpha} \ar@{=}[r] &
c^*t^*\mathcal{F} \ar[ru]_{c^*\alpha}
}
$$
is a commutative in the category of
$\mathcal{O}_{R \times_{s, U, t} R}$-modules, and
\item the pullback
$$
e^*\alpha : \mathcal{F} \longrightarrow \mathcal{F}
$$
is the identity map.
\end{enumerate}
Compare with the commutative diagrams of Lemma \ref{lemma-diagram}.
\end{definition}

\noindent
The commutativity of the first diagram forces the operator $e^*\alpha$
to be idempotent. Hence the second condition can be reformulated as saying
that $e^*\alpha$ is an isomorphism. In fact, the condition implies that
$\alpha$ is an isomorphism.

\begin{lemma}
\label{lemma-isomorphism}
Let $S$ be a scheme, let $(U, R, s, t, c)$ be a groupoid scheme over $S$.
If $(\mathcal{F}, \alpha)$ is a quasi-coherent module on $(U, R, s, t, c)$
then $\alpha$ is an isomorphism.
\end{lemma}

\begin{proof}
Pull back the commutative diagram of
Definition \ref{definition-groupoid-module}
by the morphism $(i, 1) : R \to R \times_{s, U, t} R$.
Then we see that $i^*\alpha \circ \alpha = s^*e^*\alpha$.
Pulling back by the morphism $(1, i)$ we obtain the relation
$\alpha \circ i^*\alpha = t^*e^*\alpha$. By the second assumption 
these morphisms are the identity. Hence $i^*\alpha$ is an inverse of
$\alpha$.
\end{proof}

\begin{lemma}
\label{lemma-pullback}
Let $S$ be a scheme. Consider a morphism
$f : (U, R, s, t, c) \to (U', R', s', t', c')$
of groupoid schemes over $S$. Then pullback $f^*$ given by
$$
(\mathcal{F}, \alpha) \mapsto (f^*\mathcal{F}, f^*\alpha)
$$
defines a functor from the category of quasi-coherent sheaves on
$(U', R', s', t', c')$ to the category of quasi-coherent sheaves on
$(U, R, s, t, c)$.
\end{lemma}

\begin{proof}
Omitted.
\end{proof}

\begin{lemma}
\label{lemma-pushforward}
Let $S$ be a scheme. Consider a morphism
$f : (U, R, s, t, c) \to (U', R', s', t', c')$
of groupoid schemes over $S$. Assume that
\begin{enumerate}
\item $f : U \to U'$ is quasi-compact and quasi-separated,
\item the square
$$
\xymatrix{
R \ar[d]_t \ar[r]_f & R' \ar[d]^{t'} \\
U \ar[r]^f & U'
}
$$
is cartesian, and
\item $s'$ and $t'$ are flat.
\end{enumerate}
Then pushforward $f_*$ given by
$$
(\mathcal{F}, \alpha) \mapsto (f_*\mathcal{F}, f_*\alpha)
$$
defines a functor from the category of quasi-coherent sheaves on
$(U, R, s, t, c)$ to the category of quasi-coherent sheaves on
$(U', R', s', t', c')$ which is right adjoint to pullback as defined in
Lemma \ref{lemma-pullback}.
\end{lemma}

\begin{proof}
Since $U \to U'$ is quasi-compact and quasi-separated we see that
$f_*$ transforms quasi-coherent sheaves into quasi-coherent sheaves
(Schemes, Lemma \ref{schemes-lemma-push-forward-quasi-coherent}).
Moreover, since the squares
$$
\vcenter{
\xymatrix{
R \ar[d]_t \ar[r]_f & R' \ar[d]^{t'} \\
U \ar[r]^f & U'
}
}
\quad\text{and}\quad
\vcenter{
\xymatrix{
R \ar[d]_s \ar[r]_f & R' \ar[d]^{s'} \\
U \ar[r]^f & U'
}
}
$$
are cartesian we find that $(t')^*f_*\mathcal{F} = f_*t^*\mathcal{F}$
and $(s')^*f_*\mathcal{F} = f_*s^*\mathcal{F}$ , see
Cohomology of Schemes, Lemma
\ref{coherent-lemma-flat-base-change-cohomology}.
Thus it makes sense to think of $f_*\alpha$ as a map
$(t')^*f_*\mathcal{F} \to (s')^*f_*\mathcal{F}$. A similar argument
shows that $f_*\alpha$ satisfies the cocycle condition.
The functor is adjoint to the pullback functor since pullback
and pushforward on modules on ringed spaces are adjoint.
Some details omitted.
\end{proof}

\begin{lemma}
\label{lemma-colimits}
Let $S$ be a scheme. Let $(U, R, s, t, c)$ be a groupoid scheme over $S$.
The category of quasi-coherent modules on $(U, R, s, t, c)$ has colimits.
\end{lemma}

\begin{proof}
Let $i \mapsto (\mathcal{F}_i, \alpha_i)$ be a diagram over the index
category $\mathcal{I}$. We can form the colimit
$\mathcal{F} = \colim \mathcal{F}_i$
which is a quasi-coherent sheaf on $U$, see
Schemes, Section \ref{schemes-section-quasi-coherent}.
Since colimits commute with pullback we see that
$s^*\mathcal{F} = \colim s^*\mathcal{F}_i$ and similarly
$t^*\mathcal{F} = \colim t^*\mathcal{F}_i$. Hence we can set
$\alpha = \colim \alpha_i$. We omit the proof that $(\mathcal{F}, \alpha)$
is the colimit of the diagram in the category of quasi-coherent modules
on $(U, R, s, t, c)$.
\end{proof}

\begin{lemma}
\label{lemma-abelian}
Let $S$ be a scheme.
Let $(U, R, s, t, c)$ be a groupoid scheme over $S$.
If $s$, $t$ are flat, then the category of quasi-coherent modules on
$(U, R, s, t, c)$ is abelian.
\end{lemma}

\begin{proof}
Let $\varphi : (\mathcal{F}, \alpha) \to (\mathcal{G}, \beta)$ be a
homomorphism of quasi-coherent modules on $(U, R, s, t, c)$. Since
$s$ is flat we see that
$$
0 \to s^*\Ker(\varphi)
\to s^*\mathcal{F} \to s^*\mathcal{G} \to s^*\Coker(\varphi) \to 0
$$
is exact and similarly for pullback by $t$. Hence $\alpha$ and $\beta$
induce isomorphisms
$\kappa : t^*\Ker(\varphi) \to s^*\Ker(\varphi)$ and
$\lambda : t^*\Coker(\varphi) \to s^*\Coker(\varphi)$
which satisfy the cocycle condition. Then it is straightforward to
verify that $(\Ker(\varphi), \kappa)$ and
$(\Coker(\varphi), \lambda)$ are a kernel and cokernel in the
category of quasi-coherent modules on $(U, R, s, t, c)$. Moreover,
the condition $\Coim(\varphi) = \Im(\varphi)$ follows
because it holds over $U$.
\end{proof}













\section{Colimits of quasi-coherent modules}
\label{section-colimits}

\noindent
In this section we prove some technical results saying that under
suitable assumptions every quasi-coherent module on a groupoid is
a filtered colimit of ``small'' quasi-coherent modules.

\begin{lemma}
\label{lemma-construct-quasi-coherent}
Let $(U, R, s, t, c)$ be a groupoid scheme over $S$.
Assume $s, t$ are flat, quasi-compact, and quasi-separated.
For any quasi-coherent module $\mathcal{G}$ on $U$, there exists
a canonical isomorphism
$\alpha : t^*t_*s^*\mathcal{G} \to s^*t_*s^*\mathcal{G}$
which turns $(t_*s^*\mathcal{G}, \alpha)$ into a quasi-coherent module
on $(U, R, s, t, c)$. This construction defines a functor
$$
\QCoh(\mathcal{O}_U) \longrightarrow \QCoh(U, R, s, t, c)
$$
which is a right adjoint to the forgetful functor
$(\mathcal{F}, \beta) \mapsto \mathcal{F}$.
\end{lemma}

\begin{proof}
The pushforward of a quasi-coherent module along a quasi-compact and
quasi-separated morphism is quasi-coherent, see Schemes, Lemma
\ref{schemes-lemma-push-forward-quasi-coherent}. Hence $t_*s^*\mathcal{G}$
is quasi-coherent. With notation as in Lemma \ref{lemma-diagram} we have
$$
t^*t_*s^*\mathcal{G} =
\text{pr}_{0, *}c^* s^*\mathcal{G} =
\text{pr}_{0, *}\text{pr}_1^*s^*\mathcal{G} =
s^*t_*s^*\mathcal{G}
$$
The middle equality because $s \circ c = s \circ \text{pr}_1$ as
morphisms $R \times_{s, U, t} R \to U$, and the first and the last
equality because we know that base change and pushforward commute in
these steps by Cohomology of Schemes, Lemma
\ref{coherent-lemma-flat-base-change-cohomology}.

\medskip\noindent
To verify the cocycle condition of Definition \ref{definition-groupoid-module}
for $\alpha$ and the adjointness property we describe the construction
$\mathcal{G} \mapsto (\mathcal{G}, \alpha)$ in another way.
Consider the groupoid scheme
$(R, R \times_{s, U, s} R, \text{pr}_0, \text{pr}_1, \text{pr}_{02})$
associated to the equivalence relation $R \times_{s, U, s} R$
on $R$, see Lemma \ref{lemma-equivalence-groupoid}.
There is a morphism
$$
f :
(R, R \times_{s, U, s} R, \text{pr}_1, \text{pr}_0, \text{pr}_{02})
\longrightarrow
(U, R, s, t, c)
$$
of groupoid schemes given by $t : R \to U$ and $R \times_{t, U, t} R \to R$
given by $(r_0, r_1) \mapsto r_0 \circ r_1^{-1}$ (we omit the verification
of the commutativity of the required diagrams). Since
$t, s : R \to U$ are quasi-compact, quasi-separated, and flat,
and since we have a cartesian square
$$
\xymatrix{
R \times_{s, U, s} R \ar[d]_{\text{pr}_0}
\ar[rr]_-{(r_0, r_1) \mapsto r_0 \circ r_1^{-1}} & & R \ar[d]^t \\
R \ar[rr]^t & & U
}
$$
by Lemma \ref{lemma-diagram-pull} it follows that
Lemma \ref{lemma-pushforward} applies to $f$. Note that
$$
\QCoh(R, R \times_{s, U, s} R, \text{pr}_1, \text{pr}_0, \text{pr}_{02})
= \QCoh(\mathcal{O}_U)
$$
by the theory of descent of quasi-coherent sheaves as $\{t : R \to U\}$
is an fpqc covering, see
Descent, Proposition \ref{descent-proposition-fpqc-descent-quasi-coherent}.
Observe that pullback along $f$ agrees with the forgetful functor and
that pushforward agrees with the construction that assigns to
$\mathcal{G}$ the pair $(\mathcal{G}, \alpha)$. We omit the precise
verifications. Thus the lemma follows from Lemma \ref{lemma-pushforward}.
\end{proof}

\begin{lemma}
\label{lemma-push-pull}
Let $f : Y \to X$ be a morphism of schemes. Let $\mathcal{F}$
be a quasi-coherent $\mathcal{O}_X$-module, let $\mathcal{G}$
be a quasi-coherent $\mathcal{O}_Y$-module, and let
$\varphi : \mathcal{G} \to f^*\mathcal{F}$ be a module map. Assume
\begin{enumerate}
\item $\varphi$ is injective,
\item $f$ is quasi-compact, quasi-separated, flat, and surjective,
\item $X$, $Y$ are locally Noetherian, and
\item $\mathcal{G}$ is a coherent $\mathcal{O}_Y$-module.
\end{enumerate}
Then $\mathcal{F} \cap f_*\mathcal{G}$ defined as the pullback
$$
\xymatrix{
\mathcal{F} \ar[r] & f_*f^*\mathcal{F} \\
\mathcal{F} \cap f_*\mathcal{G} \ar[u] \ar[r] &
f_*\mathcal{G} \ar[u]
}
$$
is a coherent $\mathcal{O}_X$-module.
\end{lemma}

\begin{proof}
We will freely use the characterization of coherent modules of
Cohomology of Schemes, Lemma \ref{coherent-lemma-coherent-Noetherian}
as well as the fact that coherent modules form a Serre subcategory
of $\QCoh(\mathcal{O}_X)$, see
Cohomology of Schemes,
Lemma \ref{coherent-lemma-coherent-Noetherian-quasi-coherent-sub-quotient}.
If $f$ has a section $\sigma$, then we see that
$\mathcal{F} \cap f_*\mathcal{G}$ is contained in the image of
$\sigma^*\mathcal{G} \to \sigma^*f^*\mathcal{F} = \mathcal{F}$,
hence coherent. In general, to show that $\mathcal{F} \cap f_*\mathcal{G}$
is coherent, it suffices the show that
$f^*(\mathcal{F} \cap f_*\mathcal{G})$ is coherent (see
Descent, Lemma \ref{descent-lemma-finite-type-descends}).
Since $f$ is flat this is equal to $f^*\mathcal{F} \cap f^*f_*\mathcal{G}$.
Since $f$ is flat, quasi-compact, and quasi-separated we see
$f^*f_*\mathcal{G} = p_*q^*\mathcal{G}$ where $p, q : Y \times_X Y \to Y$
are the projections, see
Cohomology of Schemes, Lemma \ref{coherent-lemma-flat-base-change-cohomology}.
Since $p$ has a section we win.
\end{proof}

\noindent
Let $S$ be a scheme. Let $(U, R, s, t, c)$ be a groupoid in schemes over $S$.
Assume that $U$ is locally Noetherian. In the lemma below we say that a
quasi-coherent sheaf $(\mathcal{F}, \alpha)$ on $(U, R, s, t, c)$ is
{\it coherent} if $\mathcal{F}$ is a coherent $\mathcal{O}_U$-module.

\begin{lemma}
\label{lemma-colimit-coherent}
Let $(U, R, s, t, c)$ be a groupoid scheme over $S$.
Assume that
\begin{enumerate}
\item $U$, $R$ are Noetherian,
\item $s, t$ are flat, quasi-compact, and quasi-separated.
\end{enumerate}
Then every quasi-coherent module $(\mathcal{F}, \alpha)$ on $(U, R, s, t, c)$
is a filtered colimit of coherent modules.
\end{lemma}

\begin{proof}
We will use the characterization of Cohomology of Schemes, Lemma
\ref{coherent-lemma-coherent-Noetherian} of coherent modules on locally
Noetherian scheme without further mention. Write
$\mathcal{F} = \colim \mathcal{H}_i$ with $\mathcal{H}_i$ coherent, see
Properties, Lemma \ref{properties-lemma-directed-colimit-finite-presentation}.
Given a quasi-coherent sheaf $\mathcal{H}$ on $U$ we denote $t_*s^*\mathcal{H}$
the quasi-coherent sheaf on $(U, R, s, t, c)$ of
Lemma \ref{lemma-construct-quasi-coherent}. There is an adjunction map
$\mathcal{F} \to t_*s^*\mathcal{F}$ in $\QCoh(U, R, s, t, c)$.
Consider the pullback diagram
$$
\xymatrix{
\mathcal{F} \ar[r] & t_*s^*\mathcal{F} \\
\mathcal{F}_i \ar[r] \ar[u] & t_*s^*\mathcal{H}_i \ar[u]
}
$$
in other words $\mathcal{F}_i = \mathcal{F} \cap t_*s^*\mathcal{H}_i$.
Then $\mathcal{F}_i$ is coherent by Lemma \ref{lemma-push-pull}.
On the other hand, the diagram above is a pullback diagram in
$\QCoh(U, R, s, t, c)$ also as restriction to $U$ is an
exact functor by (the proof of) Lemma \ref{lemma-abelian}. Finally,
because $t$ is quasi-compact and quasi-separated we see that
$t_*$ commutes with colimits (see
Cohomology of Schemes, Lemma \ref{coherent-lemma-colimit-cohomology}).
Hence $t_*s^*\mathcal{F} = \colim t_*\mathcal{H}_i$ and hence
$\mathcal{F} = \colim \mathcal{F}_i$ as desired.
\end{proof}

\noindent
Here is a curious lemma that is useful when working with groupoids
on fields. In fact, this is the standard argument to prove that any
representation of an algebraic group is a colimit of finite dimensional
representations.

\begin{lemma}
\label{lemma-colimit-finite-type}
Let $(U, R, s, t, c)$ be a groupoid scheme over $S$.
Assume that
\begin{enumerate}
\item $U$, $R$ are affine,
\item there exist $e_i \in \mathcal{O}_R(R)$ such that
every element $g \in \mathcal{O}_R(R)$ can be uniquely written as
$\sum s^*(f_i)e_i$ for some $f_i \in \mathcal{O}_U(U)$.
\end{enumerate}
Then every quasi-coherent module $(\mathcal{F}, \alpha)$ on $(U, R, s, t, c)$
is a filtered colimit of finite type quasi-coherent modules.
\end{lemma}

\begin{proof}
The assumption means that $\mathcal{O}_R(R)$ is a free
$\mathcal{O}_U(U)$-module via $s$ with basis $e_i$. Hence
for any quasi-coherent $\mathcal{O}_U$-module $\mathcal{G}$
we see that $s^*\mathcal{G}(R) = \bigoplus_i \mathcal{G}(U)e_i$.
We will write $s(-)$ to indicate pullback of sections by $s$ and
similarly for other morphisms.
Let $(\mathcal{F}, \alpha)$ be a quasi-coherent module on
$(U, R, s, t, c)$. Let $\sigma \in \mathcal{F}(U)$. By the above
we can write
$$
\alpha(t(\sigma)) = \sum s(\sigma_i) e_i
$$
for some unique $\sigma_i \in \mathcal{F}(U)$ (all but finitely many
are zero of course). We can also write
$$
c(e_i) = \sum \text{pr}_1(f_{ij}) \text{pr}_0(e_j)
$$
as functions on $R \times_{s, U, t}R$. Then the commutativity of the diagram
in Definition \ref{definition-groupoid-module} means that
$$
\sum \text{pr}_1(\alpha(t(\sigma_i))) \text{pr}_0(e_i)
=
\sum \text{pr}_1(s(\sigma_i)f_{ij}) \text{pr}_0(e_j)
$$
(calculation omitted). Picking off the coefficients of $\text{pr}_0(e_l)$
we see that $\alpha(t(\sigma_l)) = \sum s(\sigma_i)f_{il}$. Hence
the submodule $\mathcal{G} \subset \mathcal{F}$ generated by the
elements $\sigma_i$ defines a finite type quasi-coherent module
preserved by $\alpha$. Hence it is a subobject of $\mathcal{F}$ in
$\QCoh(U, R, s, t, c)$. This submodule contains $\sigma$
(as one sees by pulling back the first relation by $e$). Hence we win.
\end{proof}

\noindent
We suggest the reader skip the rest of this section. Let $S$ be a scheme.
Let $(U, R, s, t, c)$ be a groupoid in schemes over $S$. Let $\kappa$ be a
cardinal. In the following we will say that a quasi-coherent sheaf
$(\mathcal{F}, \alpha)$ on $(U, R, s, t, c)$ is $\kappa$-generated if
$\mathcal{F}$ is a $\kappa$-generated $\mathcal{O}_U$-module, see
Properties, Definition \ref{properties-definition-kappa-generated}.

\begin{lemma}
\label{lemma-set-of-iso-classes}
Let $(U, R, s, t, c)$ be a groupoid scheme over $S$.
Let $\kappa$ be a cardinal.
There exists a set $T$ and a family $(\mathcal{F}_t, \alpha_t)_{t \in T}$ of
$\kappa$-generated quasi-coherent modules on $(U, R, s, t, c)$
such that every $\kappa$-generated quasi-coherent module on
$(U, R, s, t, c)$ is isomorphic to one of the $(\mathcal{F}_t, \alpha_t)$.
\end{lemma}

\begin{proof}
For each quasi-coherent module $\mathcal{F}$ on $U$ there is a
(possibly empty) set of maps $\alpha : t^*\mathcal{F} \to s^*\mathcal{F}$
such that $(\mathcal{F}, \alpha)$ is a quasi-coherent modules on
$(U, R, s, t, c)$. By
Properties, Lemma \ref{properties-lemma-set-of-iso-classes}
there exists a set of isomorphism classes of $\kappa$-generated
quasi-coherent $\mathcal{O}_U$-modules.
\end{proof}

\begin{lemma}
\label{lemma-colimit-kappa}
Let $(U, R, s, t, c)$ be a groupoid scheme over $S$.
Assume that $s, t$ are flat. There exists a
cardinal $\kappa$ such that every quasi-coherent module
$(\mathcal{F}, \alpha)$ on $(U, R, s, t, c)$
is the directed colimit of its $\kappa$-generated
quasi-coherent submodules.
\end{lemma}

\begin{proof}
In the statement of the lemma and in this proof
a {\it submodule} of a quasi-coherent module $(\mathcal{F}, \alpha)$
is a quasi-coherent submodule $\mathcal{G} \subset \mathcal{F}$
such that $\alpha(t^*\mathcal{G}) = s^*\mathcal{G}$ as subsheaves of
$s^*\mathcal{F}$. This makes sense because since $s, t$ are flat the
pullbacks $s^*$ and $t^*$ are exact, i.e., preserve subsheaves.
The proof will be a repeat of the proof of
Properties, Lemma \ref{properties-lemma-colimit-kappa}.
We urge the reader to read that proof first.

\medskip\noindent
Choose an affine open covering $U = \bigcup_{i \in I} U_i$.
For each pair $i, j$ choose affine open coverings
$$
U_i \cap U_j = \bigcup\nolimits_{k \in I_{ij}} U_{ijk}
\quad\text{and}\quad
s^{-1}(U_i) \cap t^{-1}(U_j) = \bigcup\nolimits_{k \in J_{ij}} W_{ijk}.
$$
Write $U_i = \Spec(A_i)$, $U_{ijk} = \Spec(A_{ijk})$,
$W_{ijk} = \Spec(B_{ijk})$.
Let $\kappa$ be any infinite cardinal $\geq$ than the cardinality
of any of the sets $I$, $I_{ij}$, $J_{ij}$.

\medskip\noindent
Let $(\mathcal{F}, \alpha)$ be a quasi-coherent module on $(U, R, s, t, c)$.
Set $M_i = \mathcal{F}(U_i)$, $M_{ijk} = \mathcal{F}(U_{ijk})$.
Note that
$$
M_i \otimes_{A_i} A_{ijk} = M_{ijk} = M_j \otimes_{A_j} A_{ijk}
$$
and that $\alpha$ gives isomorphisms
$$
\alpha|_{W_{ijk}} :
M_i \otimes_{A_i, t} B_{ijk}
\longrightarrow
M_j \otimes_{A_j, s} B_{ijk}
$$
see
Schemes, Lemma \ref{schemes-lemma-widetilde-pullback}.
Using the axiom of choice we choose a map
$$
(i, j, k, m) \mapsto S(i, j, k, m)
$$
which associates to every $i, j \in I$, $k \in I_{ij}$ or $k \in J_{ij}$
and $m \in M_i$ a finite subset $S(i, j, k, m) \subset M_j$
such that we have
$$
m \otimes 1 = \sum\nolimits_{m' \in S(i, j, k, m)} m' \otimes a_{m'}
\quad\text{or}\quad
\alpha(m \otimes 1) = \sum\nolimits_{m' \in S(i, j, k, m)} m' \otimes b_{m'}
$$
in $M_{ijk}$ for some $a_{m'} \in A_{ijk}$ or $b_{m'} \in B_{ijk}$.
Moreover, let's agree that $S(i, i, k, m) = \{m\}$ for all
$i, j = i, k, m$ when $k \in I_{ij}$. Fix such a collection $S(i, j, k, m)$

\medskip\noindent
Given a family $\mathcal{S} = (S_i)_{i \in I}$ of subsets
$S_i \subset M_i$ of cardinality at most $\kappa$ we set
$\mathcal{S}' = (S'_i)$ where
$$
S'_j = \bigcup\nolimits_{(i, j, k, m)\text{ such that }m \in S_i}
S(i, j, k, m)
$$
Note that $S_i \subset S'_i$. Note that $S'_i$ has cardinality at most
$\kappa$ because it is a union over a set of cardinality at most $\kappa$
of finite sets. Set $\mathcal{S}^{(0)} = \mathcal{S}$,
$\mathcal{S}^{(1)} = \mathcal{S}'$ and by induction
$\mathcal{S}^{(n + 1)} = (\mathcal{S}^{(n)})'$. Then set
$\mathcal{S}^{(\infty)} = \bigcup_{n \geq 0} \mathcal{S}^{(n)}$.
Writing $\mathcal{S}^{(\infty)} = (S^{(\infty)}_i)$ we see that
for any element $m \in S^{(\infty)}_i$ the image of $m$ in
$M_{ijk}$ can be written as a finite sum $\sum m' \otimes a_{m'}$
with $m' \in S_j^{(\infty)}$. In this way we see that setting
$$
N_i = A_i\text{-submodule of }M_i\text{ generated by }S^{(\infty)}_i
$$
we have
$$
N_i \otimes_{A_i} A_{ijk} = N_j \otimes_{A_j} A_{ijk}
\quad\text{and}\quad
\alpha(N_i \otimes_{A_i, t} B_{ijk}) = N_j \otimes_{A_j, s} B_{ijk}
$$
as submodules of $M_{ijk}$ or $M_j \otimes_{A_j, s} B_{ijk}$.
Thus there exists a quasi-coherent submodule
$\mathcal{G} \subset \mathcal{F}$ with $\mathcal{G}(U_i) = N_i$
such that $\alpha(t^*\mathcal{G}) = s^*\mathcal{G}$ as submodules
of $s^*\mathcal{F}$. In other words,
$(\mathcal{G}, \alpha|_{t^*\mathcal{G}})$ is a submodule of
$(\mathcal{F}, \alpha)$.
Moreover, by construction $\mathcal{G}$ is $\kappa$-generated.

\medskip\noindent
Let $\{(\mathcal{G}_t, \alpha_t)\}_{t \in T}$ be the set of
$\kappa$-generated quasi-coherent submodules of $(\mathcal{F}, \alpha)$.
If $t, t' \in T$ then $\mathcal{G}_t + \mathcal{G}_{t'}$ is also a
$\kappa$-generated quasi-coherent submodule as it is the image of the map
$\mathcal{G}_t \oplus \mathcal{G}_{t'} \to \mathcal{F}$.
Hence the system (ordered by inclusion) is directed.
The arguments above show that every section of $\mathcal{F}$ over $U_i$
is in one of the $\mathcal{G}_t$ (because we can start with $\mathcal{S}$
such that the given section is an element of $S_i$). Hence
$\colim_t \mathcal{G}_t \to \mathcal{F}$ is both injective and surjective
as desired.
\end{proof}





\section{Groupoids and group schemes}
\label{section-groupoids-group-schemes}

\noindent
There are many ways to construct a groupoid out of an action $a$
of a group $G$ on a set $V$. We choose the one where we think
of an element $g \in G$ as an arrow with source $v$ and target $a(g, v)$.
This leads to the following construction for group actions of
schemes.

\begin{lemma}
\label{lemma-groupoid-from-action}
Let $S$ be a scheme.
Let $Y$ be a scheme over $S$.
Let $(G, m)$ be a group scheme over $Y$ with
identity $e_G$ and inverse $i_G$.
Let $X/Y$ be a scheme over $Y$ and let $a : G \times_Y X \to X$
be an action of $G$ on $X/Y$.
Then we get a groupoid scheme $(U, R, s, t, c, e, i)$ over $S$
in the following manner:
\begin{enumerate}
\item We set $U = X$, and $R = G \times_Y X$.
\item We set $s : R \to U$ equal to $(g, x) \mapsto x$.
\item We set $t : R \to U$ equal to $(g, x) \mapsto a(g, x)$.
\item We set $c : R \times_{s, U, t} R \to R$ equal to
$((g, x), (g', x')) \mapsto (m(g, g'), x')$.
\item We set $e : U \to R$ equal to $x \mapsto (e_G(x), x)$.
\item We set $i : R \to R$ equal to $(g, x) \mapsto (i_G(g), a(g, x))$.
\end{enumerate}
\end{lemma}

\begin{proof}
Omitted. Hint: It is enough to show that this works on the set
level. For this use the description above the lemma describing
$g$ as an arrow from $v$ to $a(g, v)$.
\end{proof}

\begin{lemma}
\label{lemma-action-groupoid-modules}
Let $S$ be a scheme.
Let $Y$ be a scheme over $S$.
Let $(G, m)$ be a group scheme over $Y$.
Let $X$ be a scheme over $Y$ and let $a : G \times_Y X \to X$
be an action of $G$ on $X$ over $Y$. Let $(U, R, s, t, c)$ be
the groupoid scheme constructed in Lemma \ref{lemma-groupoid-from-action}.
The rule
$(\mathcal{F}, \alpha) \mapsto (\mathcal{F}, \alpha)$ defines
an equivalence of categories between $G$-equivariant
$\mathcal{O}_X$-modules and the category of quasi-coherent
modules on $(U, R, s, t, c)$.
\end{lemma}

\begin{proof}
The assertion makes sense because $t = a$ and $s = \text{pr}_1$
as morphisms $R = G \times_Y X \to X$, see
Definitions \ref{definition-equivariant-module} and
\ref{definition-groupoid-module}.
Using the translation in Lemma \ref{lemma-groupoid-from-action}
the commutativity requirements
of the two definitions match up exactly.
\end{proof}





\section{The stabilizer group scheme}
\label{section-stabilizer}

\noindent
Given a groupoid scheme we get a group scheme as follows.

\begin{lemma}
\label{lemma-groupoid-stabilizer}
Let $S$ be a scheme.
Let $(U, R, s, t, c)$ be a groupoid over $S$.
The scheme $G$ defined by the cartesian square
$$
\xymatrix{
G \ar[r] \ar[d] & R \ar[d]^{j = (t, s)} \\
U \ar[r]^-{\Delta} & U \times_S U
}
$$
is a group scheme over $U$ with composition law
$m$ induced by the composition law $c$.
\end{lemma}

\begin{proof}
This is true because in a groupoid category the
set of self maps of any object forms a group.
\end{proof}

\noindent
Since $\Delta$ is an immersion we see that $G = j^{-1}(\Delta_{U/S})$
is a locally closed subscheme of $R$. Thinking of it in this way,
the structure morphism $j^{-1}(\Delta_{U/S}) \to U$ is induced by
either $s$ or $t$ (it is the same), and $m$ is induced by $c$.

\begin{definition}
\label{definition-stabilizer-groupoid}
Let $S$ be a scheme.
Let $(U, R, s, t, c)$ be a groupoid over $S$.
The group scheme $j^{-1}(\Delta_{U/S})\to U$
is called the {\it stabilizer of the groupoid scheme
$(U, R, s, t, c)$}.
\end{definition}

\noindent
In the literature the stabilizer group scheme is often denoted $S$
(because the word stabilizer starts with an ``s'' presumably);
we cannot do this since we have already used $S$ for the base scheme.

\begin{lemma}
\label{lemma-groupoid-action-stabilizer}
Let $S$ be a scheme.
Let $(U, R, s, t, c)$ be a groupoid over $S$, and let $G/U$ be its stabilizer.
Denote $R_t/U$ the scheme $R$ seen as a scheme over $U$ via the
morphism $t : R \to U$.
There is a canonical left action
$$
a : G \times_U R_t \longrightarrow R_t
$$
induced by the composition law $c$.
\end{lemma}

\begin{proof}
In terms of points over $T/S$ we define $a(g, r) = c(g, r)$.
\end{proof}

\begin{lemma}
\label{lemma-groupoid-action-stabilizer-pseudo-torsor}
Let $S$ be a scheme. Let $(U, R, s, t, c)$ be a groupoid scheme
over $S$. Let $G$ be the stabilizer group scheme of $R$.
Let
$$
G_0 = G \times_{U, \text{pr}_0} (U \times_S U) = G \times_S U
$$
as a group scheme over $U \times_S U$. The action of $G$ on $R$ of
Lemma \ref{lemma-groupoid-action-stabilizer}
induces an action of $G_0$ on $R$ over $U \times_S U$
which turns $R$ into a pseudo $G_0$-torsor over $U \times_S U$.
\end{lemma}

\begin{proof}
This is true because in a groupoid category $\mathcal{C}$ the set
$\Mor_\mathcal{C}(x, y)$ is a principal homogeneous set
under the group $\Mor_\mathcal{C}(y, y)$.
\end{proof}

\begin{lemma}
\label{lemma-fibres-j}
Let $S$ be a scheme. Let $(U, R, s, t, c)$ be a groupoid scheme over $S$.
Let $p \in U \times_S U$ be a point. Denote
$R_p$ the scheme theoretic fibre of $j = (t, s) : R \to U \times_S U$.
If $R_p \not = \emptyset$, then the action
$$
G_{0, \kappa(p)} \times_{\kappa(p)} R_p \longrightarrow R_p
$$
(see
Lemma \ref{lemma-groupoid-action-stabilizer-pseudo-torsor})
which turns $R_p$ into a $G_{\kappa(p)}$-torsor over $\kappa(p)$.
\end{lemma}

\begin{proof}
The action is a pseudo-torsor by the lemma cited in the statement.
And if $R_p$ is not the empty scheme, then $\{R_p \to p\}$
is an fpqc covering which trivializes the pseudo-torsor.
\end{proof}







\section{Restricting groupoids}
\label{section-restrict-groupoid}

\noindent
Consider a (usual) groupoid
$\mathcal{C} = (\text{Ob}, \text{Arrows}, s, t, c)$.
Suppose we have a map of sets $g : \text{Ob}' \to \text{Ob}$.
Then we can construct a groupoid
$\mathcal{C}' = (\text{Ob}', \text{Arrows}', s', t', c')$
by thinking of a morphism between elements $x', y'$ of $\text{Ob}'$
as a morphism in $\mathcal{C}$ between $g(x'), g(y')$.
In other words we set
$$
\text{Arrows}' =
\text{Ob}'
\times_{g, \text{Ob}, t}
\text{Arrows}
\times_{s, \text{Ob}, g}
\text{Ob}'.
$$
with obvious choices for $s'$, $t'$, and $c'$. There is a canonical
functor $\mathcal{C}' \to \mathcal{C}$ which is fully faithful,
but not necessarily essentially surjective. This groupoid $\mathcal{C}'$
endowed with the functor $\mathcal{C}' \to \mathcal{C}$
is called the {\it restriction} of the groupoid
$\mathcal{C}$ to $\text{Ob}'$.

\begin{lemma}
\label{lemma-restrict-groupoid}
Let $S$ be a scheme.
Let $(U, R, s, t, c)$ be a groupoid scheme over $S$.
Let $g : U' \to U$ be a morphism of schemes.
Consider the following diagram
$$
\xymatrix{
R' \ar[d] \ar[r] \ar@/_3pc/[dd]_{t'} \ar@/^1pc/[rr]^{s'}&
R \times_{s, U} U' \ar[r] \ar[d] &
U' \ar[d]^g \\
U' \times_{U, t} R \ar[d] \ar[r] &
R \ar[r]^s \ar[d]_t &
U \\
U' \ar[r]^g &
U
}
$$
where all the squares are fibre product squares. Then there is a
canonical composition law $c' : R' \times_{s', U', t'} R' \to R'$
such that $(U', R', s', t', c')$ is a groupoid scheme over
$S$ and such that $U' \to U$, $R' \to R$ defines a morphism
$(U', R', s', t', c') \to (U, R, s, t, c)$ of groupoid schemes over $S$.
Moreover, for any scheme $T$ over $S$ the functor of groupoids
$$
(U'(T), R'(T), s', t', c') \to (U(T), R(T), s, t, c)
$$
is the restriction (see above) of $(U(T), R(T), s, t, c)$ via the map
$U'(T) \to U(T)$.
\end{lemma}

\begin{proof}
Omitted.
\end{proof}

\begin{definition}
\label{definition-restrict-groupoid}
Let $S$ be a scheme.
Let $(U, R, s, t, c)$ be a groupoid scheme over $S$.
Let $g : U' \to U$ be a morphism of schemes.
The morphism of groupoids
$(U', R', s', t', c') \to (U, R, s, t, c)$
constructed in Lemma \ref{lemma-restrict-groupoid} is called
the {\it restriction of $(U, R, s, t, c)$ to $U'$}.
We sometime use the notation $R' = R|_{U'}$ in this case.
\end{definition}

\begin{lemma}
\label{lemma-restrict-groupoid-relation}
The notions of restricting groupoids and
(pre-)equivalence relations defined in Definitions
\ref{definition-restrict-groupoid} and \ref{definition-restrict-relation}
agree via the constructions of
Lemmas \ref{lemma-groupoid-pre-equivalence} and
\ref{lemma-equivalence-groupoid}.
\end{lemma}

\begin{proof}
What we are saying here is that $R'$ of
Lemma \ref{lemma-restrict-groupoid} is also
equal to
$$
R' = (U' \times_S U')\times_{U \times_S U} R
\longrightarrow
U' \times_S U'
$$
In fact this might have been a clearer way to state that lemma.
\end{proof}

\begin{lemma}
\label{lemma-restrict-stabilizer}
Let $S$ be a scheme.
Let $(U, R, s, t, c)$ be a groupoid scheme over $S$.
Let $g : U' \to U$ be a morphism of schemes.
Let $(U', R', s', t', c')$ be the restriction of $(U, R, s, t, c)$ via $g$.
Let $G$ be the stabilizer of $(U, R, s, t, c)$ and let
$G'$ be the stabilizer of $(U', R', s', t', c')$.
Then $G'$ is the base change of $G$ by $g$, i.e.,
there is a canonical identification $G' = U' \times_{g, U} G$.
\end{lemma}

\begin{proof}
Omitted.
\end{proof}






\section{Invariant subschemes}
\label{section-invariant}

\noindent
In this section we discuss briefly the notion of an invariant subscheme.

\begin{definition}
\label{definition-invariant-open}
Let $(U, R, s, t, c)$ be a groupoid scheme over the base scheme $S$.
\begin{enumerate}
\item A subset $W \subset U$ is {\it set-theoretically $R$-invariant}
if $t(s^{-1}(W)) \subset W$.
\item An open $W \subset U$ is {\it $R$-invariant} if
$t(s^{-1}(W)) \subset W$.
\item A closed subscheme $Z \subset U$ is called {\it $R$-invariant}
if $t^{-1}(Z) = s^{-1}(Z)$. Here we use the scheme theoretic inverse image, see
Schemes, Definition \ref{schemes-definition-inverse-image-closed-subscheme}.
\item A monomorphism of schemes $T \to U$ is {\it $R$-invariant} if
$T \times_{U, t} R = R \times_{s, U} T$ as schemes over $R$.
\end{enumerate}
\end{definition}

\noindent
For subsets and open subschemes $W \subset U$ the $R$-invariance
is also equivalent to requiring that $s^{-1}(W) = t^{-1}(W)$
as subsets of $R$. If $W \subset U$ is an $R$-equivariant open subscheme
then the restriction of $R$ to $W$ is just $R_W = s^{-1}(W) = t^{-1}(W)$.
Similarly, if $Z \subset U$ is an $R$-invariant closed subscheme, then
the restriction of $R$ to $Z$ is just $R_Z = s^{-1}(Z) = t^{-1}(Z)$.

\begin{lemma}
\label{lemma-constructing-invariant-opens}
Let $S$ be a scheme.
Let $(U, R, s, t, c)$ be a groupoid scheme over $S$.
\begin{enumerate}
\item For any subset $W \subset U$ the subset $t(s^{-1}(W))$
is set-theoretically $R$-invariant.
\item If $s$ and $t$ are open, then for every open $W \subset U$
the open $t(s^{-1}(W))$ is an $R$-invariant open subscheme.
\item If $s$ and $t$ are open and quasi-compact, then $U$ has an open
covering consisting of $R$-invariant quasi-compact open subschemes.
\end{enumerate}
\end{lemma}

\begin{proof}
Part (1) follows from
Lemmas \ref{lemma-pre-equivalence-equivalence-relation-points} and
\ref{lemma-groupoid-pre-equivalence}, namely, $t(s^{-1}(W))$
is the set of points of $U$ equivalent to a point of $W$.
Next, assume $s$ and $t$ open and $W \subset U$ open.
Since $s$ is open the set $W' = t(s^{-1}(W))$ is an open subset of $U$.
Finally, assume that $s$, $t$ are both open and quasi-compact.
Then, if $W \subset U$ is a quasi-compact open, then also
$W' = t(s^{-1}(W))$ is a quasi-compact open, and invariant by the
discussion above. Letting $W$ range over all affine opens of $U$
we see (3).
\end{proof}

\begin{lemma}
\label{lemma-first-observation}
Let $S$ be a scheme. Let $(U, R, s, t, c)$ be a groupoid scheme over $S$.
Assume $s$ and $t$ quasi-compact and flat and $U$ quasi-separated.
Let $W \subset U$ be quasi-compact open. Then $t(s^{-1}(W))$
is an intersection of a nonempty family of quasi-compact open subsets of $U$.
\end{lemma}

\begin{proof}
Note that $s^{-1}(W)$ is quasi-compact open in $R$.
As a continuous map $t$ maps the quasi-compact subset
$s^{-1}(W)$ to a quasi-compact subset $t(s^{-1}(W))$.
As $t$ is flat and $s^{-1}(W)$ is closed under generalization,
so is $t(s^{-1}(W))$, see
(Morphisms, Lemma \ref{morphisms-lemma-generalizations-lift-flat} and
Topology, Lemma \ref{topology-lemma-lift-specializations-images}).
Pick a quasi-compact open $W' \subset U$ containing $t(s^{-1}(W))$. By
Properties, Lemma \ref{properties-lemma-quasi-compact-quasi-separated-spectral}
we see that $W'$ is a spectral space (here we use that $U$ is quasi-separated).
Then the lemma follows from
Topology, Lemma \ref{topology-lemma-make-spectral-space}
applied to $t(s^{-1}(W)) \subset W'$.
\end{proof}

\begin{lemma}
\label{lemma-second-observation}
Assumptions and notation as in Lemma \ref{lemma-first-observation}.
There exists an $R$-invariant open $V \subset U$ and a quasi-compact
open $W'$ such that $W \subset V \subset W' \subset U$.
\end{lemma}

\begin{proof}
Set $E = t(s^{-1}(W))$. Recall that $E$ is set-theoretically $R$-invariant
(Lemma \ref{lemma-constructing-invariant-opens}).
By Lemma \ref{lemma-first-observation} there exists a quasi-compact
open $W'$ containing $E$. Let $Z = U \setminus W'$ and consider
$T = t(s^{-1}(Z))$. Observe that $Z \subset T$ and that
$E \cap T = \emptyset$ because $s^{-1}(E) = t^{-1}(E)$ is disjoint
from $s^{-1}(Z)$. Since $T$ is the image of the closed subset
$s^{-1}(Z) \subset R$ under the quasi-compact morphism $t : R \to U$
we see that any point $\xi$ in the closure $\overline{T}$
is the specialization of a point of $T$, see
Morphisms, Lemma \ref{morphisms-lemma-reach-points-scheme-theoretic-image} (and
Morphisms, Lemma \ref{morphisms-lemma-quasi-compact-scheme-theoretic-image}
to see that the scheme theoretic image is the closure of the image).
Say $\xi' \leadsto \xi$ with $\xi' \in T$. Suppose that $r \in R$ and
$s(r) = \xi$. Since $s$ is flat we can find a specialization $r' \leadsto r$
in $R$ such that $s(r') = \xi'$
(Morphisms, Lemma \ref{morphisms-lemma-generalizations-lift-flat}).
Then $t(r') \leadsto t(r)$. We conclude that $t(r') \in T$ as $T$
is set-theoretically invariant by
Lemma \ref{lemma-constructing-invariant-opens}.
Thus $\overline{T}$ is a set-theoretically $R$-invariant closed subset
and $V = U \setminus \overline{T}$ is the open we are
looking for. It is contained in $W'$ which finishes the proof.
\end{proof}




\section{Quotient sheaves}
\label{section-quotient-sheaves}

\noindent
Let $\tau \in \{Zariski, \etale, fppf, smooth, syntomic\}$.
Let $S$ be a scheme.
Let $j : R \to U \times_S U$ be a pre-relation over $S$.
Say $U, R, S$ are objects of a $\tau$-site $\Sch_\tau$
(see Topologies, Section \ref{topologies-section-procedure}).
Then we can consider the functors
$$
h_U, h_R :
(\Sch/S)_\tau^{opp}
\longrightarrow
\textit{Sets}.
$$
These are sheaves, see
Descent, Lemma \ref{descent-lemma-fpqc-universal-effective-epimorphisms}.
The morphism $j$ induces a map $j : h_R \to h_U \times h_U$.
For each object $T \in \Ob((\Sch/S)_\tau)$
we can take the equivalence relation $\sim_T$ generated by
$j(T) : R(T) \to U(T) \times U(T)$ and consider the quotient.
Hence we get a presheaf
\begin{equation}
\label{equation-quotient-presheaf}
(\Sch/S)_\tau^{opp}
\longrightarrow
\textit{Sets}, \quad
T \longmapsto U(T)/\sim_T
\end{equation}

\begin{definition}
\label{definition-quotient-sheaf}
Let $\tau$, $S$, and the pre-relation $j : R \to U \times_S U$ be as above.
In this setting the {\it quotient sheaf $U/R$} associated
to $j$ is the sheafification of the presheaf
(\ref{equation-quotient-presheaf}) in the $\tau$-topology.
If $j : R \to U \times_S U$ comes from the action of a group scheme
$G/S$ on $U$ as in Lemma \ref{lemma-groupoid-from-action} then we
sometimes denote the quotient sheaf $U/G$.
\end{definition}

\noindent
This means exactly that the diagram
$$
\xymatrix{
h_R \ar@<1ex>[r] \ar@<-1ex>[r] &
h_U \ar[r] &
U/R
}
$$
is a coequalizer diagram in the category of sheaves of sets
on $(\Sch/S)_\tau$. Using the Yoneda embedding we
may view $(\Sch/S)_\tau$ as a full subcategory of
sheaves on $(\Sch/S)_\tau$ and hence identify schemes
with representable functors. Using this abuse of notation
we will often depict the diagram above simply
$$
\xymatrix{
R \ar@<1ex>[r]^s \ar@<-1ex>[r]_t &
U \ar[r] &
U/R
}
$$
We will mostly work with the fppf topology when considering
quotient sheaves of groupoids/equivalence relations.

\begin{definition}
\label{definition-representable-quotient}
In the situation of Definition \ref{definition-quotient-sheaf}.
We say that the pre-relation $j$ has a
{\it representable quotient} if the sheaf $U/R$ is representable.
We will say a groupoid $(U, R, s, t, c)$ has a
{\it representable quotient}
if the quotient $U/R$ with $j = (t, s)$ is representable.
\end{definition}

\noindent
The following lemma characterizes schemes $M$ representing the quotient.
It applies for example if $\tau = fppf$, $U \to M$ is flat,
of finite presentation and surjective, and $R \cong U \times_M U$.

\begin{lemma}
\label{lemma-criterion-quotient-representable}
In the situation of Definition \ref{definition-quotient-sheaf}.
Assume there is a scheme $M$, and a morphism $U \to M$ such that
\begin{enumerate}
\item the morphism $U \to M$ equalizes $s, t$,
\item the morphism $U \to M$ induces a surjection of sheaves
$h_U \to h_M$ in the $\tau$-topology, and
\item the induced map $(t, s) : R \to U \times_M U$ induces a
surjection of sheaves $h_R \to h_{U \times_M U}$ in the $\tau$-topology.
\end{enumerate}
In this case $M$ represents the quotient sheaf $U/R$.
\end{lemma}

\begin{proof}
Condition (1) says that $h_U \to h_M$ factors through $U/R$.
Condition (2) says that $U/R \to h_M$ is surjective as a map of sheaves.
Condition (3) says that $U/R \to h_M$ is injective as a map of sheaves.
Hence the lemma follows.
\end{proof}

\noindent
The following lemma is wrong if we do not require $j$ to be a
pre-equivalence relation (but just a pre-relation say).

\begin{lemma}
\label{lemma-quotient-pre-equivalence}
Let $\tau \in \{Zariski, \etale, fppf, smooth, syntomic\}$.
Let $S$ be a scheme.
Let $j : R \to U \times_S U$ be a pre-equivalence relation over $S$.
Assume $U, R, S$ are objects of a $\tau$-site $\Sch_\tau$.
For $T \in \Ob((\Sch/S)_\tau)$ and
$a, b \in U(T)$ the following are equivalent:
\begin{enumerate}
\item $a$ and $b$ map to the same element of $(U/R)(T)$, and
\item there exists a $\tau$-covering $\{f_i : T_i \to T\}$ of $T$
and morphisms $r_i : T_i \to R$ such that
$a \circ f_i = s \circ r_i$ and $b \circ f_i = t \circ r_i$.
\end{enumerate}
In other words, in this case the map of $\tau$-sheaves
$$
h_R \longrightarrow h_U \times_{U/R} h_U
$$
is surjective.
\end{lemma}

\begin{proof}
Omitted. Hint: The reason this works is that the presheaf
(\ref{equation-quotient-presheaf}) in this case is really given
by $T \mapsto U(T)/j(R(T))$ as $j(R(T)) \subset U(T) \times U(T)$
is an equivalence relation, see
Definition \ref{definition-equivalence-relation}.
\end{proof}

\begin{lemma}
\label{lemma-quotient-pre-equivalence-relation-restrict}
Let $\tau \in \{Zariski, \etale, fppf, smooth, syntomic\}$.
Let $S$ be a scheme.
Let $j : R \to U \times_S U$ be a pre-equivalence relation over $S$
and $g : U' \to U$ a morphism of schemes over $S$.
Let $j' : R' \to U' \times_S U'$ be the restriction of $j$ to $U'$.
Assume  $U, U', R, S$ are objects of a $\tau$-site $\Sch_\tau$.
The map of quotient sheaves
$$
U'/R' \longrightarrow U/R
$$
is injective. If $g$ defines a surjection $h_{U'} \to h_U$ of sheaves
in the $\tau$-topology (for example if $\{g : U' \to U\}$ is a
$\tau$-covering), then $U'/R' \to U/R$ is an isomorphism.
\end{lemma}

\begin{proof}
Suppose $\xi, \xi' \in (U'/R')(T)$ are sections which
map to the same section of $U/R$.
Then we can find a $\tau$-covering $\mathcal{T} = \{T_i \to T\}$ of $T$
such that $\xi|_{T_i}, \xi'|_{T_i}$ are given by $a_i, a_i' \in U'(T_i)$. By
Lemma \ref{lemma-quotient-pre-equivalence}
and the axioms of a site we may after refining
$\mathcal{T}$ assume there exist morphisms $r_i : T_i \to R$
such that $g \circ a_i = s \circ r_i$, $g \circ a_i' = t \circ r_i$.
Since by construction
$R' = R \times_{U \times_S U} (U' \times_S U')$
we see that $(r_i, (a_i, a_i')) \in R'(T_i)$ and this
shows that $a_i$ and $a_i'$ define the same section
of $U'/R'$ over $T_i$. By the sheaf condition this implies
$\xi = \xi'$.

\medskip\noindent
If $h_{U'} \to h_U$ is a surjection
of sheaves, then of course $U'/R' \to U/R$ is surjective also.
If $\{g : U' \to U\}$ is a $\tau$-covering, then
the map of sheaves $h_{U'} \to h_U$ is surjective, see
Sites, Lemma \ref{sites-lemma-covering-surjective-after-sheafification}.
Hence $U'/R' \to U/R$ is surjective also in this case.
\end{proof}

\begin{lemma}
\label{lemma-quotient-groupoid-restrict}
Let $\tau \in \{Zariski, \etale, fppf, smooth, syntomic\}$.
Let $S$ be a scheme.
Let $(U, R, s, t, c)$ be a groupoid scheme over $S$.
Let $g : U' \to U$ a morphism of schemes over $S$.
Let $(U', R', s', t', c')$ be the restriction of $(U, R, s, t, c)$ to $U'$.
Assume  $U, U', R, S$ are objects of a $\tau$-site $\Sch_\tau$.
The map of quotient sheaves
$$
U'/R' \longrightarrow U/R
$$
is injective. If the composition
$$
\xymatrix{
U' \times_{g, U, t} R \ar[r]_-{\text{pr}_1} \ar@/^3ex/[rr]^h
& R \ar[r]_s & U
}
$$
defines a surjection of sheaves in the $\tau$-topology  then
the map is bijective. This holds for example if
$\{h : U' \times_{g, U, t} R \to U\}$ is a $\tau$-covering, or
if $U' \to U$ defines a surjection of sheaves in the $\tau$-topology, or if
$\{g : U' \to U\}$ is a covering in the $\tau$-topology.
\end{lemma}

\begin{proof}
Injectivity follows on combining
Lemmas \ref{lemma-groupoid-pre-equivalence} and
\ref{lemma-quotient-pre-equivalence-relation-restrict}.
To see surjectivity (see
Sites, Section \ref{sites-section-sheaves-injective}
for a characterization of surjective maps of sheaves) we argue as follows.
Suppose that $T$ is a scheme and $\sigma \in U/R(T)$.
There exists a covering $\{T_i \to T\}$ such that $\sigma|_{T_i}$
is the image of some element $f_i \in U(T_i)$. Hence we
may assume that $\sigma$ is the image of $f \in U(T)$.
By the assumption that $h$ is a surjection of sheaves, we
can find a $\tau$-covering $\{\varphi_i : T_i \to T\}$ and morphisms
$f_i : T_i \to U' \times_{g, U, t} R$ such that
$f \circ \varphi_i = h \circ f_i$. Denote
$f'_i = \text{pr}_0 \circ f_i : T_i \to U'$. Then we see that
$f'_i \in U'(T_i)$ maps to $g \circ f'_i \in U(T_i)$ and
that $g \circ f'_i \sim_{T_i} h \circ f_i = f \circ \varphi_i$
notation as in (\ref{equation-quotient-presheaf}). Namely, the
element of $R(T_i)$ giving the relation is $\text{pr}_1 \circ f_i$.
This means that the restriction
of $\sigma$ to $T_i$ is in the image of $U'/R'(T_i) \to U/R(T_i)$
as desired.

\medskip\noindent
If $\{h\}$ is a $\tau$-covering, then it induces a surjection of sheaves, see
Sites, Lemma \ref{sites-lemma-covering-surjective-after-sheafification}.
If $U' \to U$ is surjective, then also $h$ is surjective as $s$ has a section
(namely the neutral element $e$ of the groupoid scheme).
\end{proof}

\begin{lemma}
\label{lemma-criterion-fibre-product}
Let $S$ be a scheme. Let $f : (U, R, j) \to (U', R', j')$ be a morphism
between equivalence relations over $S$. Assume that
$$
\xymatrix{
R \ar[d]_s \ar[r]_f & R' \ar[d]^{s'} \\
U \ar[r]^f & U'
}
$$
is cartesian. For any
$\tau \in  \{Zariski, \etale, fppf, smooth, syntomic\}$
the diagram
$$
\xymatrix{
U \ar[d] \ar[r] & U/R \ar[d]^f \\
U' \ar[r] & U'/R'
}
$$
is a fibre product square of $\tau$-sheaves.
\end{lemma}

\begin{proof}
By Lemma \ref{lemma-quotient-pre-equivalence} the quotient sheaves
have a simple description which we will use below without further mention.
We first show that
$$
U \longrightarrow U' \times_{U'/R'} U/R
$$
is injective. Namely, assume $a, b \in U(T)$ map to the same element
on the right hand side. Then $f(a) = f(b)$. After replacing $T$ by the
members of a $\tau$-covering we may assume that there exists an
$r \in R(T)$ such that $a = s(r)$ and $b = t(r)$. Then $r' = f(r)$
is a $T$-valued point of $R'$ with $s'(r') = t'(r')$. Hence
$r' = e'(f(a))$ (where $e'$ is the identity of the groupoid
scheme associated to $j'$, see Lemma \ref{lemma-equivalence-groupoid}).
Because the first diagram of the lemma is cartesian this implies
that $r$ has to equal $e(a)$. Thus $a = b$.

\medskip\noindent
Finally, we show that the displayed arrow is surjective. Let
$T$ be a scheme over $S$ and let $(a', \overline{b})$ be a section
of the sheaf $U' \times_{U'/R'} U/R$ over $T$. After replacing $T$
by the members of a $\tau$-covering we may assume that $\overline{b}$
is the class of an element $b \in U(T)$. After replacing $T$
by the members of a $\tau$-covering we may assume that there exists
an $r' \in R'(T)$ such that $a' = t(r')$ and $s'(r') = f(b)$.
Because the first diagram of the lemma is cartesian we can find
$r \in R(T)$ such that $s(r) = b$ and $f(r) = r'$. Then it is clear
that $a = t(r) \in U(T)$ is a section which maps to
$(a', \overline{b})$.
\end{proof}








\section{Descent in terms of groupoids}
\label{section-groupoids-descent}

\noindent
Cartesian morphisms are defined as follows.

\begin{definition}
\label{definition-cartesian-morphism}
Let $S$ be a scheme. Let $f : (U', R', s', t', c') \to (U, R, s, t, c)$ be
a morphism of groupoid schemes over $S$. We say $f$ is {\it cartesian}, or
that {\it $(U', R', s', t', c')$ is cartesian over $(U, R, s, t, c)$},
if the diagram
$$
\xymatrix{
R' \ar[r]_f \ar[d]_{s'} & R \ar[d]^s \\
U' \ar[r]^f & U
}
$$
is a fibre square in the category of schemes. A {\it morphism of groupoid
schemes cartesian over $(U, R, s, t, c)$} is a morphism of groupoid
schemes compatible with the structure morphisms towards $(U, R, s, t, c)$.
\end{definition}

\noindent
Cartesian morphisms are related to descent data. First we prove a general
lemma describing the category of cartesian groupoid schemes over a
fixed groupoid scheme.

\begin{lemma}
\label{lemma-characterize-cartesian-schemes}
Let $S$ be a scheme. Let $(U, R, s, t, c)$ be a groupoid scheme over $S$.
The category of groupoid schemes cartesian over $(U, R, s, t, c)$
is equivalent to the category of pairs $(V, \varphi)$ where $V$ is a
scheme over $U$ and
$$
\varphi :
V \times_{U, t} R
\longrightarrow
R \times_{s, U} V
$$
is an isomorphism over $R$ such that $e^*\varphi = \text{id}_V$ and such that
$$
c^*\varphi = \text{pr}_1^*\varphi \circ \text{pr}_0^*\varphi
$$
as morphisms of schemes over $R \times_{s, U, t} R$.
\end{lemma}

\begin{proof}
The pullback notation in the lemma signifies base change. The displayed
formula makes sense because
$$
(R \times_{s, U, t} R) \times_{\text{pr}_1, R, \text{pr}_1} (V \times_{U, t} R)
=
(R \times_{s, U, t} R) \times_{\text{pr}_0, R, \text{pr}_0} (R \times_{s, U} V)
$$
as schemes over $R \times_{s, U, t} R$.

\medskip\noindent
Given $(V, \varphi)$ we set $U' = V$ and $R' = V \times_{U, t} R$.
We set $t' : R' \to U'$ equal to the projection $V \times_{U, t} R \to V$.
We set $s'$ equal to $\varphi$ followed by the projection
$R \times_{s, U} V \to V$. We set $c'$ equal to the composition
\begin{align*}
R' \times_{s', U', t'} R'
& \xrightarrow{\varphi, 1}
(R \times_{s, U} V) \times_V (V \times_{U, t} R) \\
& \xrightarrow{}
R \times_{s, U} V \times_{U, t} R \\
& \xrightarrow{\varphi^{-1}, 1}
V \times_{U, t} (R \times_{s, U, t} R) \\
& \xrightarrow{1, c}
V \times_{U, t} R = R'
\end{align*}
A computation, which we omit shows that we obtain a groupoid scheme
over $(U, R, s, t, c)$. It is clear that this groupoid scheme is
cartesian over $(U, R, s, t, c)$.

\medskip\noindent
Conversely, given $f : (U', R', s', t', c') \to (U, R, s, t, c)$
cartesian then the morphisms
$$
U' \times_{U, t} R \xleftarrow{t', f} R' \xrightarrow{f, s'} R \times_{s, U} U'
$$
are isomorphisms and we can set $V = U'$ and $\varphi$ equal to the
composition $(f, s') \circ (t', f)^{-1}$. We omit the proof that
$\varphi$ satisfies the conditions in the lemma. We omit the proof that
these constructions are mutually inverse.
\end{proof}

\noindent
Let $S$ be a scheme. Let $f : X \to Y$ be a morphism of schemes over $S$. Then
we obtain a groupoid scheme $(X, X \times_Y X, \text{pr}_1, \text{pr}_0, c)$
over $S$. Namely, $j : X \times_Y X \to X \times_S X$ is an equivalence
relation and we can take the associated groupoid, see
Lemma \ref{lemma-equivalence-groupoid}.

\begin{lemma}
\label{lemma-cartesian-equivalent-descent-datum}
Let $S$ be a scheme. Let $f : X \to Y$ be a morphism of schemes over $S$.
The construction of Lemma \ref{lemma-characterize-cartesian-schemes}
determines an equivalence
$$
\begin{matrix}
\text{category of groupoid schemes} \\
\text{cartesian over } (X, X \times_Y X, \ldots)
\end{matrix}
\longrightarrow
\begin{matrix}
\text{ category of descent data} \\
\text{ relative to } X/Y
\end{matrix}
$$
\end{lemma}

\begin{proof}
This is clear from
Lemma \ref{lemma-characterize-cartesian-schemes}
and the definition of descent data on schemes in
Descent, Definition \ref{descent-definition-descent-datum}.
\end{proof}







\section{Separation conditions}
\label{section-separation}

\noindent
This really means conditions on the morphism $j : R \to U \times_S U$
when given a groupoid $(U, R, s, t, c)$ over $S$. As in the previous
section we first formulate the corresponding diagram.

\begin{lemma}
\label{lemma-diagram-diagonal}
Let $S$ be a scheme.
Let $(U, R, s, t, c)$ be a groupoid over $S$.
Let $G \to U$ be the stabilizer group scheme.
The commutative diagram
$$
\xymatrix{
R \ar[d]^{\Delta_{R/U \times_S U}} \ar[rrr]_{f \mapsto (f, s(f))} & & &
R \times_{s, U} U \ar[d] \ar[r] & U \ar[d] \\
R \times_{(U \times_S U)} R \ar[rrr]^{(f, g) \mapsto (f, f^{-1} \circ g)} & & &
R \times_{s, U} G \ar[r] & G
}
$$
the two left horizontal arrows are isomorphisms
and the right square is a fibre product square.
\end{lemma}

\begin{proof}
Omitted.
Exercise in the definitions and the functorial point of
view in algebraic geometry.
\end{proof}

\begin{lemma}
\label{lemma-diagonal}
Let $S$ be a scheme.
Let $(U, R, s, t, c)$ be a groupoid over $S$.
Let $G \to U$ be the stabilizer group scheme.
\begin{enumerate}
\item The following are equivalent
\begin{enumerate}
\item $j : R \to U \times_S U$ is separated,
\item $G \to U$ is separated, and
\item $e : U \to G$ is a closed immersion.
\end{enumerate}
\item The following are equivalent
\begin{enumerate}
\item $j : R \to U \times_S U$ is quasi-separated,
\item $G \to U$ is quasi-separated, and
\item $e : U \to G$ is quasi-compact.
\end{enumerate}
\end{enumerate}
\end{lemma}

\begin{proof}
The group scheme $G \to U$ is the base change of $R \to U \times_S U$
by the diagonal morphism $U \to U \times_S U$, see
Lemma \ref{lemma-groupoid-stabilizer}. Hence if
$j$ is separated (resp.\ quasi-separated),
then $G \to U$ is separated (resp.\ quasi-separated).
(See Schemes, Lemma
\ref{schemes-lemma-separated-permanence}).
Thus (a) $\Rightarrow$ (b) in both (1) and (2).

\medskip\noindent
If $G \to U$ is separated (resp.\ quasi-separated), then the morphism
$U \to G$, as a section of the structure morphism $G \to U$ is a closed
immersion (resp.\ quasi-compact), see
Schemes, Lemma \ref{schemes-lemma-section-immersion}.
Thus (b) $\Rightarrow$ (a) in both (1) and (2).

\medskip\noindent
By the result of
Lemma \ref{lemma-diagram-diagonal}
(and Schemes, Lemmas \ref{schemes-lemma-base-change-immersion}
and \ref{schemes-lemma-quasi-compact-preserved-base-change})
we see that if $e$ is a closed immersion (resp.\ quasi-compact)
$\Delta_{R/U \times_S U}$ is a closed
immersion (resp.\ quasi-compact).
Thus (c) $\Rightarrow$ (a) in both (1) and (2).
\end{proof}









\section{Finite flat groupoids, affine case}
\label{section-finite-flat}

\noindent
Let $S$ be a scheme.
Let $(U, R, s, t, c)$ be a groupoid scheme over $S$.
Assume $U = \Spec(A)$, and $R = \Spec(B)$ are affine.
In this case we get two ring maps
$s^\sharp, t^\sharp : A \longrightarrow B$.
Let $C$ be the equalizer of $s^\sharp$ and $t^\sharp$. In a formula
\begin{equation}
\label{equation-invariants}
C = \{a \in A \mid t^\sharp(a) = s^\sharp(a) \}.
\end{equation}
We will sometimes call this the {\it ring of $R$-invariant functions on $U$}.
What properties does $M = \Spec(C)$ have? The first observation is
that the diagram
$$
\xymatrix{
R \ar[r]_s \ar[d]_t & U \ar[d] \\
U \ar[r] & M
}
$$
is commutative, i.e., the morphism $U \to M$ equalizes $s, t$.
Moreover, if $T$ is any affine scheme, and if $U \to T$ is
a morphism which equalizes $s, t$, then $U \to T$ factors through $U \to M$.
In other words, $U \to M$ is a coequalizer in the category of affine schemes.

\medskip\noindent
We would like to find conditions that guarantee the morphism $U \to M$ is
really a ``quotient'' in the category of schemes. We will discuss this at
length elsewhere (insert future reference here); here we just discuss some
special cases. Namely, we will focus on the case where $s, t$ are finite
locally free.

\begin{example}
\label{example-quotient-projective-line}
Let $k$ be a field. Let $U = \text{GL}_{2, k}$. Let $B \subset \text{GL}_2$
be the closed subgroup scheme of upper triangular matrices.
Then the quotient sheaf $\text{GL}_{2, k}/B$ (in the Zariski, \'etale or
fppf topology, see Definition \ref{definition-quotient-sheaf}) is
representable by the projective line: $\mathbf{P}^1 = \text{GL}_{2, k}/B$.
(Details omitted.)
On the other hand, the ring of invariant functions in this case is just $k$.
Note that in this case the morphisms
$s, t : R = \text{GL}_{2, k} \times_k B \to \text{GL}_{2, k} = U$ are smooth
of relative dimension $3$.
\end{example}

\noindent
Recall that in Exercises, Exercises \ref{exercises-exercise-trace-det} and
\ref{exercises-exercise-trace-det-rings} we have defined the determinant
and the norm for finitely locally free modules and finite locally free ring
extensions. If $\varphi : A \to B$ is a finite locally free ring map, then
we will denote $\text{Norm}_\varphi(b) \in A$ the norm of $b \in B$. In the
case of a finite locally free morphism of schemes, the norm was constructed
in Divisors, Lemma \ref{divisors-lemma-finite-locally-free-has-norm}.

\begin{lemma}
\label{lemma-determinant-trick}
Let $S$ be a scheme. Let $(U, R, s, t, c)$ be a groupoid scheme over $S$.
Assume $U = \Spec(A)$ and $R = \Spec(B)$ are affine and
$s, t : R \to U$ finite locally free.
Let $C$ be as in (\ref{equation-invariants}).
Let $f \in A$. Then $\text{Norm}_{s^\sharp}(t^\sharp(f)) \in C$.
\end{lemma}

\begin{proof}
Consider the commutative diagram
$$
\xymatrix{
& U & \\
R \ar[d]_s \ar[ru]^t &
R \times_{s, U, t} R
\ar[l]^-{\text{pr}_0} \ar[d]^{\text{pr}_1} \ar[r]_-c &
R \ar[d]^s \ar[lu]_t \\
U & R \ar[l]_t \ar[r]^s & U
}
$$
of Lemma \ref{lemma-diagram}.
Think of $f \in \Gamma(U, \mathcal{O}_U)$. The commutativity of the
top part of the diagram shows that
$pr_0^\sharp(t^\sharp(f)) = c^\sharp(t^\sharp(f))$ as elements of
$\Gamma(R \times_{S, U, t} R, \mathcal{O})$.
Looking at the right lower cartesian square
the compatibility of the norm construction with base change shows that
$s^\sharp(\text{Norm}_{s^\sharp}(t^\sharp(f))) =
\text{Norm}_{\text{pr}_1}(c^\sharp(t^\sharp(f)))$.
Similarly we get
$t^\sharp(\text{Norm}_{s^\sharp}(t^\sharp(f))) =
\text{Norm}_{\text{pr}_1}(\text{pr}_0^\sharp(t^\sharp(f)))$.
Hence by the first equality of this proof we see that
$s^\sharp(\text{Norm}_{s^\sharp}(t^\sharp(f))) =
t^\sharp(\text{Norm}_{s^\sharp}(t^\sharp(f)))$ as desired.
\end{proof}

\begin{lemma}
\label{lemma-finite-locally-free-disjoint-free}
Let $S$ be a scheme. Let $(U, R, s, t, c)$ be a groupoid scheme over $S$.
Assume $s, t : R \to U$ finite locally free.
Then
$$
U = \coprod\nolimits_{r \geq 1} U_r
$$
is a disjoint union of $R$-invariant opens such that the restriction $R_r$ of
$R$ to $U_r$ has the property that $s, t : R_r \to U_r$ are finite locally
free of rank $r$.
\end{lemma}

\begin{proof}
By
Morphisms, Lemma \ref{morphisms-lemma-finite-locally-free}
there exists a decomposition
$U = \coprod\nolimits_{r \geq 0} U_r$
such that $s : s^{-1}(U_r) \to U_r$ is finite locally free of rank $r$.
As $s$ is surjective we see that $U_0 = \emptyset$.
Note that $u \in U_r \Leftrightarrow$ if and only if the scheme theoretic fibre
$s^{-1}(u)$ has degree $r$ over $\kappa(u)$. Now, if $z \in R$ with $s(z) = u$
and $t(z) = u'$ then using notation as in Lemma \ref{lemma-diagram}
$$
\text{pr}_1^{-1}(z) \to \Spec(\kappa(z))
$$
is the base change of both
$s^{-1}(u) \to \Spec(\kappa(u))$ and $s^{-1}(u') \to \Spec(\kappa(u'))$
by the lemma cited. Hence $u \in U_r \Leftrightarrow u' \in U_r$,
in other words, the open subsets $U_r$ are $R$-invariant.
In particular the restriction of $R$ to $U_r$ is just
$s^{-1}(U_r)$ and $s : R_r \to U_r$ is finite locally free of rank $r$.
As $t : R_r \to U_r$ is isomorphic to $s$ by the inverse of $R_r$
we see that it has also rank $r$.
\end{proof}

\begin{lemma}
\label{lemma-integral-over-invariants}
Let $S$ be a scheme. Let $(U, R, s, t, c)$ be a groupoid scheme over $S$.
Assume $U = \Spec(A)$ and $R = \Spec(B)$ are affine and
$s, t : R \to U$ finite locally free.
Let $C \subset A$ be as in (\ref{equation-invariants}).
Then $A$ is integral over $C$.
\end{lemma}

\begin{proof}
First, by Lemma \ref{lemma-finite-locally-free-disjoint-free}
we know that $(U, R, s, t, c)$ is a disjoint union
of groupoid schemes $(U_r, R_r, s, t, c)$ such that each $s, t : R_r \to U_r$
has constant rank $r$. As $U$ is quasi-compact, we have $U_r = \emptyset$ for
almost all $r$. It suffices to prove the lemma for each $(U_r, R_r, s, t, c)$
and hence we may assume that $s, t$ are finite locally free of rank $r$.

\medskip\noindent
Assume that $s, t$ are finite locally free of rank $r$.
Let $f \in A$. Consider the element $x - f \in A[x]$, where we think
of $x$ as the coordinate on $\mathbf{A}^1$.
Since
$$
(U \times \mathbf{A}^1, R \times \mathbf{A}^1,
s \times \text{id}_{\mathbf{A}^1},
t \times \text{id}_{\mathbf{A}^1},
c \times \text{id}_{\mathbf{A}^1})
$$
is also a groupoid scheme with finite source and target, we may apply
Lemma \ref{lemma-determinant-trick} to it and we see that
$P(x) = \text{Norm}_{s^\sharp}(t^\sharp(x - f))$
is an element of $C[x]$. Because $s^\sharp : A \to B$ is finite locally
free of rank $r$ we see that $P$ is monic of degree $r$.
Moreover $P(f) = 0$ by Cayley-Hamilton
(Algebra, Lemma \ref{algebra-lemma-charpoly}).
\end{proof}

\begin{lemma}
\label{lemma-invariants-base-change}
Let $S$ be a scheme. Let $(U, R, s, t, c)$ be a groupoid scheme over $S$.
Assume $U = \Spec(A)$ and $R = \Spec(B)$ are affine and
$s, t : R \to U$ finite locally free. Let $C \subset A$ be as in
(\ref{equation-invariants}). Let $C \to C'$ be a ring map, and set
$U' = \Spec(A \otimes_C C')$,
$R' = \Spec(B \otimes_C C')$.
Then
\begin{enumerate}
\item The maps $s, t, c$ induce maps $s', t', c'$ such that
$(U', R', s', t', c')$ is a groupoid scheme. Let $C^1 \subset A'$
be the $R'$-invariant functions on $U'$.
\item The canonical map $\varphi : C' \to C^1$ satisfies
\begin{enumerate}
\item for every $f \in C^1$ there exists an $n > 0$ and a
polynomial $P \in C'[x]$ whose image in $C^1[x]$ is
$(x - f)^n$, and
\item for every $f \in \Ker(\varphi)$ there exists
an $n > 0$ such that $f^n = 0$.
\end{enumerate}
\item If $C \to C'$ is flat then $\varphi$ is an isomorphism.
\end{enumerate}
\end{lemma}

\begin{proof}
The proof of part (1) is omitted. Let us denote $A' = A \otimes_C C'$ and
$B' = B \otimes_C C'$. Then we have
$$
C^1
= \{a \in A' \mid (t')^\sharp(a) = (s')^\sharp(a) \}
= \{a \in A \otimes_C C' \mid t^\sharp \otimes 1(a) = s^\sharp \otimes 1(a) \}.
$$
In other words, $C^1$ is the kernel of the difference map
$(t^\sharp - s^\sharp) \otimes 1$ which is just the base change
of the $C$-linear map $t^\sharp - s^\sharp : A \to B$ by $C \to C'$.
Hence (3) follows.

\medskip\noindent
Proof of part (2)(b). Since $C \to A$ is integral
(Lemma \ref{lemma-integral-over-invariants}) and injective we see that
$\Spec(A) \to \Spec(C)$ is surjective, see
Algebra, Lemma \ref{algebra-lemma-integral-overring-surjective}.
Thus also $\Spec(A') \to \Spec(C')$ is surjective
as a base change of a surjective morphism
(Morphisms, Lemma \ref{morphisms-lemma-base-change-surjective}).
Hence $\Spec(C^1) \to \Spec(C')$ is surjective also.
This implies that the kernel of $\varphi$ is contained in the
radical of the ring $C'$, i.e., (2)(b) holds.

\medskip\noindent
Proof of part (2)(a). By Lemma \ref{lemma-finite-locally-free-disjoint-free}
our groupoid scheme $(U, R, s, t, c)$ decomposes as a finite disjoint union
of groupoid schemes $(U_r, R_r, s, t, c)$ such that $s, t : R_r \to U_r$
are finite locally free of rank $r$. Pulling back by $U' = \Spec(C') \to U$
we obtain a similar decomposition of $U'$ and $U^1 = \Spec(C^1)$.
We will show in the next paragraph that (2)(a) holds for the corresponding
system of rings $A_r, B_r, C_r, C'_r, C^1_r$ with $n = r$.
Then given $f \in C^1$ let $P_r \in C_r[x]$ be the polynomial
whose image in $C^1_r[x]$ is the image of $(x - f)^r$.
Choosing a sufficiently divisible integer $n$ we see that
there is a polynomial $P \in C'[x]$ whose image in $C^1[x]$ is
$(x - f)^n$; namely, we take $P$ to be the unique element of
$C'[x]$ whose image in $C'_r[x]$ is $P_r^{n/r}$.

\medskip\noindent
In this paragraph we prove (2)(a) in case the ring maps
$s^\sharp, t^\sharp : A \to B$ are finite locally free of a fixed rank $r$.
Let $f \in C^1 \subset A' = A \otimes_C C'$. Choose a flat
$C$-algebra $D$ and a surjection $D \to C'$. Choose a lift
$g \in A \otimes_C D$ of $f$.
Consider the polynomial
$$
P = \text{Norm}_{s^\sharp \otimes 1}(t^\sharp \otimes 1(x - g))
$$
in $(A \otimes_C D)[x]$. By Lemma \ref{lemma-determinant-trick}
and part (3) of the current lemma the coefficients of $P$ are in $D$
(compare with the proof of Lemma \ref{lemma-integral-over-invariants}).
On the other hand, the image of $P$ in $(A \otimes_C C')[x]$ is
$(x - f)^r$ because $t^\sharp \otimes 1(x - f) = s^\sharp(x - f)$
and $s^\sharp$ is finite locally free of rank $r$.
This proves what we want with $P$ as in the statement (2)(a)
given by the image of our $P$ under the map $D[x] \to C'[x]$.
\end{proof}

\begin{lemma}
\label{lemma-points}
Let $S$ be a scheme. Let $(U, R, s, t, c)$ be a groupoid scheme over $S$.
Assume $U = \Spec(A)$ and $R = \Spec(B)$ are affine and
$s, t : R \to U$ finite locally free. Let $C \subset A$ be as in
(\ref{equation-invariants}). Then $U \to M = \Spec(C)$ has
the following properties:
\begin{enumerate}
\item the map on points $|U| \to |M|$ is surjective and
$u_0, u_1 \in |U|$ map to the same point if and only if
there exists a $r \in |R|$ with $t(r) = u_0$ and $s(r) = u_1$, in
a formula
$$
|M| = |U|/|R|
$$
\item for any algebraically closed field $k$ we have
$$
M(k) = U(k)/R(k)
$$
\end{enumerate}
\end{lemma}

\begin{proof}
Let $k$ be an algebraically closed field.
Since $C \to A$ is integral (Lemma \ref{lemma-integral-over-invariants})
and injective we see that
$\Spec(A) \to \Spec(C)$ is surjective, see
Algebra, Lemma \ref{algebra-lemma-integral-overring-surjective}.
Thus $|U| \to |M|$ is surjective.
Let $C \to k$ be a ring map. Since surjective morphisms are
preserved under base change
(Morphisms, Lemma \ref{morphisms-lemma-base-change-surjective}) we see that
$A \otimes_C k$ is not zero. Now $k \subset A \otimes_C k$ is a
nonzero integral extension. Hence any residue field of $A \otimes_C k$
is an algebraic extension of $k$, hence equal to $k$. Thus we see that
$U(k) \to M(k)$ is surjective.

\medskip\noindent
Let $a_0, a_1 : A \to k$ be ring maps. If there exists a ring map
$b : B \to k$ such that $a_0 = b \circ t^\sharp$ and $a_1 = b \circ s^\sharp$
then we see that $a_0|_C = a_1|_C$ by definition.
Conversely, suppose that $a_0|_C = a_1|_C$. Let us name this algebra
map $c : C \to k$. Consider the diagram
$$
\xymatrix{
& &
B \ar@{-->}[lld] \\
k & &
A
\ar@<0.5ex>[ll]^{a_0}
\ar@<-0.5ex>[ll]_{a_1}
\ar@<1ex>[u]
\ar@<-1ex>[u] \\
& &
C \ar[u] \ar[llu]^c
}
$$
We are trying to construct the dotted arrow, and if we do then
part (2) follows, which in turn implies part (1).
Since $A \to B$ is finite and faithfully flat
there exist finitely many ring maps
$b_1, \ldots, b_n : B \to k$ such that $b_i \circ s^\sharp = a_1$.
If the dotted arrow does not exist, then we see that none of the
$a'_i = b_i \circ t^\sharp$, $i = 1, \ldots, n$ is equal to $a_0$.
Hence the maximal ideals
$$
\mathfrak m'_i = \Ker(a_i' \otimes 1 : A \otimes_C k \to k)
$$
of $A \otimes_C k$ are distinct from
$\mathfrak m = \Ker(a_0 \otimes 1 : A \otimes_C k \to k)$.
By Algebra, Lemma \ref{algebra-lemma-silly} we would get an element
$f \in A \otimes_C k$ with $f \in \mathfrak m$, but
$f \not \in \mathfrak m_i'$ for $i = 1, \ldots, n$.
Consider the norm
$$
g = \text{Norm}_{s^\sharp \otimes 1}(t^\sharp \otimes 1(f))
\in
A \otimes_C k
$$
By Lemma \ref{lemma-determinant-trick} this lies in the invariants
$C^1 \subset A \otimes_C k$ of the base change
groupoid (base change via the map $c : C \to k$). On the one hand,
$a_1(g) \in k^*$ since
the value of $t^\sharp(f)$ at all the points (which correspond to
$b_1, \ldots, b_n$) lying over $a_1$ is
invertible (insert future reference on property determinant here).
On the other hand, since $f \in \mathfrak m$, we see that
$f$ is not a unit, hence $t^\sharp(f)$ is not a unit
(as $t^\sharp \otimes 1$ is faithfully flat),
hence its norm is not a unit (insert future reference
on property determinant here). We conclude that $C^1$ contains
an element which is not nilpotent
and not a unit. We will now show that this leads to a contradiction.
Namely, apply Lemma \ref{lemma-invariants-base-change}
to the map $c : C \to C' = k$, then
we see that the map of $k$ into the invariants $C^1$ is injective
and moreover, that for any element $x \in C^1$ there exists an integer
$n > 0$ such that $x^n \in k$. Hence every element of $C^1$ is
either a unit or nilpotent.
\end{proof}

\begin{lemma}
\label{lemma-etale}
Let $S$ be a scheme. Let $f : (U', R', s', t') \to (U, R, s, t, c)$ be a
morphism of groupoid schemes over $S$.
\begin{enumerate}
\item $U$, $R$, $U'$, $R'$ are affine,
\item $s, t, s', t'$ are finite locally free,
\item the diagrams
$$
\xymatrix{
R' \ar[d]_{s'} \ar[r]_f & R \ar[d]^s \\
U' \ar[r]^f & U
}
\quad
\quad
\xymatrix{
R' \ar[d]_{t'} \ar[r]_f & R \ar[d]^t \\
U' \ar[r]^f & U
}
\quad
\quad
\xymatrix{
G' \ar[d] \ar[r]_f & G \ar[d] \\
U' \ar[r]^f & U
}
$$
are cartesian where $G$ and $G'$ are the stabilizer group schemes, and
\item $f : U' \to U$ is \'etale.
\end{enumerate}
Then the map $C \to C'$ from the $R$-invariant functions on $U$
to the $R'$-invariant functions on $U'$ is \'etale and
$U' = \Spec(C') \times_{\Spec(C)} U$.
\end{lemma}

\begin{proof}
Set $M = \Spec(C)$ and $M' = \Spec(C')$.
Write $U = \Spec(A)$, $U' = \Spec(A')$, $R = \Spec(B)$, and
$R' = \Spec(B')$. We will use the results of
Lemmas \ref{lemma-integral-over-invariants},
\ref{lemma-invariants-base-change}, and
\ref{lemma-points}
without further mention.

\medskip\noindent
Assume $C$ is a strictly henselian local ring. Let $p \in M$
be the closed point and let $p' \in M'$ map to $p$.
Claim: in this case there is a disjoint union decomposition
$(U', R', s', t', c') = (U, R, s, t, c) \amalg (U'', R'', s'', t'', c'')$
over $(U, R, s, t, c)$ such that for the corresponding
disjoint union decomposition $M' = M \amalg M''$ over $M$
the point $p'$ corresponds to $p \in M$.

\medskip\noindent
The claim implies the lemma. Suppose that $M_1 \to M$ is a flat morphism
of affine schemes. Then we can base change everything to $M_1$
without affecting the hypotheses (1) -- (4).
From Lemma \ref{lemma-invariants-base-change}
we see $M_1$, resp.\ $M_1'$ is the spectrum of the
$R_1$-invariant functions on $U_1$,
resp.\ the $R'_1$-invariant functions on $U'_1$.
Suppose that $p' \in M'$ maps to $p \in M$.
Let $M_1$ be the spectrum of the strict henselization of
$\mathcal{O}_{M, p}$ with closed point $p_1 \in M_1$.
Choose a point $p'_1 \in M'_1$ mapping to $p_1$ and $p'$.
From the claim we get
$$
(U'_1, R'_1, s'_1, t'_1, c'_1) =
(U_1, R_1, s_1, t_1, c_1) \amalg
(U''_1, R''_1, s''_1, t''_1, c''_1)
$$
and correspondingly $M'_1 = M_1 \amalg M''_1$ as a scheme over $M_1$.
Write $M_1 = \Spec(C_1)$ and write $C_1 = \colim C_i$ as a filtered
colimit of \'etale $C$-algebras. Set $M_i = \Spec(C_i)$.
The $M_1 = \lim M_i$ and similarly for the other schemes.
By Limits, Lemmas \ref{limits-lemma-descend-opens} and
\ref{limits-lemma-descend-isomorphism}
we can find an $i$ such that
$$
(U'_i, R'_i, s'_i, t'_i, c'_i) =
(U_i, R_i, s_i, t_i, c_i) \amalg
(U''_i, R''_i, s''_i, t''_i, c''_i)
$$
We conclude that $M'_i = M_i \amalg M''_i$. In particular
$M' \to M$ becomes \'etale at a point over $p'$ after an
\'etale base change. This implies that $M' \to M$ is \'etale at $p'$
(for example by Morphisms, Lemma
\ref{morphisms-lemma-set-points-where-fibres-etale}).
We will prove $U' \cong M' \times_M U$ after we prove the claim.

\medskip\noindent
Proof of the claim. Observe that $U_p$ and $U'_{p'}$ have finitely many points.
For $u \in U_p$ we have $\kappa(u)/\kappa(p)$ is algebraic,
hence $\kappa(u)$ is separably closed.
As $U' \to U$ is \'etale, we conclude the morphism $U'_{p'} \to U_p$
induces isomorphisms on residue field extensions.
Let $u' \in U'_{p'}$ with image $u \in U_p$.
By assumption (3) the morphism of scheme theoretic fibres
$(s')^{-1}(u') \to s^{-1}(u)$,
$(t')^{-1}(u') \to t^{-1}(u)$, and
$G'_{u'} \to G_u$ are isomorphisms. Observing that $U_p = t(s^{-1}(u))$
(set theoretically) we conclude that the points of $U'_{p'}$
surject onto the points of $U_p$.
Suppose that $u'_1$ and $u'_2$ are points of $U'_{p'}$ mapping
to the same point $u$ of $U_p$. Then there exists a point
$r' \in R'_{p'}$ with $s'(r') = u'_1$ and $t'(r') = u'_2$.
Consider the two towers of fields
$$
\kappa(r')/\kappa(u'_1)/\kappa(u)/\kappa(p) \quad
\kappa(r')/\kappa(u'_2)/\kappa(u)/\kappa(p)
$$
whose ``ends'' are the same as the two ``ends'' of the two towers
$$
\kappa(r')/\kappa(u'_1)/\kappa(p')/\kappa(p) \quad
\kappa(r')/\kappa(u'_2)/\kappa(p')/\kappa(p)
$$
These two induce the same maps $\kappa(p') \to \kappa(r')$ as
$(U'_{p'}, R'_{p'}, s', t', c')$ is a groupoid over $p'$.
Since $\kappa(u)/\kappa(p)$ is purely inseparable,
we conclude that the two induced maps
$\kappa(u) \to \kappa(r')$ are the same.
Therefore $r'$ maps to a point of the fibre $G_u$.
By assumption (3) we conclude that $r' \in (G')_{u'_1}$.
Namely, we may think of $G$ as a closed subscheme of $R$
viewed as a scheme over $U$ via $s$ and use that
the base change to $U'$ gives $G' \subset R'$.
In particular we have $u'_1 = u'_2$.
We conclude that $U'_{p'} \to U_p$ is a bijective
map on points inducing isomorphisms on residue fields.
It follows that $U'_{p'}$ is a finite set of closed points
(Algebra, Lemma \ref{algebra-lemma-finite-residue-extension-closed})
and hence $U'_{p'}$ is closed in $U'$.
Let $J' \subset A'$ be the radical ideal cutting out $U'_{p'}$
set theoretically.

\medskip\noindent
Second part proof of the claim.
Let $\mathfrak m \subset C$ be the maximal ideal.
Observe that $(A, \mathfrak m A)$ is a henselian pair by
More on Algebra, Lemma \ref{more-algebra-lemma-integral-over-henselian-pair}.
Let $J = \sqrt{\mathfrak m A}$.
Then $(A, J)$ is a henselian pair
(More on Algebra, Lemma \ref{more-algebra-lemma-change-ideal-henselian-pair})
and the \'etale ring map
$A \to A'$ induces an isomorphism $A/J \to A'/J'$
by our deliberations above.
We conclude that $A' = A \times A''$ by
More on Algebra, Lemma \ref{more-algebra-lemma-characterize-henselian-pair}.
Consider the corresponding disjoint union
decomposition $U' = U \amalg U''$. The open $(s')^{-1}(U)$ is the
set of points of $R'$ specializing to a point of $R'_{p'}$.
Similarly for $(t')^{-1}(U)$. Similarly we have
$(s')^{-1}(U'') = (t')^{-1}(U'')$ as this is the set of
points which do not specialize to $R'_{p'}$.
Hence we obtain a disjoint union decomposition
$$
(U', R', s', t', c') =
(U, R, s, t, c) \amalg
(U'', R'', s'', t'', c'')
$$
This immediately gives $M' = M \amalg M''$ and the proof of the claim
is complete.

\medskip\noindent
We still have to prove that the canonical map $U' \to M' \times_M U$
is an isomorphism. It is an \'etale morphism
(Morphisms, Lemma \ref{morphisms-lemma-etale-permanence}).
On the other hand, by base changing to strictly henselian local rings
(as in the third paragraph of the proof) and using the bijectivity
$U'_{p'} \to U_p$ esthablished in the course of the proof of the claim,
we see that $U' \to M' \times_M U$ is universally bijective
(some details omitted). However, a universally bijective
\'etale morphism is an isomorphism
(Descent, Lemma \ref{descent-lemma-universally-injective-etale-open-immersion})
and the proof is complete.
\end{proof}

\begin{lemma}
\label{lemma-basis}
Let $S$ be a scheme.
Let $(U, R, s, t, c)$ be a groupoid scheme over $S$.
Assume
\begin{enumerate}
\item $U = \Spec(A)$, and $R = \Spec(B)$ are affine, and
\item there exist elements $x_i \in A$, $i \in I$ such that
$B = \bigoplus_{i \in I} s^\sharp(A)t^\sharp(x_i)$.
\end{enumerate}
Then $A = \bigoplus_{i\in I} Cx_i$, and $B \cong A \otimes_C A$
where $C \subset A$ is the $R$-invariant
functions on $U$ as in (\ref{equation-invariants}).
\end{lemma}

\begin{proof}
During this proof we will write $s, t : A \to B$ instead of
$s^\sharp, t^\sharp$, and similarly $c : B \to B \otimes_{s, A, t} B$.
We write $p_0 : B \to B \otimes_{s, A, t} B$, $b \mapsto b \otimes 1$ and
$p_1 : B \to B \otimes_{s, A, t} B$, $b \mapsto 1 \otimes b$. By
Lemma \ref{lemma-diagram-pull}
and the definition of $C$ we have the following
commutative diagram
$$
\xymatrix{
B \otimes_{s, A, t} B &
B \ar@<-1ex>[l]_-c \ar@<1ex>[l]^-{p_0} &
A \ar[l]^t \\
B \ar[u]^{p_1} &
A \ar@<-1ex>[l]_s \ar@<1ex>[l]^t \ar[u]_s &
C \ar[u] \ar[l]
}
$$
Moreover the tow left squares are cocartesian in the category of rings, and
the top row is isomorphic to the diagram
$$
\xymatrix{
B \otimes_{t, A, t} B &
B \ar@<-1ex>[l]_-{p_1} \ar@<1ex>[l]^-{p_0} &
A \ar[l]^t
}
$$
which is an equalizer diagram according to
Descent, Lemma \ref{descent-lemma-ff-exact} because condition (2) implies
in particular that $s$ (and hence also then isomorphic arrow $t$)
is faithfully flat.
The lower row is an equalizer diagram by definition of $C$.
We can use the $x_i$ and get a commutative diagram
$$
\xymatrix{
B \otimes_{s, A, t} B &
B \ar@<-1ex>[l]_-c \ar@<1ex>[l]^-{p_0} &
A \ar[l]^t \\
\bigoplus_{i \in I} B x_i \ar[u]^{p_1} &
\bigoplus_{i \in I} A x_i \ar@<-1ex>[l]_s \ar@<1ex>[l]^t \ar[u]_s &
\bigoplus_{i \in I} C x_i \ar[u] \ar[l]
}
$$
where in the right vertical arrow we map $x_i$ to $x_i$,
in the middle vertical arrow we map $x_i$ to $t(x_i)$ and
in the left vertical arrow we map $x_i$ to
$c(t(x_i)) = t(x_i) \otimes 1 = p_0(t(x_i))$ (equality by the commutativity
of the top part of the diagram in Lemma \ref{lemma-diagram}). Then the diagram
commutes. Moreover the middle vertical arrow is an isomorphism
by assumption. Since the left two squares are cocartesian we
conclude that also the left vertical arrow is an isomorphism.
On the other hand, the horizontal rows are exact (i.e., they are
equalizers). Hence we conclude that also the right vertical arrow
is an isomorphism.
\end{proof}

\begin{proposition}
\label{proposition-finite-flat-equivalence}
Let $S$ be a scheme.
Let $(U, R, s, t, c)$ be a groupoid scheme over $S$.
Assume
\begin{enumerate}
\item $U = \Spec(A)$, and $R = \Spec(B)$ are affine,
\item $s, t : R \to U$ finite locally free, and
\item $j = (t, s)$ is an equivalence.
\end{enumerate}
In this case, let $C \subset A$ be as in
(\ref{equation-invariants}). Then $U \to M = \Spec(C)$
is finite locally free and $R = U \times_M U$.
Moreover, $M$ represents the quotient sheaf $U/R$
in the fppf topology (see Definition \ref{definition-quotient-sheaf}).
\end{proposition}

\begin{proof}
During this proof we use the notation $s, t : A \to B$
instead of the notation $s^\sharp, t^\sharp$.
By Lemma \ref{lemma-criterion-quotient-representable}
it suffices to show that $C \to A$ is finite locally free
and that the map
$$
t \otimes s : A \otimes_C A \longrightarrow B
$$
is an isomorphism. First, note that $j$ is a monomorphism, and
also finite (since already $s$ and $t$ are finite). Hence we see
that $j$ is a closed immersion by
Morphisms, Lemma \ref{morphisms-lemma-finite-monomorphism-closed}.
Hence $A \otimes_C A \to B$ is surjective.

\medskip\noindent
We will perform base change by flat ring maps $C \to C'$ as in
Lemma \ref{lemma-invariants-base-change}, and we will use that
formation of invariants commutes with flat base change, see
part (3) of the lemma cited.
We will show below that for every prime $\mathfrak p \subset C$, there exists
a local flat ring map $C_{\mathfrak p} \to C_{\mathfrak p}'$
such that the result holds after a base change to $C_{\mathfrak p}'$.
This implies immediately
that $A \otimes_C A \to B$ is injective (use
Algebra, Lemma \ref{algebra-lemma-characterize-zero-local}).
It also implies that $C \to A$ is flat, by combining
Algebra, Lemmas \ref{algebra-lemma-local-flat-ff},
\ref{algebra-lemma-flat-localization}, and
\ref{algebra-lemma-flatness-descends}. Then since $U \to \Spec(C)$
is surjective also (Lemma \ref{lemma-points}) we conclude that $C \to A$
is faithfully flat. Then the isomorphism $B \cong A \otimes_C A$
implies that $A$ is a finitely presented $C$-module, see
Algebra, Lemma \ref{algebra-lemma-descend-properties-modules}.
Hence $A$ is finite locally free over $C$, see
Algebra, Lemma \ref{algebra-lemma-finite-projective}.

\medskip\noindent
By Lemma \ref{lemma-finite-locally-free-disjoint-free}
we know that $A$ is a finite
product of rings $A_r$ and $B$ is a finite product of rings $B_r$
such that the groupoid scheme decomposes accordingly (see the proof
of Lemma \ref{lemma-integral-over-invariants}).
Then also $C$ is a product of rings $C_r$ and
correspondingly $C'$ decomposes as a product. Hence we may and do
assume that the ring maps $s, t : A \to B$ are finite
locally free of a fixed rank $r$.

\medskip\noindent
The local ring maps $C_{\mathfrak p} \to C_{\mathfrak p}'$ we are going
to use are any local flat ring maps such that the residue field of
$C_{\mathfrak p}'$ is infinite.
By Algebra, Lemma \ref{algebra-lemma-flat-local-given-residue-field}
such local ring maps exist.

\medskip\noindent
Assume $C$ is a local ring with maximal ideal $\mathfrak m$ and
infinite residue field, and assume that $s, t : A \to B$ is
finite locally free of constant rank $r > 0$.
Since $C \subset A$ is integral (Lemma \ref{lemma-integral-over-invariants})
all primes lying over $\mathfrak m$ are maximal, and all maximal
ideals of $A$ lie over $\mathfrak m$. Similarly for $C \subset B$.
Pick a maximal ideal $\mathfrak m'$
of $A$ lying over $\mathfrak m$ (exists by Lemma \ref{lemma-points}).
Since $t : A \to B$ is finite locally free there exist at most finitely
many maximal ideals of $B$ lying over $\mathfrak m'$. Hence we conclude
(by Lemma \ref{lemma-points} again)
that $A$ has finitely many maximal ideals, i.e.,
$A$ is semi-local. This in turn implies that $B$ is semi-local as
well. OK, and now, because $t \otimes s : A \otimes_C A \to B$ is surjective,
we can apply
Algebra, Lemma \ref{algebra-lemma-semi-local-module-basis-in-submodule}
to the ring map $C \to A$, the $A$-module $M = B$ (seen as an $A$-module
via $t$) and the $C$-submodule $s(A) \subset B$. This lemma implies that there
exist $x_1, \ldots, x_r \in A$ such that $M$ is free over $A$
on the basis $s(x_1), \ldots, s(x_r)$. Hence we conclude that $C \to A$
is finite free and $B \cong A \otimes_C A$ by applying
Lemma \ref{lemma-basis}.
\end{proof}



\section{Finite flat groupoids}
\label{section-finite-flat-general}

\noindent
In this section we prove a lemma that will help to show that the quotient
of a scheme by a finite flat equivalence relation is a scheme, provided that
each equivalence class is contained in an affine. See
Properties of Spaces,
Proposition \ref{spaces-properties-proposition-finite-flat-equivalence-global}.

\begin{lemma}
\label{lemma-find-invariant-affine}
Let $S$ be a scheme.
Let $(U, R, s, t, c)$ be a groupoid scheme over $S$.
Assume $s$, $t$ are finite locally free.
Let $u \in U$ be a point such that $t(s^{-1}(\{u\}))$
is contained in an affine open of $U$.
Then there exists an $R$-invariant affine open neighbourhood
of $u$ in $U$.
\end{lemma}

\begin{proof}
Since $s$ is finite locally free it has finite fibres. Hence
$t(s^{-1}(\{u\})) = \{u_1, \ldots, u_n\}$ is a finite set.
Note that $u \in \{u_1, \ldots, u_n\}$.
Let $W \subset U$ be an affine open containing $\{u_1, \ldots, u_n\}$,
in particular $u \in W$. Consider
$Z = R \setminus s^{-1}(W) \cap t^{-1}(W)$. This is a closed subset
of $R$. The image $t(Z)$ is a closed subset of $U$ which can be loosely
described as the set of points of $U$ which are $R$-equivalent to a point
of $U \setminus W$. Hence $W' = U \setminus t(Z)$ is an $R$-invariant, open
subscheme of $U$ contained in $W$, and $\{u_1, \ldots, u_n\} \subset W'$.
Picture
$$
\{u_1, \ldots, u_n\} \subset W' \subset W \subset U.
$$
Let $f \in \Gamma(W, \mathcal{O}_W)$ be an element such that
$\{u_1, \ldots, u_n\} \subset D(f) \subset W'$. Such an $f$ exists by
Algebra, Lemma \ref{algebra-lemma-silly}. By our choice of $W'$ we
have $s^{-1}(W') \subset t^{-1}(W)$, and hence we get a diagram
$$
\xymatrix{
s^{-1}(W') \ar[d]_s \ar[r]_-t & W \\
W'
}
$$
The vertical arrow is finite locally free by assumption. Set
$$
g = \text{Norm}_s(t^\sharp f) \in \Gamma(W', \mathcal{O}_{W'})
$$
By construction $g$ is a function on $W'$ which is
nonzero in $u$, as $t^\sharp(f)$ is nonzero in each of the points of
$R$ lying over $u$, since $f$ is nonzero in $u_1, \ldots, u_n$.
Similarly, $D(g) \subset W'$ is equal to the
set of points $w$ such that $f$ is not zero in any of the points
equivalent to $w$. This means that $D(g)$ is an
$R$-invariant affine open of $W'$. The final picture is
$$
\{u_1, \ldots, u_n\} \subset D(g) \subset D(f) \subset W' \subset W \subset U
$$
and hence we win.
\end{proof}







\section{Descending quasi-projective schemes}
\label{section-quasi-projective}

\noindent
We can use Lemma \ref{lemma-find-invariant-affine}
to show that a certain type of descent datum is effective.

\begin{lemma}
\label{lemma-descend-along-finite}
Let $X \to Y$ be a surjective finite locally free morphism.
Let $V$ be a scheme over $X$ such that for all
$(y, v_1, \ldots, v_d)$ where $y \in Y$ and
$v_1, \ldots, v_d \in V_y$ there exists an affine open
$U \subset V$ with $v_1, \ldots, v_d \in U$.
Then any descent datum on $V/X/Y$ is effective.
\end{lemma}

\begin{proof}
Let $\varphi$ be a descent datum as in
Descent, Definition \ref{descent-definition-descent-datum}.
Recall that the functor from schemes over $Y$ to descent data
relative to $\{X \to Y\}$ is fully faithful, see
Descent, Lemma \ref{descent-lemma-refine-coverings-fully-faithful}.
Thus using Constructions, Lemma \ref{constructions-lemma-relative-glueing}
it suffices to prove the lemma in the case that $Y$ is affine.
Some details omitted (this argument can be avoided if $Y$ is
separated or has affine diagonal, because then every morphism from
an affine scheme to $X$ is affine).

\medskip\noindent
Assume $Y$ is affine. If $V$ is also affine, then we have effectivity by
Descent, Lemma \ref{descent-lemma-affine}. Hence by
Descent, Lemma \ref{descent-lemma-effective-for-fpqc-is-local-upstairs}
it suffices to prove that every point $v$ of $V$ has a $\varphi$-invariant
affine open neighbourhood. Consider the groupoid
$(X, X \times_Y X, \text{pr}_1, \text{pr}_0, \text{pr}_{02})$.
By Lemma \ref{lemma-cartesian-equivalent-descent-datum}
the descent datum $\varphi$ determines and is determined by
a cartesian morphism of groupoid schemes
$$
(V, R, s, t, c)
\longrightarrow
(X, X \times_Y X, \text{pr}_1, \text{pr}_0, \text{pr}_{02})
$$
over $\Spec(\mathbf{Z})$.
Since $X \to Y$ is finite locally free, we see that
$\text{pr}_i : X \times_Y X \to X$ and hence $s$ and $t$
are finite locally free. In particular the $R$-orbit
$t(s^{-1}(\{v\}))$ of our point $v \in V$
is finite. Using the equivalence of categories of
Lemma \ref{lemma-cartesian-equivalent-descent-datum}
once more we see that $\varphi$-invariant opens of $V$
are the same thing as $R$-invariant opens of $V$.
Our assumption shows there exists an affine open of $V$
containing the orbit $t(s^{-1}(\{v\}))$ as all the points
in this orbit map to the same point of $Y$.
Thus Lemma \ref{lemma-find-invariant-affine}
provides an $R$-invariant affine open containing $v$.
\end{proof}

\begin{lemma}
\label{lemma-descend-along-finite-quasi-projective}
Let $X \to Y$ be a surjective finite locally free morphism.
Let $V$ be a scheme over $X$ such that one of the following holds
\begin{enumerate}
\item $V \to X$ is projective,
\item $V \to X$ is quasi-projective,
\item there exists an ample invertible sheaf on $V$,
\item there exists an $X$-ample invertible sheaf on $V$,
\item there exists an $X$-very ample invertible sheaf on $V$.
\end{enumerate}
Then any descent datum on $V/X/Y$ is effective.
\end{lemma}

\begin{proof}
We check the condition in Lemma \ref{lemma-descend-along-finite}.
Let $y \in Y$ and $v_1, \ldots, v_d \in V$ points over $y$.
Case (1) is a special case of (2), see
Morphisms, Lemma \ref{morphisms-lemma-projective-quasi-projective}.
Case (2) is a special case of (4), see
Morphisms, Definition \ref{morphisms-definition-quasi-projective}.
If there exists an ample invertible sheaf on $V$, then
there exists an affine open containing $v_1, \ldots, v_d$ by
Properties, Lemma \ref{properties-lemma-ample-finite-set-in-affine}.
Thus (3) is true.
In cases (4) and (5) it is harmless to replace $Y$ by an
affine open neighbourhood of $y$.
Then $X$ is affine too.
In case (4) we see that $V$ has an ample invertible sheaf
by Morphisms, Definition \ref{morphisms-definition-relatively-ample}
and the result follows from case (3).
In case (5) we can replace $V$ by a quasi-compact open containing
$v_1, \ldots, v_d$ and we reduce to case (4) by
Morphisms, Lemma \ref{morphisms-lemma-ample-very-ample}.
\end{proof}













\begin{multicols}{2}[\section{Other chapters}]
\noindent
Preliminaries
\begin{enumerate}
\item \hyperref[introduction-section-phantom]{Introduction}
\item \hyperref[conventions-section-phantom]{Conventions}
\item \hyperref[sets-section-phantom]{Set Theory}
\item \hyperref[categories-section-phantom]{Categories}
\item \hyperref[topology-section-phantom]{Topology}
\item \hyperref[sheaves-section-phantom]{Sheaves on Spaces}
\item \hyperref[sites-section-phantom]{Sites and Sheaves}
\item \hyperref[stacks-section-phantom]{Stacks}
\item \hyperref[fields-section-phantom]{Fields}
\item \hyperref[algebra-section-phantom]{Commutative Algebra}
\item \hyperref[brauer-section-phantom]{Brauer Groups}
\item \hyperref[homology-section-phantom]{Homological Algebra}
\item \hyperref[derived-section-phantom]{Derived Categories}
\item \hyperref[simplicial-section-phantom]{Simplicial Methods}
\item \hyperref[more-algebra-section-phantom]{More on Algebra}
\item \hyperref[smoothing-section-phantom]{Smoothing Ring Maps}
\item \hyperref[modules-section-phantom]{Sheaves of Modules}
\item \hyperref[sites-modules-section-phantom]{Modules on Sites}
\item \hyperref[injectives-section-phantom]{Injectives}
\item \hyperref[cohomology-section-phantom]{Cohomology of Sheaves}
\item \hyperref[sites-cohomology-section-phantom]{Cohomology on Sites}
\item \hyperref[dga-section-phantom]{Differential Graded Algebra}
\item \hyperref[dpa-section-phantom]{Divided Power Algebra}
\item \hyperref[sdga-section-phantom]{Differential Graded Sheaves}
\item \hyperref[hypercovering-section-phantom]{Hypercoverings}
\end{enumerate}
Schemes
\begin{enumerate}
\setcounter{enumi}{25}
\item \hyperref[schemes-section-phantom]{Schemes}
\item \hyperref[constructions-section-phantom]{Constructions of Schemes}
\item \hyperref[properties-section-phantom]{Properties of Schemes}
\item \hyperref[morphisms-section-phantom]{Morphisms of Schemes}
\item \hyperref[coherent-section-phantom]{Cohomology of Schemes}
\item \hyperref[divisors-section-phantom]{Divisors}
\item \hyperref[limits-section-phantom]{Limits of Schemes}
\item \hyperref[varieties-section-phantom]{Varieties}
\item \hyperref[topologies-section-phantom]{Topologies on Schemes}
\item \hyperref[descent-section-phantom]{Descent}
\item \hyperref[perfect-section-phantom]{Derived Categories of Schemes}
\item \hyperref[more-morphisms-section-phantom]{More on Morphisms}
\item \hyperref[flat-section-phantom]{More on Flatness}
\item \hyperref[groupoids-section-phantom]{Groupoid Schemes}
\item \hyperref[more-groupoids-section-phantom]{More on Groupoid Schemes}
\item \hyperref[etale-section-phantom]{\'Etale Morphisms of Schemes}
\end{enumerate}
Topics in Scheme Theory
\begin{enumerate}
\setcounter{enumi}{41}
\item \hyperref[chow-section-phantom]{Chow Homology}
\item \hyperref[intersection-section-phantom]{Intersection Theory}
\item \hyperref[pic-section-phantom]{Picard Schemes of Curves}
\item \hyperref[weil-section-phantom]{Weil Cohomology Theories}
\item \hyperref[adequate-section-phantom]{Adequate Modules}
\item \hyperref[dualizing-section-phantom]{Dualizing Complexes}
\item \hyperref[duality-section-phantom]{Duality for Schemes}
\item \hyperref[discriminant-section-phantom]{Discriminants and Differents}
\item \hyperref[derham-section-phantom]{de Rham Cohomology}
\item \hyperref[local-cohomology-section-phantom]{Local Cohomology}
\item \hyperref[algebraization-section-phantom]{Algebraic and Formal Geometry}
\item \hyperref[curves-section-phantom]{Algebraic Curves}
\item \hyperref[resolve-section-phantom]{Resolution of Surfaces}
\item \hyperref[models-section-phantom]{Semistable Reduction}
\item \hyperref[functors-section-phantom]{Functors and Morphisms}
\item \hyperref[equiv-section-phantom]{Derived Categories of Varieties}
\item \hyperref[pione-section-phantom]{Fundamental Groups of Schemes}
\item \hyperref[etale-cohomology-section-phantom]{\'Etale Cohomology}
\item \hyperref[crystalline-section-phantom]{Crystalline Cohomology}
\item \hyperref[proetale-section-phantom]{Pro-\'etale Cohomology}
\item \hyperref[relative-cycles-section-phantom]{Relative Cycles}
\item \hyperref[more-etale-section-phantom]{More \'Etale Cohomology}
\item \hyperref[trace-section-phantom]{The Trace Formula}
\end{enumerate}
Algebraic Spaces
\begin{enumerate}
\setcounter{enumi}{64}
\item \hyperref[spaces-section-phantom]{Algebraic Spaces}
\item \hyperref[spaces-properties-section-phantom]{Properties of Algebraic Spaces}
\item \hyperref[spaces-morphisms-section-phantom]{Morphisms of Algebraic Spaces}
\item \hyperref[decent-spaces-section-phantom]{Decent Algebraic Spaces}
\item \hyperref[spaces-cohomology-section-phantom]{Cohomology of Algebraic Spaces}
\item \hyperref[spaces-limits-section-phantom]{Limits of Algebraic Spaces}
\item \hyperref[spaces-divisors-section-phantom]{Divisors on Algebraic Spaces}
\item \hyperref[spaces-over-fields-section-phantom]{Algebraic Spaces over Fields}
\item \hyperref[spaces-topologies-section-phantom]{Topologies on Algebraic Spaces}
\item \hyperref[spaces-descent-section-phantom]{Descent and Algebraic Spaces}
\item \hyperref[spaces-perfect-section-phantom]{Derived Categories of Spaces}
\item \hyperref[spaces-more-morphisms-section-phantom]{More on Morphisms of Spaces}
\item \hyperref[spaces-flat-section-phantom]{Flatness on Algebraic Spaces}
\item \hyperref[spaces-groupoids-section-phantom]{Groupoids in Algebraic Spaces}
\item \hyperref[spaces-more-groupoids-section-phantom]{More on Groupoids in Spaces}
\item \hyperref[bootstrap-section-phantom]{Bootstrap}
\item \hyperref[spaces-pushouts-section-phantom]{Pushouts of Algebraic Spaces}
\end{enumerate}
Topics in Geometry
\begin{enumerate}
\setcounter{enumi}{81}
\item \hyperref[spaces-chow-section-phantom]{Chow Groups of Spaces}
\item \hyperref[groupoids-quotients-section-phantom]{Quotients of Groupoids}
\item \hyperref[spaces-more-cohomology-section-phantom]{More on Cohomology of Spaces}
\item \hyperref[spaces-simplicial-section-phantom]{Simplicial Spaces}
\item \hyperref[spaces-duality-section-phantom]{Duality for Spaces}
\item \hyperref[formal-spaces-section-phantom]{Formal Algebraic Spaces}
\item \hyperref[restricted-section-phantom]{Algebraization of Formal Spaces}
\item \hyperref[spaces-resolve-section-phantom]{Resolution of Surfaces Revisited}
\end{enumerate}
Deformation Theory
\begin{enumerate}
\setcounter{enumi}{89}
\item \hyperref[formal-defos-section-phantom]{Formal Deformation Theory}
\item \hyperref[defos-section-phantom]{Deformation Theory}
\item \hyperref[cotangent-section-phantom]{The Cotangent Complex}
\item \hyperref[examples-defos-section-phantom]{Deformation Problems}
\end{enumerate}
Algebraic Stacks
\begin{enumerate}
\setcounter{enumi}{93}
\item \hyperref[algebraic-section-phantom]{Algebraic Stacks}
\item \hyperref[examples-stacks-section-phantom]{Examples of Stacks}
\item \hyperref[stacks-sheaves-section-phantom]{Sheaves on Algebraic Stacks}
\item \hyperref[criteria-section-phantom]{Criteria for Representability}
\item \hyperref[artin-section-phantom]{Artin's Axioms}
\item \hyperref[quot-section-phantom]{Quot and Hilbert Spaces}
\item \hyperref[stacks-properties-section-phantom]{Properties of Algebraic Stacks}
\item \hyperref[stacks-morphisms-section-phantom]{Morphisms of Algebraic Stacks}
\item \hyperref[stacks-limits-section-phantom]{Limits of Algebraic Stacks}
\item \hyperref[stacks-cohomology-section-phantom]{Cohomology of Algebraic Stacks}
\item \hyperref[stacks-perfect-section-phantom]{Derived Categories of Stacks}
\item \hyperref[stacks-introduction-section-phantom]{Introducing Algebraic Stacks}
\item \hyperref[stacks-more-morphisms-section-phantom]{More on Morphisms of Stacks}
\item \hyperref[stacks-geometry-section-phantom]{The Geometry of Stacks}
\end{enumerate}
Topics in Moduli Theory
\begin{enumerate}
\setcounter{enumi}{107}
\item \hyperref[moduli-section-phantom]{Moduli Stacks}
\item \hyperref[moduli-curves-section-phantom]{Moduli of Curves}
\end{enumerate}
Miscellany
\begin{enumerate}
\setcounter{enumi}{109}
\item \hyperref[examples-section-phantom]{Examples}
\item \hyperref[exercises-section-phantom]{Exercises}
\item \hyperref[guide-section-phantom]{Guide to Literature}
\item \hyperref[desirables-section-phantom]{Desirables}
\item \hyperref[coding-section-phantom]{Coding Style}
\item \hyperref[obsolete-section-phantom]{Obsolete}
\item \hyperref[fdl-section-phantom]{GNU Free Documentation License}
\item \hyperref[index-section-phantom]{Auto Generated Index}
\end{enumerate}
\end{multicols}


\bibliography{my}
\bibliographystyle{amsalpha}

\end{document}
