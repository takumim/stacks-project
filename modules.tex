\IfFileExists{stacks-project.cls}{%
\documentclass{stacks-project}
}{%
\documentclass{amsart}
}

% For dealing with references we use the comment environment
\usepackage{verbatim}
\newenvironment{reference}{\comment}{\endcomment}
%\newenvironment{reference}{}{}
\newenvironment{slogan}{\comment}{\endcomment}
\newenvironment{history}{\comment}{\endcomment}

% For commutative diagrams we use Xy-pic
\usepackage[all]{xy}

% We use 2cell for 2-commutative diagrams.
\xyoption{2cell}
\UseAllTwocells

% We use multicol for the list of chapters between chapters
\usepackage{multicol}

% This is generall recommended for better output
\usepackage[T1]{fontenc}

% For cross-file-references
\usepackage{xr-hyper}

% Package for hypertext links:
\usepackage{hyperref}

% For any local file, say "hello.tex" you want to link to please
% use \externaldocument[hello-]{hello}
\externaldocument[introduction-]{introduction}
\externaldocument[conventions-]{conventions}
\externaldocument[sets-]{sets}
\externaldocument[categories-]{categories}
\externaldocument[topology-]{topology}
\externaldocument[sheaves-]{sheaves}
\externaldocument[sites-]{sites}
\externaldocument[stacks-]{stacks}
\externaldocument[fields-]{fields}
\externaldocument[algebra-]{algebra}
\externaldocument[brauer-]{brauer}
\externaldocument[homology-]{homology}
\externaldocument[derived-]{derived}
\externaldocument[simplicial-]{simplicial}
\externaldocument[more-algebra-]{more-algebra}
\externaldocument[smoothing-]{smoothing}
\externaldocument[modules-]{modules}
\externaldocument[sites-modules-]{sites-modules}
\externaldocument[injectives-]{injectives}
\externaldocument[cohomology-]{cohomology}
\externaldocument[sites-cohomology-]{sites-cohomology}
\externaldocument[dga-]{dga}
\externaldocument[dpa-]{dpa}
\externaldocument[hypercovering-]{hypercovering}
\externaldocument[schemes-]{schemes}
\externaldocument[constructions-]{constructions}
\externaldocument[properties-]{properties}
\externaldocument[morphisms-]{morphisms}
\externaldocument[coherent-]{coherent}
\externaldocument[divisors-]{divisors}
\externaldocument[limits-]{limits}
\externaldocument[varieties-]{varieties}
\externaldocument[topologies-]{topologies}
\externaldocument[descent-]{descent}
\externaldocument[perfect-]{perfect}
\externaldocument[more-morphisms-]{more-morphisms}
\externaldocument[flat-]{flat}
\externaldocument[groupoids-]{groupoids}
\externaldocument[more-groupoids-]{more-groupoids}
\externaldocument[etale-]{etale}
\externaldocument[chow-]{chow}
\externaldocument[intersection-]{intersection}
\externaldocument[pic-]{pic}
\externaldocument[adequate-]{adequate}
\externaldocument[dualizing-]{dualizing}
\externaldocument[duality-]{duality}
\externaldocument[discriminant-]{discriminant}
\externaldocument[local-cohomology-]{local-cohomology}
\externaldocument[curves-]{curves}
\externaldocument[resolve-]{resolve}
\externaldocument[models-]{models}
\externaldocument[pione-]{pione}
\externaldocument[etale-cohomology-]{etale-cohomology}
\externaldocument[proetale-]{proetale}
\externaldocument[crystalline-]{crystalline}
\externaldocument[spaces-]{spaces}
\externaldocument[spaces-properties-]{spaces-properties}
\externaldocument[spaces-morphisms-]{spaces-morphisms}
\externaldocument[decent-spaces-]{decent-spaces}
\externaldocument[spaces-cohomology-]{spaces-cohomology}
\externaldocument[spaces-limits-]{spaces-limits}
\externaldocument[spaces-divisors-]{spaces-divisors}
\externaldocument[spaces-over-fields-]{spaces-over-fields}
\externaldocument[spaces-topologies-]{spaces-topologies}
\externaldocument[spaces-descent-]{spaces-descent}
\externaldocument[spaces-perfect-]{spaces-perfect}
\externaldocument[spaces-more-morphisms-]{spaces-more-morphisms}
\externaldocument[spaces-flat-]{spaces-flat}
\externaldocument[spaces-groupoids-]{spaces-groupoids}
\externaldocument[spaces-more-groupoids-]{spaces-more-groupoids}
\externaldocument[bootstrap-]{bootstrap}
\externaldocument[spaces-pushouts-]{spaces-pushouts}
\externaldocument[spaces-chow-]{spaces-chow}
\externaldocument[groupoids-quotients-]{groupoids-quotients}
\externaldocument[spaces-more-cohomology-]{spaces-more-cohomology}
\externaldocument[spaces-simplicial-]{spaces-simplicial}
\externaldocument[spaces-duality-]{spaces-duality}
\externaldocument[formal-spaces-]{formal-spaces}
\externaldocument[restricted-]{restricted}
\externaldocument[spaces-resolve-]{spaces-resolve}
\externaldocument[formal-defos-]{formal-defos}
\externaldocument[defos-]{defos}
\externaldocument[cotangent-]{cotangent}
\externaldocument[examples-defos-]{examples-defos}
\externaldocument[algebraic-]{algebraic}
\externaldocument[examples-stacks-]{examples-stacks}
\externaldocument[stacks-sheaves-]{stacks-sheaves}
\externaldocument[criteria-]{criteria}
\externaldocument[artin-]{artin}
\externaldocument[quot-]{quot}
\externaldocument[stacks-properties-]{stacks-properties}
\externaldocument[stacks-morphisms-]{stacks-morphisms}
\externaldocument[stacks-limits-]{stacks-limits}
\externaldocument[stacks-cohomology-]{stacks-cohomology}
\externaldocument[stacks-perfect-]{stacks-perfect}
\externaldocument[stacks-introduction-]{stacks-introduction}
\externaldocument[stacks-more-morphisms-]{stacks-more-morphisms}
\externaldocument[stacks-geometry-]{stacks-geometry}
\externaldocument[moduli-]{moduli}
\externaldocument[moduli-curves-]{moduli-curves}
\externaldocument[examples-]{examples}
\externaldocument[exercises-]{exercises}
\externaldocument[guide-]{guide}
\externaldocument[desirables-]{desirables}
\externaldocument[coding-]{coding}
\externaldocument[obsolete-]{obsolete}
\externaldocument[fdl-]{fdl}
\externaldocument[index-]{index}

% Theorem environments.
%
\theoremstyle{plain}
\newtheorem{theorem}[subsection]{Theorem}
\newtheorem{proposition}[subsection]{Proposition}
\newtheorem{lemma}[subsection]{Lemma}

\theoremstyle{definition}
\newtheorem{definition}[subsection]{Definition}
\newtheorem{example}[subsection]{Example}
\newtheorem{exercise}[subsection]{Exercise}
\newtheorem{situation}[subsection]{Situation}

\theoremstyle{remark}
\newtheorem{remark}[subsection]{Remark}
\newtheorem{remarks}[subsection]{Remarks}

\numberwithin{equation}{subsection}

% Macros
%
\def\lim{\mathop{\mathrm{lim}}\nolimits}
\def\colim{\mathop{\mathrm{colim}}\nolimits}
\def\Spec{\mathop{\mathrm{Spec}}}
\def\Hom{\mathop{\mathrm{Hom}}\nolimits}
\def\Ext{\mathop{\mathrm{Ext}}\nolimits}
\def\SheafHom{\mathop{\mathcal{H}\!\mathit{om}}\nolimits}
\def\SheafExt{\mathop{\mathcal{E}\!\mathit{xt}}\nolimits}
\def\Sch{\mathit{Sch}}
\def\Mor{\mathop{Mor}\nolimits}
\def\Ob{\mathop{\mathrm{Ob}}\nolimits}
\def\Sh{\mathop{\mathit{Sh}}\nolimits}
\def\NL{\mathop{N\!L}\nolimits}
\def\proetale{{pro\text{-}\acute{e}tale}}
\def\etale{{\acute{e}tale}}
\def\QCoh{\mathit{QCoh}}
\def\Ker{\mathop{\mathrm{Ker}}}
\def\Im{\mathop{\mathrm{Im}}}
\def\Coker{\mathop{\mathrm{Coker}}}
\def\Coim{\mathop{\mathrm{Coim}}}

%
% Macros for moduli stacks/spaces
%
\def\QCohstack{\mathcal{QC}\!\mathit{oh}}
\def\Cohstack{\mathcal{C}\!\mathit{oh}}
\def\Spacesstack{\mathcal{S}\!\mathit{paces}}
\def\Quotfunctor{\mathrm{Quot}}
\def\Hilbfunctor{\mathrm{Hilb}}
\def\Curvesstack{\mathcal{C}\!\mathit{urves}}
\def\Polarizedstack{\mathcal{P}\!\mathit{olarized}}
\def\Complexesstack{\mathcal{C}\!\mathit{omplexes}}
% \Pic is the operator that assigns to X its picard group, usage \Pic(X)
% \Picardstack_{X/B} denotes the Picard stack of X over B
% \Picardfunctor_{X/B} denotes the Picard functor of X over B
\def\Pic{\mathop{\mathrm{Pic}}\nolimits}
\def\Picardstack{\mathcal{P}\!\mathit{ic}}
\def\Picardfunctor{\mathrm{Pic}}
\def\Deformationcategory{\mathcal{D}\!\mathit{ef}}


% OK, start here.
%
\begin{document}

\title{Sheaves of Modules}


\maketitle

\phantomsection
\label{section-phantom}

\tableofcontents

\section{Introduction}
\label{section-introduction}

\noindent
In this chapter we work out basic notions of sheaves of modules.
This in particular includes the case of abelian sheaves, since
these may be viewed as sheaves of $\underline{\mathbf{Z}}$-modules.
Basic references are \cite{FAC}, \cite{EGA} and \cite{SGA4}.

\medskip\noindent
We work out what happens for sheaves of modules on ringed topoi
in another chapter (see
Modules on Sites, Section \ref{sites-modules-section-introduction}),
although there we will mostly just duplicate the discussion
from this chapter.





\section{Pathology}
\label{section-pathology}

\noindent
A ringed space is a pair consisting of a topological space $X$
and a sheaf of rings $\mathcal{O}$. We allow $\mathcal{O} = 0$
in the definition. In this case the category of modules has a
single object (namely $0$). It is still an abelian category etc,
but it is a little degenerate. Similarly the sheaf $\mathcal{O}$
may be zero over open subsets of $X$, etc.

\medskip\noindent
This doesn't happen when considering locally ringed spaces (as we
will do later).








\section{The abelian category of sheaves of modules}
\label{section-kernels}

\noindent
Let $(X, \mathcal{O}_X)$ be a ringed space, see
Sheaves, Definition \ref{sheaves-definition-ringed-space}.
Let $\mathcal{F}$, $\mathcal{G}$ be sheaves of $\mathcal{O}_X$-modules, see
Sheaves, Definition \ref{sheaves-definition-sheaf-modules}.
Let $\varphi, \psi : \mathcal{F} \to \mathcal{G}$
be morphisms of sheaves of $\mathcal{O}_X$-modules.
We define $\varphi + \psi : \mathcal{F} \to \mathcal{G}$
to be the map which on each open $U \subset X$ is the
sum of the maps induced by $\varphi$, $\psi$. This is
clearly again a map of sheaves of $\mathcal{O}_X$-modules.
It is also clear that composition of maps of
$\mathcal{O}_X$-modules is bilinear with respect to this
addition. Thus $\textit{Mod}(\mathcal{O}_X)$ is a pre-additive
category, see Homology, Definition \ref{homology-definition-preadditive}.

\medskip\noindent
We will denote $0$ the sheaf of $\mathcal{O}_X$-modules
which has constant value $\{0\}$ for all open $U \subset X$.
Clearly this is both a final and an initial object of
$\textit{Mod}(\mathcal{O}_X)$. Given a morphism
of $\mathcal{O}_X$-modules $\varphi : \mathcal{F} \to \mathcal{G}$
the following are equivalent:
(a) $\varphi$ is zero, (b) $\varphi$ factors through $0$,
(c) $\varphi$ is zero on sections over each open $U$, and
(d) $\varphi_x = 0$ for all $x \in X$. See
Sheaves, Lemma \ref{sheaves-lemma-points-exactness}.

\medskip\noindent
Moreover, given a pair
$\mathcal{F}$, $\mathcal{G}$ of sheaves of $\mathcal{O}_X$-modules
we may define the direct sum as
$$
\mathcal{F} \oplus \mathcal{G} = \mathcal{F} \times \mathcal{G}
$$
with obvious maps $(i, j, p, q)$ as in Homology, Definition
\ref{homology-definition-direct-sum}. Thus $\textit{Mod}(\mathcal{O}_X)$
is an additive category, see
Homology, Definition \ref{homology-definition-additive-category}.

\medskip\noindent
Let $\varphi : \mathcal{F} \to \mathcal{G}$ be a morphism
of $\mathcal{O}_X$-modules. We may define $\Ker(\varphi)$
to be the subsheaf of $\mathcal{F}$ with sections
$$
\Ker(\varphi)(U) =
\{ s \in \mathcal{F}(U) \mid \varphi(s) = 0 \text{ in } \mathcal{G}(U)\}
$$
for all open $U \subset X$. It is easy to see that this is indeed
a kernel in the category of $\mathcal{O}_X$-modules. In other words,
a morphism $\alpha : \mathcal{H} \to \mathcal{F}$ factors
through $\Ker(\varphi)$ if and only if $\varphi \circ \alpha = 0$.
Moreover, on the level of stalks we have
$\Ker(\varphi)_x = \Ker(\varphi_x)$.

\medskip\noindent
On the other hand, we define
$\Coker(\varphi)$ as the sheaf of $\mathcal{O}_X$-modules
associated to the presheaf of $\mathcal{O}_X$-modules defined
by the rule
$$
U
\longmapsto
\Coker(\mathcal{G}(U)\to \mathcal{F}(U)) =
\mathcal{F}(U)/\varphi(\mathcal{G}(U)).
$$
Since taking stalks commutes with taking sheafification, see
Sheaves, Lemma \ref{sheaves-lemma-stalk-sheafification} we
see that $\Coker(\varphi)_x = \Coker(\varphi_x)$.
Thus the map $\mathcal{G} \to \Coker(\varphi)$ is surjective
(as a map of sheaves of sets),
see Sheaves, Section \ref{sheaves-section-exactness-points}.
To show that this is a cokernel, note that if
$\beta : \mathcal{G} \to \mathcal{H}$ is a morphism of $\mathcal{O}_X$-modules
such that $\beta \circ \varphi$ is zero, then you get for every
open $U \subset X$ a map induced by $\beta$ from
$\mathcal{G}(U)/\varphi(\mathcal{F}(U))$ into $\mathcal{H}(U)$.
By the universal property of sheafification (see
Sheaves, Lemma \ref{sheaves-lemma-sheafification-presheaf-modules})
we obtain a canonical map $\Coker(\varphi) \to \mathcal{H}$
such that the original $\beta$ is equal to
the composition
$\mathcal{G} \to \Coker(\varphi) \to \mathcal{H}$.
The morphism $\Coker(\varphi) \to \mathcal{H}$ is unique
because of the surjectivity mentioned above.

\begin{lemma}
\label{lemma-abelian}
Let $(X, \mathcal{O}_X)$ be a ringed space. The category
$\textit{Mod}(\mathcal{O}_X)$ is an abelian category. Moreover
a complex
$$
\mathcal{F} \to \mathcal{G} \to \mathcal{H}
$$
is exact at $\mathcal{G}$ if and only if for all $x \in X$ the
complex
$$
\mathcal{F}_x \to \mathcal{G}_x \to \mathcal{H}_x
$$
is exact at $\mathcal{G}_x$.
\end{lemma}

\begin{proof}
By Homology, Definition \ref{homology-definition-abelian-category}
we have to show that image and coimage agree. By Sheaves,
Lemma \ref{sheaves-lemma-points-exactness} it is enough to show
that image and coimage have the same stalk at every $x \in X$.
By the constructions of kernels and cokernels above these stalks
are the coimage and image in the categories of $\mathcal{O}_{X, x}$-modules.
Thus we get the result from the fact that the category of modules
over a ring is abelian.
\end{proof}

\noindent
Actually the category $\textit{Mod}(\mathcal{O}_X)$ has many more properties.
Here are two constructions we can do.
\begin{enumerate}
\item Given any set $I$ and for each $i \in I$ a $\mathcal{O}_X$-module
we can form the product
$$
\prod\nolimits_{i \in I} \mathcal{F}_i
$$
which is the sheaf that associates to each open $U$ the
product of the modules $\mathcal{F}_i(U)$. This is also the
categorical product, as in
Categories, Definition \ref{categories-definition-product}.
\item Given any set $I$ and for each $i \in I$ a $\mathcal{O}_X$-module
we can form the direct sum
$$
\bigoplus\nolimits_{i \in I} \mathcal{F}_i
$$
which is the {\it sheafification} of the presheaf
that associates to each open $U$ the
direct sum of the modules $\mathcal{F}_i(U)$.
This is also the categorical coproduct, as in
Categories, Definition \ref{categories-definition-coproduct}.
To see this you use the universal property of sheafification.
\end{enumerate}
Using these we conclude that all limits and colimits exist in
$\textit{Mod}(\mathcal{O}_X)$.

\begin{lemma}
\label{lemma-limits-colimits}
Let $(X, \mathcal{O}_X)$ be a ringed space.
\begin{enumerate}
\item All limits exist in $\textit{Mod}(\mathcal{O}_X)$.
Limits are the same as the corresponding limits of presheaves of
$\mathcal{O}_X$-modules (i.e., commute with taking
sections over opens).
\item All colimits exist in $\textit{Mod}(\mathcal{O}_X)$.
Colimits are the sheafification of the corresponding colimit in
the category of presheaves. Taking colimits commutes with taking
stalks.
\item Filtered colimits are exact.
\item Finite direct sums are the same as the corresponding
finite direct sums of presheaves of $\mathcal{O}_X$-modules.
\end{enumerate}
\end{lemma}

\begin{proof}
As $\textit{Mod}(\mathcal{O}_X)$ is abelian (Lemma \ref{lemma-abelian})
it has all finite limits and colimits
(Homology, Lemma \ref{homology-lemma-colimit-abelian-category}).
Thus the existence of limits and colimits and their description follows from
the existence of products and coproducts and their description
(see discussion above) and
Categories, Lemmas \ref{categories-lemma-limits-products-equalizers} and
\ref{categories-lemma-colimits-coproducts-coequalizers}.
Since sheafification commutes with taking stalks we see that
colimits commute with taking stalks. Part (3) signifies that given
a system $0 \to \mathcal{F}_i \to \mathcal{G}_i \to \mathcal{H}_i \to 0$
of exact sequences of $\mathcal{O}_X$-modules over a directed set $I$
the sequence $0 \to \colim \mathcal{F}_i \to \colim \mathcal{G}_i \to
\colim \mathcal{H}_i \to 0$ is exact as well. Since we can check
exactness on stalks (Lemma \ref{lemma-abelian}) this follows from the case
of modules which is
Algebra, Lemma \ref{algebra-lemma-directed-colimit-exact}.
We omit the proof of (4).
\end{proof}

\noindent
The existence of limits and colimits
allows us to consider exactness properties of
functors defined on the category of $\mathcal{O}$-modules
in terms of limits and colimits, as in
Categories, Section \ref{categories-section-exact-functor}.
See Homology, Lemma \ref{homology-lemma-exact-functor} for a
description of exactness
properties in terms of short exact sequences.

\begin{lemma}
\label{lemma-exactness-pushforward-pullback}
Let $f : (X, \mathcal{O}_X) \to (Y, \mathcal{O}_Y)$
be a morphism of ringed spaces.
\begin{enumerate}
\item The functor
$f_* : \textit{Mod}(\mathcal{O}_X) \to \textit{Mod}(\mathcal{O}_Y)$
is left exact. In fact it commutes with all limits.
\item The functor
$f^* : \textit{Mod}(\mathcal{O}_Y) \to \textit{Mod}(\mathcal{O}_X)$
is right exact. In fact it commutes with all colimits.
\item Pullback $f^{-1} : \textit{Ab}(Y) \to \textit{Ab}(X)$
on abelian sheaves is exact.
\end{enumerate}
\end{lemma}

\begin{proof}
Parts (1) and (2) hold because $(f^*, f_*)$ is an adjoint pair
of functors, see
Sheaves, Lemma \ref{sheaves-lemma-adjoint-pullback-pushforward-modules}
and
Categories, Section \ref{categories-section-adjoint}.
Part (3) holds because exactness can be checked on stalks
(Lemma \ref{lemma-abelian})
and the description of stalks of the pullback, see
Sheaves, Lemma \ref{sheaves-lemma-pullback-abelian-stalk}.
\end{proof}

\begin{lemma}
\label{lemma-j-shriek-exact}
Let $j : U \to X$ be an open immersion of topological spaces.
The functor $j_! : \textit{Ab}(U) \to \textit{Ab}(X)$
is exact.
\end{lemma}

\begin{proof}
Follows from the description of stalks
given in Sheaves, Lemma \ref{sheaves-lemma-j-shriek-abelian}.
\end{proof}

\begin{lemma}
\label{lemma-section-direct-sum-quasi-compact}
Let $(X, \mathcal{O}_X)$ be a ringed space.
Let $I$ be a set. For $i \in I$, let $\mathcal{F}_i$
be a sheaf of $\mathcal{O}_X$-modules.
For $U \subset X$ quasi-compact open the map
$$
\bigoplus\nolimits_{i \in I} \mathcal{F}_i(U)
\longrightarrow
\left(\bigoplus\nolimits_{i \in I} \mathcal{F}_i\right)(U)
$$
is bijective.
\end{lemma}

\begin{proof}
If $s$ is an element of the right hand side, then
there exists an open covering $U = \bigcup_{j \in J} U_j$
such that $s|_{U_j}$ is a finite sum
$\sum_{i \in I_j} s_{ji}$ with $s_{ji} \in \mathcal{F}_i(U_j)$.
Because $U$ is quasi-compact we may assume that the
covering is finite, i.e., that $J$ is finite.
Then $I' = \bigcup_{j \in J} I_j$ is a finite subset of
$I$. Clearly, $s$ is a section of the subsheaf
$\bigoplus_{i \in I'} \mathcal{F}_i$. The result follows
from the fact that for a finite direct sum sheafification
is not needed, see Lemma \ref{lemma-limits-colimits} above.
\end{proof}





\section{Sections of sheaves of modules}
\label{section-sections}

\noindent
Let $(X, \mathcal{O}_X)$ be a ringed space.
Let $\mathcal{F}$ be a sheaf of $\mathcal{O}_X$-modules.
Let $s \in \Gamma(X, \mathcal{F}) = \mathcal{F}(X)$ be a
global section. There is a unique {\it map of $\mathcal{O}_X$-modules
$$
\mathcal{O}_X \longrightarrow \mathcal{F}, \ f \longmapsto fs
$$
associated to $s$}. The notation above signifies that a local
section $f$ of $\mathcal{O}_X$, i.e., a section $f$ over some open $U$,
is mapped to the multiplication of $f$ with the restriction of $s$ to
$U$. Conversely, any map $\varphi : \mathcal{O}_X \to \mathcal{F}$
gives rise to a section $s = \varphi(1)$ such that $\varphi$ is
the morphism associated to $s$.

\begin{definition}
\label{definition-globally-generated}
Let $(X, \mathcal{O}_X)$ be a ringed space.
Let $\mathcal{F}$ be a sheaf of $\mathcal{O}_X$-modules.
We say that $\mathcal{F}$ is {\it generated by global
sections} if there exist a set $I$, and
global sections $s_i \in \Gamma(X, \mathcal{F})$, $i \in I$
such that the map
$$
\bigoplus\nolimits_{i \in I}
\mathcal{O}_X \longrightarrow \mathcal{F}
$$
which is the map associated to $s_i$ on the summand corresponding to $i$,
is surjective. In this case we say that the sections $s_i$
{\it generate} $\mathcal{F}$.
\end{definition}

\noindent
We often use the abuse of notation introduced in
Sheaves, Section \ref{sheaves-section-stalks} where, given a local
section $s$ of $\mathcal{F}$ defined in an open neighbourhood
of a point $x \in X$, we denote $s_x$, or even $s$ the image of $s$
in the stalk $\mathcal{F}_x$.

\begin{lemma}
\label{lemma-globally-generated}
Let $(X, \mathcal{O}_X)$ be a ringed space.
Let $\mathcal{F}$ be a sheaf of $\mathcal{O}_X$-modules.
Let $I$ be a set. Let
$s_i \in \Gamma(X, \mathcal{F})$, $i \in I$
be global sections. The sections $s_i$ generate
$\mathcal{F}$ if and only if for all $x\in X$ the
elements $s_{i, x} \in \mathcal{F}_x$ generate
the $\mathcal{O}_{X, x}$-module $\mathcal{F}_x$.
\end{lemma}

\begin{proof}
Omitted.
\end{proof}

\begin{lemma}
\label{lemma-tensor-product-globally-generated}
\begin{slogan}
The tensor product of globally generated sheaves of modules is
globally generated.
\end{slogan}
Let $(X, \mathcal{O}_X)$ be a ringed space.
Let $\mathcal{F}$, $\mathcal{G}$ be sheaves of $\mathcal{O}_X$-modules.
If $\mathcal{F}$ and $\mathcal{G}$ are generated by global sections
then so is $\mathcal{F} \otimes_{\mathcal{O}_X} \mathcal{G}$.
\end{lemma}

\begin{proof}
Omitted.
\end{proof}

\begin{lemma}
\label{lemma-generated-by-local-sections}
Let $(X, \mathcal{O}_X)$ be a ringed space.
Let $\mathcal{F}$ be a sheaf of $\mathcal{O}_X$-modules.
Let $I$ be a set. Let $s_i$, $i \in I$ be a collection
of local sections of $\mathcal{F}$, i.e., $s_i \in \mathcal{F}(U_i)$
for some opens $U_i \subset X$. There exists a unique smallest
subsheaf of $\mathcal{O}_X$-modules $\mathcal{G}$ such
that each $s_i$ corresponds to a local section of
$\mathcal{G}$.
\end{lemma}

\begin{proof}
Consider the subpresheaf of $\mathcal{O}_X$-modules
defined by the rule
$$
U
\longmapsto
\{
\text{sums } \sum\nolimits_{i \in J} f_i (s_i|_U)
\text{ where } J \text{ is finite, }
U \subset U_i \text{ for } i\in J, \text{ and }
f_i \in \mathcal{O}_X(U)
\}
$$
Let $\mathcal{G}$ be the sheafification of this subpresheaf.
This is a subsheaf of $\mathcal{F}$ by
Sheaves, Lemma \ref{sheaves-lemma-characterize-epi-mono}.
Since all the finite sums clearly have to be in $\mathcal{G}$
this is the smallest subsheaf as desired.
\end{proof}

\begin{definition}
\label{definition-generated-by-local-sections}
Let $(X, \mathcal{O}_X)$ be a ringed space.
Let $\mathcal{F}$ be a sheaf of $\mathcal{O}_X$-modules.
Given a set $I$, and
local sections $s_i$, $i \in I$ of $\mathcal{F}$
we say that the subsheaf $\mathcal{G}$ of
Lemma \ref{lemma-generated-by-local-sections}
above is the {\it subsheaf generated by the $s_i$}.
\end{definition}

\begin{lemma}
\label{lemma-generated-by-local-sections-stalk}
Let $(X, \mathcal{O}_X)$ be a ringed space.
Let $\mathcal{F}$ be a sheaf of $\mathcal{O}_X$-modules.
Given a set $I$, and
local sections $s_i$, $i \in I$ of $\mathcal{F}$.
Let $\mathcal{G}$ be the subsheaf generated by the
$s_i$ and let $x\in X$.
Then $\mathcal{G}_x$ is the $\mathcal{O}_{X, x}$-submodule of
$\mathcal{F}_x$ generated by the elements $s_{i, x}$
for those $i$ such that $s_i$ is defined at $x$.
\end{lemma}

\begin{proof}
This is clear from the construction of $\mathcal{G}$
in the proof of Lemma \ref{lemma-generated-by-local-sections}.
\end{proof}










\section{Supports of modules and sections}
\label{section-support}

\begin{definition}
\label{definition-support}
Let $(X, \mathcal{O}_X)$ be a ringed space.
Let $\mathcal{F}$ be a sheaf of $\mathcal{O}_X$-modules.
\begin{enumerate}
\item The {\it support of $\mathcal{F}$} is the set of
points $x \in X$ such that $\mathcal{F}_x \not = 0$.
\item We denote $\text{Supp}(\mathcal{F})$ the support of $\mathcal{F}$.
\item Let $s \in \Gamma(X, \mathcal{F})$ be a global section.
The {\it support of $s$} is the set of points $x \in X$
such that the image $s_x \in \mathcal{F}_x$ of $s$ is
not zero.
\end{enumerate}
\end{definition}

\noindent
Of course the support of a local section is then defined also
since a local section is a global section of the restriction of
$\mathcal{F}$.

\begin{lemma}
\label{lemma-support-section-closed}
Let $(X, \mathcal{O}_X)$ be a ringed space.
Let $\mathcal{F}$ be a sheaf of $\mathcal{O}_X$-modules.
Let $U \subset X$ open.
\begin{enumerate}
\item The support of $s \in \mathcal{F}(U)$ is closed in $U$.
\item The support of $fs$ is contained in the intersections
of the supports of $f \in \mathcal{O}_X(U)$ and $s \in \mathcal{F}(U)$.
\item The support of $s + s'$ is contained in the union of
the supports of $s, s' \in \mathcal{F}(U)$.
\item The support of $\mathcal{F}$ is the union of the supports
of all local sections of $\mathcal{F}$.
\item If $\varphi : \mathcal{F} \to \mathcal{G}$ is a morphism of
$\mathcal{O}_X$-modules, then the support of $\varphi(s)$ is
contained in the support of $s \in \mathcal{F}(U)$.
\end{enumerate}
\end{lemma}

\begin{proof}
This is true because if $s_x = 0$, then $s$ is zero
in an open neighbourhood of $x$ by definition of stalks.
Similarly for $f$. Details omitted.
\end{proof}

\noindent
In general the support of a sheaf of modules is not closed.
Namely, the sheaf could be an abelian sheaf on $\mathbf{R}$
(with the usual archimedean topology)
which is the direct sum of infinitely many nonzero skyscraper
sheaves each supported at a single point $p_i$ of $\mathbf{R}$.
Then the support would be the set of points $p_i$
which may not be closed.

\medskip\noindent
Another example is to consider the open immersion
$j : U = (0 , \infty) \to \mathbf{R} = X$, and the abelian sheaf
$j_!\underline{\mathbf{Z}}_U$. By Sheaves, Section
\ref{sheaves-section-open-immersions} the support of
this sheaf is exactly $U$.

\begin{lemma}
\label{lemma-support-sheaf-rings-closed}
Let $X$ be a topological space.
The support of a sheaf of rings is closed.
\end{lemma}

\begin{proof}
This is true because (according to our conventions)
a ring is $0$ if and only if
$1 = 0$, and hence the support of a sheaf of rings
is the support of the unit section.
\end{proof}






\section{Closed immersions and abelian sheaves}
\label{section-closed-immersions}

\noindent
Recall that we think of an abelian sheaf on a topological space $X$ as a
sheaf of $\underline{\mathbf{Z}}_X$-modules. Thus we may apply any results,
definitions for sheaves of modules to abelian sheaves.

\begin{lemma}
\label{lemma-i-star-exact}
Let $X$ be a topological space. Let $Z \subset X$ be a closed subset.
Denote $i : Z \to X$ the inclusion map. The functor
$$
i_* : \textit{Ab}(Z) \longrightarrow \textit{Ab}(X)
$$
is exact, fully faithful, with essential image exactly those
abelian sheaves whose support is contained in $Z$. The functor $i^{-1}$
is a left inverse to $i_*$.
\end{lemma}

\begin{proof}
Exactness follows from the description of
stalks in Sheaves, Lemma \ref{sheaves-lemma-stalks-closed-pushforward}
and Lemma \ref{lemma-abelian}. The rest was shown in
Sheaves, Lemma \ref{sheaves-lemma-equivalence-categories-closed-abelian}.
\end{proof}

\noindent
Let $\mathcal{F}$ be a sheaf on $X$. There is a canonical subsheaf
of $\mathcal{F}$ which consists of exactly those sections whose
support is contained in $Z$. Here is the exact statement.

\begin{lemma}
\label{lemma-sections-support-in-closed}
Let $X$ be a topological space. Let $Z \subset X$ be a closed subset.
Let $\mathcal{F}$ be a sheaf on $X$. For $U \subset X$ open set
$$
\Gamma(U, \mathcal{H}_Z(\mathcal{F})) =
\{s \in \mathcal{F}(U) \mid
\text{ the support of }s\text{ is contained in }Z \cap U\}
$$
Then $\mathcal{H}_Z(\mathcal{F})$ is an abelian subsheaf of
$\mathcal{F}$. It is the largest abelian subsheaf of
$\mathcal{F}$ whose support is contained in $Z$. The construction
$\mathcal{F} \mapsto \mathcal{H}_Z(\mathcal{F})$ is functorial
in the abelian sheaf $\mathcal{F}$.
\end{lemma}

\begin{proof}
This follows from Lemma \ref{lemma-support-section-closed}.
\end{proof}

\noindent
This seems like a good opportunity to show that the functor
$i_*$ has a right adjoint on abelian sheaves.

\begin{lemma}
\label{lemma-i-star-right-adjoint}
Let $i : Z \to X$ be the inclusion of a closed subset into the
topological space $X$.
Denote\footnote{This is likely nonstandard notation.}
$i^! : \textit{Ab}(X) \to \textit{Ab}(Z)$ the functor
$\mathcal{F} \mapsto i^{-1}\mathcal{H}_Z(\mathcal{F})$.
Then $i^!$ is a right adjoint to $i_*$, in a formula
$$
\Mor_{\textit{Ab}(X)}(i_*\mathcal{G}, \mathcal{F})
=
\Mor_{\textit{Ab}(Z)}(\mathcal{G}, i^!\mathcal{F}).
$$
In particular $i_*$ commutes with arbitrary colimits.
\end{lemma}

\begin{proof}
Note that $i_*i^!\mathcal{F} = \mathcal{H}_Z(\mathcal{F})$.
Since $i_*$ is fully faithful we are reduced to showing that
$$
\Mor_{\textit{Ab}(X)}(i_*\mathcal{G}, \mathcal{F})
=
\Mor_{\textit{Ab}(X)}(i_*\mathcal{G}, \mathcal{H}_Z(\mathcal{F})).
$$
This follows since the support of the image via any homomorphism
of a section of $i_*\mathcal{G}$ is contained in $Z$, see
Lemma \ref{lemma-support-section-closed}.
\end{proof}

\begin{remark}
\label{remark-i-star-right-adjoint}
In Sheaves, Remark \ref{sheaves-remark-i-star-not-exact}
we showed that $i_*$ as a functor
on the categories of sheaves of sets
does not have a right adjoint simply because
it is not exact. However, it is very close to being
true, in fact, the functor $i_*$ is exact on sheaves
of pointed sets, sections with support in $Z$ can
be defined for sheaves of pointed sets, and $i^!$
makes sense and is a right adjoint to $i_*$.
\end{remark}









\section{A canonical exact sequence}
\label{section-canonical-exact-sequence}

\noindent
We give this exact sequence its own section.

\begin{lemma}
\label{lemma-canonical-exact-sequence}
Let $X$ be a topological space.
Let $U \subset X$ be an open subset with complement $Z \subset X$.
Denote $j : U \to X$ the open immersion and
$i : Z \to X$ the closed immersion.
For any sheaf of abelian groups $\mathcal{F}$ on $X$
the adjunction mappings $j_{!}j^*\mathcal{F} \to \mathcal{F}$ and
$\mathcal{F} \to i_*i^*\mathcal{F}$ give a short exact
sequence
$$
0 \to j_{!}j^*\mathcal{F} \to \mathcal{F} \to i_*i^*\mathcal{F} \to 0
$$
of sheaves of abelian groups. For any morphism
$\varphi : \mathcal{F} \to \mathcal{G}$ of abelian sheaves on $X$
we obtain a morphism of short exact sequences
$$
\xymatrix{
0 \ar[r] &
j_{!}j^*\mathcal{F} \ar[r] \ar[d] &
\mathcal{F} \ar[r] \ar[d] &
i_*i^*\mathcal{F} \ar[r] \ar[d] &
0 \\
0 \ar[r] &
j_{!}j^*\mathcal{G} \ar[r] &
\mathcal{G} \ar[r] &
i_*i^*\mathcal{G} \ar[r] &
0
}
$$
\end{lemma}

\begin{proof}
The functoriality of the short exact sequence is
immediate from the naturality of the adjunction mappings.
We may check exactness on stalks (Lemma \ref{lemma-abelian}).
For a description of the stalks in question see
Sheaves, Lemmas \ref{sheaves-lemma-j-shriek-abelian}
and \ref{sheaves-lemma-stalks-closed-pushforward}.
\end{proof}








\section{Modules locally generated by sections}
\label{section-locally-generated}

\noindent
Let $(X, \mathcal{O}_X)$ be a ringed space.
In this and the following section we will often restrict
sheaves to open subspaces $U \subset X$, see
Sheaves, Section \ref{sheaves-section-open-immersions}.
In particular, we will often denote the open subspace
by $(U, \mathcal{O}_U)$ instead of the more correct
notation $(U, \mathcal{O}_X|_U)$, see
Sheaves, Definition \ref{sheaves-definition-restriction}.

\medskip\noindent
Consider the open immersion
$j : U = (0 , \infty) \to \mathbf{R} = X$, and the abelian sheaf
$j_!\underline{\mathbf{Z}}_U$. By Sheaves, Section
\ref{sheaves-section-open-immersions} the stalk of
$j_!\underline{\mathbf{Z}}_U$ at $x = 0$ is $0$. In fact the
sections of this sheaf over any open interval containing $0$
are $0$. Thus there is no open neighbourhood of the point
$0$ over which the sheaf can be generated by sections.

\begin{definition}
\label{definition-locally-generated}
Let $(X, \mathcal{O}_X)$ be a ringed space.
Let $\mathcal{F}$ be a sheaf of $\mathcal{O}_X$-modules.
We say that $\mathcal{F}$ is {\it locally generated by sections}
if for every $x \in X$ there exists an open
neighbourhood $U$ such that $\mathcal{F}|_U$
is globally generated as a sheaf of $\mathcal{O}_U$-modules.
\end{definition}

\noindent
In other words there exists a set $I$ and for
each $i$ a section $s_i \in \mathcal{F}(U)$ such
that the associated map
$$
\bigoplus\nolimits_{i \in I} \mathcal{O}_U
\longrightarrow
\mathcal{F}|_U
$$
is surjective.

\begin{lemma}
\label{lemma-pullback-locally-generated}
Let $f : (X, \mathcal{O}_X) \to (Y, \mathcal{O}_Y)$
be a morphism of ringed spaces.
The pullback $f^*\mathcal{G}$ is locally generated by sections
if $\mathcal{G}$ is locally generated by sections.
\end{lemma}

\begin{proof}
Given an open subspace $V$ of $Y$ we may
consider the commutative diagram of ringed spaces
$$
\xymatrix{
(f^{-1}V, \mathcal{O}_{f^{-1}V}) \ar[r]_{j'} \ar[d]_{f'} &
(X, \mathcal{O}_X) \ar[d]^f \\
(V, \mathcal{O}_V) \ar[r]^j &
(Y, \mathcal{O}_Y)
}
$$
We know that $f^*\mathcal{G}|_{f^{-1}V} \cong (f')^*(\mathcal{G}|_V)$,
see Sheaves, Lemma \ref{sheaves-lemma-push-pull-composition-modules}.
Thus we may assume that $\mathcal{G}$ is globally generated.

\medskip\noindent
We have seen that $f^*$ commutes with all colimits,
and is right exact, see Lemma \ref{lemma-exactness-pushforward-pullback}.
Thus if we have a surjection
$$
\bigoplus\nolimits_{i \in I}
\mathcal{O}_Y
\to
\mathcal{G}
\to
0
$$
then upon applying $f^*$ we obtain the surjection
$$
\bigoplus\nolimits_{i \in I}
\mathcal{O}_X
\to
f^*\mathcal{G}
\to
0.
$$
This implies the lemma.
\end{proof}












\section{Modules of finite type}
\label{section-finite-type}

\begin{definition}
\label{definition-finite-type}
Let $(X, \mathcal{O}_X)$ be a ringed space.
Let $\mathcal{F}$ be a sheaf of $\mathcal{O}_X$-modules.
We say that $\mathcal{F}$ is of {\it finite type}
if for every $x \in X$ there exists an open
neighbourhood $U$ such that $\mathcal{F}|_U$
is generated by finitely many sections.
\end{definition}

\begin{lemma}
\label{lemma-pullback-finite-type}
Let $f : (X, \mathcal{O}_X) \to (Y, \mathcal{O}_Y)$
be a morphism of ringed spaces.
The pullback $f^*\mathcal{G}$ of a finite type
$\mathcal{O}_Y$-module is a finite type $\mathcal{O}_X$-module.
\end{lemma}

\begin{proof}
Arguing as in the proof of Lemma \ref{lemma-pullback-locally-generated}
we may assume $\mathcal{G}$ is globally generated by finitely
many sections.
We have seen that $f^*$ commutes with all colimits,
and is right exact, see Lemma \ref{lemma-exactness-pushforward-pullback}.
Thus if we have a surjection
$$
\bigoplus\nolimits_{i = 1, \ldots, n}
\mathcal{O}_Y
\to
\mathcal{G}
\to
0
$$
then upon applying $f^*$ we obtain the surjection
$$
\bigoplus\nolimits_{i = 1, \ldots, n}
\mathcal{O}_X
\to
f^*\mathcal{G}
\to
0.
$$
This implies the lemma.
\end{proof}

\begin{lemma}
\label{lemma-extension-finite-type}
Let $X$ be a ringed space.
The image of a morphism of $\mathcal{O}_X$-modules of finite
type is of finite type.
Let
$0 \to \mathcal{F}_1 \to \mathcal{F}_2 \to \mathcal{F}_3 \to 0$
be a short exact sequence of $\mathcal{O}_X$-modules.
If $\mathcal{F}_1$ and $\mathcal{F}_3$ are of finite type,
so is $\mathcal{F}_2$.
\end{lemma}

\begin{proof}
The statement on images is trivial.
The statement on short exact sequences comes from the
fact that sections of $\mathcal{F}_3$ locally lift to sections
of $\mathcal{F}_2$ and the corresponding result in
the category of modules over a ring (applied to the stalks
for example).
\end{proof}

\begin{lemma}
\label{lemma-finite-type-surjective-on-stalk}
Let $X$ be a ringed space.
Let $\varphi : \mathcal{G} \to \mathcal{F}$ be a homomorphism
of $\mathcal{O}_X$-modules.
Let $x \in X$. Assume $\mathcal{F}$ of finite type and
the map on stalks
$\varphi_x : \mathcal{G}_x \to \mathcal{F}_x$ surjective.
Then there exists an open neighbourhood
$x \in U \subset X$ such that $\varphi|_U$ is surjective.
\end{lemma}

\begin{proof}
Choose an open neighbourhood $U \subset X$ of $x$ such that $\mathcal{F}$ is
generated by $s_1, \ldots, s_n \in \mathcal{F}(U)$ over $U$.
By assumption of surjectivity of $\varphi_x$,
after shrinking $U$ we may assume that $s_i = \varphi(t_i)$
for some $t_i \in \mathcal{G}(U)$.
Then $U$ works.
\end{proof}

\begin{lemma}
\label{lemma-finite-type-stalk-zero}
Let $X$ be a ringed space.
Let $\mathcal{F}$ be an $\mathcal{O}_X$-module.
Let $x \in X$.
Assume $\mathcal{F}$ of finite type and $\mathcal{F}_x = 0$.
Then there exists an open neighbourhood
$x \in U \subset X$ such that $\mathcal{F}|_U$ is zero.
\end{lemma}

\begin{proof}
This is a special case of
Lemma \ref{lemma-finite-type-surjective-on-stalk}
applied to the morphism $0 \to \mathcal{F}$.
\end{proof}

\begin{lemma}
\label{lemma-support-finite-type-closed}
\begin{slogan}
Over any ringed space, sheaves of modules of finite type have closed support.
\end{slogan}
Let $(X, \mathcal{O}_X)$ be a ringed space.
Let $\mathcal{F}$ be a sheaf of $\mathcal{O}_X$-modules.
If $\mathcal{F}$ is of finite type then support of $\mathcal{F}$ is closed.
\end{lemma}

\begin{proof}
This is a reformulation of Lemma \ref{lemma-finite-type-stalk-zero}.
\end{proof}

\begin{lemma}
\label{lemma-finite-type-quasi-compact-colimit}
Let $X$ be a ringed space. Let $I$ be a preordered set and
let $(\mathcal{F}_i, f_{ii'})$ be a system over $I$ consisting of sheaves
of $\mathcal{O}_X$-modules (see
Categories, Section \ref{categories-section-posets-limits}).
Let $\mathcal{F} = \colim \mathcal{F}_i$ be the colimit. Assume
(a) $I$ is directed,
(b) $\mathcal{F}$ is a finite type $\mathcal{O}_X$-module, and
(c) $X$ is quasi-compact. Then there exists an $i$ such that
$\mathcal{F}_i \to \mathcal{F}$ is surjective.
If the transition maps $f_{ii'}$ are injective
then we conclude that $\mathcal{F} = \mathcal{F}_i$ for some $i \in I$.
\end{lemma}

\begin{proof}
Let $x \in X$. There exists an open neighbourhood
$U \subset X$ of $x$ and finitely many sections
$s_j \in \mathcal{F}(U)$, $j = 1, \ldots, m$ such that
$s_1, \ldots, s_m$ generate $\mathcal{F}$ as $\mathcal{O}_U$-module.
After possibly shrinking $U$ to a smaller open neighbourhood of $x$
we may assume that each $s_j$ comes from a section of $\mathcal{F}_i$
for some $i \in I$.
Hence, since $X$ is quasi-compact we can find a finite open
covering $X = \bigcup_{j = 1, \ldots, m} U_j$, and for each $j$
an index $i_j$ and finitely many sections $s_{jl} \in \mathcal{F}_{i_j}(U_j)$
whose images generate the restriction of $\mathcal{F}$ to
$U_j$. Clearly, the lemma holds for any index $i \in I$ which
is $\geq$ all $i_j$.
\end{proof}

\begin{lemma}
\label{lemma-set-isomorphism-classes-finite-type-modules}
Let $X$ be a ringed space.
There exists a set of $\mathcal{O}_X$-modules
$\{\mathcal{F}_i\}_{i \in I}$ of finite type
such that each finite type $\mathcal{O}_X$-module
on $X$ is isomorphic to exactly one of the $\mathcal{F}_i$.
\end{lemma}

\begin{proof}
For each open covering $\mathcal{U} : X = \bigcup U_j$ consider the
sheaves of $\mathcal{O}_X$-modules $\mathcal{F}$ such that each
restriction $\mathcal{F}|_{U_j}$ is a quotient of
$\mathcal{O}_{U_j}^{\oplus r}$ for some $r_j \geq 0$.
These are parametrized by subsheaves
$\mathcal{K}_i \subset \mathcal{O}_{U_j}^{\oplus r_j}$ and glueing
data
$$
\varphi_{jj'} :
\mathcal{O}_{U_j \cap U_{j'}}^{\oplus r_j}/
(\mathcal{K}_j|_{U_j \cap U_{j'}})
\longrightarrow
\mathcal{O}_{U_j \cap U_{j'}}^{\oplus r_{j'}}/
(\mathcal{K}_{j'}|_{U_j \cap U_{j'}})
$$
see Sheaves, Section \ref{sheaves-section-glueing-sheaves}.
Note that the collection of all glueing data forms a set.
The collection of all coverings $\mathcal{U} : X = \bigcup_{j \in J} U_i$
where $J \to \mathcal{P}(X)$, $j \mapsto U_j$ is injective forms a set as
well. Hence the collection of all sheaves of $\mathcal{O}_X$-modules
gotten from glueing quotients as above forms a set $\mathcal{I}$.
By definition every finite type $\mathcal{O}_X$-module
is isomorphic to an element of $\mathcal{I}$. Choosing an
element out of each isomorphism class inside $\mathcal{I}$
gives the desired set of sheaves (uses axiom of choice).
\end{proof}













\section{Quasi-coherent modules}
\label{section-quasi-coherent}

\noindent
In this section we introduce an abstract notion of
quasi-coherent $\mathcal{O}_X$-module. This notion is very
useful in algebraic geometry, since quasi-coherent modules
on a scheme have a good description on any affine open.
However, we warn the reader that
in the general setting of (locally) ringed spaces
this notion is not well behaved at all. The category of
quasi-coherent sheaves is not abelian in general, infinite
direct sums of quasi-coherent sheaves aren't quasi-coherent, etc, etc.

\begin{definition}
\label{definition-quasi-coherent}
Let $(X, \mathcal{O}_X)$ be a ringed space.
Let $\mathcal{F}$ be a sheaf of $\mathcal{O}_X$-modules.
We say that $\mathcal{F}$ is a {\it quasi-coherent
sheaf of $\mathcal{O}_X$-modules} if for every
point $x \in X$ there exists an open neighbourhood
$x\in U \subset X$ such that $\mathcal{F}|_U$
is isomorphic to the cokernel of a map
$$
\bigoplus\nolimits_{j \in J}
\mathcal{O}_U
\longrightarrow
\bigoplus\nolimits_{i \in I}
\mathcal{O}_U
$$
The category of quasi-coherent $\mathcal{O}_X$-modules
is denoted $\QCoh(\mathcal{O}_X)$.
\end{definition}

\noindent
The definition means that $X$ is covered by open sets $U$
such that $\mathcal{F}|_U$ has a {\it presentation}
of the form
$$
\bigoplus\nolimits_{j \in J}
\mathcal{O}_U
\longrightarrow
\bigoplus\nolimits_{i \in I}
\mathcal{O}_U
\longrightarrow
\mathcal{F}|_U
\longrightarrow
0.
$$
Here presentation signifies that the displayed
sequence is exact. In other words
\begin{enumerate}
\item for every point $x$ of $X$ there exists
an open neighbourhood such that $\mathcal{F}|_U$
is generated by global sections, and
\item for a suitable choice of these sections
the kernel of the associated surjection is also
generated by global sections.
\end{enumerate}

\begin{lemma}
\label{lemma-direct-sum-quasi-coherent}
Let $(X, \mathcal{O}_X)$ be a ringed space.
The direct sum of two quasi-coherent $\mathcal{O}_X$-modules is
a quasi-coherent $\mathcal{O}_X$-module.
\end{lemma}

\begin{proof}
Omitted.
\end{proof}

\begin{remark}
\label{remark-infinite-direct-sum-quasi-coherent-not}
Warning: It is not true in general that an infinite
direct sum of quasi-coherent $\mathcal{O}_X$-modules
is quasi-coherent. For more esoteric behaviour of quasi-coherent
modules see Example \ref{example-quasi-coherent}.
\end{remark}

\begin{lemma}
\label{lemma-pullback-quasi-coherent}
Let $f : (X, \mathcal{O}_X) \to (Y, \mathcal{O}_Y)$
be a morphism of ringed spaces.
The pullback $f^*\mathcal{G}$ of a quasi-coherent
$\mathcal{O}_Y$-module is quasi-coherent.
\end{lemma}

\begin{proof}
Arguing as in the proof of Lemma \ref{lemma-pullback-locally-generated}
we may assume $\mathcal{G}$ has a global presentation by
direct sums of copies of $\mathcal{O}_Y$.
We have seen that $f^*$ commutes with all colimits,
and is right exact, see Lemma \ref{lemma-exactness-pushforward-pullback}.
Thus if we have an exact sequence
$$
\bigoplus\nolimits_{j \in J}
\mathcal{O}_Y
\longrightarrow
\bigoplus\nolimits_{i \in I}
\mathcal{O}_Y
\longrightarrow
\mathcal{G}
\longrightarrow
0
$$
then upon applying $f^*$ we obtain the exact sequence
$$
\bigoplus\nolimits_{j \in J}
\mathcal{O}_X
\longrightarrow
\bigoplus\nolimits_{i \in I}
\mathcal{O}_X
\longrightarrow
f^*\mathcal{G}
\longrightarrow
0.
$$
This implies the lemma.
\end{proof}

\noindent
This gives plenty of examples of quasi-coherent sheaves.

\begin{lemma}
\label{lemma-construct-quasi-coherent-sheaves}
Let $(X, \mathcal{O}_X)$ be ringed space.
Let $\alpha : R \to \Gamma(X, \mathcal{O}_X)$ be a ring homomorphism from
a ring $R$ into the ring of global sections on $X$.
Let $M$ be an $R$-module.
The following three constructions give canonically isomorphic
sheaves of $\mathcal{O}_X$-modules:
\begin{enumerate}
\item Let $\pi : (X, \mathcal{O}_X) \longrightarrow (\{*\}, R)$
be the morphism of ringed spaces with $\pi : X \to \{*\}$
the unique map and with $\pi$-map $\pi^\sharp$ the given map
$\alpha : R \to \Gamma(X, \mathcal{O}_X)$. Set $\mathcal{F}_1 = \pi^*M$.
\item Choose a presentation
$\bigoplus_{j \in J} R \to \bigoplus_{i \in I} R \to M \to 0$.
Set
$$
\mathcal{F}_2 = \Coker\left(
\bigoplus\nolimits_{j \in J} \mathcal{O}_X
\to
\bigoplus\nolimits_{i \in I} \mathcal{O}_X
\right).
$$
Here the map on the component $\mathcal{O}_X$ corresponding to $j \in J$
given by the section $\sum_i \alpha(r_{ij})$ where the $r_{ij}$
are the matrix coefficients of the map in the presentation of $M$.
\item Set $\mathcal{F}_3$ equal to the sheaf associated to the presheaf
$U \mapsto \mathcal{O}_X(U) \otimes_R M$, where the map
$R \to \mathcal{O}_X(U)$ is the composition of $\alpha$ and
the restriction map $\mathcal{O}_X(X) \to \mathcal{O}_X(U)$.
\end{enumerate}
This construction has the following properties:
\begin{enumerate}
\item The resulting sheaf of $\mathcal{O}_X$-modules
$\mathcal{F}_M = \mathcal{F}_1 = \mathcal{F}_2 = \mathcal{F}_3$
is quasi-coherent.
\item The construction gives a functor from
the category of $R$-modules to the category of quasi-coherent
sheaves on $X$ which commutes with arbitrary colimits.
\item For any $x \in X$ we have
$\mathcal{F}_{M, x} = \mathcal{O}_{X, x} \otimes_R M$
functorial in $M$.
\item Given any $\mathcal{O}_X$-module
$\mathcal{G}$ we have
$$
\Mor_{\mathcal{O}_X}(\mathcal{F}_M, \mathcal{G})
=
\Hom_R(M, \Gamma(X, \mathcal{G}))
$$
where the $R$-module structure on $\Gamma(X, \mathcal{G})$
comes from the $\Gamma(X, \mathcal{O}_X)$-module structure via
$\alpha$.
\end{enumerate}
\end{lemma}

\begin{proof}
The isomorphism between $\mathcal{F}_1$ and $\mathcal{F}_3$
comes from the fact that $\pi^*$ is defined as the sheafification
of the presheaf in (3), see Sheaves, Section
\ref{sheaves-section-ringed-spaces-functoriality-modules}.
The isomorphism between the constructions in (2) and (1) comes
from the fact that the functor $\pi^*$ is right exact, so
$\pi^*(\bigoplus_{j \in J} R) \to \pi^*(\bigoplus_{i \in I} R) \to
\pi^*M \to 0$ is exact, $\pi^*$ commutes with arbitrary
direct sums, see Lemma \ref{lemma-exactness-pushforward-pullback},
and finally the fact that $\pi^*(R) = \mathcal{O}_X$.

\medskip\noindent
Assertion (1) is clear from construction (2).
Assertion (2) is clear since $\pi^*$ has these properties.
Assertion (3) follows from the description of stalks of
pullback sheaves, see
Sheaves, Lemma \ref{sheaves-lemma-stalk-pullback-modules}.
Assertion (4) follows from adjointness of $\pi_*$ and
$\pi^*$.
\end{proof}

\begin{definition}
\label{definition-sheaf-associated}
In the situation of Lemma \ref{lemma-construct-quasi-coherent-sheaves}
we say $\mathcal{F}_M$ is the {\it sheaf associated to the module $M$
and the ring map $\alpha$}. If $R = \Gamma(X, \mathcal{O}_X)$
and $\alpha = \text{id}_R$ we simply say $\mathcal{F}_M$ is the
{\it sheaf associated to the module $M$}.
\end{definition}


\begin{lemma}
\label{lemma-restrict-quasi-coherent}
Let $(X, \mathcal{O}_X)$ be a ringed space.
Set $R = \Gamma(X, \mathcal{O}_X)$.
Let $M$ be an $R$-module.
Let $\mathcal{F}_M$ be the quasi-coherent sheaf of
$\mathcal{O}_X$-modules associated to $M$.
If $g : (Y, \mathcal{O}_Y) \to (X, \mathcal{O}_X)$
is a morphism of ringed spaces, then
$g^*\mathcal{F}_M$ is the sheaf associated
to the $\Gamma(Y, \mathcal{O}_Y)$-module
$\Gamma(Y, \mathcal{O}_Y) \otimes_R M$.
\end{lemma}

\begin{proof}
The assertion follows from the first description
of $\mathcal{F}_M$ in Lemma \ref{lemma-construct-quasi-coherent-sheaves}
as $\pi^*M$, and the following commutative diagram
of ringed spaces
$$
\xymatrix{
(Y, \mathcal{O}_Y) \ar[r]_-\pi \ar[d]_g &
(\{*\}, \Gamma(Y, \mathcal{O}_Y)) \ar[d]^{\text{induced by }g^\sharp} \\
(X, \mathcal{O}_X) \ar[r]^-\pi &
(\{*\}, \Gamma(X, \mathcal{O}_X))
}
$$
(Also use Sheaves, Lemma \ref{sheaves-lemma-push-pull-composition-modules}.)
\end{proof}

\begin{lemma}
\label{lemma-quasi-coherent-module}
Let $(X, \mathcal{O}_X)$ be a ringed space.
Let $x \in X$ be a point.
Assume that $x$ has a fundamental system of quasi-compact neighbourhoods.
Consider any quasi-coherent $\mathcal{O}_X$-module $\mathcal{F}$.
Then there exists an open neighbourhood $U$ of $x$
such that $\mathcal{F}|_U$ is isomorphic to the
sheaf of modules $\mathcal{F}_M$ on $(U, \mathcal{O}_U)$
associated to some $\Gamma(U, \mathcal{O}_U)$-module $M$.
\end{lemma}

\begin{proof}
First we may replace $X$ by an open neighbourhood of $x$
and assume that $\mathcal{F}$ is isomorphic to the
cokernel of a map
$$
\Psi :
\bigoplus\nolimits_{j \in J}
\mathcal{O}_X
\longrightarrow
\bigoplus\nolimits_{i \in I}
\mathcal{O}_X.
$$
The problem is that this map may not be given by
a ``matrix'', because the module of global sections of a direct sum
is in general different from the direct sum of the modules
of global sections.

\medskip\noindent
Let $x \in E \subset X$ be a quasi-compact
neighbourhood of $x$ (note: $E$ may not be open).
Let $x \in U \subset E$ be an open neighbourhood of
$x$ contained in $E$.
Next, we proceed as in the proof of
Lemma \ref{lemma-section-direct-sum-quasi-compact}.
For each $j \in J$ denote
$s_j \in \Gamma(X, \bigoplus\nolimits_{i \in I} \mathcal{O}_X)$
the image of the section $1$ in the summand $\mathcal{O}_X$
corresponding to $j$. There exists a finite collection of opens
$U_{jk}$, $k \in K_j$ such that $E \subset \bigcup_{k \in K_j} U_{jk}$
and such that each restriction $s_j|_{U_{jk}}$
is a finite sum $\sum_{i \in I_{jk}} f_{jki}$
with $I_{jk} \subset I$, and $f_{jki}$ in the summand
$\mathcal{O}_X$ corresponding to $i \in I$. Set
$I_j = \bigcup_{k \in k_J} I_{jk}$. This is a finite set.
Since $U \subset E \subset \bigcup_{k \in K_j} U_{jk}$
the section $s_j|_U$ is a section of the finite direct sum
$\bigoplus_{i \in I_j} \mathcal{O}_X$.
By Lemma \ref{lemma-limits-colimits}
we see that actually $s_j|_U$ is a sum
$\sum_{i \in I_j} f_{ij}$ and
$f_{ij} \in \mathcal{O}_X(U) = \Gamma(U, \mathcal{O}_U)$.

\medskip\noindent
At this point we can define a module $M$ as the cokernel of the map
$$
\bigoplus\nolimits_{j \in J}
\Gamma(U, \mathcal{O}_U)
\longrightarrow
\bigoplus\nolimits_{i \in I}
\Gamma(U, \mathcal{O}_U)
$$
with matrix given by the $(f_{ij})$. By construction (2) of
Lemma \ref{lemma-construct-quasi-coherent-sheaves} we see that
$\mathcal{F}_M$ has the same presentation as $\mathcal{F}|_U$
and therefore $\mathcal{F}_M \cong \mathcal{F}|_U$.
\end{proof}

\begin{example}
\label{example-quasi-coherent}
Let $X$ be countably many copies $L_1, L_2, L_3, \ldots$
of the real line all glued together at $0$; a fundamental
system of neighbourhoods of $0$ being the collection
$\{U_n\}_{n \in \mathbf{N}}$, with $U_n \cap L_i = (-1/n, 1/n)$.
Let $\mathcal{O}_X$ be the sheaf of continuous real valued functions.
Let $f : \mathbf{R} \to \mathbf{R}$ be a continuous function
which is identically zero on $(-1, 1)$ and identically $1$
on $(-\infty, -2) \cup (2, \infty)$. Denote $f_n$ the continuous
function on $X$ which is equal to $x \mapsto f(nx)$ on each
$L_j = \mathbf{R}$. Let $1_{L_j}$ be the characteristic function
of $L_j$. We consider the map
$$
\bigoplus\nolimits_{j \in \mathbf{N}}
\mathcal{O}_X
\longrightarrow
\bigoplus\nolimits_{j, i \in \mathbf{N}}
\mathcal{O}_X, \quad
e_j \longmapsto \sum\nolimits_{i \in \mathbf{N}} f_j 1_{L_i} e_{ij}
$$
with obvious notation. This makes sense because this sum is locally
finite as $f_j$ is zero in a neighbourhood of $0$. Over $U_n$
the image of $e_j$, for $j > 2n$ is not a finite linear combination
$\sum g_{ij} e_{ij}$ with $g_{ij}$ continuous. Thus there is
no neighbourhood of $0 \in X$ such that the displayed map is given by
a ``matrix'' as in the proof of Lemma \ref{lemma-quasi-coherent-module} above.

\medskip\noindent
Note that $\bigoplus\nolimits_{j \in \mathbf{N}} \mathcal{O}_X$
is the sheaf associated to the free module with basis $e_j$
and similarly for the other direct sum. Thus we see that a
morphism of sheaves associated to modules in general even
locally on $X$ does not come from a morphism of modules.
Similarly there should be an example of a ringed space $X$
and a quasi-coherent $\mathcal{O}_X$-module $\mathcal{F}$
such that $\mathcal{F}$ is not locally of the form $\mathcal{F}_M$.
(Please email if you find one.)
Moreover, there should be examples of locally compact spaces
$X$ and maps $\mathcal{F}_M \to \mathcal{F}_N$ which also do
not locally come from maps of modules (the proof of Lemma
\ref{lemma-quasi-coherent-module} shows this cannot happen
if $N$ is free).
\end{example}












\section{Modules of finite presentation}
\label{section-finite-presentation}


\begin{definition}
\label{definition-finite-presentation}
Let $(X, \mathcal{O}_X)$ be a ringed space.
Let $\mathcal{F}$ be a sheaf of $\mathcal{O}_X$-modules.
We say that $\mathcal{F}$ is of {\it finite presentation}
if for every point $x \in X$ there exists an open neighbourhood
$x\in U \subset X$, and  $n, m \in \mathbf{N}$ such that $\mathcal{F}|_U$
is isomorphic to the cokernel of a map
$$
\bigoplus\nolimits_{j = 1, \ldots, m}
\mathcal{O}_U
\longrightarrow
\bigoplus\nolimits_{i = 1, \ldots, n}
\mathcal{O}_U
$$
\end{definition}

\noindent
This means that $X$ is covered by open sets $U$
such that $\mathcal{F}|_U$ has a {\it presentation}
of the form
$$
\bigoplus\nolimits_{j = 1, \ldots, m}
\mathcal{O}_U
\longrightarrow
\bigoplus\nolimits_{i = 1, \ldots, n}
\mathcal{O}_U
\to
\mathcal{F}|_U
\to
0.
$$
Here presentation signifies that the displayed
sequence is exact. In other words
\begin{enumerate}
\item for every point $x$ of $X$ there exists
an open neighbourhood such that $\mathcal{F}|_U$
is generated by finitely many global sections, and
\item for a suitable choice of these sections
the kernel of the associated surjection is also
generated by finitely many global sections.
\end{enumerate}

\begin{lemma}
\label{lemma-finite-presentation-quasi-coherent}
Let $(X, \mathcal{O}_X)$ be a ringed space.
Any $\mathcal{O}_X$-module of finite presentation
is quasi-coherent.
\end{lemma}

\begin{proof}
Immediate from definitions.
\end{proof}

\begin{lemma}
\label{lemma-kernel-surjection-finite-free-onto-finite-presentation}
Let $(X, \mathcal{O}_X)$ be a ringed space.
Let $\mathcal{F}$ be a $\mathcal{O}_X$-module of finite presentation.
\begin{enumerate}
\item If $\psi : \mathcal{O}_X^{\oplus r} \to \mathcal{F}$ is a surjection,
then $\Ker(\psi)$ is of finite type.
\item If $\theta : \mathcal{G} \to \mathcal{F}$ is surjective with
$\mathcal{G}$ of finite type, then $\Ker(\theta)$ is of finite type.
\end{enumerate}
\end{lemma}

\begin{proof}
Proof of (1).
Let $x \in X$. Choose an open neighbourhood $U \subset X$ of $x$
such that there exists a presentation
$$
\mathcal{O}_U^{\oplus m}
\xrightarrow{\chi}
\mathcal{O}_U^{\oplus n}
\xrightarrow{\varphi}
\mathcal{F}|_U
\to
0.
$$
Let $e_k$ be the section generating the $k$th factor of
$\mathcal{O}_X^{\oplus r}$. For every $k = 1, \ldots, r$
we can, after shrinking $U$ to a small neighbourhood of $x$,
lift $\psi(e_k)$ to a section $\tilde e_k$ of
$\mathcal{O}_U^{\oplus n}$ over $U$. This gives a morphism
of sheaves $\alpha : \mathcal{O}_U^{\oplus r} \to \mathcal{O}_U^{\oplus n}$
such that $\varphi \circ \alpha = \psi$.
Similarly, after shrinking $U$, we can find a morphism
$\beta : \mathcal{O}_U^{\oplus n} \to \mathcal{O}_U^{\oplus r}$
such that $\psi \circ \beta = \varphi$. Then the map
$$
\mathcal{O}_U^{\oplus m} \oplus
\mathcal{O}_U^{\oplus r}
\xrightarrow{\beta \circ \chi, 1 - \beta \circ \alpha}
\mathcal{O}_U^{\oplus r}
$$
is a surjection onto the kernel of $\psi$.

\medskip\noindent
To prove (2) we may locally choose a surjection
$\eta : \mathcal{O}_X^{\oplus r} \to \mathcal{G}$.
By part (1) we see $\Ker(\theta \circ \eta)$ is of finite type.
Since $\Ker(\theta) = \eta(\Ker(\theta \circ \eta))$ we win.
\end{proof}

\begin{lemma}
\label{lemma-pullback-finite-presentation}
Let $f : (X, \mathcal{O}_X) \to (Y, \mathcal{O}_Y)$
be a morphism of ringed spaces.
The pullback $f^*\mathcal{G}$ of a module of finite presentation
is of finite presentation.
\end{lemma}

\begin{proof}
Exactly the same as the proof of Lemma \ref{lemma-pullback-quasi-coherent}
but with finite index sets.
\end{proof}

\begin{lemma}
\label{lemma-quasi-coherent-limit-finite-presentation}
Let $(X, \mathcal{O}_X)$ be a ringed space.
Set $R = \Gamma(X, \mathcal{O}_X)$.
Let $M$ be an $R$-module.
The $\mathcal{O}_X$-module $\mathcal{F}_M$ associated to $M$
is a directed colimit of finitely presented $\mathcal{O}_X$-modules.
\end{lemma}

\begin{proof}
This follows immediately from
Lemma \ref{lemma-construct-quasi-coherent-sheaves}
and the fact that any module is a directed colimit
of finitely presented modules, see
Algebra, Lemma \ref{algebra-lemma-module-colimit-fp}.
\end{proof}

\begin{lemma}
\label{lemma-finite-presentation-quasi-compact-colimit}
Let $X$ be a ringed space.
Let $I$ be a preordered set and
let $(\mathcal{F}_i, \varphi_{ii'})$ be a system over $I$
consisting of sheaves of $\mathcal{O}_X$-modules
(see Categories, Section \ref{categories-section-posets-limits}).
Assume
\begin{enumerate}
\item $I$ is directed,
\item $\mathcal{G}$ is an $\mathcal{O}_X$-module of finite presentation, and
\item $X$ has a cofinal system of open coverings
$\mathcal{U} : X = \bigcup_{j\in J} U_j$ with
$J$ finite and $U_j \cap U_{j'}$ quasi-compact
for all $j, j' \in J$.
\end{enumerate}
Then we have
$$
\colim_i \Hom_X(\mathcal{G}, \mathcal{F}_i)
=
\Hom_X(\mathcal{G}, \colim_i \mathcal{F}_i).
$$
\end{lemma}

\begin{proof}
Let $\alpha$ be an element of the right hand side.
For every point $x \in X$ we may choose an open neighbourhood
$U \subset X$ and finitely many sections
$s_j \in \mathcal{G}(U)$ which generate $\mathcal{G}$ over $U$
and finitely many relations $\sum f_{kj} s_j = 0$, $k = 1, \ldots, n$
with $f_{kj} \in \mathcal{O}_X(U)$ which generate the kernel
of $\bigoplus_{j = 1, \ldots, m} \mathcal{O}_U \to \mathcal{G}$.
After possibly shrinking $U$ to a smaller open neighbourhood of $x$
we may assume there exists an index $i \in I$ such that
the sections $\alpha(s_j)$ all come from sections
$s_j' \in \mathcal{F}_i(U)$.
After possibly shrinking $U$ to a smaller open neighbourhood of $x$
and increasing $i$ we may assume the relations
$\sum f_{kj} s'_j = 0$ hold in $\mathcal{F}_i(U)$.
Hence we see that $\alpha|_U$ lifts to a morphism
$\mathcal{G}|_U \to \mathcal{F}_i|_U$ for some index $i \in I$.

\medskip\noindent
By condition (3) and the preceding arguments, we may choose
a finite open covering $X = \bigcup_{j = 1, \ldots, m} U_j$
such that (a) $\mathcal{G}|_{U_j}$ is generated by finitely
many sections $s_{jk} \in \mathcal{G}(U_j)$, (b) the restriction
$\alpha|_{U_j}$ comes from a morphism
$\alpha_j : \mathcal{G} \to \mathcal{F}_{i_j}$
for some $i_j \in I$, and (c) the intersections
$U_j \cap U_{j'}$ are all quasi-compact.
For every pair $(j, j') \in \{1, \ldots, m\}^2$
and any $k$ we can find
we can find an index $i \geq \max(i_j, i_{j'})$ such
that
$$
\varphi_{i_ji}(\alpha_j(s_{jk}|_{U_j \cap U_{j'}})) =
\varphi_{i_{j'}i}(\alpha_{j'}(s_{jk}|_{U_j \cap U_{j'}}))
$$
see Sheaves, Lemma \ref{sheaves-lemma-directed-colimits-sections} (2).
Since there are finitely many of these pairs $(j, j')$ and
finitely many $s_{jk}$ we see that we can find a single $i$
which works for all of them. For this index $i$ all of the maps
$\varphi_{i_ji} \circ \alpha_j$ agree on the overlaps $U_j \cap U_{j'}$
as the sections $s_{jk}$ generate $\mathcal{G}$ over this overlap.
Hence we get a morphism $\mathcal{G} \to \mathcal{F}_i$ as desired.
\end{proof}

\begin{remark}
\label{remark-condition-necessary}
In the lemma above some condition beyond the condition that $X$
is quasi-compact is necessary. See
Sheaves, Example \ref{sheaves-example-conditions-needed-colimit}.
\end{remark}

\begin{lemma}
\label{lemma-finite-presentation-stalk-free}
Let $(X, \mathcal{O}_X)$ be a ringed space. Let $\mathcal{F}$ be
a finitely presented $\mathcal{O}_X$-module. Let $x \in X$ such that
$\mathcal{F}_x \cong \mathcal{O}_{X, x}^{\oplus r}$. Then there exists
an open neighbourhood $U$ of $x$ such that
$\mathcal{F}|_U \cong \mathcal{O}_U^{\oplus r}$.
\end{lemma}

\begin{proof}
Choose $s_1, \ldots, s_r \in \mathcal{F}_x$ mapping to a basis of
$\mathcal{O}_{X, x}^{\oplus r}$ by the isomorphism. Choose an open
neighbourhood $U$ of $x$ such that $s_i$ lifts to $s_i \in \mathcal{F}(U)$.
After shrinking $U$ we see that the induced map
$\psi : \mathcal{O}_U^{\oplus r} \to \mathcal{F}|_U$ is surjective
(Lemma \ref{lemma-finite-type-surjective-on-stalk}).
By Lemma \ref{lemma-kernel-surjection-finite-free-onto-finite-presentation}
we see that $\Ker(\psi)$ is of finite type.
Then $\Ker(\psi)_x = 0$ implies that $\Ker(\psi)$ becomes zero
after shrinking $U$ once more (Lemma \ref{lemma-finite-type-stalk-zero}).
\end{proof}





\section{Coherent modules}
\label{section-coherent}

\noindent
The category of coherent sheaves on a ringed space $X$
is a more reasonable object
than the category of quasi-coherent sheaves, in the sense
that it is at least an abelian subcategory of $\textit{Mod}(\mathcal{O}_X)$
no matter what $X$ is. On the other hand, the pullback of a
coherent module is ``almost never'' coherent in the general setting
of ringed spaces.

\begin{definition}
\label{definition-coherent}
Let $(X, \mathcal{O}_X)$ be a ringed space.
Let $\mathcal{F}$ be a sheaf of $\mathcal{O}_X$-modules.
We say that $\mathcal{F}$ is a {\it coherent $\mathcal{O}_X$-module}
if the following two conditions hold:
\begin{enumerate}
\item $\mathcal{F}$ is of finite type, and
\item for every open $U \subset X$ and every finite
collection $s_i \in \mathcal{F}(U)$, $i = 1, \ldots, n$
the kernel of the associated map
$\bigoplus_{i = 1, \ldots, n} \mathcal{O}_U \to \mathcal{F}|_U$
is of finite type.
\end{enumerate}
The category of coherent $\mathcal{O}_X$-modules is denoted
$\textit{Coh}(\mathcal{O}_X)$.
\end{definition}

\begin{lemma}
\label{lemma-coherent-finite-presentation}
Let $(X, \mathcal{O}_X)$ be a ringed space.
Any coherent $\mathcal{O}_X$-module is of finite presentation
and hence quasi-coherent.
\end{lemma}

\begin{proof}
Let $\mathcal{F}$ be a coherent sheaf on $X$.
Pick a point $x \in X$.
By (1) of the definition of coherent, we may find an open neighbourhood $U$
and sections $s_i$, $i = 1, \ldots, n$ of $\mathcal{F}$ over $U$
such that $\Psi : \bigoplus_{i = 1, \ldots, n} \mathcal{O}_U \to \mathcal{F}$
is surjective. By (2) of the definition of coherent, we may find
an open neighbourhood $V$, $x \in V \subset U$ and sections
$t_1, \ldots, t_m$ of $\bigoplus_{i = 1, \ldots, n} \mathcal{O}_V$
which generate the kernel of $\Psi|_V$. Then over $V$ we get the
presentation
$$
\bigoplus\nolimits_{j = 1, \ldots, m}
\mathcal{O}_V
\longrightarrow
\bigoplus\nolimits_{i = 1, \ldots, n}
\mathcal{O}_V
\to
\mathcal{F}|_V
\to
0
$$
as desired.
\end{proof}

\begin{example}
\label{example-coherent-not-Noetherian}
Suppose that $X$ is a point. In this case the definition
above gives a notion for modules over rings.
What does the definition of coherent mean?
It is closely related to the notion of Noetherian,
but it is not the same: Namely, the ring
$R = \mathbf{C}[x_1, x_2, x_3, \ldots]$ is coherent
as a module over itself but not Noetherian as a module
over itself. See Algebra, Section \ref{algebra-section-coherent}
for more discussion.
\end{example}

\begin{lemma}
\label{lemma-coherent-abelian}
Let $(X, \mathcal{O}_X)$ be a ringed space.
\begin{enumerate}
\item Any finite type subsheaf of a coherent sheaf is coherent.
\item Let $\varphi : \mathcal{F} \to \mathcal{G}$
be a morphism from a finite type sheaf $\mathcal{F}$
to a coherent sheaf $\mathcal{G}$. Then $\Ker(\varphi)$ is finite type.
\item Let $\varphi : \mathcal{F} \to \mathcal{G}$ be a morphism
of coherent $\mathcal{O}_X$-modules. Then
$\Ker(\varphi)$ and
$\Coker(\varphi)$ are coherent.
\item Given a short exact sequence of $\mathcal{O}_X$-modules
$0 \to \mathcal{F}_1 \to \mathcal{F}_2 \to \mathcal{F}_3 \to 0$
if two out of three are coherent so is the third.
\item The category $\textit{Coh}(\mathcal{O}_X)$ is a weak Serre subcategory
of $\textit{Mod}(\mathcal{O}_X)$. In particular, the category of
coherent modules is abelian and the inclusion functor
$\textit{Coh}(\mathcal{O}_X) \to \textit{Mod}(\mathcal{O}_X)$
is exact.
\end{enumerate}
\end{lemma}

\begin{proof}
Condition (2) of Definition \ref{definition-coherent}
holds for any subsheaf of a coherent sheaf. Thus we get (1).

\medskip\noindent
Assume the hypotheses of (2).
Let us show that $\Ker(\varphi)$ is of finite type. Pick $x \in X$.
Choose an open neighbourhood $U$ of $x$ in $X$ such
that $\mathcal{F}|_U$ is generated by $s_1, \ldots, s_n$.
By Definition \ref{definition-coherent} the kernel $\mathcal{K}$
of the induced map
$\bigoplus_{i = 1}^n \mathcal{O}_U \to \mathcal{G}$,
$e_i \mapsto \varphi(s_i)$ is of finite type.
Hence $\Ker(\varphi)$ which is the image of the
composition
$\mathcal{K} \to \bigoplus_{i = 1}^n \mathcal{O}_U \to \mathcal{F}$
is of finite type.

\medskip\noindent
Assume the hypotheses of (3).
By (2) the kernel of $\varphi$ is of finite type and
hence by (1) it is coherent.

\medskip\noindent
With the same hypotheses let us show that $\Coker(\varphi)$ is coherent.
Since $\mathcal{G}$ is of finite type so is $\Coker(\varphi)$.
Let $U \subset X$ be open and let $\overline{s}_i \in \Coker(\varphi)(U)$,
$i = 1, \ldots, n$ be sections. We have to show that the kernel of the
associated morphism
$\overline{\Psi} : \bigoplus_{i = 1}^n \mathcal{O}_U \to \Coker(\varphi)$
has finite type. There exists an open covering
of $U$ such that on each open all the sections $\overline{s}_i$
lift to sections $s_i$ of $\mathcal{G}$. Hence we may assume
this is the case over $U$. We may in addition assume there are
sections $t_j$, $j = 1, \ldots, m$ of $\Im(\varphi)$ over $U$
which generate $\Im(\varphi)$ over $U$.
Let $\Phi : \bigoplus_{j = 1}^m \mathcal{O}_U \to \Im(\varphi)$
be defined using $t_j$ and
$\Psi :
\bigoplus_{j = 1}^m \mathcal{O}_U \oplus
\bigoplus_{i = 1}^n \mathcal{O}_U \to \mathcal{G}$
using $t_j$ and $s_i$.
Consider the following commutative diagram
$$
\xymatrix{
0 \ar[r] &
\bigoplus_{j = 1}^m \mathcal{O}_U \ar[d]_\Phi \ar[r] &
\bigoplus_{j = 1}^m \mathcal{O}_U \oplus
\bigoplus_{i = 1}^n \mathcal{O}_U \ar[d]_\Psi \ar[r] &
\bigoplus_{i = 1}^n \mathcal{O}_U \ar[d]_{\overline{\Psi}} \ar[r] &
0 \\
0 \ar[r] &
\Im(\varphi) \ar[r] &
\mathcal{G} \ar[r] &
\Coker(\varphi) \ar[r] &
0
}
$$
By the snake lemma we get an exact sequence
$\Ker(\Psi) \to \Ker(\overline{\Psi}) \to 0$. Since $\Ker(\Psi)$
is a finite type module, we see that $\Ker(\overline{\Psi})$ has finite type.

\medskip\noindent
Proof of part (4).
Let $0 \to \mathcal{F}_1 \to \mathcal{F}_2 \to \mathcal{F}_3 \to 0$
be a short exact sequence of $\mathcal{O}_X$-modules. By part
(3) it suffices
to prove that if $\mathcal{F}_1$ and $\mathcal{F}_3$ are coherent
so is $\mathcal{F}_2$. By Lemma \ref{lemma-extension-finite-type} we
see that $\mathcal{F}_2$ has finite type. Let
$s_1, \ldots, s_n$ be finitely many local
sections of $\mathcal{F}_2$ defined over a common open $U$ of $X$.
We have to show that the module of relations $\mathcal{K}$
between them is of finite type.
Consider the following commutative diagram
$$
\xymatrix{
0 \ar[r] &
0 \ar[r] \ar[d] &
\bigoplus_{i = 1}^{n} \mathcal{O}_U \ar[r] \ar[d] &
\bigoplus_{i = 1}^{n} \mathcal{O}_U \ar[r] \ar[d] &
0 \\
0 \ar[r] &
\mathcal{F}_1 \ar[r] &
\mathcal{F}_2 \ar[r] &
\mathcal{F}_3 \ar[r] &
0
}
$$
with obvious notation. By the snake lemma
we get a short exact sequence
$0 \to \mathcal{K} \to \mathcal{K}_3 \to \mathcal{F}_1$
where $\mathcal{K}_3$ is the module of relations among
the images of the sections $s_i$ in $\mathcal{F}_3$.
Since $\mathcal{F}_3$ is coherent we see that
$\mathcal{K}_3$ is finite type. Since $\mathcal{F}_1$
is coherent we see that the image $\mathcal{I}$
of $\mathcal{K}_3 \to \mathcal{F}_1$
is coherent. Hence $\mathcal{K}$
is the kernel of the map $\mathcal{K}_3 \to \mathcal{I}$
between a finite type sheaf and a coherent sheaves and hence
finite type by (2).

\medskip\noindent
Proof of (5). This follows because (3) and (4) show that
Homology, Lemma \ref{homology-lemma-characterize-weak-serre-subcategory}
applies.
\end{proof}

\begin{lemma}
\label{lemma-coherent-structure-sheaf}
Let $(X, \mathcal{O}_X)$ be a ringed space.
Let $\mathcal{F}$ be an $\mathcal{O}_X$-module.
Assume $\mathcal{O}_X$ is a coherent $\mathcal{O}_X$-module.
Then $\mathcal{F}$ is coherent if and only if it is
of finite presentation.
\end{lemma}

\begin{proof}
Omitted.
\end{proof}

\begin{lemma}
\label{lemma-finite-type-to-coherent-injective-on-stalk}
Let $X$ be a ringed space.
Let $\varphi : \mathcal{G} \to \mathcal{F}$ be a homomorphism
of $\mathcal{O}_X$-modules.
Let $x \in X$. Assume $\mathcal{G}$ of finite type,
$\mathcal{F}$ coherent and the map on stalks
$\varphi_x : \mathcal{G}_x \to \mathcal{F}_x$ injective.
Then there exists an open neighbourhood
$x \in U \subset X$ such that $\varphi|_U$ is injective.
\end{lemma}

\begin{proof}
Denote $\mathcal{K} \subset \mathcal{G}$ the kernel of $\varphi$.
By Lemma \ref{lemma-coherent-abelian} we see that $\mathcal{K}$ is
a finite type $\mathcal{O}_X$-module. Our assumption is that
$\mathcal{K}_x = 0$. By Lemma \ref{lemma-finite-type-stalk-zero}
there exists an open neighbourhood $U$ of $x$ such that $\mathcal{K}|_U = 0$.
Then $U$ works.
\end{proof}













\section{Closed immersions of ringed spaces}
\label{section-closed-immersion}

\noindent
When do we declare a morphism of ringed spaces
$i : (Z, \mathcal{O}_Z) \to (X, \mathcal{O}_X)$
to be a closed immersion?

\medskip\noindent
Motivated by the example of a closed immersion of normal topological spaces
(ringed with the sheaf of continuous functors), or differential manifolds
(ringed with the sheaf of differentiable functions), it seems natural to
assume at least:
\begin{enumerate}
\item The map $i$ is a closed immersion of topological spaces.
\item The associated map $\mathcal{O}_X \to i_*\mathcal{O}_Z$
is surjective. Denote the kernel by $\mathcal{I}$.
\end{enumerate}
Already these conditions imply a number of pleasing results: For example
we prove that the category of $\mathcal{O}_Z$-modules is equivalent to
the category of $\mathcal{O}_X$-modules annihilated by $\mathcal{I}$
generalizing the result on abelian sheaves of
Section \ref{section-closed-immersions}

\medskip\noindent
However, in the Stacks project we choose the
definition that guarantees that if $i$ is a closed immersion
and $(X, \mathcal{O}_X)$ is a scheme, then also $(Z, \mathcal{O}_Z)$
is a scheme. Moreover, in this situation we want $i_*$ and $i^*$
to provide an equivalence between the category of quasi-coherent
$\mathcal{O}_Z$-modules and the category of quasi-coherent
$\mathcal{O}_X$-modules annihilated by $\mathcal{I}$.
A minimal condition is that $i_*\mathcal{O}_Z$ is a
quasi-coherent sheaf of $\mathcal{O}_X$-modules.
A good way to guarantee that $i_*\mathcal{O}_Z$ is a
quasi-coherent $\mathcal{O}_X$-module is to assume that
$\mathcal{I}$ is locally generated by sections.
We can interpret this condition as saying ``$(Z, \mathcal{O}_Z)$ is
locally on $(X, \mathcal{O}_X)$ defined by setting some regular functions
$f_i$, i.e., local sections of $\mathcal{O}_X$, equal to zero''.
This leads to the following definition.

\begin{definition}
\label{definition-closed-immersion}
A {\it closed immersion of ringed spaces}\footnote{This is
nonstandard notation; see discussion above.} is a morphism
$i : (Z, \mathcal{O}_Z) \to (X, \mathcal{O}_X)$
with the following properties:
\begin{enumerate}
\item The map $i$ is a closed immersion of topological spaces.
\item The associated map $\mathcal{O}_X \to i_*\mathcal{O}_Z$
is surjective. Denote the kernel by $\mathcal{I}$.
\item The $\mathcal{O}_X$-module $\mathcal{I}$ is locally
generated by sections.
\end{enumerate}
\end{definition}

\noindent
Actually, this definition still does not guarantee that
$i_*$ of a quasi-coherent $\mathcal{O}_Z$-module is a
quasi-coherent $\mathcal{O}_X$-module. The problem is that
it is not clear how to convert a local presentation of
a quasi-coherent $\mathcal{O}_Z$-module into a local
presentation for the pushforward. However, the following
is trivial.

\begin{lemma}
\label{lemma-i-star-quasi-coherent}
Let $i : (Z, \mathcal{O}_Z) \to (X, \mathcal{O}_X)$
be a closed immersion of ringed spaces.
Let $\mathcal{F}$ be a quasi-coherent $\mathcal{O}_Z$-module.
Then $i_*\mathcal{F}$ is locally on $X$ the cokernel of
a map of quasi-coherent $\mathcal{O}_X$-modules.
\end{lemma}

\begin{proof}
This is true because $i_*\mathcal{O}_Z$ is quasi-coherent
by definition. And locally on $Z$ the sheaf $\mathcal{F}$
is a cokernel of a map between direct sums of copies
of $\mathcal{O}_Z$. Moreover, any direct sum of copies of the
{\it the same} quasi-coherent sheaf is quasi-coherent.
And finally, $i_*$ commutes with arbitrary colimits,
see Lemma \ref{lemma-i-star-right-adjoint}. Some details omitted.
\end{proof}

\begin{lemma}
\label{lemma-i-star-reflects-finite-type}
Let $i : (Z, \mathcal{O}_Z) \to (X, \mathcal{O}_X)$ be a morphism of ringed
spaces. Assume $i$ is a homeomorphism onto a closed subset of $X$ and
that $\mathcal{O}_X \to i_*\mathcal{O}_Z$ is surjective.
Let $\mathcal{F}$ be an $\mathcal{O}_Z$-module.
Then $i_*\mathcal{F}$ is of finite type if and only if
$\mathcal{F}$ is of finite type.
\end{lemma}

\begin{proof}
Suppose that $\mathcal{F}$ is of finite type.
Pick $x \in X$. If $x \not \in Z$, then $i_*\mathcal{F}$
is zero in a neighbourhood of $x$ and hence finitely generated in
a neighbourhood of $x$. If $x = i(z)$, then choose an open neighbourhood
$z \in V \subset Z$ and sections $s_1, \ldots, s_n \in \mathcal{F}(V)$
which generate $\mathcal{F}$ over $V$. Write $V = Z \cap U$ for some open
$U \subset X$. Note that $U$ is a neighbourhood of $x$. Clearly the
sections $s_i$ give sections $s_i$ of $i_*\mathcal{F}$ over $U$.
The resulting map
$$
\bigoplus\nolimits_{i = 1, \ldots, n} \mathcal{O}_U
\longrightarrow
i_*\mathcal{F}|_U
$$
is surjective by inspection of what it does on stalks
(here we use that $\mathcal{O}_X \to i_*\mathcal{O}_Z$
is surjective). Hence $i_*\mathcal{F}$
is of finite type.

\medskip\noindent
Conversely, suppose that $i_*\mathcal{F}$ is of finite type.
Choose $z \in Z$. Set $x = i(z)$. By assumption there exists
an open neighbourhood $U \subset X$ of $x$, and sections
$s_1, \ldots, s_n \in (i_*\mathcal{F})(U)$ which generate $i_*\mathcal{F}$
over $U$. Set $V = Z \cap U$. By definition of $i_*$ the sections
$s_i$ correspond to sections $s_i$ of $\mathcal{F}$ over $V$.
The resulting map
$$
\bigoplus\nolimits_{i = 1, \ldots, n} \mathcal{O}_V
\longrightarrow
\mathcal{F}|_V
$$
is surjective by inspection of what it does on stalks. Hence
$\mathcal{F}$ is of finite type.
\end{proof}

\begin{lemma}
\label{lemma-i-star-equivalence}
Let $i : (Z, \mathcal{O}_Z) \to (X, \mathcal{O}_X)$ be a morphism
of ringed spaces. Assume $i$ is a homeomorphism onto a closed subset of $X$
and $i^\sharp : \mathcal{O}_X \to i_*\mathcal{O}_Z$ is surjective.
Denote $\mathcal{I} \subset \mathcal{O}_X$ the kernel of $i^\sharp$.
The functor
$$
i_* :
\textit{Mod}(\mathcal{O}_Z)
\longrightarrow
\textit{Mod}(\mathcal{O}_X)
$$
is exact, fully faithful, with essential image those
$\mathcal{O}_X$-modules $\mathcal{G}$ such that $\mathcal{I}\mathcal{G} = 0$.
\end{lemma}

\begin{proof}
We claim that for a $\mathcal{O}_Z$-module $\mathcal{F}$ the canonical map
$$
i^*i_*\mathcal{F} \longrightarrow \mathcal{F}
$$
is an isomorphism. We check this on stalks. Say $z \in Z$ and $x = i(z)$.
We have
$$
(i^*i_*\mathcal{F})_z =
(i_*\mathcal{F})_x \otimes_{\mathcal{O}_{X, x}} \mathcal{O}_{Z, z} =
\mathcal{F}_z \otimes_{\mathcal{O}_{X, x}} \mathcal{O}_{Z, z} =
\mathcal{F}_z
$$
by Sheaves, Lemma \ref{sheaves-lemma-stalk-pullback-modules},
the fact that $\mathcal{O}_{Z, z}$ is a quotient of $\mathcal{O}_{X, x}$, and
Sheaves, Lemma \ref{sheaves-lemma-stalks-closed-pushforward}.
It follows that $i_*$ is fully faithful.

\medskip\noindent
Let $\mathcal{G}$ be a $\mathcal{O}_X$-module with
$\mathcal{I}\mathcal{G} = 0$. We will prove the canonical map
$$
\mathcal{G} \longrightarrow i_*i^*\mathcal{G}
$$
is an isomorphism. This proves that $\mathcal{G} = i_*\mathcal{F}$
with $\mathcal{F} = i^*\mathcal{G}$ which finishes the proof.
We check the displayed map induces an isomorphism on stalks.
If $x \in X$, $x \not \in i(Z)$, then $\mathcal{G}_x = 0$
because $\mathcal{I}_x = \mathcal{O}_{X, x}$ in this
case. As above $(i_*i^*\mathcal{G})_x = 0$ by
Sheaves, Lemma \ref{sheaves-lemma-stalks-closed-pushforward}.
On the other hand, if $x \in Z$, then we obtain the map
$$
\mathcal{G}_x
\longrightarrow
\mathcal{G}_x \otimes_{\mathcal{O}_{X, x}} \mathcal{O}_{Z, x}
$$
by Sheaves, Lemmas \ref{sheaves-lemma-stalk-pullback-modules} and
\ref{sheaves-lemma-stalks-closed-pushforward}. This map is an isomorphism
because $\mathcal{O}_{Z, x} = \mathcal{O}_{X, x}/\mathcal{I}_x$
and because $\mathcal{G}_x$ is annihilated by $\mathcal{I}_x$ by assumption.
\end{proof}










\section{Locally free sheaves}
\label{section-locally-free}

\noindent
Let $(X, \mathcal{O}_X)$ be a ringed space.
Our conventions allow (some of) the stalks $\mathcal{O}_{X, x}$
to be the zero ring. This means we have to be a little careful
when defining the rank of a locally free sheaf.

\begin{definition}
\label{definition-locally-free}
Let $(X, \mathcal{O}_X)$ be a ringed space.
Let $\mathcal{F}$ be a sheaf of $\mathcal{O}_X$-modules.
\begin{enumerate}
\item We say $\mathcal{F}$ is {\it locally free} if for every
point $x \in X$ there exists a set $I$ and an open
neighbourhood $x \in U \subset X$
such that $\mathcal{F}|_U$ is isomorphic to
$\bigoplus_{i \in I} \mathcal{O}_X|_U$ as an $\mathcal{O}_X|_U$-module.
\item We say $\mathcal{F}$ is {\it finite locally free} if we may
choose the index sets $I$ to be finite.
\item We say $\mathcal{F}$ is {\it finite locally free of rank $r$}
if we may choose the index sets $I$ to have cardinality $r$.
\end{enumerate}
\end{definition}

\noindent
A finite direct sum of (finite) locally free sheaves is (finite)
locally free. However, it may not be the case that an infinite direct
sum of locally free sheaves is locally free.

\begin{lemma}
\label{lemma-locally-free-quasi-coherent}
Let $(X, \mathcal{O}_X)$ be a ringed space.
Let $\mathcal{F}$ be a sheaf of $\mathcal{O}_X$-modules.
If $\mathcal{F}$ is locally free then it is quasi-coherent.
\end{lemma}

\begin{proof}
Omitted.
\end{proof}

\begin{lemma}
\label{lemma-pullback-locally-free}
Let $f : (X, \mathcal{O}_X) \to (Y, \mathcal{O}_Y)$
be a morphism of ringed spaces. If $\mathcal{G}$ is
a locally free $\mathcal{O}_Y$-module, then
$f^*\mathcal{G}$ is a locally free $\mathcal{O}_X$-module.
\end{lemma}

\begin{proof}
Omitted.
\end{proof}

\begin{lemma}
\label{lemma-rank}
Let $(X, \mathcal{O}_X)$ be a ringed space.
Suppose that the support of $\mathcal{O}_X$ is $X$,
i.e., all stalk of $\mathcal{O}_X$ are nonzero rings.
Let $\mathcal{F}$ be a locally free sheaf of $\mathcal{O}_X$-modules.
There exists a locally constant function
$$
\text{rank}_\mathcal{F} :
X \longrightarrow \{0, 1, 2, \ldots\}\cup\{\infty\}
$$
such that for any point $x \in X$ the cardinality of any
set $I$ such that $\mathcal{F}$ is isomorphic to
$\bigoplus_{i\in I} \mathcal{O}_X$ in a neighbourhood
of $x$ is $\text{rank}_\mathcal{F}(x)$.
\end{lemma}

\begin{proof}
Under the assumption of the lemma the cardinality of $I$ can be read off
from the rank of the free module $\mathcal{F}_x$ over the nonzero ring
$\mathcal{O}_{X, x}$, and it is constant in a neighbourhood of $x$.
\end{proof}

\begin{lemma}
\label{lemma-map-finite-locally-free}
Let $(X, \mathcal{O}_X)$ be a ringed space. Let $r \geq 0$.
Let $\varphi : \mathcal{F} \to \mathcal{G}$ be a map of finite
locally free $\mathcal{O}_X$-modules of rank $r$.
Then $\varphi$ is an isomorphism if and only if $\varphi$
is surjective.
\end{lemma}

\begin{proof}
Assume $\varphi$ is surjective. Pick $x \in X$.
There exists an open neighbourhood $U$ of $x$
such that both $\mathcal{F}|_U$ and $\mathcal{G}|_U$ are
isomorphic to $\mathcal{O}_U^{\oplus r}$.
Pick lifts of the free generators of $\mathcal{G}|_U$
to obtain a map $\psi : \mathcal{G}|_U \to \mathcal{F}|_U$
such that $\varphi|_U \circ \psi = \text{id}$.
Hence we conclude that the map
$\Gamma(U, \mathcal{F}) \to \Gamma(U, \mathcal{G})$
induced by $\varphi$ is surjective. Since both
$\Gamma(U, \mathcal{F})$ and $\Gamma(U, \mathcal{G})$
are isomorphic to $\Gamma(U, \mathcal{O}_U)^{\oplus r}$
as an $\Gamma(U, \mathcal{O}_U)$-module we may
apply Algebra, Lemma \ref{algebra-lemma-fun} to see that
$\Gamma(U, \mathcal{F}) \to \Gamma(U, \mathcal{G})$
is injective. This finishes the proof.
\end{proof}

\begin{lemma}
\label{lemma-direct-summand-of-locally-free-is-locally-free}
Let $(X, \mathcal{O}_X)$ be a ringed space. If all stalks $\mathcal{O}_{X, x}$
are local rings, then any direct summand of a finite locally free
$\mathcal{O}_X$-module is finite locally free.
\end{lemma}

\begin{proof}
Assume $\mathcal{F}$ is a direct summand of the finite locally free
$\mathcal{O}_X$-module $\mathcal{H}$.
Let $x \in X$ be a point. Then $\mathcal{H}_x$ is a finite free
$\mathcal{O}_{X, x}$-module.
Because $\mathcal{O}_{X, x}$ is local, we see that
$\mathcal{F}_x \cong \mathcal{O}_{X, x}^{\oplus r}$ for some $r$, see
Algebra, Lemma \ref{algebra-lemma-finite-projective}.
By Lemma \ref{lemma-finite-presentation-stalk-free}
we see that $\mathcal{F}$ is free of rank $r$ in an open neighbourhood of $x$.
(Note that $\mathcal{F}$ is of finite presentation as a summand of
$\mathcal{H}$.)
\end{proof}







\section{Tensor product}
\label{section-tensor-product}

\noindent
Let $(X, \mathcal{O}_X)$ be a ringed space.
Let $\mathcal{F}$, $\mathcal{G}$ be $\mathcal{O}_X$-modules.
We have briefly discussed the tensor product in the setting
of change of rings in Sheaves, Sections
\ref{sheaves-section-presheaves-modules} and
\ref{sheaves-section-sheafification-presheaves-modules}.
In exactly the same way we define first the {\it tensor
product presheaf}
$$
\mathcal{F} \otimes_{p, \mathcal{O}_X} \mathcal{G}
$$
as the rule which assigns to $U \subset X$ open the
$\mathcal{O}_X(U)$-module
$\mathcal{F}(U) \otimes_{\mathcal{O}_X(U)} \mathcal{G}(U)$.
Having defined this we define the {\it tensor product sheaf}
as the sheafification of the above:
$$
\mathcal{F} \otimes_{\mathcal{O}_X} \mathcal{G}
=
(\mathcal{F} \otimes_{p, \mathcal{O}_X} \mathcal{G})^\#
$$
This can be characterized as the sheaf of
$\mathcal{O}_X$-modules such that for any third
sheaf of $\mathcal{O}_X$-modules $\mathcal{H}$
we have
$$
\Hom_{\mathcal{O}_X}
(\mathcal{F} \otimes_{\mathcal{O}_X} \mathcal{G}, \mathcal{H})
=
\text{Bilin}_{\mathcal{O}_X}(\mathcal{F} \times \mathcal{G}, \mathcal{H}).
$$
Here the right hand side indicates the set of bilinear maps of sheaves
of $\mathcal{O}_X$-modules (definition omitted).

\medskip\noindent
The tensor product of modules $M, N$ over a ring $R$ satisfies symmetry,
namely $M \otimes_R N = N \otimes_R M$, hence the same holds for
tensor products of sheaves of modules, i.e., we have
$$
\mathcal{F} \otimes_{\mathcal{O}_X} \mathcal{G}
=
\mathcal{G} \otimes_{\mathcal{O}_X} \mathcal{F}
$$
functorial in $\mathcal{F}$, $\mathcal{G}$.
And since tensor product of modules satisfies associativity we
also get canonical functorial isomorphisms
$$
(\mathcal{F} \otimes_{\mathcal{O}_X} \mathcal{G})
\otimes_{\mathcal{O}_X} \mathcal{H}
=
\mathcal{F} \otimes_{\mathcal{O}_X}
(\mathcal{G} \otimes_{\mathcal{O}_X} \mathcal{H})
$$
functorial in $\mathcal{F}$, $\mathcal{G}$, and $\mathcal{H}$.

\begin{lemma}
\label{lemma-stalk-tensor-product}
Let $(X, \mathcal{O}_X)$ be a ringed space.
Let $\mathcal{F}$, $\mathcal{G}$ be $\mathcal{O}_X$-modules.
Let $x \in X$. There is a canonical isomorphism
of $\mathcal{O}_{X, x}$-modules
$$
(\mathcal{F} \otimes_{\mathcal{O}_X} \mathcal{G})_x
=
\mathcal{F}_x \otimes_{\mathcal{O}_{X, x}} \mathcal{G}_x
$$
functorial in $\mathcal{F}$ and $\mathcal{G}$.
\end{lemma}

\begin{proof}
Omitted.
\end{proof}

\begin{lemma}
\label{lemma-tensor-product-sheafification}
Let $(X, \mathcal{O}_X)$ be a ringed space.
Let $\mathcal{F}'$, $\mathcal{G}'$ be presheaves of $\mathcal{O}_X$-modules
with sheafifications $\mathcal{F}$, $\mathcal{G}$. Then
$\mathcal{F} \otimes_{\mathcal{O}_X} \mathcal{G} =
(\mathcal{F}' \otimes_{p, \mathcal{O}_X} \mathcal{G}')^\#$.
\end{lemma}

\begin{proof}
Omitted.
\end{proof}


\begin{lemma}
\label{lemma-tensor-product-exact}
Let $(X, \mathcal{O}_X)$ be a ringed space.
Let $\mathcal{G}$ be an $\mathcal{O}_X$-module.
If
$\mathcal{F}_1
\to \mathcal{F}_2
\to \mathcal{F}_3
\to 0$
is an exact sequence of $\mathcal{O}_X$-modules then
the induced sequence
$$
\mathcal{F}_1 \otimes_{\mathcal{O}_X} \mathcal{G} \to
\mathcal{F}_2 \otimes_{\mathcal{O}_X} \mathcal{G} \to
\mathcal{F}_3 \otimes_{\mathcal{O}_X} \mathcal{G} \to
0
$$
is exact.
\end{lemma}

\begin{proof}
This follows from the fact that exactness may be checked at stalks
(Lemma \ref{lemma-abelian}), the description of stalks
(Lemma \ref{lemma-stalk-tensor-product}) and the corresponding
result for tensor products of modules
(Algebra, Lemma \ref{algebra-lemma-tensor-product-exact}).
\end{proof}

\begin{lemma}
\label{lemma-tensor-product-pullback}
Let $f : (X, \mathcal{O}_X) \to (Y, \mathcal{O}_Y)$ be
a morphism of ringed spaces. Let $\mathcal{F}$, $\mathcal{G}$
be $\mathcal{O}_Y$-modules. Then
$f^*(\mathcal{F} \otimes_{\mathcal{O}_Y} \mathcal{G})
= f^*\mathcal{F} \otimes_{\mathcal{O}_X} f^*\mathcal{G}$
functorially in $\mathcal{F}$, $\mathcal{G}$.
\end{lemma}

\begin{proof}
Omitted.
\end{proof}

\begin{lemma}
\label{lemma-tensor-product-permanence}
Let $(X, \mathcal{O}_X)$ be a ringed space.
Let $\mathcal{F}$, $\mathcal{G}$ be $\mathcal{O}_X$-modules.
\begin{enumerate}
\item If $\mathcal{F}$, $\mathcal{G}$ are locally generated
by sections, so is $\mathcal{F} \otimes_{\mathcal{O}_X} \mathcal{G}$.
\item If $\mathcal{F}$, $\mathcal{G}$ are of finite type,
so is $\mathcal{F} \otimes_{\mathcal{O}_X} \mathcal{G}$.
\item If $\mathcal{F}$, $\mathcal{G}$ are quasi-coherent,
so is $\mathcal{F} \otimes_{\mathcal{O}_X} \mathcal{G}$.
\item If $\mathcal{F}$, $\mathcal{G}$ are of finite presentation,
so is $\mathcal{F} \otimes_{\mathcal{O}_X} \mathcal{G}$.
\item If $\mathcal{F}$ is of finite presentation and $\mathcal{G}$ is coherent,
then $\mathcal{F} \otimes_{\mathcal{O}_X} \mathcal{G}$ is coherent.
\item If $\mathcal{F}$, $\mathcal{G}$ are coherent,
so is $\mathcal{F} \otimes_{\mathcal{O}_X} \mathcal{G}$.
\item If $\mathcal{F}$, $\mathcal{G}$ are locally free,
so is $\mathcal{F} \otimes_{\mathcal{O}_X} \mathcal{G}$.
\end{enumerate}
\end{lemma}

\begin{proof}
We first prove that the tensor product of locally free
$\mathcal{O}_X$-modules is locally free. This follows if we show
that
$(\bigoplus_{i \in I} \mathcal{O}_X) \otimes_{\mathcal{O}_X}
(\bigoplus_{j \in J} \mathcal{O}_X) \cong
\bigoplus_{(i, j) \in I \times J} \mathcal{O}_X$.
The sheaf $\bigoplus_{i \in I} \mathcal{O}_X$ is the sheaf associated
to the presheaf $U \mapsto \bigoplus_{i \in I} \mathcal{O}_X(U)$.
Hence the tensor product is the sheaf associated
to the presheaf
$$
U \longmapsto
(\bigoplus\nolimits_{i \in I} \mathcal{O}_X(U))
\otimes_{\mathcal{O}_X(U)}
(\bigoplus\nolimits_{j \in J} \mathcal{O}_X(U)).
$$
We deduce what we want since for any ring $R$ we have
$(\bigoplus_{i \in I} R) \otimes_R (\bigoplus_{j \in J} R) =
\bigoplus_{(i, j) \in I \times J} R$.

\medskip\noindent
If $\mathcal{F}_2 \to \mathcal{F}_1 \to \mathcal{F} \to 0$
is exact, then by Lemma \ref{lemma-tensor-product-exact}
the complex
$\mathcal{F}_2 \otimes \mathcal{G} \to
\mathcal{F}_1 \otimes \mathcal{G} \to
\mathcal{F} \otimes \mathcal{G} \to 0$
is exact. Using this we can prove (5). Namely, in this case there
exists locally such an exact sequence with $\mathcal{F}_i$, $i = 1, 2$
finite free. Hence the two terms $\mathcal{F}_2 \otimes \mathcal{G}$
are isomorphic to finite direct sums of $\mathcal{G}$.
Since finite direct sums are coherent sheaves, these are coherent
and so is the cokernel of the map, see Lemma \ref{lemma-coherent-abelian}.

\medskip\noindent
And if also
$\mathcal{G}_2 \to \mathcal{G}_1 \to \mathcal{G} \to 0$
is exact, then we see that
$$
\mathcal{F}_2 \otimes_{\mathcal{O}_X} \mathcal{G}_1
\oplus
\mathcal{F}_1 \otimes_{\mathcal{O}_X} \mathcal{G}_2
\to
\mathcal{F}_1 \otimes_{\mathcal{O}_X} \mathcal{G}_1
\to
\mathcal{F} \otimes_{\mathcal{O}_X} \mathcal{G}
\to 0
$$
is exact. Using this we can for example prove (3).
Namely, the assumption means that we can locally find presentations
as above with $\mathcal{F}_i$ and $\mathcal{G}_i$
free $\mathcal{O}_X$-modules. Hence the displayed presentation
is a presentation of the tensor product by free sheaves as well.

\medskip\noindent
The proof of the other statements is omitted.
\end{proof}

\begin{lemma}
\label{lemma-tensor-commute-colimits}
Let $(X, \mathcal{O}_X)$ be a ringed space.
For any $\mathcal{O}_X$-module $\mathcal{F}$ the functor
$$
\textit{Mod}(\mathcal{O}_X) \longrightarrow \textit{Mod}(\mathcal{O}_X)
, \quad
\mathcal{G} \longmapsto \mathcal{F} \otimes_\mathcal{O} \mathcal{G}
$$
commutes with arbitrary colimits.
\end{lemma}

\begin{proof}
Let $I$ be a preordered set and let $\{\mathcal{G}_i\}$ be
a system over $I$. Set $\mathcal{G} = \colim_i \mathcal{G}_i$.
Recall that $\mathcal{G}$ is the sheaf associated to the presheaf
$\mathcal{G}' : U \mapsto \colim_i \mathcal{G}_i(U)$, see
Sheaves, Section \ref{sheaves-section-limits-sheaves}.
By
Lemma \ref{lemma-tensor-product-sheafification}
the tensor product $\mathcal{F} \otimes_{\mathcal{O}_X} \mathcal{G}$
is the sheafification of the presheaf
$$
U \longmapsto
\mathcal{F}(U) \otimes_{\mathcal{O}_X(U)} \colim_i \mathcal{G}_i(U) =
\colim_i \mathcal{F}(U) \otimes_{\mathcal{O}_X(U)} \mathcal{G}_i(U)
$$
where the equality sign is
Algebra, Lemma \ref{algebra-lemma-tensor-products-commute-with-limits}.
Hence the lemma follows from the description of colimits in
$\textit{Mod}(\mathcal{O}_X)$.
\end{proof}





\section{Flat modules}
\label{section-flat}

\noindent
We can define flat modules exactly as in the case of modules over rings.

\begin{definition}
\label{definition-flat}
Let $(X, \mathcal{O}_X)$ be a ringed space.
An $\mathcal{O}_X$-module $\mathcal{F}$ is {\it flat} if the functor
$$
\textit{Mod}(\mathcal{O}_X)
\longrightarrow
\textit{Mod}(\mathcal{O}_X), \quad
\mathcal{G} \mapsto \mathcal{G} \otimes_\mathcal{O} \mathcal{F}
$$
is exact.
\end{definition}

\noindent
We can characterize flatness by looking at the stalks.

\begin{lemma}
\label{lemma-flat-stalks-flat}
Let $(X, \mathcal{O}_X)$ be a ringed space.
An $\mathcal{O}_X$-module $\mathcal{F}$ is flat
if and only if the stalk $\mathcal{F}_x$ is a flat
$\mathcal{O}_{X, x}$-module for all $x \in X$.
\end{lemma}

\begin{proof}
Assume $\mathcal{F}_x$ is a flat $\mathcal{O}_{X, x}$-module for all
$x \in X$. In this case, if $\mathcal{G} \to \mathcal{H} \to \mathcal{K}$
is exact, then also
$\mathcal{G} \otimes_{\mathcal{O}_X} \mathcal{F} \to
\mathcal{H} \otimes_{\mathcal{O}_X} \mathcal{F} \to
\mathcal{K} \otimes_{\mathcal{O}_X} \mathcal{F}$
is exact because we can check exactness at stalks and because
tensor product commutes with taking stalks, see
Lemma \ref{lemma-stalk-tensor-product}.
Conversely, suppose that $\mathcal{F}$ is flat, and let $x \in X$.
Consider the skyscraper sheaves $i_{x, *} M$ where $M$ is a
$\mathcal{O}_{X, x}$-module. Note that
$$
M \otimes_{\mathcal{O}_{X, x}} \mathcal{F}_x =
\left(i_{x, *} M \otimes_{\mathcal{O}_X} \mathcal{F}\right)_x
$$
again by
Lemma \ref{lemma-stalk-tensor-product}.
Since $i_{x, *}$ is exact, we see that the fact that $\mathcal{F}$
is flat implies that $M \mapsto M \otimes_{\mathcal{O}_{X, x}} \mathcal{F}_x$
is exact. Hence $\mathcal{F}_x$ is a flat $\mathcal{O}_{X, x}$-module.
\end{proof}

\noindent
Thus the following definition makes sense.

\begin{definition}
\label{definition-flat-at-point}
Let $(X, \mathcal{O}_X)$ be a ringed space. Let $x \in X$.
An $\mathcal{O}_X$-module $\mathcal{F}$ is
{\it flat at $x$} if $\mathcal{F}_x$ is a flat
$\mathcal{O}_{X, x}$-module.
\end{definition}

\noindent
Hence we see that $\mathcal{F}$ is a flat $\mathcal{O}_X$-module
if and only if it is flat at every point.

\begin{lemma}
\label{lemma-colimits-flat}
Let $(X, \mathcal{O}_X)$ be a ringed space.
A filtered colimit of flat $\mathcal{O}_X$-modules is flat.
A direct sum of flat $\mathcal{O}_X$-modules is flat.
\end{lemma}

\begin{proof}
This follows from
Lemma \ref{lemma-tensor-commute-colimits},
Lemma \ref{lemma-stalk-tensor-product},
Algebra, Lemma \ref{algebra-lemma-directed-colimit-exact},
and the fact that we can check exactness at stalks.
\end{proof}

\begin{lemma}
\label{lemma-j-shriek-flat}
Let $(X, \mathcal{O}_X)$ be a ringed space.
Let $U \subset X$ be open. The sheaf $j_{U!}\mathcal{O}_U$
is a flat sheaf of $\mathcal{O}_X$-modules.
\end{lemma}

\begin{proof}
The stalks of $j_{U!}\mathcal{O}_U$ are either zero or equal
to $\mathcal{O}_{X, x}$. Apply
Lemma \ref{lemma-flat-stalks-flat}.
\end{proof}

\begin{lemma}
\label{lemma-module-quotient-flat}
Let $(X, \mathcal{O}_X)$ be a ringed space.
\begin{enumerate}
\item Any sheaf of $\mathcal{O}_X$-modules is a quotient of
a direct sum $\bigoplus j_{U_i!}\mathcal{O}_{U_i}$.
\item Any $\mathcal{O}_X$-module is a quotient of
a flat $\mathcal{O}_X$-module.
\end{enumerate}
\end{lemma}

\begin{proof}
Let $\mathcal{F}$ be an $\mathcal{O}_X$-module.
For every open $U \subset X$ and every
$s \in \mathcal{F}(U)$ we get a morphism
$j_{U!}\mathcal{O}_U \to \mathcal{F}$, namely the adjoint to
the morphism $\mathcal{O}_U \to \mathcal{F}|_U$, $1 \mapsto s$.
Clearly the map
$$
\bigoplus\nolimits_{(U, s)} j_{U!}\mathcal{O}_U
\longrightarrow
\mathcal{F}
$$
is surjective, and the source is flat by combining Lemmas
\ref{lemma-colimits-flat} and \ref{lemma-j-shriek-flat}.
\end{proof}

\begin{lemma}
\label{lemma-flat-tor-zero}
Let $(X, \mathcal{O}_X)$ be a ringed space.
Let
$$
0 \to \mathcal{F}'' \to \mathcal{F}' \to \mathcal{F} \to 0
$$
be a short exact sequence of $\mathcal{O}_X$-modules.
Assume $\mathcal{F}$ is flat. Then for any $\mathcal{O}_X$-module
$\mathcal{G}$ the sequence
$$
0 \to
\mathcal{F}'' \otimes_\mathcal{O} \mathcal{G} \to
\mathcal{F}' \otimes_\mathcal{O} \mathcal{G} \to
\mathcal{F} \otimes_\mathcal{O} \mathcal{G} \to 0
$$
is exact.
\end{lemma}

\begin{proof}
Using that $\mathcal{F}_x$ is a flat $\mathcal{O}_{X, x}$-module
for every $x \in X$ and that exactness can be checked on stalks, this
follows from
Algebra, Lemma \ref{algebra-lemma-flat-tor-zero}.
\end{proof}

\begin{lemma}
\label{lemma-flat-ses}
\begin{slogan}
Kernels of epimorphisms and extensions of flat sheaves of modules over
a ringed space are again flat.
\end{slogan}
Let $(X, \mathcal{O}_X)$ be a ringed space.
Let
$$
0 \to
\mathcal{F}_2 \to
\mathcal{F}_1 \to
\mathcal{F}_0 \to 0
$$
be a short exact sequence of $\mathcal{O}_X$-modules.
\begin{enumerate}
\item If $\mathcal{F}_2$ and $\mathcal{F}_0$ are flat so is
$\mathcal{F}_1$.
\item If $\mathcal{F}_1$ and $\mathcal{F}_0$ are flat so is
$\mathcal{F}_2$.
\end{enumerate}
\end{lemma}

\begin{proof}
Since exactness and flatness may be checked at the level of stalks
this follows from
Algebra, Lemma \ref{algebra-lemma-flat-ses}.
\end{proof}

\begin{lemma}
\label{lemma-flat-resolution-of-flat}
Let $(X, \mathcal{O}_X)$ be a ringed space.
Let
$$
\ldots \to
\mathcal{F}_2 \to
\mathcal{F}_1 \to
\mathcal{F}_0 \to
\mathcal{Q} \to 0
$$
be an exact complex of $\mathcal{O}_X$-modules.
If $\mathcal{Q}$ and all $\mathcal{F}_i$ are flat $\mathcal{O}_X$-modules,
then for any $\mathcal{O}_X$-module $\mathcal{G}$ the complex
$$
\ldots \to
\mathcal{F}_2 \otimes_{\mathcal{O}_X} \mathcal{G} \to
\mathcal{F}_1 \otimes_{\mathcal{O}_X} \mathcal{G} \to
\mathcal{F}_0 \otimes_{\mathcal{O}_X} \mathcal{G} \to
\mathcal{Q} \otimes_{\mathcal{O}_X} \mathcal{G} \to 0
$$
is exact also.
\end{lemma}

\begin{proof}
Follows from Lemma \ref{lemma-flat-tor-zero} by splitting the complex
into short exact sequences and using Lemma \ref{lemma-flat-ses} to
prove inductively that $\Im(\mathcal{F}_{i + 1} \to \mathcal{F}_i)$
is flat.
\end{proof}

\noindent
The following lemma gives one direction of the equational criterion of
flatness (Algebra, Lemma \ref{algebra-lemma-flat-eq}).

\begin{lemma}
\label{lemma-flat-eq}
Let $(X, \mathcal{O}_X)$ be a ringed space. Let $\mathcal{F}$ be a flat
$\mathcal{O}_X$-module. Let $U \subset X$ be open and let
$$
\mathcal{O}_U \xrightarrow{(f_1, \ldots, f_n)}
\mathcal{O}_U^{\oplus n} \xrightarrow{(s_1, \ldots, s_n)}
\mathcal{F}|_U
$$
be a complex of $\mathcal{O}_U$-modules. For every $x \in U$ there
exists an open neighbourhood $V \subset U$ of $x$ and a factorization
$$
\mathcal{O}_V^{\oplus n}
\xrightarrow{A}
\mathcal{O}_V^{\oplus m} \xrightarrow{(t_1, \ldots, t_m)}
\mathcal{F}|_V
$$
of $(s_1, \ldots, s_n)|_V$ such that $A \circ (f_1, \ldots, f_n)|_V = 0$.
\end{lemma}

\begin{proof}
Let $\mathcal{I} \subset \mathcal{O}_U$ be the sheaf of ideals
generated by $f_1, \ldots, f_n$. Then $\sum f_i \otimes s_i$ is
a section of $\mathcal{I} \otimes_{\mathcal{O}_U} \mathcal{F}|_U$
which maps to zero in $\mathcal{F}|_U$. As $\mathcal{F}|_U$ is flat
the map
$\mathcal{I} \otimes_{\mathcal{O}_U} \mathcal{F}|_U \to \mathcal{F}|_U$
is injective. Since $\mathcal{I} \otimes_{\mathcal{O}_U} \mathcal{F}|_U$
is the sheaf associated to the presheaf tensor product, we see
there exists an open neighbourhood $V \subset U$ of $x$ such
that $\sum f_i|_V \otimes s_i|_V$ is zero in
$\mathcal{I}(V) \otimes_{\mathcal{O}(V)} \mathcal{F}(V)$.
Unwinding the definitions using Algebra, Lemma \ref{algebra-lemma-relations}
we find $t_1, \ldots, t_m \in \mathcal{F}(V)$ and $a_{ij} \in \mathcal{O}(V)$
such that $\sum a_{ij}f_i|_V = 0$ and $s_i|_V = \sum a_{ij}t_j$.
\end{proof}

\begin{lemma}
\label{lemma-flat-locally-finite-presentation}
Let $(X, \mathcal{O}_X)$ be a ringed space. Let $\mathcal{F}$
be locally of finite presentation and flat. Then $\mathcal{F}$ is
locally a direct summand of a finite free $\mathcal{O}_X$-module.
\end{lemma}

\begin{proof}
After replacing $X$ by the members of an open covering, we may
assume there exists a presentation
$$
\mathcal{O}_X^{\oplus r} \to
\mathcal{O}_X^{\oplus n} \to \mathcal{F} \to 0
$$
Let $x \in X$. By Lemma \ref{lemma-flat-eq}
we can, after shrinking $X$ to an open
neighbourhood of $x$, assume there exists a factorization
$$
\mathcal{O}_X^{\oplus n} \to
\mathcal{O}_X^{\oplus n_1} \to \mathcal{F}
$$
such that the composition
$\mathcal{O}_X^{\oplus r} \to \mathcal{O}_X^{\oplus n} \to
\mathcal{O}_X^{\oplus n_1}$
annihilates the first summand of $\mathcal{O}_X^{\oplus r}$.
Repeating this argument $r - 1$ more times we obtain a factorization
$$
\mathcal{O}_X^{\oplus n} \to
\mathcal{O}_X^{\oplus n_r} \to \mathcal{F}
$$
such that the composition
$\mathcal{O}_X^{\oplus r} \to \mathcal{O}_X^{\oplus n}
\to \mathcal{O}_X^{\oplus n_r}$ is zero.
This means that the surjection $\mathcal{O}_X^{\oplus n_r} \to \mathcal{F}$
has a section and we win.
\end{proof}







\section{Constructible sheaves of sets}
\label{section-constructible}

\noindent
Let $X$ be a topological space. Given a set $S$ recall that $\underline{S}$
or $\underline{S}_X$ denotes the constant sheaf with value $S$, see
Sheaves, Definition \ref{sheaves-definition-constant-sheaf}.
Let $U \subset X$ be an open of a topological space $X$.
We will denote $j_U$ the inclusion morphism and we will denote
$j_{U!} : \Sh(U) \to \Sh(X)$ the extension by the empty set
described in Sheaves, Section \ref{sheaves-section-open-immersions}.

\begin{lemma}
\label{lemma-surjection}
Let $X$ be a topological space. Let $\mathcal{B}$ be a basis for the
topology on $X$. Let $\mathcal{F}$ be a sheaf of sets on $X$.
There exists a set $I$ and for each $i \in I$ an element
$U_i \in \mathcal{B}$ and a finite set $S_i$ such that there exists
a surjection $\coprod_{i \in I} j_{U_i!}\underline{S_i} \to \mathcal{F}$.
\end{lemma}

\begin{proof}
Let $S$ be a singleton set. We will prove the result with $S_i = S$.
For every $x \in X$ and element $s \in \mathcal{F}_x$ we can choose
a $U(x, s) \in \mathcal{B}$ and $s(x, s) \in \mathcal{F}(U(x, s))$
which maps to $s$ in $\mathcal{F}_x$. By
Sheaves, Lemma \ref{sheaves-lemma-j-shriek}
the section $s(x, s)$
corresponds to a map of sheaves $j_{U(x, s)!}\underline{S} \to \mathcal{F}$.
Then
$$
\coprod\nolimits_{(x, s)} j_{U(x, s)!}\underline{S} \to \mathcal{F}
$$
is surjective on stalks and hence surjective.
\end{proof}

\begin{lemma}
\label{lemma-filtered-colimit-constructibles}
Let $X$ be a topological space. Let $\mathcal{B}$ be a basis for the
topology of $X$ and assume that each $U \in \mathcal{B}$ is quasi-compact.
Then every sheaf of sets on $X$ is a filtered colimit of sheaves of the form
\begin{equation}
\label{equation-towards-constructible-sets}
\text{Coequalizer}\left(
\xymatrix{
\coprod\nolimits_{b = 1, \ldots, m} j_{V_b!}\underline{S_b}
\ar@<1ex>[r] \ar@<-1ex>[r] &
\coprod\nolimits_{a = 1, \ldots, n} j_{U_a!}\underline{S_a}
}
\right)
\end{equation}
with $U_a$ and $V_b$ in $\mathcal{B}$ and $S_a$ and $S_b$ finite sets.
\end{lemma}

\begin{proof}
By Lemma \ref{lemma-surjection} every sheaf of sets $\mathcal{F}$
is the target of a surjection whose source is a coprod $\mathcal{F}_0$
of sheaves the form $j_{U!}\underline{S}$ with $U \in \mathcal{B}$
and $S$ finite. Applying this to
$\mathcal{F}_0 \times_\mathcal{F} \mathcal{F}_0$
we find that $\mathcal{F}$ is a coequalizer of a pair of maps
$$
\xymatrix{
\coprod\nolimits_{b \in B} j_{V_b!}\underline{S_b}
\ar@<1ex>[r] \ar@<-1ex>[r] &
\coprod\nolimits_{a \in A} j_{U_a!}\underline{S_a}
}
$$
for some index sets $A$, $B$ and $V_b$ and $U_a$ in $\mathcal{B}$ and
$S_a$ and $S_b$ finite. For every finite subset $B' \subset B$
there is a finite subset $A' \subset A$ such that the coproduct
over $b \in B'$ maps into the coprod over $a \in A'$ via both maps.
Namely, we can view the right hand side as a filtered colimit with
injective transition maps. Hence taking sections over the quasi-compact
opens $U_b$, $b \in B'$ commutes with this coproduct,
see Sheaves, Lemma \ref{sheaves-lemma-directed-colimits-sections}.
Thus our sheaf is the colimit of the cokernels of these maps
between finite coproducts.
\end{proof}

\begin{lemma}
\label{lemma-constructible-comes-from-finite}
Let $X$ be a spectral topological space. Let $\mathcal{B}$ be
the set of quasi-compact open subsets of $X$.
Let $\mathcal{F}$ be a sheaf of sets as in
Equation (\ref{equation-towards-constructible-sets}).
Then there exists a continuous spectral map $f : X \to Y$
to a finite sober topological space $Y$ and a sheaf
of sets $\mathcal{G}$ on $Y$ with finite stalks
such that $f^{-1}\mathcal{G} \cong \mathcal{F}$.
\end{lemma}

\begin{proof}
We can write $X = \lim X_i$ as a directed limit
of finite sober spaces, see Topology, Lemma
\ref{topology-lemma-spectral-inverse-limit-finite-sober-spaces}.
Of course the transition maps $X_{i'} \to X_i$ are spectral and hence
by Topology, Lemma \ref{topology-lemma-directed-inverse-limit-spectral-spaces}
the maps $p_i : X \to X_i$ are spectral.
For some $i$ we can find opens $U_{a, i}$ and $V_{b, i}$
of $X_i$ whose inverse images are $U_a$ and $V_b$, see
Topology, Lemma \ref{topology-lemma-descend-opens}.
The two maps map
$$
\beta, \gamma :
\coprod\nolimits_{b \in B} j_{V_b!}\underline{S_b}
\longrightarrow
\coprod\nolimits_{a \in A} j_{U_a!}\underline{S_a}
$$
whose coequalizer is $\mathcal{F}$ correspond by adjunction to two families
$$
\beta_b, \gamma_b :
S_b
\longrightarrow
\Gamma(V_b, \coprod\nolimits_{a \in A} j_{U_a!}\underline{S_a}), \quad
b \in B
$$
of maps of sets. Observe that
$p_i^{-1}(j_{U_{a, i}!}\underline{S_a}) = j_{U_a!}\underline{S_a}$
and $(X_{i'} \to X_i)^{-1}(j_{U_{a, i}!}\underline{S_a}) =
j_{U_{a, i'}!}\underline{S_a}$. It follows from
Sheaves, Lemma \ref{sheaves-lemma-compute-pullback-to-limit}
(and using that $S_b$ and $B$ are finite sets) that
after increasing $i$ we find maps
$$
\beta_{b, i}, \gamma_{b, i} :
S_b
\longrightarrow
\Gamma(V_{b, i}, \coprod\nolimits_{a \in A} j_{U_{a, i}!}\underline{S_a})
, \quad b \in B
$$
which give rise to the maps $\beta_b$ and $\gamma_b$ after pulling
back by $p_i$. These maps correspond in turn to maps of sheaves
$$
\beta_i, \gamma_i :
\coprod\nolimits_{b \in B} j_{V_{b, i}!}\underline{S_b}
\longrightarrow
\coprod\nolimits_{a \in A} j_{U_{a, i}!}\underline{S_a}
$$
on $X_i$. Then we can take $Y = X_i$ and
$$
\mathcal{G} =
\text{Coequalizer}\left(
\xymatrix{
\coprod\nolimits_{b = 1, \ldots, m} j_{V_{b, i}!}\underline{S_b}
\ar@<1ex>[r] \ar@<-1ex>[r] &
\coprod\nolimits_{a = 1, \ldots, n} j_{U_{a, i}!}\underline{S_a}
}
\right)
$$
We omit some details.
\end{proof}

\begin{lemma}
\label{lemma-constructible-in-constant}
Let $X$ be a spectral topological space. Let $\mathcal{B}$ be
the set of quasi-compact open subsets of $X$.
Let $\mathcal{F}$ be a sheaf of sets as in
Equation (\ref{equation-towards-constructible-sets}).
Then there exist finitely many constructible closed subsets
$Z_1, \ldots, Z_n \subset X$ and finite sets $S_i$
such that $\mathcal{F}$ is isomorphic to a subsheaf of
$\prod (Z_i \to X)_*\underline{S_i}$.
\end{lemma}

\begin{proof}
By Lemma \ref{lemma-constructible-comes-from-finite}
we reduce to the case of a finite sober topological space
and a sheaf with finite stalks.
In this case $\mathcal{F} \subset \prod_{x \in X} i_{x, *}\mathcal{F}_x$
where $i_x : \{x\} \to X$ is the embedding. We omit the proof
that $i_{x, *}\mathcal{F}_x$ is a constant sheaf on $\overline{\{x\}}$.
\end{proof}










\section{Flat morphisms of ringed spaces}
\label{section-flat-morphisms}

\noindent
The pointwise definition is motivated by
Lemma \ref{lemma-flat-stalks-flat}
and
Definition \ref{definition-flat-at-point}
above.

\begin{definition}
\label{definition-flat-morphism}
Let $f : X \to Y$ be a morphism of ringed spaces.
Let $x \in X$. We say $f$ is said to be {\it flat at $x$}
if the map of rings $\mathcal{O}_{Y, f(x)} \to \mathcal{O}_{X, x}$ is flat.
We say $f$ is {\it flat} if $f$ is flat at every $x \in X$.
\end{definition}

\noindent
Consider the map of sheaves of rings
$f^\sharp : f^{-1}\mathcal{O}_Y \to \mathcal{O}_X$.
We see that the stalk at $x$ is the ring map
$f^\sharp_x : \mathcal{O}_{Y, f(x)} \to \mathcal{O}_{X, x}$.
Hence $f$ is flat at $x$ if and only if $\mathcal{O}_X$ is flat at $x$
as an $f^{-1}\mathcal{O}_Y$-module. And $f$ is flat if and only if
$\mathcal{O}_X$ is flat as an $f^{-1}\mathcal{O}_Y$-module.
A very special case of a flat morphism is an open immersion.

\begin{lemma}
\label{lemma-pullback-flat}
Let $f : X \to Y$ be a flat morphism of ringed spaces.
Then the pullback functor
$f^* : \textit{Mod}(\mathcal{O}_Y) \to \textit{Mod}(\mathcal{O}_X)$
is exact.
\end{lemma}

\begin{proof}
The functor $f^*$ is the composition of the exact functor
$f^{-1} : \textit{Mod}(\mathcal{O}_Y) \to \textit{Mod}(f^{-1}\mathcal{O}_Y)$
and the change of rings functor
$$
\textit{Mod}(f^{-1}\mathcal{O}_Y) \to \textit{Mod}(\mathcal{O}_X), \quad
\mathcal{F} \longmapsto
\mathcal{F} \otimes_{f^{-1}\mathcal{O}_Y} \mathcal{O}_X.
$$
Thus the result follows from the discussion following
Definition \ref{definition-flat-morphism}.
\end{proof}

\begin{definition}
\label{definition-flat-module}
Let $f : (X, \mathcal{O}_X) \to (Y, \mathcal{O}_Y)$ be a morphism of
ringed spaces. Let $\mathcal{F}$ be a sheaf of $\mathcal{O}_X$-modules.
\begin{enumerate}
\item We say that $\mathcal{F}$ is {\it flat over $Y$ at a point $x \in X$}
if the stalk $\mathcal{F}_x$ is a flat $\mathcal{O}_{Y, f(x)}$-module.
\item We say that $\mathcal{F}$ is {\it flat over $Y$} if
$\mathcal{F}$ is flat over $Y$ at every point $x$ of $X$.
\end{enumerate}
\end{definition}

\noindent
With this definition we see that $\mathcal{F}$ is flat over $Y$ at $x$
if and only if $\mathcal{F}$ is flat at $x$ as an
$f^{-1}\mathcal{O}_Y$-module because
$(f^{-1}\mathcal{O}_Y)_x = \mathcal{O}_{Y, f(x)}$ by
Sheaves, Lemma \ref{sheaves-lemma-stalk-pullback}.















\section{Symmetric and exterior powers}
\label{section-symmetric-exterior}

\noindent
Let $(X, \mathcal{O}_X)$ be a ringed space.
Let $\mathcal{F}$ be an $\mathcal{O}_X$-module.
We define the {\it tensor algebra} of $\mathcal{F}$ to be
the sheaf of noncommutative $\mathcal{O}_X$-algebras
$$
\text{T}(\mathcal{F})
=
\text{T}_{\mathcal{O}_X}(\mathcal{F})
= \bigoplus\nolimits_{n \geq 0} \text{T}^n(\mathcal{F}).
$$
Here $\text{T}^0(\mathcal{F}) = \mathcal{O}_X$,
$\text{T}^1(\mathcal{F}) = \mathcal{F}$
and for $n \geq 2$ we have
$$
\text{T}^n(\mathcal{F}) =
\mathcal{F} \otimes_{\mathcal{O}_X} \ldots \otimes_{\mathcal{O}_X} \mathcal{F}
\ \ (n\text{ factors})
$$
We define $\wedge(\mathcal{F})$ to be the quotient of
$\text{T}(\mathcal{F})$ by the two sided ideal generated by
local sections $s \otimes s$ of $\text{T}^2(\mathcal{F})$ where
$s$ is a local section of $\mathcal{F}$. This is
called the {\it exterior algebra} of $\mathcal{F}$.
Similarly, we define $\text{Sym}(\mathcal{F})$ to be
the quotient of $\text{T}(\mathcal{F})$ by the two
sided ideal generated by local sections of the form
$s \otimes t - t \otimes s$ of $\text{T}^2(\mathcal{F})$.

\medskip\noindent
Both $\wedge(\mathcal{F})$ and $\text{Sym}(\mathcal{F})$ are graded
$\mathcal{O}_X$-algebras, with grading inherited from $\text{T}(\mathcal{F})$.
Moreover $\text{Sym}(\mathcal{F})$
is commutative, and $\wedge(\mathcal{F})$ is graded commutative.

\begin{lemma}
\label{lemma-local-tensor-algebra}
In the situation described above.
The sheaf $\wedge^n\mathcal{F}$ is the sheafification of the
presheaf
$$
U \longmapsto \wedge^n_{\mathcal{O}_X(U)}(\mathcal{F}(U)).
$$
See Algebra, Section \ref{algebra-section-tensor-algebra}.
Similarly, the sheaf $\text{Sym}^n\mathcal{F}$ is the sheafification
of the presheaf
$$
U \longmapsto \text{Sym}^n_{\mathcal{O}_X(U)}(\mathcal{F}(U)).
$$
\end{lemma}

\begin{proof}
Omitted. It may be more efficient to define $\text{Sym}(\mathcal{F})$
and $\wedge(\mathcal{F})$ in this way instead of the method
given above.
\end{proof}

\begin{lemma}
\label{lemma-stalk-tensor-algebra}
In the situation described above. Let $x \in X$.
There are canonical isomorphisms of $\mathcal{O}_{X, x}$-modules
$\text{T}(\mathcal{F})_x = \text{T}(\mathcal{F}_x)$,
$\text{Sym}(\mathcal{F})_x = \text{Sym}(\mathcal{F}_x)$, and
$\wedge(\mathcal{F})_x = \wedge(\mathcal{F}_x)$.
\end{lemma}

\begin{proof}
Clear from Lemma \ref{lemma-local-tensor-algebra} above, and
Algebra, Lemma \ref{algebra-lemma-colimit-tensor-algebra}.
\end{proof}

\begin{lemma}
\label{lemma-pullback-tensor-algebra}
Let $f : (X, \mathcal{O}_X) \to (Y, \mathcal{O}_Y)$ be a morphism of
ringed spaces. Let $\mathcal{F}$ be a sheaf of $\mathcal{O}_Y$-modules.
Then $f^*\text{T}(\mathcal{F}) = \text{T}(f^*\mathcal{F})$,
and similarly for the exterior and symmetric algebras associated
to $\mathcal{F}$.
\end{lemma}

\begin{proof}
Omitted.
\end{proof}

\begin{lemma}
\label{lemma-presentation-sym-exterior}
Let $(X, \mathcal{O}_X)$ be a ringed space.
Let $\mathcal{F}_2 \to \mathcal{F}_1 \to \mathcal{F} \to 0$
be an exact sequence of sheaves of $\mathcal{O}_X$-modules.
For each $n \geq 1$ there is an exact sequence
$$
\mathcal{F}_2 \otimes_{\mathcal{O}_X} \text{Sym}^{n - 1}(\mathcal{F}_1)
\to
\text{Sym}^n(\mathcal{F}_1)
\to
\text{Sym}^n(\mathcal{F})
\to
0
$$
and similarly an exact sequence
$$
\mathcal{F}_2 \otimes_{\mathcal{O}_X} \wedge^{n - 1}(\mathcal{F}_1)
\to
\wedge^n(\mathcal{F}_1)
\to
\wedge^n(\mathcal{F})
\to
0
$$
\end{lemma}

\begin{proof}
See Algebra, Lemma \ref{algebra-lemma-presentation-sym-exterior}.
\end{proof}

\begin{lemma}
\label{lemma-tensor-algebra-permanence}
Let $(X, \mathcal{O}_X)$ be a ringed space.
Let $\mathcal{F}$ be a sheaf of $\mathcal{O}_X$-modules.
\begin{enumerate}
\item If $\mathcal{F}$ is locally generated by sections,
then so is each $\text{T}^n(\mathcal{F})$,
$\wedge^n(\mathcal{F})$, and $\text{Sym}^n(\mathcal{F})$.
\item If $\mathcal{F}$ is of finite type,
then so is each $\text{T}^n(\mathcal{F})$,
$\wedge^n(\mathcal{F})$, and $\text{Sym}^n(\mathcal{F})$.
\item If $\mathcal{F}$ is of finite presentation,
then so is each $\text{T}^n(\mathcal{F})$,
$\wedge^n(\mathcal{F})$, and $\text{Sym}^n(\mathcal{F})$.
\item If $\mathcal{F}$ is coherent,
then for $n > 0$ each $\text{T}^n(\mathcal{F})$,
$\wedge^n(\mathcal{F})$, and $\text{Sym}^n(\mathcal{F})$
is coherent.
\item If $\mathcal{F}$ is quasi-coherent,
then so is each $\text{T}^n(\mathcal{F})$,
$\wedge^n(\mathcal{F})$, and $\text{Sym}^n(\mathcal{F})$.
\item If $\mathcal{F}$ is locally free,
then so is each $\text{T}^n(\mathcal{F})$,
$\wedge^n(\mathcal{F})$, and $\text{Sym}^n(\mathcal{F})$.
\end{enumerate}
\end{lemma}

\begin{proof}
These statements for $\text{T}^n(\mathcal{F})$ follow
from Lemma \ref{lemma-tensor-product-permanence}.

\medskip\noindent
Statements (1) and (2) follow from the fact that
$\wedge^n(\mathcal{F})$ and $\text{Sym}^n(\mathcal{F})$
are quotients of $\text{T}^n(\mathcal{F})$.

\medskip\noindent
Statement (6) follows from
Algebra, Lemma \ref{algebra-lemma-free-tensor-algebra}.

\medskip\noindent
For (3) and (5) we will use
Lemma \ref{lemma-presentation-sym-exterior} above.
By locally choosing a presentation
$\mathcal{F}_2 \to \mathcal{F}_1 \to \mathcal{F} \to 0$
with $\mathcal{F}_i$ free, or finite free and applying the
lemma we see that $\text{Sym}^n(\mathcal{F})$, $\wedge^n(\mathcal{F})$
has a similar presentation; here we use (6) and
Lemma \ref{lemma-tensor-product-permanence}.

\medskip\noindent
To prove (4) we will use
Algebra, Lemma \ref{algebra-lemma-present-sym-wedge}.
We may localize on $X$ and assume that
$\mathcal{F}$ is generated by a finite set
$(s_i)_{i \in I}$ of global sections.
The lemma mentioned above
combined with Lemma \ref{lemma-local-tensor-algebra} above
implies that for $n \geq 2$
there exists an exact sequence
$$
\bigoplus\nolimits_{j \in J}
\text{T}^{n - 2}(\mathcal{F})
\to
\text{T}^n(\mathcal{F})
\to
\text{Sym}^n(\mathcal{F})
\to
0
$$
where the index set $J$ is finite. Now we know that
$\text{T}^{n - 2}(\mathcal{F})$ is finitely generated
and hence the image of the first arrow is a coherent
subsheaf of $\text{T}^n(\mathcal{F})$, see Lemma \ref{lemma-coherent-abelian}.
By that same lemma we conclude that $\text{Sym}^n(\mathcal{F})$ is
coherent.
\end{proof}

\begin{lemma}
\label{lemma-whole-tensor-algebra-permanence}
Let $(X, \mathcal{O}_X)$ be a ringed space.
Let $\mathcal{F}$ be a sheaf of $\mathcal{O}_X$-modules.
\begin{enumerate}
\item If $\mathcal{F}$ is quasi-coherent,
then so is each $\text{T}(\mathcal{F})$,
$\wedge(\mathcal{F})$, and $\text{Sym}(\mathcal{F})$.
\item If $\mathcal{F}$ is locally free,
then so is each $\text{T}(\mathcal{F})$,
$\wedge(\mathcal{F})$, and $\text{Sym}(\mathcal{F})$.
\end{enumerate}
\end{lemma}

\begin{proof}
It is not true that an infinite direct sum $\bigoplus \mathcal{G}_i$ of
locally free modules is locally free, or that an
infinite direct sum of quasi-coherent modules
is quasi-coherent. The problem is that given a
point $x \in X$ the open neighbourhoods $U_i$ of $x$ on which $\mathcal{G}_i$
becomes free (resp.\ has a suitable presentation) may have an intersection
which is not an open neighbourhood of $x$. However, in the
proof of Lemma \ref{lemma-tensor-algebra-permanence} we saw that
once a suitable open neighbourhood for $\mathcal{F}$ has been chosen,
then this open neighbourhood works for each of the sheaves
$\text{T}^n(\mathcal{F})$, $\wedge^n(\mathcal{F})$ and
$\text{Sym}^n(\mathcal{F})$.
The lemma follows.
\end{proof}







\section{Internal Hom}
\label{section-internal-hom}

\noindent
Let $(X, \mathcal{O}_X)$ be a ringed space.
Let $\mathcal{F}$, $\mathcal{G}$ be $\mathcal{O}_X$-modules.
Consider the rule
$$
U \longmapsto \Hom_{\mathcal{O}_X|_U}(\mathcal{F}|_U, \mathcal{G}|_U).
$$
It follows from the discussion in Sheaves, Section
\ref{sheaves-section-glueing-sheaves} that this is a sheaf of
abelian groups. In addition, given an element
$\varphi \in \Hom_{\mathcal{O}_X|_U}(\mathcal{F}|_U, \mathcal{G}|_U)$
and a section $f \in \mathcal{O}_X(U)$ then we can define
$f\varphi \in \Hom_{\mathcal{O}_X|_U}(\mathcal{F}|_U, \mathcal{G}|_U)$
by either precomposing with multiplication by $f$ on $\mathcal{F}|_U$
or postcomposing with multiplication by $f$ on $\mathcal{G}|_U$ (it gives
the same result). Hence we in fact get a sheaf of $\mathcal{O}_X$-modules.
We will denote this sheaf
$\SheafHom_{\mathcal{O}_X}(\mathcal{F}, \mathcal{G})$.
There is a canonical ``evaluation'' morphism
$$
\mathcal{F}
\otimes_{\mathcal{O}_X}
\SheafHom_{\mathcal{O}_X}(\mathcal{F}, \mathcal{G})
\longrightarrow
\mathcal{G}.
$$
For every $x \in X$ there is also a canonical morphism
$$
\SheafHom_{\mathcal{O}_X}(\mathcal{F}, \mathcal{G})_x
\to
\Hom_{\mathcal{O}_{X, x}}(\mathcal{F}_x, \mathcal{G}_x)
$$
which is rarely an isomorphism.

\begin{lemma}
\label{lemma-internal-hom}
Let $(X, \mathcal{O}_X)$ be a ringed space.
Let $\mathcal{F}$, $\mathcal{G}$, $\mathcal{H}$ be $\mathcal{O}_X$-modules.
There is a canonical isomorphism
$$
\SheafHom_{\mathcal{O}_X}
(\mathcal{F} \otimes_{\mathcal{O}_X} \mathcal{G}, \mathcal{H})
\longrightarrow
\SheafHom_{\mathcal{O}_X}
(\mathcal{F}, \SheafHom_{\mathcal{O}_X}(\mathcal{G}, \mathcal{H}))
$$
which is functorial in all three entries (sheaf Hom in
all three spots). In particular, to give a
morphism $\mathcal{F} \otimes_{\mathcal{O}_X} \mathcal{G} \to \mathcal{H}$
is the same as giving a morphism
$\mathcal{F} \to \SheafHom_{\mathcal{O}_X}(\mathcal{G}, \mathcal{H})$.
\end{lemma}

\begin{proof}
This is the analogue of
Algebra, Lemma \ref{algebra-lemma-hom-from-tensor-product}.
The proof is the same, and is omitted.
\end{proof}

\begin{lemma}
\label{lemma-internal-hom-exact}
Let $(X, \mathcal{O}_X)$ be a ringed space.
Let $\mathcal{F}$, $\mathcal{G}$ be $\mathcal{O}_X$-modules.
\begin{enumerate}
\item If $\mathcal{F}_2 \to \mathcal{F}_1 \to \mathcal{F} \to 0$
is an exact sequence of $\mathcal{O}_X$-modules, then
$$
0 \to
\SheafHom_{\mathcal{O}_X}(\mathcal{F}, \mathcal{G}) \to
\SheafHom_{\mathcal{O}_X}(\mathcal{F}_1, \mathcal{G}) \to
\SheafHom_{\mathcal{O}_X}(\mathcal{F}_2, \mathcal{G})
$$
is exact.
\item If $0 \to \mathcal{G} \to \mathcal{G}_1 \to \mathcal{G}_2$
is an exact sequence of $\mathcal{O}_X$-modules, then
$$
0 \to
\SheafHom_{\mathcal{O}_X}(\mathcal{F}, \mathcal{G}) \to
\SheafHom_{\mathcal{O}_X}(\mathcal{F}, \mathcal{G}_1) \to
\SheafHom_{\mathcal{O}_X}(\mathcal{F}, \mathcal{G}_2)
$$
is exact.
\end{enumerate}
\end{lemma}

\begin{proof}
Omitted.
\end{proof}

\begin{lemma}
\label{lemma-stalk-internal-hom}
Let $(X, \mathcal{O}_X)$ be a ringed space.
Let $\mathcal{F}$, $\mathcal{G}$ be $\mathcal{O}_X$-modules.
If $\mathcal{F}$ is finitely presented then the canonical map
$$
\SheafHom_{\mathcal{O}_X}(\mathcal{F}, \mathcal{G})_x
\to
\Hom_{\mathcal{O}_{X, x}}(\mathcal{F}_x, \mathcal{G}_x)
$$
is an isomorphism.
\end{lemma}

\begin{proof}
By localizing on $X$ we may assume that $\mathcal{F}$ has a presentation
$$
\bigoplus\nolimits_{j = 1, \ldots, m}
\mathcal{O}_X
\longrightarrow
\bigoplus\nolimits_{i = 1, \ldots, n}
\mathcal{O}_X
\to
\mathcal{F}
\to
0.
$$
By Lemma \ref{lemma-internal-hom-exact} this gives an exact sequence
$
0 \to
\SheafHom_{\mathcal{O}_X}(\mathcal{F}, \mathcal{G}) \to
\bigoplus\nolimits_{i = 1, \ldots, n} \mathcal{G}
\longrightarrow
\bigoplus\nolimits_{j = 1, \ldots, m} \mathcal{G}.
$
Taking stalks we get an exact sequence
$
0 \to
\SheafHom_{\mathcal{O}_X}(\mathcal{F}, \mathcal{G})_x \to
\bigoplus\nolimits_{i = 1, \ldots, n} \mathcal{G}_x
\longrightarrow
\bigoplus\nolimits_{j = 1, \ldots, m} \mathcal{G}_x
$
and the result follows since $\mathcal{F}_x$ sits in
an exact sequence
$
\bigoplus\nolimits_{j = 1, \ldots, m}
\mathcal{O}_{X, x}
\longrightarrow
\bigoplus\nolimits_{i = 1, \ldots, n}
\mathcal{O}_{X, x}
\to
\mathcal{F}_x
\to
0
$
which induces the exact sequence
$
0 \to
\Hom_{\mathcal{O}_{X, x}}(\mathcal{F}_x, \mathcal{G}_x) \to
\bigoplus\nolimits_{i = 1, \ldots, n} \mathcal{G}_x
\longrightarrow
\bigoplus\nolimits_{j = 1, \ldots, m} \mathcal{G}_x
$
which is the same as the one above.
\end{proof}

\begin{lemma}
\label{lemma-pullback-internal-hom}
Let $f : (X, \mathcal{O}_X) \to (Y, \mathcal{O}_Y)$ be a morphism
of ringed spaces. Let $\mathcal{F}$, $\mathcal{G}$ be $\mathcal{O}_Y$-modules.
If $\mathcal{F}$ is finitely presented and $f$ is flat,
then the canonical map
$$
f^*\SheafHom_{\mathcal{O}_Y}(\mathcal{F}, \mathcal{G})
\longrightarrow
\SheafHom_{\mathcal{O}_X}(f^*\mathcal{F}, f^*\mathcal{G})
$$
is an isomorphism.
\end{lemma}

\begin{proof}
Note that $f^*\mathcal{F}$ is also finitely presented
(Lemma \ref{lemma-pullback-finite-presentation}).
Let $x \in X$ map to $y \in Y$. Looking at the stalks
at $x$ we get an isomorphism by
Lemma \ref{lemma-stalk-internal-hom} and
More on Algebra, Remark
\ref{more-algebra-remark-pseudo-coherence-and-base-change-ext}
to see that in this case $\Hom$ commutes with base change by
$\mathcal{O}_{Y, y} \to \mathcal{O}_{X, x}$.
Second proof: use the exact same argument as given
in the proof of Lemma \ref{lemma-stalk-internal-hom}.
\end{proof}

\begin{lemma}
\label{lemma-internal-hom-locally-kernel-direct-sum}
Let $(X, \mathcal{O}_X)$ be a ringed space.
Let $\mathcal{F}$, $\mathcal{G}$ be $\mathcal{O}_X$-modules.
If $\mathcal{F}$ is finitely presented then the sheaf
$\SheafHom_{\mathcal{O}_X}(\mathcal{F}, \mathcal{G})$ is
locally a kernel of a map between finite direct sums
of copies of $\mathcal{G}$.
In particular, if $\mathcal{G}$ is coherent then
$\SheafHom_{\mathcal{O}_X}(\mathcal{F}, \mathcal{G})$
is coherent too.
\end{lemma}

\begin{proof}
The first assertion
we saw in the proof of Lemma \ref{lemma-stalk-internal-hom}.
And the result for coherent sheaves then follows from
Lemma \ref{lemma-coherent-abelian}.
\end{proof}

\begin{lemma}
\label{lemma-adjoint-tensor-restrict}
Let $X$ be a topological space. Let $\mathcal{O}_1 \to \mathcal{O}_2$
be a homomorphism of sheaves of rings. Then we have
$$
\Hom_{\mathcal{O}_1}(\mathcal{F}_{\mathcal{O}_1}, \mathcal{G}) =
\Hom_{\mathcal{O}_2}(\mathcal{F},
\SheafHom_{\mathcal{O}_1}(\mathcal{O}_2, \mathcal{G}))
$$
bifunctorially in $\mathcal{F} \in \textit{Mod}(\mathcal{O}_2)$
and $\mathcal{G} \in \textit{Mod}(\mathcal{O}_1)$.
\end{lemma}

\begin{proof}
Omitted. This is the analogue of
Algebra, Lemma \ref{algebra-lemma-adjoint-hom-restrict}
and is proved in exactly the same way.
\end{proof}





\section{Koszul complexes}
\label{section-koszul-complex}

\noindent
We suggest first reading the section on Koszul complexes in
More on Algebra, Section \ref{more-algebra-section-koszul}.
We define the Koszul complex in the category of $\mathcal{O}_X$-modules
as follows.

\begin{definition}
\label{definition-koszul}
Let $X$ be a ringed space. Let $\varphi : \mathcal{E} \to \mathcal{O}_X$
be an $\mathcal{O}_X$-module map. The
{\it Koszul complex} $K_\bullet(\varphi)$ associated to $\varphi$
is the sheaf of commutative differential graded algebras defined as follows:
\begin{enumerate}
\item the underlying graded algebra is the exterior algebra
$K_\bullet(\varphi) = \wedge(\mathcal{E})$,
\item the differential $d : K_\bullet(\varphi) \to K_\bullet(\varphi)$
is the unique derivation such that $d(e) = \varphi(e)$ for all
local sections $e$ of $\mathcal{E} = K_1(\varphi)$.
\end{enumerate}
\end{definition}

\noindent
Explicitly, if $e_1 \wedge \ldots \wedge e_n$ is a wedge product of local
sections of $\mathcal{E}$, then
$$
d(e_1 \wedge \ldots \wedge e_n) =
\sum\nolimits_{i = 1, \ldots, n} (-1)^{i + 1}
\varphi(e_i)e_1 \wedge \ldots \wedge \widehat{e_i} \wedge \ldots \wedge e_n.
$$
It is straightforward to see that this gives a well defined derivation
on the tensor algebra, which annihilates $e \wedge e$ and hence factors
through the exterior algebra.

\begin{definition}
\label{definition-koszul-complex}
Let $X$ be a ringed space and let
$f_1, \ldots, f_n \in \Gamma(X, \mathcal{O}_X)$. The
{\it Koszul complex on $f_1, \ldots, f_r$} is the Koszul complex
associated to the map
$(f_1, \ldots, f_n) : \mathcal{O}_X^{\oplus n} \to \mathcal{O}_X$.
Notation $K_\bullet(\mathcal{O}_X, f_1, \ldots, f_n)$,
or $K_\bullet(\mathcal{O}_X, f_\bullet)$.
\end{definition}

\noindent
Of course, given an $\mathcal{O}_X$-module map
$\varphi : \mathcal{E} \to \mathcal{O}_X$,
if $\mathcal{E}$ is finite locally free, then
$K_\bullet(\varphi)$ is locally on $X$ isomorphic to a Koszul complex
$K_\bullet(\mathcal{O}_X, f_1, \ldots, f_n)$.



\section{Invertible modules}
\label{section-invertible}

\noindent
Similarly to the case of modules over rings
(More on Algebra, Section \ref{more-algebra-section-picard})
we have the following definition.

\begin{definition}
\label{definition-invertible}
Let $(X, \mathcal{O}_X)$ be a ringed space. An
{\it invertible $\mathcal{O}_X$-module} is a sheaf
of $\mathcal{O}_X$-modules $\mathcal{L}$ such that
the functor
$$
\textit{Mod}(\mathcal{O}_X) \longrightarrow \textit{Mod}(\mathcal{O}_X),\quad
\mathcal{F} \longmapsto \mathcal{L} \otimes_{\mathcal{O}_X} \mathcal{F}
$$
is an equivalence of categories. We say that $\mathcal{L}$ is
{\it trivial} if it is isomorphic as an $\mathcal{O}_X$-module
to $\mathcal{O}_X$.
\end{definition}

\noindent
Lemma \ref{lemma-invertible-is-locally-free-rank-1}
below explains the relationship with locally free modules
of rank $1$.

\begin{lemma}
\label{lemma-invertible}
Let $(X, \mathcal{O}_X)$ be a ringed space. Let $\mathcal{L}$
be an $\mathcal{O}_X$-module. Equivalent are
\begin{enumerate}
\item $\mathcal{L}$ is invertible, and
\item there exists an $\mathcal{O}_X$-module $\mathcal{N}$
such that
$\mathcal{L} \otimes_{\mathcal{O}_X} \mathcal{N} \cong \mathcal{O}_X$.
\end{enumerate}
In this case $\mathcal{L}$ is locally a direct summand of a finite free
$\mathcal{O}_X$-module and the module $\mathcal{N}$ in (2) is isomorphic to
$\SheafHom_{\mathcal{O}_X}(\mathcal{L}, \mathcal{O}_X)$.
\end{lemma}

\begin{proof}
Assume (1). Then the functor $- \otimes_{\mathcal{O}_X} \mathcal{L}$
is essentially surjective, hence there exists an $\mathcal{O}_X$-module
$\mathcal{N}$ as in (2). If (2) holds, then the functor
$- \otimes_{\mathcal{O}_X} \mathcal{N}$ is a quasi-inverse
to the functor $- \otimes_{\mathcal{O}_X} \mathcal{L}$ and
we see that (1) holds.

\medskip\noindent
Assume (1) and (2) hold. Denote
$\psi : \mathcal{L} \otimes_{\mathcal{O}_X} \mathcal{N} \to \mathcal{O}_X$
the given isomorphism. Let $x \in X$. Choose an open neighbourhood
$U$ an integer $n \geq 1$ and sections $s_i \in \mathcal{L}(U)$,
$t_i \in \mathcal{N}(U)$ such that $\psi(\sum s_i \otimes t_i) = 1$.
Consider the isomorphisms
$$
\mathcal{L}|_U \to
\mathcal{L}|_U \otimes_{\mathcal{O}_U}
\mathcal{L}|_U \otimes_{\mathcal{O}_U} \mathcal{N}|_U \to \mathcal{L}|_U
$$
where the first arrow sends $s$ to $\sum s_i \otimes s \otimes t_i$
and the second arrow sends $s \otimes s' \otimes t$ to $\psi(s' \otimes t)s$.
We conclude that $s \mapsto \sum \psi(s \otimes t_i)s_i$ is
an automorphism of $\mathcal{L}|_U$. This automorphism factors as
$$
\mathcal{L}|_U \to \mathcal{O}_U^{\oplus n} \to \mathcal{L}|_U
$$
where the first arrow is given by
$s \mapsto (\psi(s \otimes t_1), \ldots, \psi(s \otimes t_n))$
and the second arrow by $(a_1, \ldots, a_n) \mapsto \sum a_i s_i$.
In this way we conclude that $\mathcal{L}|_U$ is a direct summand
of a finite free $\mathcal{O}_U$-module.

\medskip\noindent
Assume (1) and (2) hold. Consider the evaluation map
$$
\mathcal{L} \otimes_{\mathcal{O}_X}
\SheafHom_{\mathcal{O}_X}(\mathcal{L}, \mathcal{O}_X)
\longrightarrow \mathcal{O}_X
$$
To finish the proof of the lemma
we will show this is an isomorphism by checking it induces
isomorphisms on stalks. Let $x \in X$.
Since we know (by the previous paragraph)
that $\mathcal{L}$ is a finitely presented
$\mathcal{O}_X$-module
we can use Lemma \ref{lemma-stalk-internal-hom}
to see that it suffices to show that
$$
\mathcal{L}_x \otimes_{\mathcal{O}_{X, x}}
\Hom_{\mathcal{O}_{X, x}}(\mathcal{L}_x, \mathcal{O}_{X, x})
\longrightarrow \mathcal{O}_{X, x}
$$
is an isomorphism. Since
$\mathcal{L}_x \otimes_{\mathcal{O}_{X, x}} \mathcal{N}_x =
(\mathcal{L} \otimes_{\mathcal{O}_X} \mathcal{N})_x =
\mathcal{O}_{X, x}$ (Lemma \ref{lemma-stalk-tensor-product})
the desired result follows from
More on Algebra, Lemma \ref{more-algebra-lemma-invertible}.
\end{proof}

\begin{lemma}
\label{lemma-pullback-invertible}
Let $f : (X, \mathcal{O}_X) \to (Y, \mathcal{O}_Y)$ be a
morphism of ringed spaces. The pullback $f^*\mathcal{L}$ of an
invertible $\mathcal{O}_Y$-module is invertible.
\end{lemma}

\begin{proof}
By Lemma \ref{lemma-invertible}
there exists an $\mathcal{O}_Y$-module $\mathcal{N}$ such that
$\mathcal{L} \otimes_{\mathcal{O}_Y} \mathcal{N} \cong \mathcal{O}_Y$.
Pulling back we get
$f^*\mathcal{L} \otimes_{\mathcal{O}_X} f^*\mathcal{N} \cong \mathcal{O}_X$
by Lemma \ref{lemma-tensor-product-pullback}.
Thus $f^*\mathcal{L}$ is invertible by Lemma \ref{lemma-invertible}.
\end{proof}

\begin{lemma}
\label{lemma-invertible-is-locally-free-rank-1}
Let $(X, \mathcal{O}_X)$ be a ringed space. Any locally free
$\mathcal{O}_X$-module of rank $1$ is invertible.
If all stalks $\mathcal{O}_{X, x}$ are local rings, then
the converse holds as well (but in general this is not the case).
\end{lemma}

\begin{proof}
The parenthetical statement follows by considering a one point
space $X$ with sheaf of rings $\mathcal{O}_X$ given by a ring $R$.
Then invertible $\mathcal{O}_X$-modules correspond to invertible
$R$-modules, hence as soon as $\Pic(R)$ is not the trivial group,
then we get an example.

\medskip\noindent
Assume $\mathcal{L}$ is locally free of rank $1$ and consider the
evaluation map
$$
\mathcal{L} \otimes_{\mathcal{O}_X}
\SheafHom_{\mathcal{O}_X}(\mathcal{L}, \mathcal{O}_X)
\longrightarrow \mathcal{O}_X
$$
Looking over an open covering trivialization $\mathcal{L}$, we see
that this map is an isomorphism. Hence $\mathcal{L}$ is invertible
by Lemma \ref{lemma-invertible}.

\medskip\noindent
Assume all stalks $\mathcal{O}_{X, x}$ are local rings and $\mathcal{L}$
invertible. In the proof of Lemma \ref{lemma-invertible}
we have seen that $\mathcal{L}_x$ is an invertible
$\mathcal{O}_{X, x}$-module for all $x \in X$. Since
$\mathcal{O}_{X, x}$ is local, we see that
$\mathcal{L}_x \cong \mathcal{O}_{X, x}$
(More on Algebra, Section \ref{more-algebra-section-picard}).
Since $\mathcal{L}$ is of finite presentation by
Lemma \ref{lemma-invertible} we conclude that $\mathcal{L}$
is locally free of rank $1$ by
Lemma \ref{lemma-finite-presentation-stalk-free}.
\end{proof}

\begin{lemma}
\label{lemma-constructions-invertible}
Let $(X, \mathcal{O}_X)$ be a ringed space.
\begin{enumerate}
\item If $\mathcal{L}$, $\mathcal{N}$ are invertible
$\mathcal{O}_X$-modules, then so is
$\mathcal{L} \otimes_{\mathcal{O}_X} \mathcal{N}$.
\item If $\mathcal{L}$ is an invertible $\mathcal{O}_X$-module, then so is
$\SheafHom_{\mathcal{O}_X}(\mathcal{L}, \mathcal{O}_X)$ and the evaluation map
$\mathcal{L} \otimes_{\mathcal{O}_X}
\SheafHom_{\mathcal{O}_X}(\mathcal{L}, \mathcal{O}_X) \to \mathcal{O}_X$
is an isomorphism.
\end{enumerate}
\end{lemma}

\begin{proof}
Part (1) is clear from the definition and part (2) follows from
Lemma \ref{lemma-invertible} and its proof.
\end{proof}

\begin{definition}
\label{definition-powers}
Let $(X, \mathcal{O}_X)$ be a ringed space. Given an invertible sheaf
$\mathcal{L}$ on $X$ and $n \in \mathbf{Z}$ we define the
$n$th {\it tensor power} $\mathcal{L}^{\otimes n}$ of $\mathcal{L}$
as the image of $\mathcal{O}_X$ under applying the equivalence
$\mathcal{F} \mapsto\mathcal{F} \otimes_{\mathcal{O}_X} \mathcal{L}$
exactly $n$ times.
\end{definition}

\noindent
This makes sense also for negative $n$ as we've defined an invertible
$\mathcal{O}_X$-module as one for which tensoring is an equivalence.
More explicitly, we have
$$
\mathcal{L}^{\otimes n} =
\left\{
\begin{matrix}
\mathcal{O}_X & \text{if} & n = 0 \\
\SheafHom_{\mathcal{O}_X}(\mathcal{L}, \mathcal{O}_X) & \text{if} & n = -1\\
\mathcal{L} \otimes_{\mathcal{O}_X} \ldots \otimes_{\mathcal{O}_X} \mathcal{L}
& \text{if} & n > 0 \\
\mathcal{L}^{\otimes -1} \otimes_{\mathcal{O}_X} \ldots
\otimes_{\mathcal{O}_X} \mathcal{L}^{\otimes -1}
& \text{if} & n < -1
\end{matrix}
\right.
$$
see Lemma \ref{lemma-constructions-invertible}.
With this definition we have canonical isomorphisms
$\mathcal{L}^{\otimes n} \otimes_{\mathcal{O}_X}
\mathcal{L}^{\otimes m} \to
\mathcal{L}^{\otimes n + m}$, and these isomorphisms
satisfy a commutativity and an associativity constraint
(formulation omitted).

\medskip\noindent
Let $(X, \mathcal{O}_X)$ be a ringed space. We can define a $\mathbf{Z}$-graded
ring structure on $\bigoplus \Gamma(X, \mathcal{L}^{\otimes n})$ by mapping
$s \in \Gamma(X, \mathcal{L}^{\otimes n})$ and
$t \in \Gamma(X, \mathcal{L}^{\otimes m})$ to the
section corresponding to $s \otimes t$ in
$\Gamma(X, \mathcal{L}^{\otimes n + m})$.
We omit the verification that this defines a commutative
and associative ring with $1$. However, by our conventions in
Algebra, Section \ref{algebra-section-graded}
a graded ring has no nonzero elements in negative degrees.
This leads to the following definition.

\begin{definition}
\label{definition-gamma-star}
Let $(X, \mathcal{O}_X)$ be a ringed space.
Given an invertible sheaf $\mathcal{L}$ on $X$ we define
the {\it associated graded ring} to be
$$
\Gamma_*(X, \mathcal{L})
=
\bigoplus\nolimits_{n \geq 0} \Gamma(X, \mathcal{L}^{\otimes n})
$$
Given a sheaf of $\mathcal{O}_X$-modules $\mathcal{F}$ we set
$$
\Gamma_*(X, \mathcal{L}, \mathcal{F})
=
\bigoplus\nolimits_{n \in \mathbf{Z}} \Gamma(X,
\mathcal{F} \otimes_{\mathcal{O}_X} \mathcal{L}^{\otimes n})
$$
which we think of as a graded $\Gamma_*(X, \mathcal{L})$-module.
\end{definition}

\noindent
We often write simply $\Gamma_*(\mathcal{L})$ and $\Gamma_*(\mathcal{F})$
(although this is ambiguous if $\mathcal{F}$ is invertible).
The multiplication of $\Gamma_*(\mathcal{L})$ on
$\Gamma_*(\mathcal{F})$ is defined using the isomorphisms
above. If $\gamma : \mathcal{F} \to \mathcal{G}$ is a $\mathcal{O}_X$-module
map, then we get an $\Gamma_*(\mathcal{L})$-module homomorphism
$\gamma : \Gamma_*(\mathcal{F}) \to \Gamma_*(\mathcal{G})$.
If $\alpha : \mathcal{L} \to \mathcal{N}$ is an $\mathcal{O}_X$-module
map between invertible $\mathcal{O}_X$-modules, then we obtain
a graded ring homomorphism $\Gamma_*(\mathcal{L}) \to \Gamma_*(\mathcal{N})$.
If $f : (Y, \mathcal{O}_Y) \to (X, \mathcal{O}_X)$
is a morphism of ringed spaces and if $\mathcal{L}$ is invertible
on $X$, then we get an invertible sheaf $f^*\mathcal{L}$ on $Y$
(Lemma \ref{lemma-pullback-invertible})
and an induced homomorphism of graded rings
$$
f^* :
\Gamma_*(X, \mathcal{L})
\longrightarrow
\Gamma_*(Y, f^*\mathcal{L})
$$
Furthermore, there are some compatibilities between the constructions
above whose statements we omit.

\begin{lemma}
\label{lemma-pic-set}
Let $(X, \mathcal{O}_X)$ be a ringed space.
There exists a set of invertible modules $\{\mathcal{L}_i\}_{i \in I}$
such that each invertible module on $X$ is isomorphic to exactly
one of the $\mathcal{L}_i$.
\end{lemma}

\begin{proof}
Recall that any invertible $\mathcal{O}_X$-module is locally
a direct summand of a finite free $\mathcal{O}_X$-module, see
Lemma \ref{lemma-invertible}.
For each open covering $\mathcal{U} : X = \bigcup_{j \in J} U_j$
and map $r : J \to \mathbf{N}$ consider the sheaves of
$\mathcal{O}_X$-modules $\mathcal{F}$ such that
$\mathcal{F}_j = \mathcal{F}|_{U_j}$ is a direct summand of
$\mathcal{O}_{U_j}^{\oplus r(j)}$.
The collection of isomorphism classes of $\mathcal{F}_j$ is a set, because
$\Hom_{\mathcal{O}_U}(\mathcal{O}_U^{\oplus r}, \mathcal{O}_U^{\oplus r})$
is a set. The sheaf $\mathcal{F}$ is gotten by glueing $\mathcal{F}_j$,
see Sheaves, Section
\ref{sheaves-section-glueing-sheaves}. Note that the collection of
all glueing data forms a set. The collection of all coverings
$\mathcal{U} : X = \bigcup_{j \in J} U_i$ where $J \to \mathcal{P}(X)$,
$j \mapsto U_j$ is injective forms a set as well. For each covering
there is a set of maps $r : J \to \mathbf{N}$. Hence the collection
of all $\mathcal{F}$ forms a set.
\end{proof}

\noindent
This lemma says roughly speaking that the collection of
isomorphism classes of invertible sheaves forms a set.
Lemma \ref{lemma-constructions-invertible} says that
tensor product defines the structure of an abelian group
on this set.

\begin{definition}
\label{definition-pic}
Let $(X, \mathcal{O}_X)$ be a ringed space.
The {\it Picard group} $\Pic(X)$ of $X$ is the
abelian group whose elements are isomorphism classes of
invertible $\mathcal{O}_X$-modules, with addition
corresponding to tensor product.
\end{definition}

\begin{lemma}
\label{lemma-s-open}
\begin{slogan}
A (local) trivialisation of a linebundle
is the same as a (local) nonvanishing section.
\end{slogan}
Let $X$ be a ringed space. Assume that each stalk $\mathcal{O}_{X, x}$
is a local ring with maximal ideal $\mathfrak m_x$.
Let $\mathcal{L}$ be an invertible $\mathcal{O}_X$-module.
For any section $s \in \Gamma(X, \mathcal{L})$ the set
$$
X_s = \{x \in X \mid \text{image }s \not\in \mathfrak m_x\mathcal{L}_x\}
$$
is open in $X$. The map $s : \mathcal{O}_{X_s} \to \mathcal{L}|_{X_s}$
is an isomorphism, and there exists a section $s'$
of $\mathcal{L}^{\otimes -1}$ over $X_s$ such that $s' (s|_{X_s}) = 1$.
\end{lemma}

\begin{proof}
Suppose $x \in X_s$.
We have an isomorphism
$$
\mathcal{L}_x \otimes_{\mathcal{O}_{X, x}} (\mathcal{L}^{\otimes -1})_x
\longrightarrow
\mathcal{O}_{X, x}
$$
by Lemma \ref{lemma-constructions-invertible}.
Both $\mathcal{L}_x$ and $(\mathcal{L}^{\otimes -1})_x$
are free $\mathcal{O}_{X, x}$-modules of rank $1$. We conclude
from Algebra, Nakayama's Lemma \ref{algebra-lemma-NAK} that
$s_x$ is a basis for $\mathcal{L}_x$. Hence there exists
a basis element $t_x \in (\mathcal{L}^{\otimes -1})_x$
such that $s_x \otimes t_x$ maps to $1$.
Choose an open neighbourhood $U$ of
$x$ such that $t_x$ comes from a section $t$
of $\mathcal{L}^{\otimes -1}$ over $U$ and such that
$s \otimes t$ maps to $1 \in \mathcal{O}_X(U)$.
Clearly, for every $x' \in U$ we see that $s$ generates
the module $\mathcal{L}_{x'}$. Hence $U \subset X_s$.
This proves that $X_s$ is open. Moreover, the section
$t$ constructed over $U$ above is unique, and hence
these glue to give te section $s'$ of the lemma.
\end{proof}

\noindent
It is also true that, given a morphism of locally ringed
spaces $f : Y \to X$
(see Schemes, Definition \ref{schemes-definition-locally-ringed-space})
that the inverse image $f^{-1}(X_s)$ is equal to $Y_{f^*s}$, where
$f^*s \in \Gamma(Y, f^*\mathcal{L})$ is the pullback of $s$.



\section{Rank and determinant}
\label{section-rank-and-det}

\noindent
Let $(X, \mathcal{O}_X)$ be a ringed space. Consider the category
$\mathcal{C}$ of finite locally free $\mathcal{O}_X$-modules.
This is an exact category
(see Injectives, Remark \ref{injectives-remark-embed-exact-category})
whose admissible epimorphisms are
surjections and whose admissible monomorphisms are kernels of
surjections. Moreover, there is a set of isomorphism classes
of objects of $\mathcal{C}$ (proof omitted). Thus we can form
the Grothendieck $K$-group $K(\mathcal{C})$, which is often denoted
$K_0^{naive}(X)$. Explicitly, in this case $K_0^{naive}(X)$ is
the abelian group generated by $[\mathcal{E}]$ for $\mathcal{E}$
a finite locally free $\mathcal{O}_X$-module, subject to the relations
$$
[\mathcal{E}'] = [\mathcal{E}] + [\mathcal{E}'']
$$
whenever there is a short exact sequence
$0 \to \mathcal{E}' \to \mathcal{E} \to \mathcal{E}'' \to 0$
of finite locally free $\mathcal{O}_X$-modules.

\medskip\noindent
Ranks. Given a finite locally free $\mathcal{O}_X$-module $\mathcal{E}$,
the {\it rank} is a locally constant function
$$
r = r_\mathcal{E} : X \longrightarrow \mathbf{Z}_{\geq 0},\quad
x \longmapsto \text{rank}_{\mathcal{O}_{X, x}} \mathcal{E}_x
$$
This makes sense as $\mathcal{E}_x \cong \mathcal{O}_{X, x}^{\oplus r(x)}$
and this uniquely determines $r(x)$. By definition of locally free
modules the function $r$ is locally constant. If 
$0 \to \mathcal{E}' \to \mathcal{E} \to \mathcal{E}'' \to 0$
is a short exact sequence of finite locally free $\mathcal{O}_X$-modules,
then $r_\mathcal{E} = r_{\mathcal{E}'} + r_{\mathcal{E}''}$,
Thus the rank defines a homomorphism
$$
K_0^{naive}(X) \longrightarrow \text{Map}_{cont}(X, \mathbf{Z}),\quad
[\mathcal{E}] \longmapsto r_\mathcal{E}
$$

\medskip\noindent
Determinants. Given a  finite locally free
$\mathcal{O}_X$-module $\mathcal{E}$ we obtain a disjoint union
decomposition
$$
X = X_0 \amalg X_1 \amalg X_2 \amalg \ldots
$$
with $X_i$ open and closed, such that $\mathcal{E}$ is finite locally
free of rank $i$ on $X_i$ (this is exactly the same as saying the
$r_\mathcal{E}$ is locally constant). In this case we define
$\det(\mathcal{E})$ as the invertible sheaf on $X$ which is equal to
$\wedge^i(\mathcal{E}|_{X_i})$ on $X_i$ for all $i \geq 0$.
Since the decomposition above is disjoint, there are no glueing
conditions to check. By Lemma \ref{lemma-det-ses} below
this defines a homomorphism
$$
\det : K_0^{naive}(X) \longrightarrow \Pic(X),\quad
[\mathcal{E}] \longmapsto \det(\mathcal{E})
$$
of abelian groups.

\begin{lemma}
\label{lemma-det-ses}
Let $X$ be a ringed space. Let
$0 \to \mathcal{E}' \to \mathcal{E} \to \mathcal{E}'' \to 0$
be a short exact sequence of finite locally free $\mathcal{O}_X$-modules,
Then there is a canonical isomorphism
$$
\det(\mathcal{E}') \otimes_{\mathcal{O}_X}\det(\mathcal{E}'')
\longrightarrow
\det(\mathcal{E})
$$
of $\mathcal{O}_X$-modules.
\end{lemma}

\begin{proof}
We can decompose $X$ into disjoint open and closed subsets such that
both $\mathcal{E}'$ and $\mathcal{E}''$ have constant rank on them.
Thus we reduce to the case where $\mathcal{E}'$ and $\mathcal{E}''$
have constant rank, say $r'$ and $r''$. In this situation we
define
$$
\wedge^{r'}(\mathcal{E}') \otimes_{\mathcal{O}_X} \wedge^{r''}(\mathcal{E}'')
\longrightarrow
\wedge^{r' + r''}(\mathcal{E})
$$
as follows. Given local sections $s'_1, \ldots, s'_{r'}$ of $\mathcal{E}'$
and local sections $s''_1, \ldots, s''_{r''}$ of $\mathcal{E}''$
we map
$$
s'_1 \wedge \ldots \wedge s'_{r'} \otimes
s''_1 \wedge \ldots \wedge s''_{r''}
\quad\text{to}\quad
s'_1 \wedge \ldots \wedge s'_{r'} \wedge
\tilde s''_1 \wedge \ldots \wedge \tilde s''_{r''}
$$
where $\tilde s''_i$ is a local lift of the section
$s''_i$ to a section of $\mathcal{E}$. We omit the details.
\end{proof}






\section{Localizing sheaves of rings}
\label{section-localizing-sheaves-rings}

\noindent
Let $X$ be a topological space and let $\mathcal{O}_X$ be a
presheaf of rings. Let $\mathcal{S} \subset \mathcal{O}_X$
be a presheaf of sets contained in $\mathcal{O}_X$.
Suppose that for every open $U \subset X$ the set
$\mathcal{S}(U) \subset \mathcal{O}_X(U)$ is a multiplicative subset, see
Algebra, Definition \ref{algebra-definition-multiplicative-subset}.
In this case we can consider the presheaf of rings
$$
\mathcal{S}^{-1}\mathcal{O}_X :
U \longmapsto \mathcal{S}(U)^{-1}\mathcal{O}_X(U).
$$
The restriction mapping sends the section $f/s$, $f \in \mathcal{O}_X(U)$,
$s \in \mathcal{S}(U)$ to $(f|_V)/(s|_V)$ if $V \subset U$ are opens
of $X$.

\begin{lemma}
\label{lemma-simple-invert}
Let $X$ be a topological space and let $\mathcal{O}_X$ be a
presheaf of rings. Let $\mathcal{S} \subset \mathcal{O}_X$
be a pre-sheaf of sets contained in $\mathcal{O}_X$.
Suppose that for every open $U \subset X$ the set
$\mathcal{S}(U) \subset \mathcal{O}_X(U)$ is a multiplicative subset.
\begin{enumerate}
\item There is a map of presheaves of rings
$\mathcal{O}_X \to \mathcal{S}^{-1}\mathcal{O}_X$
such that every local section of $\mathcal{S}$ maps to an invertible
section of $\mathcal{O}_X$.
\item For any homomorphism of presheaves of rings
$\mathcal{O}_X \to \mathcal{A}$ such that each local section
of $\mathcal{S}$ maps to an invertible section of $\mathcal{A}$
there exists a unique factorization
$\mathcal{S}^{-1}\mathcal{O}_X \to \mathcal{A}$.
\item For any $x \in X$ we have
$$
(\mathcal{S}^{-1}\mathcal{O}_X)_x = \mathcal{S}_x^{-1} \mathcal{O}_{X, x}.
$$
\item The sheafification $(\mathcal{S}^{-1}\mathcal{O}_X)^\#$ is a sheaf
of rings with a map of sheaves of rings
$(\mathcal{O}_X)^\# \to (\mathcal{S}^{-1}\mathcal{O}_X)^\#$
which is universal for maps of $(\mathcal{O}_X)^\#$ into sheaves
of rings such that each local section of $\mathcal{S}$ maps
to an invertible section.
\item For any $x \in X$ we have
$$
(\mathcal{S}^{-1}\mathcal{O}_X)^\#_x = \mathcal{S}_x^{-1} \mathcal{O}_{X, x}.
$$
\end{enumerate}
\end{lemma}

\begin{proof}
Omitted.
\end{proof}

\noindent
Let $X$ be a topological space and let $\mathcal{O}_X$ be a
presheaf of rings. Let $\mathcal{S} \subset \mathcal{O}_X$
be a presheaf of sets contained in $\mathcal{O}_X$.
Suppose that for every open $U \subset X$ the set
$\mathcal{S}(U) \subset \mathcal{O}_X(U)$ is a multiplicative subset.
Let $\mathcal{F}$ be a presheaf of $\mathcal{O}_X$-modules
In this case we can consider the presheaf of
$\mathcal{S}^{-1}\mathcal{O}_X$-modules
$$
\mathcal{S}^{-1}\mathcal{F} :
U \longmapsto \mathcal{S}(U)^{-1}\mathcal{F}(U).
$$
The restriction mapping sends the section $t/s$, $t \in \mathcal{F}(U)$,
$s \in \mathcal{S}(U)$ to $(t|_V)/(s|_V)$ if $V \subset U$ are opens
of $X$.

\begin{lemma}
\label{lemma-simple-invert-module}
Let $X$ be a topological space.
Let $\mathcal{O}_X$ be a presheaf of rings.
Let $\mathcal{S} \subset \mathcal{O}_X$ be a pre-sheaf of sets contained
in $\mathcal{O}_X$. Suppose that for every open $U \subset X$ the set
$\mathcal{S}(U) \subset \mathcal{O}_X(U)$ is a multiplicative subset.
For any presheaf of $\mathcal{O}_X$-modules $\mathcal{F}$ we
have
$$
\mathcal{S}^{-1}\mathcal{F}
=
\mathcal{S}^{-1}\mathcal{O}_X \otimes_{p, \mathcal{O}_X} \mathcal{F}
$$
(see Sheaves, Section \ref{sheaves-section-presheaves-modules} for notation)
and if $\mathcal{F}$ and $\mathcal{O}_X$ are sheaves then
$$
(\mathcal{S}^{-1}\mathcal{F})^\#
=
(\mathcal{S}^{-1}\mathcal{O}_X)^\# \otimes_{\mathcal{O}_X} \mathcal{F}
$$
(see Sheaves, Section \ref{sheaves-section-sheafification-presheaves-modules}
for notation).
\end{lemma}

\begin{proof}
Omitted.
\end{proof}








\section{Modules of differentials}
\label{section-differentials}

\noindent
In this section we briefly explain how to define the module of relative
differentials for a morphism of ringed spaces. We suggest the reader take
a look at the corresponding section in the chapter on commutative algebra
(Algebra, Section \ref{algebra-section-differentials}).

\begin{definition}
\label{definition-derivation}
Let $X$ be a topological space. Let $\varphi : \mathcal{O}_1 \to \mathcal{O}_2$
be a homomorphism of sheaves of rings. Let $\mathcal{F}$ be an
$\mathcal{O}_2$-module. A {\it $\mathcal{O}_1$-derivation} or more precisely
a {\it $\varphi$-derivation} into $\mathcal{F}$ is a map
$D : \mathcal{O}_2 \to \mathcal{F}$ which is additive, annihilates the image
of $\mathcal{O}_1 \to \mathcal{O}_2$, and satisfies the
{\it Leibniz rule}
$$
D(ab) = aD(b) + D(a)b
$$
for all $a, b$ local sections of $\mathcal{O}_2$ (wherever they are both
defined). We denote $\text{Der}_{\mathcal{O}_1}(\mathcal{O}_2, \mathcal{F})$
the set of $\varphi$-derivations into $\mathcal{F}$.
\end{definition}

\noindent
This is the sheaf theoretic analogue of
Algebra, Definition \ref{algebra-definition-derivation}.
Given a derivation $D : \mathcal{O}_2 \to \mathcal{F}$
as in the definition the map on global sections
$$
D : \Gamma(X, \mathcal{O}_2) \longrightarrow \Gamma(X, \mathcal{F})
$$
is a $\Gamma(X, \mathcal{O}_1)$-derivation as in the algebra definition.
Note that if $\alpha : \mathcal{F} \to \mathcal{G}$ is a map of
$\mathcal{O}_2$-modules, then there is an induced map
$$
\text{Der}_{\mathcal{O}_1}(\mathcal{O}_2, \mathcal{F})
\longrightarrow
\text{Der}_{\mathcal{O}_1}(\mathcal{O}_2, \mathcal{G})
$$
given by the rule $D \mapsto \alpha \circ D$. In other words
we obtain a functor.

\begin{lemma}
\label{lemma-universal-module}
Let $X$ be a topological space. Let $\varphi : \mathcal{O}_1 \to \mathcal{O}_2$
be a homomorphism of sheaves of rings. The functor
$$
\textit{Mod}(\mathcal{O}_2) \longrightarrow \textit{Ab}, \quad
\mathcal{F} \longmapsto \text{Der}_{\mathcal{O}_1}(\mathcal{O}_2, \mathcal{F})
$$
is representable.
\end{lemma}

\begin{proof}
This is proved in exactly the same way as the analogous statement in algebra.
During this proof, for any sheaf of sets $\mathcal{F}$ on $X$,
let us denote $\mathcal{O}_2[\mathcal{F}]$ the sheafification of the
presheaf $U \mapsto \mathcal{O}_2(U)[\mathcal{F}(U)]$ where this denotes
the free $\mathcal{O}_1(U)$-module on the set $\mathcal{F}(U)$.
For $s \in \mathcal{F}(U)$ we denote $[s]$ the corresponding section
of $\mathcal{O}_2[\mathcal{F}]$ over $U$. If $\mathcal{F}$ is a sheaf of
$\mathcal{O}_2$-modules, then there is a canonical map
$$
c : \mathcal{O}_2[\mathcal{F}] \longrightarrow \mathcal{F}
$$
which on the presheaf level is given by the rule
$\sum f_s[s] \mapsto \sum f_s s$. We will employ the short hand
$[s] \mapsto s$ to
describe this map and similarly for other maps below. Consider
the map of $\mathcal{O}_2$-modules
\begin{equation}
\label{equation-define-module-differentials}
\begin{matrix}
\mathcal{O}_2[\mathcal{O}_2 \times \mathcal{O}_2] \oplus
\mathcal{O}_2[\mathcal{O}_2 \times \mathcal{O}_2] \oplus
\mathcal{O}_2[\mathcal{O}_1] &
\longrightarrow &
\mathcal{O}_2[\mathcal{O}_2] \\
[(a, b)] \oplus [(f, g)] \oplus [h] & \longmapsto & [a + b] - [a] - [b] + \\
& & [fg] - g[f] - f[g] + \\
& & [\varphi(h)]
\end{matrix}
\end{equation}
with short hand notation as above. Set $\Omega_{\mathcal{O}_2/\mathcal{O}_1}$
equal to the cokernel of this map. Then it is clear that there exists
a map of sheaves of sets
$$
\text{d} : \mathcal{O}_2 \longrightarrow \Omega_{\mathcal{O}_2/\mathcal{O}_1}
$$
mapping a local section $f$ to the image of $[f]$ in
$\Omega_{\mathcal{O}_2/\mathcal{O}_1}$. By construction $\text{d}$
is a $\mathcal{O}_1$-derivation. Next, let $\mathcal{F}$
be a sheaf of $\mathcal{O}_2$-modules and let
$D : \mathcal{O}_2 \to \mathcal{F}$ be a $\mathcal{O}_1$-derivation.
Then we can consider the $\mathcal{O}_2$-linear map
$\mathcal{O}_2[\mathcal{O}_2] \to \mathcal{F}$ which sends $[g]$ to $D(g)$.
It follows from the definition of a derivation that this map annihilates
sections in the image of the map (\ref{equation-define-module-differentials})
and hence defines a map
$$
\alpha_D : \Omega_{\mathcal{O}_2/\mathcal{O}_1} \longrightarrow \mathcal{F}
$$
Since it is clear that $D = \alpha_D \circ \text{d}$ the lemma is proved.
\end{proof}

\begin{definition}
\label{definition-module-differentials}
Let $X$ be a topological space. Let $\varphi : \mathcal{O}_1 \to \mathcal{O}_2$
be a homomorphism of sheaves of rings on $X$. The {\it module of differentials}
of $\varphi$ is the object representing the functor
$\mathcal{F} \mapsto \text{Der}_{\mathcal{O}_1}(\mathcal{O}_2, \mathcal{F})$
which exists by Lemma \ref{lemma-universal-module}.
It is denoted $\Omega_{\mathcal{O}_2/\mathcal{O}_1}$, and the {\it universal
$\varphi$-derivation} is denoted
$\text{d} : \mathcal{O}_2 \to \Omega_{\mathcal{O}_2/\mathcal{O}_1}$.
\end{definition}

\noindent
Note that $\Omega_{\mathcal{O}_2/\mathcal{O}_1}$ is the cokernel of
the map (\ref{equation-define-module-differentials}) of
$\mathcal{O}_2$-modules. Moreover the map $\text{d}$ is described
by the rule that $\text{d}f$ is the image of the local section $[f]$.

\begin{lemma}
\label{lemma-differentials-sheafify}
Let $X$ be a topological space. Let $\varphi : \mathcal{O}_1 \to \mathcal{O}_2$
be a homomorphism of sheaves of rings on $X$. Then
$\Omega_{\mathcal{O}_2/\mathcal{O}_1}$ is the sheaf associated to the
presheaf $U \mapsto \Omega_{\mathcal{O}_2(U)/\mathcal{O}_1(U)}$.
\end{lemma}

\begin{proof}
Consider the map (\ref{equation-define-module-differentials}). There is
a similar map of presheaves whose value on the open $U$ is
$$
\mathcal{O}_2(U)[\mathcal{O}_2(U) \times \mathcal{O}_2(U)] \oplus
\mathcal{O}_2(U)[\mathcal{O}_2(U) \times \mathcal{O}_2(U)] \oplus
\mathcal{O}_2(U)[\mathcal{O}_1(U)]
\longrightarrow
\mathcal{O}_2(U)[\mathcal{O}_2(U)]
$$
The cokernel of this map has value $\Omega_{\mathcal{O}_2(U)/\mathcal{O}_1(U)}$
over $U$ by the construction of the module of differentials in 
Algebra, Definition \ref{algebra-definition-differentials}.
On the other hand, the sheaves in (\ref{equation-define-module-differentials})
are the sheafifications of the presheaves above. Thus the result follows
as sheafification is exact.
\end{proof}

\begin{lemma}
\label{lemma-localize-differentials}
Let $X$ be a topological space. Let $\varphi : \mathcal{O}_1 \to \mathcal{O}_2$
be a homomorphism of sheaves of rings. For $U \subset X$ open
there is a canonical isomorphism
$$
\Omega_{\mathcal{O}_2/\mathcal{O}_1}|_U =
\Omega_{(\mathcal{O}_2|_U)/(\mathcal{O}_1|_U)}
$$
compatible with universal derivations.
\end{lemma}

\begin{proof}
Holds because $\Omega_{\mathcal{O}_2/\mathcal{O}_1}$
is the cokernel of the map (\ref{equation-define-module-differentials}).
\end{proof}

\begin{lemma}
\label{lemma-pullback-differentials}
Let $f : Y \to X$ be a continuous map of topological spaces.
Let $\varphi : \mathcal{O}_1 \to \mathcal{O}_2$
be a homomorphism of sheaves of rings on $X$.
Then there is a canonical identification
$f^{-1}\Omega_{\mathcal{O}_2/\mathcal{O}_1} =
\Omega_{f^{-1}\mathcal{O}_2/f^{-1}\mathcal{O}_1}$
compatible with universal derivations.
\end{lemma}

\begin{proof}
This holds because the sheaf $\Omega_{\mathcal{O}_2/\mathcal{O}_1}$
is the cokernel of the map (\ref{equation-define-module-differentials})
and a similar statement holds for
$\Omega_{f^{-1}\mathcal{O}_2/f^{-1}\mathcal{O}_1}$,
because the functor $f^{-1}$ is exact, and because
$f^{-1}(\mathcal{O}_2[\mathcal{O}_2]) =
f^{-1}\mathcal{O}_2[f^{-1}\mathcal{O}_2]$,
$f^{-1}(\mathcal{O}_2[\mathcal{O}_2 \times \mathcal{O}_2]) =
f^{-1}\mathcal{O}_2[f^{-1}\mathcal{O}_2 \times f^{-1}\mathcal{O}_2]$, and
$f^{-1}(\mathcal{O}_2[\mathcal{O}_1]) =
f^{-1}\mathcal{O}_2[f^{-1}\mathcal{O}_1]$.
\end{proof}

\begin{lemma}
\label{lemma-stalk-module-differentials}
Let $X$ be a topological space. Let $\mathcal{O}_1 \to \mathcal{O}_2$
be a homomorphism of sheaves of rings on $X$. Let $x \in X$. Then we have
$\Omega_{\mathcal{O}_2/\mathcal{O}_1, x} =
\Omega_{\mathcal{O}_{2, x}/\mathcal{O}_{1, x}}$.
\end{lemma}

\begin{proof}
This is a special case of Lemma \ref{lemma-pullback-differentials}
for the inclusion map $\{x\} \to X$. An alternative proof is the use
Lemma \ref{lemma-differentials-sheafify},
Sheaves, Lemma \ref{sheaves-lemma-stalk-sheafification}, and
Algebra, Lemma \ref{algebra-lemma-colimit-differentials}
\end{proof}

\begin{lemma}
\label{lemma-functoriality-differentials}
Let $X$ be a topological space. Let
$$
\xymatrix{
\mathcal{O}_2 \ar[r]_\varphi & \mathcal{O}_2' \\
\mathcal{O}_1 \ar[r] \ar[u] & \mathcal{O}'_1 \ar[u]
}
$$
be a commutative diagram of sheaves of rings on $X$. The map
$\mathcal{O}_2 \to \mathcal{O}'_2$ composed with the map
$\text{d} : \mathcal{O}'_2 \to \Omega_{\mathcal{O}'_2/\mathcal{O}'_1}$
is a $\mathcal{O}_1$-derivation. Hence we obtain a canonical map of
$\mathcal{O}_2$-modules
$\Omega_{\mathcal{O}_2/\mathcal{O}_1} \to
\Omega_{\mathcal{O}'_2/\mathcal{O}'_1}$.
It is uniquely characterized by the property that
$\text{d}(f) \mapsto \text{d}(\varphi(f))$
for any local section $f$ of $\mathcal{O}_2$.
In this way $\Omega_{-/-}$ becomes a functor on the category
of arrows of sheaves of rings.
\end{lemma}

\begin{proof}
This lemma proves itself.
\end{proof}

\begin{lemma}
\label{lemma-differential-seq}
In Lemma \ref{lemma-functoriality-differentials} suppose that
$\mathcal{O}_2 \to \mathcal{O}'_2$ is surjective with kernel
$\mathcal{I} \subset \mathcal{O}_2$ and assume that
$\mathcal{O}_1 = \mathcal{O}'_1$. Then there is a canonical exact
sequence of $\mathcal{O}'_2$-modules
$$
\mathcal{I}/\mathcal{I}^2
\longrightarrow
\Omega_{\mathcal{O}_2/\mathcal{O}_1} \otimes_{\mathcal{O}_2} \mathcal{O}'_2
\longrightarrow
\Omega_{\mathcal{O}'_2/\mathcal{O}_1}
\longrightarrow
0
$$
The leftmost map is characterized by the rule that a local section
$f$ of $\mathcal{I}$ maps to $\text{d}f \otimes 1$.
\end{lemma}

\begin{proof}
For a local section $f$ of $\mathcal{I}$ denote $\overline{f}$ the image of
$f$ in $\mathcal{I}/\mathcal{I}^2$. To show that the map
$\overline{f} \mapsto \text{d}f \otimes 1$ is well defined we just have to
check that $\text{d} f_1f_2 \otimes 1 = 0$ if $f_1, f_2$ are local sections
of $\mathcal{I}$. And this is clear from the Leibniz rule
$\text{d} f_1f_2 \otimes 1 =
(f_1 \text{d}f_2 + f_2 \text{d} f_1 )\otimes 1 =
\text{d}f_2 \otimes f_1 + \text{d}f_1 \otimes f_2 = 0$.
A similar computation show this map is
$\mathcal{O}'_2 = \mathcal{O}_2/\mathcal{I}$-linear. The map on the right
is the one from Lemma \ref{lemma-functoriality-differentials}. To see
that the sequence is exact, we can check on stalks
(Lemma \ref{lemma-abelian}). By
Lemma \ref{lemma-stalk-module-differentials}
this follows from
Algebra, Lemma \ref{algebra-lemma-differential-seq}.
\end{proof}

\begin{definition}
\label{definition-differentials}
Let $(f, f^\sharp) : (X, \mathcal{O}_X) \to (S, \mathcal{O}_S)$
be a morphism of ringed spaces.
\begin{enumerate}
\item Let $\mathcal{F}$ be an $\mathcal{O}_X$-module. An {\it $S$-derivation}
into $\mathcal{F}$ is a $f^{-1}\mathcal{O}_S$-derivation, or more
precisely a $f^\sharp$-derivation in the sense of
Definition \ref{definition-derivation}.
We denote $\text{Der}_S(\mathcal{O}_X, \mathcal{F})$
the set of $S$-derivations into $\mathcal{F}$.
\item The {\it sheaf of differentials $\Omega_{X/S}$ of $X$ over $S$}
is the module of differentials $\Omega_{\mathcal{O}_X/f^{-1}\mathcal{O}_S}$
endowed with its universal
$S$-derivation $\text{d}_{X/S} : \mathcal{O}_X \to \Omega_{X/S}$.
\end{enumerate}
\end{definition}

\noindent
Here is a particular situation where derivations come up
naturally.

\begin{lemma}
\label{lemma-double-structure-gives-derivation}
Let $(f, f^\sharp) : (X, \mathcal{O}_X) \to (S, \mathcal{O}_S)$
be a morphism of ringed spaces. Consider a short exact sequence
$$
0 \to \mathcal{I} \to \mathcal{A} \to \mathcal{O}_X \to 0
$$
Here $\mathcal{A}$ is a sheaf of $f^{-1}\mathcal{O}_S$-algebras,
$\pi : \mathcal{A} \to \mathcal{O}_X$ is a surjection
of sheaves of $f^{-1}\mathcal{O}_S$-algebras, and
$\mathcal{I} = \Ker(\pi)$ is its kernel. Assume $\mathcal{I}$ an ideal
sheaf with square zero in $\mathcal{A}$. So $\mathcal{I}$
has a natural structure of an $\mathcal{O}_X$-module.
A section $s : \mathcal{O}_X \to \mathcal{A}$ of $\pi$
is a $f^{-1}\mathcal{O}_S$-algebra map such that $\pi \circ s = \text{id}$.
Given any section $s : \mathcal{O}_X \to \mathcal{A}$
of $\pi$ and any $S$-derivation $D : \mathcal{O}_X \to \mathcal{I}$
the map
$$
s + D : \mathcal{O}_X \to \mathcal{A}
$$
is a section of $\pi$ and every section $s'$ is of the form $s + D$
for a unique $S$-derivation $D$.
\end{lemma}

\begin{proof}
Recall that the $\mathcal{O}_X$-module structure on $\mathcal{I}$
is given by $h \tau = \tilde h \tau$ (multiplication in $\mathcal{A}$)
where $h$ is a local section of $\mathcal{O}_X$, and
$\tilde h$ is a local lift of $h$ to a local
section of $\mathcal{A}$, and $\tau$ is a local section of $\mathcal{I}$.
In particular, given $s$, we may use $\tilde h = s(h)$.
To verify that $s + D$ is a homomorphism of sheaves of rings we
compute
\begin{eqnarray*}
(s + D)(ab) & = & s(ab) + D(ab) \\
& = & s(a)s(b) + aD(b) + D(a)b \\
& = & s(a) s(b) + s(a)D(b) + D(a)s(b) \\
& = & (s(a) + D(a))(s(b) + D(b))
\end{eqnarray*}
by the Leibniz rule. In the same manner one shows
$s + D$ is a $f^{-1}\mathcal{O}_S$-algebra
map because $D$ is an $S$-derivation. Conversely, given $s'$ we set
$D = s' - s$. Details omitted.
\end{proof}

\begin{lemma}
\label{lemma-functoriality-differentials-ringed-spaces}
Let
$$
\xymatrix{
X' \ar[d]_{h'} \ar[r]_f & X \ar[d]^h \\
S' \ar[r]^g & S
}
$$
be a commutative diagram of ringed spaces.
\begin{enumerate}
\item The canonical map $\mathcal{O}_X \to f_*\mathcal{O}_{X'}$ composed with
$f_*\text{d}_{X'/S'} : f_*\mathcal{O}_{X'} \to f_*\Omega_{X'/S'}$ is a
$S$-derivation and we obtain a canonical map of $\mathcal{O}_X$-modules
$\Omega_{X/S} \to f_*\Omega_{X'/S'}$.
\item The commutative diagram
$$
\xymatrix{
f^{-1}\mathcal{O}_X \ar[r] & \mathcal{O}_{X'} \\
f^{-1}h^{-1}\mathcal{O}_S \ar[u] \ar[r] & (h')^{-1}\mathcal{O}_{S'} \ar[u]
}
$$
induces by Lemmas \ref{lemma-pullback-differentials} and
\ref{lemma-functoriality-differentials}
a canonical map $f^{-1}\Omega_{X/S} \to \Omega_{X'/S'}$.
\end{enumerate}
These two maps correspond (via adjointness of $f_*$ and $f^*$ and
via $f^*\Omega_{X/S} =
f^{-1}\Omega_{X/S} \otimes_{f^{-1}\mathcal{O}_X} \mathcal{O}_{X'}$ and
Sheaves, Lemma \ref{sheaves-lemma-adjointness-tensor-restrict})
to the same $\mathcal{O}_{X'}$-module homomorphism
$$
c_f : f^*\Omega_{X/S} \longrightarrow \Omega_{X'/S'}
$$
which is uniquely characterized by the property that
$f^*\text{d}_{X/S}(a)$ maps to $\text{d}_{X'/S'}(f^*a)$
for any local section $a$ of $\mathcal{O}_X$.
\end{lemma}

\begin{proof}
Omitted.
\end{proof}

\begin{lemma}
\label{lemma-check-functoriality-differentials}
Let
$$
\xymatrix{
X'' \ar[d] \ar[r]_g & X' \ar[d] \ar[r]_f & X \ar[d] \\
S'' \ar[r] & S' \ar[r] & S
}
$$
be a commutative diagram of ringed spaces. With notation as in
Lemma \ref{lemma-functoriality-differentials-ringed-spaces} we have
$$
c_{f \circ g} = c_g \circ g^* c_f
$$
as maps $(f \circ g)^*\Omega_{X/S} \to \Omega_{X''/S''}$.
\end{lemma}

\begin{proof}
Omitted.
\end{proof}


\section{The naive cotangent complex}
\label{section-netherlander}

\noindent
This section is the analogue of
Algebra, Section \ref{algebra-section-netherlander} for morphisms of ringed
spaces. We urge the reader to read that section first.

\medskip\noindent
Let $X$ be a topological space. Let $\mathcal{A} \to \mathcal{B}$ be a
homomorphism of sheaves of rings. In this section, for any sheaf of sets
$\mathcal{E}$ on $X$ we denote $\mathcal{A}[\mathcal{E}]$ the sheafification
of the presheaf $U \mapsto \mathcal{A}(U)[\mathcal{E}(U)]$. Here
$\mathcal{A}(U)[\mathcal{E}(U)]$
denotes the polynomial algebra over $\mathcal{A}(U)$
whose variables correspond to the elements of $\mathcal{E}(U)$.
We denote $[e] \in \mathcal{A}(U)[\mathcal{E}(U)]$ the variable
corresponding to $e \in \mathcal{E}(U)$.
There is a canonical surjection of $\mathcal{A}$-algebras
\begin{equation}
\label{equation-canonical-presentation}
\mathcal{A}[\mathcal{B}] \longrightarrow \mathcal{B},\quad [b] \longmapsto b
\end{equation}
whose kernel we denote $\mathcal{I} \subset \mathcal{A}[\mathcal{B}]$.
It is a simple observation that $\mathcal{I}$ is generated by the
local sections $[b][b'] - [bb']$ and $[a] - a$. According to
Lemma \ref{lemma-differential-seq} there is a canonical map
\begin{equation}
\label{equation-naive-cotangent-complex}
\mathcal{I}/\mathcal{I}^2
\longrightarrow
\Omega_{\mathcal{A}[\mathcal{B}]/\mathcal{A}}
\otimes_{\mathcal{A}[\mathcal{B}]} \mathcal{B}
\end{equation}
whose cokernel is canonically isomorphic to $\Omega_{\mathcal{B}/\mathcal{A}}$.

\begin{definition}
\label{definition-naive-cotangent-complex}
Let $X$ be a topological space. Let $\mathcal{A} \to \mathcal{B}$ be a
homomorphism of sheaves of rings. The {\it naive cotangent complex}
$\NL_{\mathcal{B}/\mathcal{A}}$ is the chain complex
(\ref{equation-naive-cotangent-complex})
$$
\NL_{\mathcal{B}/\mathcal{A}} =
\left(\mathcal{I}/\mathcal{I}^2
\longrightarrow
\Omega_{\mathcal{A}[\mathcal{B}]/\mathcal{A}}
\otimes_{\mathcal{A}[\mathcal{B}]} \mathcal{B}\right)
$$
with $\mathcal{I}/\mathcal{I}^2$ placed in degree $-1$ and
$\Omega_{\mathcal{A}[\mathcal{B}]/\mathcal{A}}
\otimes_{\mathcal{A}[\mathcal{B}]} \mathcal{B}$
placed in degree $0$.
\end{definition}

\noindent
This construction satisfies a functoriality similar to that discussed
in Lemma \ref{lemma-functoriality-differentials} for modules of differentials.
Namely, given a commutative diagram
\begin{equation}
\label{equation-commutative-square-sheaves}
\vcenter{
\xymatrix{
\mathcal{B} \ar[r] & \mathcal{B}' \\
\mathcal{A} \ar[u] \ar[r] & \mathcal{A}' \ar[u]
}
}
\end{equation}
of sheaves of rings on $X$ there is a canonical
$\mathcal{B}$-linear map of complexes
$$
\NL_{\mathcal{B}/\mathcal{A}} \longrightarrow \NL_{\mathcal{B}'/\mathcal{A}'}
$$
Namely, the maps in the commutative diagram give rise to a canonical map
$\mathcal{A}[\mathcal{B}] \to \mathcal{A}'[\mathcal{B}']$
which maps $\mathcal{I}$ into
$\mathcal{I}' = \Ker(\mathcal{A}'[\mathcal{B}'] \to \mathcal{B}')$.
Thus a map $\mathcal{I}/\mathcal{I}^2 \to \mathcal{I}'/(\mathcal{I}')^2$
and a map between modules of differentials, which together give the
desired map between the naive cotangent complexes.
The map is compatible with compositions in the following sense:
given a commutative diagram
$$
\xymatrix{
\mathcal{B} \ar[r] &
\mathcal{B}' \ar[r] &
\mathcal{B}'' \\
\mathcal{A} \ar[u] \ar[r] &
\mathcal{A}' \ar[u] \ar[r] &
\mathcal{A}'' \ar[u]
}
$$
of sheaves of rings then the composition
$$
\NL_{\mathcal{B}/\mathcal{A}} \longrightarrow
\NL_{\mathcal{B}'/\mathcal{A}'} \longrightarrow
\NL_{\mathcal{B}''/\mathcal{A}''}
$$
is the map for the outer rectangle.

\medskip\noindent
We can choose a different presentation of $\mathcal{B}$ as a quotient of a
polynomial algebra over $\mathcal{A}$ and still obtain the same object
of $D(\mathcal{B})$. To explain this, suppose that $\mathcal{E}$ is
a sheaves of sets on $X$ and $\alpha : \mathcal{E} \to \mathcal{B}$
a map of sheaves of sets. Then we obtain an $\mathcal{A}$-algebra
homomorphism $\mathcal{A}[\mathcal{E}] \to \mathcal{B}$. If this map
is surjective, i.e., if $\alpha(\mathcal{E})$ generates $\mathcal{B}$
as an $\mathcal{A}$-algebra, then we set
$$
\NL(\alpha) = \left(
\mathcal{J}/\mathcal{J}^2
\longrightarrow
\Omega_{\mathcal{A}[\mathcal{E}]/\mathcal{A}}
\otimes_{\mathcal{A}[\mathcal{E}]} \mathcal{B}\right)
$$
where $\mathcal{J} \subset \mathcal{A}[\mathcal{E}]$ is the kernel
of the surjection $\mathcal{A}[\mathcal{E}] \to \mathcal{B}$.
Here is the result.

\begin{lemma}
\label{lemma-NL-up-to-qis}
In the situation above there is a canonical isomorphism
$\NL(\alpha) = \NL_{\mathcal{B}/\mathcal{A}}$ in $D(\mathcal{B})$.
\end{lemma}

\begin{proof}
Observe that $\NL_{\mathcal{B}/\mathcal{A}} = \NL(\text{id}_\mathcal{B})$.
Thus it suffices to show that given two maps
$\alpha_i : \mathcal{E}_i \to \mathcal{B}$ as above, there is a
canonical quasi-isomorphism $\NL(\alpha_1) = \NL(\alpha_2)$ in $D(\mathcal{B})$.
To see this set $\mathcal{E} = \mathcal{E}_1 \amalg \mathcal{E}_2$ and
$\alpha = \alpha_1 \amalg \alpha_2 : \mathcal{E} \to \mathcal{B}$.
Set
$\mathcal{J}_i = \Ker(\mathcal{A}[\mathcal{E}_i] \to \mathcal{B})$
and
$\mathcal{J} = \Ker(\mathcal{A}[\mathcal{E}] \to \mathcal{B})$.
We obtain maps $\mathcal{A}[\mathcal{E}_i] \to \mathcal{A}[\mathcal{E}]$
which send $\mathcal{J}_i$ into $\mathcal{J}$.
Thus we obtain canonical maps of complexes
$$
\NL(\alpha_i) \longrightarrow \NL(\alpha)
$$
and it suffices to show these maps are quasi-isomorphism. To see this
it suffices to check on stalks (Lemma \ref{lemma-abelian}). If $x \in X$
then the stalk of $\NL(\alpha)$ is the complex $\NL(\alpha_x)$ of
Algebra, Section \ref{algebra-section-netherlander}
associated to the presentation $\mathcal{A}_x[\mathcal{E}_x] \to \mathcal{B}_x$
coming from the map $\alpha_x : \mathcal{E}_x \to \mathcal{B}_x$.
(Some details omitted; use Lemma \ref{lemma-stalk-module-differentials}
to see compatibility of forming differentials and taking stalks.)
We conclude the result holds by
Algebra, Lemma \ref{algebra-lemma-NL-homotopy}.
\end{proof}

\begin{lemma}
\label{lemma-pullback-NL}
Let $f : X \to Y$ be a continuous map of topological spaces.
Let $\mathcal{A} \to \mathcal{B}$ be a homomorphism of sheaves of rings
on $Y$. Then $f^{-1}\NL_{\mathcal{B}/\mathcal{A}} =
\NL_{f^{-1}\mathcal{B}/f^{-1}\mathcal{A}}$.
\end{lemma}

\begin{proof}
Omitted. Hint: Use Lemma \ref{lemma-pullback-differentials}.
\end{proof}

\begin{lemma}
\label{lemma-stalk-NL}
Let $X$ be a topological space. Let $\mathcal{A} \to \mathcal{B}$
be a homomorphism of sheaves of rings on $X$. Let $x \in X$.
Then we have $\NL_{\mathcal{B}/\mathcal{A}, x} =
\NL_{\mathcal{B}_x/\mathcal{A}_x}$.
\end{lemma}

\begin{proof}
This is a special case of Lemma \ref{lemma-pullback-NL}
for the inclusion map $\{x\} \to X$.
\end{proof}

\begin{lemma}
\label{lemma-exact-sequence-NL}
Let $X$ be a topological space. Let
$\mathcal{A} \to \mathcal{B} \to \mathcal{C}$
be maps of sheaves of rings. Let $C$ be the cone
(Derived Categories, Definition \ref{derived-definition-cone})
of the map of complexes
$\NL_{\mathcal{C}/\mathcal{A}} \to \NL_{\mathcal{C}/\mathcal{B}}$.
There is a canonical map
$$
c :
\NL_{\mathcal{B}/\mathcal{A}} \otimes_\mathcal{B} \mathcal{C}
\longrightarrow
C[-1]
$$
of complexes of $\mathcal{C}$-modules
which produces a canonical six term exact sequence
$$
\xymatrix{
H^0(\NL_{\mathcal{B}/\mathcal{A}} \otimes_\mathcal{B} \mathcal{C}) \ar[r] &
H^0(\NL_{\mathcal{C}/\mathcal{A}}) \ar[r] &
H^0(\NL_{\mathcal{C}/\mathcal{B}}) \ar[r] &
0 \\
H^{-1}(\NL_{\mathcal{B}/\mathcal{A}} \otimes_\mathcal{B} \mathcal{C}) \ar[r] &
H^{-1}(\NL_{\mathcal{C}/\mathcal{A}}) \ar[r] &
H^{-1}(\NL_{\mathcal{C}/\mathcal{B}}) \ar[llu]
}
$$
of cohomology sheaves.
\end{lemma}

\begin{proof}
To give the map $c$ we have to give a map
$c_1 : \NL_{\mathcal{B}/\mathcal{A}} \otimes_\mathcal{B} \mathcal{C}
\to \NL_{\mathcal{C}/\mathcal{A}}$ and an explicity homotopy
between the composition
$$
\NL_{\mathcal{B}/\mathcal{A}} \otimes_\mathcal{B} \mathcal{C} \to
\NL_{\mathcal{C}/\mathcal{A}} \to \NL_{\mathcal{C}/\mathcal{B}}
$$
and the zero map, see
Derived Categories, Lemma \ref{derived-lemma-map-from-cone}.
For $c_1$ we use the functoriality described
above for the obvious diagram. For the homotopy we use the map
$$
\NL_{\mathcal{B}/\mathcal{A}}^0 \otimes_\mathcal{B} \mathcal{C}
\longrightarrow
\NL_{\mathcal{C}/\mathcal{B}}^{-1},\quad
\text{d}[b] \otimes 1 \longmapsto [\varphi(b)] - b[1]
$$
where $\varphi : \mathcal{B} \to \mathcal{C}$ is the given map.
Please compare with
Algebra, Remark \ref{algebra-remark-composition-homotopy-equivalent-to-zero}.
To see the consequence for cohomology sheaves, it suffices to show
that $H^0(c)$ is an isomorphism and $H^{-1}(c)$ surjective.
To see this we can look at stalks, see Lemma \ref{lemma-stalk-NL},
and then we can use the corresponding result in commutative algebra,
see Algebra, Lemma \ref{algebra-lemma-exact-sequence-NL}.
Some details omitted.
\end{proof}

\noindent
The cotangent complex of a morphism of ringed spaces is defined
in terms of the cotangent complex we defined above.

\begin{definition}
\label{definition-cotangent-complex-morphism-ringed-topoi}
The {\it naive cotangent complex} $\NL_f = \NL_{X/Y}$ of a morphism of ringed
spaces $f : (X, \mathcal{O}_X) \to (Y, \mathcal{O}_Y)$ is
$\NL_{\mathcal{O}_X/f^{-1}\mathcal{O}_Y}$.
\end{definition}

\noindent
Given a commutative diagram
$$
\xymatrix{
X' \ar[r]_g \ar[d]_{f'} & X \ar[d]^f \\
Y' \ar[r]^h & Y
}
$$
of ringed spaces, there is a canonical map $c : g^*\NL_{X/Y} \to \NL_{X'/Y'}$.
Namely, it is the map
$$
g^*\NL_{X/Y} =
\mathcal{O}_{X'} \otimes_{g^{-1}\mathcal{O}_X}
\NL_{g^{-1}\mathcal{O}_X/g^{-1}f^{-1}\mathcal{O}_Y}
\longrightarrow
\NL_{\mathcal{O}_{X'}/(f')^{-1}\mathcal{O}_{Y'}} = \NL_{X'/Y'}
$$
where the arrow comes from the commutative diagram of sheaves of rings
$$
\xymatrix{
g^{-1}\mathcal{O}_X \ar[r]_{g^\sharp} &
\mathcal{O}_{X'} \\
g^{-1}f^{-1}\mathcal{O}_Y
\ar[r]^{g^{-1}h^\sharp}
\ar[u]^{g^{-1}f^\sharp} &
(f')^{-1}\mathcal{O}_{Y'} \ar[u]_{(f')^\sharp}
}
$$
as in (\ref{equation-commutative-square-sheaves}) above.
Given a second such diagram
$$
\xymatrix{
X'' \ar[r]_{g'} \ar[d] & X' \ar[d] \\
Y'' \ar[r] & Y'
}
$$
the composition of $(g')^*c$ and the map
$c' : (g')^*\NL_{X'/Y'} \to \NL_{X''/Y''}$
is the map $(g \circ g')^*\NL_{X''/Y''} \to \NL_{X/Y}$.

\begin{lemma}
\label{lemma-exact-sequence-NL-ringed-topoi}
Let $f : X \to Y$ and $g : Y \to Z$ be morphisms of ringed spaces.
Let $C$ be the cone of the map $\NL_{X/Z} \to \NL_{X/Y}$ of complexes
of $\mathcal{O}_X$-modules. There is a canonical map
$$
f^*\NL_{Y/Z} \to C[-1]
$$
which produces a canonical six term exact sequence
$$
\xymatrix{
H^0(f^*\NL_{Y/Z}) \ar[r] &
H^0(\NL_{X/Z}) \ar[r] &
H^0(\NL_{X/Y}) \ar[r] &
0 \\
H^{-1}(f^*\NL_{Y/Z}) \ar[r] &
H^{-1}(\NL_{X/Z}) \ar[r] &
H^{-1}(\NL_{X/Y}) \ar[llu]
}
$$
of cohomology sheaves.
\end{lemma}

\begin{proof}
Consider the maps of sheaves rings
$$
(g \circ f)^{-1}\mathcal{O}_Z \to f^{-1}\mathcal{O}_Y \to \mathcal{O}_X
$$
and apply Lemma \ref{lemma-exact-sequence-NL}.
\end{proof}







\begin{multicols}{2}[\section{Other chapters}]
\noindent
Preliminaries
\begin{enumerate}
\item \hyperref[introduction-section-phantom]{Introduction}
\item \hyperref[conventions-section-phantom]{Conventions}
\item \hyperref[sets-section-phantom]{Set Theory}
\item \hyperref[categories-section-phantom]{Categories}
\item \hyperref[topology-section-phantom]{Topology}
\item \hyperref[sheaves-section-phantom]{Sheaves on Spaces}
\item \hyperref[sites-section-phantom]{Sites and Sheaves}
\item \hyperref[stacks-section-phantom]{Stacks}
\item \hyperref[fields-section-phantom]{Fields}
\item \hyperref[algebra-section-phantom]{Commutative Algebra}
\item \hyperref[brauer-section-phantom]{Brauer Groups}
\item \hyperref[homology-section-phantom]{Homological Algebra}
\item \hyperref[derived-section-phantom]{Derived Categories}
\item \hyperref[simplicial-section-phantom]{Simplicial Methods}
\item \hyperref[more-algebra-section-phantom]{More on Algebra}
\item \hyperref[smoothing-section-phantom]{Smoothing Ring Maps}
\item \hyperref[modules-section-phantom]{Sheaves of Modules}
\item \hyperref[sites-modules-section-phantom]{Modules on Sites}
\item \hyperref[injectives-section-phantom]{Injectives}
\item \hyperref[cohomology-section-phantom]{Cohomology of Sheaves}
\item \hyperref[sites-cohomology-section-phantom]{Cohomology on Sites}
\item \hyperref[dga-section-phantom]{Differential Graded Algebra}
\item \hyperref[dpa-section-phantom]{Divided Power Algebra}
\item \hyperref[sdga-section-phantom]{Differential Graded Sheaves}
\item \hyperref[hypercovering-section-phantom]{Hypercoverings}
\end{enumerate}
Schemes
\begin{enumerate}
\setcounter{enumi}{25}
\item \hyperref[schemes-section-phantom]{Schemes}
\item \hyperref[constructions-section-phantom]{Constructions of Schemes}
\item \hyperref[properties-section-phantom]{Properties of Schemes}
\item \hyperref[morphisms-section-phantom]{Morphisms of Schemes}
\item \hyperref[coherent-section-phantom]{Cohomology of Schemes}
\item \hyperref[divisors-section-phantom]{Divisors}
\item \hyperref[limits-section-phantom]{Limits of Schemes}
\item \hyperref[varieties-section-phantom]{Varieties}
\item \hyperref[topologies-section-phantom]{Topologies on Schemes}
\item \hyperref[descent-section-phantom]{Descent}
\item \hyperref[perfect-section-phantom]{Derived Categories of Schemes}
\item \hyperref[more-morphisms-section-phantom]{More on Morphisms}
\item \hyperref[flat-section-phantom]{More on Flatness}
\item \hyperref[groupoids-section-phantom]{Groupoid Schemes}
\item \hyperref[more-groupoids-section-phantom]{More on Groupoid Schemes}
\item \hyperref[etale-section-phantom]{\'Etale Morphisms of Schemes}
\end{enumerate}
Topics in Scheme Theory
\begin{enumerate}
\setcounter{enumi}{41}
\item \hyperref[chow-section-phantom]{Chow Homology}
\item \hyperref[intersection-section-phantom]{Intersection Theory}
\item \hyperref[pic-section-phantom]{Picard Schemes of Curves}
\item \hyperref[weil-section-phantom]{Weil Cohomology Theories}
\item \hyperref[adequate-section-phantom]{Adequate Modules}
\item \hyperref[dualizing-section-phantom]{Dualizing Complexes}
\item \hyperref[duality-section-phantom]{Duality for Schemes}
\item \hyperref[discriminant-section-phantom]{Discriminants and Differents}
\item \hyperref[derham-section-phantom]{de Rham Cohomology}
\item \hyperref[local-cohomology-section-phantom]{Local Cohomology}
\item \hyperref[algebraization-section-phantom]{Algebraic and Formal Geometry}
\item \hyperref[curves-section-phantom]{Algebraic Curves}
\item \hyperref[resolve-section-phantom]{Resolution of Surfaces}
\item \hyperref[models-section-phantom]{Semistable Reduction}
\item \hyperref[functors-section-phantom]{Functors and Morphisms}
\item \hyperref[equiv-section-phantom]{Derived Categories of Varieties}
\item \hyperref[pione-section-phantom]{Fundamental Groups of Schemes}
\item \hyperref[etale-cohomology-section-phantom]{\'Etale Cohomology}
\item \hyperref[crystalline-section-phantom]{Crystalline Cohomology}
\item \hyperref[proetale-section-phantom]{Pro-\'etale Cohomology}
\item \hyperref[relative-cycles-section-phantom]{Relative Cycles}
\item \hyperref[more-etale-section-phantom]{More \'Etale Cohomology}
\item \hyperref[trace-section-phantom]{The Trace Formula}
\end{enumerate}
Algebraic Spaces
\begin{enumerate}
\setcounter{enumi}{64}
\item \hyperref[spaces-section-phantom]{Algebraic Spaces}
\item \hyperref[spaces-properties-section-phantom]{Properties of Algebraic Spaces}
\item \hyperref[spaces-morphisms-section-phantom]{Morphisms of Algebraic Spaces}
\item \hyperref[decent-spaces-section-phantom]{Decent Algebraic Spaces}
\item \hyperref[spaces-cohomology-section-phantom]{Cohomology of Algebraic Spaces}
\item \hyperref[spaces-limits-section-phantom]{Limits of Algebraic Spaces}
\item \hyperref[spaces-divisors-section-phantom]{Divisors on Algebraic Spaces}
\item \hyperref[spaces-over-fields-section-phantom]{Algebraic Spaces over Fields}
\item \hyperref[spaces-topologies-section-phantom]{Topologies on Algebraic Spaces}
\item \hyperref[spaces-descent-section-phantom]{Descent and Algebraic Spaces}
\item \hyperref[spaces-perfect-section-phantom]{Derived Categories of Spaces}
\item \hyperref[spaces-more-morphisms-section-phantom]{More on Morphisms of Spaces}
\item \hyperref[spaces-flat-section-phantom]{Flatness on Algebraic Spaces}
\item \hyperref[spaces-groupoids-section-phantom]{Groupoids in Algebraic Spaces}
\item \hyperref[spaces-more-groupoids-section-phantom]{More on Groupoids in Spaces}
\item \hyperref[bootstrap-section-phantom]{Bootstrap}
\item \hyperref[spaces-pushouts-section-phantom]{Pushouts of Algebraic Spaces}
\end{enumerate}
Topics in Geometry
\begin{enumerate}
\setcounter{enumi}{81}
\item \hyperref[spaces-chow-section-phantom]{Chow Groups of Spaces}
\item \hyperref[groupoids-quotients-section-phantom]{Quotients of Groupoids}
\item \hyperref[spaces-more-cohomology-section-phantom]{More on Cohomology of Spaces}
\item \hyperref[spaces-simplicial-section-phantom]{Simplicial Spaces}
\item \hyperref[spaces-duality-section-phantom]{Duality for Spaces}
\item \hyperref[formal-spaces-section-phantom]{Formal Algebraic Spaces}
\item \hyperref[restricted-section-phantom]{Algebraization of Formal Spaces}
\item \hyperref[spaces-resolve-section-phantom]{Resolution of Surfaces Revisited}
\end{enumerate}
Deformation Theory
\begin{enumerate}
\setcounter{enumi}{89}
\item \hyperref[formal-defos-section-phantom]{Formal Deformation Theory}
\item \hyperref[defos-section-phantom]{Deformation Theory}
\item \hyperref[cotangent-section-phantom]{The Cotangent Complex}
\item \hyperref[examples-defos-section-phantom]{Deformation Problems}
\end{enumerate}
Algebraic Stacks
\begin{enumerate}
\setcounter{enumi}{93}
\item \hyperref[algebraic-section-phantom]{Algebraic Stacks}
\item \hyperref[examples-stacks-section-phantom]{Examples of Stacks}
\item \hyperref[stacks-sheaves-section-phantom]{Sheaves on Algebraic Stacks}
\item \hyperref[criteria-section-phantom]{Criteria for Representability}
\item \hyperref[artin-section-phantom]{Artin's Axioms}
\item \hyperref[quot-section-phantom]{Quot and Hilbert Spaces}
\item \hyperref[stacks-properties-section-phantom]{Properties of Algebraic Stacks}
\item \hyperref[stacks-morphisms-section-phantom]{Morphisms of Algebraic Stacks}
\item \hyperref[stacks-limits-section-phantom]{Limits of Algebraic Stacks}
\item \hyperref[stacks-cohomology-section-phantom]{Cohomology of Algebraic Stacks}
\item \hyperref[stacks-perfect-section-phantom]{Derived Categories of Stacks}
\item \hyperref[stacks-introduction-section-phantom]{Introducing Algebraic Stacks}
\item \hyperref[stacks-more-morphisms-section-phantom]{More on Morphisms of Stacks}
\item \hyperref[stacks-geometry-section-phantom]{The Geometry of Stacks}
\end{enumerate}
Topics in Moduli Theory
\begin{enumerate}
\setcounter{enumi}{107}
\item \hyperref[moduli-section-phantom]{Moduli Stacks}
\item \hyperref[moduli-curves-section-phantom]{Moduli of Curves}
\end{enumerate}
Miscellany
\begin{enumerate}
\setcounter{enumi}{109}
\item \hyperref[examples-section-phantom]{Examples}
\item \hyperref[exercises-section-phantom]{Exercises}
\item \hyperref[guide-section-phantom]{Guide to Literature}
\item \hyperref[desirables-section-phantom]{Desirables}
\item \hyperref[coding-section-phantom]{Coding Style}
\item \hyperref[obsolete-section-phantom]{Obsolete}
\item \hyperref[fdl-section-phantom]{GNU Free Documentation License}
\item \hyperref[index-section-phantom]{Auto Generated Index}
\end{enumerate}
\end{multicols}


\bibliography{my}
\bibliographystyle{amsalpha}

\end{document}
