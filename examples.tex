\IfFileExists{stacks-project.cls}{%
\documentclass{stacks-project}
}{%
\documentclass{amsart}
}

% For dealing with references we use the comment environment
\usepackage{verbatim}
\newenvironment{reference}{\comment}{\endcomment}
%\newenvironment{reference}{}{}
\newenvironment{slogan}{\comment}{\endcomment}
\newenvironment{history}{\comment}{\endcomment}

% For commutative diagrams we use Xy-pic
\usepackage[all]{xy}

% We use 2cell for 2-commutative diagrams.
\xyoption{2cell}
\UseAllTwocells

% We use multicol for the list of chapters between chapters
\usepackage{multicol}

% This is generall recommended for better output
\usepackage[T1]{fontenc}

% For cross-file-references
\usepackage{xr-hyper}

% Package for hypertext links:
\usepackage{hyperref}

% For any local file, say "hello.tex" you want to link to please
% use \externaldocument[hello-]{hello}
\externaldocument[introduction-]{introduction}
\externaldocument[conventions-]{conventions}
\externaldocument[sets-]{sets}
\externaldocument[categories-]{categories}
\externaldocument[topology-]{topology}
\externaldocument[sheaves-]{sheaves}
\externaldocument[sites-]{sites}
\externaldocument[stacks-]{stacks}
\externaldocument[fields-]{fields}
\externaldocument[algebra-]{algebra}
\externaldocument[brauer-]{brauer}
\externaldocument[homology-]{homology}
\externaldocument[derived-]{derived}
\externaldocument[simplicial-]{simplicial}
\externaldocument[more-algebra-]{more-algebra}
\externaldocument[smoothing-]{smoothing}
\externaldocument[modules-]{modules}
\externaldocument[sites-modules-]{sites-modules}
\externaldocument[injectives-]{injectives}
\externaldocument[cohomology-]{cohomology}
\externaldocument[sites-cohomology-]{sites-cohomology}
\externaldocument[dga-]{dga}
\externaldocument[dpa-]{dpa}
\externaldocument[hypercovering-]{hypercovering}
\externaldocument[schemes-]{schemes}
\externaldocument[constructions-]{constructions}
\externaldocument[properties-]{properties}
\externaldocument[morphisms-]{morphisms}
\externaldocument[coherent-]{coherent}
\externaldocument[divisors-]{divisors}
\externaldocument[limits-]{limits}
\externaldocument[varieties-]{varieties}
\externaldocument[topologies-]{topologies}
\externaldocument[descent-]{descent}
\externaldocument[perfect-]{perfect}
\externaldocument[more-morphisms-]{more-morphisms}
\externaldocument[flat-]{flat}
\externaldocument[groupoids-]{groupoids}
\externaldocument[more-groupoids-]{more-groupoids}
\externaldocument[etale-]{etale}
\externaldocument[chow-]{chow}
\externaldocument[intersection-]{intersection}
\externaldocument[pic-]{pic}
\externaldocument[adequate-]{adequate}
\externaldocument[dualizing-]{dualizing}
\externaldocument[duality-]{duality}
\externaldocument[discriminant-]{discriminant}
\externaldocument[local-cohomology-]{local-cohomology}
\externaldocument[curves-]{curves}
\externaldocument[resolve-]{resolve}
\externaldocument[models-]{models}
\externaldocument[pione-]{pione}
\externaldocument[etale-cohomology-]{etale-cohomology}
\externaldocument[proetale-]{proetale}
\externaldocument[crystalline-]{crystalline}
\externaldocument[spaces-]{spaces}
\externaldocument[spaces-properties-]{spaces-properties}
\externaldocument[spaces-morphisms-]{spaces-morphisms}
\externaldocument[decent-spaces-]{decent-spaces}
\externaldocument[spaces-cohomology-]{spaces-cohomology}
\externaldocument[spaces-limits-]{spaces-limits}
\externaldocument[spaces-divisors-]{spaces-divisors}
\externaldocument[spaces-over-fields-]{spaces-over-fields}
\externaldocument[spaces-topologies-]{spaces-topologies}
\externaldocument[spaces-descent-]{spaces-descent}
\externaldocument[spaces-perfect-]{spaces-perfect}
\externaldocument[spaces-more-morphisms-]{spaces-more-morphisms}
\externaldocument[spaces-flat-]{spaces-flat}
\externaldocument[spaces-groupoids-]{spaces-groupoids}
\externaldocument[spaces-more-groupoids-]{spaces-more-groupoids}
\externaldocument[bootstrap-]{bootstrap}
\externaldocument[spaces-pushouts-]{spaces-pushouts}
\externaldocument[spaces-chow-]{spaces-chow}
\externaldocument[groupoids-quotients-]{groupoids-quotients}
\externaldocument[spaces-more-cohomology-]{spaces-more-cohomology}
\externaldocument[spaces-simplicial-]{spaces-simplicial}
\externaldocument[spaces-duality-]{spaces-duality}
\externaldocument[formal-spaces-]{formal-spaces}
\externaldocument[restricted-]{restricted}
\externaldocument[spaces-resolve-]{spaces-resolve}
\externaldocument[formal-defos-]{formal-defos}
\externaldocument[defos-]{defos}
\externaldocument[cotangent-]{cotangent}
\externaldocument[examples-defos-]{examples-defos}
\externaldocument[algebraic-]{algebraic}
\externaldocument[examples-stacks-]{examples-stacks}
\externaldocument[stacks-sheaves-]{stacks-sheaves}
\externaldocument[criteria-]{criteria}
\externaldocument[artin-]{artin}
\externaldocument[quot-]{quot}
\externaldocument[stacks-properties-]{stacks-properties}
\externaldocument[stacks-morphisms-]{stacks-morphisms}
\externaldocument[stacks-limits-]{stacks-limits}
\externaldocument[stacks-cohomology-]{stacks-cohomology}
\externaldocument[stacks-perfect-]{stacks-perfect}
\externaldocument[stacks-introduction-]{stacks-introduction}
\externaldocument[stacks-more-morphisms-]{stacks-more-morphisms}
\externaldocument[stacks-geometry-]{stacks-geometry}
\externaldocument[moduli-]{moduli}
\externaldocument[moduli-curves-]{moduli-curves}
\externaldocument[examples-]{examples}
\externaldocument[exercises-]{exercises}
\externaldocument[guide-]{guide}
\externaldocument[desirables-]{desirables}
\externaldocument[coding-]{coding}
\externaldocument[obsolete-]{obsolete}
\externaldocument[fdl-]{fdl}
\externaldocument[index-]{index}

% Theorem environments.
%
\theoremstyle{plain}
\newtheorem{theorem}[subsection]{Theorem}
\newtheorem{proposition}[subsection]{Proposition}
\newtheorem{lemma}[subsection]{Lemma}

\theoremstyle{definition}
\newtheorem{definition}[subsection]{Definition}
\newtheorem{example}[subsection]{Example}
\newtheorem{exercise}[subsection]{Exercise}
\newtheorem{situation}[subsection]{Situation}

\theoremstyle{remark}
\newtheorem{remark}[subsection]{Remark}
\newtheorem{remarks}[subsection]{Remarks}

\numberwithin{equation}{subsection}

% Macros
%
\def\lim{\mathop{\mathrm{lim}}\nolimits}
\def\colim{\mathop{\mathrm{colim}}\nolimits}
\def\Spec{\mathop{\mathrm{Spec}}}
\def\Hom{\mathop{\mathrm{Hom}}\nolimits}
\def\Ext{\mathop{\mathrm{Ext}}\nolimits}
\def\SheafHom{\mathop{\mathcal{H}\!\mathit{om}}\nolimits}
\def\SheafExt{\mathop{\mathcal{E}\!\mathit{xt}}\nolimits}
\def\Sch{\mathit{Sch}}
\def\Mor{\mathop{Mor}\nolimits}
\def\Ob{\mathop{\mathrm{Ob}}\nolimits}
\def\Sh{\mathop{\mathit{Sh}}\nolimits}
\def\NL{\mathop{N\!L}\nolimits}
\def\proetale{{pro\text{-}\acute{e}tale}}
\def\etale{{\acute{e}tale}}
\def\QCoh{\mathit{QCoh}}
\def\Ker{\mathop{\mathrm{Ker}}}
\def\Im{\mathop{\mathrm{Im}}}
\def\Coker{\mathop{\mathrm{Coker}}}
\def\Coim{\mathop{\mathrm{Coim}}}

%
% Macros for moduli stacks/spaces
%
\def\QCohstack{\mathcal{QC}\!\mathit{oh}}
\def\Cohstack{\mathcal{C}\!\mathit{oh}}
\def\Spacesstack{\mathcal{S}\!\mathit{paces}}
\def\Quotfunctor{\mathrm{Quot}}
\def\Hilbfunctor{\mathrm{Hilb}}
\def\Curvesstack{\mathcal{C}\!\mathit{urves}}
\def\Polarizedstack{\mathcal{P}\!\mathit{olarized}}
\def\Complexesstack{\mathcal{C}\!\mathit{omplexes}}
% \Pic is the operator that assigns to X its picard group, usage \Pic(X)
% \Picardstack_{X/B} denotes the Picard stack of X over B
% \Picardfunctor_{X/B} denotes the Picard functor of X over B
\def\Pic{\mathop{\mathrm{Pic}}\nolimits}
\def\Picardstack{\mathcal{P}\!\mathit{ic}}
\def\Picardfunctor{\mathrm{Pic}}
\def\Deformationcategory{\mathcal{D}\!\mathit{ef}}


% OK, start here.
%
\begin{document}

\title{Examples}


\maketitle

\phantomsection
\label{section-phantom}

\tableofcontents

\section{Introduction}
\label{section-introduction}

\noindent
This chapter will contain examples which illuminate the theory.





\section{An empty limit}
\label{section-empty-limit}

\noindent
This example is due to Waterhouse, see \cite{Waterhouse}.
Let $S$ be an uncountable set. For every finite subset
$T \subset S$ consider the set $M_T$ of injective maps $T \to \mathbf{N}$.
For $T \subset T' \subset S$ finite the restriction $M_{T'} \to M_T$
is surjective. Thus we have an inverse system over the
directed partially ordered set of finite subsets of $S$
with surjective transition maps.
But $\lim M_T = \emptyset$ as an element in the limit would
define an injective map $S \to \mathbf{N}$.



\section{A zero limit}
\label{section-zero-limit}

\noindent
Let $(S_i)_{i \in I}$ be a directed inverse system of nonempty sets
with surjective transition maps and
with $\lim S_i = \emptyset$, see Section \ref{section-empty-limit}.
Let $K$ be a field and set
$$
V_i = \bigoplus\nolimits_{s \in S_i} K
$$
Then the transition maps $V_i \to V_j$ are surjective for $i \geq j$.
However, $\lim V_i = 0$. Namely, if $v = (v_i)$ is an element of the
limit, then the support of $v_i$ would be a finite subset $T_i \subset S_i$
with $\lim T_i \not = \emptyset$, see
Categories, Lemma \ref{categories-lemma-nonempty-limit}.

\medskip\noindent
For each $i$ consider the unique $K$-linear map $V_i \to K$ which sends
each basis vector $s \in S_i$ to $1$. Let $W_i \subset V_i$
be the kernel. Then
$$
0 \to (W_i) \to (V_i) \to (K) \to 0
$$
is a nonsplit short exact sequence of inverse systems of vector spaces
over the directed set $I$. Hence
$W_i$ is a directed system of $K$-vector spaces
with surjective transition maps, vanishing limit, and nonvanishing
$R^1\lim$.



\section{Non-quasi-compact inverse limit of quasi-compact spaces}
\label{section-lim-not-quasi-compact}

\noindent
Let $\mathbf{N}$ denote the set of natural numbers.
For every integer $n$, let $I_n$ denote the set of all natural numbers $> n$.
Define $T_n$ to be the unique topology on $\mathbf{N}$ with basis
$\{1\}, \ldots , \{n\}, I_n$.  Denote by $X_n$ the topological space
$(\mathbf{N}, T_n)$.  For each $m < n$, the identity map,
$$
f_{n, m} : X_n \longrightarrow X_m
$$
is continuous.  Obviously for $m < n < p$, the composition
$f_{p, n} \circ f_{n, m}$ equals $f_{p, m}$.  So $((X_n), (f_{n,m}))$
is a directed inverse system of quasi-compact topological spaces.

\medskip\noindent
Let $T$ be the discrete topology on $\mathbf{N}$, and let $X$ be
$(\mathbf{N}, T)$. Then for every integer $n$, the identity map,
$$
f_n : X \longrightarrow X_n
$$
is continuous. We claim that this is the inverse limit of the directed
system above. Let $(Y, S)$ be any topological space. For every integer $n$,
let
$$
g_n : (Y, S) \longrightarrow (\mathbf{N}, T_n)
$$
be a continuous map. Assume that for every $m < n$ we have
$f_{n,m} \circ g_n = g_m$, i.e., the system $(g_n)$ is compatible
with the directed system above. In particular, all of the set maps
$g_n$ are equal to a common set map
$$
g : Y \longrightarrow \mathbf{N}.
$$
Moreover, for every integer $n$, since $\{n\}$ is open in $X_n$,
also $g^{-1}(\{n\}) = g_n^{-1}(\{n\})$ is open in $Y$.
Therefore the set map $g$ is continuous for the topology $S$ on $Y$
and the topology $T$ on $\mathbf{N}$. Thus $(X, (f_n))$ is the inverse
limit of the directed system above.

\medskip\noindent
However, clearly $X$ is not quasi-compact, since the infinite open
covering by singleton sets has no inverse limit.

\begin{lemma}
\label{lemma-lim-not-quasi-compact}
There exists an inverse system of quasi-compact topological spaces
over $\mathbf{N}$ whose limit is not quasi-compact.
\end{lemma}

\begin{proof}
See discussion above.
\end{proof}






\section{The structure sheaf on the fibre product}
\label{section-silly}

\noindent
Let $X, Y, S, a, b, p, q, f$ be as in the introduction to
Derived Categories of Schemes,
Section \ref{perfect-section-kunneth}. Picture:
$$
\xymatrix{
& X \times_S Y \ar[ld]^p \ar[rd]_q \ar[dd]^f \\
X \ar[rd]_a & & Y \ar[ld]^b \\
& S
}
$$
Then we have a canonical map
$$
can :
p^{-1}\mathcal{O}_X \otimes_{f^{-1}\mathcal{O}_S} q^{-1}\mathcal{O}_Y
\longrightarrow
\mathcal{O}_{X \times_S Y}
$$
which is not an isomorphism in general.

\medskip\noindent
For example, let $S = \Spec(\mathbf{R})$, $X = \Spec(\mathbf{C})$, and
$Y = \Spec(\mathbf{C})$. Then
$X \times_S Y = \Spec(\mathbf{C}) \amalg \Spec(\mathbf{C})$
is a discrete space with two points
and the sheaves $p^{-1}\mathcal{O}_X$, $q^{-1}\mathcal{O}_Y$
and $f^{-1}\mathcal{O}_S$ are the constant sheaves with values
$\mathbf{C}$, $\mathbf{C}$, and $\mathbf{R}$.
Hence the source of $can$ is the constant sheaf
with value $\mathbf{C} \otimes_\mathbf{R} \mathbf{C}$
on the discrete space with two points. Thus its global
sections have dimension $8$ as an $\mathbf{R}$-vector space
whereas taking global sections of the target of $can$
we obtain $\mathbf{C} \times \mathbf{C}$
which has dimension $4$ as an $\mathbf{R}$-vector space.

\medskip\noindent
Another example is the following. Let $k$ be an algebraically
closed field. Consider
$S = \Spec(k)$, $X = \mathbf{A}^1_k$, and $Y = \mathbf{A}^1_k$.
Then for $U \subset X \times_S Y = \mathbf{A}^2_k$ nonempty open
the images $p(U) \subset X = \mathbf{A}^1_k$ and $q(U) \subset \mathbf{A}^1_k$
are open and the reader can show that
$$
\left(
p^{-1}\mathcal{O}_X \otimes_{f^{-1}\mathcal{O}_S} q^{-1}\mathcal{O}_Y
\right)(U) = \mathcal{O}_X(p(U)) \otimes_k \mathcal{O}_Y(q(U))
$$
This is not equal to $\mathcal{O}_{X \times_S Y}(U)$
if $U$ is the complement of an irreducible curve $C$ in
$X \times_S Y = \mathbf{A}^2_k$ such that both $p|_C$ and $q|_C$
are nonconstant.

\medskip\noindent
Returning to the general case, let $z = (x, y, s, \mathfrak p)$
be a point of $X \times_S Y$ as in
Schemes, Lemma \ref{schemes-lemma-points-fibre-product}.
Then on stalks at $z$ the map $can$ gives the map
$$
can_z :
\mathcal{O}_{X, x} \otimes_{\mathcal{O}_{S, s}} \mathcal{O}_{Y, y}
\longrightarrow
\mathcal{O}_{X \times_S Y, z}
$$
This is a flat ring homomorphism as the target is a localization of the
source (details omitted; hint reduce to the case that $X$, $Y$, and $S$
are affine).
Observe that the source is in general not a local ring, and this gives
another way to see that $can$ is not an isomorphism in general.

\medskip\noindent
More generally, suppose we have an $\mathcal{O}_X$-module $\mathcal{F}$ and an
$\mathcal{O}_Y$-module $\mathcal{G}$. Then there is a canonical map
\begin{align*}
& p^{-1}\mathcal{F} \otimes_{f^{-1}\mathcal{O}_S} q^{-1}\mathcal{G} \\
& =
p^{-1}(\mathcal{F} \otimes_{\mathcal{O}_X} \mathcal{O}_X)
\otimes_{f^{-1}\mathcal{O}_S}
q^{-1}(\mathcal{O}_Y \otimes_{\mathcal{O}_Y} \mathcal{G}) \\
& =
p^{-1}\mathcal{F} \otimes_{p^{-1}\mathcal{O}_X} p^{-1}\mathcal{O}_X
\otimes_{f^{-1}\mathcal{O}_S}
q^{-1}\mathcal{O}_Y \otimes_{q^{-1}\mathcal{O}_Y} q^{-1}\mathcal{G} \\
& \xrightarrow{can}
p^{-1}\mathcal{F} \otimes_{q^{-1}\mathcal{O}_X}
\mathcal{O}_{X \times_S Y}
\otimes_{q^{-1}\mathcal{O}_Y} q^{-1}\mathcal{G}  \\
& =
p^{-1}\mathcal{F} \otimes_{q^{-1}\mathcal{O}_X}
\mathcal{O}_{X \times_S Y}
\otimes_{\mathcal{O}_{X \times_S Y}}
\mathcal{O}_{X \times_S Y}
\otimes_{q^{-1}\mathcal{O}_Y} q^{-1}\mathcal{G}  \\
& =
p^*\mathcal{F} \otimes_{\mathcal{O}_{X \times_S Y}} q^*\mathcal{G}
\end{align*}
which is rarely an isomorphism.





\section{A nonintegral connected scheme whose local rings are domains}
\label{section-connected-locally-integral-not-integral}

\noindent
We give an example of an affine scheme $X = \Spec(A)$ which is
connected, all of whose local rings are domains, but which is not integral.
Connectedness of $X$ means $A$ has no nontrivial idempotents, see
Algebra, Lemma \ref{algebra-lemma-disjoint-decomposition}.
The local rings of $X$ are domains if, whenever $fg = 0$ in $A$, every
point of $X$ has a neighborhood where either $f$ or $g$ vanishes.
As long as $A$ is not a domain, then $X$ is not integral
(Properties, Definition \ref{properties-definition-integral}).

\medskip\noindent
Roughly speaking, the construction is as follows: let $X_0$ be the cross
(the union of coordinate axes) on the affine plane. Then let $X_1$ be
the (reduced) full preimage of $X_0$ on the blowup of the plane ($X_1$
has three rational components forming a chain).  Then blow up the
resulting surface at the two singularities of $X_1$, and let $X_2$ be
the reduced preimage of $X_1$ (which has five rational components), etc.
Take $X$ to be the inverse limit. The only problem with this construction
is that blowups glue in a projective line, so $X_1$ is not affine. Let us
correct this by glueing in an affine line instead (so our scheme will be an
open subset in what was described above).

\medskip\noindent
Here is a completely algebraic construction: For every $k \ge 0$, let $A_k$
be the following ring: its elements are collections of
polynomials $p_i \in \mathbf{C}[x]$ where $i = 0, \ldots, 2^k$ such that
$p_i(1) = p_{i + 1}(0)$. Set $X_k = \Spec(A_k)$. Observe that $X_k$ is
a union of $2^k + 1$ affine lines that meet transversally in a chain.
Define a ring homomorphism $A_k \to A_{k + 1}$ by
$$
(p_0, \ldots, p_{2^k})
\longmapsto
(p_0, p_0(1), p_1, p_1(1), \ldots, p_{2^k}),
$$
in other words, every other polynomial is constant. This identifies
$A_k$ with a subring of $A_{k + 1}$. Let $A$ be the direct limit of $A_k$
(basically, their union). Set $X = \Spec(A)$. For every $k$, we have
a natural embedding $A_k \to A$, that is, a map $X\to X_k$.
Each $A_k$ is connected but not integral; this implies that $A$ is
connected but not integral. It remains to show that the local rings of
$A$ are domains.

\medskip\noindent
Take $f, g \in A$ with $fg = 0$ and $x \in X$. Let us construct a
neighborhood of $x$ on which one of $f$ and $g$ vanishes. Choose $k$
such that $f, g \in A_{k - 1}$ (note the $k - 1$ index).
Let $y$ be the image of $x$ in $X_k$. It suffices to prove that $y$ has
a neighborhood on which either $f$ or $g$ viewed as sections of
$\mathcal{O}_{X_k}$ vanishes.
If $y$ is a smooth point of $X_k$, that is, it lies on only one of the
$2^k + 1$ lines, this is obvious. We can therefore assume that $y$ is one
of the $2^k$ singular points, so two components of $X_k$ pass through
$y$. However, on one of these two components (the one with odd index),
both $f$ and $g$ are constant, since they are pullbacks of functions on
$X_{k - 1}$. Since $fg = 0$ everywhere, either $f$ or $g$ (say, $f$)
vanishes on the other component.
This implies that $f$ vanishes on both components, as required.






\section{Noncomplete completion}
\label{section-noncomplete-completion}

\noindent
Let $R$ be a ring and let $\mathfrak m$ be a maximal ideal. Consider the
completion
$$
R^\wedge = \lim R/\mathfrak m^n.
$$
Note that $R^\wedge$ is a local ring with maximal ideal
$\mathfrak m' = \Ker(R^\wedge \to R/\mathfrak m)$.
Namely, if $x = (x_n) \in R^\wedge$ is not in $\mathfrak m'$, then
$y = (x_n^{-1}) \in R^\wedge$ satisfies $xy = 1$, whence $R^\wedge$ is local by
Algebra, Lemma \ref{algebra-lemma-characterize-local-ring}. Now it is
always true that $R^\wedge$ complete in its limit topology (see the
discussion in
More on Algebra, Section \ref{more-algebra-section-topological-ring}).
But beyond that, we have the following questions:
\begin{enumerate}
\item Is it true that $\mathfrak m R^\wedge = \mathfrak m'$?
\item Is $R^\wedge$ viewed as an $R^\wedge$-module $\mathfrak m'$-adically
complete?
\item Is $R^\wedge$ viewed as an $R$-module $\mathfrak m$-adically complete?
\end{enumerate}
It turns out that these questions all have a negative answer.
The example below was taken from an unpublished note of
Bart de Smit and Hendrik Lenstra. See also
\cite[Exercise III.2.12]{Bourbaki-CA} and
\cite[Example 1.8]{Yekutieli}

\medskip\noindent
Let $k$ be a field, $R = k[x_1, x_2, x_3, \ldots]$, and
$\mathfrak m = (x_1, x_2, x_3, \ldots)$.
We will think of an element $f$ of $R^\wedge$ as a (possibly) infinite sum
$$
f = \sum a_I x^I
$$
(using multi-index notation) such that for each $d \geq 0$ there
are only finitely many nonzero $a_I$ for $|I| = d$. The maximal
ideal $\mathfrak m' \subset R^\wedge$ is the collection of $f$ with
zero constant term. In particular, the element
$$
f = x_1 + x_2^2 + x_3^3 + \ldots
$$
is in $\mathfrak m'$ but not in $\mathfrak m R^\wedge$ which
shows that (1) is false in this example. However, if (1) is
false, then (3) is necessarily false because
$\mathfrak m' = \Ker(R^\wedge \to R/\mathfrak m)$
and we can apply
Algebra, Lemma \ref{algebra-lemma-hathat} with $n = 1$.

\medskip\noindent
To finish we prove that $R^\wedge$ is not $\mathfrak m'$-adically complete.
For $n \geq 1$ let $K_n = \Ker(R^\wedge \to R/\mathfrak m^n)$. Then
we have short exact sequences
$$
0 \to K_n/(\mathfrak m')^n \to R^\wedge/(\mathfrak m')^n \to
R/\mathfrak m^n \to 0
$$
The projection map $R^\wedge \to R/\mathfrak m^{n + 1}$ sends
$(\mathfrak m')^n$ onto $\mathfrak m^n/\mathfrak m^{n + 1}$.
It follows that $K_{n + 1} \to K_n/(\mathfrak m')^n$ is surjective.
Hence the inverse system $\left(K_n/(\mathfrak m')^n\right)$
has surjective transition maps and
taking inverse limits we obtain an exact sequence
$$
0 \to \lim K_n/(\mathfrak m')^n \to
\lim R^\wedge/(\mathfrak m')^n \to
\lim R/\mathfrak m^n \to 0
$$
by Algebra, Lemma \ref{algebra-lemma-Mittag-Leffler}.
Thus we see that $R^\wedge$ is complete with respect to $\mathfrak m'$
if and only if $K_n = (\mathfrak m')^n$ for all $n \geq 1$.

\medskip\noindent
To show that $R^\wedge$ is not $\mathfrak m'$-adically complete
in our example we show that $K_2 = \Ker(R^\wedge \to R/\mathfrak m^2)$
is not equal to $(\mathfrak m')^2$.
Note that an element of $(\mathfrak m')^2$
can be written as a finite sum
\begin{equation}
\label{equation-sum}
\sum\nolimits_{i = 1, \ldots, t} f_i g_i
\end{equation}
with $f_i, g_i \in R^\wedge$ having vanishing constant terms.
To get an example we are going to choose an $z \in K_2$
of the form
$$
z = z_1 + z_2 + z_3 + \ldots
$$
with the following properties
\begin{enumerate}
\item there exist sequences $1 < d_1 < d_2 < d_3 < \ldots $ and
$0 < n_1 < n_2 < n_3 < \ldots$ such that
$z_i \in k[x_{n_i}, x_{n_i + 1}, \ldots, x_{n_{i + 1} - 1}]$
homogeneous of degree $d_i$, and
\item in the ring $k[[x_{n_i}, x_{n_i + 1}, \ldots, x_{n_{i + 1} - 1}]]$
the element $z_i$ cannot be written as a sum (\ref{equation-sum})
with $t \leq i$.
\end{enumerate}
Clearly this implies that $z$ is not in $(\mathfrak m')^2$
because the image of the relation (\ref{equation-sum}) in the
ring $k[[x_{n_i}, x_{n_i + 1}, \ldots, x_{n_{i + 1} - 1}]]$
for $i$ large enough would produce a contradiction. Hence it suffices
to prove that for all $t > 0$ there exists a $d \gg 0$ and an integer
$n$ such that we can find an homogeneous element
$z \in k[x_1, \ldots, x_n]$ of degree $d$ which cannot be written as
a sum (\ref{equation-sum}) for the given $t$ in $k[[x_1, \ldots, x_n]]$.
Take $n > 2t$ and any $d > 1$ prime to the characteristic of $k$ and
set $z = \sum_{i = 1, \ldots, n} x_i^d$. Then the vanishing locus
of the ideal
$$
(\frac{\partial z}{\partial x_1}, \ldots, \frac{\partial z}{\partial x_n})
=
(dx_1^{d - 1}, \ldots, dx_n^{d - 1})
$$
consists of one point. On the other hand,
$$
\frac{\partial ( \sum\nolimits_{i = 1, \ldots, t} f_i g_i ) }{\partial x_j}
\in (f_1, \ldots, f_t, g_1, \ldots, g_t)
$$
by the Leibniz rule and hence the vanishing locus of these derivatives
contains at least
$$
V(f_1, \ldots, f_t, g_1, \ldots, g_t) \subset
\Spec(k[[x_1, \ldots, x_n]]).
$$
Hence this is a contradiction as the dimension of
$V(f_1, \ldots, f_t, g_1, \ldots, g_t)$ is at least $n - 2t \geq 1$.

\begin{lemma}
\label{lemma-noncomplete-completion}
There exists a local ring $R$ and a maximal ideal $\mathfrak m$ such that
the completion $R^\wedge$ of $R$ with respect to $\mathfrak m$ has the
following properties
\begin{enumerate}
\item $R^\wedge$ is local, but its maximal ideal is not equal to
$\mathfrak m R^\wedge$,
\item $R^\wedge$ is not a complete local ring, and
\item $R^\wedge$ is not $\mathfrak m$-adically complete as an $R$-module.
\end{enumerate}
\end{lemma}

\begin{proof}
This follows from the discussion above as (with $R = k[x_1, x_2, x_3, \ldots]$)
the completion of the localization $R_{\mathfrak m}$ is equal to the
completion of $R$.
\end{proof}





\section{Noncomplete quotient}
\label{section-noncomplete-quotient}

\noindent
Let $k$ be a field. Let
$$
R = k[t, z_1, z_2, z_3, \ldots, w_1, w_2, w_3, \ldots, x]/
(z_it - x^iw_i, z_i w_j)
$$
Note that in particular $z_iz_jt = 0$ in this ring. Any element $f$ of $R$
can be uniquely written as a finite sum
$$
f = \sum\nolimits_{i = 0, \ldots, d} f_i x^i
$$
where each $f_i \in k[t, z_i, w_j]$ has no terms involving the products
$z_it$ or $z_iw_j$. Moreover, if $f$ is written in this way, then
$f \in (x^n)$ if and only if $f_i = 0$ for $i < n$.
So $x$ is a nonzerodivisor and $\bigcap (x^n) = 0$.
Let $R^\wedge$ be the completion of $R$ with respect to the ideal $(x)$.
Note that $R^\wedge$ is $(x)$-adically complete, see
Algebra, Lemma \ref{algebra-lemma-hathat-finitely-generated}.
By the above we see that an element of $R^\wedge$ can be uniquely written
as an infinite sum
$$
f = \sum\nolimits_{i = 0}^\infty f_i x^i
$$
where each $f_i \in k[t, z_i, w_j]$ has no terms involving the products
$z_it$ or $z_iw_j$. Consider the element
$$
f = \sum\nolimits_{i = 1}^\infty x^i w_i =
xw_1 + x^2w_2 + x^3w_3 + \ldots
$$
i.e., we have $f_n = w_n$. Note that $f \in (t , x^n)$ for every $n$
because $x^mw_m \in (t)$ for all $m$.
We claim that $f \not \in (t)$. To prove this assume that
$tg = f$ where $g = \sum g_lx^l$ in canonical form as above.
Since $tz_iz_j = 0$ we may as well assume that none of the $g_l$ have
terms involving the products $z_iz_j$. Examining the process to
get $tg$ in canonical form we see the following:
Given any term $c m$ of $g_l$ where $c \in k$ and $m$ is a
monomial in $t, z_i, w_j$ and we make the following replacement
\begin{enumerate}
\item if the monomial $m$ does not involve any $z_i$, then $ctm$ is
a term of $f_l$, and
\item if the monomial $m$ does involve a $z_i$ then it is equal to
$m = z_i$ and we see that $cw_i$ is term of $f_{l + i}$.
\end{enumerate}
Since $g_0$ is a polynomial only finitely many of the variables $z_i$
occur in it. Pick $n$ such that $z_n$ does not occur in $g_0$.
Then the rules above show that $w_n$ does not occur in $f_n$ which is
a contradiction. It follows that $R^\wedge/(t)$ is not complete, see
Algebra, Lemma \ref{algebra-lemma-quotient-complete}.

\begin{lemma}
\label{lemma-noncomplete-quotient}
There exists a ring $R$ complete with respect to a principal ideal
$I$ and a principal ideal $J$ such that $R/J$ is not $I$-adically
complete.
\end{lemma}

\begin{proof}
See discussion above.
\end{proof}



\section{Completion is not exact}
\label{section-completion-not-exact}

\noindent
A quick example is the following. Suppose that $R = k[t]$. Let
$P = K = \bigoplus_{n \in \mathbf{N}} R$ and
$M = \bigoplus_{n \in \mathbf{N}} R/(t^n)$. Then there is a short exact
sequence $0 \to K \to P \to M \to 0$ where the first map is given by
multiplication by $t^n$ on the $n$th summand. We claim that
$0 \to K^\wedge \to P^\wedge \to M^\wedge \to 0$ is not exact in the middle.
Namely, $\xi = (t^2, t^3, t^4, \ldots) \in P^\wedge$ maps to zero in
$M^\wedge$ but is not in the image of $K^\wedge \to P^\wedge$, because
it would be the image of $(t, t, t, \ldots)$ which is not an element of
$K^\wedge$.

\medskip\noindent
A ``smaller'' example is the following. In the situation of
Lemma \ref{lemma-noncomplete-quotient}
the short exact sequence $0 \to J \to R \to R/J \to 0$ does not remain
exact after completion. Namely, if $f \in J$ is a generator, then
$f : R \to J$ is surjective, hence $R \to J^\wedge$ is surjective, hence
the image of $J^\wedge \to R$ is $(f) = J$ but the fact that
$R/J$ is noncomplete means that the kernel of the surjection
$R \to (R/J)^\wedge$ is strictly bigger than $J$, see
Algebra, Lemmas \ref{algebra-lemma-completion-generalities} and
\ref{algebra-lemma-quotient-complete}.
By the same token the sequence
$R \to R \to R/(f) \to 0$ does not remain exact on completion.

\begin{lemma}
\label{lemma-completion-not-exact}
\begin{slogan}
Completion is neither left nor right exact in general.
\end{slogan}
Completion is not an exact functor in general; it is not even
right exact in general. This holds even when $I$ is finitely
generated on the category of finitely presented modules.
\end{lemma}

\begin{proof}
See discussion above.
\end{proof}



\section{The category of complete modules is not abelian}
\label{section-non-abelian}


\noindent
Let $R$ be a ring and let $I \subset R$ be a finitely generated ideal.
Consider the category $\mathcal{A}$ of $I$-adically complete
$R$-modules, see
Algebra, Definition \ref{algebra-definition-complete}.
Let $\varphi : M \to N$ be a morphism of $\mathcal{A}$.
The cokernel of $\varphi$ in $\mathcal{A}$ is the completion
$(\Coker(\varphi))^\wedge$ of the usual cokernel
(as $I$ is finitely generated this completion is complete, see
Algebra, Lemma \ref{algebra-lemma-hathat-finitely-generated}).
Let $K = \Ker(\varphi)$. We claim that $K$ is complete and
hence is the kernel of $\varphi$ in $\mathcal{A}$. Namely, let
$K^\wedge$ be the completion. As $M$ is complete we obtain a factorization
$$
K \to K^\wedge \to M \xrightarrow{\varphi} N
$$
Since $\varphi$ is continuous for the $I$-adic topology, $K \to K^\wedge$
has dense image, and $K = \Ker(\varphi)$ we conclude that $K^\wedge$
maps into $K$. Thus $K^\wedge = K \oplus C$ and $K$ is a direct summand
of a complete module, hence complete.

\medskip\noindent
We will give an example that shows that $\Im \not = \Coim$
in general. We take $R = \mathbf{Z}_p = \lim_n \mathbf{Z}/p^n\mathbf{Z}$
to be the ring of $p$-adic integers and we take $I = (p)$.
Consider the map
$$
\text{diag}(1, p, p^2, \ldots) :
\left(\bigoplus\nolimits_{n \geq 1} \mathbf{Z}_p\right)^\wedge
\longrightarrow
\prod\nolimits_{n \geq 1} \mathbf{Z}_p
$$
where the left hand side is the $p$-adic completion of the direct sum.
Hence an element of the left hand side is a vector $(x_1, x_2, x_3, \ldots)$
with $x_i \in \mathbf{Z}_p$ with $p$-adic valuation $v_p(x_i) \to \infty$ as
$i \to \infty$. This maps to $(x_1, px_2, p^2x_3, \ldots)$. Hence we see
that $(1, p, p^2, \ldots)$ is in the closure of the image but not in
the image. By our description of kernels and cokernels above it is
clear that $\Im \not = \Coim$ for this map.

\begin{lemma}
\label{lemma-complete-modules-not-abelian}
Let $R$ be a ring and let $I \subset R$ be a finitely generated ideal.
The category of $I$-adically complete $R$-modules has kernels and
cokernels but is not abelian in general.
\end{lemma}

\begin{proof}
See above.
\end{proof}





\section{The category of derived complete modules}
\label{section-derived-complete-modules}

\noindent
Please read More on Algebra, Section
\ref{more-algebra-section-derived-complete-modules}
before reading this section.

\medskip\noindent
Let $A$ be a ring, let $I$ be an ideal of $A$, and denote $\mathcal{C}$
the category of derived complete modules as defined in
More on Algebra, Definition \ref{more-algebra-definition-derived-complete}.

\medskip\noindent
Let $T$ be a set and let $M_t$, $t \in T$ be a family of derived complete
modules. We claim that in general $\bigoplus M_t$ is not a derived complete
module. For a specific example, let $A = \mathbf{Z}_p$ and $I = (p)$ and
consider $\bigoplus_{n \in \mathbf{N}} \mathbf{Z}_p$. The map from
$\bigoplus_{n \in \mathbf{N}} \mathbf{Z}_p$ to its $p$-adic completion
isn't surjective. This means that $\bigoplus_{n \in \mathbf{N}} \mathbf{Z}_p$
cannot be derived complete as this would imply otherwise, see
More on Algebra, Lemma \ref{more-algebra-lemma-complete-derived-complete}.
Hence the inclusion functor $\mathcal{C} \to \text{Mod}_A$ does not commute
with either direct sums or (filtered) colimits.

\medskip\noindent
Assume $I$ is finitely generated. By the discussion in More on Algebra, Section
\ref{more-algebra-section-derived-complete-modules} the category
$\mathcal{C}$ has arbitrary colimits. However, we claim that filtered colimits
are not exact in the category $\mathcal{C}$. Namely, suppose that
$A = \mathbf{Z}_p$ and $I = (p)$. One has inclusions
$f_n : \mathbf{Z}_p/p\mathbf{Z}_p \to \mathbf{Z}_p/p^n\mathbf{Z}_p$
of $p$-adically complete $A$-modules given by multiplication by
$p^{n - 1}$. There are commutative diagrams
$$
\xymatrix{
\mathbf{Z}_p/p\mathbf{Z}_p \ar[r]_{f_n} \ar[d]^1 &
\mathbf{Z}_p/p^n\mathbf{Z}_p \ar[d]_p \\
\mathbf{Z}_p/p\mathbf{Z}_p \ar[r]^{f_{n + 1}} &
\mathbf{Z}_p/p^{n + 1}\mathbf{Z}_p
}
$$
We claim: the colimit of these inclusions in the category $\mathcal{C}$
gives the map $\mathbf{Z}_p/p\mathbf{Z}_p \to 0$. Namely, the colimit in
$\text{Mod}_A$ of the system on the right is $\mathbf{Q}_p/\mathbf{Z}_p$.
Thus the colimit in $\mathcal{C}$ is
$$
H^0((\mathbf{Q}_p/\mathbf{Z}_p)^\wedge) =
H^0(\mathbf{Z}_p[1]) = 0
$$
by More on Algebra, Section
\ref{more-algebra-section-derived-complete-modules}
where ${}^\wedge$ is derived completion.
This proves our claim.

\begin{lemma}
\label{lemma-derived-complete-modules}
Let $A$ be a ring and let $I \subset A$ be an ideal.
The category $\mathcal{C}$ of derived complete modules
is abelian and the inclusion functor $F : \mathcal{C} \to \text{Mod}_A$
is exact and commutes with arbitrary limits.
If $I$ is finitely generated, then $\mathcal{C}$ has
arbitrary direct sums and colimits, but $F$ does not commute with these
in general. Finally, filtered colimits are not exact in $\mathcal{C}$
in general, hence $\mathcal{C}$ is not a Grothendieck abelian category.
\end{lemma}

\begin{proof}
See More on Algebra, Lemma \ref{more-algebra-lemma-derived-complete-modules}
and discussion above.
\end{proof}





\section{Nonflat completions}
\label{section-nonflat}

\noindent
The completion of a ring with respect to an ideal isn't always flat,
contrary to the Noetherian case. We have seen two examples of this
phenomenon in
More on Algebra, Example \ref{more-algebra-example-not-glueing-pair}.
In this section we give two more examples.

\begin{lemma}
\label{lemma-countable-fg-tensor}
Let $R$ be a ring. Let $M$ be an $R$-module which is countable.
Then $M$ is a finite $R$-module if and only if
$M \otimes_R R^\mathbf{N} \to M^\mathbf{N}$ is surjective.
\end{lemma}

\begin{proof}
If $M$ is a finite module, then the map is surjective by Algebra, Proposition
\ref{algebra-proposition-fg-tensor}. Conversely, assume the map is surjective.
Let $m_1, m_2, m_3, \ldots$ be an enumeration of the elements of $M$.
Let $\sum_{j = 1, \ldots, m} x_j \otimes a_j$ be an element of the
tensor product mapping to the element $(m_n) \in M^\mathbf{N}$. Then
we see that $x_1, \ldots, x_m$ generate $M$ over $R$ as in the proof of
Algebra, Proposition \ref{algebra-proposition-fg-tensor}.
\end{proof}

\begin{lemma}
\label{lemma-countable-fp-tensor}
Let $R$ be a countable ring. Let $M$ be a countable $R$-module. Then $M$
is finitely presented if and only if the canonical map
$M \otimes_R R^\mathbf{N} \to M^\mathbf{N}$ is an isomorphism.
\end{lemma}

\begin{proof}
If $M$ is a finitely presented module, then the map is an isomorphism
by Algebra, Proposition \ref{algebra-proposition-fp-tensor}. Conversely,
assume the map is an isomorphism. By Lemma \ref{lemma-countable-fg-tensor}
the module $M$ is finite. Choose a surjection $R^{\oplus m} \to M$ with
kernel $K$. Then $K$ is countable as a submodule of $R^{\oplus m}$.
Arguing as in the proof of Algebra, Proposition
\ref{algebra-proposition-fp-tensor} we see that
$K \otimes_R R^\mathbf{N} \to K^\mathbf{N}$ is surjective.
Hence we conclude that $K$ is a finite $R$-module by
Lemma \ref{lemma-countable-fg-tensor}.
Thus $M$ is finitely presented.
\end{proof}

\begin{lemma}
\label{lemma-countable-coherent}
Let $R$ be a countable ring. Then $R$ is coherent if and only if
$R^\mathbf{N}$ is a flat $R$-module.
\end{lemma}

\begin{proof}
If $R$ is coherent, then $R^\mathbf{N}$ is a flat module by
Algebra, Proposition \ref{algebra-proposition-characterize-coherent}.
Assume $R^\mathbf{N}$ is flat. Let $I \subset R$ be a finitely
generated ideal. To prove the lemma we show that $I$ is finitely
presented as an $R$-module. Namely, the map
$I \otimes_R R^\mathbf{N} \to R^\mathbf{N}$ is
injective as $R^\mathbf{N}$ is flat and its image is
$I^\mathbf{N}$ by Lemma \ref{lemma-countable-fg-tensor}.
Thus we conclude by Lemma \ref{lemma-countable-fp-tensor}.
\end{proof}

\noindent
Let $R$ be a countable ring. Observe that $R[[x]]$ is isomorphic to
$R^\mathbf{N}$ as an $R$-module. By Lemma \ref{lemma-countable-coherent}
we see that $R \to R[[x]]$ is flat if and only if $R$ is coherent.
There are plenty of noncoherent countable rings, for example
$$
R = k[y, z, a_1, b_1, a_2, b_2, a_3, b_3, \ldots]/
(a_1 y + b_1 z, a_2 y + b_2 z, a_3 y + b_3 z, \ldots)
$$
where $k$ is a countable field. This ring is not coherent because
the ideal $(y, z)$ of $R$ is not a finitely presented $R$-module.
Note that $R[[x]]$ is the completion of $R[x]$ by the principal
ideal $(x)$.

\begin{lemma}
\label{lemma-completion-polynomial-ring-not-flat}
There exists a ring such that the completion $R[[x]]$ of $R[x]$
at $(x)$ is not flat over $R$ and a fortiori not flat over $R[x]$.
\end{lemma}

\begin{proof}
See discussion above.
\end{proof}

\noindent
It turns out there is a ring $R$ such that $R[[x]]$ is flat
over $R$, but $R[[x]]$ is not flat over $R[x]$. See
\href{https://math.stackexchange.com/users/164860/badam-baplan}{this post}
by Badam Baplan. Namely, let
$R$ be a valuation ring. Then $R$ is coherent (Algebra, Example
\ref{algebra-example-valuation-ring-coherent}) and hence
$R[[x]]$ is flat over $R$ by 
Algebra, Proposition \ref{algebra-proposition-characterize-coherent}.
On the other hand, we have the following lemma.

\begin{lemma}
\label{lemma-almost-integral-when-powerseries-flat}
Let $R$ be a domain with fraction field $K$.
If $R[[x]]$ is flat over $R[x]$, then $R$ is normal if and only
if $R$ is completely normal
(Algebra, Definition \ref{algebra-definition-almost-integral}).
\end{lemma}

\begin{proof}
Suppose we have $\alpha \in K$ and a nonzero $r \in R$ such that
$r \alpha^n \in R$ for all $n \geq 1$. Then we consider
$f = \sum r \alpha^{n - 1} x^n$ in $R[[x]]$. Write $\alpha = a/b$
for $a, b \in R$ with $b$ nonzero. Then we see that $(a x - b)f = -rb$.
It follows that $rb$ is in the ideal $(ax - b)R[[x]]$.
Let $S = \{h \in R[x] : h(0) = 1\}$. This is a multiplicative subset
and flatness of $R[x] \to R[[x]]$ implies that $S^{-1}R[x] \to R[[x]]$
is faithfully flat (details omitted; hint: use Algebra, Lemma
\ref{algebra-lemma-ff-rings}). Hence
$$
S^{-1}R/(ax - b)S^{-1}R \to R[[x]]/(ax - b)R[[x]]
$$
is injective. We conclude that
$h rb = (ax - b) g$ for some $h \in S$ and $g \in R[x]$.
Writing $h = 1 + h_1 x + \ldots + h_d x^d$ shows that we obtain
$$
1 + h_1 x + \ldots + h_d x^d = (1/r)(\alpha x - 1)g
$$
This factorization in $K[x]$ gives a corresponding factorization
in $K[x^{-1}]$ which shows that $\alpha$ is the root of a monic
polynomial with coefficients in $R$ as desired.
\end{proof}

\begin{lemma}
\label{lemma-completion-polynomial-ring-not-flat-bis}
If $R$ is a valuation ring of dimension $> 1$, then $R[[x]]$
is flat over $R$ but not flat over $R[x]$.
\end{lemma}

\begin{proof}
The arguments above show that this is true if we can show that
$R$ is not completely normal (valuation rings are
normal, see Algebra, Lemma \ref{algebra-lemma-valuation-ring-normal}).
Let $\mathfrak p \subset \mathfrak m \subset R$ be a chain of primes.
Pick nonzero $x \in \mathfrak p$ and $y \in \mathfrak m \setminus \mathfrak p$.
Then $x y^{-n} \in R$ for all $n \geq 1$ (if not then $y^n/x \in R$
which is absurd because $y \not \in \mathfrak p$). Hence $1/y$ is
almost integral over $R$ but not in $R$.
\end{proof}

\noindent
Next, we will construct an example where the completion of a localization
is nonflat. To do this consider the ring
$$
R = k[y, z, a_1, a_2, a_3, \ldots]/(ya_i, a_i a_j)
$$
Denote $f \in R$ the residue class of $z$. We claim the ring map
\begin{equation}
\label{equation-nonflat}
R[[x]] \longrightarrow R_f[[x]]
\end{equation}
isn't flat. Let $I$ be the kernel of $y : R[[x]] \to R[[x]]$. A typical
element $g$ of $I$ looks like $g = \sum g_{n, m} a_mx^n$
where $g_{n, m} \in k[z]$ and for a given $n$ only a finite number of
nonzero $g_{n, m}$. Let $J$ be the kernel of $y : R_f[[x]] \to R_f[[x]]$.
We claim that $J \not = I R_f[[x]]$. Namely, if this were true then we
would have
$$
\sum z^{-n} a_n x^n = \sum\nolimits_{i = 1, \ldots, m} h_i g_i
$$
for some $m \geq 1$, $g_i \in I$, and $h_i \in R_f[[x]]$. Say
$h_i = \bar h_i \bmod (y, a_1, a_2, a_3, \ldots)$
with $\bar h_i \in k[z, 1/z][[x]]$.  Looking at the coefficient of
$a_n$ and using the description of the elements $g_i$ above we would get
$$
z^{-n} x^n = \sum \bar h_i \bar g_{i, n}
$$
for some $\bar g_{i, n} \in k[z][[x]]$. This would mean that
all $z^{-n}x^n$ are contained in the finite $k[z][[x]]$-module
generated by the elements $\bar h_i$. Since $k[z][[x]]$ is Noetherian
this implies that the $R[z][[x]]$-submodule of $k[z, 1/z][[x]]$
generated by $1, z^{-1}x, z^{-2}x^2, \ldots$ is finite. By
Algebra, Lemma \ref{algebra-lemma-characterize-integral-element}
we would conclude that $z^{-1}x$ is integral over $k[z][[x]]$
which is absurd. On the other hand,
if (\ref{equation-nonflat}) were flat, then we would
get $J = IR_f[[x]]$ by tensoring the exact sequence
$0 \to I \to R[[x]] \xrightarrow{y} R[[x]]$ with $R_f[[x]]$.

\begin{lemma}
\label{lemma-nonflat-completion-localization}
There exists a ring $A$ complete with respect to a principal ideal $I$
and an element $f \in A$ such that the $I$-adic completion
$A_f^\wedge$ of $A_f$ is not flat over $A$.
\end{lemma}

\begin{proof}
Set $A = R[[x]]$ and $I = (x)$ and observe that $R_f[[x]]$
is the completion of $R[[x]]_f$.
\end{proof}





\section{Nonabelian category of quasi-coherent modules}
\label{section-nonabelian-QCoh}

\noindent
In Sheaves on Stacks, Section \ref{stacks-sheaves-section-quasi-coherent}
we defined the category of quasi-coherent modules on a category fibred in
groupoids over $\Sch$. Although we show in
Sheaves on Stacks, Section
\ref{stacks-sheaves-section-quasi-coherent-algebraic-stacks}
that this category is abelian for algebraic stacks, in this
section we show that this is not the case for formal algebraic spaces.

\medskip\noindent
Namely, consider $\mathbf{Z}_p$ viewed as topological ring using
the $p$-adic topology. Let $X = \text{Spf}(\mathbf{Z}_p)$, see
Formal Spaces, Definition
\ref{formal-spaces-definition-affine-formal-spectrum}.
Then $X$ is a sheaf in sets on $(\Sch/\mathbf{Z})_{fppf}$
and gives rise to a stack in setoids $\mathcal{X}$, see 
Stacks, Lemma \ref{stacks-lemma-when-stack-in-sets}.
Thus the discussion of Sheaves on Stacks, Section
\ref{stacks-sheaves-section-quasi-coherent-algebraic-stacks}
applies. 

\medskip\noindent
Let $\mathcal{F}$ be a quasi-coherent module on $\mathcal{X}$.
Since $X = \colim \Spec(\mathbf{Z}/p^n\mathbf{Z})$ it is clear from
Sheaves on Stacks, Lemma \ref{stacks-sheaves-lemma-quasi-coherent}
that $\mathcal{F}$ is given by a sequence $(\mathcal{F}_n)$ where
\begin{enumerate}
\item $\mathcal{F}_n$ is a quasi-coherent module on
$\Spec(\mathbf{Z}/p^n\mathbf{Z})$, and
\item the transition maps give isomorphisms
$\mathcal{F}_n = \mathcal{F}_{n + 1}/p^n\mathcal{F}_{n + 1}$.
\end{enumerate}
Converting into modules we see that $\mathcal{F}$ corresponds to a
system $(M_n)$ where each $M_n$ is an abelian group annihilated
by $p^n$ and the transition maps induce isomorphisms
$M_n = M_{n + 1}/p^n M_{n + 1}$. In this situation the module
$M = \lim M_n$ is a $p$-adically complete module and $M_n = M/p^n M$, see
Algebra, Lemma \ref{algebra-lemma-limit-complete}.
We conclude that the category of quasi-coherent modules on $X$
is equivalent to the category of $p$-adically complete
abelian groups. This category is not abelian, see
Section \ref{section-non-abelian}.

\begin{lemma}
\label{lemma-quasi-coherent-not-abelian}
The category of quasi-coherent\footnote{With quasi-coherent modules
as defined above. Due to how things are setup in the Stacks project,
this is really the correct definition; as seen above our definition
agrees with what one would naively have defined to be quasi-coherent modules
on $\text{Spf}(A)$, namely complete $A$-modules.}
modules on a formal algebraic space
$X$ is not abelian in general, even if $X$ is a Noetherian affine
formal algebraic space.
\end{lemma}

\begin{proof}
See discussion above.
\end{proof}


\section{Regular sequences and base change}
\label{section-regular-base-change}

\noindent
We are going to construct a ring $R$ with a regular sequence
$(x, y, z)$ such that there exists a nonzero element $\delta \in R/zR$
with $x\delta = y\delta = 0$.

\medskip\noindent
To construct our example we first
construct a peculiar module $E$ over the ring $k[x, y, z]$
where $k$ is any field. Namely, $E$ will be a push-out as
in the following diagram
$$
\xymatrix{
\frac{xk[x, y, z, y^{-1}]}{xyk[x, y, z]} \ar[r] \ar[d]^{z/x} &
\frac{k[x, y, z, x^{-1}, y^{-1}]}{yk[x, y, z, x^{-1}]} \ar[r] \ar[d] &
\frac{k[x, y, z, x^{-1}, y^{-1}]}{yk[x, y, z, x^{-1}] + xk[x, y, z, y^{-1}]}
\ar[d] \\
\frac{k[x, y, z, y^{-1}]}{yzk[x, y, z]} \ar[r] &
E \ar[r] &
\frac{k[x, y, z, x^{-1}, y^{-1}]}{yk[x, y, z, x^{-1}] + xk[x, y, z, y^{-1}]}
}
$$
where the rows are short exact sequences (we dropped the outer zeros due
to typesetting problems). Another way to describe $E$ is as
$$
E = \{(f, g) \mid f \in k[x, y, z, x^{-1}, y^{-1}],
g \in k[x, y, z, y^{-1}] \}/\sim
$$
where $(f, g) \sim (f', g')$ if and only if there exists a
$h \in k[x, y, z, y^{-1}]$ such that
$$
f = f' + xh \bmod yk[x, y, z, x^{-1}], \quad
g = g' - zh \bmod yzk[x, y, z]
$$
We claim: (a) $x : E \to E$ is injective, (b)
$y : E/xE \to E/xE$ is injective, (c) $E/(x, y)E = 0$, (d) there
exists a nonzero element $\delta \in E/zE$ such that
$x\delta = y\delta = 0$.

\medskip\noindent
To prove (a) suppose that $(f, g)$ is a pair that gives rise to an
element of $E$ and that $(xf, xg) \sim 0$. Then there exists a
$h \in k[x, y, z, y^{-1}]$ such that $xf + xh \in yk[x, y, z, x^{-1}]$
and $xg - zh \in yzk[x, y, z]$. We may assume that
$h = \sum a_{i, j, k}x^iy^jz^k$ is a sum of monomials where only
$j \leq 0$ occurs. Then $xg - zh \in yzk[x, y, z]$ implies that
only $i > 0$ occurs, i.e., $h = xh'$ for some $h' \in k[x, y, z, y^{-1}]$.
Then $(f, g) \sim (f + xh', g - zh')$ and we see that we may assume
that $g = 0$ and $h = 0$. In this case $xf \in yk[x, y, z, x^{-1}]$
implies $f \in yk[x, y, z, x^{-1}]$ and we see that $(f, g) \sim 0$.
Thus $x : E \to E$ is injective.

\medskip\noindent
Since multiplication by $x$ is an isomorphism on
$\frac{k[x, y, z, x^{-1}, y^{-1}]}{yk[x, y, z, x^{-1}]}$ we see that
$E/xE$ is isomorphic to
$$
\frac{k[x, y, z, y^{-1}]}{
yzk[x, y, z] + xk[x, y, z, y^{-1}] + zk[x, y, z, y^{-1}]}
=
\frac{k[x, y, z, y^{-1}]}{xk[x, y, z, y^{-1}] + zk[x, y, z, y^{-1}]}
$$
and hence multiplication by $y$ is an isomorphism on $E/xE$. This clearly
implies (b) and (c).

\medskip\noindent
Let $e \in E$ be the equivalence class of $(1, 0)$.
Suppose that $e \in zE$. Then there exist $f \in k[x, y, z, x^{-1}, y^{-1}]$,
$g \in k[x, y, z, y^{-1}]$, and $h \in k[x, y, z, y^{-1}]$ such that
$$
1 + zf + xh \in yk[x, y, z, x^{-1}], \quad
0 + zg - zh \in yzk[x, y, z].
$$
This is impossible: the monomial $1$ cannot occur in
$zf$, nor in $xh$. On the other hand, we have $ye = 0$ and
$xe = (x, 0) \sim (0, -z) = z(0, -1)$. Hence setting $\delta$
equal to the congruence class of $e$ in $E/zE$ we obtain (d).

\begin{lemma}
\label{lemma-strange-regular-sequence}
There exists a local ring $R$ and a regular sequence $x, y, z$
(in the maximal ideal) such that there exists a nonzero element
$\delta \in R/zR$ with $x\delta = y\delta = 0$.
\end{lemma}

\begin{proof}
Let $R = k[x, y, z] \oplus E$ where $E$ is the module above considered
as a square zero ideal. Then it is clear that $x, y, z$ is a regular
sequence in $R$, and that the element $\delta \in E/zE \subset R/zR$
gives an element with the desired properties. To get a local example
we may localize $R$ at the maximal ideal $\mathfrak m = (x, y, z, E)$.
The sequence $x, y, z$ remains a regular sequence (as localization is
exact), and the element $\delta$ remains nonzero as it is supported
at $\mathfrak m$.
\end{proof}

\begin{lemma}
\label{lemma-base-change-regular-sequence}
There exists a local homomorphism of local rings $A \to B$
and a regular sequence $x, y$ in the maximal ideal of $B$ such that
$B/(x, y)$ is flat over $A$, but such that the images
$\overline{x}, \overline{y}$ of $x, y$ in $B/\mathfrak m_AB$ do not
form a regular sequence, nor even a Koszul-regular sequence.
\end{lemma}

\begin{proof}
Set $A = k[z]_{(z)}$ and let $B = (k[x, y, z] \oplus E)_{(x, y, z, E)}$.
Since $x, y, z$ is a regular sequence in $B$, see proof of
Lemma \ref{lemma-strange-regular-sequence},
we see that $x, y$ is a regular sequence in $B$ and that
$B/(x, y)$ is a torsion free $A$-module, hence flat.
On the other hand, there exists a nonzero element
$\delta \in B/\mathfrak m_AB = B/zB$ which is annihilated
by $\overline{x}, \overline{y}$. Hence
$H_2(K_\bullet(B/\mathfrak m_AB, \overline{x}, \overline{y})) \not = 0$.
Thus $\overline{x}, \overline{y}$ is not Koszul-regular, in particular
it is not a regular sequence, see
More on Algebra, Lemma \ref{more-algebra-lemma-regular-koszul-regular}.
\end{proof}





\section{A Noetherian ring of infinite dimension}
\label{section-Noetherian-infinite-dimension}

\noindent
A Noetherian local ring has finite dimension as we saw in
Algebra, Proposition \ref{algebra-proposition-dimension}.
But there exist Noetherian rings of infinite dimension.
See \cite[Appendix, Example 1]{Nagata}.

\medskip\noindent
Namely, let $k$ be a field, and consider the ring
$$
R = k[x_1, x_2, x_3, \ldots ].
$$
Let $\mathfrak p_i = (x_{2^{i - 1}}, x_{2^{i - 1} + 1}, \ldots, x_{2^i - 1})$
for $i = 1, 2, \ldots$ which are prime ideals of $R$. Let $S$
be the multiplicative subset
$$
S = \bigcap\nolimits_{i \geq 1} (R \setminus \mathfrak p_i).
$$
Consider the ring $A = S^{-1}R$.
We claim that
\begin{enumerate}
\item The maximal ideals of the ring $A$ are the ideals
$\mathfrak m_i = \mathfrak p_iA$.
\item We have $A_{\mathfrak m_i} = R_{\mathfrak p_i}$ which is
a Noetherian local ring of dimension $2^i$.
\item The ring $A$ is Noetherian.
\end{enumerate}
Hence it is clear that this is the example we are looking for.
Details omitted.




\section{Local rings with nonreduced completion}
\label{section-local-completion-nonreduced}

\noindent
In Algebra, Example \ref{algebra-example-bad-dvr-char-p} we gave an example
of a characteristic $p$ Noetherian local domain $R$ of dimension $1$
whose completion is nonreduced. In this section we present the example
of \cite[Proposition 3.1]{Ferrand-Raynaud} which gives a similar
ring in characteristic zero.

\medskip\noindent
Let $\mathbf{C}\{x\}$ be the ring of convergent power series over
the field $\mathbf{C}$ of complex numbers. The ring of all power series
$\mathbf{C}[[x]]$ is its completion. Let $K = \mathbf{C}\{x\}[1/x]$
be the field of convergent Laurent series. The $K$-module
$\Omega_{K/\mathbf{C}}$ of algebraic differentials
of $K$ over $\mathbf{C}$ is an infinite dimensional $K$-vector space
(proof omitted). We may choose $f_n \in x\mathbf{C}\{x\}$,
$n \geq 1$ such that
$
\text{d}x, \text{d}f_1, \text{d}f_2, \ldots
$
are part of a basis of $\Omega_{K/\mathbf{C}}$. Thus we can
find a $\mathbf{C}$-derivation
$$
D : \mathbf{C}\{x\} \longrightarrow \mathbf{C}((x))
$$
such that $D(x) = 0$ and $D(f_i) = x^{-n}$. Let
$$
A = \{f \in \mathbf{C}\{x\} \mid D(f) \in \mathbf{C}[[x]]\}
$$
We claim that
\begin{enumerate}
\item $\mathbf{C}\{x\}$ is integral over $A$,
\item $A$ is a local domain,
\item $\dim(A) = 1$,
\item the maximal ideal of $A$ is generated by $x$ and $xf_1$,
\item $A$ is Noetherian, and
\item the completion of $A$ is equal to the ring of dual numbers
over $\mathbf{C}[[x]]$.
\end{enumerate}
Since the dual numbers are nonreduced the ring $A$ gives the example.

\medskip\noindent
Note that if $0 \not = f \in x\mathbf{C}\{x\}$ then
we may write $D(f) = h/f^n$ for some $n \geq 0$ and $h \in \mathbf{C}[[x]]$.
Hence $D(f^{n + 1}/(n + 1)) \in \mathbf{C}[[x]]$
and $D(f^{n + 2}/(n + 2)) \in \mathbf{C}[[x]]$. Thus we
see $f^{n + 1}, f^{n + 2} \in A$!
In particular we see (1) holds. We also conclude that
the fraction field of $A$ is equal to the fraction field of
$\mathbf{C}\{x\}$. It also follows immediately that
$A \cap x\mathbf{C}\{x\}$ is the set of nonunits of $A$, hence
$A$ is a local domain of dimension $1$. If we can show (4)
then it will follow that $A$ is Noetherian (proof omitted).
Suppose that $f \in A \cap x\mathbf{C}\{x\}$. Write
$D(f) = h$, $h \in \mathbf{C}[[x]]$. Write $h = c + xh'$
with $c \in \mathbf{C}$, $h' \in \mathbf{C}[[x]]$. Then
$D(f - cxf_1) = c + xh' - c = xh'$. On the other hand
$f - cxf_1 = xg$ with $g \in \mathbf{C}\{x\}$, but by the
computation above we have $D(g) = h' \in \mathbf{C}[[x]]$
and hence $g \in A$. Thus $f = cxf_1 + xg \in (x, xf_1)$ as desired.

\medskip\noindent
Finally, why is the completion of $A$ nonreduced? Denote $\hat A$ the
completion of $A$. Of course this maps surjectively to the completion
$\mathbf{C}[[x]]$ of $\mathbf{C}\{x\}$ because $x \in A$. Denote
this map $\psi : \hat A \to \mathbf{C}[[x]]$.
Above we saw that $\mathfrak m_A = (x, xf_1)$
and hence $D(\mathfrak m_A^n) \subset (x^{n - 1})$ by an easy
computation. Thus $D : A \to \mathbf{C}[[x]]$ is continuous and
gives rise to a continuous derivation $\hat D : \hat A \to \mathbf{C}[[x]]$
over $\psi$. Hence we get a ring map
$$
\psi + \epsilon \hat D :
\hat A
\longrightarrow
\mathbf{C}[[x]][\epsilon].
$$
Since $\hat A$ is a one dimensional Noetherian complete local ring, if we
can show this arrow is surjective then it will follow that $\hat A$
is nonreduced. Actually the map is an isomorphism but we omit the
verification of this. The subring $\mathbf{C}[x]_{(x)} \subset A$
gives rise to a map $i : \mathbf{C}[[x]] \to \hat A$ on completions such
that $i \circ \psi = \text{id}$ and such that $D \circ i = 0$
(as $D(x) = 0$ by construction). Consider the elements $x^nf_n \in A$.
We have
$$
(\psi + \epsilon D)(x^nf_n) = x^n f_n + \epsilon
$$
for all $n \geq 1$. Surjectivity easily follows from these remarks.





\section{Another local ring with nonreduced completion}
\label{section-another-local-completion-nonreduced}

\noindent
In this section we make an example of a Noetherian local domain
of dimension $2$ complete with respect to a principal ideal
such that the recompletion of a localization is nonreduced.

\medskip\noindent
Let $p$ be a prime number. Let $k$ be a field of characteristic $p$
such that $k$ has infinite degree over its subfield $k^p$ of $p$th powers.
For example $k = \mathbf{F}_p(t_1, t_2, t_3, \ldots)$.
Consider the ring
$$
A =
\left\{
\begin{matrix}
\sum a_{i, j} x^iy^j \in k[[x, y]] \text{ such that for all }n \geq 0
\text{ we have } \\
[k^p(a_{n, n}, a_{n, n + 1}, a_{n + 1, n},
a_{n, n + 2}, a_{n + 2, n}, \ldots) : k^p] < \infty
\end{matrix}
\right\}
$$
As a set we have
$$
k^p[[x, y]] \subset A  \subset k[[x, y]]
$$
Every element $f$ of $A$ can be uniquely written as a series
$$
f = f_0 + f_1 xy + f_2 (xy)^2 + f_3 (xy)^3 + \ldots
$$
with
$$
f_n = a_{n, n} + a_{n, n + 1} y + a_{n + 1, n} x +
a_{n, n + 2} y^2 + a_{n + 2, n} x^2 + \ldots
$$
and the condition in the formula defining $A$ means that the
coefficients of $f_n$ generate a finite extension of $k^p$.
From this presentation it is clear that $A$ is an
$k^p[[x, y]]$-subalgebra of $k[[x, y]]$ complete with
respect to the ideal $xy$. Moreover, we clearly have
$$
A/xy A = C \times_k D
$$
where $k^p[[x]] \subset C \subset k[[x]]$ and
$k^p[[y]] \subset D \subset k[[y]]$ are the subrings
of power series from
Algebra, Example \ref{algebra-example-bad-dvr-char-p}.
Hence $C$ and $D$ are dvrs and we see that $A/ xy A$ is
Noetherian. By Algebra, Lemma \ref{algebra-lemma-completion-Noetherian} we
conclude that $A$ is Noetherian. Since $\dim(k[[x, y]]) = 2$ using
Algebra, Lemma \ref{algebra-lemma-integral-sub-dim-equal}
we conclude that $\dim(A) = 2$.

\medskip\noindent
Let $f = \sum a_i x^i$ be a power series such that
$k^p(a_0, a_1, a_2, \ldots)$ has infinite degree over $k^p$.
Then $f \not \in A$ but $f^p \in A$. We set
$$
B = A[f] \subset k[[x, y]]
$$
Since $B$ is finite over $A$ we see that $B$ is Noetherian.
Also, $B$ is  complete with respect to the ideal generated by $xy$, see
Algebra, Lemma \ref{algebra-lemma-completion-tensor}.
In fact $B$ is free over $A$ with basis $1, f, f^2, \ldots, f^{p - 1}$;
we omit the proof.

\medskip\noindent
We claim the ring
$$
(B_y)^\wedge = (B[1/y])^\wedge = \lim B[1/y]/(xy)^n B[1/y] =
\lim B[1/y] / x^n B[1/y]
$$
is nonreduced. Namely, this ring is free over
$$
(A_y)^\wedge = (A[1/y])^\wedge = \lim A[1/y]/(xy)^n A[1/y] =
\lim A[1/y] / x^n A[1/y]
$$
with basis $1, f, \ldots, f^{p - 1}$. However, there is an
element $g \in (A_y)^\wedge$ such that $f^p = g^p$. Namely,
we can just take $g = \sum a_i x^i$ (the same expression as
we used for $f$) which makes sense in $(A_y)^\wedge$.
Hence we see that
$$
(B_y)^\wedge = (A_y)^\wedge[f]/(f^p - g^p) \cong
(A_y)^\wedge[\tau]/(\tau^p)
$$
is nonreduced. In fact, this example shows slightly more.
Namely, observe that $(A_y)^\wedge$ is a dvr with
uniformizer $x$ and residue field the fraction field
of the dvr $D$ given above. Hence we see that even
$$
(B_y)^\wedge[1/(xy)] = ((B_y)^\wedge)_{xy}
$$
is nonreduced. This produces an example of the following kind.

\begin{lemma}
\label{lemma-nonreduced-recompletion}
There exists a local Noetherian $2$-dimensional domain $(B, \mathfrak m)$
complete with respect to a principal ideal $I = (b)$ and an
element $f \in \mathfrak m$, $f \not \in I$ such that
the $I$-adic completion $C = (B_f)^\wedge$ of the principal
localization $B_f$ is nonreduced and even such that
$C_b = C[1/b] = (B_f)^\wedge[1/b]$ is nonreduced.
\end{lemma}

\begin{proof}
See discussion above.
\end{proof}








\section{A non catenary Noetherian local ring}
\label{section-non-catenary-Noetherian-local}

\noindent
Even though there is a succesful dimension theory of Noetherian local rings
there are non-catenary Noetherian local rings. An example may be found in
\cite[Appendix, Example 2]{Nagata}. In fact, we will present this example
in the simplest case. Namely, we will construct a local Noetherian domain $A$
of dimension $2$ which is not universally catenary. (Note that $A$ is
automatically catenary, see
Exercises, Exercise
\ref{exercises-exercise-Noetherian-local-domain-dim-2-catenary}.)
The existence of a Noetherian local ring which is not universally
catenary implies the existence of a Noetherian local ring which
is not catenary -- and we spell this out at the end of this section
in the particular example at hand.

\medskip\noindent
Let $k$ be a field, and consider the formal power series ring
$k[[x]]$ in one variable over $k$. Let
$$
z = \sum\nolimits_{i = 1}^\infty a_i x^i
$$
be a formal power series. We assume $z$ as an element of the Laurent
series field $k((x)) = k[[x]][1/x]$ is transcendental over $k(x)$.
Put
$$
z_j
=
x^{-j}(z - \sum\nolimits_{i = 1, \ldots, j - 1} a_i x^i)
=
\sum\nolimits_{i = j}^\infty a_i x^{i - j}
\in k[[x]].
$$
Note that $z = xz_1$.
Let $R$ be the subring of $k[[x]]$ generated by $x$, $z$ and all of the
$z_j$, in other words
$$
R = k[x, z_1, z_2, z_3, \ldots ] \subset k[[x]].
$$
Consider the ideals $\mathfrak m = (x)$ and
$\mathfrak n = (x - 1, z_1, z_2, \ldots)$ of $R$.

\medskip\noindent
We have $xz_{j + 1} + a_j = z_j$. Hence $R/\mathfrak m = k$
and $\mathfrak m$ is a maximal ideal. Moreover, any element of $R$
not in $\mathfrak m$ maps to a unit in $k[[x]]$ and hence
$R_{\mathfrak m} \subset k[[x]]$. In fact it is easy to deduce
that $R_{\mathfrak m}$ is a discrete valuation ring and residue
field $k$.

\medskip\noindent
We claim that
$$
R/(x - 1) =
k[x, z_1, z_2, z_3, \ldots ]/(x - 1)
\cong
k[z].
$$
Namely, the relation above implies that
$z_{j + 1} = z_j - a_j - (x - 1)z_{j + 1}$, and hence
we may express the class of $z_{j + 1}$ in terms of $z_j$ in
the quotient $R/(x - 1)$. Since the fraction field of $R$
has transcendence degree $2$ over $k$ by construction we see that $z$ is
transcendental over $k$ in $R/(x - 1)$, whence the desired isomorphism.
Hence $\mathfrak n = (x - 1, z)$ and is a maximal ideal. In fact the
map
$$
k[x, x^{-1}, z]_{(x - 1, z)} \longrightarrow R_{\mathfrak n}
$$
is an isomorphism (since $x^{-1}$ is invertible in $R_{\mathfrak n}$
and since $z_{j + 1} = x^{-1}z_j - a_j = \ldots = f_j(x, x^{-1}, z)$).
This shows that $R_{\mathfrak n}$ is a regular local ring
of dimension $2$ and residue field $k$.

\medskip\noindent
Let $S$ be the multiplicative subset
$$
S =
(R \setminus \mathfrak m) \cap (R \setminus \mathfrak n) =
R \setminus (\mathfrak m \cup \mathfrak n)
$$
and set $B = S^{-1}R$. We claim that
\begin{enumerate}
\item The ring $B$ is a $k$-algebra.
\item The maximal ideals of the ring $B$ are the two ideals
$\mathfrak mB$ and $\mathfrak nB$.
\item The residue field at these maximal ideals is $k$.
\item We have $B_{\mathfrak mB} = R_{\mathfrak m}$
and $B_{\mathfrak nB} = R_{\mathfrak n}$
which are Noetherian regular local rings of dimensions $1$ and $2$.
\item The ring $B$ is Noetherian.
\end{enumerate}
We omit the details of the verifications.

\medskip\noindent
Whenever given a $k$-algebra $B$ with the properties listed above we
get an example as follows. Take $A = k + \text{rad}(B) \subset B$
with $\text{rad}(B) = \mathfrak mB \cap \mathfrak nB$ the Jacobson radical.
It is easy to see that $B$ is finite over $A$ and hence $A$ is
Noetherian by Eakin's theorem (see \cite{Eakin}, or
\cite[Appendix A1]{Nagata}, or insert future reference here).
Also $A$ is a local domain with the same fraction field as $B$ and
residue field $k$. Since the dimension of $B$ is $2$ we see that $A$
has dimension $2$ as well, by
Algebra, Lemma \ref{algebra-lemma-integral-sub-dim-equal}.

\medskip\noindent
If $A$ were universally catenary then the dimension formula,
Algebra, Lemma \ref{algebra-lemma-dimension-formula}
would give $\dim(B_{\mathfrak mB}) = 2$ contradiction.

\medskip\noindent
Note that $B$ is generated by one element over $A$.
Hence $B = A[x]/\mathfrak p$ for some prime
$\mathfrak p$ of $A[x]$. Let $\mathfrak m' \subset A[x]$ be
the maximal ideal corresponding to $\mathfrak mB$. Then on
the one hand $\dim(A[x]_{\mathfrak m'}) = 3$ and on the
other hand
$$
(0)
\subset \mathfrak pA[x]_{\mathfrak m'}
\subset \mathfrak m'A[x]_{\mathfrak m'}
$$
is a maximal chain of primes. Hence $A[x]_{\mathfrak m'}$ is
an example of a non catenary Noetherian local ring.





\section{Existence of bad local Noetherian rings}
\label{section-bad}

\noindent
Let $(A, \mathfrak m, \kappa)$ be a Noetherian complete local ring.
In \cite{Lech} it was shown that $A$ is the completion of a Noetherian
local domain if $\text{depth}(A) \geq 1$ and $A$ contains either
$\mathbf{Q}$ or $\mathbf{F}_p$ as a subring, or contains $\mathbf{Z}$
as a subring and $A$ is torsion free as a $\mathbf{Z}$-module.
This produces many examples of Noetherian local domains with
``bizarre'' properties.

\medskip\noindent
Applying this for example to $A = \mathbf{C}[[x, y]]/(y^2)$ we find
a Noetherian local domain whose completion is nonreduced.
Please compare with
Section \ref{section-local-completion-nonreduced}.

\medskip\noindent
In \cite{LLPY} conditions were found that characterize when $A$ is
the completion of a reduced local Noetherian ring.

\medskip\noindent
In \cite{Heitmann-completion-UFD} it was shown that $A$ is the completion
of a local Noetherian UFD $R$ if $\text{depth}(A) \geq 2$ and $A$ contains
either $\mathbf{Q}$ or $\mathbf{F}_p$ as a subring, or contains $\mathbf{Z}$
as a subring and $A$ is torsion free as a $\mathbf{Z}$-module.
In particular $R$ is normal (Algebra, Lemma \ref{algebra-lemma-UFD-normal})
hence the henselization of $R$ is a normal domain too
(More on Algebra, Lemma \ref{more-algebra-lemma-henselization-normal}).
Thus $A$ as above is the completion of a henselian Noetherian local
normal domain (because the completion of $R$ and its henselization agree,
see More on Algebra, Lemma \ref{more-algebra-lemma-henselization-noetherian}).

\medskip\noindent
Apply this to find a Noetherian local UFD $R$ such that
$R^\wedge \cong \mathbf{C}[[x, y, z, w]]/(wx, wy)$.
Note that $\Spec(R^\wedge)$ is the
union of a regular $2$-dimensional and a regular $3$-dimensional component.
The ring $R$ cannot be universally catenary: Let
$$
X \longrightarrow \Spec(R)
$$
be the blowing up of the maximal ideal. Then $X$ is an integral scheme.
There is a closed point $x \in X$ such that $\dim(\mathcal{O}_{X, x}) = 2$,
namely, on the level of the complete local ring we pick $x$ to lie on the
strict transform of the $2$-dimensional component and not on the strict
transform of the $3$-dimensional component. By
Morphisms, Lemma \ref{morphisms-lemma-dimension-formula}
we see that $R$ is not universally catenary. Please compare with
Section \ref{section-non-catenary-Noetherian-local}.

\medskip\noindent
The ring above is catenary (being a $3$-dimensional local Noetherian UFD).
However, in \cite{Ogoma-example} the author constructs a normal local
Noetherian domain $R$ with $R^\wedge \cong \mathbf{C}[[x, y, z, w]]/(wx, wy)$
such that $R$ is not catenary. See also \cite{Heitmann-Ogoma} and
\cite{Lech-YAPO}.

\medskip\noindent
In \cite{Heitmann-isolated} it was shown that $A$ is the completion
of a local Noetherian ring $R$ with an isolated singularity
provided $A$ contains either $\mathbf{Q}$ or $\mathbf{F}_p$ as a subring
or $A$ has residue characteristic $p > 0$ and $p$ cannot map to a
nonzero zerodivisor in any proper localization of $A$.
Here we say a Noetherian local ring $R$
has an isolated singularity if $R_\mathfrak p$ is
a regular local ring for all nonmaximal primes $\mathfrak p \subset R$.

\medskip\noindent
The papers \cite{Nishimura-few} and \cite{Nishimura-few-II}
contain long lists of ``bad'' Noetherian
local rings with given completions. In particular it constructs
an example of a $2$-dimensional Nagata local normal domain whose
completion is $\mathbf{C}[[x, y, z]]/(yz)$ and one whose completion
is $\mathbf{C}[[x, y, z]]/(y^2 - z^3)$.

\medskip\noindent
As an aside, in \cite{Loepp} it was shown that $A$ is the completion of an
excellent Noetherian local domain if $A$ is reduced, equidimensional,
and no integer in $A$ is a zero divisor. However, this doesn't lead
to ``bad'' Noetherian local rings as we obtain excellent ones!




\section{Dimension in Noetherian Jacobson rings}
\label{section-noetherian-jacobson}

\noindent
Let $k$ be the algebraic closure of a finite field.
Let $A = k[x, y]$ and $X = \Spec(A)$.
Let $C = V(x)$ be the $y$-axis (this could be any other
$1$-dimensional integral closed subscheme of $X$).
Let $C_1, C_2, C_3, \ldots$ be an enumeration
of the other integral closed subschemes of $X$ of dimension $1$.
Let $p_1, p_2, p_3, \ldots$ be an
enumeration of the closed points of $C$.

\medskip\noindent
Claim: for every $n$ there exists an irreducible closed
$Z_n \subset X$ of dimension $1$ such that
$$
\{p_n\} = Z_n \cap (C \cup C_1 \cup \ldots \cup C_n)
$$
set theoretically. To do this set
$Y = C \cup C_1 \cup C_2 \cup \ldots \cup C_n$.
This is a reduced affine algebraic scheme of dimension $1$ over $k$.
It is enough to find $f \in k[x, y]$ with $V(f) \cap  Y = \{p_n\}$
set theoretically because then we can take $Z_n$ to be a suitable
irreducible component of $V(f)$. Since the restriction map
$$
k[x, y] \longrightarrow \Gamma(Y, \mathcal{O}_Y)
$$
is surjective, it suffices to find a regular function $g$ on $Y$
whose zero set is $\{p_n\}$ set theoretically.
To see this is possible, we choose an effective Cartier divisor
$D \subset Y$ whose support is $p_n$ (this is possible by
Varieties, Lemma \ref{varieties-lemma-complement-codim-1-closed-points}).
Thus it suffices to show that $\mathcal{O}_X(ND) \cong \mathcal{O}_X$
for some $N > 0$. But the Picard group of an affine $1$-dimensional
algebraic scheme over the algebraic closure of a finite field is torsion
(insert future reference here) and we conclude the claim is true.

\medskip\noindent
Choose $Z_n$ as above for all $n$. Since $k[x, y]$ is a UFD
we may write $Z_n = V(f_n)$ for some irreducible element $f_n \in A$.
Let $S \subset k[x, y]$ be the multiplicative subset generated
by $f_1, f_2, f_3, \ldots$. Consider the Noetherian ring $B = S^{-1}A$.

\medskip\noindent
Obviously, the ring map $A \to B$ identifies local rings
and induces an injection $\Spec(B) \to \Spec(A)$.
Moreover, looking at the curve $C_1$ we see that
only the points of $C \cap C_1$ are removed when passing
from $\Spec(A)$ to $\Spec(B)$.
In particular, we see that $\Spec(B)$ has an infinite
number of maximal ideals corresponding to maximal
ideals of $A$. On the other hand, $xB$ is a maximal ideal because
the spectrum of $B/xB$ consists of a unique prime ideal
as we removed all the closed points of $C = V(x)$ (but not
the generic point).
Finally, for $i \geq 1$ consider the curve $C_i$.
Write $C_i = V(g_i)$ for $g_i \in A$ irreducible.
If $C_i = Z_n$ for some $n$, then $g_iB$ is the unit ideal.
If not, then all but finitely many of the closed points
of $C_i$ survive the passage from $A$ to $B$:
namely, only the points of
$(Z_1 \cup \ldots \cup Z_{i - 1} \cup C) \cap C_i$
are removed from $C_i$.

\medskip\noindent
The structure of the prime spectrum of $B$ given above
shows that $B$ is Jacobson by
Algebra, Lemma \ref{algebra-lemma-noetherian-dim-1-Jacobson}.
The maximal ideals are the maximal ideals of $A$
which are in $\Spec(B)$ (and there an inifinitude of these)
together with the maximal ideal $xB$. Thus we see that we have
local rings of dimensions $1$ and $2$.

\begin{lemma}
\label{lemma-Noetherian-Jacobson}
There exists a Jacobson, universally catenary, Noetherian domain $B$
with maximal ideals $\mathfrak m_1, \mathfrak m_2$ such that
$\dim(B_{\mathfrak m_1}) = 1$ and $\dim(B_{\mathfrak m_2}) = 2$.
\end{lemma}

\begin{proof}
The construction of $B$ is given above. We just point out that
$B$ is universally catenary by
Algebra, Lemma \ref{algebra-lemma-localization-catenary} and
Morphisms, Lemma \ref{morphisms-lemma-ubiquity-uc}.
\end{proof}




\section{Underlying space Noetherian not Noetherian}
\label{section-noetherian-not-noetherian}

\noindent
We give two examples to show that a scheme whose underlying topological space
is Noetherian may not be a Noetherian scheme.

\begin{example}
\label{example-many-variables}
Let $k$ be a field, and let $A = k[x_1, x_2, x_3, \dots] /
(x_1^2, x_2^2, x_3^2, \dots)$.  Any prime ideal of $A$ contains the
nilpotents $x_1, x_2, x_3, \dots$, so $\mathfrak p = (x_1, x_2, x_3,
\dots)$ is the only prime ideal of $A$.  Therefore the underlying
topological space of $\operatorname{Spec} A$ is a single point and in
particular is Noetherian.  However $\mathfrak p$ is clearly not finitely
generated.
\end{example}

\begin{example}
\label{example-many-mononials}
Let $k$ be a field, and let $A \subseteq k[x, y]$ be the
subring generated by $k$ and the monomials $\{xy^i\}_{i \ge 0}$.  The prime
ideals of $A$ that do not contain $x$ are in one-to-one correspondence with
the prime ideals of $A_x \cong k[x, x^{-1}, y]$.  If $\mathfrak p$ is a
prime ideal that does contain $x$, then it contains every $xy^i$, $i \ge 0$,
because $(xy^i)^2 = x(xy^{2i}) \in \mathfrak p$ and $\mathfrak p$ is
radical.  Consequently $\mathfrak p = (\{xy^i\}_{i \ge 0})$.  Therefore the
underlying topological space of $\operatorname{Spec} A$ is Noetherian, since
it consists of the points of the Noetherian scheme
$\Spec(A[x, x^{-1}, y])$ and the prime ideal $\mathfrak p$.
But the ring $A$ is non-Noetherian because $\mathfrak p$
is not finitely generated.  Note that
in this example, $A$ also has the property of being a domain.
\end{example}






\section{Non-quasi-affine variety with quasi-affine normalization}
\label{section-nonquasi-affine}

\noindent
The existence of an example of this kind is mentioned in
\cite[II Remark 6.6.13]{EGA}. They refer to the fifth volume of
EGA for such an example, but the fifth volume did not appear.

\medskip\noindent
Let $k$ be a field.
Let $Y = \mathbf{A}^2_k \setminus \{(0, 0)\}$.
We are going to construct a finite surjective birational morphism
$\pi : Y \longrightarrow X$
with $X$ a variety over $k$ such that $X$ is not quasi-affine.
Namely, consider the following curves in $Y$:
$$
\begin{matrix}
C_1 & : & x = 0 \\
C_2 & : & y = 0
\end{matrix}
$$
Note that $C_1 \cap C_2 = \emptyset$. We choose the isomorphism
$\varphi : C_1 \to C_2$, $(0, y) \mapsto (y^{-1}, 0)$.
We claim there is a unique morphism $\pi : Y \to X$ as above
such that
$$
\xymatrix{
C_1
\ar@<1ex>[rr]^{\text{id}} \ar@<-1ex>[rr]_{\varphi}
& &
Y \ar[r]^\pi & X
}
$$
is a coequalizer diagram in the category of varieties (and even in
the category of schemes). Accepting this for the moment let us
show that such an $X$ cannot be quasi-affine. Namely, it is clear
that we would get
$$
\Gamma(X, \mathcal{O}_X) =
\{ f \in k[x, y] \mid f(0, y) = f(y^{-1}, 0)\} =
k \oplus (xy) \subset k[x, y].
$$
In particular these functions do not separate the points $(1, 0)$
and $(-1, 0)$ whose images in $X$ (we will see below) are distinct
(if the characteristic of $k$ is not $2$).

\medskip\noindent
To show that $X$ exists consider the Zariski open
$D(x + y) \subset Y$ of $Y$. This is the spectrum
of the ring
$k[x, y, 1/(x + y)]$
and the curves $C_1$, $C_2$ are completely contained in
$D(x + y)$. Moreover the morphism
$$
C_1 \amalg C_2
\longrightarrow
D(x + y) \cap Y = \Spec(k[x, y, 1/(x + y)])
$$
is a closed immersion. It follows from
More on Algebra, Lemma \ref{more-algebra-lemma-fibre-product-finite-type}
that the ring
$$
A =
\{f \in k[x, y, 1/(x + y)] \mid f(0, y) = f(y^{-1}, 0)\}
$$
is of finite type over $k$. On the other hand we have the open
$D(xy) \subset Y$ of $Y$ which is disjoint from the curves $C_1$
and $C_2$. It is the spectrum of the ring
$$
B = k[x, y, 1/xy].
$$
Note that we have $A_{xy} \cong B_{x + y}$ (since $A$ clearly contains
the elements $xyP(x, y)$ any polynomial $P$ and the element $xy/(x + y)$).
The scheme $X$ is obtained by glueing the affine schemes
$\Spec(A)$ and $\Spec(B)$ using the isomorphism
$A_{xy} \cong B_{x + y}$ and hence is clearly of finite type over
$k$. To see that it is separated one has to show that the
ring map $A \otimes_k B \to B_{x + y}$ is surjective. To see
this use that $A \otimes_k B$ contains the element
$xy/(x + y) \otimes 1/xy$ which maps to $1/(x + y)$.
The morphism $Y \to X$ is given by the natural maps
$D(x + y) \to \Spec(A)$ and $D(xy) \to \Spec(B)$.
Since these are both finite we deduce that $Y \to X$ is finite
as desired. We omit the verification that $X$ is indeed the
coequalizer of the displayed diagram above, however, see
(insert future reference for pushouts in the category of schemes
here). Note that the morphism $\pi : Y \to X$ does
map the points  $(1, 0)$
and $(-1, 0)$ to distinct points in $X$ because the
function $(x + y^3)/(x + y)^2 \in A$ has value
$1/1$, resp.\ $-1/(-1)^2 = -1$ which are always distinct
(unless the characteristic is $2$ -- please find your own points
for characteristic $2$). We summarize this discussion in the
form of a lemma.

\begin{lemma}
\label{lemma-quasi-affine-normalization-not-quasi-affine}
Let $k$ be a field.
There exists a variety $X$ whose normalization is quasi-affine but
which is itself not quasi-affine.
\end{lemma}

\begin{proof}
See discussion above and (insert future reference on normalization here).
\end{proof}



\section{Taking scheme theoretic images}
\label{section-scheme-theoretic-image}

\noindent
Let $k$ be a field. Let $t$ be a variable. Let $Y = \Spec(k[t])$
and $X = \coprod_{n \geq 1} \Spec(k[t]/(t^n))$. Denote
$f : X \to Y$ the morphism
using the closed immersion $\Spec(k[t]/(t^n)) \to \Spec(k[t])$
for each $n \geq 1$. In this case we have
\begin{enumerate}
\item The scheme theoretic image
(Morphisms, Definition \ref{morphisms-definition-scheme-theoretic-image})
of $f$ is $Y$. On the other hand, the image of $f$ is the closed
point $t = 0$ in $Y$. Thus the underlying closed subset of the
scheme theoretic image of $f$
is not equal to the closure of the image of $f$.
\item The formation of the scheme theoretic image does not commute
with restriction to the open subscheme $V = \Spec(k[t, 1/t]) \subset Y$.
Namely, the preimage of $V$ in $X$ is empty and hence the scheme theoretic
image of $f|_{f^{-1}(V)} : f^{-1}(V) \to V$ is the empty scheme.
This is not equal to $Y \cap V$.
\end{enumerate}




\section{Images of locally closed subsets}
\label{section-non-chevalley}

\noindent
Chevalley's theorem says that the image of a constructible set
by a finitely presented morphism of affine schemes is constructible, see
Algebra, Theorem \ref{algebra-theorem-chevalley} and
Morphisms, Section \ref{morphisms-section-constructible}.
We will see the same thing does not hold
for images of locally closed subsets.

\medskip\noindent
Let $k$ be a field of characteristic $0$. Consider the projection morphism
$$
f :
X = \Spec(k[t, x_1, x_2, \ldots, y_1, y_2, \ldots])
\longrightarrow
\Spec(k[x_1, x_2, \ldots, y_1, y_2, \ldots]) = Y
$$
This is a morphism of finite presentation.
Let $Z$ be the closed subset of $X$ defined by
$$
x_1(t - 1) = 0,\quad
x_2(t - 1)(t - 2) = 0,\quad
x_3(t - 1)(t - 2)(t - 3) = 0,\quad \ldots
$$
Let $U = \bigcup_{j \geq 1} U_j$ be the open of $X$ defined by
$$
U_j = \text{points where }y_j(t - 1)(t - 2) ... (t - j)\text{ is nonzero}
$$
Then we have
$$
f(Z \cap U_j) =
\text{points where }x_1, \ldots, x_j\text{ are zero and }y_j\text{ is nonzero}
$$
We claim that $B = f(Z \cap U) = \bigcup_{j \geq 1} f(Z \cap U_j)$ is
not a finite union of locally closed subsets of $Y$.

\medskip\noindent
Proof of the claim. Suppose that $B = A_1 \cup \cdots \cup A_m$
is a finite cover of $B$ by locally closed subsets of $Y$.
We will show by induction on $n$ that $m \geq n$. The base
case $n = 1$ is OK as $B$ is nonempty. Assume $n > 1$ and that
the induction hypothesis holds for $n - 1$. Since the closure of $B$ is
$(x_1 = 0)$, one of the $A_i$ must contain some nonempty open subset of
$(x_1 = 0)$. Then $A_i$ must be open in $(x_1 = 0)$.
But any such open subset cannot contain a point with
$y_1 = 0$; indeed, for points of $B$, $y_1 = 0$ forces $x_2 = 0$,
and this shows $B$ contains no neighborhood of $(x, y)$ inside $(x_1 = 0)$.
Therefore, the remaining $m - 1$ elements restrict
to a constructible cover of $B \cap (y_1 = 0)$.
However, observe that the right shift map
$x_i \mapsto x_{i + 1}$, $y_i \mapsto y_{i + 1}$ identifies $B$
with $B \cap (y_1 = 0)$! Thus by induction hypothesis, we see that
$m - 1 \geq n - 1$ and we conclde $m \geq n$. This finishes
the proof of the induction step and thereby establishes the claim.

\begin{lemma}
\label{lemma-no-chevalley}
There exists a morphism $f : X \to Y$ of finite presentation
between affine schemes and a locally closed subset $T$ of $X$
such that $f(T)$ is not a finite union of locally closed subsets of $Y$.
\end{lemma}

\begin{proof}
See discussion above.
\end{proof}










\section{A locally closed subscheme which is not open in closed}
\label{section-strange-immersion}

\noindent
This is a copy of
Morphisms, Example \ref{morphisms-example-thibaut}.
Here is an example of an immersion which is not a composition of an
open immersion followed by a closed immersion.
Let $k$ be a field.
Let $X = \Spec(k[x_1, x_2, x_3, \ldots])$.
Let $U = \bigcup_{n = 1}^{\infty} D(x_n)$.
Then $U \to X$ is an open immersion.
Consider the ideals
$$
I_n =
(x_1^n, x_2^n, \ldots, x_{n - 1}^n, x_n - 1, x_{n + 1}, x_{n + 2}, \ldots)
\subset
k[x_1, x_2, x_3, \ldots][1/x_n].
$$
Note that $I_n k[x_1, x_2, x_3, \ldots][1/x_nx_m] = (1)$
for any $m \not = n$. Hence the quasi-coherent ideals
$\widetilde I_n$ on $D(x_n)$ agree on $D(x_nx_m)$, namely
$\widetilde I_n|_{D(x_nx_m)} = \mathcal{O}_{D(x_n x_m)}$ if
$n \not = m$. Hence these ideals glue to a quasi-coherent sheaf of ideals
$\mathcal{I} \subset \mathcal{O}_U$.
Let $Z \subset U$ be the closed subscheme corresponding to
$\mathcal{I}$. Thus $Z \to X$ is an immersion.

\medskip\noindent
We claim that we cannot factor $Z \to X$ as
$Z \to \overline{Z} \to X$, where $\overline{Z} \to X$ is closed
and $Z \to \overline{Z}$ is open. Namely, $\overline{Z}$ would
have to be defined by an ideal $I \subset k[x_1, x_2, x_3, \ldots]$
such that $I_n = I k[x_1, x_2, x_3, \ldots][1/x_n]$.
But the only element $f \in k[x_1, x_2, x_3, \ldots]$
which ends up in all $I_n$ is $0$! Hence $I$ does not exist.

\medskip\noindent
The morphism $Z \to X$ also gives an example of bad behaviour
of scheme theoretic images of immersions. Namely, the arguments above
show that the scheme theoretic image of the immersion $Z \to X$ is $X$.
On the other hand, we see
\begin{enumerate}
\item $Z$ is not topologically dense in $X$, and
\item the scheme theoretic image of $Z = Z \cap U \to U$ is just
$Z$. This is not equal to $U \cap X = U$ and hence formation
of the scheme theoretic image in this case
does not commute with restrictions to opens.
\end{enumerate}






\section{Nonexistence of suitable opens}
\label{section-nonexistence-opens}

\noindent
This section complements the results of
Properties, Section \ref{properties-section-finding-affine-opens}.

\medskip\noindent
Let $k$ be a field and let $A = k[z_1, z_2, z_3, \ldots]/I$ where $I$
is the ideal generated by all pairwise products
$z_iz_j$, $i \not = j$, $i, j \in \mathbf{N}$. Set $S = \Spec(A)$.
Let $s \in S$ be the closed point corresponding to the maximal
ideal $(z_i)$. We claim there is no
quasi-compact open $V \subset S \setminus \{s\}$ which is
dense in $S \setminus \{s\}$. Note that $S \setminus \{s\} = \bigcup D(z_i)$.
Each $D(z_i)$ is open and irreducible with generic point
$\eta_i$. We conclude that $\eta_i \in V$ for all $i$.
However, a principal affine open of $S \setminus \{s\}$ is
of the form $D(f)$ where $f \in (z_1, z_2, \ldots)$. Then
$f \in (z_1, \ldots, z_n)$ for some $n$ and we see that $D(f)$ contains
only finitely many of the points $\eta_i$. Thus $V$ cannot be quasi-compact.

\medskip\noindent
Let $k$ be a field and let $B = k[x, z_1, z_2, z_3, \ldots]/J$ where $J$
is the ideal generated by the products $xz_i$, $i \in \mathbf{N}$ and by
all pairwise products $z_iz_j$, $i \not = j$, $i, j \in \mathbf{N}$.
Set $T = \Spec(B)$. Consider the principal open $U = D(x)$.
We claim there is no quasi-compact open $V \subset S$ such that
$V \cap U = \emptyset$ and $V \cup U$ is dense in $S$.
Let $t \in T$ be the closed point corresponding to the maximal
ideal $(x, z_i)$. The closure of $U$ in $T$ is
$\overline{U} = U \cup \{t\}$. Hence $V \subset \bigcup_i D(z_i)$
is a quasi-compact open. By the arguments of the previous paragraph
we see that $V$ cannot be dense in $\bigcup D(z_i)$.

\begin{lemma}
\label{lemma-complement-of-affine-does-not-contain-qc-dense-open}
Nonexistence quasi-compact opens of affines:
\begin{enumerate}
\item There exist an affine scheme $S$ and affine open $U \subset S$
such that there is no quasi-compact open $V \subset S$ with
$U \cap V = \emptyset$ and $U \cup V$ dense in $S$.
\item There exists an affine scheme $S$ and a closed point $s \in S$ such that
$S \setminus \{s\}$ does not contain a quasi-compact dense open.
\end{enumerate}
\end{lemma}

\begin{proof}
See discussion above.
\end{proof}

\noindent
Let $X$ be the glueing of two copies of the affine scheme $T$ (see above)
along the affine open $U$. Thus there is a morphism $\pi : X \to T$ and
$X = U_1 \cup U_2$ such that $\pi$ maps $U_i$ isomorphically to $T$ and
$U_1 \cap U_2$ isomorphically to $U$. Note that $X$ is quasi-separated
(by Schemes, Lemma \ref{schemes-lemma-characterize-quasi-separated})
and quasi-compact. We claim there does not exist a separated, dense,
quasi-compact open $W \subset X$. Namely, consider the two closed
points $x_1 \in U_1$, $x_2 \in U_2$ mapping to the closed point $t \in T$
introduced above. Let $\tilde \eta \in U_1 \cap U_2$ be the generic point
mapping to the (unique) generic point $\eta$ of $U$.
Note that $\tilde\eta \leadsto x_1$ and $\tilde\eta \leadsto x_2$
lying over the specialization $\eta \leadsto s$.
Since $\pi|_W : W \to T$ is separated we conclude that we cannot have both
$x_1$ and $x_2 \in W$ (by the valuative criterion of separatedness
Schemes, Lemma \ref{schemes-lemma-valuative-criterion-separatedness}).
Say $x_1 \not \in W$. Then $W \cap U_1$ is a quasi-compact (as $X$
is quasi-separated) dense open of $U_1$ which does not contain $x_1$.
Now observe that there exists an isomorphism $(T, t) \cong (S, s)$
of schemes (by sending $x$ to $z_1$ and $z_i$ to $z_{i + 1}$).
Hence by the first paragraph of this section we arrive at a contradiction.

\begin{lemma}
\label{lemma-no-dense-separated-quasi-compact-open-in-qcqs}
There exists a quasi-compact and quasi-separated scheme $X$ which does
not contain a separated quasi-compact dense open.
\end{lemma}

\begin{proof}
See discussion above.
\end{proof}







\section{Nonexistence of quasi-compact dense open subscheme}
\label{section-nonexistence-qc-dense-open-subscheme}

\noindent
Let $X$ be a quasi-compact and quasi-separated algebraic space over a field
$k$. We know that the schematic locus $X' \subset X$ is a dense open
subspace, see
Properties of Spaces, Proposition
\ref{spaces-properties-proposition-locally-quasi-separated-open-dense-scheme}.
In fact, this result holds when $X$ is reasonable, see
Decent Spaces, Proposition
\ref{decent-spaces-proposition-reasonable-open-dense-scheme}.
A natural question is whether one can find a quasi-compact dense
open subscheme of $X$. It turns out this is not possible in general.

\medskip\noindent
Assume the characteristic of $k$ is not 2.
Let $B = k[x, z_1, z_2, z_3, \ldots]/J$ where $J$ is the ideal generated by
the products $xz_i$, $i \in \mathbf{N}$ and by all pairwise products
$z_iz_j$, $i \not = j$, $i, j \in \mathbf{N}$. Set $U = \Spec(B)$.
Denote $0 \in U$ the closed point all of whose coordinates are zero.
Set
$$
j : R = \Delta \amalg \Gamma \longrightarrow U \times_k U
$$
where $\Delta$ is the image of the diagonal morphism of $U$ over $k$ and
$$
\Gamma = \{((x, 0, 0, 0, \ldots), (-x, 0, 0, 0, \ldots))
\mid x \in \mathbf{A}^1_k, x \not = 0\}.
$$
It is clear that $s, t : R \to U$ are \'etale, and hence
$j$ is an \'etale equivalence relation. The quotient $X = U/R$
is an algebraic space (Spaces, Theorem \ref{spaces-theorem-presentation}).
Note that $j$ is not an immersion because
$(0, 0) \in \Delta$ is in the closure of $\Gamma$.
Hence $X$ is not a scheme. On the other hand, $X$ is quasi-separated
as $R$ is quasi-compact. Denote $0_X$ the image of the point $0 \in U$.
We claim that $X \setminus \{0_X\}$ is a scheme, namely
$$
X \setminus \{0_X\} =
\Spec\left(k[x^2, x^{-2}]\right) \amalg
\Spec\left(k[z_1, z_2, z_3, \ldots]/(z_iz_j)\right) \setminus \{0\}
$$
(details omitted). On the other hand, we have seen in
Section \ref{section-nonexistence-opens} that the scheme
on the right hand side does not contain
a quasi-compact dense open.

\begin{lemma}
\label{lemma-nonexistence-qc-dense-open-subscheme}
There exists a quasi-compact and quasi-separated algebraic space
which does not contain a quasi-compact dense open subscheme.
\end{lemma}

\begin{proof}
See discussion above.
\end{proof}

\noindent
Using the construction of 
Spaces, Example \ref{spaces-example-non-representable-descent}
in the same manner as we used the construction of
Spaces, Example \ref{spaces-example-affine-line-involution}
above, one obtains an example of a quasi-compact, quasi-separated, and
locally separated algebraic space which does not contain a quasi-compact
dense open subscheme.


\section{Affines over algebraic spaces}
\label{section-embedding-affines}

\medskip\noindent
Suppose that $f : Y \to X$ is a morphism of schemes with $f$
locally of finite type and $Y$ affine. Then there exists an immersion
$Y \to \mathbf{A}^n_X$ of $Y$ into affine $n$-space over $X$.
See the slightly more general
Morphisms, Lemma \ref{morphisms-lemma-quasi-affine-finite-type-over-S}.

\medskip\noindent
Now suppose that $f : Y \to X$ is a morphism of algebraic spaces with
$f$ locally of finite type and $Y$ an affine scheme. Then it is not
true in general that we can find an immersion of $Y$ into affine
$n$-space over $X$.

\medskip\noindent
A first (nasty) counter example is $Y = \Spec(k)$ and
$X = [\mathbf{A}^1_k/\mathbf{Z}]$ where $k$ is a field of characteristic zero
and $\mathbf{Z}$ acts on $\mathbf{A}^1_k$ by translation $(n, t) \mapsto t + n$.
Namely, for any morphism $Y \to \mathbf{A}^n_X$ over $X$ we can pullback to
the covering $\mathbf{A}^1_k$ of $X$ and we get an infinite disjoint union of
$\mathbf{A}^1_k$'s mapping into $\mathbf{A}^{n + 1}_k$ which is not an
immersion.

\medskip\noindent
A second counter example is $Y = \mathbf{A}^1_k \to X = \mathbf{A}^1_k/R$
with $R = \{(t, t)\} \amalg \{(t, -t), t \not = 0\}$. Namely, in
this case the morphism $Y \to \mathbf{A}^n_X$ would be given by some
regular functions $f_1, \ldots, f_n$ on $Y$ and hence the
fibre product of $Y$ with the covering
$\mathbf{A}^{n + 1}_k \to \mathbf{A}^n_X$
would be the scheme
$$
\{(f_1(t), \ldots, f_n(t), t)\} \amalg
\{(f_1(t), \ldots, f_n(t), -t), t \not = 0\}
$$
with obvious morphism to $\mathbf{A}^{n + 1}_k$ which is not an immersion.
Note that this gives a counter example with $X$ quasi-separated.

\begin{lemma}
\label{lemma-cannot-embed-into-affine}
There exists a finite type morphism of algebraic spaces $Y \to X$
with $Y$ affine and $X$ quasi-separated, such that there does not exist
an immersion $Y \to \mathbf{A}^n_X$ over $X$.
\end{lemma}

\begin{proof}
See discussion above.
\end{proof}










\section{Pushforward of quasi-coherent modules}
\label{section-push-quasi-coherent}

\noindent
In Schemes, Lemma \ref{schemes-lemma-push-forward-quasi-coherent}
we proved that $f_*$ transforms quasi-coherent modules into quasi-coherent
modules when $f$ is quasi-compact and quasi-separated. Here are some
examples to show that these conditions are both necessary.

\medskip\noindent
Suppose that $Y = \Spec(A)$ is an affine scheme and that
$X = \coprod_{n \in \mathbf{N}} Y$. We claim that $f_*\mathcal{O}_X$
is not quasi-coherent where $f : X \to Y$ is the obvious morphism.
Namely, for $a \in A$ we have
$$
f_*\mathcal{O}_X(D(a)) = \prod\nolimits_{n \in \mathbf{N}} A_a
$$
Hence, in order for $f_*\mathcal{O}_X$ to be quasi-coherent we would need
$$
\prod\nolimits_{n \in \mathbf{N}} A_a
=
\left(\prod\nolimits_{n \in \mathbf{N}} A\right)_a
$$
for all $a \in A$. This isn't true in general, for example if
$A = \mathbf{Z}$ and $a = 2$, then $(1, 1/2, 1/4, 1/8, \ldots)$
is an element of the left hand side which is not in the right hand side.
Note that $f$ is a non-quasi-compact separated morphism.

\medskip\noindent
Let $k$ be a field. Set
$$
A = k[t, z, x_1, x_2, x_3, \ldots]/(tx_1z, t^2x_2^2z, t^3x_3^3z, \ldots)
$$
Let $Y = \Spec(A)$. Let $V \subset Y$ be the open subscheme
$V = D(x_1) \cup D(x_2) \cup \ldots$. Let $X$ be two copies of $Y$
glued along $V$. Let $f : X \to Y$ be the obvious morphism. Then we
have an exact sequence
$$
0 \to f_*\mathcal{O}_X \to
\mathcal{O}_Y \oplus \mathcal{O}_Y \xrightarrow{(1, -1)} j_*\mathcal{O}_V
$$
where $j : V \to Y$ is the inclusion morphism. Since
$$
A \longrightarrow \prod A_{x_n}
$$
is injective (details omitted) we see that $\Gamma(Y, f_*\mathcal{O}_X) = A$.
On the other hand, the kernel of the map
$$
A_t \longrightarrow \prod A_{tx_n}
$$
is nonzero because it contains the element $z$. Hence
$\Gamma(D(t), f_*\mathcal{O}_X)$ is strictly bigger than
$A_t$ because it contains $(z, 0)$. Thus we see that $f_*\mathcal{O}_X$
is not quasi-coherent. Note that $f$ is quasi-compact but
non-quasi-separated.

\begin{lemma}
\label{lemma-pushforward-quasi-coherent}
Schemes, Lemma \ref{schemes-lemma-push-forward-quasi-coherent}
is sharp in the sense that one can neither drop the assumption
of quasi-compactness nor the assumption of quasi-separatedness.
\end{lemma}

\begin{proof}
See discussion above.
\end{proof}








\section{A nonfinite module with finite free rank 1 stalks}
\label{section-nonfree}

\noindent
Let $R = \mathbf{Q}[x]$. Set
$M = \sum_{n \in \mathbf{N}} \frac{1}{x - n}R$ as a submodule of
the fraction field of $R$. Then $M$ is not finitely generated, but
for every prime $\mathfrak p$ of $R$ we have
$M_{\mathfrak p} \cong R_{\mathfrak p}$ as an
$R_{\mathfrak p}$-module.

\medskip\noindent
An example of a similar flavor is $R = \mathbf{Z}$ and
$M = \sum_{p \text{ prime}} \frac{1}{p} \mathbf{Z} \subset \mathbf{Q}$,
which equals the set of fractions $\frac{a}{b}$ with $b$ nonzero and
squarefree.








\section{A noninvertible ideal invertible in stalks}
\label{section-locally-invertible-not-invertible}

\noindent
Let $A$ be a domain and let $I \subset A$ be a nonzero ideal.
Recall that when we say $I$ is invertible, we mean that
$I$ is invertible as an $A$-module.
We are going to make an example of this situation where
$I$ is not invertible, yet $I_\mathfrak q = (f) \subset A_\mathfrak q$
is a (nonzero) principal ideal for every prime ideal $\mathfrak q \subset A$.
In the literature the property that $I_\mathfrak q$ is principal
for all primes $\mathfrak q$ is sometimes expressed by
saying ``$I$ is a locally principal ideal''. We can't use
this terminology as our ``local'' always means
``local in the Zariski topology'' (or whatever topology we are
currently working with).

\medskip\noindent
Let $R = \mathbf{Q}[x]$ and let $M = \sum \frac{1}{x - n}R$
be the module constructed in Section \ref{section-nonfree}.
Consider the ring\footnote{The ring $A$ is an example
of a non-Noetherian domain whose local rings are Noetherian.}
$$
A = \text{Sym}^*_R(M)
$$
and the ideal $I = M A = \bigoplus_{d \geq 1} \text{Sym}^d_R(M)$.
Since $M$ is not finitely generated as an $R$-module we see that
$I$ cannot be generated by finitely many elements as an ideal in $A$.
Since an invertible module is finitely generated, this means that
$I$ is not invertible.
On the other hand, let $\mathfrak p \subset R$ be a prime ideal.
By construction $M_\mathfrak p \cong R_\mathfrak p$. Hence
$$
A_\mathfrak p = \text{Sym}^*_{R_\mathfrak p}(M_\mathfrak p) \cong
\text{Sym}^*_{R_\mathfrak p}(R_\mathfrak p) =
R_\mathfrak p[T]
$$
as a graded $R_\mathfrak p$-algebra. It follows that
$I_\mathfrak p \subset A_\mathfrak p$ is generated by
the nonzerodivisor $T$.
Thus certainly for any prime ideal $\mathfrak q \subset A$
we see that $I_\mathfrak q$ is generated by a single element.

\begin{lemma}
\label{lemma-locally-principal-not-invertible}
There exists a domain $A$ and a nonzero ideal $I \subset A$
such that $I_\mathfrak q \subset A_\mathfrak q$ is a principal
ideal for all primes $\mathfrak q \subset A$ but $I$ is not an invertible
$A$-module.
\end{lemma}

\begin{proof}
See discussion above.
\end{proof}









\section{A finite flat module which is not projective}
\label{section-finite-flat-not-projective}

\noindent
This is a copy of
Algebra, Remark \ref{algebra-remark-warning}.
It is not true that a finite $R$-module which is
$R$-flat is automatically projective. A counter
example is where $R = \mathcal{C}^\infty(\mathbf{R})$
is the ring of infinitely differentiable functions on
$\mathbf{R}$, and $M = R_{\mathfrak m} = R/I$ where
$\mathfrak m = \{f \in R \mid f(0) = 0\}$ and
$I = \{f \in R \mid \exists \epsilon, \epsilon > 0 :
f(x) = 0\ \forall x, |x| < \epsilon\}$.

\medskip\noindent
The morphism $\Spec(R/I) \to \Spec(R)$ is also
an example of a flat closed immersion which is not open.

\begin{lemma}
\label{lemma-finite-flat-non-projective}
Strange flat modules.
\begin{enumerate}
\item There exists a ring $R$ and a finite flat $R$-module $M$ which is
not projective.
\item There exists a closed immersion which is flat but not open.
\end{enumerate}
\end{lemma}

\begin{proof}
See discussion above.
\end{proof}



\section{A projective module which is not locally free}
\label{section-projective-not-locally-free}

\noindent
We give two examples. One where the rank is between $0$ and $1$
and one where the rank is $\aleph_0$.

\begin{lemma}
\label{lemma-ideal-generated-by-idempotents-projective}
Let $R$ be a ring. Let $I \subset R$ be an ideal generated by
a countable collection of idempotents. Then $I$ is projective
as an $R$-module.
\end{lemma}

\begin{proof}
Say $I = (e_1, e_2, e_3, \ldots)$ with $e_n$ an idempotent of $R$.
After inductively replacing $e_{n + 1}$ by $e_n + (1 - e_n)e_{n + 1}$
we may assume that $(e_1) \subset (e_2) \subset (e_3) \subset \ldots$
and hence $I = \bigcup_{n \geq 1} (e_n) = \colim_n e_nR$.
In this case
$$
\Hom_R(I, M) = \Hom_R(\colim_n e_nR, M)
= \lim_n \Hom_R(e_nR, M) = \lim_n e_nM
$$
Note that the transition maps $e_{n + 1}M \to e_nM$ are given
by multiplication by $e_n$ and are surjective. Hence by
Algebra, Lemma \ref{algebra-lemma-ML-exact-sequence}
the functor $\Hom_R(I, M)$ is exact, i.e., $I$ is a projective
$R$-module.
\end{proof}

\noindent
Suppose that $P \subset Q$ is an inclusion of $R$-modules with $Q$ a
finite $R$-module and $P$ locally free, see
Algebra, Definition \ref{algebra-definition-locally-free}.
Suppose that $Q$ can be generated by $N$ elements as an $R$-module.
Then it follows from
Algebra, Lemma \ref{algebra-lemma-map-cannot-be-injective}
that $P$ is finite locally free (with the free parts having rank
at most $N$). And in this case $P$ is a finite $R$-module, see
Algebra, Lemma \ref{algebra-lemma-finite-projective}.

\medskip\noindent
Combining this with the above we see that a non-finitely-generated
ideal which is generated by a countable collection of idempotents
is projective but not locally free. An explicit example is
$R = \prod_{n \in \mathbf{N}} \mathbf{F}_2$ and
$I$ the ideal generated by the idempotents
$$
e_n = (1, 1, \ldots, 1, 0, \ldots )
$$
where the sequence of $1$'s has length $n$.

\begin{lemma}
\label{lemma-ideal-projective-not-locally-free}
There exists a ring $R$ and an ideal $I$ such that $I$ is projective as
an $R$-module but not locally free as an $R$-module.
\end{lemma}

\begin{proof}
See above.
\end{proof}

\begin{lemma}
\label{lemma-chow-group-product}
Let $K$ be a field.
Let $C_i$, $i = 1, \ldots, n$ be smooth, projective, geometrically irreducible
curves over $K$. Let $P_i \in C_i(K)$ be a rational point and
let $Q_i \in C_i$ be a point such that $[\kappa(Q_i) : K] = 2$.
Then $[P_1 \times \ldots \times P_n]$ is nonzero in
$\CH_0(U_1 \times_K \ldots \times_K U_n)$ where $U_i = C_i \setminus \{Q_i\}$.
\end{lemma}

\begin{proof}
There is a degree map
$\deg : \CH_0(C_1 \times_K \ldots \times_K C_n) \to \mathbf{Z}$
Because each $Q_i$ has degree $2$ over $K$ we see that
any zero cycle supported on the ``boundary''
$$
C_1 \times_K \ldots \times_K C_n
\setminus
U_1 \times_K \ldots \times_K U_n
$$
has degree divisible by $2$.
\end{proof}

\noindent
We can construct another example of a projective but not locally free
module using the lemma above as follows. Let
$C_n$, $n = 1, 2, 3, \ldots$ be smooth, projective, geometrically irreducible
curves over $\mathbf{Q}$ each with a pair of points
$P_n, Q_n \in C_n$ such that $\kappa(P_n) = \mathbf{Q}$ and
$\kappa(Q_n)$ is a quadratic extension of $\mathbf{Q}$.
Set $U_n = C_n \setminus \{Q_n\}$; this is an affine curve.
Let $\mathcal{L}_n$ be the inverse of the ideal sheaf of $P_n$
on $U_n$. Note that $c_1(\mathcal{L}_n) = [P_n]$ in the group of
zero cycles $\CH_0(U_n)$. Set $A_n = \Gamma(U_n, \mathcal{O}_{U_n})$.
Let $L_n = \Gamma(U_n, \mathcal{L}_n)$ which is a locally
free module of rank $1$ over $A_n$. Set
$$
B_n = A_1 \otimes_{\mathbf{Q}} A_2 \otimes_{\mathbf{Q}} \ldots
\otimes_{\mathbf{Q}} A_n
$$
so that $\Spec(B_n) = U_1 \times \ldots \times U_n$ all products
over $\Spec(\mathbf{Q})$. For $i \leq n$ we set
$$
L_{n, i} =
A_1 \otimes_{\mathbf{Q}} \ldots \otimes_{\mathbf{Q}} L_i
\otimes_{\mathbf{Q}} \ldots \otimes_{\mathbf{Q}} A_n
$$
which is a locally free $B_n$-module of rank $1$. Note that this is
also the global sections of $\text{pr}_i^*\mathcal{L}_n$. Set
$$
B_\infty = \colim_n B_n
\quad\text{and}\quad
L_{\infty, i} = \colim_n L_{n, i}
$$
Finally, set
$$
M = \bigoplus\nolimits_{i \geq 1} L_{\infty, i}.
$$
This is a direct sum of finite locally free modules, hence projective.
We claim that $M$ is not locally free. Namely, suppose that
$f \in B_\infty$ is a nonzero function such that $M_f$ is free
over $(B_\infty)_f$. Let $e_1, e_2, \ldots$ be a basis. Choose
$n \geq 1$ such that $f \in B_n$.
Choose $m \geq n + 1$ such that $e_1, \ldots, e_{n + 1}$ are in
$$
\bigoplus\nolimits_{1 \leq i \leq m} L_{m, i}.
$$
Because the elements $e_1, \ldots, e_{n + 1}$ are part of a basis
after a faithfully flat base change we conclude that
the Chern classes
$$
c_i(\text{pr}_1^*\mathcal{L}_1 \oplus \ldots \oplus
\text{pr}_m^*\mathcal{L}_m), \quad i = m, m - 1, \ldots, m - n
$$
are zero in the chow group of
$$
D(f) \subset U_1 \times \ldots \times U_m
$$
Since $f$ is the pullback of a function on $U_1 \times \ldots \times U_n$
this implies in particular that
$$
c_{m - n}(\mathcal{O}_W^{\oplus n} \oplus
\text{pr}_1^*\mathcal{L}_{n + 1} \oplus \ldots
\oplus \text{pr}_{m - n}^*\mathcal{L}_m) = 0.
$$
on the variety
$$
W = (C_{n + 1} \times \ldots \times C_m)_K
$$
over the field $K = \mathbf{Q}(C_1 \times \ldots \times C_n)$.
In other words the cycle
$$
[(P_{n + 1} \times \ldots \times P_m)_K]
$$
is zero in the chow group of zero cycles on $W$. This contradicts
Lemma \ref{lemma-chow-group-product}
above because the points $Q_i$, $n + 1 \leq i \leq m$
induce corresponding points $Q_i'$ on $(C_n)_K$ and as $K/\mathbf{Q}$
is geometrically irreducible we have $[\kappa(Q_i') : K] = 2$.

\begin{lemma}
\label{lemma-projective-not-locally-free}
There exists a countable ring $R$ and a projective module $M$
which is a direct sum of countably many locally free rank $1$
modules such that $M$ is not locally free.
\end{lemma}

\begin{proof}
See above.
\end{proof}







\section{Zero dimensional local ring with nonzero flat ideal}
\label{section-zero-dimensional-flat-ideal}

\noindent
In \cite{Lazard} and \cite{Autour}
there is an example of a zero dimensional local ring with a
nonzero flat ideal. Here is the construction. Let $k$ be a field.
Let $X_i, Y_i$, $i \geq 1$ be variables. Take
$R = k[X_i, Y_i]/(X_i - Y_i X_{i + 1}, Y_i^2)$. Denote $x_i$, resp.\ $y_i$
the image of $X_i$, resp.\ $Y_i$ in this ring. Note that
$$
x_i = y_i x_{i + 1} = y_i y_{i +1} x_{i + 2} =
y_i y_{i + 1} y_{i + 2} x_{i + 3} = \ldots
$$
in this ring. The ring $R$ has only
one prime ideal, namely $\mathfrak m = (x_i, y_i)$. We claim that
the ideal $I = (x_i)$ is flat as an $R$-module.

\medskip\noindent
Note that the annihilator of $x_i$ in $R$ is the ideal
$(x_1, x_2, x_3, \ldots, y_i, y_{i + 1}, y_{i + 2}, \ldots)$.
Consider the $R$-module $M$ generated by elements $e_i$, $i \geq 1$ and
relations $e_i = y_i e_{i + 1}$. Then $M$ is flat as it is the
colimit $\colim_i R$ of copies of $R$ with transition maps
$$
R \xrightarrow{y_1} R \xrightarrow{y_2} R \xrightarrow{y_3} \ldots
$$
Note that the annihilator of $e_i$ in $M$ is the ideal
$(x_1, x_2, x_3, \ldots, y_i, y_{i + 1}, y_{i + 2}, \ldots)$.
Since every element of $M$, resp.\ $I$ can be written as
$f e_i$, resp.\ $h x_i$ for some $f, h \in R$ we see that the
map $M \to I$, $e_i \to x_i$ is an isomorphism and $I$ is flat.

\begin{lemma}
\label{lemma-zero-dimensional-flat-ideal}
\begin{slogan}
Zero dimensional ring with flat ideal.
\end{slogan}
There exists a local ring $R$ with a unique prime ideal
and a nonzero ideal $I \subset R$ which is a flat $R$-module
\end{lemma}

\begin{proof}
See discussion above.
\end{proof}








\section{An epimorphism of zero-dimensional rings which is not surjective}
\label{section-epimorphism-not-surjective}

\noindent
In \cite{Lazard-deux} and \cite{Autour} one can find the following example.
Let $k$ be a field. Consider the ring homomorphism
$$
k[x_1, x_2, \ldots, z_1, z_2, \ldots]/
(x_i^{4^i}, z_i^{4^i})
\longrightarrow
k[x_1, x_2, \ldots, y_1, y_2, \ldots]/
(x_i^{4^i}, y_i - x_{i + 1}y_{i + 1}^2)
$$
which maps $x_i$ to $x_i$ and $z_i$ to $x_iy_i$. Note that $y_i^{4^{i + 1}}$
is zero in the right hand side but that $y_1$ is not zero (details omitted).
This map is not surjective: we can think of the above as a map of
$\mathbf{Z}$-graded algebras by setting $\deg(x_i) = -1$, $\deg(z_i) = 0$,
and $\deg(y_i) = 1$ and then it is clear that $y_1$ is not in the image.
Finally, the map is an epimorphism because
$$
y_{i - 1} \otimes 1 = x_i y_i^2 \otimes 1 = y_i \otimes x_i y_i =
x_i y_i \otimes y_i = 1 \otimes x_i y_i^2.
$$
hence the tensor product of the target over the source is isomorphic
to the target.

\begin{lemma}
\label{lemma-epi-not-surjective}
There exists an epimorphism of local rings of dimension $0$
which is not a surjection.
\end{lemma}

\begin{proof}
See discussion above.
\end{proof}



\section{Finite type, not finitely presented, flat at prime}
\label{section-ft-not-fp-flat-at-prime}

\noindent
Let $k$ be a field. Consider the local ring $A_0 = k[x, y]_{(x, y)}$.
Denote $\mathfrak p_{0, n} = (y + x^n + x^{2n + 1})$. This is a prime ideal.
Set
$$
A = A_0[z_1, z_2, z_3, \ldots]/(z_n z_m, z_n(y + x^n + x^{2n + 1}))
$$
Note that $A \to A_0$ is a surjection whose kernel is an ideal of
square zero. Hence $A$ is also a local ring and the prime ideals of $A$
are in one-to-one correspondence with the prime ideals of $A_0$.
Denote $\mathfrak p_n$ the prime ideal of $A$ corresponding to
$\mathfrak p_{0, n}$. Observe that $\mathfrak p_n$ is the annihilator
of $z_n$ in $A$. Let
$$
C = A[z]/(xz^2 + z + y)[\frac{1}{2zx + 1}]
$$
Note that $A \to C$ is an \'etale ring map, see
Algebra, Example \ref{algebra-example-make-standard-smooth}.
Let $\mathfrak q \subset C$ be the maximal ideal generated by
$x$, $y$, $z$ and all $z_n$. As $A \to C$ is flat we see that the
annihilator of $z_n$ in $C$ is $\mathfrak p_nC$. We compute
\begin{align*}
C/\mathfrak p_n C
& =
A_0[z]/(xz^2 + z + y, y + x^n + x^{2n + 1})[1/(2zx + 1)] \\
& =
k[x]_{(x)}[z]/(xz^2 + z - x^n - x^{2n + 1})[1/(2zx + 1)] \\
& =
k[x]_{(x)}[z]/(z - x^n) \times
k[x]_{(x)}[z]/(xz + x^{n + 1} + 1)[1/(2zx + 1)] \\
& =
k[x]_{(x)} \times k(x)
\end{align*}
because $(z - x^n)(xz + x^{n + 1} + 1) = xz^2 + z - x^n - x^{2n + 1}$.
Hence we see that $\mathfrak p_nC = \mathfrak r_n \cap \mathfrak q_n$
with $\mathfrak r_n = \mathfrak p_nC + (z - x^n)C$ and
$\mathfrak q_n = \mathfrak p_nC + (xz + x^{n + 1} + 1)C$.
Since $\mathfrak q_n + \mathfrak r_n = C$ we also get
$\mathfrak p_nC = \mathfrak r_n \mathfrak q_n$.
It follows that $\mathfrak q_n$ is the annihilator of $\xi_n = (z - x^n)z_n$.
Observe that on the one hand $\mathfrak r_n \subset \mathfrak q$, and
on the other hand $\mathfrak q_n + \mathfrak q = C$. This follows for example
because $\mathfrak q_n$ is a maximal ideal of $C$ distinct from $\mathfrak q$.
Similarly we have $\mathfrak q_n + \mathfrak q_m = C$ for $n \not = m$.
At this point we let
$$
B = \Im(C \longrightarrow C_{\mathfrak q})
$$
We observe that the elements $\xi_n$ map to zero in $B$ as $xz + x^{n + 1} + 1$
is not in $\mathfrak q$. Denote $\mathfrak q' \subset B$ the image of
$\mathfrak q$. By construction $B$ is a finite type $A$-algebra, with
$B_{\mathfrak q'} \cong C_{\mathfrak q}$. In particular we see that
$B_{\mathfrak q'}$ is flat over $A$.

\medskip\noindent
We claim there does not exist an element $g' \in B$, $g' \not \in \mathfrak q'$
such that $B_{g'}$ is of finite presentation over $A$. We sketch a proof of
this claim. Choose an element $g \in C$ which maps to $g' \in B$.
Consider the map $C_g \to B_{g'}$. By
Algebra, Lemma \ref{algebra-lemma-finite-presentation-independent}
we see that $B_g$ is finitely presented over $A$ if and only if the kernel
of $C_g \to B_{g'}$ is finitely generated. But the element $g \in C$ is
not contained in $\mathfrak q$, hence maps to a nonzero element of
$A_0[z]/(xz^2 + z + y)$. Hence $g$ can only be contained in finitely
many of the prime ideals $\mathfrak q_n$, because the primes
$(y + x^n + x^{2n + 1}, xz + x^{n + 1} + 1)$ are an infinite collection
of codimension 1 points of the 2-dimensional irreducible Noetherian space
$\Spec(k[x, y, z]/(xz^2 + z + y))$. The map
$$
\bigoplus\nolimits_{g \not \in \mathfrak q_n} C/\mathfrak q_n
\longrightarrow
C_g, \quad
(c_n) \longrightarrow \sum c_n \xi_n
$$
is injective and its image is the kernel of $C_g \to B_{g'}$. We omit the
proof of this statement. (Hint: Write $A = A_0 \oplus I$ as an $A_0$-module
where $I$ is the kernel of $A \to A_0$. Similarly, write $C = C_0 \oplus IC$.
Write
$IC = \bigoplus Cz_n \cong \bigoplus (C/\mathfrak r_n \oplus C/\mathfrak q_n)$
and study the effect of multiplication by $g$ on the summands.)
This concludes the sketch of the proof of the claim.
This also proves that $B_{g'}$ is not flat over $A$ for any $g'$ as above.
Namely, if it were flat, then the annihilator of the image of $z_n$ in
$B_{g'}$ would be $\mathfrak p_nB_{g'}$, and would not contain $z - x^n$.

\medskip\noindent
As a consequence we can answer (negatively) a question posed in
\cite[Part I, Remarques (3.4.7) (\romannumeral 5)]{GruRay}.
Here is a precise statement.

\begin{lemma}
\label{lemma-example-raynaud-gruson}
There exists a local ring $A$, a finite type ring map $A \to B$ and a prime
$\mathfrak q$ lying over $\mathfrak m_A$ such that $B_{\mathfrak q}$ is flat
over $A$, and for any element $g \in B$, $g \not \in \mathfrak q$
the ring $B_g$ is neither finitely presented over $A$ nor flat over $A$.
\end{lemma}

\begin{proof}
See discussion above.
\end{proof}


\section{Finite type, flat and not of finite presentation}
\label{section-finite-type-flat-not-finite-presentation}

\noindent
In this section we give some examples of ring maps and morphisms
which are of finite type and flat but not of finite presentation.

\medskip\noindent
Let $R$ be a ring which has an ideal $I$ such that $R/I$ is a finite
flat module but not projective, see
Section \ref{section-finite-flat-not-projective}
for an explicit example. Note that this means that $I$ is not
finitely generated, see
Algebra, Lemma \ref{algebra-lemma-finitely-generated-pure-ideal}.
Note that $I = I^2$, see
Algebra, Lemma \ref{algebra-lemma-pure}.
The base ring in our examples will be
$R$ and correspondingly the base scheme $S = \Spec(R)$.

\medskip\noindent
Consider the ring map $R \to R \oplus R/I\epsilon$ where
$\epsilon^2 = 0$ by convention. This is a finite, flat
ring map which is not of finite presentation. All the fibre rings
are complete intersections and geometrically irreducible.

\medskip\noindent
Let $A = R[x, y]/(xy, ay; a \in I)$. Note that as an $R$-module
we have $A = \bigoplus_{i \geq 0} Ry^i \oplus \bigoplus_{j > 0} R/Ix^j$.
Hence $R \to A$ is a flat finite type ring map which is not of finite
presentation. Each fibre ring
is isomorphic to either $\kappa(\mathfrak p)[x, y]/(xy)$
or $\kappa(\mathfrak p)[x]$.

\medskip\noindent
We can turn the previous example into a projective morphism by
taking $B = R[X_0, X_1, X_2]/(X_1X_2, aX_2; a \in I)$.
In this case $X = \text{Proj}(B) \to S$ is a proper flat morphism
which is not of finite presentation such that for each $s \in S$
the fibre $X_s$ is isomorphic either to $\mathbf{P}^1_s$ or to
the closed subscheme of $\mathbf{P}^2_s$ defined by the vanishing of
$X_1X_2$ (this is a projective nodal curve of arithmetic genus $0$).

\medskip\noindent
Let $M = R \oplus R \oplus R/I$. Set $B = \text{Sym}_R(M)$ the
symmetric algebra on $M$. Set $X = \text{Proj}(B)$.
Then $X \to S$ is a proper flat morphism, not of finite presentation
such that for $s \in S$ the geometric fibre is isomorphic to either
$\mathbf{P}^1_s$ or $\mathbf{P}^2_s$. In particular these fibres are
smooth and geometrically irreducible.

\begin{lemma}
\label{lemma-finite-type-flat-not-finitely-presented}
There exist examples of
\begin{enumerate}
\item a flat finite type ring map with geometrically irreducible
complete intersection fibre rings which is not of finite presentation,
\item a flat finite type ring map with geometrically connected,
geometrically reduced, dimension 1, complete intersection fibre rings
which is not of finite presentation,
\item a proper flat morphism of schemes $X \to S$ each of whose fibres
is isomorphic to either $\mathbf{P}^1_s$ or to the vanishing locus of
$X_1X_2$ in $\mathbf{P}^2_s$ which is not of finite presentation, and
\item a proper flat morphism of schemes $X \to S$ each of whose
fibres is isomorphic to either $\mathbf{P}^1_s$ or $\mathbf{P}^2_s$
which is not of finite presentation.
\end{enumerate}
\end{lemma}

\begin{proof}
See discussion above.
\end{proof}



\section{Topology of a finite type ring map}
\label{section-topology-finite-type}

\noindent
Let $A \to B$ be a local map of local domains.
If $A$ is Noetherian, $A \to B$ is essentially of finite type,
and $A/\mathfrak m_A \subset B/\mathfrak m_B$ is finite then there
exists a prime $\mathfrak q \subset B$, $\mathfrak q \not = \mathfrak m_B$
such that $A \to B/\mathfrak q$ is the localization of a quasi-finite ring
map. See
More on Morphisms, Lemma
\ref{more-morphisms-lemma-quasi-finite-quasi-section-meeting-nearby-open}.

\medskip\noindent
In this section we give an example that shows this result is false $A$
is no longer Noetherian. Namely, let $k$ be a field and set
$$
A = \{a_0 + a_1 x + a_2 x^2 + \ldots \mid a_0 \in k, a_i \in k((y))
\text{ for }i\geq 1\}
$$
and
$$
C = \{a_0 + a_1 x + a_2 x^2 + \ldots \mid a_0 \in k[y], a_i \in k((y))
\text{ for }i\geq 1\}.
$$
The inclusion $A \to C$ is of finite type as $C$ is generated by $y$
over $A$. We claim that $A$ is a local ring with maximal ideal
$\mathfrak m = \{a_1 x + a_2 x^2 + \ldots \in A\}$ and no prime
ideals besides $(0)$ and $\mathfrak m$. Namely, an element
$f = a_0 + a_1 x + a_2 x^2 + \ldots$ of $A$ is invertible as soon as
$a_0 \not = 0$. If $\mathfrak q \subset A$ is a nonzero prime ideal,
and $f = a_i x^i + \ldots \in \mathfrak q$, then using properties of
power series one sees that for any $g \in k((y))$ the element
$g^{i + 1} x^{i + 1} \in \mathfrak q$, i.e., $gx \in \mathfrak q$.
This proves that $\mathfrak q = \mathfrak m$.

\medskip\noindent
As to the spectrum of the ring $C$, arguing in the same way as above
we see that any nonzero prime ideal contains the prime
$\mathfrak p = \{a_1 x + a_2 x^2 + \ldots \in C\}$ which lies
over $\mathfrak m$. Thus the only prime of $C$ which lies over
$(0)$ is $(0)$. Set $\mathfrak m_C = yC + \mathfrak p$ and
$B = C_{\mathfrak m_C}$. Then $A \to B$ is the desired example.

\begin{lemma}
\label{lemma-topology-finite-type}
There exists a local homomorphism $A \to B$ of local domains which is
essentially of finite type and such that $A/\mathfrak m_A \to B/\mathfrak m_B$
is finite such that for every prime
$\mathfrak q \not = \mathfrak m_B$ of $B$ the ring map
$A \to B/\mathfrak q$ is not the localization of a quasi-finite ring map.
\end{lemma}

\begin{proof}
See the discussion above.
\end{proof}



\section{Pure not universally pure}
\label{section-pure-not-universally}

\noindent
Let $k$ be a field. Let
$$
R = k[[x, xy, xy^2, \ldots]] \subset k[[x, y]].
$$
In other words, a power series $f \in k[[x, y]]$ is in $R$ if and only
if $f(0, y)$ is a constant. In particular $R[1/x] = k[[x, y]][1/x]$
and $R/xR$ is a local ring with a maximal ideal whose square
is zero. Denote $R[y] \subset k[[x, y]]$ the set of power series
$f \in k[[x, y]]$ such that $f(0, y)$ is a polynomial in $y$. Then
$R \to R[y]$ is a finite type but not finitely presented ring map
which induces an isomorphism after inverting $x$. Also there is a
surjection $R[y]/xR[y] \to k[y]$ whose kernel has square zero. Consider
the finitely presented ring map $R \to S = R[t]/(xt - xy)$.
Again $R[1/x] \to S[1/x]$ is an isomorphism and in this case
$S/xS \cong (R/xR)[t]/(xy)$ maps onto $k[t]$ with nilpotent kernel.
There is a surjection $S \to R[y]$, $t \longmapsto y$ which induces
an isomorphism on inverting $x$ and a surjection with nilpotent kernel
modulo $x$. Hence the kernel of $S \to R[y]$ is locally nilpotent.
In particular $S \to R[y]$ induces a universal homeomorphism on spectra.

\medskip\noindent
First we claim that $S$ is an $S$-module which is relatively pure
over $R$. Since on inverting $x$ we obtain an isomorphism we only need
to check this at the maximal ideal $\mathfrak m \subset R$. Since
$R$ is complete with respect to its maximal ideal it is henselian
hence we need only check that every prime $\mathfrak p \subset R$,
$\mathfrak p \not = \mathfrak m$, the unique prime $\mathfrak q$
of $S$ lying over $\mathfrak p$ satisfies
$\mathfrak mS + \mathfrak q \not = S$. Since $\mathfrak p \not = \mathfrak m$
it corresponds to a unique prime ideal of $k[[x, y]][1/x]$. Hence
either $\mathfrak p = (0)$ or $\mathfrak p = (f)$ for some
irreducible element $f \in k[[x, y]]$ which is not associated to $x$
(here we use that $k[[x, y]]$ is a UFD -- insert future reference here).
In the first case $\mathfrak q = (0)$ and the result is clear. In the
second case we may multiply $f$ by a unit so that $f \in R[y]$
(Weierstrass preparation; details omitted). Then it is easy to see that
$R[y]/fR[y] \cong k[[x, y]]/(f)$ hence $f$ defines a prime ideal
of $R[y]$ and $\mathfrak mR[y] + fR[y] \not = R[y]$.
Since $S \to R[y]$ induces a universal homeomorphism on spectra we deduce the
desired result for $S$ also.

\medskip\noindent
Second we claim that $S$ is not universally relatively pure over $R$.
Namely, to see this it suffices to find a valuation ring
$\mathcal{O}$ and a local ring map $R \to \mathcal{O}$ such that
$\Spec(R[y] \otimes_R \mathcal{O}) \to \Spec(\mathcal{O})$
does not hit the closed point of $\Spec(\mathcal{O})$.
Equivalently, we have to find $\varphi : R \to \mathcal{O}$ such that
$\varphi(x) \not = 0$ and $v(\varphi(x)) > v(\varphi(xy))$ where $v$
is the valuation of $\mathcal{O}$.
(Because this means that the valuation of $y$ is negative.)
To do this consider the ring map
$$
R
\longrightarrow
\{a_0 + a_1 x + a_2 x^2 + \ldots \mid a_0 \in k[y^{-1}], a_i \in k((y))\}
$$
defined in the obvious way. We can find a valuation ring $\mathcal{O}$
dominating the localization of the right hand side at the maximal
ideal $(y^{-1}, x)$ and we win.

\begin{lemma}
\label{lemma-pure-not-universally-pure}
There exists a morphism of affine schemes of finite presentation
$X \to S$ and an $\mathcal{O}_X$-module $\mathcal{F}$ of finite presentation
such that $\mathcal{F}$ is pure relative to $S$, but not universally
pure relative to $S$.
\end{lemma}

\begin{proof}
See discussion above.
\end{proof}



\section{A formally smooth non-flat ring map}
\label{section-formally-smooth-nonflat}

\noindent
Let $k$ be a field. Consider the $k$-algebra $k[\mathbf{Q}]$.
This is the $k$-algebra with basis $x_\alpha, \alpha \in \mathbf{Q}$
and multiplication determined by $x_\alpha x_\beta = x_{\alpha + \beta}$.
(In particular $x_0 = 1$.)
Consider the $k$-algebra homomorphism
$$
k[\mathbf{Q}] \longrightarrow k, \quad
x_\alpha \longmapsto 1.
$$
It is surjective with kernel $J$ generated by the elements $x_\alpha - 1$.
Let us compute $J/J^2$. Note that multiplication by $x_\alpha$ on $J/J^2$
is the identity map. Denote $z_\alpha$ the class of $x_\alpha - 1$ modulo
$J^2$. These classes generate $J/J^2$. Since
$$
(x_\alpha - 1)(x_\beta - 1) = x_{\alpha + \beta} - x_\alpha - x_\beta + 1 =
(x_{\alpha + \beta} - 1) - (x_\alpha - 1) - (x_\beta - 1)
$$
we see that $z_{\alpha + \beta} = z_\alpha + z_\beta$ in $J/J^2$.
A general element of $J/J^2$ is of the form $\sum \lambda_\alpha z_\alpha$
with $\lambda_\alpha \in k$ (only finitely many nonzero). Note that if the
characteristic of $k$ is $p > 0$ then
$$
0 = pz_{\alpha/p} = z_{\alpha/p} + \ldots + z_{\alpha/p} = z_\alpha
$$
and we see that $J/J^2 = 0$. If the characteristic of $k$ is zero, then
$$
J/J^2 = \mathbf{Q} \otimes_{\mathbf{Z}} k \cong k
$$
(details omitted) is not zero.

\medskip\noindent
We claim that $k[\mathbf{Q}] \to k$ is a formally smooth ring map
if the characteristic of $k$ is positive. Namely, suppose given a
solid commutative diagram
$$
\xymatrix{
k \ar[r] \ar@{..>}[rd] & A \\
k[\mathbf{Q}] \ar[u] \ar[r]^\varphi & A' \ar[u]
}
$$
with $A' \to A$ a surjection whose kernel $I$ has square zero.
To show that $k[\mathbf{Q}] \to k$ is formally smooth we have to prove
that $\varphi$ factors through $k$. Since $\varphi(x_\alpha - 1)$ maps
to zero in $A$ we see that $\varphi$ induces a map
$\overline{\varphi} : J/J^2 \to I$ whose vanishing is the obstruction
to the desired factorization. Since $J/J^2 = 0$ if the characteristic
is $p > 0$ we get the result we want, i.e., $k[\mathbf{Q}] \to k$ is
formally smooth in this case. Finally, this ring map is not flat, for
example as the nonzerodivisor $x_2 - 1$ is mapped to zero.

\begin{lemma}
\label{lemma-formally-smooth-nonflat}
There exists a formally smooth ring map which is not flat.
\end{lemma}

\begin{proof}
See discussion above.
\end{proof}


\section{A formally \'etale non-flat ring map}
\label{section-formally-etale-nonflat}

\noindent
In this section we give a counterexample to the final sentence in
\cite[0, Example 19.10.3(i)]{EGA} (this was not one of the items caught
in their later errata lists). Consider $A \to A/J$ for a local ring $A$
and a nonzero proper ideal $J$ such that $J^2 = J$ (so $J$ isn't finitely
generated); the valuation ring of an algebraically closed non-archimedean
field with $J$ its maximal ideal is a source of such $(A, J)$. These
non-flat quotient maps are formally \'etale. Namely, suppose given a
commutative diagram
$$
\xymatrix{
A/J \ar[r] & R/I \\
A \ar[u] \ar[r]^\varphi & R \ar[u]
}
$$
where $I$ is an ideal of the ring $R$ with $I^2 = 0$. Then $A \to R$
factors uniquely through $A/J$ because
$$
\varphi(J) = \varphi(J^2) \subset (\varphi(J)A)^2 \subset I^2 = 0.
$$
Hence this also provides a counterexample to the formally \'etale case
of the ``structure theorem'' for locally finite type and formally \'etale
morphisms in \cite[IV, Theorem 18.4.6(i)]{EGA}
(but not a counterexample to part (ii), which
is what people actually use in practice). The error in the proof of
the latter is that the very last step of the proof is to invoke the incorrect
\cite[0, Example 19.3.10(i)]{EGA},
which is how the counterexample just mentioned creeps in.

\begin{lemma}
\label{lemma-formally-etale-not-presented}
There exist formally \'etale nonflat ring maps.
\end{lemma}

\begin{proof}
See discussion above.
\end{proof}



\section{A formally \'etale ring map with nontrivial cotangent complex}
\label{section-formally-etale-nontrivial-cotangent-complex}

\noindent
Let $k$ be a field. Consider the ring
$$
R = k[\{x_n\}_{n \geq 1}, \{y_n\}_{n \geq 1}]/(
x_1y_1, x_{nm}^m - x_n, y_{nm}^m - y_n)
$$
Let $A$ be the localization at the maximal ideal generated by
all $x_n, y_n$ and denote $J \subset A$ the maximal ideal. Set $B = A/J$.
By construction $J^2 = J$ and hence $A \to B$ is formally \'etale (see
Section \ref{section-formally-etale-nonflat}).
We claim that the element $x_1 \otimes y_1$ is a nonzero
element in the kernel of
$$
J \otimes_A J \longrightarrow J.
$$
Namely, $(A, J)$ is the colimit of the localizations
$(A_n, J_n)$ of the rings
$$
R_n = k[x_n, y_n]/(x_n^n y_n^n)
$$
at their corresponding maximal ideals. Then
$x_1 \otimes y_1$ corresponds to the element
$x_n^n \otimes y_n^n \in J_n \otimes_{A_n} J_n$ and is nonzero
(by an explicit computation which we omit). Since $\otimes$ commutes
with colimits we conclude. By
\cite[III Section 3.3]{cotangent}
we see that $J$ is not weakly regular. Hence by
\cite[III Proposition 3.3.3]{cotangent}
we see that the cotangent complex $L_{B/A}$ is not zero. In fact, we can
be more precise. We have $H_0(L_{B/A}) = \Omega_{B/A}$ and
$H_1(L_{B/A}) = 0$ because $J/J^2 = 0$. But from the five-term exact sequence
of Quillen's fundamental spectral sequence
(see Cotangent, Remark \ref{cotangent-remark-elucidate-ss} or
\cite[Corollary 8.2.6]{Reinhard})
and the nonvanishing of
$\text{Tor}_2^A(B, B) = \Ker(J \otimes_A J \to J)$ we conclude that
$H_2(L_{B/A})$ is nonzero.

\begin{lemma}
\label{lemma-formally-etale-nontrivial-cotangent-complex}
There exists a formally \'etale surjective ring map $A \to B$
with $L_{B/A}$ not equal to zero.
\end{lemma}

\begin{proof}
See discussion above.
\end{proof}






\section{Flat and formally unramified is not formally \'etale}
\label{section-flat-formally-unramified-not-formally-etale}

\noindent
In More on Morphisms, Lemma
\ref{more-morphisms-lemma-unramified-flat-formally-etale}
it is shown that an unramified flat morphism of schemes $X \to S$
is formally \'etale. The goal of this section is to give two examples
that illustrate that we cannot replace `unramified' by `formally unramified'.
The first example exploits special properties of perfect rings, while the
second example shows the result fails even for maps of Noetherian regular rings.

\begin{lemma}
\label{lemma-perfect-closure-polynomial-ring}
Let $A = \mathbb{F}_p[T]$ be the polynomial ring in one variable over
$\mathbb{F}_p$. Let $A_{perf}$  denote the perfect closure of $A$.
Then $A \rightarrow A_{perf}$ is flat and formally unramified,
but not formally \'etale.
\end{lemma}

\begin{proof}
Note that under the Frobenius map $F_A : A \to A$, the target copy of $A$
is a free-module over the domain with basis $\{1, T, \dots, T^{p - 1}\}$.
Thus, $F_A$ is faithfully flat, and consequently,  so is
$A \to A_{perf}$ since it is a colimit of faithfully flat maps.
Since $A_{perf}$ is a perfect ring, the relative Frobenius
$F_{A_{perf}/A}$ is a surjection. In  other words,
$A_{perf} = A[A_{perf}^p]$, which readily implies
$\Omega_{A_{perf}/A} = 0$. Then
$A \rightarrow A_{perf}$ is formally unramified by
More on Morphisms, Lemma
\ref{more-morphisms-lemma-formally-unramified-differentials}

\medskip\noindent
It suffices to show that $A \rightarrow A_{perf}$
is not formally smooth. Note that since $A$ is a
smooth $\mathbb{F}_p$-algebra, the cotangent complex
$L_{A/\mathbb{F}_P} \simeq \Omega_{A/\mathbb{F}_p}[0]$
is concentrated in degree $0$, see
Cotangent, Lemma \ref{cotangent-lemma-when-projective}. Moreover,
$L_{A_{perf}/\mathbb{F}_p} = 0$ in $D(A_{perf})$
by Cotangent, Lemma \ref{cotangent-lemma-perfect-zero}.
Consider the distinguished triangle of cotangent complexes
$$
L_{A/\mathbb{F}_p} \otimes_A A_{perf} \to
L_{A_{perf}/\mathbb{F}_p} \to
L_{A_{perf}/A} \to
(L_{A/\mathbb{F}_p} \otimes_A A_{perf})[1]
$$
in $D(A_{perf})$, see Cotangent, Section \ref{cotangent-section-triangle}.
We find $L_{A_{perf}/A} = \Omega_{A/\mathbb{F}_p} \otimes_A A_{perf}[1]$,
that is, $L_{A_{perf}/A}$ is equal to a free rank $1$ $A_{perf}$
module placed in degree $-1$. Thus $A \rightarrow A_{perf}$
is not formally smooth by
More on Morphisms, Lemma \ref{more-morphisms-lemma-NL-formally-smooth}
and
Cotangent, Lemma \ref{cotangent-lemma-relation-with-naive-cotangent-complex}.
\end{proof}

\noindent
The next example also involves rings of prime characteristic, but is perhaps
a little more surprising. The drawback is that it requires more knowledge of
characteristic $p$ phenomena than the previous example. Recall that we say a
ring $A$ of prime characteristic is $F$-finite if the Frobenius map on $A$
is finite. 

\begin{lemma}
\label{lemma-completion-etale}
Let $(A, \mathfrak m, \kappa)$ be a Noetherian local ring of prime
characteristic $p > 0$ such that $[\kappa : \kappa^p] < \infty$.
Then the  canonical map  $A \to A^\wedge$ to the completion of $A$
is flat and formally unramified. However, if $A$ is regular but not
excellent, then this map is not formally \'etale.
\end{lemma}

\begin{proof}
Flatness of the completion is
Algebra, Lemma \ref{algebra-lemma-completion-flat}.
To show that the map is formally unramified, it
suffices to show that $\Omega_{A^\wedge/A} = 0$, see
Algebra, Lemma \ref{algebra-lemma-characterize-formally-unramified}.

\medskip\noindent
We sketch a proof. Choose $x_1, \ldots, x_r \in A$ which map to a $p$-basis
$\overline{x}_1, \ldots, \overline{x}_r$ of $\kappa$, i.e.,
such that $\kappa$ is minimally generated by $\overline{x}_i$ over $\kappa^p$.
Choose a minimal set of generators $y_1, \ldots, y_s$ of $\mathfrak m$.
For each $n$ the elements $x_1, \ldots, x_r, y_1, \ldots, y_s$ generate
$A/\mathfrak m^n$ over $(A/\mathfrak m^n)^p$ by Frobenius.
Some details omitted. We conclude that $F : A^\wedge \to A^\wedge$
is finite. Hence $\Omega_{A^\wedge/A}$ is a finite $A^\wedge$-module.
On the other hand, for any $a \in A^\wedge$ and $n$ we can find
$a_0 \in A$ such that $a - a_0 \in \mathfrak m^nA^\wedge$.
We conclude that $\text{d}(a) \in \bigcap \mathfrak m^n \Omega_{A^\wedge/A}$
which implies that $\text{d}(a)$ is zero by
Algebra, Lemma \ref{algebra-lemma-intersect-powers-ideal-module-zero}.
Thus $\Omega_{A^\wedge/A} = 0$.

\medskip\noindent
Suppose $A$ is regular. Then, using the Cohen structure theorem
$x_1, \ldots, x_r, y_1, \ldots, y_s$ is a $p$-basis for the ring
$A^\wedge$, i.e., we have
$$
A^\wedge = \bigoplus\nolimits_{I, J} (A^\wedge)^p
x_1^{i_1} \ldots x_r^{i_r} y_1^{j_1} \ldots y_s^{j_s}
$$
with $I = (i_1, \ldots, i_r)$, $J = (j_1, \ldots, j_s)$ and
$0 \leq i_a, j_b \leq p - 1$. Details omitted. In particular, we see that
$\Omega_{A^\wedge}$ is a free $A^\wedge$-module with basis
$\text{d}(x_1), \ldots, \text{d}(x_r), \text{d}(y_1), \ldots, \text{d}(y_s)$.

\medskip\noindent
Now if $A \to A^\wedge$ is formally \'etale or even just formally smooth,
then we see that $\NL_{A^\wedge/A}$ has vanishing cohomology in degrees $-1, 0$
by Algebra, Proposition \ref{algebra-proposition-characterize-formally-smooth}.
It follows from the Jacobi-Zariski sequence
(Algebra, Lemma \ref{algebra-lemma-exact-sequence-NL}) for the ring
maps $\mathbf{F}_p \to A \to A^\wedge$ that we get an isomorphism
$\Omega_A \otimes_A A^\wedge \cong \Omega_{A^\wedge}$.
Hence we find that $\Omega_A$ is free on
$\text{d}(x_1), \ldots, \text{d}(x_r), \text{d}(y_1), \ldots, \text{d}(y_s)$.
Looking at fraction fields and using that $A$ is normal
we conclude that $a \in A$ is a $p$th power if and only if its image in
$A^\wedge$ is a $p$th power (details omitted; use
Algebra, Lemma \ref{algebra-lemma-derivative-zero-pth-power}).
A second consequence is that the operators $\partial/\partial x_a$ and
$\partial/\partial y_b$ are defined on $A$.

\medskip\noindent
We will show that the above lead to the conclusion that $A$ is finite
over $A^p$ with $p$-basis $x_1, \ldots, x_r, y_1, \ldots, y_s$.
This will contradict the non-excellency of $A$ by a result of
Kunz, see \cite[Corollary 2.6]{Kun76}.
Namely, say $a \in A$ and write
$$
a =  \sum\nolimits_{I, J} (a_{I, J})^p
x_1^{i_1} \ldots x_r^{i_r} y_1^{j_1} \ldots y_s^{j_s}
$$
with $a_{I, J} \in A^\wedge$. To finish the proof it suffices to
show that $a_{I, J} \in A$. Applying the operator
$$
(\partial/\partial x_1)^{p - 1} \ldots
(\partial/\partial x_r)^{p - 1}
(\partial/\partial y_1)^{p - 1} \ldots
(\partial/\partial y_s)^{p - 1}
$$
to both sides we conclude that $a_{I, J}^p \in A$ where
$I = (p - 1, \ldots, p - 1)$ and $J = (p - 1, \ldots, p - 1)$.
By our remark above, this also implies $a_{I, J} \in A$.
After replacing $a$ by $a' = a - a_{I, J}^p x^I y^J$
we can use a $1$-order lower differential operators to get
another coefficient $a_{I, J}$ to be in $A$. Etc.
\end{proof}

\begin{remark}
\label{remark-reference-existence-regular-nonexcellent-rings}
Non-excellent regular rings whose residue fields have a finite $p$-basis
can be constructed even in the function field of $\mathbb{P}^2_k$, over a 
characteristic $p$ field $k = \overline{k}$. See
\cite[$\mathsection 4.1$]{DS18}.
\end{remark}

\noindent
The proof of Lemma \ref{lemma-completion-etale} actually shows a little more.

\begin{lemma}
\label{lemma-excellent-regular-local-rings}
Let $(A, \mathfrak m, \kappa)$ be a regular local ring of characteristic
$p > 0$. Suppose $[\kappa : \kappa^p] < \infty$. Then $A$ is excellent
if and only if $A \to A^\wedge$ is formally \'etale.
\end{lemma}

\begin{proof}
The backward implication follows from Lemma \ref{lemma-completion-etale}.
For the forward implication, note that we already know from
Lemma \ref{lemma-completion-etale}
that $A \to A^\wedge$ is formally unramified or equivalently
that $\Omega_{A^\wedge/A}$ is zero.
Thus, it suffices to show that the completion map is formally smooth when
$A$ is excellent. By N\'eron-Popescu
desingularization $A \to A^\wedge$ can be  written as a filtered
colimit of smooth $A$-algebras
(Smoothing Ring Maps, Theorem \ref{smoothing-theorem-popescu}).
Hence $\NL_{A^\wedge/A}$ has vanishing cohomology in degree $-1$.
Thus $A \to A^\wedge$ is formally smooth by Algebra, Proposition
\ref{algebra-proposition-characterize-formally-smooth}.
\end{proof}










\section{Ideals generated by sets of idempotents and localization}
\label{section-ideal-locally-idempotents}

\noindent
Let $R$ be a ring. Consider the ring
$$
B(R) = R[x_n; n \in \mathbf{Z}]/(x_n(x_n - 1), x_nx_m; n \not = m)
$$
It is easy to show that every prime $\mathfrak q \subset B(R)$
is either of the form
$$
\mathfrak q = \mathfrak pB(R) + (x_n; n \in \mathbf{Z})
$$
or of the form
$$
\mathfrak q =
\mathfrak pB(R) + (x_n - 1) + (x_m; n \not = m, m \in \mathbf{Z}).
$$
Hence we see that
$$
\Spec(B(R)) =
\Spec(R) \amalg \coprod\nolimits_{n \in \mathbf{Z}} \Spec(R)
$$
where the topology is not just the disjoint union topology. It has the
following properties: Each of the copies indexed by $n \in \mathbf{Z}$
is an open subscheme, namely it is the standard open $D(x_n)$.
The "central" copy of $\Spec(R)$ is in the closure of the union
of any infinitely many of the other copies of $\Spec(R)$.
Note that this last copy of $\Spec(R)$ is cut out by the ideal
$(x_n, n \in \mathbf{Z})$ which is generated by the idempotents $x_n$.
Hence we see that if $\Spec(R)$ is connected,
then the decomposition above is exactly the decomposition of
$\Spec(B(R))$ into connected components.

\medskip\noindent
Next, let $A = \mathbf{C}[x, y]/((y - x^2 + 1)(y + x^2 - 1))$.
The spectrum of $A$ consists of two irreducible components
$C_1 = \Spec(A_1)$, $C_2 = \Spec(A_2)$
with $A_1 = \mathbf{C}[x, y]/(y - x^2 + 1)$ and
$A_2 = \mathbf{C}[x, y]/(y + x^2 - 1)$. Note that these are
parametrized by $(x, y) = (t, t^2 - 1)$ and $(x, y) = (t, -t^2 + 1)$
which meet in $P = (-1, 0)$ and $Q = (1, 0)$. We can make a twisted
version of $B(A)$ where we glue $B(A_1)$ to $B(A_2)$ in the following
way: Above $P$ we let $x_n \in B(A_1) \otimes \kappa(P)$
correspond to $x_n \in B(A_2) \otimes \kappa(P)$, but above $Q$
we let $x_n \in B(A_1) \otimes \kappa(Q)$
correspond to $x_{n + 1} \in B(A_2) \otimes \kappa(Q)$.
Let $B^{twist}(A)$ denote the resulting $A$-algebra.
Details omitted. By construction
$B^{twist}(A)$ is Zariski locally over $A$ isomorphic to the untwisted
version. Namely, this happens over both the principal open
$\Spec(A) \setminus \{P\}$
and the principal open $\Spec(A) \setminus \{Q\}$.
However, our choice of glueing produces enough "monodromy" such that
$\Spec(B^{twist}(A))$ is connected (details omitted).
Finally, there is a central copy of
$\Spec(A) \to \Spec(B^{twist}(A))$
which gives a closed subscheme whose ideal is Zariski locally
on $B^{twist}(A)$ cut out by ideals generated by idempotents, but
not globally (as $B^{twist}(A)$ has no nontrivial idempotents).

\begin{lemma}
\label{lemma-not-generated-idempotents}
There exists an affine scheme $X = \Spec(A)$ and a
closed subscheme $T \subset X$ such that $T$ is Zariski locally
on $X$ cut out by ideals generated by idempotents, but
$T$ is not cut out by an ideal generated by idempotents.
\end{lemma}

\begin{proof}
See above.
\end{proof}



\section{A ring map which identifies local rings which is not ind-\'etale}
\label{section-not-ind-etale}

\noindent
Note that the ring map $R \to B(R)$ constructed in
Section \ref{section-ideal-locally-idempotents} is
a colimit of finite products of copies of $R$. Hence $R \to B(R)$
is ind-Zariski, see
Pro-\'etale Cohomology, Definition \ref{proetale-definition-ind-zariski}.
Next, consider the ring map $A \to B^{twist}(A)$
constructed in Section \ref{section-ideal-locally-idempotents}.
Since this ring map is Zariski locally on $\Spec(A)$ isomorphic to an
ind-Zariski ring map $R \to B(R)$ we conclude that it identifies local rings
(see Pro-\'etale Cohomology, Lemma \ref{proetale-lemma-ind-zariski-implies}).
The discussion in Section \ref{section-ideal-locally-idempotents}
shows there is a section
$B^{twist}(A) \to A$ whose kernel is not generated by idempotents.
Now, if $A \to B^{twist}(A)$ were ind-\'etale, i.e.,
$B^{twist}(A) = \colim A_i$ with $A \to A_i$ \'etale,
then the kernel of $A_i \to A$ would be generated by an idempotent
(Algebra, Lemmas \ref{algebra-lemma-map-between-etale} and
\ref{algebra-lemma-surjective-flat-finitely-presented}).
This would contradict the result mentioned above.

\begin{lemma}
\label{lemma-not-ind-etale}
There is a ring map $A \to B$ which identifies local rings but
which is not ind-\'etale. A fortiori it is not ind-Zariski.
\end{lemma}

\begin{proof}
See discussion above.
\end{proof}




\section{Non flasque quasi-coherent sheaf associated to injective module}
\label{section-nonflasque}

\noindent
For more examples of this type see \cite[Expos\'e II, Appendix I]{SGA6}
where Illusie explains some examples due to Verdier.

\medskip\noindent
Consider the affine scheme $X = \Spec(A)$ where
$$
A = k[x, y, z_1, z_2, \ldots]/(x^nz_n)
$$
is the ring from
Properties, Example \ref{properties-example-does-not-work-in-general}.
Set $I = (x) \subset A$. Consider the quasi-compact open $U = D(x)$ of $X$.
We have seen in loc.\ cit.\ that there is a section
$s \in \mathcal{O}_X(U)$ which does not come from an $A$-module
map $I^n \to A$ for any $n \geq 0$.

\medskip\noindent
Let $\alpha : A \to J$ be the embedding of $A$ into an injective $A$-module.
Let $Q = J/\alpha(A)$ and denote $\beta : J \to Q$ the quotient map.
We claim that the map
$$
\Gamma(X, \widetilde{J})
\longrightarrow
\Gamma(U, \widetilde{J})
$$
is not surjective. Namely, we claim that $\alpha(s)$ is not in the image.
To see this, we argue by contradiction. So assume that $x \in J$ is an
element which restricts to $\alpha(s)$ over $U$. Then $\beta(x) \in Q$
is an element which restricts to $0$ over $U$. Hence we know that
$I^n\beta(x) = 0$ for some $n$, see
Properties,
Lemma \ref{properties-lemma-sections-over-quasi-compact-open-in-affine}.
This implies that we get a morphism
$\varphi : I^n \to A$, $h \mapsto \alpha^{-1}(hx)$. It is easy to see that
this morphism $\varphi$ gives rise to the section $s$ via the map of
Properties,
Lemma \ref{properties-lemma-sections-over-quasi-compact-open-in-affine}
which is a contradiction.

\begin{lemma}
\label{lemma-nonflasque}
There exists an affine scheme $X = \Spec(A)$ and an injective
$A$-module $J$ such that $\widetilde{J}$ is not a flasque sheaf on $X$.
Even the restriction $\Gamma(X, \widetilde{J}) \to \Gamma(U, \widetilde{J})$
with $U$ a standard open need not be surjective.
\end{lemma}

\begin{proof}
See above.
\end{proof}

\noindent
In fact, we can use a similar construction to get an example of an
injective module whose associated quasi-coherent sheaf has nonzero
cohomology over a quasi-compact open. Namely, we start with the ring
$$
A = k[x, y, w_1, u_1, w_2, u_2, \ldots]/(x^nw_n, y^nu_n, u_n^2, w_n^2)
$$
where $k$ is a field. Choose an injective map $A \to I$ where $I$ is an
injective $A$-module. We claim that the element $1/xy$ in
$A_{xy} \subset I_{xy}$ is not in the image of $I_x \oplus I_y \to I_{xy}$.
Arguing by contradiction, suppose that
$$
\frac{1}{xy} = \frac{i}{x^n} + \frac{j}{y^n}
$$
for some $n \geq 1$ and $i, j \in I$. Clearing denominators we obtain
$$
(xy)^{n + m - 1} = x^my^{n + m}i + x^{n + m}y^mj
$$
for some $m \geq 0$. Multiplying with $u_{n + m}w_{n + m}$ we see
that $u_{n + m}w_{n + m}(xy)^{n + m - 1} = 0$ in $A$ which is the
desired contradiction.
Let $U = D(x) \cup D(y) \subset X = \Spec(A)$. For any $A$-module
$M$ we have an exact sequence
$$
0 \to H^0(U, \widetilde{M}) \to M_x \oplus M_y \to M_{xy} \to
H^1(U, \widetilde{M}) \to 0
$$
by Mayer-Vietoris. We conclude that $H^1(U, \widetilde{I})$ is nonzero.

\begin{lemma}
\label{lemma-nonvanishing}
There exists an affine scheme $X = \Spec(A)$ whose underlying
topological space is Noetherian and an injective
$A$-module $I$ such that $\widetilde{I}$ has nonvanishing $H^1$
on some quasi-compact open $U$ of $X$.
\end{lemma}

\begin{proof}
See above. Note that $\Spec(A) = \Spec(k[x, y])$ as topological spaces.
\end{proof}









\section{A non-separated flat group scheme}
\label{section-non-separated-group-scheme}

\noindent
Every group scheme over a field is separated, see
Groupoids, Lemma \ref{groupoids-lemma-group-scheme-over-field-separated}.
This is not true for group schemes over a base.

\medskip\noindent
Let $k$ be a field. Let $S = \Spec(k[x]) = \mathbf{A}^1_k$.
Let $G$ be the affine line with $0$ doubled (see
Schemes, Example \ref{schemes-example-affine-space-zero-doubled})
seen as a scheme over $S$. Thus a fibre of $G \to S$ is either a
singleton or a set with two elements (one in $U$ and one in $V$).
Thus we can endow these fibres with the structure of a group (by
letting the element in $U$ be the zero of the group structure).
More precisely, $G$ has two opens $U, V$ which map isomorphically
to $S$ such that $U \cap V$ is mapped isomorphically to
$S \setminus \{0\}$. Then
$$
G \times_S G = U \times_S U \cup V \times_S U \cup
U \times_S V \cup V \times_S V
$$
where each piece is isomorphic to $S$. Hence we can define a multiplication
$m : G \times_S G \to G$ as the unique $S$-morphism which maps the first
and the last piece into $U$ and the two middle pieces into $V$. This matches
the pointwise description given above. We omit the
verification that this defines a group scheme structure.

\begin{lemma}
\label{lemma-non-separated-group-scheme}
There exists a flat group scheme of finite type over the affine line
which is not separated.
\end{lemma}

\begin{proof}
See the discussion above.
\end{proof}

\begin{lemma}
\label{lemma-non-quasi-separated-group-scheme}
There exists a flat group scheme of finite type over the infinite
dimensional affine space which is not quasi-separated.
\end{lemma}

\begin{proof}
The same construction as above can be carried out with the infinite dimensional
affine space $S = \mathbf{A}^\infty_k = \Spec k[x_1, x_2, \ldots]$ as the base
and the origin $0 \in S$ corresponding to the maximal ideal
$(x_1, x_2, \ldots)$ as the closed point which is doubled in $G$.
The resulting group scheme $G \rightarrow S$ is 
not quasi-separated as explained in
Schemes, Example \ref{schemes-example-not-quasi-separated}. 
\end{proof}



\section{A non-flat group scheme with flat identity component}
\label{section-non-flat-group-scheme}

\noindent
Let $X \to S$ be a monomorphism of schemes. Let $G = S \amalg X$.
Let $m : G \times_S G \to G$ be the $S$-morphism
$$
G \times_S G = X \times_S X \amalg X \amalg X \amalg S
\longrightarrow G = X \amalg S
$$
which maps the summands $X \times_S X$ and $S$ into $S$ and
maps the summands $X$ into $X$ by the identity morphism.
This defines a group law. To see this we have to show that
$m \circ (m \times \text{id}_G) = m \circ (\text{id}_G \times m)$
as maps $G \times_S G \times_S G \to G$. Decomposing
$G \times_S G \times_S G$ into components as above, we see that
we need to verify this for the restriction to each of the $8$-pieces.
Each piece is isomorphic to either $S$, $X$, $X \times_S X$, or
$X \times_S X \times_S X$. Moreover, both maps map these pieces
to $S$, $X$, $S$, $X$ respectively. Having said this, the fact that
$X \to S$ is a monomorphism implies that $X \times_S X \cong X$
and $X \times_S X \times_S X \cong X$ and that there is in each case
exactly one $S$-morphism $S \to S$ or $X \to X$. Thus we see
that $m \circ (m \times \text{id}_G) = m \circ (\text{id}_G \times m)$.
Thus taking $X \to S$ to be any
nonflat monomorphism of schemes (e.g., a closed immersion)
we get an example of a group scheme
over a base $S$ whose identity component is $S$ (hence flat)
but which is not flat.

\begin{lemma}
\label{lemma-non-flat-group-scheme}
There exists a group scheme $G$ over a base $S$ whose identity
component is flat over $S$ but which is not flat over $S$.
\end{lemma}

\begin{proof}
See discussion above.
\end{proof}




\section{A non-separated group algebraic space over a field}
\label{section-non-separated-group-space}

\noindent
Every group scheme over a field is separated, see
Groupoids, Lemma \ref{groupoids-lemma-group-scheme-over-field-separated}.
This is not true for group algebraic spaces over a field
(but see end of this section for positive results).

\medskip\noindent
Let $k$ be a field of characteristic zero.
Consider the algebraic space $G = \mathbf{A}^1_k/\mathbf{Z}$ from
Spaces, Example \ref{spaces-example-affine-line-translation}.
By construction $G$ is the fppf sheaf associated to the presheaf
$$
T \longmapsto \Gamma(T, \mathcal{O}_T) / \mathbf{Z}
$$
on the category of schemes over $k$. The obvious addition rule on the presheaf
induces an addition $m : G \times G \to G$ which turns $G$ into a group
algebraic space over $\Spec(k)$. Note that $G$ is not separated
(and not even quasi-separated or locally separated). On the other hand
$G \to \Spec(k)$ is of finite type!

\begin{lemma}
\label{lemma-non-separated-group-space}
There exists a group algebraic space of finite type over a field
which is not separated (and not even quasi-separated or locally separated).
\end{lemma}

\begin{proof}
See discussion above.
\end{proof}

\noindent
Positive results: If the group algebraic space $G$ is either
quasi-separated, or locally separated, or more generally a
decent algebraic space, then $G$ is in fact separated, see
More on Groupoids in Spaces, Lemma
\ref{spaces-more-groupoids-lemma-group-scheme-over-field-separated}.
Moreover, a finite type, separated group algebraic space over a
field is in fact a scheme by More on Groupoids in Spaces, Lemma
\ref{spaces-more-groupoids-lemma-group-space-scheme-locally-finite-type-over-k}.
The idea of the proof is that the schematic locus is open dense, see
Properties of Spaces, Proposition
\ref{spaces-properties-proposition-locally-quasi-separated-open-dense-scheme}
or
Decent Spaces, Theorem \ref{decent-spaces-theorem-decent-open-dense-scheme}.
By translating this open we see that
every point of $G$ has an open neighbourhood which is a scheme.




\section{Specializations between points in fibre \'etale morphism}
\label{section-specializations-fibre-etale}

\noindent
If $f : X \to Y$ is an \'etale, or more generally a locally quasi-finite
morphism of schemes, then there are no specializations between points of
fibres, see
Morphisms, Lemma \ref{morphisms-lemma-locally-quasi-finite-fibres}.
However, for morphisms of algebraic spaces this doesn't hold in general.

\medskip\noindent
To give an example, let $k$ be a field.
Set
$$
P = k[u, u^{-1}, y, \{x_n\}_{n \in \mathbf{Z}}].
$$
Consider the action
of $\mathbf{Z}$ on $P$ by $k$-algebra maps generated by the automorphism
$\tau$ given by the rules $\tau(u) = u$, $\tau(y) = uy$, and
$\tau(x_n) = x_{n + 1}$. For $d \geq 1$ set
$I_d = ((1 - u^d)y, x_n - x_{n + d}, n \in \mathbf{Z})$.
Then $V(I_d) \subset \Spec(P)$ is the fix point locus of $\tau^d$.
Let $S \subset P$ be the multiplicative subset generated by $y$ and
all $1 - u^d$, $d \in \mathbf{N}$. Then we
see that $\mathbf{Z}$ acts freely on $U = \Spec(S^{-1}P)$.
Let $X = U/\mathbf{Z}$ be the quotient algebraic space, see
Spaces, Definition \ref{spaces-definition-quotient}.

\medskip\noindent
Consider the prime ideals $\mathfrak p_n = (x_n, x_{n + 1}, \ldots)$ in
$S^{-1}P$. Note that $\tau(\mathfrak p_n) = \mathfrak p_{n + 1}$.
Hence each of these define point $\xi_n \in U$ whose image in $X$ is
the same point $x$ of $X$. Moreover we have the specializations
$$
\ldots \leadsto \xi_n \leadsto \xi_{n - 1} \leadsto \ldots
$$
We conclude that $U \to X$ is an example of the promised type.

\begin{lemma}
\label{lemma-specializations-fibre-etale}
There exists an \'etale morphism of algebraic spaces $f : X \to Y$
and a nontrivial specialization of points $x \leadsto x'$ in $|X|$ with
$f(x) = f(x')$ in $|Y|$.
\end{lemma}

\begin{proof}
See discussion above.
\end{proof}







\section{A torsor which is not an fppf torsor}
\label{section-torsor-not-fppf}

\noindent
In
Groupoids, Remark \ref{groupoids-remark-fun-with-torsors}
we raise the question whether any $G$-torsor is a $G$-torsor for the
fppf topology. In this section we show that this is not always the case.

\medskip\noindent
Let $k$ be a field. All schemes and stacks are over $k$ in what follows.
Let $G \to \Spec(k)$ be the group scheme
$$
G = (\mu_{2, k})^\infty =
\mu_{2, k} \times_k \mu_{2, k} \times_k \mu_{2, k} \times_k \ldots =
\lim_n (\mu_{2, k})^n
$$
where $\mu_{2, k}$ is the group scheme of second roots of unity over
$\Spec(k)$, see
Groupoids, Example \ref{groupoids-example-roots-of-unity}.
As an inverse limit of affine schemes we see that $G$ is an affine group
scheme. In fact it is the spectrum of the ring
$k[t_1, t_2, t_3, \ldots]/(t_i^2 - 1)$. The multiplication map
$m : G \times_k G \to G$ is on the algebra level given by
$t_i \mapsto t_i \otimes t_i$.

\medskip\noindent
We claim that any $G$-torsor over $k$ is of the form
$$
P = \Spec(k[x_1, x_2, x_3, \ldots]/(x_i^2 - a_i))
$$
for certain $a_i \in k^*$ and with $G$-action $G \times_k P \to P$
given by $x_i \to t_i \otimes x_i$ on the algebra level.
We omit the proof.
Actually for the example we only need that $P$ is a $G$-torsor
which is clear since over $k' = k(\sqrt{a_1}, \sqrt{a_2}, \ldots)$
the scheme $P$ becomes isomorphic to $G$ in a $G$-equivariant manner.
Note that $P$ is trivial if and only if $k' = k$ since if
$P$ has a $k$-rational point then all of the $a_i$ are squares.

\medskip\noindent
We claim that $P$ is an fppf torsor if and only if the field extension
$k' = k(\sqrt{a_1}, \sqrt{a_2}, \ldots)/k$ is finite.
If $k'$ is finite over $k$, then
$\{\Spec(k') \to \Spec(k)\}$
is an fppf covering which trivializes $P$ and we see that $P$ is indeed
an fppf torsor. Conversely, suppose that $P$ is an fppf $G$-torsor.
This means that there exists an fppf covering
$\{S_i \to \Spec(k)\}$ such that each $P_{S_i}$ is trivial.
Pick an $i$ such that $S_i$ is not empty. Let $s \in S_i$ be a closed
point. By
Varieties, Lemma \ref{varieties-lemma-locally-finite-type-Jacobson}
the field extension $\kappa(s)/k$ is finite, and by construction
$P_{\kappa(s)}$ has a $\kappa(s)$-rational point. Thus we see that
$k \subset k' \subset \kappa(s)$ and $k'$ is finite over $k$.

\medskip\noindent
To get an explicit example take $k = \mathbf{Q}$ and $a_i = i$
for example (or $a_i$ is the $i$th prime if you like).

\begin{lemma}
\label{lemma-torsors-principal-spaces-not-equal}
Let $S$ be a scheme. Let $G$ be a group scheme over $S$.
The stack $G\textit{-Principal}$ classifying principal homogeneous $G$-spaces
(see Examples of Stacks, Subsection
\ref{examples-stacks-subsection-principal-homogeneous-spaces})
and the stack $G\textit{-Torsors}$ classifying fppf $G$-torsors
(see Examples of Stacks, Subsection
\ref{examples-stacks-subsection-fppf-torsors})
are not equivalent in general.
\end{lemma}

\begin{proof}
The discussion above shows that the functor
$G\textit{-Torsors} \to G\textit{-Principal}$ isn't essentially
surjective in general.
\end{proof}









\section{Stack with quasi-compact flat covering which is not algebraic}
\label{section-not-algebraic-stack}

\noindent
In this section we briefly describe an example due to Brian Conrad.
You can find the example online at
\href{https://mathoverflow.net/questions/15082/fpqc-covers-of-stacks/15269#15269}{this location}.
Our example is slightly different.

\medskip\noindent
Let $k$ be an algebraically closed field.
All schemes and stacks are over $k$ in what follows.
Let $G \to \Spec(k)$ be an affine group scheme.
In Examples of Stacks,
Lemma \ref{examples-stacks-lemma-classifying-stacks}
we have given several different equivalent ways to view
$\mathcal{X} = [\Spec(k)/G]$ as a stack in groupoids over
$(\Sch/\Spec(k))_{fppf}$. In particular $\mathcal{X}$ classifies
fppf $G$-torsors. More precisely, a $1$-morphism $T \to \mathcal{X}$
corresponds to an fppf $G_T$-torsor $P$ over $T$ and $2$-arrows
correspond to isomorphisms of torsors. It follows that
the diagonal $1$-morphism
$$
\Delta :
\mathcal{X}
\longrightarrow
\mathcal{X} \times_{\Spec(k)} \mathcal{X}
$$
is representable and affine. Namely, given any pair
of fppf $G_T$-torsors $P_1, P_2$ over a scheme $T/k$ the
scheme $\mathit{Isom}(P_1, P_2)$ is affine over $T$.
The trivial $G$-torsor over $\Spec(k)$ defines a $1$-morphism
$$
f : \Spec(k) \longrightarrow \mathcal{X}.
$$
We claim that this is a surjective $1$-morphism. The reason is simply
that by definition for any $1$-morphism $T \to \mathcal{X}$ there exists
a fppf covering $\{T_i \to T\}$ such that $P_{T_i}$ is isomorphic
to the trivial $G_{T_i}$-torsor. Hence the compositions
$T_i \to T \to \mathcal{X}$ factor through $f$. Thus it is clear that
the projection $T \times_\mathcal{X} \Spec(k) \to T$
is surjective (which is how we define the property that $f$ is surjective, see
Algebraic Stacks,
Definition \ref{algebraic-definition-relative-representable-property}).
In a similar way you show that $f$ is quasi-compact and flat (details omitted).
We also record here the observation that
$$
\Spec(k) \times_\mathcal{X} \Spec(k) \cong G
$$
as schemes over $k$.

\medskip\noindent
Suppose there exists a surjective smooth morphism
$p : U \to \mathcal{X}$ where $U$ is a scheme.
Consider the fibre product
$$
\xymatrix{
W \ar[d] \ar[r] & U \ar[d] \\
\Spec(k) \ar[r] & \mathcal{X}
}
$$
Then we see that $W$ is a nonempty smooth scheme over $k$ which
hence has a $k$-point. This means that we can factor $f$ through $U$.
Hence we obtain
$$
G \cong
\Spec(k) \times_\mathcal{X} \Spec(k) \cong
(\Spec(k) \times_k \Spec(k))
\times_{(U \times_k U)}
(U \times_\mathcal{X} U)
$$
and since the projections $U \times_\mathcal{X} U \to U$ were
assumed smooth we conclude that
$U \times_\mathcal{X} U \to U \times_k U$ is
locally of finite type, see
Morphisms, Lemma \ref{morphisms-lemma-permanence-finite-type}.
It follows that in this case $G$ is locally of finite type over $k$.
Altogether we have proved the following lemma (which can be
significantly generalized).

\begin{lemma}
\label{lemma-BG-algebraic}
Let $k$ be a field. Let $G$ be an affine group scheme over $k$.
If the stack $[\Spec(k)/G]$ has a smooth covering by a
scheme, then $G$ is of finite type over $k$.
\end{lemma}

\begin{proof}
See discussion above.
\end{proof}

\noindent
To get an explicit example as in the title of this section, take for example
$G = (\mu_{2, k})^\infty$ the group scheme of
Section \ref{section-torsor-not-fppf}, which is not locally of finite type
over $k$. By the discussion above we see that
$\mathcal{X} = [\Spec(k)/G]$ has properties (1) and (2) of
Algebraic Stacks, Definition \ref{algebraic-definition-algebraic-stack},
but not property (3). Hence $\mathcal{X}$ is not an algebraic stack.
On the other hand, there does exist a scheme $U$ and a surjective,
flat, quasi-compact morphism $U \to \mathcal{X}$, namely the morphism
$f : \Spec(k) \to \mathcal{X}$ we studied above.






\section{Limit preserving on objects, not limit preserving}
\label{section-limit-preserving}

\noindent
Let $S$ be a nonempty scheme. Let $\mathcal{G}$ be an injective abelian sheaf
on $(\Sch/S)_{fppf}$. We obtain a stack in groupoids
$$
\mathcal{G}\textit{-Torsors} \longrightarrow (\Sch/S)_{fppf}
$$
over $S$, see Examples of Stacks, Lemma
\ref{examples-stacks-lemma-torsors-sheaf-stack-in-groupoids}.
This stack is limit preserving on objects over $(\Sch/S)_{fppf}$ (see
Criteria for Representability, Section \ref{criteria-section-limit-preserving})
because every $\mathcal{G}$-torsor is trivial. On the other hand,
$\mathcal{G}\textit{-Torsors}$ is in general not limit preserving (see
Artin's Axioms, Definition \ref{artin-definition-limit-preserving})
as $\mathcal{G}$ need not be limit preserving as a sheaf. For example,
take any nonzero injective sheaf $\mathcal{I}$ and set
$\mathcal{G} = \prod_{n \in \mathbf{Z}} \mathcal{I}$ to get an
example.

\begin{lemma}
\label{lemma-limit-preserving-on-objects-not-limit-preserving}
Let $S$ be a nonempty scheme. There exists a stack in groupoids
$p : \mathcal{X} \to (\Sch/S)_{fppf}$
such that $p$ is limit preserving on objects, but $\mathcal{X}$ is not
limit preserving.
\end{lemma}

\begin{proof}
See discussion above.
\end{proof}



\section{A non-algebraic classifying stack}
\label{section-non-algebraic}

\noindent
Let $S = \Spec(\mathbf{F}_p)$ and let $\mu_p$ denote the
group scheme of $p$th roots of unity over $S$. In
Groupoids in Spaces, Section \ref{spaces-groupoids-section-stacks}
we have introduced the quotient stack $[S/\mu_p]$ and in
Examples of Stacks, Section \ref{examples-stacks-section-group-quotient-stacks}
we have shown $[S/\mu_p]$ is the classifying stack for
fppf $\mu_p$-torsors: Given a scheme $T$ over $S$ the category
$\Mor_S(T, [S/\mu_p])$ is canonically equivalent to the
category of fppf $\mu_p$-torsors over $T$. Finally, in
Criteria for Representability, Theorem
\ref{criteria-theorem-flat-groupoid-gives-algebraic-stack}
we have seen that $[S/\mu_p]$ is an algebraic stack.

\medskip\noindent
Now we can ask the question: ``How about the category fibred in groupoids
$\mathcal{S}$ classifying \'etale $\mu_p$-torsors?'' (In other words
$\mathcal{S}$ is a category over $\Sch/S$ whose fibre category over a
scheme $T$ is the category of \'etale $\mu_p$-torsors over $T$.)

\medskip\noindent
The first objection is that this isn't a stack for the fppf topology,
because descent for objects isn't going to hold. For example the
$\mu_p$-torsor $\Spec(\mathbf{F}_p(t)[x]/(x^p - t))$ over
$T = \Spec(\mathbf{F}_p(T))$ is fppf locally trivial, but
not \'etale locally trivial.

\medskip\noindent
A fix for this first problem is to work with the \'etale topology and
in this case descent for objects does work. Indeed it is true that
$\mathcal{S}$ is a stack in groupoids over $(\Sch/S)_\etale$.
Moreover, it is also the case that the diagonal
$\Delta : \mathcal{S} \to \mathcal{S} \times \mathcal{S}$
is representable (by schemes). This is true because given two
$\mu_p$-torsors (whether they be \'etale locally trivial or not)
the sheaf of isomorphisms between them is representable by a scheme.

\medskip\noindent
Thus we can finally ask if there exists a scheme $U$ and a smooth
and surjective $1$-morphism $U \to \mathcal{S}$. We will show in two
ways that this is impossible: by a direct argument (which we advise
the reader to skip) and by an argument using a general result.

\medskip\noindent
Direct argument (sketch): Note that the $1$-morphism
$\mathcal{S} \to \Spec(\mathbf{F}_p)$ satisfies the
infinitesimal lifting criterion for formal smoothness.
This is true because given a first order infinitesimal thickening
of schemes $T \to T'$ the kernel of $\mu_p(T') \to \mu_p(T)$
is isomorphic to the sections of the ideal sheaf of $T$ in $T'$, and
hence $H^1_\etale(T, \mu_p) = H^1_\etale(T', \mu_p)$.
Moreover, $\mathcal{S}$ is a limit preserving stack. Hence
if $U \to \mathcal{S}$ is smooth, then $U \to \Spec(\mathbf{F}_p)$
is limit preserving and satisfies  the
infinitesimal lifting criterion for formal smoothness.
This implies that $U$ is smooth over $\mathbf{F}_p$.
In particular $U$ is reduced, hence $H^1_\etale(U, \mu_p) = 0$.
Thus $U \to \mathcal{S}$ factors as
$U \to \Spec(\mathbf{F}_p) \to \mathcal{S}$
and the first arrow is smooth. By descent of smoothness, we see
that $U \to \mathcal{S}$ being smooth would imply
$\Spec(\mathbf{F}_p) \to \mathcal{S}$ is smooth. However, this
is not the case as
$\Spec(\mathbf{F}_p) \times_\mathcal{S} \Spec(\mathbf{F}_p)$
is $\mu_p$ which is not smooth over $\Spec(\mathbf{F}_p)$.


\medskip\noindent
Structural argument: In
Criteria for Representability, Section \ref{criteria-section-stacks-etale}
we have seen that we can think of algebraic stacks as those
stacks in groupoids for the \'etale topology with diagonal
representable by algebraic spaces having a smooth covering.
Hence if a smooth surjective $U \to \mathcal{S}$ exists then
$\mathcal{S}$ is an algebraic stack, and in particular satisfies
descent in the fppf topology. But we've seen above that $\mathcal{S}$
does not satisfies descent in the fppf topology.

\medskip\noindent
Loosely speaking the arguments above show that the classifying
stack in the \'etale topology for \'etale locally trivial torsors
for a group scheme $G$ over a base $B$ is algebraic if and only
if $G$ is smooth over $B$. One of the advantages of working with
the fppf topology is that it suffices to assume that $G \to B$
is flat and locally of finite presentation. In fact the quotient
stack (for the fppf topology) $[B/G]$ is algebraic if and only
if $G \to B$ is flat and locally of finite presentation, see
Criteria for Representability, Lemma \ref{criteria-lemma-BG-algebraic}.







\section{Sheaf with quasi-compact flat covering which is not algebraic}
\label{section-not-algebraic}

\noindent
Consider the functor $F = (\mathbf{P}^1)^\infty$, i.e., for a scheme $T$
the value $F(T)$ is the set of $f = (f_1, f_2, f_3, \ldots)$ where each
$f_i : T \to \mathbf{P}^1$ is a morphism of schemes. Note that
$\mathbf{P}^1$ satisfies the sheaf property for fpqc coverings, see
Descent, Lemma \ref{descent-lemma-fpqc-universal-effective-epimorphisms}.
A product of sheaves is a sheaf, so $F$ also satisfies the sheaf property for
the fpqc topology. The diagonal of $F$ is representable: if $f : T \to F$
and $g : S \to F$ are morphisms, then $T \times_F S$ is the scheme theoretic
intersection of the closed subschemes $T \times_{f_i, \mathbf{P}^1, g_i} S$
inside the scheme $T \times S$. Consider the group scheme $\text{SL}_2$ which
comes with a surjective smooth affine morphism $\text{SL}_2 \to \mathbf{P}^1$.
Next, consider $U = (\text{SL}_2)^\infty$ with its canonical (product) morphism
$U \to F$. Note that $U$ is an affine scheme. We claim the morphism
$U \to F$ is flat, surjective, and universally open. Namely, suppose
$f : T \to F$ is a morphism. Then $Z = T \times_F U$ is the infinite
fibre product of the schemes $Z_i = T \times_{f_i, \mathbf{P}^1} \text{SL}_2$
over $T$. Each of the morphisms $Z_i \to T$ is surjective smooth and
affine which implies that
$$
Z = Z_1 \times_T Z_2 \times_T Z_3 \times_T \ldots
$$
is a scheme flat and affine over $Z$. A simple limit argument shows that
$Z \to T$ is open as well.

\medskip\noindent
On the other hand, we claim that $F$ isn't an algebraic space.
Namely, if $F$ where an algebraic space it would be a quasi-compact
and separated (by our description of fibre products over $F$) algebraic
space. Hence cohomology of quasi-coherent sheaves would vanish above a
certain cutoff (see
Cohomology of Spaces, Proposition \ref{spaces-cohomology-proposition-vanishing}
and remarks preceding it). But clearly by taking the pullback
of $\mathcal{O}(-2, -2, \ldots, -2)$ under the projection
$$
(\mathbf{P}^1)^\infty \longrightarrow (\mathbf{P}^1)^n
$$
(which has a section) we can obtain a quasi-coherent sheaf whose cohomology
is nonzero in degree $n$. Altogether we obtain an answer to a question asked
by Anton Geraschenko on mathoverflow.

\begin{lemma}
\label{lemma-not-algebraic}
There exists a functor $F : \Sch^{opp} \to \textit{Sets}$
which satisfies the sheaf condition for the fpqc topology, has representable
diagonal $\Delta : F \to F \times F$, and such that there exists a
surjective, flat, universally open, quasi-compact morphism
$U \to F$ where $U$ is a scheme, but such that $F$ is not an algebraic space.
\end{lemma}

\begin{proof}
See discussion above.
\end{proof}








\section{Sheaves and specializations}
\label{section-sheaves}

\noindent
In the following we fix a big \'etale site $\Sch_\etale$
as constructed in
Topologies, Definition \ref{topologies-definition-big-etale-site}.
Moreover, a scheme will be an object of this site.
Recall that if $x, x'$ are points of a scheme $X$ we say $x$ is a
{\it specialization} of $x'$ or we write $x' \leadsto x$ if
$x \in \overline{\{x'\}}$. This is true in particular if $x = x'$.

\medskip\noindent
Consider the functor $F : \Sch_\etale \to \textit{Ab}$
defined by the following rules:
$$
F(X) = \prod\nolimits_{x \in X}
\prod\nolimits_{x' \in X, x' \leadsto x, x' \not = x}
\mathbf{Z}/2\mathbf{Z}
$$
Given a scheme $X$ we denote $|X|$ the underlying set of points.
An element $a \in F(X)$ will be viewed as a map of sets
$|X| \times |X| \to \mathbf{Z}/2\mathbf{Z}$, $(x, x') \mapsto a(x, x')$
which is zero if $x = x'$ or if $x$ is not a specialization of $x'$.
Given a morphism of schemes $f : X \to Y$ we define
$$
F(f) : F(Y) \longrightarrow F(X)
$$
by the rule that for $b \in F(Y)$ we set
$$
F(f)(b)(x, x') =
\left\{
\begin{matrix}
0 & \text{if }x\text{ is not a specialization of }x' \\
b(f(x), f(x')) & \text{else.}
\end{matrix}
\right.
$$
Note that this really does define an element of $F(X)$. We claim that if
$f : X \to Y$ and $g : Y \to Z$ are composable morphisms then
$F(f) \circ F(g) = F(g \circ f)$. Namely, let $c \in F(Z)$ and let
$x' \leadsto x$ be a specialization of points in $X$, then
$$
F(g \circ f)(x, x') = c(g(f(x)), g(f(x'))) = F(g)(F(f)(c))(x, x')
$$
because $f(x') \leadsto f(x)$. (This also works if $f(x) = f(x')$.)

\medskip\noindent
Let $G$ be the sheafification of $F$ in the \'etale topology.

\medskip\noindent
I claim that if $X$ is a scheme and $x' \leadsto x$ is a specialization
and $x' \not = x$, then $G(X) \not = 0$. Namely, let $a \in F(X)$ be
an element such that when we think of $a$ as a function
$|X| \times |X| \to \mathbf{Z}/2\mathbf{Z}$ it is nonzero at $(x, x')$.
Let $\{f_i : U_i \to X\}$ be an \'etale covering of $X$. Then we can pick an
$i$ and a point $u_i \in U_i$ with $f_i(u_i) = x$. Since generalizations
lift along flat morphisms (see
Morphisms, Lemma \ref{morphisms-lemma-generalizations-lift-flat})
we can find a specialization $u'_i \leadsto u_i$
with $f_i(u'_i) = x'$. By our construction above we see that
$F(f_i)(a) \not = 0$. Hence $a$ determines a nonzero element of $G(X)$.

\medskip\noindent
Note that if $X = \Spec(k)$ where $k$ is a field (or more generally
a ring all of whose prime ideals are maximal), then $F(X) = 0$
and for every \'etale morphism $U \to X$ we have $F(U) = 0$ because there
are no specializations between distinct points in fibres of an \'etale
morphism. Hence $G(X) = 0$.

\medskip\noindent
Suppose that $X \subset X'$ is a thickening, see
More on Morphisms, Definition \ref{more-morphisms-definition-thickening}.
Then the category of schemes \'etale over $X'$ is equivalent to the
category of schemes \'etale over $X$ by the base change functor
$U' \mapsto U = U' \times_{X'} X$, see
\'Etale Cohomology,
Theorem \ref{etale-cohomology-theorem-topological-invariance}.
Since it is always the case that $F(U) = F(U')$ in this situation
we see that also $G(X) = G(X')$.

\medskip\noindent
As a variant we can consider the presheaf $F_n$ which associates
to a scheme $X$ the collection of maps
$a : |X|^{n + 1} \to \mathbf{Z}/2\mathbf{Z}$ where $a(x_0, \ldots, x_n)$
is nonzero only if $x_n \leadsto \ldots \leadsto x_0$ is a sequence of
specializations and $x_n \not = x_{n - 1} \not = \ldots \not = x_0$.
Let $G_n$ be the sheaf associated to $F_n$.
In exactly the same way as above one shows that $G_n$ is nonzero
if $\dim(X) \geq n$ and is zero if $\dim(X) < n$.

\begin{lemma}
\label{lemma-sheaf-zero-on-low-dimension}
There exists a sheaf of abelian groups $G$ on
$\Sch_\etale$ with the following properties
\begin{enumerate}
\item $G(X) = 0$ whenever $\dim(X) < n$,
\item $G(X)$ is not zero if $\dim(X) \geq n$, and
\item if $X \subset X'$ is a thickening, then $G(X) = G(X')$.
\end{enumerate}
\end{lemma}

\begin{proof}
See the discussion above.
\end{proof}

\begin{remark}
\label{remark-specialization}
Here are some remarks:
\begin{enumerate}
\item The presheaves $F$ and $F_n$ are separated presheaves.
\item It turns out that $F$, $F_n$ are not sheaves.
\item One can show that $G$, $G_n$ is actually a sheaf for the fppf topology.
\end{enumerate}
We will prove these results if we need them.
\end{remark}


\section{Sheaves and constructible functions}
\label{section-constructible-functions}

\noindent
In the following we fix a big \'etale site $\Sch_\etale$ as constructed in
Topologies, Definition \ref{topologies-definition-big-etale-site}.
Moreover, a scheme will be an object of this site.
In this section we say that a {\it constructible partition}
of a scheme $X$ is a locally finite disjoint union decomposition
$X = \coprod_{i \in I} X_i$ such that each $X_i \subset X$ is a
locally constructible subset of $X$. Locally finite means that
for any quasi-compact open $U \subset X$ there are only finitely
many $i \in I$ such that $X_i \cap U$ is not empty.
Note that if $f : X \to Y$ is a morphism of schemes and
$Y = \coprod Y_j$ is a constructible partition, then
$X = \coprod f^{-1}(Y_j)$ is a constructible partition of $X$.
Given a set $S$ and a scheme $X$ a {\it constructible function}
$f : |X| \to S$ is a map such that $X = \coprod_{s \in S} f^{-1}(s)$
is a constructible partition of $X$.
If $G$ is an (abstract group) and $a, b : |X| \to G$ are constructible
functions, then $ab : |X| \to G, x \mapsto a(x)b(x)$ is a constructible
function too. The reason is that given any two constructible partitions
there is a third one refining both.

\medskip\noindent
Let $A$ be any abelian group. For any scheme $X$ we define
$$
F(X) =
\frac{\{a : |X| \to A \mid a \text{ is a constructible function}\}}{\{\text{locally constant functions }|X| \to A\}}
$$
We think of an element $a$ of $F(X)$ simply as a function well defined
up to adding a locally constant one. Given a morphism of schemes
$f : X \to Y$ and an element $b \in F(Y)$, then we define
$F(f)(b) = b \circ f$. Thus $F$ is a presheaf on $\Sch_\etale$.

\medskip\noindent
Note that if $\{f_i : U_i \to X\}$ is an fppf covering, and $a \in F(X)$
is such that $F(f_i)(a) = 0$ in $F(U_i)$, then $a \circ f_i$ is a locally
constant function for each $i$. This means in turn that $a$ is a locally
constant function as the morphisms $f_i$ are open. Hence $a = 0$ in $F(X)$.
Thus we see that $F$ is a separated presheaf (in the fppf topology
hence a fortiori in the \'etale topology).

\medskip\noindent
Let $G$ be the sheafification of $F$ in the \'etale topology.
Since $F$ is separated, and since $F(X) \not = 0$ for example when
$X$ is the spectrum of a discrete valuation ring, we see that $G$
is not zero.

\medskip\noindent
Let $X = \Spec(k)$ where $k$ is a field.
Then any \'etale covering of $X$ can be dominated
by a covering $\{\Spec(k') \to \Spec(k)\}$ with $k'/k$
a finite separable extension of fields. Since $F(\Spec(k')) = 0$
we see that $G(X) = 0$.

\medskip\noindent
Suppose that $X \subset X'$ is a thickening, see
More on Morphisms, Definition \ref{more-morphisms-definition-thickening}.
Then the category of schemes \'etale over $X'$ is equivalent to the
category of schemes \'etale over $X$ by the base change functor
$U' \mapsto U = U' \times_{X'} X$, see
\'Etale Cohomology,
Theorem \ref{etale-cohomology-theorem-topological-invariance}.
Since $F(U) = F(U')$ in this situation we see that also $G(X) = G(X')$.

\medskip\noindent
The sheaf $G$ is limit preserving, see
Limits of Spaces,
Definition \ref{spaces-limits-definition-locally-finite-presentation}.
Namely, let $R$ be a ring which is written as a directed colimit
$R = \colim_i R_i$ of rings. Set $X = \Spec(R)$ and
$X_i = \Spec(R_i)$, so that $X = \lim_i X_i$. Then
$G(X) = \colim_i G(X_i)$. To prove this one first proves that
a constructible partition of $\Spec(R)$ comes from a constructible
partitions of some $\Spec(R_i)$. Hence the result for $F$. To
get the result for the sheafification, use that any \'etale ring map
$R \to R'$ comes from an \'etale ring map $R_i \to R_i'$ for some $i$.
Details omitted.

\begin{lemma}
\label{lemma-weird-sheaf}
There exists a sheaf of abelian groups $G$ on
$\Sch_\etale$ with the following properties
\begin{enumerate}
\item $G(\Spec(k)) = 0$ whenever $k$ is a field,
\item $G$ is limit preserving,
\item if $X \subset X'$ is a thickening, then $G(X) = G(X')$, and
\item $G$ is not zero.
\end{enumerate}
\end{lemma}

\begin{proof}
See discussion above.
\end{proof}




\section{The lisse-\'etale site is not functorial}
\label{section-lisse-etale-not-functorial}

\noindent
The {\it lisse-\'etale} site
$X_{lisse,\etale}$ of $X$ is the category of schemes smooth over $X$
endowed with (usual) \'etale coverings, see
Cohomology of Stacks, Section \ref{stacks-cohomology-section-lisse-etale}.
Let $f : X  \to Y$ be a morphism of schemes.
There is a functor
$$
u : Y_{lisse,\etale} \longrightarrow X_{lisse,\etale},\quad
V/Y \longmapsto V \times_Y X
$$
which is continuous. Hence we obtain an adjoint pair of functors
$$
u^s :
\Sh(X_{lisse,\etale})
\longrightarrow
\Sh(Y_{lisse,\etale}),
\quad
u_s :
\Sh(Y_{lisse,\etale})
\longrightarrow
\Sh(X_{lisse,\etale}),
$$
see Sites, Section \ref{sites-section-continuous-functors}.
We claim that, in general, $u$ does {\bf not} define a morphism of sites, see
Sites, Definition \ref{sites-definition-morphism-sites}.
In other words, we claim that $u_s$ is not left exact in general. Note that
representable presheaves are sheaves on lisse-\'etale sites. Hence, by
Sites, Lemma \ref{sites-lemma-pullback-representable-sheaf}
we see that $u_sh_V = h_{V \times_Y X}$. Now consider two morphisms
$$
\xymatrix{
V_1 \ar[rd] \ar@<1ex>[rr]^a \ar@<-1ex>[rr]_b & & V_2 \ar[ld] \\
& Y
}
$$
of schemes $V_1, V_2$ smooth over $Y$. Now if $u_s$ is left exact, then
we would have
$$
u_s \text{Equalizer}(h_a, h_b : h_{V_1} \to h_{V_2})
=
\text{Equalizer}(h_{a \times 1}, h_{b \times 1} :
h_{V_1 \times_Y X} \to h_{V_2 \times_Y X})
$$
We will take the morphisms $a, b : V_1 \to V_2$ such that there exists
no morphism from a scheme smooth over $Y$ into $(a = b) \subset V_1$, i.e.,
such that the left hand side is the empty sheaf, but such that after
base change to $X$ the equalizer is nonempty and smooth over $X$.
A silly example is to take $X = \Spec(\mathbf{F}_p)$,
$Y = \Spec(\mathbf{Z})$ and $V_1 = V_2 = \mathbf{A}^1_\mathbf{Z}$
with morphisms $a(x) = x$ and $b(x) = x + p$. Note that the equalizer
of $a$ and $b$ is the fibre of $\mathbf{A}^1_\mathbf{Z}$ over $(p)$.

\begin{lemma}
\label{lemma-lisse-etale-not-functorial}
The lisse-\'etale site is not functorial, even for morphisms of schemes.
\end{lemma}

\begin{proof}
See discussion above.
\end{proof}






\section{Sheaves on the category of Noetherian schemes}
\label{section-sheaves-locally-Noetherian}

\noindent
Let $S$ be a locally Noetherian scheme. As in
Artin's Axioms, Section \ref{artin-section-noetherian}
consider the inclusion functor
$$
u : (\textit{Noetherian}/S)_{fppf} \longrightarrow (\Sch/S)_{fppf}
$$
of the fppf site of locally Noetherian schemes over $S$ into a
big fppf site of $S$. As explained in the section referenced,
this functor is continuous. Hence we obtain an adjoint pair of functors
$$
u^s :
\Sh((\Sch/S)_{fppf})
\longrightarrow
\Sh((\textit{Noetherian}/S)_{fppf})
$$
and
$$
u_s :
\Sh((\textit{Noetherian}/S)_{fppf})
\longrightarrow
\Sh((\Sch/S)_{fppf})
$$
see Sites, Section \ref{sites-section-continuous-functors}.
However, we claim that $u$ in general does not define a
morphism of sites, see Sites, Definition \ref{sites-definition-morphism-sites}.
In other words, we claim that the functor $u_s$ is not left exact in general.

\medskip\noindent
Let $p$ be a prime number and set $S = \Spec(\mathbf{F}_p)$.
Consider the injective map of sheaves
$$
a : \mathcal{F} \longrightarrow \mathcal{G}
$$
on $(\textit{Noetherian}/S)_{fppf}$ defined as follows: for $U$
a locally Noetherian scheme over $S$ we define
$$
\mathcal{G}(U) = \Gamma(U, \mathcal{O}_U)^* =
\Mor_S(U, \mathbf{G}_{m, S})
$$
and we take
$$
\mathcal{F}(U) = \{f \in \mathcal{G}(U) \mid \text{fppf locally }f
\text{ has arbitrary }p\text{-power roots}\}
$$
A Noetherian $\mathbf{F}_p$-algebra $A$ has a nilpotent nilradical
$I \subset A$, the $p$-power roots of $1$ in $A$ are of the
elements of the form $1 + a$, $a \in I$, and hence no-nontrivial
$p$-power root of $1$ has arbitrary $p$-power roots. We conclude
that $\mathcal{F}(U)$ is a $p$-torsion free abelian group for any
locally Noetherian scheme $U$; some details omitted. It follows that
$p : \mathcal{F} \to \mathcal{F}$
is an injective map of abelian sheaves on $(\textit{Noetherian}/S)_{fppf}$.

\medskip\noindent
To get a contradiction, assume $u_s$ is exact. Then
$p : u_s\mathcal{F} \to u_s\mathcal{F}$ is injective too
and we find that $(u_s\mathcal{F})(V)$ is a $p$-torsion
free abelian group for any $V$ over $S$. Since
representable presheaves are sheaves on fppf sites, by
Sites, Lemma \ref{sites-lemma-pullback-representable-sheaf},
we see that $u_s\mathcal{G}$ is represented by
$\mathbf{G}_{m, S}$. Using that $u_s\mathcal{F} \to u_s\mathcal{G}$
is injective, we find a $p$-torsion free subgroup
$$
(u_s\mathcal{F})(V) \subset \Gamma(V, \mathcal{O}_V)^*
$$
for every scheme $V$ over $S$ with the following property:
for every morphism $V \to U$ of schemes over $S$ with $U$
locally Noetherian the subgroup
$$
\mathcal{F}(U) \subset \Gamma(U, \mathcal{O}_U)^*
$$
maps into the subgroup $(u_s\mathcal{F})(V)$ by the restriction
mapping $\Gamma(U, \mathcal{O}_U)^* \to \Gamma(V, \mathcal{O}_V)^*$.

\medskip\noindent
The actual contradiction now is obtained as follows:
let $k = \bigcup_{n \geq 0} \mathbf{F}_p(t^{1/{p^n}})$
and set
$$
B = k \otimes_{\mathbf{F}_p(t)} k
$$
and $V = \Spec(B)$. Since we have the two projection morphisms
$V \to \Spec(k)$ corresponding to the two coprojections $k \to B$
and since $\Spec(k)$ is Noetherian, we conclude the subgroup
$$
(u_s\mathcal{F})(V) \subset B^*
$$
contains $k^* \otimes 1$ and $1 \otimes k^*$. This is a contradiction
because
$$
(t^{1/p} \otimes 1) \cdot (1 \otimes t^{-1/p}) =
t^{1/p} \otimes t^{-1/p}
$$
is a nontrivial $p$-torsion unit of $B$.

\begin{lemma}
\label{lemma-not-a-morphism-of-sites-noetherian-to-all}
With $S = \Spec(\mathbf{F}_p)$ the inclusion functor
$(\textit{Noetherian}/S)_{fppf} \to (\Sch/S)_{fppf}$
does not define a morphism of sites.
\end{lemma}

\begin{proof}
See discussion above.
\end{proof}





\section{Derived pushforward of quasi-coherent modules}
\label{section-derived-push-quasi-coherent}

\noindent
Let $k$ be a field of characteristic $p > 0$. Let $S = \Spec(k[x])$.
Let $G = \mathbf{Z}/p\mathbf{Z}$ viewed either as an abstract group
or as a constant group scheme over $S$. Consider the algebraic stack
$\mathcal{X} = [S/G]$ where $G$ acts trivially on $S$, see
Examples of Stacks, Remark \ref{examples-stacks-remark-X-mod-G-group}
and
Criteria for Representability, Lemma \ref{criteria-lemma-BG-algebraic}.
Consider the structure morphism
$$
f : \mathcal{X} \longrightarrow S
$$
This morphism is quasi-compact and quasi-separated. Hence we get a functor
$$
Rf_{\QCoh, *} :
D^{+}_\QCoh(\mathcal{O}_\mathcal{X})
\longrightarrow
D^{+}_\QCoh(\mathcal{O}_S),
$$
see 
Derived Categories of Stacks, Proposition
\ref{stacks-perfect-proposition-derived-direct-image-quasi-coherent}.
Let's compute $Rf_{\QCoh, *}\mathcal{O}_\mathcal{X}$.
Since $D_\QCoh(\mathcal{O}_S)$ is equivalent to
the derived category of $k[x]$-modules (see
Derived Categories of Schemes, Lemma
\ref{perfect-lemma-affine-compare-bounded})
this is equivalent to computing
$R\Gamma(\mathcal{X}, \mathcal{O}_\mathcal{X})$.
For this we can use the covering $S \to \mathcal{X}$ and the spectral
sequence
$$
H^q(S \times_\mathcal{X} \ldots \times_\mathcal{X} S, O)
\Rightarrow H^{p + q}(\mathcal{X}, \mathcal{O}_\mathcal{X})
$$
see
Cohomology of Stacks, Proposition
\ref{stacks-cohomology-proposition-smooth-covering-compute-cohomology}.
Note that
$$
S \times_\mathcal{X} \ldots \times_\mathcal{X} S = S \times G^p
$$
which is affine. Thus the complex
$$
k[x] \to \text{Map}(G, k[x]) \to \text{Map}(G^2, k[x]) \to \ldots
$$
computes $R\Gamma(\mathcal{X}, \mathcal{O}_\mathcal{X})$.
Here for $\varphi \in \text{Map}(G^{p - 1}, k[x])$ its differential is
the map which sends $(g_1, \ldots, g_p)$ to
$$
\varphi(g_2, \ldots, g_p) +
\sum\nolimits_{i = 1}^{p - 1}
(-1)^i\varphi(g_1, \ldots, g_i + g_{i + 1}, \ldots, g_p)
+ (-1)^p\varphi(g_1, \ldots, g_{p - 1}).
$$
This is just the complex computing the group cohomology of $G$ acting
trivially on $k[x]$ (insert future reference here). The cohomology of
the cyclic group $G$ on $k[x]$ is exactly one copy of $k[x]$ in each
cohomological degree $\geq 0$ (insert future reference here). We conclude
that
$$
Rf_*\mathcal{O}_\mathcal{X} = \bigoplus\nolimits_{n \geq 0} \mathcal{O}_S[-n]
$$
Now, consider the complex
$$
E = \bigoplus\nolimits_{m \geq 0} \mathcal{O}_\mathcal{X}[m]
$$
This is an object of $D_\QCoh(\mathcal{O}_\mathcal{X})$. We interrupt the
discussion for a general result.

\begin{lemma}
\label{lemma-is-limit}
Let $\mathcal{X}$ be an algebraic stack. Let $K$ be an object of
$D(\mathcal{O}_\mathcal{X})$ whose cohomology sheaves are locally
quasi-coherent (Sheaves on Stacks, Definition
\ref{stacks-sheaves-definition-locally-quasi-coherent})
and satisfy the flat base change property (Cohomology of Stacks,
Definition \ref{stacks-cohomology-definition-flat-base-change}).
Then there exists a distinguished triangle
$$
K \to
\prod\nolimits_{n \geq 0} \tau_{\geq -n} K \to
\prod\nolimits_{n \geq 0} \tau_{\geq -n} K \to K[1]
$$
in $D(\mathcal{O}_\mathcal{X})$. In other words, $K$ is the derived
limit of its canonical truncations.
\end{lemma}

\begin{proof}
Recall that we work on the ``big fppf site'' $\mathcal{X}_{fppf}$
of $\mathcal{X}$ (by our conventions
for sheaves of $\mathcal{O}_\mathcal{X}$-modules in the chapters
Sheaves on Stacks and Cohomology on Stacks). Let $\mathcal{B}$ be the set
of objects $x$ of $\mathcal{X}_{fppf}$ which lie over an affine scheme $U$. 
Combining
Sheaves on Stacks, Lemmas
\ref{stacks-sheaves-lemma-compare-fppf-etale},
\ref{stacks-sheaves-lemma-cohomology-restriction},
Descent, Lemma \ref{descent-lemma-quasi-coherent-and-flat-base-change},
and
Cohomology of Schemes, Lemma
\ref{coherent-lemma-quasi-coherent-affine-cohomology-zero}
we see that $H^p(x, \mathcal{F}) = 0$ if $\mathcal{F}$ is
locally quasi-coherent and $x \in \mathcal{B}$.
Now the claim follows from
Cohomology on Sites, Lemma \ref{sites-cohomology-lemma-is-limit-dimension}
with $d = 0$.
\end{proof}

\begin{lemma}
\label{lemma-sum-is-product}
Let $\mathcal{X}$ be an algebraic stack. If $\mathcal{F}_n$ is a collection
of locally quasi-coherent sheaves with the flat base change property on
$\mathcal{X}$, then $\oplus_n \mathcal{F}_n[n] \to \prod_n \mathcal{F}_n[n]$
is an isomorphism in $D(\mathcal{O}_\mathcal{X})$.
\end{lemma}

\begin{proof}
This is true because by Lemma \ref{lemma-is-limit} we see that the direct sum
is isomorphic to the product.
\end{proof}

\noindent
We continue our discussion. Since a quasi-coherent module is locally
quasi-coherent and satisfies the flat base change property
(Sheaves on Stacks, Lemma \ref{stacks-sheaves-lemma-quasi-coherent}) we get
$$
E = \prod\nolimits_{m \geq 0} \mathcal{O}_\mathcal{X}[m]
$$
Since cohomology commutes
with limits we see that
$$
Rf_*E = \prod\nolimits_{m \geq 0}
\left(\bigoplus\nolimits_{n \geq 0} \mathcal{O}_S[m - n]\right)
$$
Note that this complex is not an object of $D_\QCoh(\mathcal{O}_S)$
because the cohomology sheaf in degree $0$ is an infinite product of copies
of $\mathcal{O}_S$ which is not even a locally quasi-coherent
$\mathcal{O}_S$-module.

\begin{lemma}
\label{lemma-push-not-OK}
A quasi-compact and quasi-separated morphism
$f : \mathcal{X} \to \mathcal{Y}$ of algebraic stacks
need not induce a functor
$Rf_* : D_\QCoh(\mathcal{O}_\mathcal{X}) \to
D_\QCoh(\mathcal{O}_\mathcal{Y})$.
\end{lemma}

\begin{proof}
See discussion above.
\end{proof}



\section{A big abelian category}
\label{section-big}

\noindent
The purpose of this section is to give an example of a ``big'' abelian category
$\mathcal{A}$ and objects $M, N$ such that the collection of isomorphism
classes of extensions $\Ext_\mathcal{A}(M, N)$ is not a set.
The example is due to Freyd, see \cite[page 131, Exercise A]{Freyd}.

\medskip\noindent
We define $\mathcal{A}$ as follows. An object of $\mathcal{A}$
consists of a triple $(M, \alpha, f)$ where $M$ is an abelian group and
$\alpha$ is an ordinal and $f : \alpha \to \text{End}(M)$ is a map.
A morphism $(M, \alpha, f) \to (M', \alpha', f')$ is given by a homomorphism
of abelian groups $\varphi : M \to M'$ such that for {\it any} ordinal $\beta$
we have
$$
\varphi \circ f(\beta) = f'(\beta) \circ \varphi
$$
Here the rule is that we set $f(\beta) = 0$ if $\beta$ is not in
$\alpha$ and similarly we set $f'(\beta)$ equal to zero if $\beta$ is not an
element of $\alpha'$. We omit the verification that the category so defined
is abelian.

\medskip\noindent
Consider the object $Z = (\mathbf{Z}, \emptyset, f)$, i.e., all the
operators are zero. The observation is that computed in
$\mathcal{A}$ the group $\Ext^1_\mathcal{A}(Z, Z)$ is a proper class
and not a set. Namely, for each ordinal $\alpha$ we can find an extension
$(M, \alpha + 1, f)$ of $Z$ by $Z$ whose underlying group is
$M = \mathbf{Z} \oplus \mathbf{Z}$ and where the value of $f$ is
always zero except for
$$
f(\alpha) =
\left(
\begin{matrix}
0 & 1 \\
0 & 0
\end{matrix}
\right).
$$
This clearly produces a proper class of isomorphism classes of extensions.
In particular, the derived category of $\mathcal{A}$ has proper classes for
its collections of morphism, see
Derived Categories, Lemma \ref{derived-lemma-ext-1}. This means that some
care has to be exercised when defining Verdier quotients of triangulated
categories.

\begin{lemma}
\label{lemma-big-abelian-category}
There exists a ``big'' abelian category $\mathcal{A}$ whose
$\Ext$-groups are proper classes.
\end{lemma}

\begin{proof}
See discussion above.
\end{proof}





\section{Weakly associated points and scheme theoretic density}
\label{section-weak-ass-dense}

\noindent
Let $k$ be a field. Let $R = k[z, x_i, y_i]/(z^2, zx_iy_i)$ where
$i$ runs over the elements of $\mathbf{N}$. Note that $R = R_0 \oplus M_0$ where
$R_0 = k[x_i, y_i]$ is a subring and $M_0$ is an ideal of square zero
with $M_0 \cong R_0/(x_iy_i)$ as $R_0$-module. The prime
$\mathfrak p = (z, x_i)$ is weakly associated to $R$ as an $R$-module
(Algebra, Definition \ref{algebra-definition-weakly-associated}).
Indeed, the element $z$ in $R_\mathfrak p$ is nonzero but annihilated
by $\mathfrak pR_\mathfrak p$. On the other hand, consider the open
subscheme
$$
U = \bigcup D(x_i) \subset \Spec(R) = S
$$
We claim that $U \subset S$ is scheme theoretically dense
(Morphisms, Definition \ref{morphisms-definition-scheme-theoretically-dense}).
To prove this it suffices to show that $\mathcal{O}_S \to j_*\mathcal{O}_U$
is injective where $j : U \to S$ is the inclusion morphism, see 
Morphisms, Lemma \ref{morphisms-lemma-characterize-scheme-theoretically-dense}.
Translated back into algebra, we have to show that for all $g \in R$
the map
$$
R_g \longrightarrow \prod R_{x_ig}
$$
is injective. Write $g = g_0 + m_0$ with $g_0 \in R_0$ and $m_0 \in M_0$.
Then $R_g = R_{g_0}$ (details omitted). Hence we may assume $g \in R_0$.
We may also assume $g$ is not zero. Now $R_g = (R_0)_g \oplus (M_0)_g$.
Since $R_0$ is a domain, the map $(R_0)_g \to \prod (R_0)_{x_ig}$ is
injective. If $g \in (x_iy_i)$ then $(M_0)_g = 0$ and there is nothing
to prove. If $g \not \in (x_iy_i)$ then, since
$(x_iy_i)$ is a radical ideal of $R_0$, we have to show that
$M_0 \to \prod (M_0)_{x_ig}$ is injective. The kernel of
$R_0 \to M_0 \to (M_0)_{x_n}$ is $(x_iy_i, y_n)$. Since
$(x_iy_i, y_n)$ is a radical ideal, if $g \not \in (x_iy_i, y_n)$
then the kernel of $R_0 \to M_0 \to (M_0)_{x_ng}$ is $(x_iy_i, y_n)$.
As $g \not \in (x_iy_i, y_n)$ for all $n \gg 0$ we conclude that
the kernel is contained in $\bigcap_{n \gg 0} (x_iy_i, y_n) = (x_iy_i)$
as desired.

\medskip\noindent
Second example due to Ofer Gabber. Let $k$ be a field and let
$R$, resp.\ $R'$ be the ring of functions $\mathbf{N} \to k$, resp.\ the ring
of eventually constant functions $\mathbf{N} \to k$. Then $\Spec(R)$,
resp.\ $\Spec(R')$ is the
Stone-{\v C}ech compactification\footnote{Every element $f \in R$ is of
the form $ue$ where $u$ is a unit and $e$ is an idempotent. Then
Algebra, Lemma \ref{algebra-lemma-ring-with-only-minimal-primes}
shows $\Spec(R)$ is Hausdorff. On the other hand, $\mathbf{N}$ with the
discrete topology can be viewed as a dense open subset. Given a set map
$\mathbf{N} \to X$ to a Hausdorff, quasi-compact topological space $X$,
we obtain a ring map $\mathcal{C}^0(X; k) \to R$ where $\mathcal{C}^0(X; k)$
is the $k$-algebra of locally constant maps $X \to k$. This gives
$\Spec(R) \to \Spec(\mathcal{C}^0(X; k)) = X$ proving the universal
property.} $\beta\mathbf{N}$, resp.\ the
one point compactification\footnote{Here one argues that there is really
only one extra maximal ideal in $R'$.}
$\mathbf{N}^* = \mathbf{N} \cup \{\infty\}$.
All points are weakly associated since all primes are minimal in the
rings $R$ and $R'$.

\begin{lemma}
\label{lemma-example-schematically-dense-missing-weakly-associated-point}
There exists a reduced scheme $X$ and a schematically dense open
$U \subset X$ such that some weakly associated point $x \in X$ is not in $U$.
\end{lemma}

\begin{proof}
In the first example we have $\mathfrak p \not \in U$ by construction.
In Gabber's examples the schemes $\Spec(R)$ or $\Spec(R')$ are reduced.
\end{proof}



\section{Example of non-additivity of traces}
\label{section-non-additive}

\noindent
Let $k$ be a field and let $R = k[\epsilon]$ be the ring of dual numbers
over $k$. In other words, $R = k[x]/(x^2)$ and $\epsilon$ is the congruence
class of $x$ in $R$. Consider the short exact sequence of complexes
$$
\xymatrix{
0 \ar[d] \ar[r] &
R \ar[d]^\epsilon \ar[r]_1 &
R \ar[d] \\
R \ar[r]^1 & R \ar[r] & 0
}
$$
Here the columns are the complexes, the first row is placed in degree $0$, and
the second row in degree $1$. Denote the first complex (i.e., the
left column) by $A^\bullet$, the second by $B^\bullet$ and the third
$C^\bullet$. We claim that the diagram
\begin{equation}
\label{equation-commutes-up-to-homotopy}
\vcenter{
\xymatrix{
A^\bullet \ar[d]_{1 + \epsilon} \ar[r] &
B^\bullet \ar[r] \ar[d]_1 &
C^\bullet \ar[d]_1 \\
A^\bullet \ar[r] & B^\bullet \ar[r] & C^\bullet
}
}
\end{equation}
commutes in $K(R)$, i.e., is a diagram of complexes commuting up to homotopy.
Namely, the square on the right commutes and the one on the left is off by
the homotopy $1 : A^1 \to B^0$. On the other hand,
$$
\text{Tr}_{A^\bullet}(1 + \epsilon) + \text{Tr}_{C^\bullet}(1)
\not = \text{Tr}_{B^\bullet}(1).
$$

\begin{lemma}
\label{lemma-nonadditivity-of-trace}
There exists a ring $R$, a distinguished triangle
$(K, L, M, \alpha, \beta, \gamma)$ in the homotopy category $K(R)$,
and an endomorphism $(a, b, c)$ of this distinguished triangle, such that
$K$, $L$, $M$ are perfect complexes and
$\text{Tr}_K(a) + \text{Tr}_M(c) \not = \text{Tr}_L(b)$.
\end{lemma}

\begin{proof}
Consider the example above. The map $\gamma : C^\bullet \to A^\bullet[1]$
is given by multiplication by $\epsilon$ in degree $0$, see
Derived Categories, Definition \ref{derived-definition-distinguished-triangle}.
Hence it is also true that
$$
\xymatrix{
C^\bullet \ar[d] \ar[r]_\gamma & A^\bullet[1] \ar[d] \\
C^\bullet \ar[r]^\gamma & A^\bullet[1]
}
$$
commutes in $K(R)$ as $\epsilon(1 + \epsilon) = \epsilon$.
Thus we indeed have a morphism of distinguished triangles.
\end{proof}






\section{Being projective is not local on the base}
\label{section-non-descending-property-projective}

\noindent
In the chapter on descent we have seen that many properties of morphisms
are local on the base, even in the fpqc topology. See
Descent, Sections \ref{descent-section-descending-properties-morphisms},
\ref{descent-section-descending-properties-morphisms-fpqc}, and
\ref{descent-section-descending-properties-morphisms-fppf}.
This is not true for projectivity of morphisms.

\begin{lemma}
\label{lemma-non-descending-property-projective}
The properties
\begin{enumerate}
\item[] $\mathcal{P}(f) =$``$f$ is projective'', and
\item[] $\mathcal{P}(f) =$``$f$ is quasi-projective''
\end{enumerate}
are not Zariski local on the base. A fortiori, they are not fpqc local
on the base.
\end{lemma}

\begin{proof}
Following Hironaka \cite[Example B.3.4.1]{H},
we define a proper morphism of smooth complex 3-folds
$f:V_Y\to Y$
which is Zariski-locally projective, but not projective. Since $f$ is proper
and not projective, it is also not quasi-projective.

\medskip\noindent
Let $Y$ be projective 3-space over the complex numbers ${\mathbf C}$.
Let $C$ and $D$ be smooth conics in $Y$ such
that the closed subscheme $C\cap D$ is reduced and consists
of two complex points $P$ and $Q$. (For example,
let $C=\{ [x,y,z,w]: xy=z^2, w=0\}$, $D=\{ [x,y,z,w]:
xy=w^2, z=0\}$, $P=[1,0,0,0]$,
and $Q=[0,1,0,0]$.) 
On $Y-Q$, first blow up the curve $C$, and then blow
up the strict transform of the curve $D$ (Divisors, Definition 
\ref{divisors-definition-strict-transform}). On $Y-P$, first blow up
the curve $D$, and then blow up the strict transform of the curve
$C$. Over $Y-P-Q$, the two varieties we have constructed are canonically
isomorphic, and so we can glue them over $Y-P-Q$. The result
is a smooth proper 3-fold $V_Y$ over ${\mathbf C}$. The morphism
$f:V_Y\to Y$ is proper and Zariski-locally projective (since
it is a blowup over $Y-P$ and over $Y-Q$), by Divisors,
Lemma \ref{divisors-lemma-blowing-up-projective}. We will show that
$V_Y$ is not projective over ${\mathbf C}$. That will imply that
$f$ is not projective.

\medskip\noindent
To do this, let $L$ be the inverse image in $V_Y$ of a complex point
of $C-P-Q$, and $M$ the inverse image of a complex point of $D-P-Q$.
Then $L$ and $M$ are isomorphic to the projective line
${\mathbf P}^1_{{\mathbf C}}$. 
Next, let $E$ be the inverse image in $V_Y$ of $C\cup D\subset Y$ in $V_Y$;
thus $E\rightarrow C\cup D$ is a proper morphism, with fibers
isomorphic to ${\mathbf P}^1$ over $(C\cup D)-\{P,Q\}$.
The inverse
image of $P$ in $E$ is a union of two lines $L_0$ and $M_0$, and we have
rational equivalences of cycles $L\sim L_0+M_0$ and $M\sim M_0$ on $E$
(using that $C$ and $D$ are isomorphic to ${\mathbf P}^1$).
Note the asymmetry resulting from the order in which we blew
up the two curves. Near $Q$, the opposite happens. So the inverse image
of $Q$ is the union of two lines $L_0'$ and $M_0'$, and we have
rational equivalences $L\sim L_0'$ and $M\sim L_0'+M_0'$ on $E$.
Combining these equivalences, we find that $L_0+M_0'\sim 0$
on $E$ and hence on $V_Y$. If $V_Y$ were projective over ${\mathbf C}$,
it would have
an ample line bundle $H$, which would have degree $> 0$ on all curves
in $V_Y$. In particular $H$ would have positive degree on $L_0+M_0'$,
contradicting that the degree of a line bundle is well-defined
on 1-cycles modulo rational equivalence on a proper scheme
over a field (Chow Homology,
Lemma \ref{chow-lemma-proper-pushforward-rational-equivalence}
and Lemma \ref{chow-lemma-factors}).
So $V_Y$ is not projective over ${\mathbf C}$.
\end{proof}

\noindent
In different terminology, Hironaka's 3-fold $V_Y$ is a small
resolution of the blowup $Y'$ of $Y$ along the reduced subscheme
$C\cup D$; here $Y'$ has two node singularities. If we define $Z$ by blowing
up $Y$ along $C$ and then along the strict transform
of $D$, then $Z$ is a smooth projective 3-fold, and the non-projective
3-fold $V_Y$ differs from $Z$ by a ``flop'' over $Y-P$.


\section{Non-effective descent data for projective schemes}
\label{section-non-effective-descent-projective}

\noindent
In the chapter on descent we have seen that descent data for schemes relative
to an fpqc morphism are effective for several classes
of morphisms. In particular, affine morphisms and more generally
quasi-affine morphisms satisfy descent for fpqc coverings
(Descent, Lemma \ref{descent-lemma-quasi-affine}).
This is not true for projective morphisms.

\begin{lemma}
\label{lemma-non-effective-descent-projective}
There is an etale covering $X\to S$ of schemes and a descent datum
$(V/X,\varphi)$ relative to $X\to S$ such that 
$V\to X$ is projective,
but the descent datum is not effective in the category of schemes.
\end{lemma}

\begin{proof}
We imitate Hironaka's example of a smooth separated complex
algebraic space of dimension 3
which is not a scheme \cite[Example B.3.4.2]{H}.

\medskip\noindent
Consider the action of the group $G = \mathbf{Z}/2 = \{1, g\}$
on projective 3-space
$\mathbf{P}^3$ over the complex numbers by
$$
g[x,y,z,w] = [y,x,w,z].
$$
The action is free outside the two disjoint lines
$L_1=\{ [x,x,z,z]\}$ and $L_2=\{ [x,-x,z,-z]\}$ in
${\mathbf P}^3$. Let $Y={\mathbf P}^3-(L_1\cup L_2)$. There is a
smooth quasi-projective scheme $S=Y/G$ over ${\mathbf C}$ such that
$Y\to S$ is a $G$-torsor (Groupoids,
Definition \ref{groupoids-definition-principal-homogeneous-space}).
Explicitly, we can define $S$ as the image of the open subset $Y$
in ${\mathbf P}^3$ under the morphism
\begin{align*}
{\mathbf P}^3 & \to \text{Proj } {\mathbf C}[x,y,z,w]^G\\
   & = \text{Proj } {\mathbf C}[u_0,u_1,v_0,v_1,v_2]/(v_0v_1=v_2^2),
\end{align*}
where $u_0=x+y$, $u_1=z+w$, $v_0=(x-y)^2$, $v_1=(z-w)^2$,
and $v_2=(x-y)(z-w)$, and the ring is graded with $u_0,u_1$
in degree 1 and $v_0,v_1,v_2$ in degree 2.

\medskip\noindent
Let $C=\{ [x,y,z,w]: xy=z^2, w=0\}$ and $D=\{ [x,y,z,w]:
xy=w^2, z=0\}$. These are smooth conic curves in ${\mathbf P}^3$, contained
in the $G$-invariant open subset $Y$, with $g(C)=D$. Also,
$C\cap D$ consists of the two points $P:=[1,0,0,0]$
and $Q:=[0,1,0,0]$, and these two points are switched by the action
of $G$. 

\medskip\noindent
Let $V_Y\to Y$ be the scheme which over $Y-P$
is defined by blowing up $D$ and then the strict transform
of $C$, and over $Y-Q$ is defined by blowing up $C$ and then
the strict transform of $D$. (This is the same construction
as in the proof of Lemma \ref{lemma-non-descending-property-projective},
except that $Y$ here denotes an open subset of ${\mathbf P}^3$
rather than all of ${\mathbf P}^3$.)
Then the action of $G$ on $Y$ lifts to an action of $G$ on $V_Y$,
which switches the inverse images of $Y-P$ and $Y-Q$. This action
of $G$ on $V_Y$ gives a descent datum $(V_Y/Y,\varphi_Y)$
on $V_Y$ relative to the $G$-torsor
$Y\to S$. The morphism $V_Y\to Y$
is proper but not projective, as shown in the proof
of Lemma \ref{lemma-non-descending-property-projective}.

\medskip\noindent
Let $X$ be the disjoint union of the open subsets $Y-P$ and $Y-Q$;
then we have surjective etale morphisms $X\to Y\to S$.
Let $V$ be the pullback of $V_Y\to Y$ to $X$; then the morphism
$V\to X$ is projective, since $V_Y\to Y$ is a blowup over each of the open
subsets $Y-P$ and $Y-Q$. Moreover, the descent datum $(V_Y/Y,\varphi_Y)$
pulls back to a descent datum $(V/X,\varphi)$ relative to the
etale covering $X\to S$.

\medskip\noindent
Suppose that this descent datum is effective in the category
of schemes. That is, there is a scheme $U\to S$
which pulls back to the morphism $V\to X$ together
with its descent datum. Then $U$ would be the quotient
of $V_Y$ by its $G$-action.
$$
\xymatrix{
V \ar[r]\ar[d]&  X\ar[d] \\
V_Y \ar[r]\ar[d]& Y\ar[d] \\
U \ar[r]& S
}
$$

\medskip\noindent
Let $E$ be the inverse image of $C\cup D\subset Y$ in $V_Y$;
thus $E\rightarrow C\cup D$ is a proper morphism, with fibers
isomorphic to ${\mathbf P}^1$ over $(C\cup D)-\{P,Q\}$.
The inverse image of $P$ in $E$ is a union of two lines $L_0$
and $M_0$. It follows that the inverse image of $Q=g(P)$ in $E$
is the union of two lines $L_0'=g(M_0)$ and $M_0'=g(L_0)$.
As shown in the proof
of Lemma \ref{lemma-non-descending-property-projective},
we have a rational equivalence $L_0+M_0'=L_0+g(L_0)\sim 0$ on $E$.

\medskip\noindent
By descent of closed subschemes, there is a curve $L_1\subset U$
(isomorphic to ${\mathbf P}^1$)
whose inverse image in $V_Y$ is $L_0\cup g(L_0)$. (Use Descent, Lemma
\ref{descent-lemma-affine}, noting that a closed immersion is an affine
morphism.)
Let $R$ be a complex point of $L_1$. Since
we assumed that $U$ is a scheme, we can choose a function
$f$ in the local ring $O_{U,R}$ that vanishes at $R$ but not
on the whole curve $L_1$. Let $D_{\text{loc}}$ be an irreducible component
of the closed subset $\{f = 0\}$ in $\Spec O_{U,R}$; then
$D_{\text{loc}}$ has codimension 1.
The closure of $D_{\text{loc}}$ in $U$ is an irreducible divisor $D_U$
in $U$ which contains the point $R$ but not the whole curve $L_1$.
The inverse image of $D_U$ in $V_Y$ is an effective divisor $D$
which intersects $L_0\cup g(L_0)$ but does not contain either
curve $L_0$ or $g(L_0)$.

\medskip\noindent
Since the complex 3-fold $V_Y$ is smooth, $O(D)$ is a line
bundle on $V_Y$. We use here that a regular local ring is factorial,
or in other words is a UFD, see
More on Algebra, Lemma \ref{more-algebra-lemma-regular-local-UFD}.
The restriction of $O(D)$ to the proper surface
$E\subset V_Y$ is a line bundle which has positive degree on the 1-cycle
$L_0+g(L_0)$, by our information on $D$. Since
$L_0+g(L_0)\sim 0$ on $E$, this contradicts 
that the degree of a line bundle is well-defined
on 1-cycles modulo rational equivalence on a proper scheme
over a field (Chow Homology,
Lemma \ref{chow-lemma-proper-pushforward-rational-equivalence}
and Lemma \ref{chow-lemma-factors}). Therefore the descent datum
$(V/X,\varphi)$ is in fact not effective; that is, $U$ does not exist
as a scheme.
\end{proof}

\noindent
In this example, the descent datum {\it is }effective in the category
of algebraic spaces. More precisely,
$U$ exists as a smooth separated algebraic space
of dimension 3 over ${\mathbf C}$,
for example by Algebraic Spaces, Lemma \ref{spaces-lemma-quotient}.
Hironaka's 3-fold $U$ is a small resolution of the blowup $S'$
of the smooth quasi-projective 3-fold $S$ along the irreducible nodal curve
$(C\cup D)/G$; the 3-fold $S'$ has a node singularity. The
other small resolution of $S'$ (differing from $U$ by a ``flop'')
is again an algebraic space which is not a scheme.






\section{A family of curves whose total space is not a scheme}
\label{section-family-of-curves}

\noindent
In Quot, Section \ref{quot-section-curves} we define a family of curves over
a scheme $S$ to be a proper, flat, finitely presented morphism
of relative dimension $\leq 1$ from an algebraic space $X$ to $S$.
If $S$ is the spectrum of a complete Noetherian local ring, then
$X$ is a scheme, see
More on Morphisms of Spaces, Lemma
\ref{spaces-more-morphisms-lemma-projective-over-complete}.
In this section we show this is not true in general.

\medskip\noindent
Let $k$ be a field. We start with a proper flat morphism
$$
Y \longrightarrow \mathbf{A}^1_k
$$
and a point $y \in Y(k)$ lying over $0 \in \mathbf{A}^1_k(k)$
with the following properties
\begin{enumerate}
\item the fibre $Y_0$ is a smooth geometrically irreducible curve over $k$,
\item for any proper closed subscheme $T \subset Y$ dominating
$\mathbf{A}^1_k$ the intersection $T \cap Y_0$ contains at least one point
distinct from $y$.
\end{enumerate}
Given such a surface we construct our example as follows.
$$
\xymatrix{
Y \ar[rd] & Z \ar[d] \ar[l] \ar[r] & X \ar[ld] \\
& \mathbf{A}^1_k
}
$$
Here $Z \to Y$ is the blowup of $Y$ in $y$. Let $E \subset Z$ be the
exceptional divisor and let $C \subset Z$ be the strict transform of $Y_0$.
We have $Z_0 = E \cup C$ scheme theoretically (to see this use that
$Y$ is smooth at $y$ and moreover $Y \to \mathbf{A}^1_k$ is smooth at $y$).
By Artin's results (\cite{ArtinII}; use
Semistable Reduction, Lemma \ref{models-lemma-properties-form}
to see that the normal bundle of $C$ is negative)
we can blow down the curve $C$ in $Z$ to obtain an algebraic space $X$
as in the diagram. Let $x \in X(k)$ be the image of $C$.

\medskip\noindent
We claim that $X$ is not a scheme. Namely, if it were a scheme,
then there would be an affine open neighbourhood $U \subset X$ of $x$.
Set $T = X \setminus U$. Then $T$ dominates $\mathbf{A}^1_k$
(as the fibres of $X \to \mathbf{A}^1_k$ are proper of dimension $1$
and the fibres of $U \to \mathbf{A}^1_k$ are affine hence different).
Let $T' \subset Z$ be the closed subscheme mapping isomorphically
to $T$ (as $x \not \in T$). Then the image of $T'$ in $X$
contradicts condition (2) above (as $T' \cap Z_0$ is contained in
the exceptional divisor $E$ of the blowing up $Z \to Y$).

\medskip\noindent
To finish the discussion we need to construct our $Y$.
We will assume the characteristic of $k$ is not $3$.
Write $\mathbf{A}^1_k = \Spec(k[t])$ and take
$$
Y \quad : \quad T_0^3 + T_1^3 + T_2^3 - tT_0T_1T_2 = 0
$$
in $\mathbf{P}^2_{k[t]}$. The fibre of this for $t = 0$
is a smooth projective genus $1$ curve. On the affine piece
$V_+(T_0)$ we get the affine equation
$$
1 + x^3 + y^3 - txy = 0
$$
which defines a smooth surface over $k$. Since the same is true on the other
affine pieces by symmetry we see that $Y$ is a smooth surface.
Finally, we see from the affine equation also that the fraction
field is $k(x, y)$ hence $Y$ is a rational surface.
Now the Picard group of a rational surface is finitely generated
(insert future reference here).
Hence in order to choose $y \in Y_0(k)$ with property (2)
it suffices to choose $y$ such that
\begin{equation}
\label{equation-three}
\mathcal{O}_{Y_0}(ny) \not \in \Im(\Pic(Y) \to \Pic(Y_0))
\text{ for all }n > 0
\end{equation}
Namely, the sum of the $1$-dimensional irreducible components of a $T$
contradicting (2) would give an effective Cartier divisor intersection $Y_0$
in the divisor $ny$ for some $n \geq 1$ and we would conclude
that $\mathcal{O}_{Y_0}(ny)$ is in the image of the restriction map.
Observe that since $Y_0$ has genus $\geq 1$ the map
$$
Y_0(k) \to \Pic(Y_0),\quad y \mapsto \mathcal{O}_{Y_0}(y)
$$
is injective. Now if $k$ is an uncountable algebraically closed field,
then using the countability of $\Pic(Y)$ and the remark
just made, we can find a $y \in Y_0(k)$ satisfying (\ref{equation-three})
and hence (2).

\begin{lemma}
\label{lemma-family-of-curves-not-scheme}
There exists a field $k$ and a family of curves
$X \to \mathbf{A}^1_k$ such that $X$ is not a scheme.
\end{lemma}

\begin{proof}
See discussion above.
\end{proof}






\section{Derived base change}
\label{section-derived-base-change}

\noindent
Let $R \to R'$ be a ring map. In
More on Algebra, Section \ref{more-algebra-section-derived-base-change}
we construct a derived base change functor
$- \otimes_R^\mathbf{L} R' : D(R) \to D(R')$.
Next, let $R \to A$ be a second ring map. Picture
$$
\xymatrix{
	A \ar[r] & A \otimes_R R' \ar@{=}[r] & A' \\
R \ar[u] \ar[r] & R' \ar[u] \ar[ur]
}
$$
Given an $A$-module $M$ the tensor product $M \otimes_R R'$ is a
$A \otimes_R R'$-module, i.e., an $A'$-module. For the ring
map $A \to A'$ there is a derived functor
$$
- \otimes_A^\mathbf{L} A' : D(A) \longrightarrow D(A')
$$
but this functor does not agree with $- \otimes_R^\mathbf{L} R'$
in general. More precisely, for $K \in D(A)$ the canonical map
$$
K \otimes_R^{\mathbf{L}} R' \longrightarrow
K \otimes_A^{\mathbf{L}} A'
$$
in $D(R')$ constructed in
More on Algebra, Equation (\ref{more-algebra-equation-comparison-map})
isn't an isomorphism in general. Thus one may wonder if there exists a
``derived base change functor'' $T : D(A) \to D(A')$, i.e., a functor
such that $T(K)$ maps to $K \otimes_R^\mathbf{L} R'$ in $D(R')$.
In this section we show it does not exist in general.

\medskip\noindent
Let $k$ be a field. Set $R = k[x, y]$. Set $R' = R/(xy)$ and $A = R/(x^2)$.
The object $A \otimes_R^\mathbf{L} R'$ in $D(R')$ is represented by
$$
x^2 : R' \longrightarrow R'
$$
and we have $H^0(A \otimes_R^\mathbf{L} R') = A \otimes_R R'$. We claim that
there does not exist an object $E$ of $D(A \otimes_R R')$ mapping to
$A \otimes_R^\mathbf{L} R'$ in $D(R')$. Namely, for such an $E$ the module
$H^0(E)$ would be free, hence $E$ would decompose as
$H^0(E)[0] \oplus H^{-1}(E)[1]$. But it is easy to see that
$A \otimes_R^\mathbf{L} R'$ is not isomorphic to the sum of its
cohomology groups in $D(R')$.

\begin{lemma}
\label{lemma-no-derived-base-change}
Let $R \to R'$ and $R \to A$ be ring maps. In general there does not
exist a functor $T : D(A) \to D(A \otimes_R R')$
of triangulated categories such that an $A$-module $M$ gives an
object $T(M)$ of $D(A \otimes_R R')$ which maps to
$M \otimes_R^\mathbf{L} R'$ under the map $D(A \otimes_R R') \to D(R')$.
\end{lemma}

\begin{proof}
See discussion above.
\end{proof}




\section{An interesting compact object}
\label{section-interesting-compact}


\noindent
Let $R$ be a ring. Let $(A, \text{d})$ be a differential graded $R$-algebra.
If $A = R$, then we know that every compact object of $D(A, \text{d}) = D(R)$
is represented by a finite complex of finite projective modules. In other
words, compact objects are perfect, see
More on Algebra, Proposition \ref{more-algebra-proposition-perfect-is-compact}.
The analogue in the language of differential graded modules would
be the question: ``Is every compact object of $D(A, \text{d})$ represented
by a differential graded $A$-module $P$ which is finite and
graded projective?''

\medskip\noindent
For general differential graded algebras, this is not true. Namely,
let $k$ be a field of characteristic $2$
(so we don't have to worry about signs).
Let $A = k[x, y]/(y^2)$
with
\begin{enumerate}
\item $x$ of degree $0$
\item $y$ of degree $-1$,
\item $\text{d}(x) = 0$, and
\item $\text{d}(y) = x^2 + x$.
\end{enumerate}
Then $x : A \to A$ is a projector in $K(A, \text{d})$.
Hence we see that
$$
A = \Ker(x) \oplus \Im(1 - x)
$$
in $K(A, \text{d})$, see
Differential Graded Algebra, Lemma \ref{dga-lemma-homotopy-direct-sums} and
Derived Categories, Lemma
\ref{derived-lemma-projectors-have-images-triangulated}.
It is clear that $A$ is a compact object of $D(A, \text{d})$.
Then $\Ker(x)$ is a compact object of $D(A, \text{d})$
as follows from
Derived Categories, Lemma \ref{derived-lemma-compact-objects-subcategory}.

\medskip\noindent
Next, suppose that $M$ is a differential graded (right) $A$-module
representing $\Ker(x)$ and suppose that $M$ is finite and
projective as a graded $A$-module. Because every finite graded projective
module over $k[x, y]/(y^2)$ is graded free, we see that $M$ is
finite free as a graded $k[x, y]/(y^2)$-module (i.e., when we forget
the differential). We set $N = M/M(x^2 + x)$.
Consider the exact sequence
$$
0 \to M \xrightarrow{x^2 + x} M \to N \to 0
$$
Since $x^2 + x$ is of degree $0$, in the center of $A$, and
$\text{d}(x^2 + x) = 0$ we see that this is a short exact sequence
of differential graded $A$-modules. Moreover, as $\text{d}(y) = x^2 + x$
we see that the differential on $N$ is linear. The maps
$$
H^{-1}(N) \to H^0(M)
\quad\text{and}\quad
H^0(M) \to H^0(N)
$$
are isomorphisms as $H^*(M) = H^0(M) = k$ since $M \cong \Ker(x)$
in $D(A, \text{d})$. A computation of the boundary map shows that
$H^*(N) = k[x, y]/(x, y^2)$ as a graded module; we omit
the details. Since $N$ is a free $k[x, y]/(y^2, x^2 + x)$-module
we have a resolution
$$
\ldots \to N[2] \xrightarrow{y} N[1] \xrightarrow{y} N \to N/Ny \to 0
$$
compatible with differentials. Since $N$ is bounded and since
$H^0(N) = k[x,y]/(x, y^2)$ it follows from
Homology, Lemma \ref{homology-lemma-first-quadrant-ss}
that $H^0(N/Ny) = k[x]/(x)$. But as $N/Ny$ is a finite complex of free
$k[x]/(x^2 + x) = k \times k$-modules, we see that its cohomology
has to have even dimension, a contradiction.

\begin{lemma}
\label{lemma-no-good-representatif-compact-object}
There exists a differential graded algebra $(A, \text{d})$ and
a compact object $E$ of $D(A, \text{d})$ such that $E$ cannot
be represented by a finite and graded projective differential
graded $A$-module.
\end{lemma}

\begin{proof}
See discussion above.
\end{proof}







\section{Two differential graded categories}
\label{section-nongraded-differential-graded}

\noindent
In this section we construct two differential graded categories satisfying
axioms (A), (B), and (C) as in
Differential Graded Algebra, Situation \ref{dga-situation-ABC}
whose objects do not come with a $\mathbf{Z}$-grading.

\medskip\noindent
{\bf Example I.} Let $X$ be a topological space. Denote
$\underline{\mathbf{Z}}$ the constant sheaf with value $\mathbf{Z}$.
Let $A$ be an $\underline{\mathbf{Z}}$-torsor. In this setting we say
a sheaf of abelian groups $\mathcal{F}$ is {\it $A$-graded} if given a
local section $a \in A(U)$ there is a projector
$p_a : \mathcal{F}|_U \to \mathcal{F}|_U$ such that
whenever we have a local isomorphism $\underline{\mathbf{Z}}|_U \to A|_U$
then $\mathcal{F}|_U = \bigoplus_{n \in \mathbf{Z}} p_n(\mathcal{F})$.
Another way to say this is that locally on $X$ the abelian sheaf
$\mathcal{F}$ has a $\mathbf{Z}$-grading, but on overlaps the different
choices of gradings differ by a shift in degree given by the transition
functions for the torsor $A$.
We say that a pair $(\mathcal{F}, \text{d})$ is an
{\it $A$-graded complex of abelian sheaves}, if $\mathcal{F}$
is an $A$-graded abelian sheaf and $\text{d} : \mathcal{F} \to \mathcal{F}$
is a differential, i.e., $\text{d}^2 = 0$ such that
$p_{a + 1} \circ \text{d} = \text{d} \circ p_a$ for every local
section $a$ of $A$. In other words, $\text{d}(p_a(\mathcal{F}))$
is contained in $p_{a + 1}(\mathcal{F})$.

\medskip\noindent
Next, consider the category $\mathcal{A}$ with
\begin{enumerate}
\item objects are $A$-graded complexes of abelian sheaves, and
\item for objects $(\mathcal{F}, \text{d})$, $(\mathcal{G}, \text{d})$ we set
$$
\Hom_\mathcal{A}((\mathcal{F}, \text{d}), (\mathcal{G}, \text{d})) =
\bigoplus \Hom^n(\mathcal{F}, \mathcal{G})
$$
where $\Hom^n(\mathcal{F}, \mathcal{G})$ is the group of
maps of abelian sheaves $f$ such that
$f(p_a(\mathcal{F})) \subset p_{a + n}(\mathcal{G})$ for all
local sections $a$ of $A$. As differential we take
$\text{d}(f) = \text{d} \circ f - (-1)^n f \circ \text{d}$, see
Differential Graded Algebra, Example \ref{dga-example-category-complexes}.
\end{enumerate}
We omit the verification that this is indeed a differential graded category
satisfying (A), (B), and (C). All the properties may be verified locally on
$X$ where one just recovers the differential graded category of complexes
of abelian sheaves. Thus we obtain a triangulated category
$K(\mathcal{A})$.

\medskip\noindent
Twisted derived category of $X$. Observe that given an object
$(\mathcal{F}, \text{d})$ of $\mathcal{A}$, there is a well defined
$A$-graded cohomology sheaf $H(\mathcal{F}, \text{d})$.
Hence it is clear what is meant by a quasi-isomorphism
in $K(\mathcal{A})$. We can invert quasi-isomorphisms
to obtain the {\it derived category $D(\mathcal{A})$ of complexes of
$A$-graded sheaves}. If $A$ is the trivial torsor, then $D(\mathcal{A})$
is equal to $D(X)$, but for nonzero torsors, one obtains a kind of
{\it twisted} derived category of $X$.

\medskip\noindent
{\bf Example II.} Let $C$ be a smooth curve over a perfect field $k$ of
characteristic $2$. Then $\Omega_{C/k}$ comes endowed with a canonical
square root. Namely, we can write $\Omega_{C/k} = \mathcal{L}^{\otimes 2}$
such that for every local function $f$ on $C$ the section
$\text{d}(f)$ is equal to $s^{\otimes 2}$ for some local section
$s$ of $\mathcal{L}$. The ``reason'' is that
$$
\text{d}(a_0 + a_1t + \ldots +a_dt^d) =
(\sum\nolimits_{i\text{ odd}} a_i^{1/2} t^{(i - 1)/2})^2\text{d}t
$$
(insert future reference here). This in particular determines a
canonical connection
$$
\nabla_{can} :
\Omega_{C/k}
\longrightarrow
\Omega_{C/k} \otimes_{\mathcal{O}_C} \Omega_{C/k}
$$
whose $2$-curvature is zero (namely, the unique connection such that the
squares have derivative equal to zero).
Observe that the category of vector bundles
with connections is a tensor category, hence we also obtain canonical
connections $\nabla_{can}$ on the invertible sheaves
$\Omega_{C/k}^{\otimes n}$ for all $n \in \mathbf{Z}$.

\medskip\noindent
Let $\mathcal{A}$ be the category with
\begin{enumerate}
\item objects are pairs $(\mathcal{F}, \nabla)$ consisting of a
finite locally free sheaf $\mathcal{F}$ endowed with a connection
$$
\nabla :
\mathcal{F}
\longrightarrow
\mathcal{F} \otimes_{\mathcal{O}_C} \Omega_{C/k}
$$
whose $2$-curvature is zero, and
\item morphisms between $(\mathcal{F}, \nabla_\mathcal{F})$ and
$(\mathcal{G}, \nabla_\mathcal{G})$
are given by
$$
\Hom_\mathcal{A}((\mathcal{F}, \nabla_\mathcal{F}),
(\mathcal{G}, \nabla_\mathcal{G})) =
\bigoplus \Hom_{\mathcal{O}_C}(\mathcal{F},
\mathcal{G} \otimes_{\mathcal{O}_C} \Omega_{C/k}^{\otimes n})
$$
For an element
$f : \mathcal{F} \to \mathcal{G} \otimes \Omega_{C/k}^{\otimes n}$
of degree $n$ we set
$$
\text{d}(f) =
\nabla_{\mathcal{G} \otimes \Omega_{C/k}^{\otimes n}} \circ f +
f \circ \nabla_\mathcal{F}
$$
with suitable identifications.
\end{enumerate}
We omit the verification that this forms a differential graded
category with properties (A), (B), (C). Thus we obtain a triangulated
homotopy category $K(\mathcal{A})$.

\medskip\noindent
If $C = \mathbf{P}^1_k$, then $K(\mathcal{A})$ is the zero category.
However, if $C$ is a smooth proper curve of genus $> 1$, then
$K(\mathcal{A})$ is not zero. Namely, suppose that $\mathcal{N}$
is an invertible sheaf
of degree $0 \leq d < g - 1$ with a nonzero section $\sigma$.
Then set
$(\mathcal{F}, \nabla_\mathcal{F}) = (\mathcal{O}_C, \text{d})$
and
$(\mathcal{G}, \nabla_\mathcal{G}) = (\mathcal{N}^{\otimes 2}, \nabla_{can})$.
We see that
$$
\Hom_\mathcal{A}^n((\mathcal{F}, \nabla_\mathcal{F}),
(\mathcal{G}, \nabla_\mathcal{G})) =
\left\{
\begin{matrix}
0 & \text{if} & n < 0 \\
\Gamma(C, \mathcal{N}^{\otimes 2}) & \text{if} & n = 0 \\
\Gamma(C, \mathcal{N}^{\otimes 2} \otimes \Omega_{C/k}) & \text{if} & n = 1
\end{matrix}
\right.
$$
The first $0$ because the degree of
$\mathcal{N}^{\otimes 2} \otimes \Omega_{C/k}^{\otimes -1}$
is negative by the condition $d < g - 1$. Now, the section
$\sigma^{\otimes 2}$ has derivative equal zero, hence the homomorphism
group
$$
\Hom_{K(\mathcal{A})}((\mathcal{F}, \nabla_\mathcal{F}),
(\mathcal{G}, \nabla_\mathcal{G}))
$$
is nonzero.






\section{The stack of proper algebraic spaces is not algebraic}
\label{section-proper-spaces-not-algebraic}

\noindent
In Quot, Section \ref{quot-section-stack-of-spaces}
we introduced and studied the stack in groupoids
$$
p'_{fp, flat, proper} :
\Spacesstack'_{fp, flat, proper}
\longrightarrow
\Sch_{fppf}
$$
the stack whose category of sections over a scheme $S$
is the category of flat, proper, finitely presented algebraic
spaces over $S$. We proved that this satisfies many of
Artin's axioms. In this section we why this stack
is not algebraic by showing that formal effectiveness
fails in general.

\medskip\noindent
The canonical example uses that the universal deformation space of
an abelian variety of dimension $g$ has $g^2$ formal parameters
whereas any effective formal deformation can be defined over a
complete local ring of dimension $\leq g(g + 1)/2$.
Our example will be constructed by writing down a
suitable non-effective deformation of a K3 surface.
We will only sketch the argument and not give all the details.

\medskip\noindent
Let $k = \mathbf{C}$ be the field of complex numbers.
Let $X \subset \mathbf{P}^3_k$ be a smooth degree
$4$ surface over $k$. We have
$\omega_X \cong \Omega^2_{X/k} \cong \mathcal{O}_X$.
Finally, we have
$\dim_k H^0(X, T_{X/k}) = 0$,
$\dim_k H^1(X, T_{X/k}) = 20$, and
$\dim_k H^2(X, T_{X/k}) = 0$.
Since $L_{X/k} = \Omega_{X/k}$ because $X$ is smooth over $k$,
and since $\Ext^i_{\mathcal{O}_X}(\Omega_{X/k}, \mathcal{O}_X) =
H^i(X, T_{X/k})$, and because we have Cotangent, Lemma
\ref{cotangent-lemma-find-obstruction-ringed-topoi}
we find that there is a universal deformation of $X$
over
$$
k[[x_1, \ldots, x_{20}]]
$$
Suppose that this universal deformation is effective
(as in Artin's Axioms, Section \ref{artin-section-formal-objects}).
Then we would get a flat, proper morphism
$$
f : Y \longrightarrow \Spec(k[[x_1, \ldots, x_{20}]])
$$
where $Y$ is an algebraic space recovering the universal deformation.
This is impossible for the following reason. Since $Y$ is separated
we can find an affine open subscheme $V \subset Y$. Since the special
fibre $X$ of $Y$ is smooth, we see that $f$ is smooth. Hence $Y$
is regular being smooth over regular and it follows
that the complement $D$ of $V$ in $Y$ is an effective Cartier divisor.
Then $\mathcal{O}_Y(D)$ is a nontrivial element of $\Pic(Y)$
(to prove this you show that the complement of a nonempty affine open in
a proper smooth algebraic space over a field is always a nontrivial
in the Picard group and you apply this to the generic fibre of $f$).
Finally, to get a contradiction, we show that $\Pic(Y) = 0$.
Namely, the map $\Pic(Y) \to \Pic(X)$ is injective,
because $H^1(X, \mathcal{O}_X) = 0$ (hence all deformations of
$\mathcal{O}_X$ to $Y \times \Spec(k[[x_i]]/\mathfrak m^n)$
are trivial) and Grothendieck's existence theorem
(which says that coherent modules giving rise to the
same sheaves on thickenings are isomorphic).
If $X$ is general enough, then $\Pic(X) = \mathbf{Z}$
generated by $\mathcal{O}_X(1)$. Hence it suffices to show that
$\mathcal{O}_X(n)$, $n > 0$ does not deform to the first
order neighbourhood\footnote{This argument works as long as
the map $c_1 : \Pic(X) \to H^1(X, \Omega_{X/k})$
is injective, which is true for $k$ any field of characteristic zero
and any smooth hypersurface $X$ of degree $4$ in $\mathbf{P}^3_k$.}.
Consider the cup-product
$$
H^1(X, \Omega_{X/k}) \times H^1(X, T_{X/k})
\longrightarrow H^2(X, \mathcal{O}_X)
$$
This is a nondegenerate pairing by coherent duality.
A computation shows that the Chern class
$c_1(\mathcal{O}_X(n)) \in H^1(X, \Omega_{X/k})$
in Hodge cohomology is nonzero.
Hence there is a first order deformation
whose cup product with $c_1(\mathcal{O}_X(n))$ is nonzero.
Then finally, one shows this cup product is the obstruction
class to lifting.

\begin{lemma}
\label{lemma-proper-spaces-not-algebraic}
The stack in groupoids
$$
p'_{fp, flat, proper} :
\Spacesstack'_{fp, flat, proper}
\longrightarrow
\Sch_{fppf}
$$
whose category of sections over a scheme $S$ is the category of
flat, proper, finitely presented algebraic spaces over $S$
(see Quot, Section \ref{quot-section-stack-of-spaces})
is not an algebraic stack.
\end{lemma}

\begin{proof}
If it was an algebraic stack, then every formal object would be
effective, see Artin's Axioms, Lemma \ref{artin-lemma-effective}.
The discussion above show this is not the case
after base change to $\Spec(\mathbf{C})$.
Hence the conclusion.
\end{proof}






\section{An example of a non-algebraic Hom-stack}
\label{section-non-algebraic-hom-stack}

\noindent
Let $\mathcal{Y}, \mathcal{Z}$ be algebraic stacks over a scheme $S$.
The {\it Hom-stack} $\underline{\Mor}_S(\mathcal{Y}, \mathcal{Z})$
is the stack in groupoids over $S$ whose category of sections over
a scheme $T$ is given by the category
$$
\Mor_T(\mathcal{Y} \times_S T, \mathcal{Z} \times_S T)
$$
whose objects are $1$-morphisms and whose morphisms are $2$-morphisms.
We omit the proof this is indeed a stack in groupoids over
$(\Sch/S)_{fppf}$ (insert future reference here). Of course, in
general the Hom-stack will not be algebraic. In this section we
give an example where it is not true and where $\mathcal{Y}$ is
representable by a proper flat scheme over $S$ and $\mathcal{Z}$
is smooth and proper over $S$.

\medskip\noindent
Let $k$ be an algebraically closed field which is not the algebraic
closure of a finite field. Let $S = \Spec(k[[t]])$
and $S_n = \Spec(k[t]/(t^n)) \subset S$. Let $f : X \to S$ be a map
satisfying the following
\begin{enumerate}
\item $f$ is projective and flat, and its fibres are geometrically
connected curves,
\item the fibre $X_0 = X \times_S S_0$ is a nodal curve with smooth
irreducible components whose dual graph has a loop consisting of
rational curves,
\item $X$ is a regular scheme.
\end{enumerate}
To make such a surface $X$ we can take for example
$$
X\quad :\quad T_0T_1T_2 - t(T_0^3 + T_1^3 + T_2^3) = 0
$$
in $\mathbf{P}^2_{k[[t]]}$. Let $A_0$ be a non-zero abelian variety over $k$
for example an elliptic curve.  Let $A = A_0 \times_{\Spec(k)} S$ be the
constant abelian scheme over $S$ associated to $A_0$. We will show that the
stack $\mathcal{X} = \underline{\Mor}_S(X, [S/A]))$ is not algebraic.

\medskip\noindent
Recall that $[S/A]$ is on the one hand the quotient stack of $A$ acting
trivially on $S$ and on
the other hand equal to the stack classifying fppf $A$-torsors, see
Examples of Stacks, Proposition
\ref{examples-stacks-proposition-equal-quotient-stacks}.
Observe that $[S/A] = [\Spec(k)/A_0] \times_{\Spec(k)} S$. This allows
us to describe the fibre category over a scheme $T$ as follows
\begin{align*}
\mathcal{X}_T
& =
\underline{\Mor}_S(X, [S/A])_T \\
& =
\Mor_T(X \times_S T, [S/A] \times_S T) \\
& =
\Mor_S(X \times_S T, [S/A]) \\
& =
\Mor_{\Spec(k)}(X \times_S T, [\Spec(k)/A_0])
\end{align*}
for any $S$-scheme $T$. In other words, the groupoid $\mathcal{X}_T$
is the groupoid of fppf $A_0$-torsors on $X \times_S T$.
Before we discuss why $\mathcal{X}$ is not an algebraic stack,
we need a few lemmas.

\begin{lemma}
\label{lemma-torsors-over-two-dimensional-regular}
Let $W$ be a two dimensional regular integral Noetherian scheme
with function field $K$. Let $G \to W$ be an abelian scheme.
Then the map $H^1_{fppf}(W, G) \to H^1_{fppf}(\Spec(K), G)$
is injective.
\end{lemma}

\begin{proof}[Sketch of proof]
Let $P \to W$ be an fppf $G$-torsor which is trivial in the generic point.
Then we have a morphism $\Spec(K) \to P$ over $W$ and we can take
its scheme theoretic image $Z \subset P$. Since $P \to W$ is proper
(as a torsor for a proper group algebraic space over $W$)
we see that $Z \to W$ is a proper birational morphism.
By Spaces over Fields, Lemma \ref{spaces-over-fields-lemma-finite-in-codim-1}
the morphism $Z \to W$ is finite away from finitely many closed points
of $W$. By (insert future reference on resolving indeterminacies
of morphisms by blowing quadratic transformations for surfaces)
the irreducible components of the geometric fibres of $Z \to W$
are rational curves. By
More on Groupoids in Spaces, Lemma
\ref{spaces-more-groupoids-lemma-no-nonconstant-morphism-from-P1-to-group}
there are no nonconstant morphisms from rational curves
to group schemes or torsors over such.
Hence $Z \to W$ is finite, whence $Z$ is a scheme and $Z \to W$
is an isomorphism by
Morphisms, Lemma \ref{morphisms-lemma-finite-birational-over-normal}.
In other words, the torsor $P$ is trivial.
\end{proof}

\begin{lemma}
\label{lemma-torsors-over-field-torsion}
Let $G$ be a smooth commutative group algebraic space over a field $K$.
Then $H^1_{fppf}(\Spec(K), G)$ is torsion.
\end{lemma}

\begin{proof}
Every $G$-torsor $P$ over $\Spec(K)$ is smooth over $K$ as a form of $G$.
Hence $P$ has a point over a finite separable extension $L/K$.
Say $[L : K] = n$. Let $[n](P)$ denote the $G$-torsor whose class is $n$
times the class of $P$ in $H^1_{fppf}(\Spec(K), G)$. There is a canonical
morphism
$$
P \times_{\Spec(K)} \ldots \times_{\Spec(K)} P \to [n](P)
$$
of algebraic spaces over $K$. This morphism is symmetric as
$G$ is abelian. Hence it factors through the quotient
$$
(P \times_{\Spec(K)} \ldots \times_{\Spec(K)} P)/S_n
$$
On the other hand, the morphism $\Spec(L) \to P$ defines a morphism
$$
(\Spec(L) \times_{\Spec(K)} \ldots \times_{\Spec(K)} \Spec(L))/S_n
\longrightarrow (P \times_{\Spec(K)} \ldots \times_{\Spec(K)} P)/S_n
$$
and the reader can verify that the scheme on the left has a $K$-rational
point. Thus we see that $[n](P)$ is the trivial torsor.
\end{proof}

\noindent
To prove $\mathcal{X} = \underline{\Mor}_S(X, [S/A])$
is not an algebraic stack, by
Artin's Axioms, Lemma \ref{artin-lemma-effective},
it is enough to show the following.

\begin{lemma}
\label{lemma-not-essentially-surjective}
The canonical map $\mathcal{X}(S) \to \lim \mathcal{X}(S_n)$
is not essentially surjective.
\end{lemma}

\begin{proof}[Sketch of proof]
Unwinding definitions, it is enough to check that
$H^1(X, A_0) \to \lim H^1(X_n, A_0)$ is not surjective.
As $X$ is regular and projective, by
Lemmas \ref{lemma-torsors-over-field-torsion} and
\ref{lemma-torsors-over-two-dimensional-regular}
each $A_0$-torsor over $X$ is torsion.
In particular, the group $H^1(X, A_0)$ is torsion.
It is thus enough to show:
(a) the group $H^1(X_0, A_0)$ is non-torsion, and
(b) the maps $H^1(X_{n + 1}, A_0) \to H^1(X_n, A_0)$ are surjective for all $n$.

\medskip\noindent
Ad (a). One constructs a nontorsion $A_0$-torsor $P_0$ on the nodal
curve $X_0$ by glueing trivial $A_0$-torsors on each component
of $X_0$ using non-torsion points on $A_0$ as the isomorphisms
over the nodes. More precisely, let $x \in X_0$ be a node
which occurs in a loop consisting of rational curves.
Let $X'_0 \to X_0$ be the normalization of $X_0$ in $X_0 \setminus \{x\}$.
Let $x', x'' \in X'_0$ be the two points mapping to $x_0$.
Then we take $A_0 \times_{\Spec(k)} X'_0$ and we identify
$A_0 \times {x'}$ with $A_0 \times \{x''\}$ using translation
$A_0 \to A_0$ by a nontorsion point $a_0 \in A_0(k)$ (there is such
a nontorsion point as $k$ is algebraically closed and not the algebraic
closure of a finite field -- this is actually not trivial to prove).
One can show that the glueing is an algebraic space (in fact one can
show it is a scheme) and that it is an nontorsion $A_0$-torsor over $X_0$.
The reason that it is nontorsion is that if $[n](P_0)$ has a section,
then that section produces a morphism $s : X'_0 \to A_0$ such that
$[n](a_0) = s(x') - s(x'')$ in the group law on $A_0(k)$. However,
since the irreducible components of the loop are rational to
section $s$ is constant on them (
More on Groupoids in Spaces, Lemma
\ref{spaces-more-groupoids-lemma-no-nonconstant-morphism-from-P1-to-group}).
Hence $s(x') = s(x'')$ and we obtain a contradiction.

\medskip\noindent
Ad (b). Deformation theory shows that the obstruction to deforming an
$A_0$-torsor $P_n \to X_n$ to an $A_0$-torsor $P_{n + 1} \to X_{n + 1}$
lies in $H^2(X_0, \omega)$ for a suitable vector bundle $\omega$ on $X_0$.
The latter vanishes as $X_0$ is a curve, proving the claim.
\end{proof}

\begin{proposition}
\label{proposition-nonalghomstack}
The stack $\mathcal{X} = \underline{\Mor}_S(X, [S/A])$ is not algebraic.
\end{proposition}

\begin{proof}
See discussion above.
\end{proof}

\begin{remark}
\label{remark-contradict-aoki}
Proposition \ref{proposition-nonalghomstack} contradicts
\cite[Theorem 1.1]{AokiHomStacks}. The problem is the non-effectivity
of formal objects for $\underline{\Mor}_S(X, [S/A])$. The same problem
is mentioned in the Erratum \cite{AokiHomStacksErr} to
\cite{AokiHomStacks}. Unfortunately, the Erratum goes on
to assert that $\underline{\Mor}_S(\mathcal{Y}, \mathcal{Z})$
is algebraic if $\mathcal{Z}$ is separated, which also contradicts
Proposition \ref{proposition-nonalghomstack} as $[S/A]$ is separated.
\end{remark}








\section{An algebraic stack not satisfying strong formal effectiveness}
\label{section-non-formal-effectiveness}

\noindent
This is \cite[Example 4.12]{Bhatt-Algebraize}.
Let $k$ be an algebraically closed field.
Let $A$ be an abelian variety over $k$.
Assume that $A(k)$ is not torsion (this always holds if $k$
is not the algebraic closure of a finite field).
Let $\mathcal{X} = [\Spec(k)/A]$.
We claim there exists an ideal $I \subset k[x, y]$
such that
$$
\mathcal{X}_{\Spec(k[x, y]^\wedge)}
\longrightarrow
\lim \mathcal{X}_{\Spec(k[x, y]/I^n)}
$$
is not essentially surjective. Namely, let $I$
be the ideal generated by $xy(x + y - 1)$.
Then $X_0 = V(I)$ consists of three copies of $\mathbf{A}^1_k$
glued into a triangle at three points. Hence we can make an infinite order
torsor $P_0$ for $A$ over $X_0$ by taking the trivial torsor
over the irreducible components of $X_0$ and glueing
using translation by nontorsion points.
Exactly as in the proof of Lemma \ref{lemma-not-essentially-surjective}
we can lift $P_0$ to a torsor $P_n$ over $X_n = \Spec(k[x, y]/I^n)$.
Since $k[x, y]^\wedge$ is a two dimensional regular domain
we see that any torsor $P$ for $A$ over $\Spec(k[x, y]^\wedge)$
is torsion (Lemmas \ref{lemma-torsors-over-two-dimensional-regular}
and \ref{lemma-torsors-over-field-torsion}). Hence the system of
torsors is not in the image of the displayed functor.

\begin{lemma}
\label{lemma-non-formal-effectiveness}
Let $k$ be an algebraically closed field which is not the closure
of a finite field. Let $A$ be an abelian variety over $k$.
Let $\mathcal{X} = [\Spec(k)/A]$.
There exists an inverse system of $k$-algebras $R_n$
with surjective transition maps whose kernels are locally nilpotent
and a system $(\xi_n)$ of $\mathcal{X}$ lying over the system
$(\Spec(R_n))$ such that this system is not effective
in the sense of Artin's Axioms, Remark \ref{artin-remark-strong-effectiveness}.
\end{lemma}

\begin{proof}
See discussion above.
\end{proof}








\section{A counter example to Grothendieck's existence theorem}
\label{section-Grothendieck-existence}

\noindent
Let $k$ be a field and let $A = k[[t]]$. Let $X$ be the glueing of
$U = \Spec(A[x])$ and $V = \Spec(A[y])$ by the identification
$$
U \setminus \{0_U\} \longrightarrow V \setminus \{0_V\}
$$
sending $x$ to $y$ where $0_U \in U$ and $O_V \in V$ are the points
corresponding to the maximal ideals $(x, t)$ and $(y, t)$. Set
$A_n = A/(t^n)$ and set $X_n = X \times_{\Spec(A)} \Spec(A_n)$.
Let $\mathcal{F}_n$ be the coherent sheaf on $X_n$
corresponding to the $A_n[x]$-module $A_n[x]/(x) \cong A_n$
and the $A_n[y]$ module $0$ with obvious glueing.
Let $\mathcal{I} \subset \mathcal{O}_X$ be the sheaf of
ideals generate by $t$. Then $(\mathcal{F}_n)$ is an object of
the category $\textit{Coh}_{\text{support proper over } A}(X, \mathcal{I})$
defined in
Cohomology of Schemes, Section \ref{coherent-section-existence-proper-support}.
On the other hand, this object is not in the image of
the functor
Cohomology of Schemes,
Equation (\ref{coherent-equation-completion-functor-proper-over-A}).
Namely, if it where there would be a finite $A[x]$-module $M$,
a finite $A[y]$-module $N$ and an isomorphism $M[1/t] \cong N[1/t]$
such that $M/t^nM \cong A_n[x]/(x)$ and $N/t^nN = 0$ for all $n$.
It is easy to see that this is impossible.

\begin{lemma}
\label{lemma-counter-Grothendieck-existence}
Counter examples to algebraization of coherent sheaves.
\begin{enumerate}
\item Grothendieck's existence theorem as stated in
Cohomology of Schemes, Theorem \ref{coherent-theorem-grothendieck-existence}
is false if we drop the assumption that $X \to \Spec(A)$ is separated.
\item The stack of coherent sheaves $\Cohstack_{X/B}$
of Quot, Theorems \ref{quot-theorem-coherent-algebraic-general} and
\ref{quot-theorem-coherent-algebraic} is in general
not algebraic if we drop the assumption that $X \to S$ is separated
\item The functor $\Quotfunctor_{\mathcal{F}/X/B}$ of
Quot, Proposition \ref{quot-proposition-quot}
is not an algebraic space in general if we drop the assumption
that $X \to B$ is separated.
\end{enumerate}
\end{lemma}

\begin{proof}
Part (1) we saw above. This shows that $\textit{Coh}_{X/A}$ fails
axiom [4] of Artin's Axioms, Section \ref{artin-section-axioms}. Hence it
cannot be an algebraic stack by Artin's Axioms, Lemma
\ref{artin-lemma-effective}.
In this way we see that (2) is true. To see (3), note that
there are compatible surjections $\mathcal{O}_{X_n} \to \mathcal{F}_n$
for all $n$. Thus we see that $\Quotfunctor_{\mathcal{O}_X/X/A}$
fails axiom [4] and we see that (3) is true as before.
\end{proof}






\section{Affine formal algebraic spaces}
\label{section-affine-formal-algebraic-space}

\noindent
Let $K$ be a field and let $(V_i)_{i \in I}$ be a directed inverse
system of nonzero vector spaces over $K$ with surjective transition maps and
with $\lim V_i = 0$, see Section \ref{section-zero-limit}.
Let $R_i = K \oplus V_i$ as $K$-algebra where $V_i$ is
an ideal of square zero. Then $R_i$ is an inverse system of
$K$-algebras with surjective transition maps with nilpotent kernels
and with $\lim R_i = K$. The affine formal
algebraic space $X = \colim \Spec(R_i)$ is an example of an affine
formal algebraic space which is not McQuillan.

\begin{lemma}
\label{lemma-affine-not-mcquillan}
There exists an affine formal algebraic space which is not McQuillan.
\end{lemma}

\begin{proof}
See discussion above.
\end{proof}

\noindent
Let $0 \to W_i \to V_i \to K \to 0$ be a system of exact sequences
as in Section \ref{section-zero-limit}. Let
$A_i = K[V_i]/(ww'; w, w' \in W_i)$.
Then there is a compatible system of surjections $A_i \to K[t]$
with nilpotent kernels and
the transition maps $A_i \to A_j$ are surjective with nilpotent
kernels as well. Recall that $V_i$ is free over $K$ with
basis given by $s \in S_i$. Then, if the characteristic of $K$ is zero,
the degree $d$ part of $A_i$ is free over $K$ with basis given by
$s^d$, $s \in S_i$ each of which map to $t^d$. Hence the inverse system of
the degree $d$ parts of the $A_i$ is isomorphic to the inverse
system of the vector spaces $V_i$. As $\lim V_i = 0$ we conclude that
$\lim A_i = K$, at least when the characteristic of $K$ is zero.
This gives an example of an affine formal algebraic space
whose ``regular functions'' do not separate points.

\begin{lemma}
\label{lemma-affine-formal-functions-do-not-separate-points}
There exists an affine formal algebraic space $X$
whose regular functions do not separate points, in the following sense:
If we write $X = \colim X_\lambda$ as in
Formal Spaces, Definition
\ref{formal-spaces-definition-affine-formal-algebraic-space}
then $\lim \Gamma(X_\lambda, \mathcal{O}_{X_\lambda})$
is a field, but $X_{red}$ has infinitely many points.
\end{lemma}

\begin{proof}
See discussion above.
\end{proof}

\noindent
Let $K$, $I$, and $(V_i)$ be as above. Consider systems
$$
\Phi = (\Lambda, J_i \subset \Lambda, (M_i) \to (V_i))
$$
where $\Lambda$ is an augmented $K$-algebra,
$J_i \subset \Lambda$ for $i \in I$ is an ideal of square zero,
$(M_i) \to (V_i)$ is a map of inverse systems of $K$-vector spaces
such that $M_i \to V_i$ is surjective for each $i$, such that
$M_i$ has a $\Lambda$-module structure, such that the transition maps
$M_i \to M_j$, $i > j$ are $\Lambda$-linear, and such that
$J_j M_i \subset \Ker(M_i \to M_j)$ for $i > j$.
Claim: There exists a system as above such that $M_j = M_i/J_j M_i$
for all $i > j$.

\medskip\noindent
If the claim is true, then we obtain a representable morphism
$$
\colim_{i \in I} \Spec(\Lambda/J_i \oplus M_i)
\longrightarrow
\text{Spf}(\lim \Lambda/J_i)
$$
of affine formal algebraic spaces whose source is not McQuillan
but the target is. Here $\Lambda/J_i \oplus M_i$ has the usual
$\Lambda/J_i$-algebra structure where $M_i$ is an ideal of square zero.
Representability translates exactly into the
condition that $M_i/J_jM_i = M_j$ for $i > j$. The source
of the morphism is not McQuillan as the projections
$\lim_{i \in I} M_i \to M_i$ are not be surjective. This is true
because the maps $\lim V_i \to V_i$ are not surjective
and we have the surjection $M_i \to V_i$. Some details omitted.

\medskip\noindent
Proof of the claim. First, note that there exists at least one
system, namely
$$
\Phi_0 = (K, J_i = (0), (V_i) \xrightarrow{\text{id}} (V_i))
$$
Given a system $\Phi$ we will prove there exists a morphism of systems
$\Phi \to \Phi'$ (morphisms of systems defined in the obvious manner)
such that $\Ker(M_i/J_j M_i \to M_j)$ maps to zero in
$M'_i/J'_j M'_i$. Once this is done we can do the usual trick
of setting $\Phi_n = (\Phi_{n - 1})'$ inductively for $n \geq 1$ and
taking $\Phi = \colim \Phi_n$ to get a system with the desired properties.
Details omitted.

\medskip\noindent
Construction of $\Phi'$ given $\Phi$. Consider the set $U$ of
triples $u = (i, j, \xi)$ where $i > j$ and $\xi \in \Ker(M_i \to M_j)$.
We will let $s, t : U \to I$ denote the maps $s(i, j, \xi) = i$
and $t(i, j, \xi) = j$. Then we set $\xi_u \in M_{s(u)}$ the third
component of $u$. We take
$$
\Lambda' = \Lambda[x_u; u \in U]/(x_u x_{u'}; u, u' \in U)
$$
with augmentation $\Lambda' \to K$ given by the augmentation of $\Lambda$
and sending $x_u$ to zero.
We take $J'_k = J_k \Lambda' + (x_{u,\ t(u) \geq k})$.
We set
$$
M'_i = M_i \oplus \bigoplus\nolimits_{s(u) \geq i} K\epsilon_{i, u}
$$
As transition maps $M'_i \to M'_j$ for $i > j$ we use
the given map $M_i \to M_j$ and we send $\epsilon_{i, u}$ to
$\epsilon_{j, u}$. The map $M'_i \to V_i$ induces the given map
$M_i \to V_i$ and sends $\epsilon_{i, u}$ to zero.
Finally, we let $\Lambda'$ act on $M'_i$ as follows:
for $\lambda \in \Lambda$ we act by the $\Lambda$-module
structure on $M_i$ and via the augmentation $\Lambda \to K$
on $\epsilon_{i, u}$. The element $x_u$ acts as $0$ on $M_i$
for all $i$. Finally, we define
$$
x_u \epsilon_{i, u} = \text{image of }\xi_u\text{ in }M_i
$$
and we set all other products $x_{u'} \epsilon_{i, u}$ equal to zero.
The displayed formula makes sense because $s(u) \geq i$ and
$\xi_u \in M_{s(u)}$. The main things the check are $J'_j M'_i \subset M'_i$
maps to zero in $M'_j$ for $i > j$ and that $\Ker(M_i \to M_j)$
maps to zero in $M'_i/J_j M'_i$. The reason for the last fact
is that $\xi = x_{(i, j, \xi)} \epsilon_{i, (i, j, \xi)} \in J'_j M'_i$
for any $\xi \in \Ker(M_i \to M_j)$.
We omit the details.

\begin{lemma}
\label{lemma-representable-morphism-affine-formal-not-mcquillan-top}
There exists a representable morphism $f : X \to Y$ of
affine formal algebraic spaces with $Y$ McQuillan, but $X$ not
McQuillan.
\end{lemma}

\begin{proof}
See discussion above.
\end{proof}









\section{Flat maps are not directed limits of finitely presented flat maps}
\label{section-flat-not-colimit-flat-finitely-presented}

\noindent
The goal of this section is to give an example of a flat ring map which
is not a filtered colimit of flat and finitely presented ring maps. In
\cite{gabber-nonexcellent} it is shown that if $A$ is a nonexcellent
local ring of dimension $1$ and residue characteristic zero, then the
(flat) ring map $A \to A^\wedge$ to its completion is not a filtered
colimit of finite type flat ring maps. The example in this section
will have a source which is an excellent ring. We encourage the reader
to submit other examples; please email
\href{mailto:stacks.project@gmail.com}{stacks.project@gmail.com}
if you have one.

\medskip\noindent
For the construction, fix a prime $p$, and let
$A = \mathbf{F}_p[x_1, \ldots, x_n]$.
Choose an absolute integral closure $A^+$ of $A$, i.e., $A^+$ is the
integral closure of $A$ in an algebraic closure of its fraction field.
In \cite[\S 6.7]{HHBigCM} it is shown that $A \to A^+$ is flat.

\medskip\noindent
We claim that the $A$-algebra $A^+$ is not a filtered colimit of finitely
presented flat $A$-algebras if $n \geq 3$.

\medskip\noindent
We sketch the argument in the case $n = 3$, and we leave the generalization to
higher $n$ to the reader. It is enough to prove the analogous statement for
the map $R \to R^+$, where $R$ is the strict henselization of $A$ at the
origin and $R^+$ is its absolute integral closure. Observe that $R$
is a henselian regular local ring whose residue field $k$ is an
algebraic closure of $\mathbf{F}_p$.

\medskip\noindent
Choose an ordinary abelian surface $X$ over $k$ and a very ample line bundle
$L$ on $X$. The section ring $\Gamma_*(X, L) = \bigoplus_n H^0(X,L^n)$ is
the coordinate ring of the affine cone over $X$ with respect to $L$. It is
a normal ring for $L$ sufficiently positive. Let $S$ denote the henselization
of $\Gamma_*(X, L)$ at vertex of the cone. Then $S$ is a henselian Noetherian
normal domain of dimension $3$. We obtain a finite injective
map $R \to S$ as the henselization of a Noether normalization
for the finite type $k$-algebra $\Gamma_*(X, L)$. As $R^+$ is an
absolute integral closure of $R$, we can also fix an embedding $S \to R^+$.
Thus $R^+$ is also the absolute integral closure of $S$. To show $R^+$
is not a filtered colimit of flat $R$-algebras, it suffices to show:
\begin{enumerate}
\item If there exists a factorization $S \to P \to R^+$ with $P$ flat
and finite type over $R$, then there exists a factorization
$S \to T \to R^+$ with $T$ finite flat over $R$.
\item For any factorization $S \to T \to R^+$ with $S \to T$ finite,
the ring $T$ is not $R$-flat.
\end{enumerate}
Indeed, since $S$ is finitely presented over $R$, if one could write
$R^+ = \colim_i P_i$ as a filtered colimit of finitely presented flat
$R$-algebras $P_i$, then $S \to R^+$ would factor as $S \to P_i \to R^+$
for $i \gg 0$, which contradicts the above pair of assertions.
Assertion (1) follows from the fact that $R$ is henselian and a
slicing argument, see More on Morphisms, Lemma
\ref{more-morphisms-lemma-qf-fp-flat-neighbourhood-dominates-fppf}.
Part (2) was proven in \cite{BhattSmallCMMod}; for the convenience of
the reader, we recall the argument.

\medskip\noindent
Let $U \subset \Spec(S)$ be the punctured spectrum, so there are natural
maps $X \leftarrow U \subset \Spec(S)$. The first map gives an
identification $H^1(U, \mathcal{O}_U) \simeq H^1(X, \mathcal{O}_X)$.
By passing to the Witt vectors of the perfection and using the
Artin-Schreier sequence\footnote{Here we use that $S$ is a strictly
henselian local ring of characteristic $p$ and hence
$S \to S$, $f \mapsto f^p - f$ is surjective.
Also $S$ is a normal domain and hence
$\Gamma(U, \mathcal{O}_U) = S$. Thus $H^1_\etale(U, \mathbf{Z}/p)$
is the kernel of the map $H^1(U, \mathcal{O}_U) \to H^1(U, \mathcal{O}_U)$
induced by $f \mapsto f^p - f$.}, this gives an identification
$H^1_\etale(U, \mathbf{Z}_p) \simeq H^1_\etale(X, \mathbf{Z}_p)$.
In particular, this group is a finite free $\mathbf{Z}_p$-module of
rank $2$ (since $X$ is ordinary). To get a contradiction assume
there exists an
$R$-flat $T$ as in (2) above. Let $V \subset \Spec(T)$ denote the preimage
of $U$, and write $f : V \to U$ for the induced finite surjective map.
Since $U$ is normal, there is a trace map $f_*\mathbf{Z}_p \to \mathbf{Z}_p$
on $U_\etale$ whose composition with the pullback
$\mathbf{Z}_p \to f_*\mathbf{Z}_p$ is multiplication by $d = \deg(f)$.
Passing to cohomology, and using that $H^1_\etale(U, \mathbf{Z}_p)$
is nontorsion, then shows that $H^1_\etale(V, \mathbf{Z}_p)$ is nonzero.
Since $H^1_\etale(V, \mathbf{Z}_p) \simeq \lim H^1_\etale(V, \mathbf{Z}/p^n)$
as there is no $R^1\lim$ interference, the group
$H^1(V_\etale,\mathbf{Z}/p)$ must be non-zero. Since $T$ is $R$-flat we
have $\Gamma(V, \mathcal{O}_V) = T$ which is strictly henselian and the
Artin-Schreier sequence shows $H^1(V, \mathcal{O}_V) \neq 0$.
This is equivalent to $H^2_\mathfrak m(T) \neq 0$, where
$\mathfrak m \subset R$ is the maximal ideal. Thus, we obtain
a contradiction since $T$ is finite flat (i.e., finite free) as an
$R$-module and $H^2_\mathfrak m(R) = 0$. This contradiction proves (2).

\begin{lemma}
\label{lemma-weird-flat-map}
There exists a commutative ring $A$ and a flat $A$-algebra $B$
which cannot be written as a filtered colimit of finitely
presented flat $A$-algebras. In fact, we may either choose $A$ to
be a finite type $\mathbf{F}_p$-algebra or a $1$-dimensional
Noetherian local ring with residue field of characteristic $0$.
\end{lemma}

\begin{proof}
See discussion above.
\end{proof}




\section{The category of modules modulo torsion modules}
\label{section-serre-quotient-modulo-torsion-modules}

\noindent
The category of torsion groups is a Serre subcategory
(Homology, Definition \ref{homology-definition-serre-subcategory})
of the category of all abelian groups. More generally, for any ring $A$,
the category of torsion $A$-modules is a Serre subcategory of the
category of all $A$-modules, see
More on Algebra, Section \ref{more-algebra-section-abelian-categories-modules}.
If $A$ is a domain, then the quotient category
(Homology, Lemma \ref{homology-lemma-serre-subcategory-is-kernel})
is equivalent to the category of vector spaces over the fraction field.
This follows from the following more general proposition.

\begin{proposition}
\label{proposition-localization-and-serre-quotients}
Let $A$ be a ring. Let $S$ be a multiplicative subset of $A$.
Let $\text{Mod}_A$ denote the category of $A$-modules and $\mathcal{T}$ its
Serre subcategory of modules for which any element is annihilated by some
element of $S$. Then there is a canonical equivalence
$\text{Mod}_A/\mathcal{T} \rightarrow \text{Mod}_{S^{-1}A}$.
\end{proposition}

\begin{proof}
The functor $\text{Mod}_A \to \text{Mod}_{S^{-1}A}$ given by $M
\mapsto M \otimes_A S^{-1}A$ is exact (by Algebra, Proposition
\ref{algebra-proposition-localization-exact})
and maps modules in $\mathcal{T}$ to zero.
Thus, by the universal property given in Homology, Lemma
\ref{homology-lemma-serre-subcategory-is-kernel}, the functor descends to a
functor $\text{Mod}_A/\mathcal{T} \to \text{Mod}_{S^{-1}A}$.

\medskip\noindent
Conversely, any $A$-module $M$ with $M \otimes_A S^{-1}A = 0$
is an object of $\mathcal{T}$, since
$M \otimes_A S^{-1}A \cong S^{-1} M$
(Algebra, Lemma \ref{algebra-lemma-tensor-localization}). Thus
Homology, Lemma \ref{homology-lemma-quotient-by-kernel-exact-functor}
shows that the functor
$\text{Mod}_A/\mathcal{T} \to \text{Mod}_{S^{-1}A}$ is faithful.

\medskip\noindent
Furthermore, this embedding is essentially surjective: a preimage to an
$S^{-1}A$-module $N$ is $N_A$, that is $N$ regarded as an $A$-module, since the
canonical map $N_A \otimes_A S^{-1}A \to N$ which maps $x \otimes a/s$ to
$(a/s) \cdot x$ is an isomorphism of $S^{-1}A$-modules.
\end{proof}

\begin{proposition}
\label{proposition-quotient-by-torsion-modules}
Let $A$ be a ring. Let $Q(A)$ denote its total quotient ring
(as in Algebra, Example \ref{algebra-example-localize-at-prime}). Let
$\text{Mod}_A$ denote the category of $A$-modules and $\mathcal{T}$ its
Serre subcategory of torsion modules. Let $\text{Mod}_{Q(A)}$
denote the category
of $Q(A)$-modules. Then there is a canonical equivalence
$\text{Mod}_A/\mathcal{T} \rightarrow \text{Mod}_{Q(A)}$.
\end{proposition}

\begin{proof}
Follows immediately from applying Proposition
\ref{proposition-localization-and-serre-quotients} to the multiplicative subset
$S = \{f \in A \mid f \text{ is not a zerodivisor in }A\}$, since a module is a
torsion module if and only if all of its elements are each annihilated by some
element of $S$.
\end{proof}

\begin{proposition}
\label{proposition-quotient-by-finitely-generated-torsion-modules}
Let $A$ be a Noetherian integral domain. Let $K$ denote its field of fractions.
Let $\text{Mod}_A^{fg}$ denote the category of finitely generated $A$-modules
and $\mathcal{T}^{fg}$ its Serre subcategory of finitely generated torsion
modules. Then $\text{Mod}_A^{fg}/\mathcal{T}^{fg}$ is canonically equivalent
to the category of finite dimensional $K$-vector spaces.
\end{proposition}

\begin{proof}
The equivalence given in Proposition
\ref{proposition-quotient-by-torsion-modules} restricts along the embedding
$\text{Mod}_A^{fg}/\mathcal{T}^{fg} \to \text{Mod}_A/\mathcal{T}$ to an
equivalence $\text{Mod}_A^{fg}/\mathcal{T}^{fg} \to \text{Vect}_K^{fd}$.
The Noetherian assumption guarantees that $\text{Mod}_A^{fg}$ is an
abelian category (see
More on Algebra, Section \ref{more-algebra-section-abelian-categories-modules})
and that the canonical functor
$\text{Mod}_A^{fg}/\mathcal{T}^{fg} \to \text{Mod}_A/\mathcal{T}$
is full (else torsion submodules of finitely
generated modules might not be objects of $\mathcal{T}^{fg}$).
\end{proof}

\begin{proposition}
\label{proposition-quotient-abelian-groups-by-torsion-groups}
The quotient of the category of abelian groups modulo its
Serre subcategory of torsion groups is the category of
$\mathbf{Q}$-vector spaces.
\end{proposition}

\begin{proof}
The claim follows directly from
Proposition \ref{proposition-quotient-by-torsion-modules}.
\end{proof}




\section{Different colimit topologies}
\label{section-colimit-topology}

\noindent
This example is \cite[Example 1.2, page 553]{TSH}. Let
$G_n = \mathbf{Q} \times \mathbf{R}^n$, $n \geq 1$ seen as a topological group
for addition endowed with the usual (Euclidean) topology. Consider the closed
embeddings $G_n \to G_{n + 1}$ mapping $(x_0, \ldots, x_n)$ to
$(x_0, \ldots, x_n, 0)$. We claim that $G = \colim G_n$ endowed with the
topology
$$
U \subset G\text{ open} \Leftrightarrow G_n \cap U\text{ open }\forall n
$$
is not a topological group.

\medskip\noindent
To see this we consider the set
$$
U = \{(x_0, x_1, x_2, \ldots)\text{ such that }
|x_j| < |\cos(jx_0)| \text{ for } j > 0\}
$$
Using that $jx_0$ is never an integral multiple of $\pi/2$ as $\pi$
is not rational it is easy to show that $U \cap G_n$ is open. Since
$0 \in U$, if the topology above made $G$ into a topological group,
then there would be an open neighbourhood $V \subset G$ of $0$
such that $V + V \subset U$. Then, for every $j \geq 0$ there would
exist $\epsilon_j > 0$ such that $(0, \ldots, 0, x_j, 0, \ldots) \in V$
for $|x_j| < \epsilon_j$. Since $V + V \subset U$ we would have
$$
(x_0, 0, \ldots, 0, x_j, 0, \ldots) \in U
$$
for $|x_0| < \epsilon_0$ and $|x_j| < \epsilon_j$. However, if we
take $j$ large enough such that $j \epsilon_0 > \pi/2$, then we can
choose $x_0 \in \mathbf{Q}$ such that $|\cos(jx_0)|$ is smaller than
$\epsilon_j$, hence there exists an $x_j$ with
$|\cos(jx_0)| < |x_j| < \epsilon_j$. This contradiction proves the claim.

\begin{lemma}
\label{lemma-colimit-topology}
There exists a system $G_1 \to G_2 \to G_3 \to \ldots$ of (abelian)
topological groups such that $\colim G_n$ taken in the category of
topological spaces is different from $\colim G_n$ taken in the category
of topological groups.
\end{lemma}

\begin{proof}
See discussion above.
\end{proof}






\section{Universally submersive but not V covering}
\label{section-universally-submersive-not-V}

\noindent
Let $A$ be a valuation ring. Let $\mathfrak p \subset A$ be a prime
ideal which is neither the minimal prime nor the maximal ideal.
(A good case to keep in mind is when $A$ has three prime ideals
and $\mathfrak p$ is the one in the ``middle''.)
Consider the morphism of affine schemes
$$
\Spec(A_\mathfrak p) \amalg \Spec(A/\mathfrak p) \longrightarrow \Spec(A)
$$
We claim this is universally submersive. In order to prove this, let
$\Spec(B) \to \Spec(A)$ be a morphism of affine schemes given by the ring
map $A \to B$. Then we have to show that
$$
\Spec(B_\mathfrak p) \amalg \Spec(B/\mathfrak pB) \to \Spec(B)
$$
is submersive. First of all it is surjective. Next, suppose that
$T \subset \Spec(B)$ is a subset such that
$T_1 = \Spec(B_\mathfrak p) \cap T$ and $T_2 = \Spec(B/\mathfrak p B) \cap T$
are closed. Then we see that $T$ is the image of the spectrum
of a $B$-algebra because both $T_1$ and $T_2$ are spectra of $B$-algebras.
Hence to show that $T$ is closed it suffices to show that
$T$ is stable under specialization, see
Algebra, Lemma \ref{algebra-lemma-image-stable-specialization-closed}.
To see this, suppose that $p \leadsto q$ is a specialization
of points in $\Spec(B)$ with $p \in T$.
Let $A'$ be a valuation ring and let $\Spec(A') \to \Spec(B)$
be a morphism such that the generic point $\eta$ of $\Spec(A')$
maps to $p$ and the closed point $s$ of $\Spec(A')$ maps to $q$, see
Schemes, Lemma \ref{schemes-lemma-points-specialize}.
Observe that the image of the composition $\gamma : \Spec(A') \to \Spec(A)$
is exactly the set of points $\xi \in \Spec(A)$ with
$\gamma(\eta) \leadsto \xi \leadsto \gamma(s)$ (details omitted).
If $\mathfrak p \not \in \Im(\gamma)$, then we see that
either both $p, q \in \Spec(B_\mathfrak p)$ or
both $p, q \in \Spec(B/\mathfrak pB)$.
In this case the fact that $T_1$, resp.\ $T_2$ is closed
implies that $q \in T_1$, resp.\ $q \in T_2$ and hence $q \in T$.
Finally, suppose $\mathfrak p \in \Im(\gamma)$, say
$\mathfrak p = \gamma(r)$. Then we have specializations
$p \leadsto r$ and $r \leadsto q$. In this case
$p, r \in \Spec(B_\mathfrak p)$ and $r, q \in \Spec(B/\mathfrak pB)$.
Then we fist conclude $r \in T_1 \subset T$, then
$r \in T_2$ as $r$ maps to $\mathfrak p$, and then $q \in T_2 \subset T$
as desired.

\medskip\noindent
On the other hand, we claim that the singleton family
$$
\{\Spec(A_\mathfrak p) \amalg \Spec(A/\mathfrak p) \longrightarrow \Spec(A)\}
$$
is not a V covering. See
Topologies, Definition \ref{topologies-definition-V-covering}.
Namely, if it where a V covering, there would
be an extension of valuation ring $A \subset B$ such that
$\Spec(B) \to \Spec(A)$ factors through
$\Spec(A_\mathfrak p) \amalg \Spec(A/\mathfrak p)$.
This would imply $\Spec(A')$ is disconnected which is absurd.

\begin{lemma}
\label{lemma-universally-submersive-not-V}
There exists a morphism $X \to Y$ of affine schemes
which is universally submersive such that $\{X \to Y\}$
is not a V covering.
\end{lemma}

\begin{proof}
See discussion above.
\end{proof}









\section{The spectrum of the integers is not quasi-compact}
\label{section-canonical}

\noindent
Of course the title of this section doesn't refer to
the spectrum of the integers as a topological space,
because any spectrum is quasi-compact as a topological space
(Algebra, Lemma \ref{algebra-lemma-quasi-compact}).
No, it refers to the spectrum of the integers in the
canonical topology on the category of schemes, and the
definition of a quasi-compact object in a site
(Sites, Definition \ref{sites-definition-quasi-compact}).

\medskip\noindent
Let $U$ be a nonprincipal ultrafilter on the set $P$ of prime numbers.
For a subset $T \subset P$ we denote $T^c = P \setminus T$ the complement.
For $A \in U$ let $S_A \subset \mathbf{Z}$ be the multiplicative
subset generated by $p \in A$. Set
$$
\mathbf{Z}_A = S_A^{-1}\mathbf{Z}
$$
Observe that $\Spec(\mathbf{Z}_A) = \{(0)\} \cup A^c \subset \Spec(\mathbf{Z})$
if we think of $P$ as the set of closed points of $\Spec(\mathbf{Z})$.
If $A, B \in U$, then $A \cap B \in U$ and $A \cup B \in U$
and we have
$$
\mathbf{Z}_{A \cap B} =
\mathbf{Z}_A \times_{\mathbf{Z}_{A \cup B}} \mathbf{Z}_B
$$
(fibre product of rings). In particular, for any
integer $n$ and elements $A_1, \ldots, A_n \in U$ the morphisms
$$
\Spec(\mathbf{Z}_{A_1}) \amalg \ldots \amalg \Spec(\mathbf{Z}_{A_n})
\longrightarrow \Spec(\mathbf{Z})
$$
factors through $\Spec(\mathbf{Z}[1/p])$ for some $p$
(namely for any $p \in A_1 \cap \ldots \cap A_n$).
We conclude that the family of flat morphisms
$\{\Spec(\mathbf{Z}_A) \to \Spec(\mathbf{Z})\}_{A \in U}$ is jointly
surjective, but no finite subset is.

\medskip\noindent
For a $\mathbf{Z}$-module $M$ we set
$$
M_A = S_A^{-1}M = M \otimes_{\mathbf{Z}} \mathbf{Z}_A
$$
Claim I: for every $\mathbf{Z}$-module $M$ we have
$$
M =
\text{Equalizer}\left(
\xymatrix{
\prod\nolimits_{A \in U} M_A \ar@<1ex>[r] \ar@<-1ex>[r] &
\prod\nolimits_{A, B \in U} M_{A \cup B}
}
\right)
$$
First, assume $M$ is torsion free. Then $M_A \subset M_P$ for all $A \in U$.
Hence we see that we have to prove
$$
M = \bigcap\nolimits_{A \in U} M_A\text{ inside }M_P = M \otimes \mathbf{Q}
$$
Namely, since $U$ is nonprincipal, for any prime $p$ we have
$\{p\}^c \in U$. Also, $M_{\{p\}^c} = M_{(p)}$ is equal to the
localization at the prime $(p)$.
Thus the above is clear because already $M_{(2)} \cap M_{(3)} = M$.
Next, assume $M$ is torsion. Then we have
$$
M = \bigoplus\nolimits_{p \in P} M[p^\infty]
$$
and correspondingly we have
$$
M_A = \bigoplus\nolimits_{p \not \in A} M[p^\infty]
$$
because we are localizing at the primes in $A$. Suppose that
$(x_A) \in \prod M_A$ is in the equalizer. Denote
$x_p = x_{\{p\}^c} \in M[p^{\infty}]$. Then the equalizer
property says
$$
x_A = (x_p)_{p \not \in A}
$$
and in particular it says that $x_p$ is zero for all but a finite
number of $p \not \in A$. To finish the proof in the torsion case
it suffices to show that $x_p$ is zero for all but a finite number
of primes $p$. If not write $\{p \in P \mid x_p \not = 0\} = T \amalg T'$
as the disjoint union of two infinite sets. Then either
$T \not \in U$ or $T' \not \in U$ because $U$ is an ultrafilter
(namely if both $T, T'$ are in $U$ then $U$ contains $T \cap T' = \emptyset$
which is not allowed). Say $T \not \in U$. Then $T = A^c$ and this
contradicts the finiteness mentioned above.
Finally, suppose that $M$ is a general module. Then we look
at the short exact sequence
$$
0 \to M_{tors} \to M \to M/M_{tors} \to 0
$$
and we look at the following large diagram
$$
\xymatrix{
M_{tors} \ar[r] \ar[d] &
\prod\nolimits_{A \in U} M_{tors, A} \ar@<1ex>[r] \ar@<-1ex>[r] \ar[d] &
\prod\nolimits_{A, B \in U} M_{tors, A \cup B} \ar[d] \\
M \ar[r] \ar[d] &
\prod\nolimits_{A \in U} M_A \ar@<1ex>[r] \ar@<-1ex>[r] \ar[d] &
\prod\nolimits_{A, B \in U} M_{A \cup B} \ar[d] \\
M/M_{tors} \ar[r] &
\prod\nolimits_{A \in U} (M/M_{tors})_A \ar@<1ex>[r] \ar@<-1ex>[r] &
\prod\nolimits_{A, B \in U} (M/M_{tors})_{A \cup B} \\
}
$$
Doing a diagram chase using exactness of the columns and the result
for the torsion module $M_{tors}$ and the torsion free module
$M/M_{tors}$ proving Claim I for $M$.
This gives an example of the phenomenon in the following lemma.

\begin{lemma}
\label{lemma-non-fpqc-descent}
There exists a ring $A$ and an infinite family of flat ring maps
$\{A \to A_i\}_{i \in I}$ such that for every $A$-module $M$ 
$$
M =
\text{Equalizer}\left(
\xymatrix{
\prod\nolimits_{i \in I} M \otimes_A A_i \ar@<1ex>[r] \ar@<-1ex>[r] &
\prod\nolimits_{i, j \in I} M \otimes_A A_i \otimes_A A_j
}
\right)
$$
but there is no finite subfamily where the same thing is true.
\end{lemma}

\begin{proof}
See discussion above.
\end{proof}

\noindent
We continue working with our nonprincipal ultrafilter $U$ on the set $P$
of prime numbers.
Let $R$ be a ring. Denote $R_A = S_A^{-1}R = R \otimes \mathbf{Z}_A$
for $A \in U$.
Claim II: given closed subsets $T_A \subset \Spec(R_A)$, $A \in U$
such that
$$
(\Spec(R_{A \cup B}) \to \Spec(R_A))^{-1}T_A =
(\Spec(R_{A \cup B}) \to \Spec(R_B))^{-1}T_B
$$
for all $A, B \in U$, there is a closed subset $T \subset \Spec(R)$
with $T_A = (\Spec(R_A) \to \Spec(R))^{-1}(T)$ for all $A \in U$.
Let $I_A \subset R_A$ for $A \in U$ be the radical ideal cutting out $T_A$.
Then the glueing condition implies
$S_{A \cup B}^{-1}I_A = S_{A \cup B}^{-1}I_B$
in $R_{A \cup B}$ for all $A, B \in U$ (because localization preserves
being a radical ideal). Let $I' \subset R$ be the set of elements
mapping into $I_P \subset R_P = R \otimes \mathbf{Q}$.
Then we see for $A \in U$ that
\begin{enumerate}
\item $I_A \subset I'_A = S_A^{-1}I'$, and
\item $M_A = I'_A/I_A$ is a torsion module.
\end{enumerate}
Of course we obtain canonical identifications
$S_{A \cup B}^{-1}M_A = S_{A \cup B}^{-1}M_B$ for $A, B \in U$.
Decomposing the torsion modules $M_A$ into their $p$-primary
components, the reader easily shows that there exist $p$-power torsion
$R$-modules $M_p$ such that
$$
M_A = \bigoplus\nolimits_{p \not \in A} M_p
$$
compatible with the canonical identifications given above.
Setting $M = \bigoplus_{p \in P} M_p$ we find canonical isomorphisms
$M_A = S_A^{-1}M$ compatible with the above canonical identifications.
Then we get a canonical map
$$
I' \longrightarrow M
$$
of $R$-modules wich recovers the map $I_A \to M_A$ for all $A \in U$.
This is true by all the compatibilities mentioned above and the
claim proved previously that $M$ is the equalizer of the two
maps from $\prod_{A \in U} M_A$ to $\prod_{A, B \in U} M_{A \cup B}$.
Let $I = \Ker(I' \to M)$. Then $I$ is an ideal and
$T = V(I)$ is a closed subset which
recovers the closed subsets $T_A$ for all $A \in U$. This proves Claim II.

\begin{lemma}
\label{lemma-Z-not-quasi-compact}
The scheme $\Spec(\mathbf{Z})$ is not quasi-compact
in the canonical topology on the category of schemes.
\end{lemma}

\begin{proof}
With notation as above consider the family of morphisms
$$
\mathcal{W} = \{\Spec(\mathbf{Z}_A) \to \Spec(\mathbf{Z})\}_{A \in U}
$$
By Descent, Lemma \ref{descent-lemma-universal-effective-epimorphism}
and the two claims proved above
this is a universal effective epimorphism.
In any category with fibre products, the universal effective epimorphisms
give $\mathcal{C}$ the structure of a site (modulo some set theoretical
issues which are easy to fix) defining the canonical topology.
Thus $\mathcal{W}$ is a covering for the canonical topology.
On the other hand, we have seen above that any finite subfamily
$$
\{\Spec(\mathbf{Z}_{A_i}) \to \Spec(\mathbf{Z})\}_{i = 1, \ldots, n},\quad
n \in \mathbf{N}, A_1, \ldots, A_n \in U
$$
factors through $\Spec(\mathbf{Z}[1/p])$ for some $p$.
Hence this finite family cannot be a universal effective epimorphism
and more generally no universal effective epimorphism
$\{g_j : T_j \to \Spec(\mathbf{Z})\}$ can refine
$\{\Spec(\mathbf{Z}_{A_i}) \to \Spec(\mathbf{Z})\}_{i = 1, \ldots, n}$.
By Sites, Definition \ref{sites-definition-quasi-compact}
this means that $\Spec(\mathbf{Z})$ is not quasi-compact
in the canonical topology. To see that our notion of quasi-compactness
agrees with the usual topos theoretic definition, see
Sites, Lemma \ref{sites-lemma-quasi-compact}.
\end{proof}







\begin{multicols}{2}[\section{Other chapters}]
\noindent
Preliminaries
\begin{enumerate}
\item \hyperref[introduction-section-phantom]{Introduction}
\item \hyperref[conventions-section-phantom]{Conventions}
\item \hyperref[sets-section-phantom]{Set Theory}
\item \hyperref[categories-section-phantom]{Categories}
\item \hyperref[topology-section-phantom]{Topology}
\item \hyperref[sheaves-section-phantom]{Sheaves on Spaces}
\item \hyperref[sites-section-phantom]{Sites and Sheaves}
\item \hyperref[stacks-section-phantom]{Stacks}
\item \hyperref[fields-section-phantom]{Fields}
\item \hyperref[algebra-section-phantom]{Commutative Algebra}
\item \hyperref[brauer-section-phantom]{Brauer Groups}
\item \hyperref[homology-section-phantom]{Homological Algebra}
\item \hyperref[derived-section-phantom]{Derived Categories}
\item \hyperref[simplicial-section-phantom]{Simplicial Methods}
\item \hyperref[more-algebra-section-phantom]{More on Algebra}
\item \hyperref[smoothing-section-phantom]{Smoothing Ring Maps}
\item \hyperref[modules-section-phantom]{Sheaves of Modules}
\item \hyperref[sites-modules-section-phantom]{Modules on Sites}
\item \hyperref[injectives-section-phantom]{Injectives}
\item \hyperref[cohomology-section-phantom]{Cohomology of Sheaves}
\item \hyperref[sites-cohomology-section-phantom]{Cohomology on Sites}
\item \hyperref[dga-section-phantom]{Differential Graded Algebra}
\item \hyperref[dpa-section-phantom]{Divided Power Algebra}
\item \hyperref[sdga-section-phantom]{Differential Graded Sheaves}
\item \hyperref[hypercovering-section-phantom]{Hypercoverings}
\end{enumerate}
Schemes
\begin{enumerate}
\setcounter{enumi}{25}
\item \hyperref[schemes-section-phantom]{Schemes}
\item \hyperref[constructions-section-phantom]{Constructions of Schemes}
\item \hyperref[properties-section-phantom]{Properties of Schemes}
\item \hyperref[morphisms-section-phantom]{Morphisms of Schemes}
\item \hyperref[coherent-section-phantom]{Cohomology of Schemes}
\item \hyperref[divisors-section-phantom]{Divisors}
\item \hyperref[limits-section-phantom]{Limits of Schemes}
\item \hyperref[varieties-section-phantom]{Varieties}
\item \hyperref[topologies-section-phantom]{Topologies on Schemes}
\item \hyperref[descent-section-phantom]{Descent}
\item \hyperref[perfect-section-phantom]{Derived Categories of Schemes}
\item \hyperref[more-morphisms-section-phantom]{More on Morphisms}
\item \hyperref[flat-section-phantom]{More on Flatness}
\item \hyperref[groupoids-section-phantom]{Groupoid Schemes}
\item \hyperref[more-groupoids-section-phantom]{More on Groupoid Schemes}
\item \hyperref[etale-section-phantom]{\'Etale Morphisms of Schemes}
\end{enumerate}
Topics in Scheme Theory
\begin{enumerate}
\setcounter{enumi}{41}
\item \hyperref[chow-section-phantom]{Chow Homology}
\item \hyperref[intersection-section-phantom]{Intersection Theory}
\item \hyperref[pic-section-phantom]{Picard Schemes of Curves}
\item \hyperref[weil-section-phantom]{Weil Cohomology Theories}
\item \hyperref[adequate-section-phantom]{Adequate Modules}
\item \hyperref[dualizing-section-phantom]{Dualizing Complexes}
\item \hyperref[duality-section-phantom]{Duality for Schemes}
\item \hyperref[discriminant-section-phantom]{Discriminants and Differents}
\item \hyperref[derham-section-phantom]{de Rham Cohomology}
\item \hyperref[local-cohomology-section-phantom]{Local Cohomology}
\item \hyperref[algebraization-section-phantom]{Algebraic and Formal Geometry}
\item \hyperref[curves-section-phantom]{Algebraic Curves}
\item \hyperref[resolve-section-phantom]{Resolution of Surfaces}
\item \hyperref[models-section-phantom]{Semistable Reduction}
\item \hyperref[functors-section-phantom]{Functors and Morphisms}
\item \hyperref[equiv-section-phantom]{Derived Categories of Varieties}
\item \hyperref[pione-section-phantom]{Fundamental Groups of Schemes}
\item \hyperref[etale-cohomology-section-phantom]{\'Etale Cohomology}
\item \hyperref[crystalline-section-phantom]{Crystalline Cohomology}
\item \hyperref[proetale-section-phantom]{Pro-\'etale Cohomology}
\item \hyperref[relative-cycles-section-phantom]{Relative Cycles}
\item \hyperref[more-etale-section-phantom]{More \'Etale Cohomology}
\item \hyperref[trace-section-phantom]{The Trace Formula}
\end{enumerate}
Algebraic Spaces
\begin{enumerate}
\setcounter{enumi}{64}
\item \hyperref[spaces-section-phantom]{Algebraic Spaces}
\item \hyperref[spaces-properties-section-phantom]{Properties of Algebraic Spaces}
\item \hyperref[spaces-morphisms-section-phantom]{Morphisms of Algebraic Spaces}
\item \hyperref[decent-spaces-section-phantom]{Decent Algebraic Spaces}
\item \hyperref[spaces-cohomology-section-phantom]{Cohomology of Algebraic Spaces}
\item \hyperref[spaces-limits-section-phantom]{Limits of Algebraic Spaces}
\item \hyperref[spaces-divisors-section-phantom]{Divisors on Algebraic Spaces}
\item \hyperref[spaces-over-fields-section-phantom]{Algebraic Spaces over Fields}
\item \hyperref[spaces-topologies-section-phantom]{Topologies on Algebraic Spaces}
\item \hyperref[spaces-descent-section-phantom]{Descent and Algebraic Spaces}
\item \hyperref[spaces-perfect-section-phantom]{Derived Categories of Spaces}
\item \hyperref[spaces-more-morphisms-section-phantom]{More on Morphisms of Spaces}
\item \hyperref[spaces-flat-section-phantom]{Flatness on Algebraic Spaces}
\item \hyperref[spaces-groupoids-section-phantom]{Groupoids in Algebraic Spaces}
\item \hyperref[spaces-more-groupoids-section-phantom]{More on Groupoids in Spaces}
\item \hyperref[bootstrap-section-phantom]{Bootstrap}
\item \hyperref[spaces-pushouts-section-phantom]{Pushouts of Algebraic Spaces}
\end{enumerate}
Topics in Geometry
\begin{enumerate}
\setcounter{enumi}{81}
\item \hyperref[spaces-chow-section-phantom]{Chow Groups of Spaces}
\item \hyperref[groupoids-quotients-section-phantom]{Quotients of Groupoids}
\item \hyperref[spaces-more-cohomology-section-phantom]{More on Cohomology of Spaces}
\item \hyperref[spaces-simplicial-section-phantom]{Simplicial Spaces}
\item \hyperref[spaces-duality-section-phantom]{Duality for Spaces}
\item \hyperref[formal-spaces-section-phantom]{Formal Algebraic Spaces}
\item \hyperref[restricted-section-phantom]{Algebraization of Formal Spaces}
\item \hyperref[spaces-resolve-section-phantom]{Resolution of Surfaces Revisited}
\end{enumerate}
Deformation Theory
\begin{enumerate}
\setcounter{enumi}{89}
\item \hyperref[formal-defos-section-phantom]{Formal Deformation Theory}
\item \hyperref[defos-section-phantom]{Deformation Theory}
\item \hyperref[cotangent-section-phantom]{The Cotangent Complex}
\item \hyperref[examples-defos-section-phantom]{Deformation Problems}
\end{enumerate}
Algebraic Stacks
\begin{enumerate}
\setcounter{enumi}{93}
\item \hyperref[algebraic-section-phantom]{Algebraic Stacks}
\item \hyperref[examples-stacks-section-phantom]{Examples of Stacks}
\item \hyperref[stacks-sheaves-section-phantom]{Sheaves on Algebraic Stacks}
\item \hyperref[criteria-section-phantom]{Criteria for Representability}
\item \hyperref[artin-section-phantom]{Artin's Axioms}
\item \hyperref[quot-section-phantom]{Quot and Hilbert Spaces}
\item \hyperref[stacks-properties-section-phantom]{Properties of Algebraic Stacks}
\item \hyperref[stacks-morphisms-section-phantom]{Morphisms of Algebraic Stacks}
\item \hyperref[stacks-limits-section-phantom]{Limits of Algebraic Stacks}
\item \hyperref[stacks-cohomology-section-phantom]{Cohomology of Algebraic Stacks}
\item \hyperref[stacks-perfect-section-phantom]{Derived Categories of Stacks}
\item \hyperref[stacks-introduction-section-phantom]{Introducing Algebraic Stacks}
\item \hyperref[stacks-more-morphisms-section-phantom]{More on Morphisms of Stacks}
\item \hyperref[stacks-geometry-section-phantom]{The Geometry of Stacks}
\end{enumerate}
Topics in Moduli Theory
\begin{enumerate}
\setcounter{enumi}{107}
\item \hyperref[moduli-section-phantom]{Moduli Stacks}
\item \hyperref[moduli-curves-section-phantom]{Moduli of Curves}
\end{enumerate}
Miscellany
\begin{enumerate}
\setcounter{enumi}{109}
\item \hyperref[examples-section-phantom]{Examples}
\item \hyperref[exercises-section-phantom]{Exercises}
\item \hyperref[guide-section-phantom]{Guide to Literature}
\item \hyperref[desirables-section-phantom]{Desirables}
\item \hyperref[coding-section-phantom]{Coding Style}
\item \hyperref[obsolete-section-phantom]{Obsolete}
\item \hyperref[fdl-section-phantom]{GNU Free Documentation License}
\item \hyperref[index-section-phantom]{Auto Generated Index}
\end{enumerate}
\end{multicols}


\bibliography{my}
\bibliographystyle{amsalpha}

\end{document}
