\IfFileExists{stacks-project.cls}{%
\documentclass{stacks-project}
}{%
\documentclass{amsart}
}

% For dealing with references we use the comment environment
\usepackage{verbatim}
\newenvironment{reference}{\comment}{\endcomment}
%\newenvironment{reference}{}{}
\newenvironment{slogan}{\comment}{\endcomment}
\newenvironment{history}{\comment}{\endcomment}

% For commutative diagrams we use Xy-pic
\usepackage[all]{xy}

% We use 2cell for 2-commutative diagrams.
\xyoption{2cell}
\UseAllTwocells

% We use multicol for the list of chapters between chapters
\usepackage{multicol}

% This is generall recommended for better output
\usepackage[T1]{fontenc}

% For cross-file-references
\usepackage{xr-hyper}

% Package for hypertext links:
\usepackage{hyperref}

% For any local file, say "hello.tex" you want to link to please
% use \externaldocument[hello-]{hello}
\externaldocument[introduction-]{introduction}
\externaldocument[conventions-]{conventions}
\externaldocument[sets-]{sets}
\externaldocument[categories-]{categories}
\externaldocument[topology-]{topology}
\externaldocument[sheaves-]{sheaves}
\externaldocument[sites-]{sites}
\externaldocument[stacks-]{stacks}
\externaldocument[fields-]{fields}
\externaldocument[algebra-]{algebra}
\externaldocument[brauer-]{brauer}
\externaldocument[homology-]{homology}
\externaldocument[derived-]{derived}
\externaldocument[simplicial-]{simplicial}
\externaldocument[more-algebra-]{more-algebra}
\externaldocument[smoothing-]{smoothing}
\externaldocument[modules-]{modules}
\externaldocument[sites-modules-]{sites-modules}
\externaldocument[injectives-]{injectives}
\externaldocument[cohomology-]{cohomology}
\externaldocument[sites-cohomology-]{sites-cohomology}
\externaldocument[dga-]{dga}
\externaldocument[dpa-]{dpa}
\externaldocument[hypercovering-]{hypercovering}
\externaldocument[schemes-]{schemes}
\externaldocument[constructions-]{constructions}
\externaldocument[properties-]{properties}
\externaldocument[morphisms-]{morphisms}
\externaldocument[coherent-]{coherent}
\externaldocument[divisors-]{divisors}
\externaldocument[limits-]{limits}
\externaldocument[varieties-]{varieties}
\externaldocument[topologies-]{topologies}
\externaldocument[descent-]{descent}
\externaldocument[perfect-]{perfect}
\externaldocument[more-morphisms-]{more-morphisms}
\externaldocument[flat-]{flat}
\externaldocument[groupoids-]{groupoids}
\externaldocument[more-groupoids-]{more-groupoids}
\externaldocument[etale-]{etale}
\externaldocument[chow-]{chow}
\externaldocument[intersection-]{intersection}
\externaldocument[pic-]{pic}
\externaldocument[adequate-]{adequate}
\externaldocument[dualizing-]{dualizing}
\externaldocument[duality-]{duality}
\externaldocument[discriminant-]{discriminant}
\externaldocument[local-cohomology-]{local-cohomology}
\externaldocument[curves-]{curves}
\externaldocument[resolve-]{resolve}
\externaldocument[models-]{models}
\externaldocument[pione-]{pione}
\externaldocument[etale-cohomology-]{etale-cohomology}
\externaldocument[proetale-]{proetale}
\externaldocument[crystalline-]{crystalline}
\externaldocument[spaces-]{spaces}
\externaldocument[spaces-properties-]{spaces-properties}
\externaldocument[spaces-morphisms-]{spaces-morphisms}
\externaldocument[decent-spaces-]{decent-spaces}
\externaldocument[spaces-cohomology-]{spaces-cohomology}
\externaldocument[spaces-limits-]{spaces-limits}
\externaldocument[spaces-divisors-]{spaces-divisors}
\externaldocument[spaces-over-fields-]{spaces-over-fields}
\externaldocument[spaces-topologies-]{spaces-topologies}
\externaldocument[spaces-descent-]{spaces-descent}
\externaldocument[spaces-perfect-]{spaces-perfect}
\externaldocument[spaces-more-morphisms-]{spaces-more-morphisms}
\externaldocument[spaces-flat-]{spaces-flat}
\externaldocument[spaces-groupoids-]{spaces-groupoids}
\externaldocument[spaces-more-groupoids-]{spaces-more-groupoids}
\externaldocument[bootstrap-]{bootstrap}
\externaldocument[spaces-pushouts-]{spaces-pushouts}
\externaldocument[spaces-chow-]{spaces-chow}
\externaldocument[groupoids-quotients-]{groupoids-quotients}
\externaldocument[spaces-more-cohomology-]{spaces-more-cohomology}
\externaldocument[spaces-simplicial-]{spaces-simplicial}
\externaldocument[spaces-duality-]{spaces-duality}
\externaldocument[formal-spaces-]{formal-spaces}
\externaldocument[restricted-]{restricted}
\externaldocument[spaces-resolve-]{spaces-resolve}
\externaldocument[formal-defos-]{formal-defos}
\externaldocument[defos-]{defos}
\externaldocument[cotangent-]{cotangent}
\externaldocument[examples-defos-]{examples-defos}
\externaldocument[algebraic-]{algebraic}
\externaldocument[examples-stacks-]{examples-stacks}
\externaldocument[stacks-sheaves-]{stacks-sheaves}
\externaldocument[criteria-]{criteria}
\externaldocument[artin-]{artin}
\externaldocument[quot-]{quot}
\externaldocument[stacks-properties-]{stacks-properties}
\externaldocument[stacks-morphisms-]{stacks-morphisms}
\externaldocument[stacks-limits-]{stacks-limits}
\externaldocument[stacks-cohomology-]{stacks-cohomology}
\externaldocument[stacks-perfect-]{stacks-perfect}
\externaldocument[stacks-introduction-]{stacks-introduction}
\externaldocument[stacks-more-morphisms-]{stacks-more-morphisms}
\externaldocument[stacks-geometry-]{stacks-geometry}
\externaldocument[moduli-]{moduli}
\externaldocument[moduli-curves-]{moduli-curves}
\externaldocument[examples-]{examples}
\externaldocument[exercises-]{exercises}
\externaldocument[guide-]{guide}
\externaldocument[desirables-]{desirables}
\externaldocument[coding-]{coding}
\externaldocument[obsolete-]{obsolete}
\externaldocument[fdl-]{fdl}
\externaldocument[index-]{index}

% Theorem environments.
%
\theoremstyle{plain}
\newtheorem{theorem}[subsection]{Theorem}
\newtheorem{proposition}[subsection]{Proposition}
\newtheorem{lemma}[subsection]{Lemma}

\theoremstyle{definition}
\newtheorem{definition}[subsection]{Definition}
\newtheorem{example}[subsection]{Example}
\newtheorem{exercise}[subsection]{Exercise}
\newtheorem{situation}[subsection]{Situation}

\theoremstyle{remark}
\newtheorem{remark}[subsection]{Remark}
\newtheorem{remarks}[subsection]{Remarks}

\numberwithin{equation}{subsection}

% Macros
%
\def\lim{\mathop{\mathrm{lim}}\nolimits}
\def\colim{\mathop{\mathrm{colim}}\nolimits}
\def\Spec{\mathop{\mathrm{Spec}}}
\def\Hom{\mathop{\mathrm{Hom}}\nolimits}
\def\Ext{\mathop{\mathrm{Ext}}\nolimits}
\def\SheafHom{\mathop{\mathcal{H}\!\mathit{om}}\nolimits}
\def\SheafExt{\mathop{\mathcal{E}\!\mathit{xt}}\nolimits}
\def\Sch{\mathit{Sch}}
\def\Mor{\mathop{Mor}\nolimits}
\def\Ob{\mathop{\mathrm{Ob}}\nolimits}
\def\Sh{\mathop{\mathit{Sh}}\nolimits}
\def\NL{\mathop{N\!L}\nolimits}
\def\proetale{{pro\text{-}\acute{e}tale}}
\def\etale{{\acute{e}tale}}
\def\QCoh{\mathit{QCoh}}
\def\Ker{\mathop{\mathrm{Ker}}}
\def\Im{\mathop{\mathrm{Im}}}
\def\Coker{\mathop{\mathrm{Coker}}}
\def\Coim{\mathop{\mathrm{Coim}}}

%
% Macros for moduli stacks/spaces
%
\def\QCohstack{\mathcal{QC}\!\mathit{oh}}
\def\Cohstack{\mathcal{C}\!\mathit{oh}}
\def\Spacesstack{\mathcal{S}\!\mathit{paces}}
\def\Quotfunctor{\mathrm{Quot}}
\def\Hilbfunctor{\mathrm{Hilb}}
\def\Curvesstack{\mathcal{C}\!\mathit{urves}}
\def\Polarizedstack{\mathcal{P}\!\mathit{olarized}}
\def\Complexesstack{\mathcal{C}\!\mathit{omplexes}}
% \Pic is the operator that assigns to X its picard group, usage \Pic(X)
% \Picardstack_{X/B} denotes the Picard stack of X over B
% \Picardfunctor_{X/B} denotes the Picard functor of X over B
\def\Pic{\mathop{\mathrm{Pic}}\nolimits}
\def\Picardstack{\mathcal{P}\!\mathit{ic}}
\def\Picardfunctor{\mathrm{Pic}}
\def\Deformationcategory{\mathcal{D}\!\mathit{ef}}


% OK, start here.
%
\begin{document}

\title{Criteria for Representability}

\maketitle

\phantomsection
\label{section-phantom}

\tableofcontents




\section{Introduction}
\label{section-introduction}

\noindent
The purpose of this chapter is to find criteria guaranteeing that a
stack in groupoids over the category of schemes with the fppf topology
is an algebraic stack. Historically, this often involved proving that
certain functors were representable, see Grothendieck's lectures
\cite{Gr-I},
\cite{Gr-II},
\cite{Gr-III},
\cite{Gr-IV},
\cite{Gr-V}, and
\cite{Gr-VI}.
This explains the title of this chapter. Another important source
of this material comes from the work of Artin, see
\cite{ArtinI},
\cite{ArtinII},
\cite{Artin-Theorem-Representability},
\cite{Artin-Construction-Techniques},
\cite{Artin-Algebraic-Spaces},
\cite{Artin-Algebraic-Approximation},
\cite{Artin-Implicit-Function}, and
\cite{ArtinVersal}.

\medskip\noindent
Some of the notation, conventions and terminology in this chapter is awkward
and may seem backwards to the more experienced reader. This is intentional.
Please see Quot, Section \ref{quot-section-conventions} for an
explanation.



\section{Conventions}
\label{section-conventions}

\noindent
The conventions we use in this chapter are the same as those in the
chapter on algebraic stacks, see
Algebraic Stacks, Section \ref{algebraic-section-conventions}.




\section{What we already know}
\label{section-done-so-far}

\noindent
The analogue of this chapter for algebraic spaces is the chapter entitled
``Bootstrap'', see
Bootstrap, Section \ref{bootstrap-section-introduction}.
That chapter already contains some representability results.
Moreover, some of the preliminary material treated there we already
have worked out in the chapter on algebraic stacks.
Here is a list:
\begin{enumerate}
\item We discuss morphisms of presheaves representable by algebraic spaces in
Bootstrap, Section
\ref{bootstrap-section-morphism-representable-by-spaces}.
In
Algebraic Stacks, Section
\ref{algebraic-section-morphisms-representable-by-algebraic-spaces}
we discuss the notion of a $1$-morphism of categories fibred in groupoids
being representable by algebraic spaces.
\item We discuss properties of morphisms of presheaves representable by
algebraic spaces in
Bootstrap, Section
\ref{bootstrap-section-representable-by-spaces-properties}.
In
Algebraic Stacks, Section
\ref{algebraic-section-representable-properties}
we discuss properties of $1$-morphisms of categories fibred in groupoids
representable by algebraic spaces.
\item We proved that if $F$ is a sheaf whose diagonal is representable
by algebraic spaces and which has an \'etale covering by an algebraic
space, then $F$ is an algebraic space, see
Bootstrap, Theorem \ref{bootstrap-theorem-bootstrap}.
(This is a weak version of the result in the next item on the list.)
\item
\label{item-bootstrap-final}
We proved that if $F$ is a sheaf and if there exists an algebraic
space $U$ and a morphism $U \to F$ which is representable by algebraic
spaces, surjective, flat, and locally of finite presentation, then
$F$ is an algebraic space, see
Bootstrap, Theorem \ref{bootstrap-theorem-final-bootstrap}.
\item We have also proved the ``smooth'' analogue of
(\ref{item-bootstrap-final}) for algebraic
stacks: If $\mathcal{X}$ is a stack in groupoids over
$(\Sch/S)_{fppf}$ and if there exists a stack in groupoids
$\mathcal{U}$ over $(\Sch/S)_{fppf}$ which is representable
by an algebraic space and a $1$-morphism $u : \mathcal{U} \to \mathcal{X}$
which is representable by algebraic spaces, surjective, and smooth
then $\mathcal{X}$ is an algebraic stack, see
Algebraic Stacks, Lemma
\ref{algebraic-lemma-smooth-surjective-morphism-implies-algebraic}.
\end{enumerate}
Our first task now is to prove the analogue of
(\ref{item-bootstrap-final}) for algebraic
stacks in general; it is
Theorem \ref{theorem-bootstrap}.



\section{Morphisms of stacks in groupoids}
\label{section-1-morphisms}

\noindent
This section is preliminary and should be skipped on a first reading.

\begin{lemma}
\label{lemma-etale-permanence}
Let $\mathcal{X} \to \mathcal{Y} \to \mathcal{Z}$
be $1$-morphisms of categories fibred in groupoids over
$(\Sch/S)_{fppf}$.
If $\mathcal{X} \to \mathcal{Z}$ and $\mathcal{Y} \to \mathcal{Z}$ are
representable by algebraic spaces and \'etale so is
$\mathcal{X} \to \mathcal{Y}$.
\end{lemma}

\begin{proof}
Let $\mathcal{U}$ be a representable category fibred in groupoids over $S$.
Let $f : \mathcal{U} \to \mathcal{Y}$ be a $1$-morphism. We have to show that
$\mathcal{X} \times_\mathcal{Y} \mathcal{U}$ is representable by an
algebraic space and \'etale over $\mathcal{U}$.
Consider the composition $h : \mathcal{U} \to \mathcal{Z}$. Then
$$
\mathcal{X} \times_\mathcal{Z} \mathcal{U}
\longrightarrow
\mathcal{Y} \times_\mathcal{Z} \mathcal{U}
$$
is a $1$-morphism between categories fibres in groupoids which are both
representable by algebraic spaces and both \'etale over $\mathcal{U}$.
Hence by
Properties of Spaces, Lemma \ref{spaces-properties-lemma-etale-permanence}
this is represented by an \'etale morphism of algebraic spaces.
Finally, we obtain the result we want as the morphism $f$ induces
a morphism $\mathcal{U} \to \mathcal{Y} \times_\mathcal{Z} \mathcal{U}$
and we have
$$
\mathcal{X} \times_\mathcal{Y} \mathcal{U} =
(\mathcal{X} \times_\mathcal{Z} \mathcal{U})
\times_{(\mathcal{Y} \times_\mathcal{Z} \mathcal{U})}
\mathcal{U}.
$$
\end{proof}

\begin{lemma}
\label{lemma-stack-in-setoids-descent}
Let $\mathcal{X}, \mathcal{Y}, \mathcal{Z}$ be stacks in groupoids
over $(\Sch/S)_{fppf}$. Suppose that $\mathcal{X} \to \mathcal{Y}$
and $\mathcal{Z} \to \mathcal{Y}$ are $1$-morphisms.
If
\begin{enumerate}
\item $\mathcal{Y}$, $\mathcal{Z}$ are representable by algebraic spaces
$Y$, $Z$ over $S$,
\item the associated morphism of algebraic spaces $Y \to Z$ is surjective,
flat and locally of finite presentation, and
\item $\mathcal{Y} \times_\mathcal{Z} \mathcal{X}$ is a stack in
setoids,
\end{enumerate}
then $\mathcal{X}$ is a stack in setoids.
\end{lemma}

\begin{proof}
This is a special case of
Stacks, Lemma \ref{stacks-lemma-stack-in-setoids-descent}.
\end{proof}

\noindent
The following lemma is the analogue of
Algebraic Stacks, Lemma
\ref{algebraic-lemma-smooth-surjective-morphism-implies-algebraic}
and will be superseded by the stronger
Theorem \ref{theorem-bootstrap}.

\begin{lemma}
\label{lemma-flat-finite-presentation-surjective-diagonal}
Let $S$ be a scheme.
Let $u : \mathcal{U} \to \mathcal{X}$ be a $1$-morphism of
stacks in groupoids over $(\Sch/S)_{fppf}$. If
\begin{enumerate}
\item $\mathcal{U}$ is representable by an algebraic space, and
\item $u$ is representable by algebraic spaces, surjective, flat and
locally of finite presentation,
\end{enumerate}
then
$\Delta : \mathcal{X} \to \mathcal{X} \times \mathcal{X}$
representable by algebraic spaces.
\end{lemma}

\begin{proof}
Given two schemes $T_1$, $T_2$ over $S$ denote
$\mathcal{T}_i = (\Sch/T_i)_{fppf}$ the associated representable
fibre categories. Suppose given $1$-morphisms
$f_i : \mathcal{T}_i \to \mathcal{X}$.
According to
Algebraic Stacks, Lemma \ref{algebraic-lemma-representable-diagonal}
it suffices to prove that the $2$-fibered
product $\mathcal{T}_1 \times_\mathcal{X} \mathcal{T}_2$
is representable by an algebraic space. By
Stacks, Lemma
\ref{stacks-lemma-2-fibre-product-stacks-in-setoids-over-stack-in-groupoids}
this is in any case a stack in setoids. Thus
$\mathcal{T}_1 \times_\mathcal{X} \mathcal{T}_2$ corresponds
to some sheaf $F$ on $(\Sch/S)_{fppf}$, see
Stacks, Lemma \ref{stacks-lemma-stack-in-setoids-characterize}.
Let $U$ be the algebraic space which represents $\mathcal{U}$.
By assumption
$$
\mathcal{T}_i' = \mathcal{U} \times_{u, \mathcal{X}, f_i} \mathcal{T}_i
$$
is representable by an algebraic space $T'_i$ over $S$. Hence
$\mathcal{T}_1' \times_\mathcal{U} \mathcal{T}_2'$ is representable
by the algebraic space $T'_1 \times_U T'_2$.
Consider the commutative diagram
$$
\xymatrix{
&
\mathcal{T}_1 \times_{\mathcal X} \mathcal{T}_2 \ar[rr]\ar'[d][dd] & &
\mathcal{T}_1 \ar[dd] \\
\mathcal{T}_1' \times_\mathcal{U} \mathcal{T}_2' \ar[ur]\ar[rr]\ar[dd] & &
\mathcal{T}_1' \ar[ur]\ar[dd] \\
&
\mathcal{T}_2 \ar'[r][rr] & &
\mathcal X \\
\mathcal{T}_2' \ar[rr]\ar[ur] & &
\mathcal{U} \ar[ur] }
$$
In this diagram the bottom square, the right square, the back square, and
the front square are $2$-fibre products. A formal argument then shows
that $\mathcal{T}_1' \times_\mathcal{U} \mathcal{T}_2' \to
\mathcal{T}_1 \times_{\mathcal X} \mathcal{T}_2$
is the ``base change'' of $\mathcal{U} \to \mathcal{X}$, more precisely
the diagram
$$
\xymatrix{
\mathcal{T}_1' \times_\mathcal{U} \mathcal{T}_2' \ar[d] \ar[r] &
\mathcal{U} \ar[d] \\
\mathcal{T}_1 \times_{\mathcal X} \mathcal{T}_2 \ar[r] &
\mathcal{X}
}
$$
is a $2$-fibre square.
Hence $T'_1 \times_U T'_2 \to F$ is representable by algebraic spaces,
flat, locally of finite presentation and surjective, see
Algebraic Stacks, Lemmas
\ref{algebraic-lemma-map-fibred-setoids-representable-algebraic-spaces},
\ref{algebraic-lemma-base-change-representable-by-spaces},
\ref{algebraic-lemma-map-fibred-setoids-property}, and
\ref{algebraic-lemma-base-change-representable-transformations-property}.
Therefore $F$ is an algebraic space by
Bootstrap, Theorem \ref{bootstrap-theorem-final-bootstrap}
and we win.
\end{proof}

\begin{lemma}
\label{lemma-second-diagonal}
Let $\mathcal{X}$ be a category fibred in groupoids over $(\Sch/S)_{fppf}$.
The following are equivalent
\begin{enumerate}
\item $\Delta_\Delta : \mathcal{X} \to
\mathcal{X} \times_{\mathcal{X} \times \mathcal{X}} \mathcal{X}$
is representable by algebraic spaces,
\item for every $1$-morphism $\mathcal{V} \to \mathcal{X} \times \mathcal{X}$
with $\mathcal{V}$ representable (by a scheme) the fibre product
$\mathcal{Y} =
\mathcal{X} \times_{\Delta, \mathcal{X} \times \mathcal{X}} \mathcal{V}$
has diagonal representable by algebraic spaces.
\end{enumerate}
\end{lemma}

\begin{proof}
Although this is a bit of a brain twister, it is completely formal.
Namely, recall that
$\mathcal{X} \times_{\mathcal{X} \times \mathcal{X}} \mathcal{X} =
\mathcal{I}_\mathcal{X}$ is the inertia of $\mathcal{X}$ and that
$\Delta_\Delta$ is the identity section of $\mathcal{I}_\mathcal{X}$, see
Categories, Section \ref{categories-section-inertia}.
Thus condition (1) says the following: Given a scheme $V$, an object $x$ of
$\mathcal{X}$ over $V$, and a morphism $\alpha : x \to x$ of $\mathcal{X}_V$
the condition ``$\alpha = \text{id}_x$'' defines an algebraic space over $V$.
(In other words, there exists a monomorphism of algebraic spaces $W \to V$
such that a morphism of schemes $f : T \to V$ factors through $W$
if and only if $f^*\alpha = \text{id}_{f^*x}$.)

\medskip\noindent
On the other hand, let $V$ be a scheme and let $x, y$ be objects of
$\mathcal{X}$ over $V$. Then $(x, y)$ define a morphism
$\mathcal{V} = (\Sch/V)_{fppf} \to \mathcal{X} \times \mathcal{X}$.
Next, let $h : V' \to V$ be a morphism of schemes and let
$\alpha : h^*x \to h^*y$ and $\beta : h^*x \to h^*y$ be morphisms
of $\mathcal{X}_{V'}$. Then $(\alpha, \beta)$ define a morphism
$\mathcal{V}' = (\Sch/V)_{fppf} \to \mathcal{Y} \times \mathcal{Y}$.
Condition (2) now says that (with any choices as above) the
condition ``$\alpha = \beta$'' defines an algebraic space over $V$.

\medskip\noindent
To see the equivalence, given $(\alpha, \beta)$ as in (2) we see that
(1) implies that ``$\alpha^{-1} \circ \beta = \text{id}_{h^*x}$''
defines an algebraic space. The implication (2) $\Rightarrow$ (1)
follows by taking $h = \text{id}_V$ and $\beta = \text{id}_x$.
\end{proof}











\section{Limit preserving on objects}
\label{section-limit-preserving}

\noindent
Let $S$ be a scheme. Let $p : \mathcal{X} \to \mathcal{Y}$ be a $1$-morphism
of categories fibred in groupoids over $(\Sch/S)_{fppf}$. We will say that
$p$ is {\it limit preserving on objects} if the following condition holds:
Given any data consisting of
\begin{enumerate}
\item an affine scheme $U = \lim_{i \in I} U_i$ which is written as the
directed limit of affine schemes $U_i$ over $S$,
\item an object $y_i$ of $\mathcal{Y}$ over $U_i$ for some $i$,
\item an object $x$ of $\mathcal{X}$ over $U$, and
\item an isomorphism $\gamma : p(x) \to y_i|_U$,
\end{enumerate}
then there exists an $i' \geq i$, an object $x_{i'}$ of
$\mathcal{X}$ over $U_{i'}$, an isomorphism
$\beta : x_{i'}|_U \to x$, and an isomorphism
$\gamma_{i'} : p(x_{i'}) \to y_i|_{U_{i'}}$
such that
\begin{equation}
\label{equation-limit-preserving}
\vcenter{
\xymatrix{
p(x_{i'}|_U) \ar[d]_{p(\beta)} \ar[rr]_{\gamma_{i'}|_U} & &
(y_i|_{U_{i'}})|_U \ar@{=}[d] \\
p(x) \ar[rr]^\gamma & & y_i|_U
}
}
\end{equation}
commutes. In this situation we say that ``$(i', x_{i'}, \beta, \gamma_{i'})$
is a {\it solution} to the problem posed by our data (1), (2), (3), (4)''.
The motivation for this definition comes from
Limits of Spaces,
Lemma \ref{spaces-limits-lemma-characterize-relative-limit-preserving}.

\begin{lemma}
\label{lemma-base-change-limit-preserving}
Let $p : \mathcal{X} \to \mathcal{Y}$ and $q : \mathcal{Z} \to \mathcal{Y}$
be $1$-morphisms of categories fibred in groupoids over $(\Sch/S)_{fppf}$.
If $p : \mathcal{X} \to \mathcal{Y}$ is limit preserving on objects, then so
is the base change
$p' : \mathcal{X} \times_\mathcal{Y} \mathcal{Z} \to \mathcal{Z}$
of $p$ by $q$.
\end{lemma}

\begin{proof}
This is formal. Let $U = \lim_{i \in I} U_i$ be the directed limit
of affine schemes $U_i$ over $S$, let $z_i$ be an object of $\mathcal{Z}$
over $U_i$ for some $i$, let $w$ be an object of
$\mathcal{X} \times_\mathcal{Y} \mathcal{Z}$ over $U$, and let
$\delta : p'(w) \to z_i|_U$ be an isomorphism.
We may write
$w = (U, x, z, \alpha)$ for some object $x$ of $\mathcal{X}$ over $U$
and object $z$ of $\mathcal{Z}$ over $U$ and isomorphism
$\alpha : p(x) \to q(z)$. Note that $p'(w) = z$ hence
$\delta : z \to z_i|_U$. Set $y_i = q(z_i)$ and
$\gamma = q(\delta) \circ \alpha : p(x) \to y_i|_U$.
As $p$ is limit preserving on objects there exists an $i' \geq i$
and an object $x_{i'}$ of $\mathcal{X}$ over $U_{i'}$ as well as
isomorphisms $\beta : x_{i'}|_U \to x$ and
$\gamma_{i'} : p(x_{i'}) \to y_i|_{U_{i'}}$ such that
(\ref{equation-limit-preserving}) commutes. Then we consider the object
$w_{i'} = (U_{i'}, x_{i'}, z_i|_{U_{i'}}, \gamma_{i'})$ of
$\mathcal{X} \times_\mathcal{Y} \mathcal{Z}$ over $U_{i'}$
and define isomorphisms
$$
w_{i'}|_U = (U, x_{i'}|_U, z_i|_U, \gamma_{i'}|_U)
\xrightarrow{(\beta, \delta^{-1})}
(U, x, z, \alpha) = w
$$
and
$$
p'(w_{i'}) = z_i|_{U_{i'}} \xrightarrow{\text{id}} z_i|_{U_{i'}}.
$$
These combine to give a solution to the problem.
\end{proof}

\begin{lemma}
\label{lemma-composition-limit-preserving}
Let $p : \mathcal{X} \to \mathcal{Y}$ and $q : \mathcal{Y} \to \mathcal{Z}$
be $1$-morphisms of categories fibred in groupoids over $(\Sch/S)_{fppf}$.
If $p$ and $q$ are limit preserving on objects, then so is the composition
$q \circ p$.
\end{lemma}

\begin{proof}
This is formal. Let $U = \lim_{i \in I} U_i$ be the directed limit
of affine schemes $U_i$ over $S$, let $z_i$ be an object of $\mathcal{Z}$
over $U_i$ for some $i$, let $x$ be an object of $\mathcal{X}$ over $U$,
and let $\gamma : q(p(x)) \to z_i|_U$ be an isomorphism. As $q$ is
limit preserving on objects there exist an $i' \geq i$, an object
$y_{i'}$ of $\mathcal{Y}$ over $U_{i'}$, an isomorphism
$\beta : y_{i'}|_U \to p(x)$, and an isomorphism
$\gamma_{i'} : q(y_{i'}) \to z_i|_{U_{i'}}$
such that (\ref{equation-limit-preserving}) is commutative. As $p$ is
limit preserving on objects there exist an $i'' \geq i'$, an object
$x_{i''}$ of $\mathcal{X}$ over $U_{i''}$, an isomorphism
$\beta' : x_{i''}|_U \to x$, and an isomorphism
$\gamma'_{i''} : p(x_{i''}) \to y_{i'}|_{U_{i''}}$
such that (\ref{equation-limit-preserving}) is commutative.
The solution is to take $x_{i''}$ over $U_{i''}$ with isomorphism
$$
q(p(x_{i''})) \xrightarrow{q(\gamma'_{i''})}
q(y_{i'})|_{U_{i''}} \xrightarrow{\gamma_{i'}|_{U_{i''}}}
z_i|_{U_{i''}}
$$
and isomorphism $\beta' : x_{i''}|_U \to x$. We omit the verification
that (\ref{equation-limit-preserving}) is commutative.
\end{proof}

\begin{lemma}
\label{lemma-representable-by-spaces-limit-preserving}
Let $p : \mathcal{X} \to \mathcal{Y}$ be a $1$-morphism of categories
fibred in groupoids over $(\Sch/S)_{fppf}$. If $p$ is
representable by algebraic spaces, then the following are equivalent:
\begin{enumerate}
\item $p$ is limit preserving on objects, and
\item $p$ is locally of finite presentation (see
Algebraic Stacks,
Definition \ref{algebraic-definition-relative-representable-property}).
\end{enumerate}
\end{lemma}

\begin{proof}
Assume (2). Let $U = \lim_{i \in I} U_i$ be the directed limit
of affine schemes $U_i$ over $S$, let $y_i$ be an object of $\mathcal{Y}$
over $U_i$ for some $i$, let $x$ be an object of $\mathcal{X}$ over $U$,
and let $\gamma : p(x) \to y_i|_U$ be an isomorphism. Let
$X_{y_i}$ denote an algebraic space over $U_i$ representing the $2$-fibre
product
$$
(\Sch/U_i)_{fppf} \times_{y_i, \mathcal{Y}, p} \mathcal{X}.
$$
Note that $\xi = (U, U \to U_i, x, \gamma^{-1})$ defines an object of
this $2$-fibre product over $U$. Via the $2$-Yoneda lemma $\xi$ corresponds
to a morphism $f_\xi : U \to X_{y_i}$ over $U_i$. By
Limits of Spaces, Proposition
\ref{spaces-limits-proposition-characterize-locally-finite-presentation}
there exists an $i' \geq i$ and a morphism $f_{i'} : U_{i'} \to X_{y_i}$
such that $f_\xi$ is the composition of $f_{i'}$ and the projection
morphism $U \to U_{i'}$. Also, the $2$-Yoneda lemma tells us that
$f_{i'}$ corresponds to an object
$\xi_{i'} = (U_{i'}, U_{i'} \to U_i, x_{i'}, \alpha)$ of
the displayed $2$-fibre product over $U_{i'}$ whose restriction to
$U$ recovers $\xi$. In particular we obtain an isomorphism
$\gamma : x_{i'}|U \to x$. Note that $\alpha : y_i|_{U_{i'}} \to p(x_{i'})$.
Hence we see that taking $x_{i'}$, the isomorphism
$\gamma : x_{i'}|U \to x$, and the isomorphism
$\beta = \alpha^{-1} : p(x_{i'}) \to y_i|_{U_{i'}}$
is a solution to the problem.

\medskip\noindent
Assume (1). Choose a scheme $T$ and a $1$-morphism
$y : (\Sch/T)_{fppf} \to \mathcal{Y}$. Let
$X_y$ be an algebraic space over $T$ representing the $2$-fibre product
$(\Sch/T)_{fppf} \times_{y, \mathcal{Y}, p} \mathcal{X}$.
We have to show that $X_y \to T$ is locally of finite presentation.
To do this we will use the criterion in
Limits of Spaces, Remark \ref{spaces-limits-remark-limit-preserving}.
Consider an affine scheme $U = \lim_{i \in I} U_i$ written as the
directed limit of affine schemes over $T$.
Pick any $i \in I$ and set $y_i = y|_{U_i}$. Also denote $i'$ an element
of $I$ which is bigger than or equal to $i$. By the $2$-Yoneda lemma
morphisms $U \to X_y$ over $T$ correspond bijectively
to isomorphism classes of pairs $(x, \alpha)$ where $x$ is an object
of $\mathcal{X}$ over $U$ and $\alpha : y|_U \to p(x)$ is an isomorphism.
Of course giving $\alpha$ is, up to an inverse, the same thing as giving
an isomorphism $\gamma : p(x) \to y_i|_U$.
Similarly for morphisms $U_{i'} \to X_y$ over $T$. Hence (1) guarantees
that the canonical map
$$
\colim_{i' \geq i} X_y(U_{i'}) \longrightarrow X_y(U)
$$
is surjective in this situation. It follows from
Limits of Spaces, Lemma \ref{spaces-limits-lemma-surjection-is-enough}
that $X_y \to T$ is locally of finite presentation.
\end{proof}

\begin{lemma}
\label{lemma-open-immersion-limit-preserving}
Let $p : \mathcal{X} \to \mathcal{Y}$ be a $1$-morphism of categories
fibred in groupoids over $(\Sch/S)_{fppf}$. Assume $p$ is representable
by algebraic spaces and an open immersion. Then $p$ is limit preserving
on objects.
\end{lemma}

\begin{proof}
This follows from
Lemma \ref{lemma-representable-by-spaces-limit-preserving}
and (via the general principle
Algebraic Stacks, Lemma
\ref{algebraic-lemma-representable-transformations-property-implication})
from the fact that an open immersion of algebraic spaces is
locally of finite presentation, see
Morphisms of Spaces, Lemma
\ref{spaces-morphisms-lemma-open-immersion-locally-finite-presentation}.
\end{proof}

\noindent
Let $S$ be a scheme. In the following lemma we need the notion of the
{\it size} of an algebraic space $X$ over $S$. Namely, given a cardinal
$\kappa$ we will say $X$ has $\text{size}(X) \leq \kappa$ if and only
if there exists a scheme $U$ with $\text{size}(U) \leq \kappa$ (see
Sets, Section \ref{sets-section-categories-schemes}) and a surjective
\'etale morphism $U \to X$.

\begin{lemma}
\label{lemma-check-representable-limit-preserving}
Let $S$ be a scheme.
Let $\kappa = \text{size}(T)$ for some $T \in \Ob((\Sch/S)_{fppf})$.
Let $f : \mathcal{X} \to \mathcal{Y}$ be a $1$-morphism
of categories fibred in groupoids over $(\Sch/S)_{fppf}$
such that
\begin{enumerate}
\item $\mathcal{Y} \to (\Sch/S)_{fppf}$ is limit preserving on objects,
\item for an affine scheme $V$ locally of finite presentation over $S$ and
$y \in \Ob(\mathcal{Y}_V)$ the fibre product
$(\Sch/V)_{fppf} \times_{y, \mathcal{Y}} \mathcal{X}$ is representable
by an algebraic space of size $\leq \kappa$\footnote{The condition on
size can be dropped by those ignoring set theoretic issues.},
\item $\mathcal{X}$ and $\mathcal{Y}$ are stacks for the Zariski topology.
\end{enumerate}
Then $f$ is representable by algebraic spaces.
\end{lemma}

\begin{proof}
Let $V$ be a scheme over $S$ and $y \in \mathcal{Y}_V$. We have to prove
$(\Sch/V)_{fppf} \times_{y, \mathcal{Y}} \mathcal{X}$ is representable
by an algebraic space.

\medskip\noindent
Case I: $V$ is affine and maps into an affine open $\Spec(\Lambda) \subset S$.
Then we can write $V = \lim V_i$ with each $V_i$ affine and of finite
presentation over $\Spec(\Lambda)$, see
Algebra, Lemma \ref{algebra-lemma-ring-colimit-fp}.
Then $y$ comes from an object $y_i$ over $V_i$ for some $i$ by assumption (1).
By assumption (3) the fibre product
$(\Sch/V_i)_{fppf} \times_{y_i, \mathcal{Y}} \mathcal{X}$ is representable
by an algebraic space $Z_i$. Then 
$(\Sch/V)_{fppf} \times_{y, \mathcal{Y}} \mathcal{X}$ is representable
by $Z \times_{V_i} V$.

\medskip\noindent
Case II: $V$ is general. Choose an affine open covering
$V = \bigcup_{i \in I} V_i$ such that each $V_i$ maps into an affine open
of $S$. We first claim
that $\mathcal{Z} = (\Sch/V)_{fppf} \times_{y, \mathcal{Y}} \mathcal{X}$
is a stack in setoids for the Zariski topology. Namely, it is a stack in
groupoids for the Zariski topology by
Stacks, Lemma \ref{stacks-lemma-2-product-stacks-in-groupoids}.
Then suppose that $z$ is an object of $\mathcal{Z}$ over a scheme $T$.
Denote $g : T \to V$ the morphism corresponding to the
projection of $z$ in $(\Sch/V)_{fppf}$. Consider the Zariski sheaf
$\mathit{I} = \mathit{Isom}_{\mathcal{Z}}(z, z)$. By Case I we see that
$\mathit{I}|_{g^{-1}(V_i)} = *$ (the singleton sheaf). Hence
$\mathcal{I} = *$. Thus $\mathcal{Z}$ is fibred in setoids. To finish
the proof we have to show that the Zariski sheaf
$Z : T \mapsto \Ob(\mathcal{Z}_T)/\cong$ is an algebraic space, see
Algebraic Stacks, Lemma
\ref{algebraic-lemma-characterize-representable-by-space}.
There is a map $p : Z \to V$ (transformation of functors) and by Case I
we know that $Z_i = p^{-1}(V_i)$ is an algebraic space. The morphisms
$Z_i \to Z$ are representable by open immersions and
$\coprod Z_i \to Z$ is surjective (in the Zariski topology).
Hence $Z$ is a sheaf for the fppf topology by
Bootstrap, Lemma \ref{bootstrap-lemma-glueing-sheaves}.
Thus Spaces, Lemma \ref{spaces-lemma-glueing-algebraic-spaces}
applies and we conclude that $Z$ is an algebraic space\footnote{
To see that the set theoretic condition of that lemma is satisfied
we argue as follows: First choose the open covering such that
$|I| \leq \text{size}(V)$. Next, choose schemes $U_i$ of size
$\leq \max(\kappa, \text{size}(V))$ and surjective \'etale morphisms
$U_i \to Z_i$; we can do this by assumption (2) and
Sets, Lemma \ref{sets-lemma-bound-size-fibre-product}
(details omitted). Then
Sets, Lemma \ref{sets-lemma-what-is-in-it}
implies that $\coprod U_i$ is an object of $(\Sch/S)_{fppf}$.
Hence $\coprod Z_i$ is an algebraic space by
Spaces, Lemma \ref{spaces-lemma-coproduct-algebraic-spaces}.
}.
\end{proof}

\begin{lemma}
\label{lemma-check-property-limit-preserving}
Let $S$ be a scheme. Let $f : \mathcal{X} \to \mathcal{Y}$ be a $1$-morphism
of categories fibred in groupoids over $(\Sch/S)_{fppf}$. Let $\mathcal{P}$
be a property of morphisms of algebraic spaces as in
Algebraic Stacks, Definition
\ref{algebraic-definition-relative-representable-property}. If
\begin{enumerate}
\item $f$ is representable by algebraic spaces,
\item $\mathcal{Y} \to (\Sch/S)_{fppf}$ is limit preserving on objects,
\item for an affine scheme $V$ locally of finite presentation over $S$ and
$y \in \mathcal{Y}_V$ the resulting morphism of algebraic spaces
$f_y : F_y \to V$, see Algebraic Stacks, Equation
(\ref{algebraic-equation-representable-by-algebraic-spaces}),
has property $\mathcal{P}$.
\end{enumerate}
Then $f$ has property $\mathcal{P}$.
\end{lemma}

\begin{proof}
Let $V$ be a scheme over $S$ and $y \in \mathcal{Y}_V$. We have to show
that $F_y \to V$ has property $\mathcal{P}$. Since $\mathcal{P}$ is
fppf local on the base we may assume that $V$ is an affine scheme which
maps into an affine open $\Spec(\Lambda) \subset S$. Thus we can write
$V = \lim V_i$ with each $V_i$ affine and of finite presentation over
$\Spec(\Lambda)$, see Algebra, Lemma \ref{algebra-lemma-ring-colimit-fp}.
Then $y$ comes from an object $y_i$ over $V_i$ for some $i$ by assumption (2).
By assumption (3) the morphism $F_{y_i} \to V_i$ has property $\mathcal{P}$.
As $\mathcal{P}$ is stable under arbitrary base change and since
$F_y = F_{y_i} \times_{V_i} V$ we conclude that $F_y \to V$ has property
$\mathcal{P}$ as desired.
\end{proof}



\section{Formally smooth on objects}
\label{section-formally-smooth}

\noindent
Let $S$ be a scheme. Let $p : \mathcal{X} \to \mathcal{Y}$ be a $1$-morphism
of categories fibred in groupoids over $(\Sch/S)_{fppf}$. We will say that
$p$ is {\it formally smooth on objects} if the following condition holds:
Given any data consisting of
\begin{enumerate}
\item a first order thickening $U \subset U'$ of affine schemes over $S$,
\item an object $y'$ of $\mathcal{Y}$ over $U'$,
\item an object $x$ of $\mathcal{X}$ over $U$, and
\item an isomorphism $\gamma : p(x) \to y'|_U$,
\end{enumerate}
then there exists an object $x'$ of
$\mathcal{X}$ over $U'$ with an isomorphism
$\beta : x'|_U \to x$ and an isomorphism $\gamma' : p(x') \to y'$
such that
\begin{equation}
\label{equation-formally-smooth}
\vcenter{
\xymatrix{
p(x'|_U) \ar[d]_{p(\beta)} \ar[rr]_{\gamma'|_U} & &
y'|_U \ar@{=}[d] \\
p(x) \ar[rr]^\gamma & & y'|_U
}
}
\end{equation}
commutes.  In this situation we say that ``$(x', \beta, \gamma')$
is a {\it solution} to the problem posed by our data (1), (2), (3), (4)''.

\begin{lemma}
\label{lemma-base-change-formally-smooth}
Let $p : \mathcal{X} \to \mathcal{Y}$ and $q : \mathcal{Z} \to \mathcal{Y}$
be $1$-morphisms of categories fibred in groupoids over $(\Sch/S)_{fppf}$.
If $p : \mathcal{X} \to \mathcal{Y}$ is formally smooth on objects, then so
is the base change
$p' : \mathcal{X} \times_\mathcal{Y} \mathcal{Z} \to \mathcal{Z}$
of $p$ by $q$.
\end{lemma}

\begin{proof}
This is formal. Let $U \subset U'$ be a first order thickening
of affine schemes over $S$, let $z'$ be an object of $\mathcal{Z}$
over $U'$, let $w$ be an object of
$\mathcal{X} \times_\mathcal{Y} \mathcal{Z}$ over $U$, and let
$\delta : p'(w) \to z'|_U$ be an isomorphism.
We may write
$w = (U, x, z, \alpha)$ for some object $x$ of $\mathcal{X}$ over $U$
and object $z$ of $\mathcal{Z}$ over $U$ and isomorphism
$\alpha : p(x) \to q(z)$. Note that $p'(w) = z$ hence
$\delta : z \to z|_U$. Set $y' = q(z')$ and
$\gamma = q(\delta) \circ \alpha : p(x) \to y'|_U$.
As $p$ is formally smooth on objects there exists an
object $x'$ of $\mathcal{X}$ over $U'$ as well as
isomorphisms $\beta : x'|_U \to x$ and $\gamma' : p(x') \to y'$ such that
(\ref{equation-formally-smooth}) commutes. Then we consider the object
$w = (U', x', z', \gamma')$ of $\mathcal{X} \times_\mathcal{Y} \mathcal{Z}$
over $U'$ and define isomorphisms
$$
w'|_U = (U, x'|_U, z'|_U, \gamma'|_U)
\xrightarrow{(\beta, \delta^{-1})}
(U, x, z, \alpha) = w
$$
and
$$
p'(w') = z' \xrightarrow{\text{id}} z'.
$$
These combine to give a solution to the problem.
\end{proof}

\begin{lemma}
\label{lemma-composition-formally-smooth}
Let $p : \mathcal{X} \to \mathcal{Y}$ and $q : \mathcal{Y} \to \mathcal{Z}$
be $1$-morphisms of categories fibred in groupoids over $(\Sch/S)_{fppf}$.
If $p$ and $q$ are formally smooth on objects, then so is the composition
$q \circ p$.
\end{lemma}

\begin{proof}
This is formal. Let $U \subset U'$ be a first order thickening
of affine schemes over $S$, let $z'$ be an object of $\mathcal{Z}$
over $U'$, let $x$ be an object of $\mathcal{X}$ over $U$,
and let $\gamma : q(p(x)) \to z'|_U$ be an isomorphism. As $q$ is
formally smooth on objects there exist an object
$y'$ of $\mathcal{Y}$ over $U'$, an isomorphism
$\beta : y'|_U \to p(x)$, and an isomorphism $\gamma' : q(y') \to z'$
such that (\ref{equation-formally-smooth}) is commutative. As $p$ is
formally smooth on objects there exist an object
$x'$ of $\mathcal{X}$ over $U'$, an isomorphism
$\beta' : x'|_U \to x$, and an isomorphism $\gamma'' : p(x') \to y'$
such that (\ref{equation-formally-smooth}) is commutative.
The solution is to take $x'$ over $U'$ with isomorphism
$$
q(p(x')) \xrightarrow{q(\gamma'')} q(y') \xrightarrow{\gamma'} z'
$$
and isomorphism $\beta' : x'|_U \to x$. We omit the verification
that (\ref{equation-formally-smooth}) is commutative.
\end{proof}

\noindent
Note that the class of formally smooth morphisms of algebraic spaces is
stable under arbitrary base change and local on the target in the
fpqc topology, see
More on Morphisms of Spaces,
Lemma \ref{spaces-more-morphisms-lemma-base-change-formally-smooth} and
\ref{spaces-more-morphisms-lemma-descending-property-formally-smooth}.
Hence condition (2) in the lemma below makes sense.

\begin{lemma}
\label{lemma-representable-by-spaces-formally-smooth}
Let $p : \mathcal{X} \to \mathcal{Y}$ be a $1$-morphism of categories
fibred in groupoids over $(\Sch/S)_{fppf}$. If $p$ is
representable by algebraic spaces, then the following are equivalent:
\begin{enumerate}
\item $p$ is formally smooth on objects, and
\item $p$ is formally smooth (see
Algebraic Stacks,
Definition \ref{algebraic-definition-relative-representable-property}).
\end{enumerate}
\end{lemma}

\begin{proof}
Assume (2). Let $U \subset U'$ be a first order thickening
of affine schemes over $S$, let $y'$ be an object of $\mathcal{Y}$
over $U'$, let $x$ be an object of $\mathcal{X}$ over $U$,
and let $\gamma : p(x) \to y'|_U$ be an isomorphism. Let
$X_{y'}$ denote an algebraic space over $U'$ representing the $2$-fibre
product
$$
(\Sch/U')_{fppf} \times_{y', \mathcal{Y}, p} \mathcal{X}.
$$
Note that $\xi = (U, U \to U', x, \gamma^{-1})$ defines an object of
this $2$-fibre product over $U$. Via the $2$-Yoneda lemma $\xi$ corresponds
to a morphism $f_\xi : U \to X_{y'}$ over $U'$. As $X_{y'} \to U'$ is
formally smooth by assumption there exists a morphism
$f' : U' \to X_{y'}$ such that $f_\xi$ is the composition of $f'$
and the morphism $U \to U'$. Also, the $2$-Yoneda lemma tells us that
$f'$ corresponds to an object $\xi' = (U', U' \to U', x', \alpha)$ of
the displayed $2$-fibre product over $U'$ whose restriction to
$U$ recovers $\xi$. In particular we obtain an isomorphism
$\gamma : x'|U \to x$. Note that $\alpha : y' \to p(x')$.
Hence we see that taking $x'$, the isomorphism
$\gamma : x'|U \to x$, and the isomorphism
$\beta = \alpha^{-1} : p(x') \to y'$
is a solution to the problem.

\medskip\noindent
Assume (1). Choose a scheme $T$ and a $1$-morphism
$y : (\Sch/T)_{fppf} \to \mathcal{Y}$. Let
$X_y$ be an algebraic space over $T$ representing the $2$-fibre product
$(\Sch/T)_{fppf} \times_{y, \mathcal{Y}, p} \mathcal{X}$.
We have to show that $X_y \to T$ is formally smooth.
Hence it suffices to show that given a first order thickening
$U \subset U'$ of affine schemes over $T$, then
$X_y(U') \to X_y(U')$ is surjective (morphisms in the
category of algebraic spaces over $T$). Set $y' = y|_{U'}$.
By the $2$-Yoneda lemma morphisms $U \to X_y$ over $T$ correspond bijectively
to isomorphism classes of pairs $(x, \alpha)$ where $x$ is an object
of $\mathcal{X}$ over $U$ and $\alpha : y|_U \to p(x)$ is an isomorphism.
Of course giving $\alpha$ is, up to an inverse, the same thing as giving
an isomorphism $\gamma : p(x) \to y'|_U$.
Similarly for morphisms $U' \to X_y$ over $T$. Hence (1) guarantees
the surjectivity of $X_y(U') \to X_y(U')$
in this situation and we win.
\end{proof}







\section{Surjective on objects}
\label{section-formally-surjective}

\noindent
Let $S$ be a scheme. Let $p : \mathcal{X} \to \mathcal{Y}$ be a $1$-morphism
of categories fibred in groupoids over $(\Sch/S)_{fppf}$. We will say that
$p$ is {\it surjective on objects} if the following condition holds:
Given any data consisting of
\begin{enumerate}
\item a field $k$ over $S$, and
\item an object $y$ of $\mathcal{Y}$ over $\Spec(k)$,
\end{enumerate}
then there exists an extension $K/k$ of fields over $S$, an
object $x$ of $\mathcal{X}$ over $\Spec(K)$
such that $p(x) \cong y|_{\Spec(K)}$.

\begin{lemma}
\label{lemma-base-change-surjective}
Let $p : \mathcal{X} \to \mathcal{Y}$ and $q : \mathcal{Z} \to \mathcal{Y}$
be $1$-morphisms of categories fibred in groupoids over $(\Sch/S)_{fppf}$.
If $p : \mathcal{X} \to \mathcal{Y}$ is surjective on objects, then so
is the base change
$p' : \mathcal{X} \times_\mathcal{Y} \mathcal{Z} \to \mathcal{Z}$
of $p$ by $q$.
\end{lemma}

\begin{proof}
This is formal. Let $z$ be an object of $\mathcal{Z}$ over a field $k$.
As $p$ is surjective on objects there exists an extension $K/k$
and an object $x$ of $\mathcal{X}$ over $K$ and an isomorphism
$\alpha : p(x) \to q(z)|_{\Spec(K)}$. Then
$w = (\Spec(K), x, z|_{\Spec(K)}, \alpha)$ is an object of
$\mathcal{X} \times_\mathcal{Y} \mathcal{Z}$ over $K$ with
$p'(w) = z|_{\Spec(K)}$.
\end{proof}

\begin{lemma}
\label{lemma-composition-surjective}
Let $p : \mathcal{X} \to \mathcal{Y}$ and $q : \mathcal{Y} \to \mathcal{Z}$
be $1$-morphisms of categories fibred in groupoids over $(\Sch/S)_{fppf}$.
If $p$ and $q$ are surjective on objects, then so is the composition
$q \circ p$.
\end{lemma}

\begin{proof}
This is formal. Let $z$ be an object of $\mathcal{Z}$ over a field $k$.
As $q$ is surjective on objects there exists a field extension $K/k$
and an object $y$ of $\mathcal{Y}$ over $K$ such that
$q(y) \cong x|_{\Spec(K)}$. As $p$ is surjective on objects there
exists a field extension $L/K$ and an object $x$ of $\mathcal{X}$
over $L$ such that $p(x) \cong y|_{\Spec(L)}$. Then the field extension
$L/k$ and the object $x$ of $\mathcal{X}$ over $L$ satisfy
$q(p(x)) \cong z|_{\Spec(L)}$ as desired.
\end{proof}

\begin{lemma}
\label{lemma-representable-by-spaces-surjective}
Let $p : \mathcal{X} \to \mathcal{Y}$ be a $1$-morphism of categories
fibred in groupoids over $(\Sch/S)_{fppf}$. If $p$ is
representable by algebraic spaces, then the following are equivalent:
\begin{enumerate}
\item $p$ is surjective on objects, and
\item $p$ is surjective (see
Algebraic Stacks,
Definition \ref{algebraic-definition-relative-representable-property}).
\end{enumerate}
\end{lemma}

\begin{proof}
Assume (2). Let $k$ be a field and let $y$ be an object of
$\mathcal{Y}$ over $k$. Let $X_y$ denote an algebraic space over $k$
representing the $2$-fibre product
$$
(\Sch/\Spec(k))_{fppf} \times_{y, \mathcal{Y}, p} \mathcal{X}.
$$
As we've assumed that $p$ is surjective we see that $X_y$ is not empty.
Hence we can find a field extension $K/k$ and a $K$-valued point
$x$ of $X_y$. Via the $2$-Yoneda lemma this corresponds to an object
$x$ of $\mathcal{X}$ over $K$ together with an isomorphism
$p(x) \cong y|_{\Spec(K)}$ and we see that (1) holds.

\medskip\noindent
Assume (1). Choose a scheme $T$ and a $1$-morphism
$y : (\Sch/T)_{fppf} \to \mathcal{Y}$. Let
$X_y$ be an algebraic space over $T$ representing the $2$-fibre product
$(\Sch/T)_{fppf} \times_{y, \mathcal{Y}, p} \mathcal{X}$.
We have to show that $X_y \to T$ is surjective. By
Morphisms of Spaces, Definition \ref{spaces-morphisms-definition-surjective}
we have to show that $|X_y| \to |T|$ is surjective.
This means exactly that given a field $k$ over $T$ and a
morphism $t : \Spec(k) \to T$ there exists a field extension
$K/k$ and a morphism $x : \Spec(K) \to X_y$ such that
$$
\xymatrix{
\Spec(K) \ar[d] \ar[r]_x & X_y \ar[d] \\
\Spec(k) \ar[r]^t & T
}
$$
commutes. By the $2$-Yoneda lemma this means exactly that we have to find
$k \subset K$ and an object $x$ of $\mathcal{X}$ over $K$ such that
$p(x) \cong t^*y|_{\Spec(K)}$. Hence (1) guarantees that this is
the case and we win.
\end{proof}












\section{Algebraic morphisms}
\label{section-algebraic}

\noindent
The following notion is occasionally useful.

\begin{definition}
\label{definition-algebraic}
Let $S$ be a scheme. Let $F : \mathcal{X} \to \mathcal{Y}$ be a
$1$-morphism of stacks in groupoids over $(\Sch/S)_{fppf}$.
We say that $F$ is {\it algebraic} if for every scheme $T$ and every
object $\xi$ of $\mathcal{Y}$ over $T$ the $2$-fibre product
$$
(\Sch/T)_{fppf} \times_{\xi, \mathcal{Y}} \mathcal{X}
$$
is an algebraic stack over $S$.
\end{definition}

\noindent
With this terminology in place we have the following result that generalizes
Algebraic Stacks, Lemma
\ref{algebraic-lemma-representable-morphism-to-algebraic}.

\begin{lemma}
\label{lemma-algebraic-morphism-to-algebraic}
Let $S$ be a scheme.
Let $F : \mathcal{X} \to \mathcal{Y}$ be a $1$-morphism of
stacks in groupoids over $(\Sch/S)_{fppf}$. If
\begin{enumerate}
\item $\mathcal{Y}$ is an algebraic stack, and
\item $F$ is algebraic (see above),
\end{enumerate}
then $\mathcal{X}$ is an algebraic stack.
\end{lemma}

\begin{proof}
By assumption (1) there exists a scheme $T$ and an object
$\xi$ of $\mathcal{Y}$ over $T$ such that the corresponding
$1$-morphism $\xi : (\Sch/T)_{fppf} \to \mathcal{Y}$
is smooth an surjective. Then
$\mathcal{U} = (\Sch/T)_{fppf} \times_{\xi, \mathcal{Y}} \mathcal{X}$
is an algebraic stack by assumption (2).
Choose a scheme $U$ and a surjective smooth $1$-morphism
$(\Sch/U)_{fppf} \to \mathcal{U}$.
The projection $\mathcal{U} \longrightarrow \mathcal{X}$
is, as the base change of the morphism
$\xi : (\Sch/T)_{fppf} \to \mathcal{Y}$,
surjective and smooth, see
Algebraic Stacks, Lemma
\ref{algebraic-lemma-base-change-representable-transformations-property}.
Then the composition
$(\Sch/U)_{fppf} \to \mathcal{U} \to \mathcal{X}$
is surjective and smooth as a composition of surjective and smooth
morphisms, see
Algebraic Stacks, Lemma
\ref{algebraic-lemma-composition-representable-transformations-property}.
Hence $\mathcal{X}$ is an algebraic stack by
Algebraic Stacks, Lemma
\ref{algebraic-lemma-smooth-surjective-morphism-implies-algebraic}.
\end{proof}

\begin{lemma}
\label{lemma-map-from-algebraic}
Let $S$ be a scheme. Let $F : \mathcal{X} \to \mathcal{Y}$ be a $1$-morphism
of stacks in groupoids over $(\Sch/S)_{fppf}$. If $\mathcal{X}$ is an
algebraic stack and $\Delta : \mathcal{Y} \to \mathcal{Y} \times \mathcal{Y}$
is representable by algebraic spaces, then $F$ is algebraic.
\end{lemma}

\begin{proof}
Choose a representable stack in groupoids $\mathcal{U}$ and a surjective
smooth $1$-morphism $\mathcal{U} \to \mathcal{X}$. Let $T$ be a scheme and
let $\xi$ be an object of $\mathcal{Y}$ over $T$. The morphism of
$2$-fibre products
$$
(\Sch/T)_{fppf} \times_{\xi, \mathcal{Y}} \mathcal{U}
\longrightarrow
(\Sch/T)_{fppf} \times_{\xi, \mathcal{Y}} \mathcal{X}
$$
is representable by algebraic spaces, surjective, and smooth as a
base change of $\mathcal{U} \to \mathcal{X}$, see
Algebraic Stacks,
Lemmas \ref{algebraic-lemma-base-change-representable-by-spaces} and
\ref{algebraic-lemma-base-change-representable-transformations-property}.
By our condition on the diagonal of $\mathcal{Y}$ we see that
the source of this morphism is representable by an algebraic space, see
Algebraic Stacks, Lemma \ref{algebraic-lemma-representable-diagonal}.
Hence the target is an algebraic stack by
Algebraic Stacks,
Lemma \ref{algebraic-lemma-smooth-surjective-morphism-implies-algebraic}.
\end{proof}

\begin{lemma}
\label{lemma-diagonals-and-algebraic-morphisms}
Let $S$ be a scheme. Let $F : \mathcal{X} \to \mathcal{Y}$ be a $1$-morphism
of stacks in groupoids over $(\Sch/S)_{fppf}$.
If $F$ is algebraic and
$\Delta : \mathcal{Y} \to \mathcal{Y} \times \mathcal{Y}$
is representable by algebraic spaces, then
$\Delta : \mathcal{X} \to \mathcal{X} \times \mathcal{X}$
is representable by algebraic spaces.
\end{lemma}

\begin{proof}
Assume $F$ is algebraic and
$\Delta : \mathcal{Y} \to \mathcal{Y} \times \mathcal{Y}$
is representable by algebraic spaces.
Take a scheme $U$ over $S$ and two objects $x_1, x_2$ of
$\mathcal{X}$ over $U$.
We have to show that $\mathit{Isom}(x_1, x_2)$ is an algebraic space
over $U$, see
Algebraic Stacks, Lemma \ref{algebraic-lemma-representable-diagonal}.
Set $y_i = F(x_i)$. We have a morphism of sheaves of sets
$$
f : \mathit{Isom}(x_1, x_2) \to \mathit{Isom}(y_1, y_2)
$$
and the target is an algebraic space by assumption.
Thus it suffices to show that $f$ is representable by
algebraic spaces, see Bootstrap, Lemma
\ref{bootstrap-lemma-representable-by-spaces-over-space}.
Thus we can choose a scheme $V$ over $U$ and an
isomorphism $\beta : y_{1, V} \to y_{2, V}$ and
we have to show the functor
$$
(\Sch/V)_{fppf} \to \textit{Sets},\quad
T/V \mapsto \{\alpha : x_{1, T} \to x_{2, T}
\text{ in }\mathcal{X}_T \mid F(\alpha) = \beta|_T\}
$$
is an algebraic space. Consider the objects
$z_1 = (V, x_{1, V}, \text{id})$ and 
$z_2 = (V, x_{2, V}, \beta)$ of
$$
\mathcal{Z} = (\Sch/V)_{fppf} \times_{y_{1, V}, \mathcal{Y}} \mathcal{X}
$$
Then it is straightforward to verify that
the functor above is equal to $\mathit{Isom}(z_1, z_2)$
on $(\Sch/V)_{fppf}$. Hence this is an algebraic space
by our assumption that $F$ is algebraic (and the definition
of algebraic stacks).
\end{proof}
















\section{Spaces of sections}
\label{section-spaces-sections}

\noindent
Given morphisms $W \to Z \to U$ we can consider the functor that associates
to a scheme $U'$ over $U$ the set of sections $\sigma : Z_{U'} \to W_{U'}$
of the base change $W_{U'} \to Z_{U'}$ of the morphism $W \to Z$.
In this section we prove some preliminary lemmas on this functor.

\begin{lemma}
\label{lemma-surjection-space-of-sections}
Let $Z \to U$ be a finite morphism of schemes.
Let $W$ be an algebraic space and let $W \to Z$ be a
surjective \'etale morphism. Then there exists a surjective
\'etale morphism $U' \to U$ and a section
$$
\sigma : Z_{U'} \to W_{U'}
$$
of the morphism $W_{U'} \to Z_{U'}$.
\end{lemma}

\begin{proof}
We may choose a separated scheme $W'$ and a surjective \'etale morphism
$W' \to W$. Hence after replacing $W$ by $W'$ we may assume that $W$
is a separated scheme. Write $f : W \to Z$ and $\pi : Z \to U$.
Note that $f \circ \pi : W \to U$ is separated as
$W$ is separated (see
Schemes, Lemma \ref{schemes-lemma-compose-after-separated}).
Let $u \in U$ be a point. Clearly it suffices
to find an \'etale neighbourhood $(U', u')$ of $(U, u)$ such that
a section $\sigma$ exists over $U'$. Let $z_1, \ldots, z_r$
be the points of $Z$ lying above $u$. For each $i$ choose a point
$w_i \in W$ which maps to $z_i$. We may pick an \'etale neighbourhood
$(U', u') \to (U, u)$ such that the conclusions of
More on Morphisms, Lemma
\ref{more-morphisms-lemma-etale-splits-off-quasi-finite-part-technical-variant}
hold for both $Z \to U$ and the points $z_1, \ldots, z_r$
and $W \to U$ and the points $w_1, \ldots, w_r$. Hence, after
replacing $(U, u)$ by $(U', u')$ and relabeling, we may assume that
all the field extensions $\kappa(z_i)/\kappa(u)$ and
$\kappa(w_i)/\kappa(u)$ are purely inseparable, and moreover
that there exist disjoint union decompositions
$$
Z = V_1 \amalg \ldots \amalg V_r \amalg A, \quad
W = W_1 \amalg \ldots \amalg W_r \amalg B
$$
by open and closed subschemes
with $z_i \in V_i$, $w_i \in W_i$ and $V_i \to U$, $W_i \to U$ finite.
After replacing $U$ by $U \setminus \pi(A)$ we may assume that
$A = \emptyset$, i.e., $Z = V_1 \amalg \ldots \amalg V_r$.
After replacing $W_i$ by $W_i \cap f^{-1}(V_i)$ and
$B$ by $B \cup \bigcup W_i \cap f^{-1}(Z \setminus V_i)$
we may assume that $f$ maps $W_i$ into $V_i$.
Then $f_i = f|_{W_i} : W_i \to V_i$ is a morphism of schemes finite over $U$,
hence finite (see
Morphisms, Lemma \ref{morphisms-lemma-finite-permanence}).
It is also \'etale (by assumption),
$f_i^{-1}(\{z_i\}) = w_i$, and induces an isomorphism of residue
fields $\kappa(z_i) = \kappa(w_i)$ (because both are purely inseparable
extensions of $\kappa(u)$ and $\kappa(w_i)/\kappa(z_i)$
is separable as $f$ is \'etale). Hence by
\'Etale Morphisms, Lemma \ref{etale-lemma-finite-etale-one-point}
we see that $f_i$ is an isomorphism in a neighbourhood $V_i'$ of
$z_i$. Since $\pi : Z \to U$ is closed, after shrinking $U$, we may assume
that $W_i \to V_i$ is an isomorphism. This proves the lemma.
\end{proof}

\begin{lemma}
\label{lemma-space-of-sections}
Let $Z \to U$ be a finite locally free morphism of schemes.
Let $W$ be an algebraic space and let $W \to Z$ be an \'etale morphism.
Then the functor
$$
F : (\Sch/U)_{fppf}^{opp} \longrightarrow \textit{Sets},
$$
defined by the rule
$$
U' \longmapsto
F(U') =
\{\sigma : Z_{U'} \to W_{U'}\text{ section of }W_{U'} \to Z_{U'}\}
$$
is an algebraic space and the morphism $F \to U$ is \'etale.
\end{lemma}

\begin{proof}
Assume first that $W \to Z$ is also separated.
Let $U'$ be a scheme over $U$ and let $\sigma \in F(U')$. By
Morphisms of Spaces, Lemma \ref{spaces-morphisms-lemma-section-immersion}
the morphism $\sigma$ is a closed immersion.
Moreover, $\sigma$ is \'etale by
Properties of Spaces, Lemma \ref{spaces-properties-lemma-etale-permanence}.
Hence $\sigma$ is also an open immersion, see
Morphisms of Spaces,
Lemma \ref{spaces-morphisms-lemma-etale-universally-injective-open}.
In other words, $Z_\sigma = \sigma(Z_{U'}) \subset W_{U'}$ is
an open subspace such that the morphism $Z_\sigma \to Z_{U'}$
is an isomorphism. In particular, the morphism $Z_\sigma \to U'$
is finite. Hence we obtain a transformation of functors
$$
F \longrightarrow (W/U)_{fin}, \quad
\sigma \longmapsto (U' \to U, Z_\sigma)
$$
where $(W/U)_{fin}$ is the finite part of the morphism $W \to U$
introduced in
More on Groupoids in Spaces, Section
\ref{spaces-more-groupoids-section-finite}.
It is clear that this transformation of functors is injective (since we can
recover $\sigma$ from $Z_\sigma$ as the inverse of the isomorphism
$Z_\sigma \to Z_{U'}$). By
More on Groupoids in Spaces, Proposition
\ref{spaces-more-groupoids-proposition-finite-algebraic-space}
we know that $(W/U)_{fin}$ is an algebraic space \'etale over $U$.
Hence to finish the proof in this case it suffices to show that
$F \to (W/U)_{fin}$ is representable and an open immersion.
To see this suppose that we are given a morphism of schemes $U' \to U$
and an open subspace $Z' \subset W_{U'}$ such that $Z' \to U'$
is finite. Then it suffices to show that there exists an
open subscheme $U'' \subset U'$ such that a morphism
$T \to U'$ factors through $U''$ if and only if $Z' \times_{U'} T$
maps isomorphically to $Z \times_{U'} T$. This follows from
More on Morphisms of Spaces, Lemma
\ref{spaces-more-morphisms-lemma-where-isomorphism}
(here we use that $Z \to B$ is flat and locally of finite presentation
as well as finite).
Hence we have proved the lemma in case $W \to Z$ is separated
as well as \'etale.

\medskip\noindent
In the general case we choose a separated scheme $W'$ and a surjective
\'etale morphism $W' \to W$. Note that the morphisms $W' \to W$ and
$W \to Z$ are separated as their source is separated. Denote $F'$ the
functor associated to $W' \to Z \to U$ as in the lemma. In the first
paragraph of the proof we showed that $F'$ is representable by an
algebraic space \'etale over $U$. By
Lemma \ref{lemma-surjection-space-of-sections}
the map of functors $F' \to F$ is surjective for the \'etale topology
on $\Sch/U$. Moreover, if $U'$ and $\sigma : Z_{U'} \to W_{U'}$
define a point $\xi \in F(U')$, then the fibre product
$$
F'' = F' \times_{F, \xi} U'
$$
is the functor on $\Sch/U'$ associated to the morphisms
$$
W'_{U'} \times_{W_{U'}, \sigma} Z_{U'} \to Z_{U'} \to U'.
$$
Since the first morphism is separated as a base change of a separated
morphism, we see that $F''$ is an algebraic space \'etale over $U'$
by the result of the first paragraph. It follows that $F' \to F$ is a
surjective \'etale transformation of functors, which is representable
by algebraic spaces. Hence $F$ is an algebraic space by
Bootstrap, Theorem \ref{bootstrap-theorem-final-bootstrap}.
Since $F' \to F$ is an \'etale surjective morphism of algebraic spaces
it follows that $F \to U$ is \'etale because $F' \to U$ is \'etale.
\end{proof}









\section{Relative morphisms}
\label{section-relative-morphisms}

\noindent
We continue the discussion started in
More on Morphisms, Section \ref{more-morphisms-section-relative-morphisms}.

\medskip\noindent
Let $S$ be a scheme. Let $Z \to B$ and $X \to B$ be morphisms of
algebraic spaces over $S$. Given a scheme $T$ we can consider pairs
$(a, b)$ where $a : T \to B$
is a morphism and $b : T \times_{a, B} Z \to T \times_{a, B} X$
is a morphism over $T$. Picture
\begin{equation}
\label{equation-hom}
\vcenter{
\xymatrix{
T \times_{a, B} Z \ar[rd] \ar[rr]_b & &
T \times_{a, B} X \ar[ld] & Z \ar[rd] & & X \ar[ld] \\
& T \ar[rrr]^a & & & B
}
}
\end{equation}
Of course, we can also think of $b$ as a morphism
$b : T \times_{a, B} Z \to X$ such that
$$
\xymatrix{
T \times_{a, B} Z \ar[r] \ar[d] \ar@/^1pc/[rrr]_-b &
Z \ar[rd] & & X \ar[ld] \\
T \ar[rr]^a & & B
}
$$
commutes. In this situation we can define a functor
\begin{equation}
\label{equation-hom-functor}
\mathit{Mor}_B(Z, X) : (\Sch/S)^{opp} \longrightarrow \textit{Sets},
\quad
T \longmapsto \{(a, b)\text{ as above}\}
\end{equation}
Sometimes we think of this as a functor defined on the category
of schemes over $B$, in which case we drop $a$ from the notation.

\begin{lemma}
\label{lemma-hom-functor-sheaf}
Let $S$ be a scheme. Let $Z \to B$ and $X \to B$ be morphisms of
algebraic spaces over $S$. Then
\begin{enumerate}
\item $\mathit{Mor}_B(Z, X)$ is a sheaf on
$(\Sch/S)_{fppf}$.
\item If $T$ is an algebraic space over $S$, then there is a
canonical bijection
$$
\Mor_{\Sh((\Sch/S)_{fppf})}(T, \mathit{Mor}_B(Z, X))
=
\{(a, b)\text{ as in }(\ref{equation-hom})\}
$$
\end{enumerate}
\end{lemma}

\begin{proof}
Let $T$ be an algebraic space over $S$. Let $\{T_i \to T\}$ be an fppf
covering of $T$ (as in
Topologies on Spaces, Section \ref{spaces-topologies-section-fppf}).
Suppose that $(a_i, b_i) \in \mathit{Mor}_B(Z, X)(T_i)$ such
that $(a_i, b_i)|_{T_i \times_T T_j} = (a_j, b_j)|_{T_i \times_T T_j}$
for all $i, j$. Then by
Descent on Spaces,
Lemma \ref{spaces-descent-lemma-fpqc-universal-effective-epimorphisms}
there exists a unique morphism $a : T \to B$ such that $a_i$ is the
composition of $T_i \to T$ and $a$. Then
$\{T_i \times_{a_i, B} Z \to T \times_{a, B} Z\}$ is an fppf covering
too and the same lemma implies there exists a unique morphism
$b : T \times_{a, B} Z \to T \times_{a, B} X$ such that $b_i$ is the
composition of $T_i \times_{a_i, B} Z \to T \times_{a, B} Z$ and $b$. Hence
$(a, b) \in \mathit{Mor}_B(Z, X)(T)$ restricts to $(a_i, b_i)$
over $T_i$ for all $i$.

\medskip\noindent
Note that the result of the preceding paragraph in particular implies (1).

\medskip\noindent
Let $T$ be an algebraic space over $S$. In order to prove (2) we will
construct mutually inverse maps between the displayed sets. In the
following when we say ``pair'' we mean a pair $(a, b)$ fitting
into (\ref{equation-hom}).

\medskip\noindent
Let $v : T \to \mathit{Mor}_B(Z, X)$ be a natural transformation.
Choose a scheme $U$ and a surjective \'etale morphism $p : U \to T$.
Then $v(p) \in \mathit{Mor}_B(Z, X)(U)$ corresponds to a pair $(a_U, b_U)$
over $U$. Let $R = U \times_T U$ with projections $t, s : R \to U$.
As $v$ is a transformation of functors we see that the pullbacks of
$(a_U, b_U)$ by $s$ and $t$ agree. Hence, since $\{U \to T\}$ is an
fppf covering, we may apply the result of the first paragraph that
deduce that there exists a unique pair $(a, b)$ over $T$.

\medskip\noindent
Conversely, let $(a, b)$ be a pair over $T$.
Let $U \to T$, $R = U \times_T U$, and $t, s : R \to U$ be as
above. Then the restriction $(a, b)|_U$ gives rise to a
transformation of functors $v : h_U \to \mathit{Mor}_B(Z, X)$ by the
Yoneda lemma
(Categories, Lemma \ref{categories-lemma-yoneda}).
As the two pullbacks $s^*(a, b)|_U$ and $t^*(a, b)|_U$
are equal, we see that $v$ coequalizes the two maps
$h_t, h_s : h_R \to h_U$. Since $T = U/R$ is the fppf quotient sheaf by
Spaces, Lemma \ref{spaces-lemma-space-presentation}
and since $\mathit{Mor}_B(Z, X)$ is an fppf sheaf by (1) we conclude
that $v$ factors through a map $T \to \mathit{Mor}_B(Z, X)$.

\medskip\noindent
We omit the verification that the two constructions above are mutually
inverse.
\end{proof}

\begin{lemma}
\label{lemma-base-change-hom-functor}
Let $S$ be a scheme. Let $Z \to B$, $X \to B$, and $B' \to B$
be morphisms of algebraic spaces over $S$. Set $Z' = B' \times_B Z$
and $X' = B' \times_B X$. Then
$$
\mathit{Mor}_{B'}(Z', X')
=
B' \times_B \mathit{Mor}_B(Z, X)
$$
in $\Sh((\Sch/S)_{fppf})$.
\end{lemma}

\begin{proof}
The equality as functors follows immediately from the definitions.
The equality as sheaves follows from this because both sides are
sheaves according to
Lemma \ref{lemma-hom-functor-sheaf}
and the fact that a fibre product of sheaves is the same as the
corresponding fibre product of pre-sheaves (i.e., functors).
\end{proof}

\begin{lemma}
\label{lemma-etale-covering-hom-functor}
Let $S$ be a scheme. Let $Z \to B$ and $X' \to X \to B$ be morphisms of
algebraic spaces over $S$. Assume
\begin{enumerate}
\item $X' \to X$ is \'etale, and
\item $Z \to B$ is finite locally free.
\end{enumerate}
Then $\mathit{Mor}_B(Z, X') \to \mathit{Mor}_B(Z, X)$ is representable
by algebraic spaces and \'etale. If $X' \to X$ is also surjective,
then $\mathit{Mor}_B(Z, X') \to \mathit{Mor}_B(Z, X)$ is surjective.
\end{lemma}

\begin{proof}
Let $U$ be a scheme and let $\xi = (a, b)$ be an element of
$\mathit{Mor}_B(Z, X)(U)$. We have to prove that the functor
$$
h_U \times_{\xi, \mathit{Mor}_B(Z, X)} \mathit{Mor}_B(Z, X')
$$
is representable by an algebraic space \'etale over $U$. Set
$Z_U = U \times_{a, B} Z$ and $W = Z_U \times_{b, X} X'$.
Then $W \to Z_U \to U$ is as in
Lemma \ref{lemma-space-of-sections}
and the sheaf $F$ defined there is identified with the fibre product
displayed above. Hence the first assertion of the lemma.
The second assertion follows from this and
Lemma \ref{lemma-surjection-space-of-sections}
which guarantees that $F \to U$ is surjective in the situation above.
\end{proof}

\begin{proposition}
\label{proposition-hom-functor-algebraic-space}
Let $S$ be a scheme. Let $Z \to B$ and $X \to B$ be morphisms of
algebraic spaces over $S$. If $Z \to B$ is finite locally free
then $\mathit{Mor}_B(Z, X)$ is an algebraic space.
\end{proposition}

\begin{proof}
Choose a scheme $B' = \coprod B'_i$ which is a disjoint union of
affine schemes $B'_i$ and an \'etale surjective morphism $B' \to B$.
We may also assume that $B'_i \times_B Z$ is the spectrum of a ring
which is finite free as a $\Gamma(B'_i, \mathcal{O}_{B'_i})$-module.
By
Lemma \ref{lemma-base-change-hom-functor}
and
Spaces, Lemma
\ref{spaces-lemma-base-change-representable-transformations-property}
the morphism $\mathit{Mor}_{B'}(Z', X') \to \mathit{Mor}_B(Z, X)$
is surjective \'etale. Hence by
Bootstrap, Theorem \ref{bootstrap-theorem-final-bootstrap}
it suffices to prove the proposition when $B = B'$ is a disjoint union of
affine schemes $B'_i$ so that each $B'_i \times_B Z$ is finite free
over $B'_i$. Then it actually suffices to prove the result for the restriction
to each $B'_i$. Thus we may assume that $B$ is affine and that
$\Gamma(Z, \mathcal{O}_Z)$ is a finite free $\Gamma(B, \mathcal{O}_B)$-module.

\medskip\noindent
Choose a scheme $X'$ which is a disjoint union of affine schemes and
a surjective \'etale morphism $X' \to X$. By
Lemma \ref{lemma-etale-covering-hom-functor}
the morphism $\mathit{Mor}_B(Z, X') \to \mathit{Mor}_B(Z, X)$
is representable by algebraic spaces, \'etale, and surjective.
Hence by
Bootstrap, Theorem \ref{bootstrap-theorem-final-bootstrap}
it suffices to prove the proposition when $X$ is a disjoint union
of affine schemes. This reduces us to the case discussed in the next
paragraph.

\medskip\noindent
Assume $X = \coprod_{i \in I} X_i$ is a disjoint union of affine
schemes, $B$ is affine, and that $\Gamma(Z, \mathcal{O}_Z)$ is a finite
free $\Gamma(B, \mathcal{O}_B)$-module. For any finite subset
$E \subset I$ set
$$
F_E = \mathit{Mor}_B(Z, \coprod\nolimits_{i \in E} X_i).
$$
By More on Morphisms,
Lemma \ref{more-morphisms-lemma-hom-from-finite-free-into-affine}
we see that $F_E$ is an algebraic space. Consider the morphism
$$
\coprod\nolimits_{E \subset I\text{ finite}} F_E
\longrightarrow
\mathit{Mor}_B(Z, X)
$$
Each of the morphisms
$F_E \to \mathit{Mor}_B(Z, X)$ is an open immersion, because it is
simply the locus parametrizing pairs $(a, b)$ where $b$ maps into
the open subscheme $\coprod\nolimits_{i \in E} X_i$ of $X$. Moreover,
if $T$ is quasi-compact, then for any pair $(a, b)$ the image
of $b$ is contained in $\coprod\nolimits_{i \in E} X_i$ for some
$E \subset I$ finite. Hence the displayed arrow is in fact an
open covering and we win\footnote{Modulo
some set theoretic arguments. Namely, we have to show that
$\coprod F_E$ is an algebraic space. This follows because
$|I| \leq \text{size}(X)$ and $\text{size}(F_E) \leq \text{size}(X)$
as follows from the explicit description of $F_E$ in the proof of
More on Morphisms,
Lemma \ref{more-morphisms-lemma-hom-from-finite-free-into-affine}.
Some details omitted.} by
Spaces, Lemma \ref{spaces-lemma-glueing-algebraic-spaces}.
\end{proof}










\section{Restriction of scalars}
\label{section-restriction-of-scalars}

\noindent
Suppose $X \to Z \to B$ are morphisms of algebraic spaces over $S$.
Given a scheme $T$ we can consider pairs $(a, b)$ where $a : T \to B$
is a morphism and $b : T \times_{a, B} Z \to X$ is a morphism over $Z$.
Picture
\begin{equation}
\label{equation-pairs}
\vcenter{
\xymatrix{
& X \ar[d] \\
T \times_{a, B} Z \ar[d] \ar[ru]^b \ar[r] & Z \ar[d] \\
T \ar[r]^a & B
}
}
\end{equation}
In this situation we can define a
functor
\begin{equation}
\label{equation-restriction-of-scalars}
\text{Res}_{Z/B}(X) : (\Sch/S)^{opp} \longrightarrow \textit{Sets},
\quad
T \longmapsto \{(a, b)\text{ as above}\}
\end{equation}
Sometimes we think of this as a functor defined on the category
of schemes over $B$, in which case we drop $a$ from the notation.

\begin{lemma}
\label{lemma-restriction-of-scalars-sheaf}
Let $S$ be a scheme. Let $X \to Z \to B$ be morphisms of
algebraic spaces over $S$. Then
\begin{enumerate}
\item $\text{Res}_{Z/B}(X)$ is a sheaf on
$(\Sch/S)_{fppf}$.
\item If $T$ is an algebraic space over $S$, then there is a
canonical bijection
$$
\Mor_{\Sh((\Sch/S)_{fppf})}(T, \text{Res}_{Z/B}(X))
=
\{(a, b)\text{ as in }(\ref{equation-pairs})\}
$$
\end{enumerate}
\end{lemma}

\begin{proof}
Let $T$ be an algebraic space over $S$. Let $\{T_i \to T\}$ be an fppf
covering of $T$ (as in
Topologies on Spaces, Section \ref{spaces-topologies-section-fppf}).
Suppose that $(a_i, b_i) \in \text{Res}_{Z/B}(X)(T_i)$ such
that $(a_i, b_i)|_{T_i \times_T T_j} = (a_j, b_j)|_{T_i \times_T T_j}$
for all $i, j$. Then by
Descent on Spaces,
Lemma \ref{spaces-descent-lemma-fpqc-universal-effective-epimorphisms}
there exists a unique morphism $a : T \to B$ such that $a_i$ is the
composition of $T_i \to T$ and $a$. Then
$\{T_i \times_{a_i, B} Z \to T \times_{a, B} Z\}$ is an fppf covering
too and the same lemma implies there exists a unique morphism
$b : T \times_{a, B} Z \to X$ such that $b_i$ is the composition
of $T_i \times_{a_i, B} Z \to T \times_{a, B} Z$ and $b$. Hence
$(a, b) \in \text{Res}_{Z/B}(X)(T)$ restricts to $(a_i, b_i)$
over $T_i$ for all $i$.

\medskip\noindent
Note that the result of the preceding paragraph in particular implies (1).

\medskip\noindent
Let $T$ be an algebraic space over $S$. In order to prove (2) we will
construct mutually inverse maps between the displayed sets. In the
following when we say ``pair'' we mean a pair $(a, b)$ fitting
into (\ref{equation-pairs}).

\medskip\noindent
Let $v : T \to \text{Res}_{Z/B}(X)$ be a natural transformation.
Choose a scheme $U$ and a surjective \'etale morphism $p : U \to T$.
Then $v(p) \in \text{Res}_{Z/B}(X)(U)$ corresponds to a pair $(a_U, b_U)$
over $U$. Let $R = U \times_T U$ with projections $t, s : R \to U$.
As $v$ is a transformation of functors we see that the pullbacks of
$(a_U, b_U)$ by $s$ and $t$ agree. Hence, since $\{U \to T\}$ is an
fppf covering, we may apply the result of the first paragraph that
deduce that there exists a unique pair $(a, b)$ over $T$.

\medskip\noindent
Conversely, let $(a, b)$ be a pair over $T$.
Let $U \to T$, $R = U \times_T U$, and $t, s : R \to U$ be as
above. Then the restriction $(a, b)|_U$ gives rise to a
transformation of functors $v : h_U \to \text{Res}_{Z/B}(X)$ by the
Yoneda lemma
(Categories, Lemma \ref{categories-lemma-yoneda}).
As the two pullbacks $s^*(a, b)|_U$ and $t^*(a, b)|_U$
are equal, we see that $v$ coequalizes the two maps
$h_t, h_s : h_R \to h_U$. Since $T = U/R$ is the fppf quotient sheaf by
Spaces, Lemma \ref{spaces-lemma-space-presentation}
and since $\text{Res}_{Z/B}(X)$ is an fppf sheaf by (1) we conclude
that $v$ factors through a map $T \to \text{Res}_{Z/B}(X)$.

\medskip\noindent
We omit the verification that the two constructions above are mutually
inverse.
\end{proof}

\noindent
Of course the sheaf $\text{Res}_{Z/B}(X)$ comes with a natural transformation
of functors $\text{Res}_{Z/B}(X) \to B$. We will use this without further
mention in the following.

\begin{lemma}
\label{lemma-etale-base-change-restriction-of-scalars}
Let $S$ be a scheme. Let $X \to Z \to B$ and $B' \to B$
be morphisms of algebraic spaces over $S$.
Set $Z' = B' \times_B Z$ and $X' = B' \times_B X$. Then
$$
\text{Res}_{Z'/B'}(X')
=
B' \times_B \text{Res}_{Z/B}(X)
$$
in $\Sh((\Sch/S)_{fppf})$.
\end{lemma}

\begin{proof}
The equality as functors follows immediately from the definitions.
The equality as sheaves follows from this because both sides are
sheaves according to
Lemma \ref{lemma-restriction-of-scalars-sheaf}
and the fact that a fibre product of sheaves is the same as the
corresponding fibre product of pre-sheaves (i.e., functors).
\end{proof}

\begin{lemma}
\label{lemma-etale-covering-restriction-of-scalars}
Let $S$ be a scheme. Let $X' \to X \to Z \to B$ be morphisms of
algebraic spaces over $S$. Assume
\begin{enumerate}
\item $X' \to X$ is \'etale, and
\item $Z \to B$ is finite locally free.
\end{enumerate}
Then $\text{Res}_{Z/B}(X') \to \text{Res}_{Z/B}(X)$ is representable
by algebraic spaces and \'etale. If $X' \to X$ is also surjective,
then $\text{Res}_{Z/B}(X') \to \text{Res}_{Z/B}(X)$ is surjective.
\end{lemma}

\begin{proof}
Let $U$ be a scheme and let $\xi = (a, b)$ be an element of
$\text{Res}_{Z/B}(X)(U)$. We have to prove that the functor
$$
h_U \times_{\xi, \text{Res}_{Z/B}(X)} \text{Res}_{Z/B}(X')
$$
is representable by an algebraic space \'etale over $U$. Set
$Z_U = U \times_{a, B} Z$ and $W = Z_U \times_{b, X} X'$.
Then $W \to Z_U \to U$ is as in
Lemma \ref{lemma-space-of-sections}
and the sheaf $F$ defined there is identified with the fibre product
displayed above. Hence the first assertion of the lemma.
The second assertion follows from this and
Lemma \ref{lemma-surjection-space-of-sections}
which guarantees that $F \to U$ is surjective in the situation above.
\end{proof}

\noindent
At this point we can use the lemmas above to prove that $\text{Res}_{Z/B}(X)$
is an algebraic space whenever $Z \to B$ is finite locally free in almost
exactly the same way as in the proof that $\mathit{Mor}_B(Z, X)$ is an
algebraic spaces, see
Proposition \ref{proposition-hom-functor-algebraic-space}.
Instead we will directly deduce this result from the following lemma
and the fact that $\mathit{Mor}_B(Z, X)$ is an algebraic space.

\begin{lemma}
\label{lemma-fibre-diagram}
Let $S$ be a scheme. Let $X \to Z \to B$ be morphisms of
algebraic spaces over $S$. The following diagram
$$
\xymatrix{
\mathit{Mor}_B(Z, X) \ar[r] & \mathit{Mor}_B(Z, Z) \\
\text{Res}_{Z/B}(X) \ar[r] \ar[u] & B \ar[u]_{\text{id}_Z}
}
$$
is a cartesian diagram of sheaves on $(\Sch/S)_{fppf}$.
\end{lemma}

\begin{proof}
Omitted. Hint: Exercise in the functorial point of view in algebraic
geometry.
\end{proof}

\begin{proposition}
\label{proposition-restriction-of-scalars-algebraic-space}
Let $S$ be a scheme. Let $X \to Z \to B$ be morphisms of
algebraic spaces over $S$. If $Z \to B$ is finite locally free
then $\text{Res}_{Z/B}(X)$ is an algebraic space.
\end{proposition}

\begin{proof}
By
Proposition \ref{proposition-hom-functor-algebraic-space}
the functors $\mathit{Mor}_B(Z, X)$ and $\mathit{Mor}_B(Z, Z)$
are algebraic spaces. Hence this follows from the cartesian
diagram of
Lemma \ref{lemma-fibre-diagram}
and the fact that fibre products of algebraic spaces exist and
are given by the fibre product in the underlying category of
sheaves of sets (see
Spaces, Lemma
\ref{spaces-lemma-fibre-product-spaces-over-sheaf-with-representable-diagonal}).
\end{proof}






\section{Finite Hilbert stacks}
\label{section-finite-hilbert-stacks}

\noindent
In this section we prove some results concerning the finite
Hilbert stacks $\mathcal{H}_d(\mathcal{X}/\mathcal{Y})$
introduced in
Examples of Stacks, Section \ref{examples-stacks-section-hilbert-d-stack}.

\begin{lemma}
\label{lemma-map-hilbert}
Consider a $2$-commutative diagram
$$
\xymatrix{
\mathcal{X}' \ar[r]_G \ar[d]_{F'} & \mathcal{X} \ar[d]^F \\
\mathcal{Y}' \ar[r]^H & \mathcal{Y}
}
$$
of stacks in groupoids over $(\Sch/S)_{fppf}$ with a given
$2$-isomorphism $\gamma : H \circ F' \to F \circ G$. In this situation we
obtain a canonical $1$-morphism
$\mathcal{H}_d(\mathcal{X}'/\mathcal{Y}') \to
\mathcal{H}_d(\mathcal{X}/\mathcal{Y})$.
This morphism is compatible with the forgetful $1$-morphisms of
Examples of Stacks,
Equation (\ref{examples-stacks-equation-diagram-hilbert-d-stack}).
\end{lemma}

\begin{proof}
We map the object $(U, Z, y', x', \alpha')$ to the object
$(U, Z, H(y'), G(x'), \gamma \star \text{id}_H \star \alpha')$
where $\star$ denotes horizontal composition of $2$-morphisms, see
Categories, Definition \ref{categories-definition-horizontal-composition}.
To a morphism
$(f, g, b, a) :
(U_1, Z_1, y_1', x_1', \alpha_1') \to (U_2, Z_2, y_2', x_2', \alpha_2')$
we assign
$(f, g, H(b), G(a))$.
We omit the verification that this defines a functor between categories over
$(\Sch/S)_{fppf}$.
\end{proof}

\begin{lemma}
\label{lemma-cartesian-map-hilbert}
In the situation of
Lemma \ref{lemma-map-hilbert}
assume that the given square is $2$-cartesian. Then the diagram
$$
\xymatrix{
\mathcal{H}_d(\mathcal{X}'/\mathcal{Y}') \ar[r] \ar[d] &
\mathcal{H}_d(\mathcal{X}/\mathcal{Y}) \ar[d] \\
\mathcal{Y}' \ar[r] &
\mathcal{Y}
}
$$
is $2$-cartesian.
\end{lemma}

\begin{proof}
We get a $2$-commutative diagram by
Lemma \ref{lemma-map-hilbert}
and hence we get a $1$-morphism (i.e., a functor)
$$
\mathcal{H}_d(\mathcal{X}'/\mathcal{Y}')
\longrightarrow
\mathcal{Y}' \times_\mathcal{Y} \mathcal{H}_d(\mathcal{X}/\mathcal{Y})
$$
We indicate why this functor is essentially surjective. Namely, an object
of the category on the right hand side is given by a scheme $U$ over $S$,
an object $y'$ of $\mathcal{Y}'_U$, an object $(U, Z, y, x, \alpha)$
of $\mathcal{H}_d(\mathcal{X}/\mathcal{Y})$ over $U$ and an isomorphism
$H(y') \to y$ in $\mathcal{Y}_U$. The assumption means exactly that
there exists an object $x'$ of $\mathcal{X}'_Z$ such that there exist
isomorphisms $G(x') \cong x$ and $\alpha' : y'|_Z \to F'(x')$ compatible
with $\alpha$. Then we see that $(U, Z, y', x', \alpha')$ is an
object of $\mathcal{H}_d(\mathcal{X}'/\mathcal{Y}')$ over $U$.
Details omitted.
\end{proof}

\begin{lemma}
\label{lemma-etale-covering-hilbert}
In the situation of
Lemma \ref{lemma-map-hilbert}
assume
\begin{enumerate}
\item $\mathcal{Y}' = \mathcal{Y}$ and $H = \text{id}_\mathcal{Y}$,
\item $G$ is representable by algebraic spaces and \'etale.
\end{enumerate}
Then $\mathcal{H}_d(\mathcal{X}'/\mathcal{Y}) \to
\mathcal{H}_d(\mathcal{X}/\mathcal{Y})$ is representable by
algebraic spaces and \'etale.
If $G$ is also surjective, then
$\mathcal{H}_d(\mathcal{X}'/\mathcal{Y}) \to
\mathcal{H}_d(\mathcal{X}/\mathcal{Y})$ is surjective.
\end{lemma}

\begin{proof}
Let $U$ be a scheme and let $\xi = (U, Z, y, x, \alpha)$ be an object of
$\mathcal{H}_d(\mathcal{X}/\mathcal{Y})$ over $U$.
We have to prove that the $2$-fibre product
\begin{equation}
\label{equation-to-show}
(\Sch/U)_{fppf}
\times_{\xi, \mathcal{H}_d(\mathcal{X}/\mathcal{Y})}
\mathcal{H}_d(\mathcal{X}'/\mathcal{Y})
\end{equation}
is representable by an algebraic space \'etale over $U$.
An object of this over $U'$ corresponds to an object
$x'$ in the fibre category of $\mathcal{X}'$ over $Z_{U'}$
such that $G(x') \cong x|_{Z_{U'}}$.
By assumption the $2$-fibre product
$$
(\Sch/Z)_{fppf} \times_{x, \mathcal{X}} \mathcal{X}'
$$
is representable by an algebraic space $W$ such that the projection
$W \to Z$ is \'etale. Then (\ref{equation-to-show})
is representable by the algebraic space $F$ parametrizing sections of
$W \to Z$ over $U$ introduced in
Lemma \ref{lemma-space-of-sections}.
Since $F \to U$ is \'etale we conclude that
$\mathcal{H}_d(\mathcal{X}'/\mathcal{Y}) \to
\mathcal{H}_d(\mathcal{X}/\mathcal{Y})$ is representable by
algebraic spaces and \'etale.
Finally, if $\mathcal{X}' \to \mathcal{X}$ is surjective also,
then $W \to Z$ is surjective, and hence $F \to U$ is surjective by
Lemma \ref{lemma-surjection-space-of-sections}.
Thus in this case
$\mathcal{H}_d(\mathcal{X}'/\mathcal{Y}) \to
\mathcal{H}_d(\mathcal{X}/\mathcal{Y})$ is also surjective.
\end{proof}

\begin{lemma}
\label{lemma-etale-map-hilbert}
In the situation of
Lemma \ref{lemma-map-hilbert}.
Assume that $G$, $H$ are representable by algebraic spaces and \'etale.
Then $\mathcal{H}_d(\mathcal{X}'/\mathcal{Y}') \to
\mathcal{H}_d(\mathcal{X}/\mathcal{Y})$ is representable by
algebraic spaces and \'etale.
If also $H$ is surjective and the induced functor
$\mathcal{X}' \to \mathcal{Y}' \times_\mathcal{Y} \mathcal{X}$
is surjective, then
$\mathcal{H}_d(\mathcal{X}'/\mathcal{Y}') \to
\mathcal{H}_d(\mathcal{X}/\mathcal{Y})$ is surjective.
\end{lemma}

\begin{proof}
Set $\mathcal{X}'' = \mathcal{Y}' \times_\mathcal{Y} \mathcal{X}$. By
Lemma \ref{lemma-etale-permanence}
the $1$-morphism $\mathcal{X}' \to \mathcal{X}''$ is representable by
algebraic spaces and \'etale (in particular the condition in the second
statement of the lemma that $\mathcal{X}' \to \mathcal{X}''$ be surjective
makes sense). We obtain a $2$-commutative diagram
$$
\xymatrix{
\mathcal{X}' \ar[r] \ar[d] &
\mathcal{X}'' \ar[r] \ar[d] &
\mathcal{X} \ar[d] \\
\mathcal{Y}' \ar[r] &
\mathcal{Y}' \ar[r] &
\mathcal{Y}
}
$$
It follows from
Lemma \ref{lemma-cartesian-map-hilbert}
that $\mathcal{H}_d(\mathcal{X}''/\mathcal{Y}')$ is the base change
of $\mathcal{H}_d(\mathcal{X}/\mathcal{Y})$ by $\mathcal{Y}' \to \mathcal{Y}$.
In particular we see that
$\mathcal{H}_d(\mathcal{X}''/\mathcal{Y}') \to
\mathcal{H}_d(\mathcal{X}/\mathcal{Y})$ is
representable by algebraic spaces and \'etale, see
Algebraic Stacks, Lemma
\ref{algebraic-lemma-base-change-representable-transformations-property}.
Moreover, it is also surjective if $H$ is.
 Hence if we can show that
the result holds for the left square in the diagram, then we're done.
In this way we reduce to the case where $\mathcal{Y}' = \mathcal{Y}$
which is the content of
Lemma \ref{lemma-etale-covering-hilbert}.
\end{proof}

\begin{lemma}
\label{lemma-relative-hilbert}
Let $F : \mathcal{X} \to \mathcal{Y}$ be a $1$-morphism of stacks in groupoids
over $(\Sch/S)_{fppf}$. Assume that
$\Delta : \mathcal{Y} \to \mathcal{Y} \times \mathcal{Y}$
is representable by algebraic spaces. Then
$$
\mathcal{H}_d(\mathcal{X}/\mathcal{Y})
\longrightarrow
\mathcal{H}_d(\mathcal{X}) \times \mathcal{Y}
$$
see
Examples of Stacks, Equation
(\ref{examples-stacks-equation-diagram-hilbert-d-stack})
is representable by algebraic spaces.
\end{lemma}

\begin{proof}
Let $U$ be a scheme and let $\xi = (U, Z, p, x, 1)$ be an object of
$\mathcal{H}_d(\mathcal{X}) = \mathcal{H}_d(\mathcal{X}/S)$ over $U$.
Here $p$ is just the structure morphism of $U$.
The fifth component $1$ exists and is unique
since everything is over $S$.
Also, let $y$ be an object of $\mathcal{Y}$ over $U$.
We have to show the $2$-fibre product
\begin{equation}
\label{equation-res-isom}
(\Sch/U)_{fppf}
\times_{\xi \times y, \mathcal{H}_d(\mathcal{X}) \times \mathcal{Y}}
\mathcal{H}_d(\mathcal{X}/\mathcal{Y})
\end{equation}
is representable by an algebraic space. To explain why this is so
we introduce
$$
I = \mathit{Isom}_\mathcal{Y}(y|_Z, F(x))
$$
which is an algebraic space over $Z$ by assumption. Let $a : U' \to U$
be a scheme over $U$. What does it mean to give an object of the fibre
category of (\ref{equation-res-isom}) over $U'$? Well, it means that we
have an object $\xi' = (U', Z', y', x', \alpha')$ of
$\mathcal{H}_d(\mathcal{X}/\mathcal{Y})$ over $U'$ and isomorphisms
$(U', Z', p', x', 1) \cong (U, Z, p, x, 1)|_{U'}$ and
$y' \cong y|_{U'}$. Thus $\xi'$ is isomorphic to
$(U', U' \times_{a, U} Z, a^*y, x|_{U' \times_{a, U} Z}, \alpha)$
for some morphism
$$
\alpha :
a^*y|_{U' \times_{a, U} Z}
\longrightarrow
F(x|_{U' \times_{a, U} Z})
$$
in the fibre category of $\mathcal{Y}$ over $U' \times_{a, U} Z$. Hence
we can view $\alpha$ as a morphism $b : U' \times_{a, U} Z \to I$.
In this way we see that (\ref{equation-res-isom})
is representable by $\text{Res}_{Z/U}(I)$ which is an algebraic space by
Proposition \ref{proposition-restriction-of-scalars-algebraic-space}.
\end{proof}

\noindent
The following lemma is a (partial) generalization of
Lemma \ref{lemma-etale-covering-hilbert}.

\begin{lemma}
\label{lemma-representable-on-top}
Let $F : \mathcal{X} \to \mathcal{Y}$ and $G : \mathcal{X}' \to \mathcal{X}$
be $1$-morphisms of stacks in groupoids over $(\Sch/S)_{fppf}$.
If $G$ is representable by algebraic spaces, then the $1$-morphism
$$
\mathcal{H}_d(\mathcal{X}'/\mathcal{Y})
\longrightarrow
\mathcal{H}_d(\mathcal{X}/\mathcal{Y})
$$
is representable by algebraic spaces.
\end{lemma}

\begin{proof}
Let $U$ be a scheme and let $\xi = (U, Z, y, x, \alpha)$ be an object of
$\mathcal{H}_d(\mathcal{X}/\mathcal{Y})$ over $U$.
We have to prove that the $2$-fibre product
\begin{equation}
\label{equation-to-show-again}
(\Sch/U)_{fppf}
\times_{\xi, \mathcal{H}_d(\mathcal{X}/\mathcal{Y})}
\mathcal{H}_d(\mathcal{X}'/\mathcal{Y})
\end{equation}
is representable by an algebraic space \'etale over $U$.
An object of this over $a : U' \to U$ corresponds to an object
$x'$ of $\mathcal{X}'$ over $U' \times_{a, U} Z$ such that
$G(x') \cong x|_{U' \times_{a, U} Z}$. By assumption the $2$-fibre product
$$
(\Sch/Z)_{fppf} \times_{x, \mathcal{X}} \mathcal{X}'
$$
is representable by an algebraic space $X$ over $Z$. It follows that
(\ref{equation-to-show-again}) is representable by $\text{Res}_{Z/U}(X)$,
which is an algebraic space by
Proposition \ref{proposition-restriction-of-scalars-algebraic-space}.
\end{proof}

\begin{lemma}
\label{lemma-limit-preserving}
Let $F : \mathcal{X} \to \mathcal{Y}$ be a $1$-morphism of stacks in groupoids
over $(\Sch/S)_{fppf}$. Assume $F$ is representable by algebraic
spaces and locally of finite presentation. Then
$$
p : \mathcal{H}_d(\mathcal{X}/\mathcal{Y}) \to \mathcal{Y}
$$
is limit preserving on objects.
\end{lemma}

\begin{proof}
This means we have to show the following: Given
\begin{enumerate}
\item an affine scheme $U = \lim_i U_i$ which is written as the
directed limit of affine schemes $U_i$ over $S$,
\item an object $y_i$ of $\mathcal{Y}$ over $U_i$ for some $i$, and
\item an object $\Xi = (U, Z, y, x, \alpha)$ of
$\mathcal{H}_d(\mathcal{X}/\mathcal{Y})$
over $U$ such that $y = y_i|_U$,
\end{enumerate}
then there exists an $i' \geq i$ and an object
$\Xi_{i'} = (U_{i'}, Z_{i'}, y_{i'}, x_{i'}, \alpha_{i'})$ of
$\mathcal{H}_d(\mathcal{X}/\mathcal{Y})$ over $U_{i'}$ with
$\Xi_{i'}|_U = \Xi$ and $y_{i'} = y_i|_{U_{i'}}$.
Namely, the last two equalities will take care of the commutativity of
(\ref{equation-limit-preserving}).

\medskip\noindent
Let $X_{y_i} \to U_i$ be an algebraic space representing the $2$-fibre
product
$$
(\Sch/U_i)_{fppf} \times_{y_i, \mathcal{Y}, F} \mathcal{X}.
$$
Note that $X_{y_i} \to U_i$ is locally of finite presentation by our
assumption on $F$. Write $\Xi $. It is clear that
$\xi = (Z, Z \to U_i, x, \alpha)$ is an object of the $2$-fibre product
displayed above, hence $\xi$ gives rise to a morphism
$f_\xi : Z \to X_{y_i}$ of algebraic spaces over $U_i$
(since $X_{y_i}$ is the functor of isomorphisms classes of objects of
$(\Sch/U_i)_{fppf} \times_{y, \mathcal{Y}, F} \mathcal{X}$, see
Algebraic Stacks,
Lemma \ref{algebraic-lemma-characterize-representable-by-space}).
By
Limits, Lemmas \ref{limits-lemma-descend-finite-presentation} and
\ref{limits-lemma-descend-finite-locally-free}
there exists an $i' \geq i$ and a finite locally free morphism
$Z_{i'} \to U_{i'}$ of degree $d$ whose base change to $U$ is $Z$. By
Limits of Spaces, Proposition
\ref{spaces-limits-proposition-characterize-locally-finite-presentation}
we may, after replacing $i'$ by a bigger index, assume there exists a
morphism $f_{i'} : Z_{i'} \to X_{y_i}$ such that
$$
\xymatrix{
Z \ar[d] \ar[r] \ar@/^3ex/[rr]^{f_\xi} &
Z_{i'} \ar[d] \ar[r]_{f_{i'}} & X_{y_i} \ar[d] \\
U \ar[r] & U_{i'} \ar[r] & U_i
}
$$
is commutative. We set
$\Xi_{i'} = (U_{i'}, Z_{i'}, y_{i'}, x_{i'}, \alpha_{i'})$
where
\begin{enumerate}
\item $y_{i'}$ is the object of $\mathcal{Y}$ over $U_{i'}$
which is the pullback of $y_i$ to $U_{i'}$,
\item $x_{i'}$ is the object of $\mathcal{X}$ over $Z_{i'}$ corresponding
via the $2$-Yoneda lemma to the $1$-morphism
$$
(\Sch/Z_{i'})_{fppf} \to
\mathcal{S}_{X_{y_i}} \to
(\Sch/U_i)_{fppf} \times_{y_i, \mathcal{Y}, F} \mathcal{X} \to
\mathcal{X}
$$
where the middle arrow is the equivalence which defines $X_{y_i}$
(notation as in
Algebraic Stacks, Sections
\ref{algebraic-section-representable-by-algebraic-spaces} and
\ref{algebraic-section-split}).
\item $\alpha_{i'} : y_{i'}|_{Z_{i'}} \to F(x_{i'})$ is the isomorphism
coming from the $2$-commutativity of the diagram
$$
\xymatrix{
(\Sch/Z_{i'})_{fppf} \ar[r] \ar[rd] &
(\Sch/U_i)_{fppf} \times_{y_i, \mathcal{Y}, F} \mathcal{X}
\ar[r] \ar[d] &
\mathcal{X} \ar[d]^F \\
& (\Sch/U_{i'})_{fppf} \ar[r] & \mathcal{Y}
}
$$
\end{enumerate}
Recall that $f_\xi : Z \to X_{y_i}$ was the morphism corresponding to
the object $\xi = (Z, Z \to U_i, x, \alpha)$ of
$(\Sch/U_i)_{fppf} \times_{y_i, \mathcal{Y}, F} \mathcal{X}$
over $Z$. By construction $f_{i'}$ is the morphism corresponding to
the object $\xi_{i'} = (Z_{i'}, Z_{i'} \to U_i, x_{i'}, \alpha_{i'})$.
As $f_\xi = f_{i'} \circ (Z \to Z_{i'})$ we see that
the object $\xi_{i'} = (Z_{i'}, Z_{i'} \to U_i, x_{i'}, \alpha_{i'})$ pulls
back to $\xi$ over $Z$. Thus $x_{i'}$ pulls back to $x$ and $\alpha_{i'}$
pulls back to $\alpha$. This means that $\Xi_{i'}$ pulls back
to $\Xi$ over $U$ and we win.
\end{proof}










\section{The finite Hilbert stack of a point}
\label{section-hilbert-point}

\noindent
Let $d \geq 1$ be an integer. In
Examples of Stacks, Definition \ref{examples-stacks-definition-hilbert-d-stack}
we defined a stack in groupoids $\mathcal{H}_d$.
In this section we prove that $\mathcal{H}_d$ is an
algebraic stack. We will throughout assume that
$S = \Spec(\mathbf{Z})$.
The general case will follow from this by base change.
Recall that the fibre category of $\mathcal{H}_d$ over a scheme $T$
is the category of finite locally free morphisms $\pi : Z \to T$ of
degree $d$. Instead of classifying these directly we first
study the quasi-coherent sheaves of algebras $\pi_*\mathcal{O}_Z$.

\medskip\noindent
Let $R$ be a ring. Let us temporarily make the following definition:
A {\it free $d$-dimensional algebra over $R$}
is given by a commutative $R$-algebra structure $m$ on $R^{\oplus d}$
such that $e_1 = (1, 0, \ldots, 0)$ is a unit\footnote{It may be better
to think of this as a pair consisting of a multiplication map
$m : R^{\oplus d} \otimes_R R^{\oplus d} \to R^{\oplus d}$ and
a ring map $\psi : R \to R^{\oplus d}$ satisfying a bunch of axioms.}.
We think of $m$ as an $R$-linear map
$$
m : R^{\oplus d} \otimes_R R^{\oplus d} \longrightarrow R^{\oplus d}
$$
such that $m(e_1, x) = m(x, e_1) = x$ and such that $m$ defines a
commutative and associative ring structure. If we write
$m(e_i, e_j) = \sum a_{ij}^ke_k$ then we see this boils down
to the conditions
$$
\left\{
\begin{matrix}
\sum_l a_{ij}^la_{lk}^m = \sum_l a_{il}^ma_{jk}^l & \forall i, j, k, m \\
a_{ij}^k = a_{ji}^k & \forall i, j, k \\
a_{i1}^j = \delta_{ij} & \forall i, j
\end{matrix}
\right.
$$
where $\delta_{ij}$ is the Kronecker $\delta$-function. OK, so let's define
$$
R_{univ} = \mathbf{Z}[a_{ij}^k]/J
$$
where the ideal $J$ is the ideal generated by the relations displayed above.
Denote
$$
m_{univ} :
R_{univ}^{\oplus d} \otimes_{R_{univ}} R_{univ}^{\oplus d}
\longrightarrow
R_{univ}^{\oplus d}
$$
the free $d$-dimensional algebra $m$ over $R_{univ}$ whose structure
constants are the classes of $a_{ij}^k$ modulo $J$.
Then it is clear that given any free $d$-dimensional algebra $m$ over a ring
$R$ there exists a unique $\mathbf{Z}$-algebra homomorphism
$\psi : R_{univ} \to R$ such that $\psi_*m_{univ} = m$ (this means that
$m$ is what you get by applying the base change functor
$- \otimes_{R_{univ}} R$ to $m_{univ}$). In other words, setting
$X = \Spec(R_{univ})$ we obtain a canonical identification
$$
X(T) = \{\text{free }d\text{-dimensional algebras }m\text{ over }R\}
$$
for varying $T = \Spec(R)$. By Zariski localization we obtain
the following seemingly more general identification
\begin{equation}
\label{equation-objects}
X(T) = \{\text{free }d\text{-dimensional algebras }
m\text{ over }\Gamma(T, \mathcal{O}_T)\}
\end{equation}
for any scheme $T$.

\medskip\noindent
Next we talk a little bit about {\it isomorphisms of free $d$-dimensional
$R$-algebras}. Namely, suppose that $m$, $m'$ are two free $d$-dimensional
algebras over a ring $R$. An {\it isomorphism from $m$ to $m'$} is given by
an invertible $R$-linear map
$$
\varphi : R^{\oplus d} \longrightarrow R^{\oplus d}
$$
such that $\varphi(e_1) = e_1$ and such that
$$
m \circ \varphi \otimes \varphi = \varphi \circ m'.
$$
Note that we can compose these so that the collection of
free $d$-dimensional algebras over $R$ becomes a category.
In this way we obtain a functor
\begin{equation}
\label{equation-FAd}
FA_d : \Sch_{fppf}^{opp} \longrightarrow \textit{Groupoids}
\end{equation}
from the category of schemes to groupoids: to a scheme $T$ we associate the
set of free $d$-dimensional algebras over $\Gamma(T, \mathcal{O}_T)$
endowed with the structure
of a category using the notion of isomorphisms just defined.

\medskip\noindent
The above suggests we consider the functor $G$ in groups
which associates to any scheme $T$ the group
$$
G(T) = \{g \in \text{GL}_d(\Gamma(T, \mathcal{O}_T)) \mid g(e_1) = e_1\}
$$
It is clear that $G \subset \text{GL}_d$ (see
Groupoids, Example \ref{groupoids-example-general-linear-group})
is the closed subgroup scheme cut out by the equations
$x_{11} = 1$ and $x_{i1} = 0$ for $i > 1$. Hence $G$ is a smooth
affine group scheme over $\Spec(\mathbf{Z})$. Consider the
action
$$
a : G \times_{\Spec(\mathbf{Z})} X \longrightarrow X
$$
which associates to a $T$-valued point $(g, m)$ with $T = \Spec(R)$
on the left hand side the free $d$-dimensional algebra over $R$
given by
$$
a(g, m) = g^{-1} \circ m \circ g \otimes g.
$$
Note that this means that $g$ defines an isomorphism $m \to a(g, m)$
of $d$-dimensional free $R$-algebras. We omit the verification that
$a$ indeed defines an action of the group scheme $G$ on the scheme $X$.

\begin{lemma}
\label{lemma-represent-FAd}
The functor in groupoids $FA_d$ defined in (\ref{equation-FAd})
is isomorphic (!) to the functor in groupoids which associates
to a scheme $T$ the category with
\begin{enumerate}
\item set of objects is $X(T)$,
\item set of morphisms is $G(T) \times X(T)$,
\item $s : G(T) \times X(T) \to X(T)$ is the projection map,
\item $t : G(T) \times X(T) \to X(T)$ is $a(T)$, and
\item composition $G(T) \times X(T) \times_{s, X(T), t} G(T) \times X(T)
\to G(T) \times X(T)$ is given by $((g, m), (g', m')) \mapsto (gg', m')$.
\end{enumerate}
\end{lemma}

\begin{proof}
We have seen the rule on objects in (\ref{equation-objects}).
We have also seen above that $g \in G(T)$ can be viewed as
a morphism from $m$ to $a(g, m)$ for any free $d$-dimensional algebra $m$.
Conversely, any morphism $m \to m'$ is given by an invertible linear
map $\varphi$ which corresponds to an element $g \in G(T)$ such
that $m' = a(g, m)$.
\end{proof}

\noindent
In fact the groupoid $(X, G \times X, s, t, c)$ described in the
lemma above is the groupoid associated to the action $a : G \times X \to X$
as defined in
Groupoids, Lemma \ref{groupoids-lemma-groupoid-from-action}.
Since $G$ is smooth over $\Spec(\mathbf{Z})$
we see that the two morphisms $s, t : G \times X \to X$ are
smooth: by symmetry it suffices to prove that one of them is, and
$s$ is the base change of $G \to \Spec(\mathbf{Z})$.
Hence $(G \times X, X, s, t, c)$ is a smooth groupoid scheme,
and the quotient stack $[X/G]$ is an algebraic stack by
Algebraic Stacks,
Theorem \ref{algebraic-theorem-smooth-groupoid-gives-algebraic-stack}.

\begin{proposition}
\label{proposition-finite-hilbert-point}
The stack $\mathcal{H}_d$ is equivalent to the quotient stack
$[X/G]$ described above. In particular $\mathcal{H}_d$ is an
algebraic stack.
\end{proposition}

\begin{proof}
Note that by
Groupoids in Spaces, Definition
\ref{spaces-groupoids-definition-quotient-stack}
the quotient stack $[X/G]$ is the stackification of the
category fibred in groupoids associated to the ``presheaf in groupoids''
which associates to a scheme $T$ the groupoid
$$
(X(T), G(T) \times X(T), s, t, c).
$$
Since this ``presheaf in groupoids'' is isomorphic to $FA_d$ by
Lemma \ref{lemma-represent-FAd}
it suffices to prove that the $\mathcal{H}_d$ is the stackification
of (the category fibred in groupoids associated to the
``presheaf in groupoids'') $FA_d$. To do this we first define a
functor
$$
\Spec : FA_d \longrightarrow \mathcal{H}_d
$$
Recall that the fibre category of $\mathcal{H}_d$ over a scheme $T$
is the category of finite locally free morphisms $Z \to T$ of degree $d$.
Thus given a scheme $T$ and a free $d$-dimensional
$\Gamma(T, \mathcal{O}_T)$-algebra $m$ we may assign to this the object
$$
Z = \underline{\Spec}_T(\mathcal{A})
$$
of $\mathcal{H}_{d, T}$
where $\mathcal{A} = \mathcal{O}_T^{\oplus d}$ endowed with a
$\mathcal{O}_T$-algebra structure via $m$. Moreover, if $m'$ is
a second such free $d$-dimensional $\Gamma(T, \mathcal{O}_T)$-algebra
and if $\varphi : m \to m'$ is an isomorphism of these, then
the induced $\mathcal{O}_T$-linear map
$\varphi : \mathcal{O}_T^{\oplus d} \to \mathcal{O}_T^{\oplus d}$
induces an isomorphism
$$
\varphi : \mathcal{A}' \longrightarrow \mathcal{A}
$$
of quasi-coherent $\mathcal{O}_T$-algebras. Hence
$$
\underline{\Spec}_T(\varphi) :
\underline{\Spec}_T(\mathcal{A})
\longrightarrow
\underline{\Spec}_T(\mathcal{A}')
$$
is a morphism in the fibre category $\mathcal{H}_{d, T}$. We omit the
verification that this construction is compatible with base change so
we get indeed a functor $\Spec : FA_d \to \mathcal{H}_d$
as claimed above.

\medskip\noindent
To show that $\Spec : FA_d \to \mathcal{H}_d$ induces an equivalence
between the stackification of $FA_d$ and $\mathcal{H}_d$ it suffices to
check that
\begin{enumerate}
\item $\mathit{Isom}(m, m') = \mathit{Isom}(\Spec(m), \Spec(m'))$
for any $m, m' \in FA_d(T)$.
\item for any scheme $T$ and any object $Z \to T$ of $\mathcal{H}_{d, T}$
there exists a covering $\{T_i \to T\}$ such that $Z|_{T_i}$ is
isomorphic to $\Spec(m)$ for some $m \in FA_d(T_i)$, and
\end{enumerate}
see
Stacks, Lemma \ref{stacks-lemma-stackify-groupoids}.
The first statement follows from the observation that any isomorphism
$$
\underline{\Spec}_T(\mathcal{A})
\longrightarrow
\underline{\Spec}_T(\mathcal{A}')
$$
is necessarily given by a global invertible matrix $g$ when
$\mathcal{A} = \mathcal{A}' = \mathcal{O}_T^{\oplus d}$ as modules.
To prove the second statement let $\pi : Z \to T$ be a finite
locally free morphism of degree $d$. Then $\mathcal{A}$ is a locally
free sheaf $\mathcal{O}_T$-modules of rank $d$.
Consider the element $1 \in \Gamma(T, \mathcal{A})$. This element is
nonzero in $\mathcal{A} \otimes_{\mathcal{O}_{T, t}} \kappa(t)$
for every $t \in T$ since the scheme
$Z_t = \Spec(\mathcal{A} \otimes_{\mathcal{O}_{T, t}} \kappa(t))$
is nonempty being of degree $d > 0$ over $\kappa(t)$. Thus
$1 : \mathcal{O}_T \to \mathcal{A}$ can locally be used as the first
basis element (for example you can use
Algebra, Lemma \ref{algebra-lemma-cokernel-flat} parts (1) and (2)
to see this). Thus, after localizing on
$T$ we may assume that there exists an isomorphism
$\varphi : \mathcal{A} \to \mathcal{O}_T^{\oplus d}$
such that $1 \in \Gamma(\mathcal{A})$ corresponds to the first basis element.
In this situation the multiplication map
$\mathcal{A} \otimes_{\mathcal{O}_T} \mathcal{A} \to \mathcal{A}$
translates via $\varphi$ into a free $d$-dimensional algebra $m$ over
$\Gamma(T, \mathcal{O}_T)$. This finishes the proof.
\end{proof}




\section{Finite Hilbert stacks of spaces}
\label{section-spaces-hilbert}

\noindent
The finite Hilbert stack of an algebraic space is an algebraic stack.

\begin{lemma}
\label{lemma-hilbert-stack-of-space}
Let $S$ be a scheme.
Let $X$ be an algebraic space over $S$.
Then $\mathcal{H}_d(X)$ is an algebraic stack.
\end{lemma}

\begin{proof}
The $1$-morphism
$$
\mathcal{H}_d(X) \longrightarrow \mathcal{H}_d
$$
is representable by algebraic spaces according to
Lemma \ref{lemma-representable-on-top}.
The stack $\mathcal{H}_d$ is an algebraic stack according to
Proposition \ref{proposition-finite-hilbert-point}.
Hence $\mathcal{H}_d(X)$ is an algebraic stack by
Algebraic Stacks,
Lemma \ref{algebraic-lemma-representable-morphism-to-algebraic}.
\end{proof}

\noindent
This lemma allows us to bootstrap.

\begin{lemma}
\label{lemma-hilbert-stack-relative-space}
Let $S$ be a scheme. Let $F : \mathcal{X} \to \mathcal{Y}$ be a $1$-morphism
of stacks in groupoids over $(\Sch/S)_{fppf}$ such that
\begin{enumerate}
\item $\mathcal{X}$ is representable by an algebraic space, and
\item $F$ is representable by algebraic spaces, surjective, flat, and
locally of finite presentation.
\end{enumerate}
Then $\mathcal{H}_d(\mathcal{X}/\mathcal{Y})$ is an algebraic stack.
\end{lemma}

\begin{proof}
Choose a representable stack in groupoids $\mathcal{U}$ over $S$ and a
$1$-morphism $f : \mathcal{U} \to \mathcal{H}_d(\mathcal{X})$
which is representable by algebraic spaces, smooth, and surjective.
This is possible because $\mathcal{H}_d(\mathcal{X})$ is an algebraic stack by
Lemma \ref{lemma-hilbert-stack-of-space}.
Consider the $2$-fibre product
$$
\mathcal{W} =
\mathcal{H}_d(\mathcal{X}/\mathcal{Y})
\times_{\mathcal{H}_d(\mathcal{X}), f}
\mathcal{U}.
$$
Since $\mathcal{U}$ is representable (in particular a stack in setoids)
it follows from
Examples of Stacks, Lemma \ref{examples-stacks-lemma-faithful-hilbert}
and
Stacks, Lemma \ref{stacks-lemma-2-fibre-product-gives-stack-in-setoids}
that $\mathcal{W}$ is a stack in setoids. The $1$-morphism
$\mathcal{W} \to \mathcal{H}_d(\mathcal{X}/\mathcal{Y})$ is
representable by algebraic spaces, smooth, and surjective as a base
change of the morphism $f$ (see
Algebraic Stacks,
Lemmas \ref{algebraic-lemma-base-change-representable-by-spaces} and
\ref{algebraic-lemma-base-change-representable-transformations-property}).
Thus, if we can show that $\mathcal{W}$ is representable by an algebraic space,
then the lemma follows from
Algebraic Stacks,
Lemma \ref{algebraic-lemma-smooth-surjective-morphism-implies-algebraic}.

\medskip\noindent
The diagonal of $\mathcal{Y}$ is representable by algebraic spaces according to
Lemma \ref{lemma-flat-finite-presentation-surjective-diagonal}.
We may apply
Lemma \ref{lemma-relative-hilbert}
to see that the $1$-morphism
$$
\mathcal{H}_d(\mathcal{X}/\mathcal{Y})
\longrightarrow
\mathcal{H}_d(\mathcal{X}) \times \mathcal{Y}
$$
is representable by algebraic spaces. Consider the $2$-fibre product
$$
\mathcal{V} =
\mathcal{H}_d(\mathcal{X}/\mathcal{Y})
\times_{(\mathcal{H}_d(\mathcal{X}) \times \mathcal{Y}), f \times F}
(\mathcal{U} \times \mathcal{X}).
$$
The projection morphism $\mathcal{V} \to \mathcal{U} \times \mathcal{X}$
is representable by algebraic spaces as a base change of the last
displayed morphism. Hence $\mathcal{V}$ is an algebraic space (see
Bootstrap, Lemma \ref{bootstrap-lemma-representable-by-spaces-over-space}
or
Algebraic Stacks,
Lemma \ref{algebraic-lemma-base-change-by-space-representable-by-space}).
The $1$-morphism $\mathcal{V} \to \mathcal{U}$ fits into the following
$2$-cartesian diagram
$$
\xymatrix{
\mathcal{V} \ar[d] \ar[r] & \mathcal{X} \ar[d]^F \\
\mathcal{W} \ar[r] & \mathcal{Y}
}
$$
because
$$
\mathcal{H}_d(\mathcal{X}/\mathcal{Y})
\times_{(\mathcal{H}_d(\mathcal{X}) \times \mathcal{Y}), f \times F}
(\mathcal{U} \times \mathcal{X})
=
(\mathcal{H}_d(\mathcal{X}/\mathcal{Y})
\times_{\mathcal{H}_d(\mathcal{X}), f}
\mathcal{U}) \times_{\mathcal{Y}, F} \mathcal{X}.
$$
Hence $\mathcal{V} \to \mathcal{W}$ is representable by algebraic spaces,
surjective, flat, and locally of finite presentation as a base change
of $F$. It follows that the same thing is true for the corresponding
sheaves of sets associated to $\mathcal{V}$ and $\mathcal{W}$, see
Algebraic Stacks, Lemma \ref{algebraic-lemma-map-fibred-setoids-property}.
Thus we conclude that the sheaf associated to $\mathcal{W}$ is an
algebraic space by
Bootstrap, Theorem \ref{bootstrap-theorem-final-bootstrap}.
\end{proof}




\section{LCI locus in the Hilbert stack}
\label{section-lci}

\noindent
Please consult
Examples of Stacks, Section \ref{examples-stacks-section-hilbert-d-stack}
for notation. Fix a $1$-morphism $F : \mathcal{X} \longrightarrow \mathcal{Y}$
of stacks in groupoids over $(\Sch/S)_{fppf}$. Assume that
$F$ is representable by algebraic spaces. Fix $d \geq 1$. Consider an
object $(U, Z, y, x, \alpha)$ of $\mathcal{H}_d(\mathcal{X}/\mathcal{Y})$.
There is an induced $1$-morphism
$$
(\Sch/Z)_{fppf}
\longrightarrow
(\Sch/U)_{fppf} \times_{y, \mathcal{Y}, F} \mathcal{X}
$$
(by the universal property of $2$-fibre products) which is representable by
a morphism of algebraic spaces over $U$.
Namely, since $F$ is representable by algebraic spaces, we may choose
an algebraic space $X_y$ over $U$ which represents the $2$-fibre product
$(\Sch/U)_{fppf} \times_{y, \mathcal{Y}, F} \mathcal{X}$.
Since $\alpha : y|_Z \to F(x)$ is an isomorphism we see that
$\xi = (Z, Z \to U, x, \alpha)$ is an object of the $2$-fibre product
$(\Sch/U)_{fppf} \times_{y, \mathcal{Y}, F} \mathcal{X}$ over $Z$.
Hence $\xi$ gives rise to a morphism $x_\alpha : Z \to X_y$ of algebraic spaces
over $U$ as $X_y$ is the functor of isomorphisms classes of objects of
$(\Sch/U)_{fppf} \times_{y, \mathcal{Y}, F} \mathcal{X}$, see
Algebraic Stacks,
Lemma \ref{algebraic-lemma-characterize-representable-by-space}.
Here is a picture
\begin{equation}
\label{equation-relative-map}
\vcenter{
\xymatrix{
Z \ar[r]_{x_\alpha} \ar[rd] & X_y \ar[d] \\
& U
}
}
\quad\quad
\vcenter{
\xymatrix{
(\Sch/Z)_{fppf} \ar[rd] \ar[r]_-{x, \alpha} &
(\Sch/U)_{fppf} \times_{y, \mathcal{Y}, F} \mathcal{X} \ar[r] \ar[d] &
\mathcal{X} \ar[d]^F \\
& (\Sch/U)_{fppf} \ar[r]^y & \mathcal{Y}
}
}
\end{equation}
We remark that if
$(f, g, b, a) : (U, Z, y, x, \alpha) \to (U', Z', y', x', \alpha')$
is a morphism between objects of $\mathcal{H}_d$, then the morphism
$x'_{\alpha'} : Z' \to X'_{y'}$ is the base change of the morphism
$x_\alpha$ by the morphism $g : U' \to U$ (details omitted).

\medskip\noindent
Now assume moreover that $F$ is flat and locally of finite presentation.
In this situation we define a full subcategory
$$
\mathcal{H}_{d, lci}(\mathcal{X}/\mathcal{Y}) \subset
\mathcal{H}_d(\mathcal{X}/\mathcal{Y})
$$
consisting of those objects $(U, Z, y, x, \alpha)$ of
$\mathcal{H}_d(\mathcal{X}/\mathcal{Y})$ such
that the corresponding morphism $x_\alpha : Z \to X_y$ is unramified
and a local complete intersection morphism (see
Morphisms of Spaces, Definition \ref{spaces-morphisms-definition-unramified}
and
More on Morphisms of Spaces,
Definition \ref{spaces-more-morphisms-definition-lci}
for definitions).

\begin{lemma}
\label{lemma-lci-locus-stack-in-groupoids}
Let $S$ be a scheme. Fix a $1$-morphism
$F : \mathcal{X} \longrightarrow \mathcal{Y}$
of stacks in groupoids over $(\Sch/S)_{fppf}$.
Assume $F$ is representable by algebraic spaces, flat, and locally
of finite presentation. Then $\mathcal{H}_{d, lci}(\mathcal{X}/\mathcal{Y})$
is a stack in groupoids and the inclusion functor
$$
\mathcal{H}_{d, lci}(\mathcal{X}/\mathcal{Y})
\longrightarrow
\mathcal{H}_d(\mathcal{X}/\mathcal{Y})
$$
is representable and an open immersion.
\end{lemma}

\begin{proof}
Let $\Xi = (U, Z, y, x, \alpha)$ be an object of $\mathcal{H}_d$. It follows
from the remark following
(\ref{equation-relative-map})
that the pullback of $\Xi$ by $U' \to U$ belongs to
$\mathcal{H}_{d, lci}(\mathcal{X}/\mathcal{Y})$ if and only if the base
change of $x_\alpha$ is unramified and a local complete intersection morphism.
Note that $Z \to U$ is finite locally free (hence flat, locally of
finite presentation and universally closed) and that $X_y \to U$ is
flat and locally of finite presentation by our assumption on $F$. Then
More on Morphisms of Spaces, Lemmas
\ref{spaces-more-morphisms-lemma-where-unramified} and
\ref{spaces-more-morphisms-lemma-where-lci}
imply exists an open subscheme $W \subset U$ such that a morphism
$U' \to U$ factors through $W$ if and only if the base change of
$x_\alpha$ via $U' \to U$ is unramified and a local complete intersection
morphism. This implies that
$$
(\Sch/U)_{fppf}
\times_{\Xi, \mathcal{H}_d(\mathcal{X}/\mathcal{Y})}
\mathcal{H}_{d, lci}(\mathcal{X}/\mathcal{Y})
$$
is representable by $W$. Hence the final statement of the lemma
holds. The first statement (that
$\mathcal{H}_{d, lci}(\mathcal{X}/\mathcal{Y})$ is a stack in groupoids)
follows from this and
Algebraic Stacks,
Lemma \ref{algebraic-lemma-open-fibred-category-is-algebraic}.
\end{proof}

\noindent
Local complete intersection morphisms are ``locally unobstructed''.
This holds in much greater generality than the special case
that we need in this chapter here.

\begin{lemma}
\label{lemma-lci-unobstructed}
Let $U \subset U'$ be a first order thickening of affine schemes.
Let $X'$ be an algebraic space flat over $U'$. Set $X = U \times_{U'} X'$.
Let $Z \to U$ be finite locally free of degree $d$. Finally, let
$f : Z \to X$ be unramified and a local complete intersection morphism.
Then there exists a commutative diagram
$$
\xymatrix{
(Z \subset Z') \ar[rd] \ar[rr]_{(f, f')} & & (X \subset X') \ar[ld] \\
& (U \subset U')
}
$$
of algebraic spaces over $U'$ such that $Z' \to U'$ is finite locally free
of degree $d$ and $Z = U \times_{U'} Z'$.
\end{lemma}

\begin{proof}
By
More on Morphisms of Spaces,
Lemma \ref{spaces-more-morphisms-lemma-unramified-lci}
the conormal sheaf $\mathcal{C}_{Z/X}$ of the unramified morphism $Z \to X$
is a finite locally free $\mathcal{O}_Z$-module and by
More on Morphisms of Spaces,
Lemma \ref{spaces-more-morphisms-lemma-transitivity-conormal-lci}
we have an exact sequence
$$
0 \to i^*\mathcal{C}_{X/X'} \to
\mathcal{C}_{Z/X'} \to
\mathcal{C}_{Z/X} \to 0
$$
of conormal sheaves. Since $Z$ is affine this sequence is split. Choose
a splitting
$$
\mathcal{C}_{Z/X'} = i^*\mathcal{C}_{X/X'} \oplus \mathcal{C}_{Z/X}
$$
Let $Z \subset Z''$ be the universal first order thickening of $Z$
over $X'$ (see
More on Morphisms of Spaces,
Section \ref{spaces-more-morphisms-section-universal-thickening}).
Denote $\mathcal{I} \subset \mathcal{O}_{Z''}$ the quasi-coherent sheaf
of ideals corresponding to $Z \subset Z''$. By definition we have
$\mathcal{C}_{Z/X'}$ is $\mathcal{I}$ viewed as a sheaf on $Z$.
Hence the splitting above determines a splitting
$$
\mathcal{I} = i^*\mathcal{C}_{X/X'} \oplus \mathcal{C}_{Z/X}
$$
Let $Z' \subset Z''$ be the closed subscheme cut out by
$\mathcal{C}_{Z/X} \subset \mathcal{I}$ viewed as a quasi-coherent sheaf
of ideals on $Z''$. It is clear that $Z'$ is a first order thickening
of $Z$ and that we obtain a commutative diagram of first order thickenings
as in the statement of the lemma.

\medskip\noindent
Since $X' \to U'$ is flat and since $X = U \times_{U'} X'$ we see that
$\mathcal{C}_{X/X'}$ is the pullback of $\mathcal{C}_{U/U'}$ to $X$, see
More on Morphisms of Spaces, Lemma \ref{spaces-more-morphisms-lemma-deform}.
Note that by construction $\mathcal{C}_{Z/Z'} = i^*\mathcal{C}_{X/X'}$
hence we conclude that $\mathcal{C}_{Z/Z'}$ is isomorphic to the pullback
of $\mathcal{C}_{U/U'}$ to $Z$. Applying
More on Morphisms of Spaces, Lemma \ref{spaces-more-morphisms-lemma-deform}
once again (or its analogue for schemes, see
More on Morphisms, Lemma \ref{more-morphisms-lemma-deform})
we conclude that $Z' \to U'$ is flat and that $Z = U \times_{U'} Z'$.
Finally,
More on Morphisms, Lemma \ref{more-morphisms-lemma-deform-property}
shows that $Z' \to U'$ is finite locally free of degree $d$.
\end{proof}

\begin{lemma}
\label{lemma-lci-formally-smooth}
Let $F : \mathcal{X} \to \mathcal{Y}$ be a $1$-morphism of stacks in groupoids
over $(\Sch/S)_{fppf}$. Assume $F$ is representable by algebraic
spaces, flat, and locally of finite presentation. Then
$$
p : \mathcal{H}_{d, lci}(\mathcal{X}/\mathcal{Y}) \to \mathcal{Y}
$$
is formally smooth on objects.
\end{lemma}

\begin{proof}
We have to show the following: Given
\begin{enumerate}
\item an object $(U, Z, y, x, \alpha)$ of
$\mathcal{H}_{d, lci}(\mathcal{X}/\mathcal{Y})$ over an affine scheme $U$,
\item a first order thickening $U \subset U'$, and
\item an object $y'$ of $\mathcal{Y}$ over $U'$ such that $y'|_U = y$,
\end{enumerate}
then there exists an object $(U', Z', y', x', \alpha')$ of
$\mathcal{H}_{d, lci}(\mathcal{X}/\mathcal{Y})$ over $U'$ with
$Z = U \times_{U'} Z'$, with $x = x'|_Z$, and with
$\alpha = \alpha'|_U$. Namely, the last two equalities will take care
of the commutativity of (\ref{equation-formally-smooth}).

\medskip\noindent
Consider the morphism $x_\alpha : Z \to X_y$ constructed in
Equation (\ref{equation-relative-map}). Denote similarly $X'_{y'}$
the algebraic space over $U'$ representing the $2$-fibre product
$(\Sch/U')_{fppf} \times_{y', \mathcal{Y}, F} \mathcal{X}$.
By assumption the morphism $X'_{y'} \to U'$ is flat (and locally of finite
presentation). As $y'|_U = y$ we see that $X_y = U \times_{U'} X'_{y'}$.
Hence we may apply
Lemma \ref{lemma-lci-unobstructed}
to find $Z' \to U'$ finite locally free of degree $d$ with
$Z = U \times_{U'} Z'$ and with $Z' \to X'_{y'}$ extending $x_\alpha$.
By construction the morphism $Z' \to X'_{y'}$ corresponds to a pair
$(x', \alpha')$. It is clear that $(U', Z', y', x', \alpha')$
is an object of $\mathcal{H}_d(\mathcal{X}/\mathcal{Y})$ over $U'$
with $Z = U \times_{U'} Z'$, with $x = x'|_Z$, and with
$\alpha = \alpha'|_U$. As we've seen in
Lemma \ref{lemma-lci-locus-stack-in-groupoids}
that $\mathcal{H}_{d, lci}(\mathcal{X}/\mathcal{Y}) \subset
\mathcal{H}_d(\mathcal{X}/\mathcal{Y})$ is an ``open substack''
it follows that $(U', Z', y', x', \alpha')$ is an object of
$\mathcal{H}_{d, lci}(\mathcal{X}/\mathcal{Y})$ as desired.
\end{proof}

\begin{lemma}
\label{lemma-lci-surjective}
Let $F : \mathcal{X} \to \mathcal{Y}$ be a $1$-morphism of stacks in groupoids
over $(\Sch/S)_{fppf}$. Assume $F$ is representable by algebraic
spaces, flat, surjective, and locally of finite presentation. Then
$$
\coprod\nolimits_{d \geq 1} \mathcal{H}_{d, lci}(\mathcal{X}/\mathcal{Y})
\longrightarrow
\mathcal{Y}
$$
is surjective on objects.
\end{lemma}

\begin{proof}
It suffices to prove the following: For any field $k$
and object $y$ of $\mathcal{Y}$ over $\Spec(k)$ there exists
an integer $d \geq 1$ and an object $(U, Z, y, x, \alpha)$ of
$\mathcal{H}_{d, lci}(\mathcal{X}/\mathcal{Y})$ with $U = \Spec(k)$.
Namely, in this case we see that $p$ is surjective on objects in the
strong sense that an extension of the field is not needed.

\medskip\noindent
Denote $X_y$ the algebraic space over $U = \Spec(k)$
representing the $2$-fibre product
$(\Sch/U')_{fppf} \times_{y', \mathcal{Y}, F} \mathcal{X}$.
By assumption the morphism $X_y \to \Spec(k)$ is surjective and
locally of finite presentation (and flat). In particular $X_y$ is
nonempty. Choose a nonempty affine scheme $V$ and an \'etale morphism
$V \to X_y$. Note that $V \to \Spec(k)$ is (flat), surjective,
and locally of finite presentation (by
Morphisms of Spaces,
Definition \ref{spaces-morphisms-definition-locally-finite-presentation}).
Pick a closed point $v \in V$ where $V \to \Spec(k)$ is Cohen-Macaulay
(i.e., $V$ is Cohen-Macaulay at $v$), see
More on Morphisms,
Lemma \ref{more-morphisms-lemma-flat-finite-presentation-CM-open}.
Applying
More on Morphisms,
Lemma \ref{more-morphisms-lemma-slice-CM}
we find a regular immersion $Z \to V$ with $Z = \{v\}$.
This implies $Z \to V$ is a closed immersion. Moreover, it follows that
$Z \to \Spec(k)$ is finite (for example by
Algebra, Lemma \ref{algebra-lemma-isolated-point}).
Hence $Z \to \Spec(k)$ is finite locally free of some degree $d$.
Now $Z \to X_y$ is unramified as the composition
of a closed immersion followed by an \'etale morphism
(see
Morphisms of Spaces, Lemmas \ref{spaces-morphisms-lemma-composition-unramified},
\ref{spaces-morphisms-lemma-etale-unramified}, and
\ref{spaces-morphisms-lemma-immersion-unramified}).
Finally, $Z \to X_y$ is a local complete intersection morphism
as a composition of a regular immersion of schemes and an \'etale
morphism of algebraic spaces (see
More on Morphisms, Lemma \ref{more-morphisms-lemma-regular-immersion-lci}
and
Morphisms of Spaces, Lemmas \ref{spaces-morphisms-lemma-etale-smooth} and
\ref{spaces-morphisms-lemma-smooth-syntomic} and
More on Morphisms of Spaces,
Lemmas \ref{spaces-more-morphisms-lemma-flat-lci} and
\ref{spaces-more-morphisms-lemma-composition-lci}).
The morphism $Z \to X_y$ corresponds to an object $x$ of $\mathcal{X}$
over $Z$ together with an isomorphism $\alpha : y|_Z \to F(x)$.
We obtain an object $(U, Z, y, x, \alpha)$ of
$\mathcal{H}_d(\mathcal{X}/\mathcal{Y})$. By what was said above about
the morphism $Z \to X_y$ we see that it actually is an object of the
subcategory $\mathcal{H}_{d, lci}(\mathcal{X}/\mathcal{Y})$ and we win.
\end{proof}




















\section{Bootstrapping algebraic stacks}
\label{section-bootstrap}

\noindent
The following theorem is one of the main results of this chapter.

\begin{theorem}
\label{theorem-bootstrap}
Let $S$ be a scheme. Let $F : \mathcal{X} \to \mathcal{Y}$
be a $1$-morphism of stacks in groupoids over $(\Sch/S)_{fppf}$. If
\begin{enumerate}
\item $\mathcal{X}$ is representable by an algebraic space, and
\item $F$ is representable by algebraic spaces, surjective, flat and
locally of finite presentation,
\end{enumerate}
then $\mathcal{Y}$ is an algebraic stack.
\end{theorem}

\begin{proof}
By
Lemma \ref{lemma-flat-finite-presentation-surjective-diagonal}
we see that the diagonal of $\mathcal{Y}$ is representable by algebraic
spaces. Hence we only need to verify the existence of a $1$-morphism
$f : \mathcal{V} \to \mathcal{Y}$ of stacks in groupoids over
$(\Sch/S)_{fppf}$ with $\mathcal{V}$ representable and
$f$ surjective and smooth. By
Lemma \ref{lemma-hilbert-stack-relative-space}
we know that
$$
\coprod\nolimits_{d \geq 1} \mathcal{H}_d(\mathcal{X}/\mathcal{Y})
$$
is an algebraic stack. It follows from
Lemma \ref{lemma-lci-locus-stack-in-groupoids}
and
Algebraic Stacks,
Lemma \ref{algebraic-lemma-open-fibred-category-is-algebraic}
that
$$
\coprod\nolimits_{d \geq 1} \mathcal{H}_{d, lci}(\mathcal{X}/\mathcal{Y})
$$
is an algebraic stack as well. Choose a representable stack in groupoids
$\mathcal{V}$ over $(\Sch/S)_{fppf}$ and a surjective and smooth
$1$-morphism
$$
\mathcal{V}
\longrightarrow
\coprod\nolimits_{d \geq 1} \mathcal{H}_{d, lci}(\mathcal{X}/\mathcal{Y}).
$$
We claim that the composition
$$
\mathcal{V}
\longrightarrow
\coprod\nolimits_{d \geq 1} \mathcal{H}_{d, lci}(\mathcal{X}/\mathcal{Y})
\longrightarrow
\mathcal{Y}
$$
is smooth and surjective which finishes the proof of the theorem. In fact,
the smoothness will be a consequence of
Lemmas \ref{lemma-limit-preserving} and \ref{lemma-lci-formally-smooth}
and the surjectivity a consequence of
Lemma \ref{lemma-lci-surjective}.
We spell out the details in the following paragraph.

\medskip\noindent
By construction $\mathcal{V} \to
\coprod\nolimits_{d \geq 1} \mathcal{H}_{d, lci}(\mathcal{X}/\mathcal{Y})$
is representable by algebraic spaces, surjective, and smooth (and hence
also locally of finite presentation and formally smooth by the general
principle
Algebraic Stacks, Lemma
\ref{algebraic-lemma-representable-transformations-property-implication}
and
More on Morphisms of Spaces,
Lemma \ref{spaces-more-morphisms-lemma-smooth-formally-smooth}).
Applying
Lemmas \ref{lemma-representable-by-spaces-limit-preserving},
\ref{lemma-representable-by-spaces-formally-smooth}, and
\ref{lemma-representable-by-spaces-surjective}
we see that $\mathcal{V} \to
\coprod\nolimits_{d \geq 1} \mathcal{H}_{d, lci}(\mathcal{X}/\mathcal{Y})$
is limit preserving on objects, formally smooth on objects, and
surjective on objects. The $1$-morphism
$\coprod\nolimits_{d \geq 1} \mathcal{H}_{d, lci}(\mathcal{X}/\mathcal{Y})
\to \mathcal{Y}$ is
\begin{enumerate}
\item limit preserving on objects: this is
Lemma \ref{lemma-limit-preserving}
for $\mathcal{H}_d(\mathcal{X}/\mathcal{Y}) \to \mathcal{Y}$
and we combine it with Lemmas
\ref{lemma-lci-locus-stack-in-groupoids},
\ref{lemma-open-immersion-limit-preserving}, and
\ref{lemma-composition-limit-preserving}
to get it for $\mathcal{H}_{d, lci}(\mathcal{X}/\mathcal{Y}) \to \mathcal{Y}$,
\item formally smooth on objects by
Lemma \ref{lemma-lci-formally-smooth},
and
\item surjective on objects by
Lemma \ref{lemma-lci-surjective}.
\end{enumerate}
Using
Lemmas \ref{lemma-composition-limit-preserving},
\ref{lemma-composition-formally-smooth}, and
\ref{lemma-composition-surjective}
we conclude that the composition $\mathcal{V} \to \mathcal{Y}$ is
limit preserving on objects, formally smooth on objects, and
surjective on objects.
Using
Lemmas \ref{lemma-representable-by-spaces-limit-preserving},
\ref{lemma-representable-by-spaces-formally-smooth}, and
\ref{lemma-representable-by-spaces-surjective}
we see that $\mathcal{V} \to \mathcal{Y}$ is
locally of finite presentation, formally smooth, and surjective.
Finally, using (via the general principle
Algebraic Stacks,
Lemma \ref{algebraic-lemma-representable-transformations-property-implication})
the infinitesimal lifting criterion
(More on Morphisms of Spaces, Lemma
\ref{spaces-more-morphisms-lemma-smooth-formally-smooth})
we see that $\mathcal{V} \to \mathcal{Y}$ is smooth and we win.
\end{proof}








\section{Applications}
\label{section-applications}

\noindent
Our first task is to show that the quotient stack $[U/R]$ associated to
a ``flat and locally finitely presented groupoid'' is an algebraic stack.
See
Groupoids in Spaces,
Definition \ref{spaces-groupoids-definition-quotient-stack}
for the definition of the quotient stack.
The following lemma is preliminary and is the analogue of
Algebraic Stacks,
Lemma \ref{algebraic-lemma-smooth-quotient-smooth-presentation}.

\begin{lemma}
\label{lemma-flat-quotient-flat-presentation}
Let $S$ be a scheme contained in $\Sch_{fppf}$.
Let $(U, R, s, t, c)$ be a groupoid in algebraic spaces over $S$.
Assume $s, t$ are flat and locally of finite presentation.
Then the morphism $\mathcal{S}_U \to [U/R]$ is flat, locally of
finite presentation, and surjective.
\end{lemma}

\begin{proof}
Let $T$ be a scheme and let $x : (\Sch/T)_{fppf} \to [U/R]$
be a $1$-morphism. We have to show that the projection
$$
\mathcal{S}_U \times_{[U/R]} (\Sch/T)_{fppf}
\longrightarrow
(\Sch/T)_{fppf}
$$
is surjective, flat, and locally of finite presentation.
We already know that the left hand side
is representable by an algebraic space $F$, see
Algebraic Stacks, Lemmas \ref{algebraic-lemma-diagonal-quotient-stack} and
\ref{algebraic-lemma-representable-diagonal}.
Hence we have to show the corresponding morphism $F \to T$ of
algebraic spaces is surjective, locally of finite presentation, and flat.
Since we are working with properties of morphisms of algebraic
spaces which are local on the target in the fppf topology we
may check this fppf locally on $T$. By construction, there exists
an fppf covering $\{T_i \to T\}$ of $T$ such that
$x|_{(\Sch/T_i)_{fppf}}$ comes from a morphism
$x_i : T_i \to U$. (Note that $F \times_T T_i$ represents the
$2$-fibre product $\mathcal{S}_U \times_{[U/R]} (\Sch/T_i)_{fppf}$
so everything is compatible with the base change via $T_i \to T$.)
Hence we may assume that $x$ comes from $x : T \to U$.
In this case we see that
$$
\mathcal{S}_U \times_{[U/R]} (\Sch/T)_{fppf}
=
(\mathcal{S}_U \times_{[U/R]} \mathcal{S}_U)
\times_{\mathcal{S}_U} (\Sch/T)_{fppf}
=
\mathcal{S}_R \times_{\mathcal{S}_U} (\Sch/T)_{fppf}
$$
The first equality by
Categories, Lemma \ref{categories-lemma-2-fibre-product-erase-factor}
and the second equality by
Groupoids in Spaces,
Lemma \ref{spaces-groupoids-lemma-quotient-stack-2-cartesian}.
Clearly the last $2$-fibre product is represented by the algebraic
space $F = R \times_{s, U, x} T$ and the projection
$R \times_{s, U, x} T \to T$ is flat and locally of finite presentation
as the base change of the flat locally finitely presented
morphism of algebraic spaces $s : R \to U$.
It is also surjective as $s$ has a section (namely the identity
$e : U \to R$ of the groupoid).
This proves the lemma.
\end{proof}

\noindent
Here is the first main result of this section.

\begin{theorem}
\label{theorem-flat-groupoid-gives-algebraic-stack}
Let $S$ be a scheme contained in $\Sch_{fppf}$.
Let $(U, R, s, t, c)$ be a groupoid in algebraic spaces over $S$.
Assume $s, t$ are flat and locally of finite presentation.
Then the quotient stack $[U/R]$ is an algebraic stack over $S$.
\end{theorem}

\begin{proof}
We check the two conditions of
Theorem \ref{theorem-bootstrap}
for the morphism
$$
(\Sch/U)_{fppf} \longrightarrow [U/R].
$$
The first is trivial (as $U$ is an algebraic space).
The second is
Lemma \ref{lemma-flat-quotient-flat-presentation}.
\end{proof}









\section{When is a quotient stack algebraic?}
\label{section-quotient-algebraic}

\noindent
In
Groupoids in Spaces, Section \ref{spaces-groupoids-section-stacks}
we have defined the quotient stack $[U/R]$ associated to a groupoid
$(U, R, s, t, c)$ in algebraic spaces. Note that $[U/R]$ is a stack
in groupoids whose diagonal is representable by algebraic spaces (see
Bootstrap, Lemma \ref{bootstrap-lemma-quotient-stack-isom}
and
Algebraic Stacks, Lemma \ref{algebraic-lemma-representable-diagonal})
and such that there exists an algebraic space $U$ and a $1$-morphism
$(\Sch/U)_{fppf} \to [U/R]$ which is an ``fppf surjection''
in the sense that it induces a map on presheaves of isomorphism classes of
objects which becomes surjective after sheafification.
However, it is not the case that $[U/R]$ is an algebraic
stack in general. This is not a contradiction with
Theorem \ref{theorem-bootstrap}
as the $1$-morphism $(\Sch/U)_{fppf} \to [U/R]$ may not
be flat and locally of finite presentation.

\medskip\noindent
The easiest way to make examples of non-algebraic quotient stacks is
to look at quotients of the form $[S/G]$ where $S$ is a scheme and $G$
is a group scheme over $S$ acting trivially on $S$. Namely, we will see
below
(Lemma \ref{lemma-BG-algebraic})
that if $[S/G]$ is algebraic, then $G \to S$ has to be flat and locally
of finite presentation. An explicit example can be found in
Examples, Section \ref{examples-section-not-algebraic-stack}.

\begin{lemma}
\label{lemma-quotient-algebraic}
Let $S$ be a scheme and let $B$ be an algebraic space over $S$.
Let $(U, R, s, t, c)$ be a groupoid in algebraic spaces over $B$.
The quotient stack $[U/R]$ is an algebraic stack if and only if
there exists a morphism of algebraic spaces $g : U' \to U$ such that
\begin{enumerate}
\item the composition
$U' \times_{g, U, t} R \to R \xrightarrow{s} U$ is a surjection of
sheaves, and
\item the morphisms $s', t' : R' \to U'$ are flat and locally of finite
presentation where $(U', R', s', t', c')$ is the restriction of
$(U, R, s, t, c)$ via $g$.
\end{enumerate}
\end{lemma}

\begin{proof}
First, assume that $g : U' \to U$ satisfies (1) and (2). Property (1)
implies that $[U'/R'] \to [U/R]$ is an equivalence, see
Groupoids in Spaces,
Lemma \ref{spaces-groupoids-lemma-quotient-stack-restrict-equivalence}.
By
Theorem \ref{theorem-flat-groupoid-gives-algebraic-stack}
the quotient stack $[U'/R']$ is an algebraic stack. Hence
$[U/R]$ is an algebraic stack too, see
Algebraic Stacks, Lemma \ref{algebraic-lemma-equivalent}.

\medskip\noindent
Conversely, assume that $[U/R]$ is an algebraic stack. We may choose a
scheme $W$ and a surjective smooth $1$-morphism
$$
f : (\Sch/W)_{fppf} \longrightarrow [U/R].
$$
By the $2$-Yoneda lemma
(Algebraic Stacks, Section \ref{algebraic-section-2-yoneda})
this corresponds to an object $\xi$ of $[U/R]$ over $W$.
By the description of $[U/R]$ in
Groupoids in Spaces, Lemma \ref{spaces-groupoids-lemma-quotient-stack-objects}
we can find a surjective, flat, locally finitely presented morphism
$b : U' \to W$ of schemes such that $\xi' = b^*\xi$ corresponds to a morphism
$g : U' \to U$. Note that the $1$-morphism
$$
f' : (\Sch/U')_{fppf} \longrightarrow [U/R].
$$
corresponding to $\xi'$ is surjective, flat, and locally of finite
presentation, see
Algebraic Stacks, Lemma
\ref{algebraic-lemma-composition-representable-transformations-property}.
Hence
$(\Sch/U')_{fppf} \times_{[U/R]} (\Sch/U')_{fppf}$
which is represented by the algebraic space
$$
\mathit{Isom}_{[U/R]}(\text{pr}_0^*\xi', \text{pr}_1^*\xi') =
(U' \times_S U')
\times_{(g \circ \text{pr}_0, g \circ \text{pr}_1), U \times_S U} R = R'
$$
(see
Groupoids in Spaces, Lemma
\ref{spaces-groupoids-lemma-quotient-stack-morphisms}
for the first equality; the second is the definition of restriction)
is flat and locally of finite presentation over $U'$ via both $s'$ and $t'$
(by base change, see
Algebraic Stacks, Lemma
\ref{algebraic-lemma-base-change-representable-transformations-property}).
By this description of $R'$ and by
Algebraic Stacks, Lemma \ref{algebraic-lemma-map-space-into-stack}
we obtain a canonical fully faithful $1$-morphism $[U'/R'] \to [U/R]$.
This $1$-morphism is essentially surjective because $f'$ is flat,
locally of finite presentation, and surjective (see
Stacks, Lemma \ref{stacks-lemma-characterize-essentially-surjective-when-ff});
another way to prove this is to use
Algebraic Stacks, Remark
\ref{algebraic-remark-flat-fp-presentation}.
Finally, we can use
Groupoids in Spaces, Lemma
\ref{spaces-groupoids-lemma-quotient-stack-restrict-equivalence}
to conclude that the composition
$U' \times_{g, U, t} R \to R \xrightarrow{s} U$ is a surjection of sheaves.
\end{proof}

\begin{lemma}
\label{lemma-group-quotient-algebraic}
Let $S$ be a scheme and let $B$ be an algebraic space over $S$.
Let $G$ be a group algebraic space over $B$.
Let $X$ be an algebraic space over $B$ and let $a : G \times_B X \to X$
be an action of $G$ on $X$ over $B$.
The quotient stack $[X/G]$ is an algebraic stack if and only if
there exists a morphism of algebraic spaces $\varphi : X' \to X$ such that
\begin{enumerate}
\item $G \times_B X' \to X$, $(g, x') \mapsto a(g, \varphi(x'))$ is a
surjection of sheaves, and
\item the two projections $X'' \to X'$ of the algebraic space $X''$
given by the rule
$$
T \longmapsto \{(x'_1, g, x'_2) \in (X' \times_B G \times_B X')(T)
\mid \varphi(x'_1) = a(g, \varphi(x'_2))\}
$$
are flat and locally of finite presentation.
\end{enumerate}
\end{lemma}

\begin{proof}
This lemma is a special case of
Lemma \ref{lemma-quotient-algebraic}.
Namely, the quotient stack $[X/G]$ is by
Groupoids in Spaces, Definition \ref{spaces-groupoids-definition-quotient-stack}
equal to the quotient stack $[X/G \times_B X]$ of the groupoid in
algebraic spaces $(X, G \times_B X, s, t, c)$ associated to
the group action in
Groupoids in Spaces, Lemma \ref{spaces-groupoids-lemma-groupoid-from-action}.
There is one small observation that is needed to get condition (1).
Namely, the morphism $s : G \times_B X \to X$ is the second projection
and the morphism $t :  G \times_B X \to X$ is the action morphism $a$.
Hence the morphism $h : U' \times_{g, U, t} R \to R \xrightarrow{s} U$ from
Lemma \ref{lemma-quotient-algebraic}
corresponds to the morphism
$$
X' \times_{\varphi, X, a} (G \times_B X) \xrightarrow{\text{pr}_1} X
$$
in the current setting. However, because of the symmetry given by
the inverse of $G$ this morphism is isomorphic to the morphism
$$
(G \times_B X) \times_{\text{pr}_1, X, \varphi} X' \xrightarrow{a} X
$$
of the statement of the lemma. Details omitted.
\end{proof}

\begin{lemma}
\label{lemma-BG-algebraic}
\begin{slogan}
Gerbes are algebraic if and only if the associated groups are flat
and locally of finite presentation
\end{slogan}
Let $S$ be a scheme and let $B$ be an algebraic space over $S$.
Let $G$ be a group algebraic space over $B$.
Endow $B$ with the trivial action of $G$.
Then the quotient stack $[B/G]$ is an algebraic stack
if and only if $G$ is flat and locally of finite presentation over $B$.
\end{lemma}

\begin{proof}
If $G$ is flat and locally of finite presentation over $B$, then
$[B/G]$ is an algebraic stack by
Theorem \ref{theorem-flat-groupoid-gives-algebraic-stack}.

\medskip\noindent
Conversely, assume that $[B/G]$ is an algebraic stack. By
Lemma \ref{lemma-group-quotient-algebraic}
and because the action is trivial, we see
there exists an algebraic space $B'$ and a morphism
$B' \to B$ such that (1) $B' \to B$ is a surjection
of sheaves and (2) the projections
$$
B' \times_B G \times_B B' \to B'
$$
are flat and locally of finite presentation. Note that the base change
$B' \times_B G \times_B B' \to G \times_B B'$ of $B' \to B$
is a surjection of sheaves also. Thus it follows from
Descent on Spaces, Lemma \ref{spaces-descent-lemma-curiosity}
that the projection $G \times_B B' \to B'$ is flat and locally
of finite presentation. By (1) we can find an fppf covering
$\{B_i \to B\}$ such that $B_i \to B$ factors through $B' \to B$.
Hence $G \times_B B_i \to B_i$ is flat and locally of finite presentation
by base change. By
Descent on Spaces, Lemmas
\ref{spaces-descent-lemma-descending-property-flat} and
\ref{spaces-descent-lemma-descending-property-locally-finite-presentation}
we conclude that $G \to B$ is flat and locally of finite presentation.
\end{proof}

\noindent
Later we will see that the quotient stack of a smooth $S$-space
by a group algebraic space $G$ is smooth, even when $G$ is not smooth
(Morphisms of Stacks, Lemma
\ref{stacks-morphisms-lemma-smooth-quotient-stack}).




\section{Algebraic stacks in the \'etale topology}
\label{section-stacks-etale}

\noindent
Let $S$ be a scheme. Instead of working with stacks in groupoids over
the big fppf site $(\Sch/S)_{fppf}$ we could work with stacks in groupoids
over the big \'etale site $(\Sch/S)_\etale$. All of the material in
Algebraic Stacks, Sections
\ref{algebraic-section-representable},
\ref{algebraic-section-2-yoneda},
\ref{algebraic-section-representable-morphism},
\ref{algebraic-section-split},
\ref{algebraic-section-representable-by-algebraic-spaces},
\ref{algebraic-section-morphisms-representable-by-algebraic-spaces},
\ref{algebraic-section-representable-properties}, and
\ref{algebraic-section-stacks}
makes sense for categories fibred in groupoids over $(\Sch/S)_\etale$.
Thus we get a second notion of an algebraic stack by working in the
\'etale topology. This notion is (a priori) weaker than the notion introduced
in Algebraic Stacks, Definition \ref{algebraic-definition-algebraic-stack}
since a stack in the fppf topology is certainly a stack in the \'etale
topology. However, the notions are equivalent as is shown by the following
lemma.

\begin{lemma}
\label{lemma-stacks-etale}
Denote the common underlying category of $\Sch_{fppf}$
and $\Sch_\etale$ by $\Sch_\alpha$ (see
Sheaves on Stacks, Section \ref{stacks-sheaves-section-sheaves} and
Topologies, Remark \ref{topologies-remark-choice-sites}). Let $S$ be an object
of $\Sch_\alpha$. Let
$$
p : \mathcal{X} \to \Sch_\alpha/S
$$
be a category fibred in groupoids with the following properties:
\begin{enumerate}
\item $\mathcal{X}$ is a stack in groupoids over $(\Sch/S)_\etale$,
\item the diagonal $\Delta : \mathcal{X} \to \mathcal{X} \times \mathcal{X}$
is representable by algebraic spaces\footnote{Here we can either mean
sheaves in the \'etale topology whose diagonal is representable and which
have an \'etale surjective covering by a scheme or algebraic spaces as
defined in
Algebraic Spaces, Definition \ref{spaces-definition-algebraic-space}.
Namely, by Bootstrap, Lemma \ref{bootstrap-lemma-spaces-etale}
there is no difference.}, and
\item there exists $U \in \Ob(\Sch_\alpha/S)$
and a $1$-morphism $(\Sch/U)_\etale \to \mathcal{X}$
which is surjective and smooth.
\end{enumerate}
Then $\mathcal{X}$ is an algebraic stack in the sense of
Algebraic Stacks, Definition \ref{algebraic-definition-algebraic-stack}.
\end{lemma}

\begin{proof}
Note that properties (2) and (3) of the lemma and the corresponding
properties (2) and (3) of
Algebraic Stacks, Definition \ref{algebraic-definition-algebraic-stack}
are independent of the topology. This is true because these properties
involve only the notion of a $2$-fibre product of categories fibred in
groupoids, $1$- and $2$-morphisms of categories fibred in groupoids, the
notion of a $1$-morphism of categories fibred in groupoids representable
by algebraic spaces, and what it means for such a $1$-morphism to be
surjective and smooth.
Thus all we have to prove is that an \'etale stack in groupoids
$\mathcal{X}$ with properties (2) and (3) is also an fppf stack in groupoids.

\medskip\noindent
Using (2) let $R$ be an algebraic space representing
$$
(\Sch_\alpha/U) \times_\mathcal{X} (\Sch_\alpha/U)
$$
By (3) the projections $s, t : R \to U$ are smooth. Exactly as in the proof of
Algebraic Stacks, Lemma \ref{algebraic-lemma-map-space-into-stack}
there exists a groupoid in spaces $(U, R, s, t, c)$ and a canonical
fully faithful $1$-morphism $[U/R]_\etale \to \mathcal{X}$
where $[U/R]_\etale$ is the \'etale stackification of presheaf
in groupoids
$$
T \longmapsto (U(T), R(T), s(T), t(T), c(T))
$$
Claim: If $V \to T$ is a surjective smooth morphism from an algebraic space
$V$ to a scheme $T$, then there exists an \'etale covering $\{T_i \to T\}$
refining the covering $\{V \to T\}$. This follows from
More on Morphisms, Lemma \ref{more-morphisms-lemma-etale-dominates-smooth}
or the more general
Sheaves on Stacks, Lemma
\ref{stacks-sheaves-lemma-surjective-flat-locally-finite-presentation}.
Using the claim and arguing exactly as in
Algebraic Stacks, Lemma \ref{algebraic-lemma-stack-presentation}
it follows that $[U/R]_\etale \to \mathcal{X}$ is an
equivalence.

\medskip\noindent
Next, let $[U/R]$ denote the quotient stack in the fppf topology
which is an algebraic stack by
Algebraic Stacks, Theorem
\ref{algebraic-theorem-smooth-groupoid-gives-algebraic-stack}.
Thus we have $1$-morphisms
$$
U \to [U/R]_\etale \to [U/R].
$$
Both $U \to [U/R]_\etale \cong \mathcal{X}$ and
$U \to [U/R]$ are surjective and smooth (the first by assumption
and the second by the theorem) and in both cases the
fibre product $U \times_\mathcal{X} U$ and $U \times_{[U/R]} U$
is representable by $R$. Hence the $1$-morphism
$[U/R]_\etale \to [U/R]$ is fully faithful (since morphisms
in the quotient stacks are given by morphisms into $R$, see
Groupoids in Spaces, Section
\ref{spaces-groupoids-section-explicit-quotient-stacks}).

\medskip\noindent
Finally, for any scheme $T$ and morphism $t : T \to [U/R]$ the fibre product
$V = T \times_{U/R} U$ is an algebraic space surjective and smooth over $T$.
By the claim above there exists an \'etale covering $\{T_i \to T\}_{i \in I}$
and morphisms $T_i \to V$ over $T$. This proves that the object
$t$ of $[U/R]$ over $T$ comes \'etale locally from $U$. We conclude that
$[U/R]_\etale \to [U/R]$ is an equivalence of stacks in
groupoids over $(\Sch/S)_\etale$ by
Stacks, Lemma \ref{stacks-lemma-characterize-essentially-surjective-when-ff}.
This concludes the proof.
\end{proof}










\begin{multicols}{2}[\section{Other chapters}]
\noindent
Preliminaries
\begin{enumerate}
\item \hyperref[introduction-section-phantom]{Introduction}
\item \hyperref[conventions-section-phantom]{Conventions}
\item \hyperref[sets-section-phantom]{Set Theory}
\item \hyperref[categories-section-phantom]{Categories}
\item \hyperref[topology-section-phantom]{Topology}
\item \hyperref[sheaves-section-phantom]{Sheaves on Spaces}
\item \hyperref[sites-section-phantom]{Sites and Sheaves}
\item \hyperref[stacks-section-phantom]{Stacks}
\item \hyperref[fields-section-phantom]{Fields}
\item \hyperref[algebra-section-phantom]{Commutative Algebra}
\item \hyperref[brauer-section-phantom]{Brauer Groups}
\item \hyperref[homology-section-phantom]{Homological Algebra}
\item \hyperref[derived-section-phantom]{Derived Categories}
\item \hyperref[simplicial-section-phantom]{Simplicial Methods}
\item \hyperref[more-algebra-section-phantom]{More on Algebra}
\item \hyperref[smoothing-section-phantom]{Smoothing Ring Maps}
\item \hyperref[modules-section-phantom]{Sheaves of Modules}
\item \hyperref[sites-modules-section-phantom]{Modules on Sites}
\item \hyperref[injectives-section-phantom]{Injectives}
\item \hyperref[cohomology-section-phantom]{Cohomology of Sheaves}
\item \hyperref[sites-cohomology-section-phantom]{Cohomology on Sites}
\item \hyperref[dga-section-phantom]{Differential Graded Algebra}
\item \hyperref[dpa-section-phantom]{Divided Power Algebra}
\item \hyperref[sdga-section-phantom]{Differential Graded Sheaves}
\item \hyperref[hypercovering-section-phantom]{Hypercoverings}
\end{enumerate}
Schemes
\begin{enumerate}
\setcounter{enumi}{25}
\item \hyperref[schemes-section-phantom]{Schemes}
\item \hyperref[constructions-section-phantom]{Constructions of Schemes}
\item \hyperref[properties-section-phantom]{Properties of Schemes}
\item \hyperref[morphisms-section-phantom]{Morphisms of Schemes}
\item \hyperref[coherent-section-phantom]{Cohomology of Schemes}
\item \hyperref[divisors-section-phantom]{Divisors}
\item \hyperref[limits-section-phantom]{Limits of Schemes}
\item \hyperref[varieties-section-phantom]{Varieties}
\item \hyperref[topologies-section-phantom]{Topologies on Schemes}
\item \hyperref[descent-section-phantom]{Descent}
\item \hyperref[perfect-section-phantom]{Derived Categories of Schemes}
\item \hyperref[more-morphisms-section-phantom]{More on Morphisms}
\item \hyperref[flat-section-phantom]{More on Flatness}
\item \hyperref[groupoids-section-phantom]{Groupoid Schemes}
\item \hyperref[more-groupoids-section-phantom]{More on Groupoid Schemes}
\item \hyperref[etale-section-phantom]{\'Etale Morphisms of Schemes}
\end{enumerate}
Topics in Scheme Theory
\begin{enumerate}
\setcounter{enumi}{41}
\item \hyperref[chow-section-phantom]{Chow Homology}
\item \hyperref[intersection-section-phantom]{Intersection Theory}
\item \hyperref[pic-section-phantom]{Picard Schemes of Curves}
\item \hyperref[weil-section-phantom]{Weil Cohomology Theories}
\item \hyperref[adequate-section-phantom]{Adequate Modules}
\item \hyperref[dualizing-section-phantom]{Dualizing Complexes}
\item \hyperref[duality-section-phantom]{Duality for Schemes}
\item \hyperref[discriminant-section-phantom]{Discriminants and Differents}
\item \hyperref[derham-section-phantom]{de Rham Cohomology}
\item \hyperref[local-cohomology-section-phantom]{Local Cohomology}
\item \hyperref[algebraization-section-phantom]{Algebraic and Formal Geometry}
\item \hyperref[curves-section-phantom]{Algebraic Curves}
\item \hyperref[resolve-section-phantom]{Resolution of Surfaces}
\item \hyperref[models-section-phantom]{Semistable Reduction}
\item \hyperref[functors-section-phantom]{Functors and Morphisms}
\item \hyperref[equiv-section-phantom]{Derived Categories of Varieties}
\item \hyperref[pione-section-phantom]{Fundamental Groups of Schemes}
\item \hyperref[etale-cohomology-section-phantom]{\'Etale Cohomology}
\item \hyperref[crystalline-section-phantom]{Crystalline Cohomology}
\item \hyperref[proetale-section-phantom]{Pro-\'etale Cohomology}
\item \hyperref[relative-cycles-section-phantom]{Relative Cycles}
\item \hyperref[more-etale-section-phantom]{More \'Etale Cohomology}
\item \hyperref[trace-section-phantom]{The Trace Formula}
\end{enumerate}
Algebraic Spaces
\begin{enumerate}
\setcounter{enumi}{64}
\item \hyperref[spaces-section-phantom]{Algebraic Spaces}
\item \hyperref[spaces-properties-section-phantom]{Properties of Algebraic Spaces}
\item \hyperref[spaces-morphisms-section-phantom]{Morphisms of Algebraic Spaces}
\item \hyperref[decent-spaces-section-phantom]{Decent Algebraic Spaces}
\item \hyperref[spaces-cohomology-section-phantom]{Cohomology of Algebraic Spaces}
\item \hyperref[spaces-limits-section-phantom]{Limits of Algebraic Spaces}
\item \hyperref[spaces-divisors-section-phantom]{Divisors on Algebraic Spaces}
\item \hyperref[spaces-over-fields-section-phantom]{Algebraic Spaces over Fields}
\item \hyperref[spaces-topologies-section-phantom]{Topologies on Algebraic Spaces}
\item \hyperref[spaces-descent-section-phantom]{Descent and Algebraic Spaces}
\item \hyperref[spaces-perfect-section-phantom]{Derived Categories of Spaces}
\item \hyperref[spaces-more-morphisms-section-phantom]{More on Morphisms of Spaces}
\item \hyperref[spaces-flat-section-phantom]{Flatness on Algebraic Spaces}
\item \hyperref[spaces-groupoids-section-phantom]{Groupoids in Algebraic Spaces}
\item \hyperref[spaces-more-groupoids-section-phantom]{More on Groupoids in Spaces}
\item \hyperref[bootstrap-section-phantom]{Bootstrap}
\item \hyperref[spaces-pushouts-section-phantom]{Pushouts of Algebraic Spaces}
\end{enumerate}
Topics in Geometry
\begin{enumerate}
\setcounter{enumi}{81}
\item \hyperref[spaces-chow-section-phantom]{Chow Groups of Spaces}
\item \hyperref[groupoids-quotients-section-phantom]{Quotients of Groupoids}
\item \hyperref[spaces-more-cohomology-section-phantom]{More on Cohomology of Spaces}
\item \hyperref[spaces-simplicial-section-phantom]{Simplicial Spaces}
\item \hyperref[spaces-duality-section-phantom]{Duality for Spaces}
\item \hyperref[formal-spaces-section-phantom]{Formal Algebraic Spaces}
\item \hyperref[restricted-section-phantom]{Algebraization of Formal Spaces}
\item \hyperref[spaces-resolve-section-phantom]{Resolution of Surfaces Revisited}
\end{enumerate}
Deformation Theory
\begin{enumerate}
\setcounter{enumi}{89}
\item \hyperref[formal-defos-section-phantom]{Formal Deformation Theory}
\item \hyperref[defos-section-phantom]{Deformation Theory}
\item \hyperref[cotangent-section-phantom]{The Cotangent Complex}
\item \hyperref[examples-defos-section-phantom]{Deformation Problems}
\end{enumerate}
Algebraic Stacks
\begin{enumerate}
\setcounter{enumi}{93}
\item \hyperref[algebraic-section-phantom]{Algebraic Stacks}
\item \hyperref[examples-stacks-section-phantom]{Examples of Stacks}
\item \hyperref[stacks-sheaves-section-phantom]{Sheaves on Algebraic Stacks}
\item \hyperref[criteria-section-phantom]{Criteria for Representability}
\item \hyperref[artin-section-phantom]{Artin's Axioms}
\item \hyperref[quot-section-phantom]{Quot and Hilbert Spaces}
\item \hyperref[stacks-properties-section-phantom]{Properties of Algebraic Stacks}
\item \hyperref[stacks-morphisms-section-phantom]{Morphisms of Algebraic Stacks}
\item \hyperref[stacks-limits-section-phantom]{Limits of Algebraic Stacks}
\item \hyperref[stacks-cohomology-section-phantom]{Cohomology of Algebraic Stacks}
\item \hyperref[stacks-perfect-section-phantom]{Derived Categories of Stacks}
\item \hyperref[stacks-introduction-section-phantom]{Introducing Algebraic Stacks}
\item \hyperref[stacks-more-morphisms-section-phantom]{More on Morphisms of Stacks}
\item \hyperref[stacks-geometry-section-phantom]{The Geometry of Stacks}
\end{enumerate}
Topics in Moduli Theory
\begin{enumerate}
\setcounter{enumi}{107}
\item \hyperref[moduli-section-phantom]{Moduli Stacks}
\item \hyperref[moduli-curves-section-phantom]{Moduli of Curves}
\end{enumerate}
Miscellany
\begin{enumerate}
\setcounter{enumi}{109}
\item \hyperref[examples-section-phantom]{Examples}
\item \hyperref[exercises-section-phantom]{Exercises}
\item \hyperref[guide-section-phantom]{Guide to Literature}
\item \hyperref[desirables-section-phantom]{Desirables}
\item \hyperref[coding-section-phantom]{Coding Style}
\item \hyperref[obsolete-section-phantom]{Obsolete}
\item \hyperref[fdl-section-phantom]{GNU Free Documentation License}
\item \hyperref[index-section-phantom]{Auto Generated Index}
\end{enumerate}
\end{multicols}


\bibliography{my}
\bibliographystyle{amsalpha}

\end{document}
