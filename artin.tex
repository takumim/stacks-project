\IfFileExists{stacks-project.cls}{%
\documentclass{stacks-project}
}{%
\documentclass{amsart}
}

% For dealing with references we use the comment environment
\usepackage{verbatim}
\newenvironment{reference}{\comment}{\endcomment}
%\newenvironment{reference}{}{}
\newenvironment{slogan}{\comment}{\endcomment}
\newenvironment{history}{\comment}{\endcomment}

% For commutative diagrams we use Xy-pic
\usepackage[all]{xy}

% We use 2cell for 2-commutative diagrams.
\xyoption{2cell}
\UseAllTwocells

% We use multicol for the list of chapters between chapters
\usepackage{multicol}

% This is generall recommended for better output
\usepackage[T1]{fontenc}

% For cross-file-references
\usepackage{xr-hyper}

% Package for hypertext links:
\usepackage{hyperref}

% For any local file, say "hello.tex" you want to link to please
% use \externaldocument[hello-]{hello}
\externaldocument[introduction-]{introduction}
\externaldocument[conventions-]{conventions}
\externaldocument[sets-]{sets}
\externaldocument[categories-]{categories}
\externaldocument[topology-]{topology}
\externaldocument[sheaves-]{sheaves}
\externaldocument[sites-]{sites}
\externaldocument[stacks-]{stacks}
\externaldocument[fields-]{fields}
\externaldocument[algebra-]{algebra}
\externaldocument[brauer-]{brauer}
\externaldocument[homology-]{homology}
\externaldocument[derived-]{derived}
\externaldocument[simplicial-]{simplicial}
\externaldocument[more-algebra-]{more-algebra}
\externaldocument[smoothing-]{smoothing}
\externaldocument[modules-]{modules}
\externaldocument[sites-modules-]{sites-modules}
\externaldocument[injectives-]{injectives}
\externaldocument[cohomology-]{cohomology}
\externaldocument[sites-cohomology-]{sites-cohomology}
\externaldocument[dga-]{dga}
\externaldocument[dpa-]{dpa}
\externaldocument[hypercovering-]{hypercovering}
\externaldocument[schemes-]{schemes}
\externaldocument[constructions-]{constructions}
\externaldocument[properties-]{properties}
\externaldocument[morphisms-]{morphisms}
\externaldocument[coherent-]{coherent}
\externaldocument[divisors-]{divisors}
\externaldocument[limits-]{limits}
\externaldocument[varieties-]{varieties}
\externaldocument[topologies-]{topologies}
\externaldocument[descent-]{descent}
\externaldocument[perfect-]{perfect}
\externaldocument[more-morphisms-]{more-morphisms}
\externaldocument[flat-]{flat}
\externaldocument[groupoids-]{groupoids}
\externaldocument[more-groupoids-]{more-groupoids}
\externaldocument[etale-]{etale}
\externaldocument[chow-]{chow}
\externaldocument[intersection-]{intersection}
\externaldocument[pic-]{pic}
\externaldocument[adequate-]{adequate}
\externaldocument[dualizing-]{dualizing}
\externaldocument[duality-]{duality}
\externaldocument[discriminant-]{discriminant}
\externaldocument[local-cohomology-]{local-cohomology}
\externaldocument[curves-]{curves}
\externaldocument[resolve-]{resolve}
\externaldocument[models-]{models}
\externaldocument[pione-]{pione}
\externaldocument[etale-cohomology-]{etale-cohomology}
\externaldocument[proetale-]{proetale}
\externaldocument[crystalline-]{crystalline}
\externaldocument[spaces-]{spaces}
\externaldocument[spaces-properties-]{spaces-properties}
\externaldocument[spaces-morphisms-]{spaces-morphisms}
\externaldocument[decent-spaces-]{decent-spaces}
\externaldocument[spaces-cohomology-]{spaces-cohomology}
\externaldocument[spaces-limits-]{spaces-limits}
\externaldocument[spaces-divisors-]{spaces-divisors}
\externaldocument[spaces-over-fields-]{spaces-over-fields}
\externaldocument[spaces-topologies-]{spaces-topologies}
\externaldocument[spaces-descent-]{spaces-descent}
\externaldocument[spaces-perfect-]{spaces-perfect}
\externaldocument[spaces-more-morphisms-]{spaces-more-morphisms}
\externaldocument[spaces-flat-]{spaces-flat}
\externaldocument[spaces-groupoids-]{spaces-groupoids}
\externaldocument[spaces-more-groupoids-]{spaces-more-groupoids}
\externaldocument[bootstrap-]{bootstrap}
\externaldocument[spaces-pushouts-]{spaces-pushouts}
\externaldocument[spaces-chow-]{spaces-chow}
\externaldocument[groupoids-quotients-]{groupoids-quotients}
\externaldocument[spaces-more-cohomology-]{spaces-more-cohomology}
\externaldocument[spaces-simplicial-]{spaces-simplicial}
\externaldocument[spaces-duality-]{spaces-duality}
\externaldocument[formal-spaces-]{formal-spaces}
\externaldocument[restricted-]{restricted}
\externaldocument[spaces-resolve-]{spaces-resolve}
\externaldocument[formal-defos-]{formal-defos}
\externaldocument[defos-]{defos}
\externaldocument[cotangent-]{cotangent}
\externaldocument[examples-defos-]{examples-defos}
\externaldocument[algebraic-]{algebraic}
\externaldocument[examples-stacks-]{examples-stacks}
\externaldocument[stacks-sheaves-]{stacks-sheaves}
\externaldocument[criteria-]{criteria}
\externaldocument[artin-]{artin}
\externaldocument[quot-]{quot}
\externaldocument[stacks-properties-]{stacks-properties}
\externaldocument[stacks-morphisms-]{stacks-morphisms}
\externaldocument[stacks-limits-]{stacks-limits}
\externaldocument[stacks-cohomology-]{stacks-cohomology}
\externaldocument[stacks-perfect-]{stacks-perfect}
\externaldocument[stacks-introduction-]{stacks-introduction}
\externaldocument[stacks-more-morphisms-]{stacks-more-morphisms}
\externaldocument[stacks-geometry-]{stacks-geometry}
\externaldocument[moduli-]{moduli}
\externaldocument[moduli-curves-]{moduli-curves}
\externaldocument[examples-]{examples}
\externaldocument[exercises-]{exercises}
\externaldocument[guide-]{guide}
\externaldocument[desirables-]{desirables}
\externaldocument[coding-]{coding}
\externaldocument[obsolete-]{obsolete}
\externaldocument[fdl-]{fdl}
\externaldocument[index-]{index}

% Theorem environments.
%
\theoremstyle{plain}
\newtheorem{theorem}[subsection]{Theorem}
\newtheorem{proposition}[subsection]{Proposition}
\newtheorem{lemma}[subsection]{Lemma}

\theoremstyle{definition}
\newtheorem{definition}[subsection]{Definition}
\newtheorem{example}[subsection]{Example}
\newtheorem{exercise}[subsection]{Exercise}
\newtheorem{situation}[subsection]{Situation}

\theoremstyle{remark}
\newtheorem{remark}[subsection]{Remark}
\newtheorem{remarks}[subsection]{Remarks}

\numberwithin{equation}{subsection}

% Macros
%
\def\lim{\mathop{\mathrm{lim}}\nolimits}
\def\colim{\mathop{\mathrm{colim}}\nolimits}
\def\Spec{\mathop{\mathrm{Spec}}}
\def\Hom{\mathop{\mathrm{Hom}}\nolimits}
\def\Ext{\mathop{\mathrm{Ext}}\nolimits}
\def\SheafHom{\mathop{\mathcal{H}\!\mathit{om}}\nolimits}
\def\SheafExt{\mathop{\mathcal{E}\!\mathit{xt}}\nolimits}
\def\Sch{\mathit{Sch}}
\def\Mor{\mathop{Mor}\nolimits}
\def\Ob{\mathop{\mathrm{Ob}}\nolimits}
\def\Sh{\mathop{\mathit{Sh}}\nolimits}
\def\NL{\mathop{N\!L}\nolimits}
\def\proetale{{pro\text{-}\acute{e}tale}}
\def\etale{{\acute{e}tale}}
\def\QCoh{\mathit{QCoh}}
\def\Ker{\mathop{\mathrm{Ker}}}
\def\Im{\mathop{\mathrm{Im}}}
\def\Coker{\mathop{\mathrm{Coker}}}
\def\Coim{\mathop{\mathrm{Coim}}}

%
% Macros for moduli stacks/spaces
%
\def\QCohstack{\mathcal{QC}\!\mathit{oh}}
\def\Cohstack{\mathcal{C}\!\mathit{oh}}
\def\Spacesstack{\mathcal{S}\!\mathit{paces}}
\def\Quotfunctor{\mathrm{Quot}}
\def\Hilbfunctor{\mathrm{Hilb}}
\def\Curvesstack{\mathcal{C}\!\mathit{urves}}
\def\Polarizedstack{\mathcal{P}\!\mathit{olarized}}
\def\Complexesstack{\mathcal{C}\!\mathit{omplexes}}
% \Pic is the operator that assigns to X its picard group, usage \Pic(X)
% \Picardstack_{X/B} denotes the Picard stack of X over B
% \Picardfunctor_{X/B} denotes the Picard functor of X over B
\def\Pic{\mathop{\mathrm{Pic}}\nolimits}
\def\Picardstack{\mathcal{P}\!\mathit{ic}}
\def\Picardfunctor{\mathrm{Pic}}
\def\Deformationcategory{\mathcal{D}\!\mathit{ef}}


% OK, start here.
%
\begin{document}

\title{Artin's axioms}

\maketitle

\phantomsection
\label{section-phantom}

\tableofcontents




\section{Introduction}
\label{section-introduction}

\noindent
In this chapter we discuss Artin's axioms for the representability of
functors by algebraic spaces. As references we suggest the papers
\cite{ArtinI}, \cite{ArtinII}, \cite{ArtinVersal}.

\medskip\noindent
Some of the notation, conventions, and terminology in this chapter is awkward
and may seem backwards to the more experienced reader. This is intentional.
Please see Quot, Section \ref{quot-section-conventions} for an
explanation.






\section{Conventions}
\label{section-conventions}

\noindent
The conventions we use in this chapter are the same as those in the
chapter on algebraic stacks, see
Algebraic Stacks, Section \ref{algebraic-section-conventions}.
In this chapter the base scheme $S$ will often be locally Noetherian
(although we will always reiterate this condition when stating
results).





\section{Predeformation categories}
\label{section-predeformation-categories}

\noindent
Let $S$ be a locally Noetherian base scheme. Let
$$
p : \mathcal{X} \longrightarrow (\Sch/S)_{fppf}
$$
be a category fibred in groupoids. Let $k$ be a field
and let $\Spec(k) \to S$ be a morphism of finite type (see
Morphisms, Lemma \ref{morphisms-lemma-point-finite-type}). We will sometimes
simply say that {\it $k$ is a field of finite type over $S$}. Let
$x_0$ be an object of $\mathcal{X}$ lying over $\Spec(k)$.
Given $S$, $\mathcal{X}$, $k$, and $x_0$ we will construct a
predeformation category, as defined in
Formal Deformation Theory,
Definition \ref{formal-defos-definition-predeformation-category}.
The construction will resemble the construction of
Formal Deformation Theory,
Remark \ref{formal-defos-remark-localize-cofibered-groupoid}.

\medskip\noindent
First, by Morphisms, Lemma \ref{morphisms-lemma-point-finite-type}
we may pick an affine open $\Spec(\Lambda) \subset S$ such that
$\Spec(k) \to S$ factors through $\Spec(\Lambda)$ and the associated
ring map $\Lambda \to k$ is finite. This provides us with the category
$\mathcal{C}_\Lambda$, see
Formal Deformation Theory, Definition \ref{formal-defos-definition-CLambda}.
The category $\mathcal{C}_\Lambda$, up to canonical equivalence,
does not depend on the choice of the affine open $\Spec(\Lambda)$ of $S$.
Namely, $\mathcal{C}_\Lambda$ is equivalent to the opposite
of the category of factorizations
\begin{equation}
\label{equation-factor}
\Spec(k) \to \Spec(A) \to S
\end{equation}
of the structure morphism such that $A$ is an Artinian local ring and
such that $\Spec(k) \to \Spec(A)$ corresponds to a ring map $A \to k$ which
identifies $k$ with the residue field of $A$.

\medskip\noindent
We let $\mathcal{F} = \mathcal{F}_{\mathcal{X}, k, x_0}$ be the
category whose
\begin{enumerate}
\item objects are morphisms $x_0 \to x$ of $\mathcal{X}$ where
$p(x) = \Spec(A)$ with $A$ an Artinian local ring and
$p(x_0) \to p(x) \to S$ a factorization as in (\ref{equation-factor}), and
\item morphisms $(x_0 \to x) \to (x_0 \to x')$ are commutative
diagrams
$$
\xymatrix{
x & & x' \ar[ll] \\
& x_0 \ar[lu] \ar[ru]
}
$$
in $\mathcal{X}$. (Note the reversal of arrows.)
\end{enumerate}
If $x_0 \to x$ is an object of $\mathcal{F}$ then writing $p(x) = \Spec(A)$
we obtain an object $A$ of $\mathcal{C}_\Lambda$. We often say that
$x_0 \to x$ or $x$ lies over $A$. A morphism of $\mathcal{F}$ between objects
$x_0 \to x$ lying over $A$ and $x_0 \to x'$ lying over $A'$
corresponds to a morphism $x' \to x$ of $\mathcal{X}$, hence a morphism
$p(x' \to x) : \Spec(A') \to \Spec(A)$ which in turn corresponds to a
ring map $A \to A'$. As $\mathcal{X}$ is a category
over the category of schemes over $S$ we see that $A \to A'$ is
$\Lambda$-algebra homomorphism. Thus we obtain a functor
\begin{equation}
\label{equation-predeformation-category}
p : \mathcal{F} = \mathcal{F}_{\mathcal{X}, k, x_0}
\longrightarrow
\mathcal{C}_\Lambda.
\end{equation}
We will use the notation $\mathcal{F}(A)$ to denote the fibre category
over an object $A$ of $\mathcal{C}_\Lambda$. An object of $\mathcal{F}(A)$
is simply a morphism $x_0 \to x$ of $\mathcal{X}$ such that
$x$ lies over $\Spec(A)$ and $x_0 \to x$ lies over $\Spec(k) \to \Spec(A)$.

\begin{lemma}
\label{lemma-predeformation-category}
The functor $p : \mathcal{F} \to \mathcal{C}_\Lambda$ defined above
is a predeformation category.
\end{lemma}

\begin{proof}
We have to show that $\mathcal{F}$ is (a) cofibred in groupoids over
$\mathcal{C}_\Lambda$ and (b) that $\mathcal{F}(k)$ is a category equivalent
to a category with a single object and a single morphism.

\medskip\noindent
Proof of (a). The fibre categories of $\mathcal{F}$
over $\mathcal{C}_\Lambda$ are groupoids as the fibre categories
of $\mathcal{X}$ are groupoids. Let $A \to A'$ be a morphism of
$\mathcal{C}_\Lambda$ and let $x_0 \to x$ be an object of $\mathcal{F}(A)$.
Because $\mathcal{X}$ is fibred in groupoids, we can find a morphism
$x' \to x$ lying over $\Spec(A') \to \Spec(A)$. Since the composition
$A \to A' \to k$ is equal the given map $A \to k$ we see (by uniqueness
of pullbacks up to isomorphism) that the pullback via $\Spec(k) \to \Spec(A')$
of $x'$ is $x_0$, i.e., that there exists a morphism $x_0 \to x'$
lying over $\Spec(k) \to \Spec(A')$ compatible with
$x_0 \to x$ and $x' \to x$. This proves that $\mathcal{F}$ has
pushforwards. We conclude by (the dual of)
Categories, Lemma \ref{categories-lemma-fibred-groupoids}.

\medskip\noindent
Proof of (b). If $A = k$, then $\Spec(k) = \Spec(A)$ and since $\mathcal{X}$
is fibred in groupoids over $(\Sch/S)_{fppf}$ we see that given any object
$x_0 \to x$ in $\mathcal{F}(k)$ the morphism $x_0 \to x$ is an isomorphism.
Hence every object of $\mathcal{F}(k)$ is isomorphic to $x_0 \to x_0$.
Clearly the only self morphism of $x_0 \to x_0$ in $\mathcal{F}$ is
the identity.
\end{proof}

\noindent
Let $S$ be a locally Noetherian base scheme. Let
$F : \mathcal{X} \to \mathcal{Y}$ be a $1$-morphism between categories
fibred in groupoids over $(\Sch/S)_{fppf}$. Let $k$ is a field
of finite type over $S$. Let $x_0$ be an object of $\mathcal{X}$ lying
over $\Spec(k)$. Set $y_0 = F(x_0)$ which is an object of $\mathcal{Y}$
lying over $\Spec(k)$. Then $F$ induces a functor
\begin{equation}
\label{equation-functoriality}
F :
\mathcal{F}_{\mathcal{X}, k, x_0}
\longrightarrow
\mathcal{F}_{\mathcal{Y}, k, y_0}
\end{equation}
of categories cofibred over $\mathcal{C}_\Lambda$. Namely, to the object
$x_0 \to x$ of $\mathcal{F}_{\mathcal{X}, k, x_0}(A)$ we associate
the object $F(x_0) \to F(x)$ of $\mathcal{F}_{\mathcal{Y}, k, y_0}(A)$.

\begin{lemma}
\label{lemma-formally-smooth-on-deformation-categories}
Let $S$ be a locally Noetherian scheme. Let $F : \mathcal{X} \to \mathcal{Y}$
be a $1$-morphism of categories fibred in groupoids over $(\Sch/S)_{fppf}$.
Assume either
\begin{enumerate}
\item $F$ is formally smooth on objects (Criteria for Representability,
Section \ref{criteria-section-formally-smooth}),
\item $F$ is representable by algebraic spaces and formally smooth, or
\item $F$ is representable by algebraic spaces and smooth.
\end{enumerate}
Then for every finite type field $k$ over $S$ and object
$x_0$ of $\mathcal{X}$ over $k$ the functor (\ref{equation-functoriality})
is smooth in the sense of
Formal Deformation Theory, Definition
\ref{formal-defos-definition-smooth-morphism}.
\end{lemma}

\begin{proof}
Case (1) is a matter of unwinding the definitions.
Assumption (2) implies (1) by
Criteria for Representability, Lemma
\ref{criteria-lemma-representable-by-spaces-formally-smooth}.
Assumption (3) implies (2) by
More on Morphisms of Spaces, Lemma
\ref{spaces-more-morphisms-lemma-smooth-formally-smooth}
and the principle of
Algebraic Stacks, Lemma
\ref{algebraic-lemma-representable-transformations-property-implication}.
\end{proof}

\begin{lemma}
\label{lemma-fibre-product-deformation-categories}
Let $S$ be a locally Noetherian scheme. Let
$$
\xymatrix{
\mathcal{W} \ar[d] \ar[r] & \mathcal{Z} \ar[d] \\
\mathcal{X} \ar[r] & \mathcal{Y}
}
$$
be a $2$-fibre product of categories fibred in groupoids over
$(\Sch/S)_{fppf}$. Let $k$ be a finite type field over $S$ and
$w_0$ an object of $\mathcal{W}$ over $k$. Let $x_0, z_0, y_0$ be
the images of $w_0$ under the morphisms in the diagram. Then
$$
\xymatrix{
\mathcal{F}_{\mathcal{W}, k, w_0} \ar[d] \ar[r] &
\mathcal{F}_{\mathcal{Z}, k, z_0} \ar[d] \\
\mathcal{F}_{\mathcal{X}, k, x_0} \ar[r] & \mathcal{F}_{\mathcal{Y}, k, y_0}
}
$$
is a fibre product of predeformation categories.
\end{lemma}

\begin{proof}
This is a matter of unwinding the definitions. Details omitted.
\end{proof}






\section{Pushouts and stacks}
\label{section-pushouts}

\noindent
In this section we show that algebraic stacks behave well with
respect to certain pushouts. The results in this section hold over
any base scheme.

\medskip\noindent
The following lemma is also correct when $Y$, $X'$, $X$, $Y'$ are
algebraic spaces, see (insert future reference here).

\begin{lemma}
\label{lemma-pushout}
Let $S$ be a scheme. Let
$$
\xymatrix{
X \ar[r] \ar[d] & X' \ar[d] \\
Y \ar[r] & Y'
}
$$
be a pushout in the category of schemes over $S$ where $X \to X'$
is a thickening and $X \to Y$ is affine, see
More on Morphisms, Lemma \ref{more-morphisms-lemma-pushout-along-thickening}.
Let $\mathcal{Z}$ be an algebraic stack over $S$.
Then the functor of fibre categories
$$
\mathcal{Z}_{Y'}
\longrightarrow
\mathcal{Z}_Y \times_{\mathcal{Z}_X} \mathcal{Z}_{X'}
$$
is an equivalence of categories.
\end{lemma}

\begin{proof}
Let $y'$ be an object of left hand side. The sheaf
$\mathit{Isom}(y', y')$ on the category of schemes over $Y'$
is representable by an algebraic space $I$ over $Y'$, see
Algebraic Stacks, Lemma \ref{algebraic-lemma-representable-diagonal}.
We conclude that the functor of the lemma is fully faithful as
$Y'$ is the pushout in the category of algebraic spaces as
well as the category of schemes, see
Pushouts of Spaces, Lemma
\ref{spaces-pushouts-lemma-pushout-along-thickening-schemes}.

\medskip\noindent
Let $(y, x', f)$ be an object of the right hand side. Here $f : y|_X \to x'|_X$
is an isomorphism. To finish the proof we have to construct an object $y'$ of
$\mathcal{Z}_{Y'}$ whose restrictions to $Y$ and $X'$ agree with $y$ and $x'$
in a manner compatible with $\varphi$. In fact, it suffices to construct $y'$
fppf locally on $Y'$, see
Stacks, Lemma \ref{stacks-lemma-characterize-essentially-surjective-when-ff}.
Choose a representable algebraic stack
$\mathcal{W}$ and a surjective smooth morphism $\mathcal{W} \to \mathcal{Z}$.
Then
$$
(\Sch/Y)_{fppf} \times_{y, \mathcal{Z}} \mathcal{W}
\quad\text{and}\quad
(\Sch/X')_{fppf} \times_{x', \mathcal{Z}} \mathcal{W}
$$
are algebraic stacks representable by algebraic spaces $V$ and $U'$
smooth over $Y$ and $X'$. The isomorphism $f$ induces an isomorphism
$\varphi : V \times_Y X \to U' \times_{X'} X$ over $X$. By
Pushouts of Spaces, Lemmas
\ref{spaces-pushouts-lemma-pushout-along-thickening} and
\ref{spaces-pushouts-lemma-equivalence-categories-spaces-pushout-flat}
we see that the pushout $V' = V \amalg_{V \times_Y X} U'$ is
an algebraic space smooth over $Y'$ whose base change to
$Y$ and $X'$ recovers $V$ and $U'$ in a manner compatible with $\varphi$.

\medskip\noindent
Let $W$ be the algebraic space representing $\mathcal{W}$.
The projections $V \to W$ and $U' \to W$ agree as morphisms
over $V \times_Y X \cong U' \times_{X'} X$ hence the universal
property of the pushout determines a morphism of algebraic spaces
$V' \to W$. Choose a scheme $Y_1'$ and a surjective \'etale morphism
$Y_1' \to V'$. Set $Y_1 = Y \times_{Y'} Y_1'$,
$X_1' = X' \times_{Y'} Y_1'$, $X_1 = X \times_{Y'} Y_1'$.
The composition
$$
(\Sch/Y_1') \to (\Sch/V') \to (\Sch/W) = \mathcal{W} \to \mathcal{Z}
$$
corresponds by the $2$-Yoneda lemma to an object $y_1'$ of $\mathcal{Z}$
over $Y_1'$ whose restriction to $Y_1$ and $X_1'$ agrees with $y|_{Y_1}$
and $x'|_{X_1'}$ in a manner compatible with $f|_{X_1}$. Thus we have
constructed our desired object smooth locally over $Y'$ and we win.
\end{proof}








\section{The Rim-Schlessinger condition}
\label{section-RS}

\noindent
The motivation for the following definition comes from
Lemma \ref{lemma-pushout}
and
Formal Deformation Theory, Definition \ref{formal-defos-definition-RS} and
Lemma \ref{formal-defos-lemma-RS-2-categorical}.

\begin{definition}
\label{definition-RS}
Let $S$ be a locally Noetherian scheme. Let $\mathcal{Z}$ be a category
fibred in groupoids over $(\Sch/S)_{fppf}$. We say $\mathcal{Z}$
satisfies {\it condition (RS)} if for every pushout
$$
\xymatrix{
X \ar[r] \ar[d] & X' \ar[d] \\
Y \ar[r] & Y' = Y \amalg_X X'
}
$$
in the category of schemes over $S$ where
\begin{enumerate}
\item $X$, $X'$, $Y$, $Y'$ are spectra of local Artinian rings,
\item $X$, $X'$, $Y$, $Y'$ are of finite type over $S$, and
\item $X \to X'$ (and hence $Y \to Y'$) is a closed immersion
\end{enumerate}
the functor of fibre categories
$$
\mathcal{Z}_{Y'}
\longrightarrow
\mathcal{Z}_Y \times_{\mathcal{Z}_X} \mathcal{Z}_{X'}
$$
is an equivalence of categories.
\end{definition}

\noindent
If $A$ is an Artinian local ring with residue field $k$, then
any morphism $\Spec(A) \to S$ is affine and of finite type if and
only if the induced morphism $\Spec(k) \to S$ is of finite type, see
Morphisms, Lemmas \ref{morphisms-lemma-Artinian-affine} and
\ref{morphisms-lemma-artinian-finite-type}.

\begin{lemma}
\label{lemma-algebraic-stack-RS}
Let $\mathcal{X}$ be an algebraic stack over a locally Noetherian base
$S$. Then $\mathcal{X}$ satisfies (RS).
\end{lemma}

\begin{proof}
Immediate from the definitions and Lemma \ref{lemma-pushout}.
\end{proof}

\begin{lemma}
\label{lemma-fibre-product-RS}
Let $S$ be a scheme. Let $p : \mathcal{X} \to \mathcal{Y}$ and
$q : \mathcal{Z} \to \mathcal{Y}$ be $1$-morphisms of categories
fibred in groupoids over $(\Sch/S)_{fppf}$. If $\mathcal{X}$, $\mathcal{Y}$,
and $\mathcal{Z}$ satisfy (RS), then so
does $\mathcal{X} \times_\mathcal{Y} \mathcal{Z}$.
\end{lemma}

\begin{proof}
This is formal. Let 
$$
\xymatrix{
X \ar[r] \ar[d] & X' \ar[d] \\
Y \ar[r] & Y' = Y \amalg_X X'
}
$$
be a diagram as in Definition \ref{definition-RS}. We have to show that
$$
(\mathcal{X} \times_{\mathcal{Y}} \mathcal{Z})_{Y'}
\longrightarrow
(\mathcal{X} \times_{\mathcal{Y}} \mathcal{Z})_Y
\times_{(\mathcal{X} \times_{\mathcal{Y}} \mathcal{Z})_X}
(\mathcal{X} \times_{\mathcal{Y}} \mathcal{Z})_{X'}
$$
is an equivalence. Using the definition of the $2$-fibre product
this becomes
\begin{equation}
\label{equation-RS-fibre-product}
\mathcal{X}_{Y'} \times_{\mathcal{Y}_{Y'}} \mathcal{Z}_{Y'}
\longrightarrow
(\mathcal{X}_Y \times_{\mathcal{Y}_Y} \mathcal{Z}_Y)
\times_{(\mathcal{X}_X \times_{\mathcal{Y}_X} \mathcal{Z}_X)}
(\mathcal{X}_{X'} \times_{\mathcal{Y}_{X'}} \mathcal{Z}_{X'}).
\end{equation}
We are given that each of the functors
$$
\mathcal{X}_{Y'} \to \mathcal{X}_Y \times_{\mathcal{Y}_Y} \mathcal{Z}_Y,
\quad
\mathcal{Y}_{Y'} \to \mathcal{X}_X \times_{\mathcal{Y}_X} \mathcal{Z}_X,
\quad
\mathcal{Z}_{Y'} \to
\mathcal{X}_{X'} \times_{\mathcal{Y}_{X'}} \mathcal{Z}_{X'}
$$
are equivalences. An object of the right hand side of
(\ref{equation-RS-fibre-product}) is a system
$$
((x_Y, z_Y, \phi_Y), (x_{X'}, z_{X'}, \phi_{X'}), (\alpha, \beta)).
$$
Then $(x_Y, x_{Y'}, \alpha)$ is isomorphic to the image of an object
$x_{Y'}$ in $\mathcal{X}_{Y'}$ and $(z_Y, z_{Y'}, \beta)$ is isomorphic
to the image of an object $z_{Y'}$ of $\mathcal{Z}_{Y'}$. The pair of
morphisms $(\phi_Y, \phi_{X'})$ corresponds to a morphism $\psi$
between the images of $x_{Y'}$ and $z_{Y'}$ in $\mathcal{Y}_{Y'}$.
Then $(x_{Y'}, z_{Y'}, \psi)$ is an object of the left hand side of
(\ref{equation-RS-fibre-product}) mapping to the given object of the
right hand side. This proves that (\ref{equation-RS-fibre-product}) is
essentially surjective. We omit the proof that it is fully faithful.
\end{proof}





\section{Deformation categories}
\label{section-deformation-categories}

\noindent
We match the notation introduced above with the notation from the
chapter ``Formal Deformation Theory''.

\begin{lemma}
\label{lemma-deformation-category}
Let $S$ be a locally Noetherian scheme. Let $\mathcal{X}$ be a category
fibred in groupoids over $(\Sch/S)_{fppf}$ satisfying (RS). For any field
$k$ of finite type over $S$ and any object $x_0$ of $\mathcal{X}$ lying
over $k$ the predeformation category
$p : \mathcal{F}_{\mathcal{X}, k, x_0} \to \mathcal{C}_\Lambda$
(\ref{equation-predeformation-category}) is a deformation category, see
Formal Deformation Theory, Definition
\ref{formal-defos-definition-deformation-category}.
\end{lemma}

\begin{proof}
Set $\mathcal{F} = \mathcal{F}_{\mathcal{X}, k, x_0}$.
Let $f_1 : A_1 \to A$ and $f_2 : A_2 \to A$ be ring maps in
$\mathcal{C}_\Lambda$ with $f_2$ surjective. We have to show that
the functor
$$
\mathcal{F}(A_1 \times_A A_2)
\longrightarrow
\mathcal{F}(A_1) \times_{\mathcal{F}(A)} \mathcal{F}(A_2)
$$
is an equivalence, see
Formal Deformation Theory, Lemma \ref{formal-defos-lemma-RS-2-categorical}.
Set $X = \Spec(A)$, $X' = \Spec(A_2)$, $Y = \Spec(A_1)$ and
$Y' = \Spec(A_1 \times_A A_2)$. Note that $Y' = Y \amalg_X X'$ in the
category of schemes, see
More on Morphisms, Lemma \ref{more-morphisms-lemma-pushout-along-thickening}.
We know that in the diagram of functors of fibre categories
$$
\xymatrix{
\mathcal{X}_{Y'} \ar[r] \ar[d] &
\mathcal{X}_Y \times_{\mathcal{X}_X} \mathcal{X}_{X'} \ar[d] \\
\mathcal{X}_{\Spec(k)} \ar@{=}[r] & \mathcal{X}_{\Spec(k)}
}
$$
the top horizontal arrow is an equivalence by
Definition \ref{definition-RS}.
Since $\mathcal{F}(B)$ is the category of objects of $\mathcal{X}_{\Spec(B)}$
with an identification with $x_0$ over $k$ we win.
\end{proof}

\begin{remark}
\label{remark-deformation-category-implies}
Let $S$ be a locally Noetherian scheme. Let $\mathcal{X}$ be fibred
in groupoids over $(\Sch/S)_{fppf}$. Let $k$ be a field of finite type over
$S$ and $x_0$ an object
of $\mathcal{X}$ over $k$. Let $p : \mathcal{F} \to \mathcal{C}_\Lambda$
be as in (\ref{equation-predeformation-category}). If $\mathcal{F}$
is a deformation category, i.e., if $\mathcal{F}$ satisfies the
Rim-Schlessinger condition (RS), then we see that $\mathcal{F}$ satisfies
Schlessinger's conditions (S1) and (S2) by
Formal Deformation Theory, Lemma \ref{formal-defos-lemma-RS-implies-S1-S2}.
Let $\overline{\mathcal{F}}$ be the functor of isomorphism classes, see
Formal Deformation Theory, Remarks
\ref{formal-defos-remarks-cofibered-groupoids}
(\ref{formal-defos-item-associated-functor-isomorphism-classes}).
Then $\overline{\mathcal{F}}$ satisfies (S1) and (S2) as well, see
Formal Deformation Theory, Lemma
\ref{formal-defos-lemma-S1-S2-associated-functor}.
This holds in particular in the situation of
Lemma \ref{lemma-deformation-category}.
\end{remark}




\section{Change of field}
\label{section-change-of-field}

\noindent
This section is the analogue of
Formal Deformation Theory, Section \ref{formal-defos-section-change-of-field}.
As pointed out there, to discuss what happens under change of field
we need to write $\mathcal{C}_{\Lambda, k}$ instead of $\mathcal{C}_\Lambda$.
In the following lemma we use the notation $\mathcal{F}_{l/k}$
introduced in Formal Deformation Theory, Situation
\ref{formal-defos-situation-change-of-fields}.

\begin{lemma}
\label{lemma-change-of-field}
Let $S$ be a locally Noetherian scheme. Let $\mathcal{X}$ be a category
fibred in groupoids over $(\Sch/S)_{fppf}$. Let $k$ be a
field of finite type over $S$ and let $l/k$ be a finite extension.
Let $x_0$ be an object of $\mathcal{F}$ lying over $\Spec(k)$.
Denote $x_{l, 0}$ the restriction of $x_0$ to $\Spec(l)$.
Then there is a canonical functor
$$
(\mathcal{F}_{\mathcal{X}, k , x_0})_{l/k}
\longrightarrow
\mathcal{F}_{\mathcal{X}, l, x_{l, 0}}
$$
of categories cofibred in groupoids over $\mathcal{C}_{\Lambda, l}$.
If $\mathcal{X}$ satisfies (RS), then this functor is an equivalence.
\end{lemma}

\begin{proof}
Consider a factorization
$$
\Spec(l) \to \Spec(B) \to S
$$
as in (\ref{equation-factor}). By definition we have
$$
(\mathcal{F}_{\mathcal{X}, k , x_0})_{l/k}(B) =
\mathcal{F}_{\mathcal{X}, k, x_0}(B \times_l k)
$$
see Formal Deformation Theory, Situation
\ref{formal-defos-situation-change-of-fields}. Thus an object of this
is a morphism $x_0 \to x$ of $\mathcal{X}$ lying over the morphism
$\Spec(k) \to \Spec(B \times_l k)$. Choosing pullback functor for $\mathcal{X}$
we can associate to $x_0 \to x$ the morphism $x_{l, 0} \to x_B$
where $x_B$ is the restriction of $x$ to $\Spec(B)$ (via the morphism
$\Spec(B) \to \Spec(B \times_l k)$ coming from $B \times_l k \subset B$).
This construction is functorial in $B$ and compatible with morphisms.

\medskip\noindent
Next, assume $\mathcal{X}$ satisfies (RS). Consider the diagrams
$$
\vcenter{
\xymatrix{
l & B \ar[l] \\
k \ar[u] & B \times_l k \ar[l] \ar[u]
}
}
\quad\text{and}\quad
\vcenter{
\xymatrix{
\Spec(l) \ar[d] \ar[r] & \Spec(B) \ar[d] \\
\Spec(k) \ar[r] & \Spec(B \times_l k)
}
}
$$
The diagram on the left is a fibre product of rings. The diagram on the
right is a pushout in the category of schemes, see
More on Morphisms, Lemma \ref{more-morphisms-lemma-pushout-along-thickening}.
These schemes are all of finite type over $S$ (see remarks following
Definition \ref{definition-RS}). Hence (RS) kicks in to give an equivalence
of fibre categories
$$
\mathcal{X}_{\Spec(B \times_l k)}
\longrightarrow
\mathcal{X}_{\Spec(k)}
\times_{\mathcal{X}_{\Spec(l)}}
\mathcal{X}_{\Spec(B)}
$$
This implies that the functor defined above gives an equivalence of
fibre categories. Hence the functor is an equivalence on categories
cofibred in groupoids by (the dual of)
Categories, Lemma \ref{categories-lemma-equivalence-fibred-categories}.
\end{proof}








\section{Tangent spaces}
\label{section-tangent-spaces}

\noindent
Let $S$ be a locally Noetherian scheme. Let $\mathcal{X}$ be a category
fibred in groupoids over $(\Sch/S)_{fppf}$. Let $k$ be a field of finite
type over $S$ and let $x_0$ be an object of $\mathcal{X}$ over $k$.
In Formal Deformation Theory, Section \ref{formal-defos-section-tangent-spaces}
we have defined the {\it tangent space}
\begin{equation}
\label{equation-tangent-space}
T\mathcal{F}_{\mathcal{X}, k, x_0} =
\left\{
\begin{matrix}
\text{isomorphism classes of morphisms}\\
x_0 \to x\text{ over }\Spec(k) \to \Spec(k[\epsilon])
\end{matrix}
\right\}
\end{equation}
of the predeformation category $\mathcal{F}_{\mathcal{X}, k, x_0}$.
In Formal Deformation Theory, Section
\ref{formal-defos-section-infinitesimal-automorphisms}
we have defined
\begin{equation}
\label{equation-infinitesimal-automorphisms}
\text{Inf}(\mathcal{F}_{\mathcal{X}, k, x_0}) =
\Ker\left(
\text{Aut}_{\Spec(k[\epsilon])}(x'_0) \to \text{Aut}_{\Spec(k)}(x_0)
\right)
\end{equation}
where $x_0'$ is the pullback of $x_0$ to $\Spec(k[\epsilon])$.
If $\mathcal{X}$ satisfies the Rim-Schlessinger condition (RS), then
$T\mathcal{F}_{\mathcal{X}, k, x_0}$ comes equipped with a natural
$k$-vector space structure by Formal Deformation Theory, Lemma
\ref{formal-defos-lemma-tangent-space-vector-space}
(assumptions hold by Lemma \ref{lemma-deformation-category} and
Remark \ref{remark-deformation-category-implies}). Moreover,
Formal Deformation Theory, Lemma \ref{formal-defos-lemma-infaut-vector-space}
shows that $\text{Inf}(\mathcal{F}_{\mathcal{X}, k, x_0})$ has a
natural $k$-vector space structure such that addition agrees with
composition of automorphisms. A natural condition
is to ask these vector spaces to have finite dimension.

\medskip\noindent
The following lemma tells us this is true if
$\mathcal{X}$ is locally of finite type over $S$ (see
Morphisms of Stacks, Section \ref{stacks-morphisms-section-finite-type}).

\begin{lemma}
\label{lemma-finite-dimension}
Let $S$ be a locally Noetherian scheme. Assume
\begin{enumerate}
\item $\mathcal{X}$ is an algebraic stack,
\item $U$ is a scheme locally of finite type over $S$, and
\item $(\Sch/U)_{fppf} \to \mathcal{X}$ is a smooth surjective
morphism.
\end{enumerate}
Then, for any $\mathcal{F} = \mathcal{F}_{\mathcal{X}, k, x_0}$ as in
Section \ref{section-predeformation-categories}
the tangent space $T\mathcal{F}$ and infinitesimal automorphism space
$\text{Inf}(\mathcal{F})$ have finite dimension over $k$
\end{lemma}

\begin{proof}
Let us write $\mathcal{U} = (\Sch/U)_{fppf}$. By our definition
of algebraic stacks the $1$-morphism $\mathcal{U} \to \mathcal{X}$
is representable by algebraic spaces. Hence in particular the
2-fibre product
$$
\mathcal{U}_{x_0} = (\Sch/\Spec(k))_{fppf} \times_\mathcal{X} \mathcal{U}
$$
is representable by an algebraic space $U_{x_0}$ over $\Spec(k)$. Then
$U_{x_0} \to \Spec(k)$ is smooth and surjective (in particular $U_{x_0}$
is nonempty). By Spaces over Fields, Lemma
\ref{spaces-over-fields-lemma-smooth-separable-closed-points-dense}
we can find a finite extension $l \supset k$ and a point
$\Spec(l) \to U_{x_0}$ over $k$. We have
$$
(\mathcal{F}_{\mathcal{X}, k , x_0})_{l/k} =
\mathcal{F}_{\mathcal{X}, l, x_{l, 0}}
$$
by Lemma \ref{lemma-change-of-field} and the fact that $\mathcal{X}$
satisfies (RS). Thus we see that
$$
T\mathcal{F} \otimes_k l \cong T\mathcal{F}_{\mathcal{X}, l, x_{l, 0}}
\quad\text{and}\quad
\text{Inf}(\mathcal{F}) \otimes_k l \cong
\text{Inf}(\mathcal{F}_{\mathcal{X}, l, x_{l, 0}})
$$
by
Formal Deformation Theory, Lemmas
\ref{formal-defos-lemma-tangent-space-change-of-field} and
\ref{formal-defos-lemma-inf-aut-change-of-field}
(these are applicable by
Lemmas \ref{lemma-algebraic-stack-RS} and
\ref{lemma-deformation-category} and
Remark \ref{remark-deformation-category-implies}).
Hence it suffices to prove that $T\mathcal{F}_{\mathcal{X}, l, x_{l, 0}}$
and $\text{Inf}(\mathcal{F}_{\mathcal{X}, l, x_{l, 0}})$
have finite dimension over $l$. Note that $x_{l, 0}$ comes from a point
$u_0$ of $\mathcal{U}$ over $l$.

\medskip\noindent
We interrupt the flow of the argument to show that the lemma for
infinitesimal automorphisms follows from the lemma for tangent spaces.
Namely, let
$\mathcal{R} = \mathcal{U} \times_\mathcal{X} \mathcal{U}$.
Let $r_0$ be the $l$-valued point $(u_0, u_0, \text{id}_{x_0})$ of
$\mathcal{R}$. Combining
Lemma \ref{lemma-fibre-product-deformation-categories} and
Formal Deformation Theory, Lemma
\ref{formal-defos-lemma-deformation-functor-diagonal}
we see that
$$
\text{Inf}(\mathcal{F}_{\mathcal{X}, l, x_{l, 0}})
\subset
T\mathcal{F}_{\mathcal{R}, l, r_0}
$$
Note that $\mathcal{R}$ is an algebraic stack, see
Algebraic Stacks, Lemma \ref{algebraic-lemma-2-fibre-product-general}.
Also, $\mathcal{R}$ is representable by an algebraic space $R$
smooth over $U$ (via either projection, see
Algebraic Stacks, Lemma \ref{algebraic-lemma-stack-presentation}).
Hence, choose an scheme $U'$ and a surjective \'etale morphism
$U' \to R$ we see that $U'$ is smooth over $U$, hence locally of
finite type over $S$. As $(\Sch/U')_{fppf} \to \mathcal{R}$ is
surjective and smooth, we have reduced the question to the case
of tangent spaces.

\medskip\noindent
The functor (\ref{equation-functoriality})
$$
\mathcal{F}_{\mathcal{U}, l, u_0}
\longrightarrow
\mathcal{F}_{\mathcal{X}, l, x_{l, 0}}
$$
is smooth by Lemma \ref{lemma-formally-smooth-on-deformation-categories}.
The induced map on tangent spaces
$$
T\mathcal{F}_{\mathcal{U}, l, u_0}
\longrightarrow
T\mathcal{F}_{\mathcal{X}, l, x_{l, 0}}
$$
is $l$-linear (by
Formal Deformation Theory, Lemma
\ref{formal-defos-lemma-k-linear-differential})
and surjective (as smooth maps of predeformation categories induce
surjective maps on tangent spaces by
Formal Deformation Theory, Lemma
\ref{formal-defos-lemma-smooth-morphism-essentially-surjective}).
Hence it suffices to prove that the tangent space of the deformation
space associated to the representable algebraic stack $\mathcal{U}$
at the point $u_0$ is finite dimensional. Let $\Spec(R) \subset U$ be
an affine open such that $u_0 : \Spec(l) \to U$ factors through $\Spec(R)$
and such that $\Spec(R) \to S$ factors through $\Spec(\Lambda) \subset S$.
Let $\mathfrak m_R \subset R$ be the kernel of the $\Lambda$-algebra map
$\varphi_0 : R \to l$ corresponding to $u_0$. Note that $R$, being of finite
type over the Noetherian ring $\Lambda$, is a Noetherian ring. Hence
$\mathfrak m_R = (f_1, \ldots, f_n)$ is a finitely generated ideal.
We have
$$
T\mathcal{F}_{\mathcal{U}, l, u_0}
=
\{\varphi : R \to l[\epsilon] \mid
\varphi \text{ is a } \Lambda\text{-algebra map and }
\varphi \bmod \epsilon = \varphi_0\}
$$
An element of the right hand side is determined by its values on
$f_1, \ldots, f_n$ hence the dimension is at most $n$ and we win.
Some details omitted.
\end{proof}

\begin{lemma}
\label{lemma-fibre-product-tangent-spaces}
Let $S$ be a locally Noetherian scheme. Let $p : \mathcal{X} \to \mathcal{Y}$
and $q : \mathcal{Z} \to \mathcal{Y}$ be $1$-morphisms of categories
fibred in groupoids over $(\Sch/S)_{fppf}$. Assume $\mathcal{X}$,
$\mathcal{Y}$, $\mathcal{Z}$ satisfy (RS).
Let $k$ be a field of finite type over $S$ and let $w_0$ be an object of
$\mathcal{W} = \mathcal{X} \times_\mathcal{Y} \mathcal{Z}$ over $k$.
Denote $x_0, y_0, z_0$ the objects of $\mathcal{X}, \mathcal{Y}, \mathcal{Z}$
you get from $w_0$. Then there is a $6$-term exact sequence
$$
\xymatrix{
0 \ar[r] &
\text{Inf}(\mathcal{F}_{\mathcal{W}, k, w_0}) \ar[r] &
\text{Inf}(\mathcal{F}_{\mathcal{X}, k, x_0}) \oplus
\text{Inf}(\mathcal{F}_{\mathcal{Z}, k, z_0}) \ar[r] &
\text{Inf}(\mathcal{F}_{\mathcal{Y}, k, y_0}) \ar[lld] \\
 &
T\mathcal{F}_{\mathcal{W}, k, w_0} \ar[r] &
T\mathcal{F}_{\mathcal{X}, k, x_0} \oplus
T\mathcal{F}_{\mathcal{Z}, k, z_0} \ar[r] &
T\mathcal{F}_{\mathcal{Y}, k, y_0}
}
$$
of $k$-vector spaces.
\end{lemma}

\begin{proof}
By Lemma \ref{lemma-fibre-product-RS} we see that $\mathcal{W}$
satisfies (RS) and hence the lemma makes sense. To see the lemma
is true, apply Lemmas \ref{lemma-fibre-product-deformation-categories} and
\ref{lemma-deformation-category}
and Formal Deformation Theory, Lemma
\ref{formal-defos-lemma-deformation-categories-fiber-product-morphisms}.
\end{proof}






\section{Formal objects}
\label{section-formal-objects}

\noindent
In this section we transfer some of the notions already defined
in the chapter ``Formal Deformation Theory'' to the current setting.
In the following we will say ``$R$ is an $S$-algebra'' to indicate
that $R$ is a ring endowed with a morphism of schemes $\Spec(R) \to S$.

\begin{definition}
\label{definition-formal-objects}
Let $S$ be a locally Noetherian scheme. Let
$p : \mathcal{X} \to (\Sch/S)_{fppf}$ be a category fibred in groupoids.
\begin{enumerate}
\item A {\it formal object} $\xi = (R, \xi_n, f_n)$ of $\mathcal{X}$ consists
of a Noetherian complete local $S$-algebra $R$, objects $\xi_n$ of
$\mathcal{X}$ lying over $\Spec(R/\mathfrak m_R^n)$, and morphisms
$f_n : \xi_n \to \xi_{n + 1}$ of $\mathcal{X}$ lying over
$\Spec(R/\mathfrak m^n) \to \Spec(R/\mathfrak m^{n + 1})$
such that $R/\mathfrak m$ is a field of finite type over $S$.
\item A {\it morphism of formal objects}
$a : \xi = (R, \xi_n, f_n) \to \eta = (T, \eta_n, g_n)$
is given by morphisms $a_n : \xi_n \to \eta_n$ such that for every $n$
the diagram
$$
\xymatrix{
\xi_{n + 1} \ar[r]_{f_n} \ar[d]_{a_{n + 1}} & \xi_n \ar[d]^{a_n} \\
\eta_{n + 1} \ar[r]^{g_n} & \eta_n
}
$$
is commutative. Applying the functor $p$ we obtain a compatible collection
of morphisms $\Spec(R/\mathfrak m_R^n) \to \Spec(T/\mathfrak m_T^n)$ and
hence a morphism $a_0 : \Spec(R) \to \Spec(T)$ over $S$. We say that
$a$ {\it lies over} $a_0$.
\end{enumerate}
\end{definition}

\noindent
Thus we obtain a category of formal objects of $\mathcal{X}$.

\begin{remark}
\label{remark-formal-objects-match}
Let $S$ be a locally Noetherian scheme. Let
$p : \mathcal{X} \to (\Sch/S)_{fppf}$ be a category fibred in groupoids.
Let $\xi = (R, \xi_n, f_n)$ be a formal object. Set $k = R/\mathfrak m$ and
$x_0 = \xi_1$. The formal object $\xi$ defines a formal object
$\xi$ of the predeformation category $\mathcal{F}_{\mathcal{X}, k, x_0}$.
This follows immediately from
Definition \ref{definition-formal-objects} above,
Formal Deformation Theory, Definition
\ref{formal-defos-definition-formal-objects},
and our construction of the predeformation category
$\mathcal{F}_{\mathcal{X}, k, x_0}$ in
Section \ref{section-predeformation-categories}.
\end{remark}

\noindent
If $F : \mathcal{X} \to \mathcal{Y}$ is a $1$-morphism of categories fibred
in groupoids over $(\Sch/S)_{fppf}$, then $F$ induces a functor between
categories of formal objects as well.

\begin{lemma}
\label{lemma-smooth-lift-formal}
Let $S$ be a locally Noetherian scheme. Let $F : \mathcal{X} \to \mathcal{Y}$
be a $1$-morphism of categories fibred in groupoids over $(\Sch/S)_{fppf}$.
Let $\eta = (R, \eta_n, g_n)$ be a formal object of $\mathcal{Y}$
and let $\xi_1$ be an object of $\mathcal{X}$ with $F(\xi_1) \cong \eta_1$.
If $F$ is formally smooth on objects (see
Criteria for Representability, Section \ref{criteria-section-formally-smooth}),
then there exists a formal object $\xi = (R, \xi_n, f_n)$ of $\mathcal{X}$
such that $F(\xi) \cong \eta$.
\end{lemma}

\begin{proof}
Note that each of the morphisms
$\Spec(R/\mathfrak m^n) \to \Spec(R/\mathfrak m^{n + 1})$ is a first order
thickening of affine schemes over $S$. Hence the assumption on $F$ means
that we can successively lift $\xi_1$ to objects $\xi_2, \xi_3, \ldots$
of $\mathcal{X}$ endowed with compatible isomorphisms
$\eta_n|_{\Spec(R/\mathfrak m^{n - 1})} \cong \eta_{n - 1}$
and $F(\eta_n) \cong \xi_n$.
\end{proof}

\noindent
Let $S$ be a locally Noetherian scheme. Let
$p : \mathcal{X} \to (\Sch/S)_{fppf}$ be a category fibred in groupoids.
Suppose that $x$ is an object of $\mathcal{X}$ over $R$, where $R$ is a
Noetherian complete local $S$-algebra with residue field of finite type
over $S$. Then we can consider the system of restrictions
$\xi_n = x|_{\Spec(R/\mathfrak m^n)}$ endowed with the natural morphisms
$\xi_1 \to \xi_2 \to \ldots$ coming from transitivity of restriction.
Thus $\xi = (R, \xi_n, \xi_n \to \xi_{n + 1})$ is a formal object of
$\mathcal{X}$. This construction is functorial in the object $x$.
Thus we obtain a functor
\begin{equation}
\label{equation-approximation}
\left\{
\begin{matrix}
\text{objects }x\text{ of }\mathcal{X} \text{ such that }p(x) = \Spec(R) \\
\text{where }R\text{ is Noetherian complete local}\\
\text{with }R/\mathfrak m\text{ of finite type over }S
\end{matrix}
\right\}
\longrightarrow
\left\{
\begin{matrix}
\text{formal objects of }\mathcal{X}
\end{matrix}
\right\}
\end{equation}
To be precise the left hand side is the full subcategory of $\mathcal{X}$
consisting of objects as indicated and the right hand side is the category
of formal objects of $\mathcal{X}$ as in
Definition \ref{definition-formal-objects}.

\begin{definition}
\label{definition-effective}
Let $S$ be a locally Noetherian scheme. Let $\mathcal{X}$ be a category
fibred in groupoids over $(\Sch/S)_{fppf}$. A formal object
$\xi = (R, \xi_n, f_n)$ of $\mathcal{X}$ is called {\it effective}
if it is in the essential image of the functor
(\ref{equation-approximation}).
\end{definition}

\noindent
If the category fibred in groupoids is an algebraic stack, then every
formal object is effective as follows from the next lemma.

\begin{lemma}
\label{lemma-effective}
Let $S$ be a locally Noetherian scheme. Let $\mathcal{X}$ be an algebraic
stack over $S$. The functor (\ref{equation-approximation}) is an equivalence.
\end{lemma}

\begin{proof}
Case I: $\mathcal{X}$ is representable (by a scheme). Say
$\mathcal{X} = (\Sch/X)_{fppf}$ for some scheme $X$ over $S$.
Unwinding the definitions we have to prove the following: Given
a Noetherian complete local $S$-algebra $R$ with $R/\mathfrak m$ of
finite type over $S$ we have
$$
\Mor_S(\Spec(R), X) \longrightarrow \lim \Mor_S(\Spec(R/\mathfrak m^n), X)
$$
is bijective. This follows from Formal Spaces, Lemma
\ref{formal-spaces-lemma-map-into-scheme}.

\medskip\noindent
Case II. $\mathcal{X}$ is representable by an algebraic space. Say
$\mathcal{X}$ is representable by $X$. Again we have to show that
$$
\Mor_S(\Spec(R), X) \longrightarrow \lim \Mor_S(\Spec(R/\mathfrak m^n), X)
$$
is bijective for $R$ as above. This is Formal Spaces, Lemma
\ref{formal-spaces-lemma-map-into-algebraic-space}.

\medskip\noindent
Case III: General case of an algebraic stack. A general remark is that
the left and right hand side of (\ref{equation-approximation}) are
categories fibred in groupoids over the category of affine schemes
over $S$ which are spectra of Noetherian complete local rings
with residue field of finite type over $S$. We will also see in the
proof below that they form stacks for a certain topology on this
category.

\medskip\noindent
We first prove fully faithfulness. Let $R$ be a Noetherian complete
local $S$-algebra with $k = R/\mathfrak m$ of finite type over $S$.
Let $x, x'$ be objects of $\mathcal{X}$ over $R$. As $\mathcal{X}$ is
an algebraic stack $\mathit{Isom}(x, x')$ is representable by an
algebraic space $I$ over $\Spec(R)$, see
Algebraic Stacks, Lemma \ref{algebraic-lemma-representable-diagonal}.
Applying Case II to $I$ over $\Spec(R)$ implies immediately that
(\ref{equation-approximation}) is fully faithful on fibre categories over
$\Spec(R)$. Hence the functor is fully faithful by
Categories, Lemma \ref{categories-lemma-equivalence-fibred-categories}.

\medskip\noindent
Essential surjectivity. Let $\xi = (R, \xi_n, f_n)$ be a formal object of
$\mathcal{X}$. Choose a scheme $U$ over $S$ and a surjective smooth morphism
$f : (\Sch/U)_{fppf} \to \mathcal{X}$. For every $n$ consider the fibre product
$$
(\Sch/\Spec(R/\mathfrak m^n))_{fppf}
\times_{\xi_n, \mathcal{X}, f}
(\Sch/U)_{fppf}
$$
By assumption this is representable by an algebraic space $V_n$ surjective and
smooth over $\Spec(R/\mathfrak m^n)$. The morphisms
$f_n : \xi_n \to \xi_{n + 1}$ induce cartesian squares
$$
\xymatrix{
V_{n + 1} \ar[d] & V_n \ar[d] \ar[l] \\
\Spec(R/\mathfrak m^{n + 1}) & \Spec(R/\mathfrak m^n) \ar[l]
}
$$
of algebraic spaces. By Spaces over Fields, Lemma
\ref{spaces-over-fields-lemma-smooth-separable-closed-points-dense}
we can find a finite separable extension $k \subset k'$ and a point
$v'_1 : \Spec(k') \to V_1$ over $k$. Let $R \subset R'$ be the finite \'etale
extension whose residue field extension is $k \subset k'$ (exists and
is unique by
Algebra, Lemmas \ref{algebra-lemma-henselian-cat-finite-etale} and
\ref{algebra-lemma-complete-henselian}).
By the infinitesimal lifting criterion of smoothness (see
More on Morphisms of Spaces, Lemma
\ref{spaces-more-morphisms-lemma-smooth-formally-smooth})
applied to $V_n \to \Spec(R/\mathfrak m^n)$ for $n = 2, 3, 4, \ldots$
we can successively find morphisms
$v'_n : \Spec(R'/(\mathfrak m')^n) \to V_n$ over $\Spec(R/\mathfrak m^n)$
fitting into commutative diagrams
$$
\xymatrix{
\Spec(R'/(\mathfrak m')^{n + 1}) \ar[d]_{v'_{n + 1}} &
\Spec(R'/(\mathfrak m')^n) \ar[d]^{v'_n} \ar[l] \\
V_{n + 1} & V_n \ar[l]
}
$$
Composing with the projection morphisms $V_n \to U$ we obtain a compatible
system of morphisms $u'_n : \Spec(R'/(\mathfrak m')^n) \to U$.
By Case I the family $(u'_n)$ comes from a unique
morphism $u' : \Spec(R') \to U$. Denote $x'$ the object of $\mathcal{X}$
over $\Spec(R')$ we get by applying the $1$-morphism $f$ to $u'$.
By construction, there exists a morphism of formal objects
$$
(\ref{equation-approximation})(x') =
(R', x'|_{\Spec(R'/(\mathfrak m')^n)}, \ldots)
\longrightarrow
(R, \xi_n, f_n)
$$
lying over $\Spec(R') \to \Spec(R)$. Note that $R' \otimes_R R'$ is a finite
product of spectra of Noetherian complete local rings to which our current
discussion applies. Denote $p_0, p_1 : \Spec(R' \otimes_R R') \to \Spec(R')$
the two projections. By the fully faithfulness shown above there exists
a canonical isomorphism $\varphi : p_0^*x' \to p_1^*x'$ because we have
such isomorphisms over
$\Spec((R' \otimes_R R')/\mathfrak m^n(R' \otimes_R R'))$.
We omit the proof that the isomorphism $\varphi$ satisfies the cocycle
condition (see Stacks, Definition \ref{stacks-definition-descent-data}).
Since $\{\Spec(R') \to \Spec(R)\}$ is an fppf covering we conclude
that $x'$ descends to an object $x$ of $\mathcal{X}$ over $\Spec(R)$.
We omit the proof that $x_n$ is the restriction of $x$ to
$\Spec(R/\mathfrak m^n)$.
\end{proof}

\begin{lemma}
\label{lemma-fibre-product-effective}
Let $S$ be a scheme. Let $p : \mathcal{X} \to \mathcal{Y}$ and
$q : \mathcal{Z} \to \mathcal{Y}$ be $1$-morphisms of categories
fibred in groupoids over $(\Sch/S)_{fppf}$. If the functor
(\ref{equation-approximation}) is an equivalence for 
$\mathcal{X}$, $\mathcal{Y}$, and $\mathcal{Z}$, then it is 
an equivalence for $\mathcal{X} \times_\mathcal{Y} \mathcal{Z}$.
\end{lemma}

\begin{proof}
The left and the right hand side of (\ref{equation-approximation})
for $\mathcal{X} \times_\mathcal{Y} \mathcal{Z}$ are simply the $2$-fibre
products of the left and the right hand side of (\ref{equation-approximation})
for $\mathcal{X}$, $\mathcal{Z}$ over $\mathcal{Y}$.
Hence the result follows as taking $2$-fibre products is compatible
with equivalences of categories, see
Categories, Lemma \ref{categories-lemma-equivalence-2-fibre-product}.
\end{proof}






\section{Approximation}
\label{section-approximation}

\noindent
A fundamental insight of Michael Artin is that you can approximate
objects of a limit preserving stack. Namely, given an object $x$
of the stack over a Noetherian complete local ring, you can find
an object $x_A$ over an algebraic ring which is ``close to'' $x$.
Here an algebraic ring means a finite type $S$-algebra and close
means adically close. In this section we present this in a simple,
yet general form.

\medskip\noindent
To formulate the result we need to pull together some definitions from
different places in the Stacks project. First, in
Criteria for Representability, Section \ref{criteria-section-limit-preserving}
we introduced {\it limit preserving on objects} for $1$-morphisms
of categories fibred in groupoids over the category of schemes.
In More on Algebra, Definition \ref{more-algebra-definition-G-ring}
we defined the notion of a {\it G-ring}. Let $S$ be a locally Noetherian scheme.
Let $A$ be an $S$-algebra. We say that $A$ is {\it of finite type over $S$}
or is a {\it finite type $S$-algebra} if $\Spec(A) \to S$ is of finite type.
In this case $A$ is a Noetherian ring. Finally, given a ring $A$ and ideal
$I$ we denote $\text{Gr}_I(A) = \bigoplus I^n/I^{n + 1}$.

\begin{lemma}
\label{lemma-approximate}
Let $S$ be a locally Noetherian scheme. Let
$p : \mathcal{X} \to (\Sch/S)_{fppf}$ be a category
fibred in groupoids. Let $x$ be an object of
$\mathcal{X}$ lying over $\Spec(R)$ where $R$ is a Noetherian complete
local ring with residue field $k$ of finite type over $S$. Let $s \in S$
be the image of $\Spec(k) \to S$. Assume that (a) $\mathcal{O}_{S, s}$ is
a G-ring and (b) $p$ is limit preserving on objects. Then for every
integer $N \geq 1$ there exist
\begin{enumerate}
\item a finite type $S$-algebra $A$,
\item a maximal ideal $\mathfrak m_A \subset A$,
\item an object $x_A$ of $\mathcal{X}$ over $\Spec(A)$,
\item an $S$-isomorphism $R/\mathfrak m_R^N \cong A/\mathfrak m_A^N$,
\item an isomorphism
$x|_{\Spec(R/\mathfrak m_R^N)} \cong x_A|_{\Spec(A/\mathfrak m_A^N)}$
compatible with (4), and
\item an isomorphism
$\text{Gr}_{\mathfrak m_R}(R) \cong \text{Gr}_{\mathfrak m_A}(A)$
of graded $k$-algebras.
\end{enumerate}
\end{lemma}

\begin{proof}
Choose an affine open $\Spec(\Lambda) \subset S$ such that $k$ is a finite
$\Lambda$-algebra, see
Morphisms, Lemma \ref{morphisms-lemma-point-finite-type}.
We may and do replace $S$ by $\Spec(\Lambda)$.

\medskip\noindent
We may write $R$ as a directed colimit $R = \colim C_j$ where each
$C_j$ is a finite type $\Lambda$-algebra (see
Algebra, Lemma \ref{algebra-lemma-ring-colimit-fp}).
By assumption (b) the object $x$ is isomorphic to the restriction of
an object over one of the $C_j$. Hence we may choose a finite type
$\Lambda$-algebra $C$, a $\Lambda$-algebra map $C \to R$, and an object
$x_C$ of $\mathcal{X}$ over $\Spec(C)$ such that $x = x_C|_{\Spec(R)}$.
The choice of $C$ is a bookkeeping device and could be avoided.
For later use, let us write $C = \Lambda[y_1, \ldots, y_u]/(f_1, \ldots, f_v)$
and we denote $\overline{a}_i \in R$ the image of $y_i$ under the
map $C \to R$. Set $\mathfrak m_C = C \cap \mathfrak m_R$.

\medskip\noindent
Choose a $\Lambda$-algebra surjection $\Lambda[x_1, \ldots, x_s] \to k$
and denote $\mathfrak m'$ the kernel.
By the universal property of polynomial rings we may lift this
to a $\Lambda$-algebra map $\Lambda[x_1, \ldots, x_s] \to R$.
We add some variables (i.e., we increase $s$ a bit) mapping to generators
of $\mathfrak m_R$. Having done this we see that
$\Lambda[x_1, \ldots, x_s] \to R/\mathfrak m_R^2$ is surjective.
Then we see that
\begin{equation}
\label{equation-surjection}
P = \Lambda[x_1, \ldots, x_s]_{\mathfrak m'}^\wedge \longrightarrow R
\end{equation}
is a surjective map of Noetherian complete local rings, see for example
Formal Deformation Theory, Lemma
\ref{formal-defos-lemma-surjective-cotangent-space}.

\medskip\noindent
Choose lifts $a_i \in P$ of $\overline{a}_i$ we found above.
Choose generators $b_1, \ldots, b_r \in P$ for the kernel of
(\ref{equation-surjection}).
Choose $c_{ji} \in P$ such that
$$
f_j(a_1, \ldots, a_u) = \sum c_{ji} b_i
$$
in $P$ which is possible by the choices made so far. Choose generators
$$
k_1, \ldots, k_t \in
\Ker(P^{\oplus r} \xrightarrow{(b_1, \ldots, b_r)} P)
$$
and write $k_i = (k_{i1}, \ldots, k_{ir})$ and $K = (k_{ij})$
so that
$$
P^{\oplus t} \xrightarrow{K}
P^{\oplus r} \xrightarrow{(b_1, \ldots, b_r)}
P \to R \to 0
$$
is an exact sequence of $P$-modules. In particular we have
$\sum k_{ij} b_j = 0$. After possibly increasing $N$ we may
assume $N - 1$ works in the Artin-Rees lemma for the first two maps of this
exact sequence (see More on Algebra, Section
\ref{more-algebra-section-artin-rees} for terminology).

\medskip\noindent
By assumption $\mathcal{O}_{S, s} = \Lambda_{\Lambda \cap \mathfrak m'}$ is
a G-ring. Hence by More on Algebra, Proposition
\ref{more-algebra-proposition-finite-type-over-G-ring}
the ring $\Lambda[x_1, \ldots, x_s]_{\mathfrak m'}$ is a $G$-ring.
Hence by Smoothing Ring Maps, Theorem
\ref{smoothing-theorem-approximation-property-variant}
there exist an \'etale ring map
$$
\Lambda[x_1, \ldots, x_s]_{\mathfrak m'} \to B,
$$
a maximal ideal $\mathfrak m_B$ of $B$ lying over $\mathfrak m'$, and
elements $a'_i, b'_i, c'_{ij}, k'_{ij} \in B'$ such that
\begin{enumerate}
\item $\kappa(\mathfrak m') = \kappa(\mathfrak m_B)$ which implies
that $\Lambda[x_1, \ldots, x_s]_{\mathfrak m'} \subset B_{\mathfrak m_B}
\subset P$ and $P$ is identified with the completion of $B$ at
$\mathfrak m_B$, see remark preceding Smoothing Ring Maps, Theorem
\ref{smoothing-theorem-approximation-property-variant},
\item $a_i - a'_i, b_i - b'_i, c_{ij} - c'_{ij}, k_{ij} - k'_{ij} \in
(\mathfrak m')^N P$, and
\item $f_j(a'_1, \ldots, a'_u) = \sum c'_{ji} b'_i$ and $\sum k'_{ij}b'_j = 0$.
\end{enumerate}
Set $A = B/(b'_1, \ldots, b'_r)$ and denote $\mathfrak m_A$ the
image of $\mathfrak m_B$ in $A$. (Note that $A$ is essentially of finite
type over $\Lambda$; at the end of the proof we will show how to obtain
an $A$ which is of finite type over $\Lambda$.) There is a ring map
$C \to A$ sending $y_i \mapsto a'_i$ because the $a'_i$ satisfy
the desired equations modulo $(b'_1, \ldots, b'_r)$.
Note that $A/\mathfrak m_A^N = R/\mathfrak m_R^N$ as quotients of
$P = B^\wedge$ by property (2) above. Set $x_A = x_C|_{\Spec(A)}$.
Since the maps
$$
C \to A \to A/\mathfrak m_A^N \cong R/\mathfrak m_R^N
\quad\text{and}\quad
C \to R \to R/\mathfrak m_R^N
$$
are equal we see that $x_A$ and $x$ agree modulo $\mathfrak m_R^N$
via the isomorphism $A/\mathfrak m_A^N = R/\mathfrak m_R^N$. At this
point we have shown properties (1) -- (5) of the statement of the lemma.
To see (6) note that
$$
P^{\oplus t} \xrightarrow{K}
P^{\oplus r} \xrightarrow{(b_1, \ldots, b_r)}
P
\quad\text{and}\quad
P^{\oplus t} \xrightarrow{K'}
P^{\oplus r} \xrightarrow{(b'_1, \ldots, b'_r)}
P
$$
are two complexes of $P$-modules which are congruent modulo
$(\mathfrak m')^N$ with the first one being exact. By our choice of $N$
above we see from
More on Algebra, Lemma \ref{more-algebra-lemma-approximate-complex-graded}
that $R = P/(b_1, \ldots, b_r)$ and
$P/(b'_1, \ldots, b'_r) = B^\wedge/(b'_1, \ldots, b'_r) = A^\wedge$
have isomorphic associated graded algebras, which is what we wanted to show.

\medskip\noindent
This last paragraph of the proof serves to clean up the issue that $A$ is
essentially of finite type over $S$ and not yet of finite type.
The construction above gives $A = B/(b'_1, \ldots, b'_r)$ and
$\mathfrak m_A \subset A$ with $B$ \'etale over
$\Lambda[x_1, \ldots, x_s]_{\mathfrak m'}$. Hence $A$ is of finite
type over the Noetherian ring $\Lambda[x_1, \ldots, x_s]_{\mathfrak m'}$.
Thus we can write $A = (A_0)_{\mathfrak m'}$ for some finite type
$\Lambda[x_1, \ldots, x_n]$ algebra $A_0$. Then
$A = \colim (A_0)_f$ where
$f \in \Lambda[x_1, \ldots, x_n] \setminus \mathfrak m'$, see
Algebra, Lemma \ref{algebra-lemma-localization-colimit}.
Because $p : \mathcal{X} \to (\Sch/S)_{fppf}$ is limit preserving on
objects, we see that
$x_A$ comes from some object $x_{(A_0)_f}$ over $\Spec((A_0)_f)$ for
an $f$ as above. After replacing $A$ by $(A_0)_f$ and $x_A$ by
$x_{(A_0)_f}$ and $\mathfrak m_A$ by $(A_0)_f \cap \mathfrak m_A$
the proof is finished.
\end{proof}





\section{Limit preserving}
\label{section-limits}

\noindent
The morphism $p : \mathcal{X} \to (\Sch/S)_{fppf}$ is limit preserving
on objects, as defined in Criteria for Representability, Section
\ref{criteria-section-limit-preserving}, if the functor of the definition
below is essentially surjective. However, the example
in Examples, Section \ref{examples-section-limit-preserving}
shows that this isn't equivalent to being limit preserving.

\begin{definition}
\label{definition-limit-preserving}
Let $S$ be a scheme. Let $\mathcal{X}$ be a category fibred in groupoids
over $(\Sch/S)_{fppf}$. We say $\mathcal{X}$ is {\it limit preserving}
if for every affine scheme $T$ over $S$ which is a limit $T = \lim T_i$
of a directed inverse system of affine schemes $T_i$ over $S$, we have
an equivalence
$$
\colim \mathcal{X}_{T_i} \longrightarrow \mathcal{X}_T
$$
of fibre categories.
\end{definition}

\noindent
We spell out what this means. First, given objects $x, y$ of $\mathcal{X}$
over $T_i$ we should have
$$
\Mor_{\mathcal{X}_T}(x|_T, y|_T) =
\colim_{i' \geq i} \Mor_{\mathcal{X}_{T_i'}}(x|_{T_i'}, y|_{T_i'})
$$
and second every object of $\mathcal{X}_T$ is isomorphic to the restriction
of an object over $T_i$ for some $i$. Note that the first condition means
that the presheaves $\mathit{Isom}_\mathcal{X}(x, y)$ (see
Stacks, Definition \ref{stacks-definition-mor-presheaf})
are limit preserving.

\begin{lemma}
\label{lemma-fibre-product-limit-preserving}
Let $S$ be a scheme. Let $p : \mathcal{X} \to \mathcal{Y}$ and
$q : \mathcal{Z} \to \mathcal{Y}$ be $1$-morphisms of categories
fibred in groupoids over $(\Sch/S)_{fppf}$.
\begin{enumerate}
\item If $\mathcal{X} \to (\Sch/S)_{fppf}$ and
$\mathcal{Z} \to (\Sch/S)_{fppf}$ are limit preserving on objects and
$\mathcal{Y}$ is limit preserving, then
$\mathcal{X} \times_\mathcal{Y} \mathcal{Z} \to (\Sch/S)_{fppf}$ is
limit preserving on objects.
\item If $\mathcal{X}$, $\mathcal{Y}$,
and $\mathcal{Z}$ are limit preserving, then so
is $\mathcal{X} \times_\mathcal{Y} \mathcal{Z}$.
\end{enumerate}
\end{lemma}

\begin{proof}
This is formal. Proof of (1). Let $T = \lim_{i \in I} T_i$ be the directed
limit of affine schemes $T_i$ over $S$. We will prove that the functor
$\colim \mathcal{X}_{T_i} \to \mathcal{X}_T$ is essentially surjective.
Recall that an object of the fibre product over $T$ is a quadruple
$(T, x, z, \alpha)$ where $x$ is an object of $\mathcal{X}$ lying over $T$,
$z$ is an object of $\mathcal{Z}$ lying over $T$, and
$\alpha : p(x) \to q(z)$ is a morphism in the fibre category of
$\mathcal{Y}$ over $T$. By assumption on $\mathcal{X}$ and $\mathcal{Z}$
we can find an $i$ and objects $x_i$ and $z_i$ over $T_i$ such that
$x_i|_T \cong T$ and $z_i|_T \cong z$. Then $\alpha$ corresponds to
an isomorphism $p(x_i)|_T \to q(z_i)|_T$ which comes from an isomorphism
$\alpha_{i'} : p(x_i)|_{T_{i'}} \to q(z_i)|_{T_{i'}}$ by our assumption on
$\mathcal{Y}$. After replacing $i$ by $i'$, $x_i$ by $x_i|_{T_{i'}}$, and
$z_i$ by $z_i|_{T_{i'}}$ we see that $(T_i, x_i, z_i, \alpha_i)$
is an object of the fibre product over $T_i$ which restricts to
an object isomorphic to $(T, x, z, \alpha)$ over $T$ as desired.

\medskip\noindent
We omit the arguments showing that $\colim \mathcal{X}_{T_i} \to \mathcal{X}_T$
is fully faithful in (2).
\end{proof}

\begin{lemma}
\label{lemma-limit-preserving-algebraic-space}
Let $S$ be a scheme. Let $\mathcal{X}$ be an algebraic stack over $S$.
Then the following are equivalent
\begin{enumerate}
\item $\mathcal{X}$ is a stack in setoids and
$\mathcal{X} \to (\Sch/S)_{fppf}$ is limit preserving on objects,
\item $\mathcal{X}$ is a stack in setoids and limit preserving,
\item $\mathcal{X}$ is representable by an algebraic space
locally of finite presentation.
\end{enumerate}
\end{lemma}

\begin{proof}
Under each of the three assumptions $\mathcal{X}$ is representable
by an algebraic space $X$ over $S$, see Algebraic Stacks, Proposition
\ref{algebraic-proposition-algebraic-stack-no-automorphisms}.
It is clear that (1) and (2) are equivalent as a functor between
setoids is an equivalence if and only if it is surjective on isomorphism
classes. Finally, (1) and (3) are equivalent by
Limits of Spaces, Proposition
\ref{spaces-limits-proposition-characterize-locally-finite-presentation}.
\end{proof}

\begin{lemma}
\label{lemma-diagonal}
Let $S$ be a scheme. Let $\mathcal{X}$ be a category fibred
in groupoids over $(\Sch/S)_{fppf}$. Assume
$\Delta : \mathcal{X} \to \mathcal{X} \times \mathcal{X}$ is
representable by algebraic spaces and $\mathcal{X}$ is limit preserving.
Then $\Delta$ is locally of finite type.
\end{lemma}

\begin{proof}
We apply Criteria for Representability, Lemma
\ref{criteria-lemma-check-property-limit-preserving}.
Let $V$ be an affine scheme $V$ of finite type over $S$
and let $\theta$ be an object of $\mathcal{X} \times \mathcal{X}$
over $V$. Let $F_\theta$ be an algebraic space representing
$\mathcal{X} \times_{\Delta, \mathcal{X} \times \mathcal{X}, \theta}
(\Sch/V)_{fppf}$ and let $f_\theta : F_\theta \to V$ be the canonical morphism
(see Algebraic Stacks, Section
\ref{algebraic-section-morphisms-representable-by-algebraic-spaces}).
It suffices to show that
$F_\theta \to V$ has the corresponding properties. By
Lemmas \ref{lemma-fibre-product-limit-preserving} and
\ref{lemma-limit-preserving-algebraic-space}
we see that $F_\theta \to S$ is locally of finite presentation.
It follows that $F_\theta \to V$ is locally of finite type
by Morphisms of Spaces, Lemma
\ref{spaces-morphisms-lemma-permanence-finite-type}.
\end{proof}






\section{Versality}
\label{section-versality}

\noindent
In the previous section we explained how to approximate objects over
complete local rings by algebraic objects. But in order to show that
a stack $\mathcal{X}$ is an algebraic stack, we need to find smooth
$1$-morphisms from schemes towards $\mathcal{X}$. Since we are not going
to assume a priori that $\mathcal{X}$ has a representable diagonal, we
cannot even speak about smooth morphisms towards $\mathcal{X}$. Instead,
borrowing terminology from deformation theory, we will introduce versal
objects.

\begin{definition}
\label{definition-versal-formal-object}
Let $S$ be a locally Noetherian scheme. Let
$p : \mathcal{X} \to (\Sch/S)_{fppf}$ be a category fibred in groupoids.
Let $\xi = (R, \xi_n, f_n)$ be a formal object. Set $k = R/\mathfrak m$ and
$x_0 = \xi_1$. We will say that $\xi$ is {\it versal} if $\xi$
as a formal object of $\mathcal{F}_{\mathcal{X}, k, x_0}$
(Remark \ref{remark-formal-objects-match}) is versal in the sense
of Formal Deformation Theory, Definition \ref{formal-defos-definition-versal}.
\end{definition}

\noindent
We briefly spell out what this means. With notation as in the definition,
suppose given morphisms $\xi_1 = x_0 \to y \to z$ of $\mathcal{X}$ lying over
closed immersions
$\Spec(k) \to \Spec(A) \to \Spec(B)$
where $A, B$ are Artinian local rings with residue field $k$.
Suppose given an $n \geq 1$ and a commutative diagram
$$
\vcenter{
\xymatrix{
& y \ar[ld] \\
\xi_n & \xi_1 \ar[u] \ar[l]
}
}
\quad\text{lying over}\quad
\vcenter{
\xymatrix{
& \Spec(A) \ar[ld] \\
\Spec(R/\mathfrak m^n) & \Spec(k) \ar[u] \ar[l]
}
}
$$
Versality means that for any data as above
there exists an $m \geq n$ and a commutative diagram
$$
\vcenter{
\xymatrix{
& & z \ar[lldd] \\
& & y \ar[ld] \ar[u] \\
\xi_m & \xi_n \ar[l] & \xi_1 \ar[u] \ar[l]
}
}
\quad\text{lying over}
\vcenter{
\xymatrix{
& & \Spec(B) \ar[lldd] \\
& & \Spec(A) \ar[ld] \ar[u] \\
\Spec(R/\mathfrak m^m) &
\Spec(R/\mathfrak m^n) \ar[l] &
\Spec(k) \ar[u] \ar[l]
}
}
$$
Please compare with Formal Deformation Theory, Remark
\ref{formal-defos-remark-versal-object}.

\medskip\noindent
Let $S$ be a locally Noetherian scheme. Let $U$ be a scheme over $S$
with structure morphism $U \to S$ locally of finite type. Let
$u_0 \in U$ be a finite type point of $U$, see
Morphisms, Definition \ref{morphisms-definition-finite-type-point}.
Set $k = \kappa(u_0)$.
Note that the composition $\Spec(k) \to S$ is also of finite type,
see Morphisms, Lemma \ref{morphisms-lemma-composition-finite-type}.
Let $p : \mathcal{X} \to (\Sch/S)_{fppf}$ be a category fibred in groupoids.
Let $x$ be an object of $\mathcal{X}$ which lies over $U$. Denote $x_0$
the pullback of $x$ by $u_0$. By the $2$-Yoneda lemma $x$ corresponds
to a $1$-morphism
$$
x : (\Sch/U)_{fppf} \longrightarrow \mathcal{X},
$$
see Algebraic Stacks, Section \ref{algebraic-section-2-yoneda}. We obtain a
morphism of predeformation categories
\begin{equation}
\label{equation-hat-x}
\hat x :
\mathcal{F}_{(\Sch/U)_{fppf}, k, u_0}
\longrightarrow
\mathcal{F}_{\mathcal{X}, k, x_0},
\end{equation}
over $\mathcal{C}_\Lambda$ see (\ref{equation-functoriality}).

\begin{definition}
\label{definition-versal}
Let $S$ be a locally Noetherian scheme.
Let $\mathcal{X}$ be fibred in groupoids over $(\Sch/S)_{fppf}$.
Let $U$ be a scheme locally of finite type over $S$.
Let $x$ be an object of $\mathcal{X}$ lying over $U$.
Let $u_0$ be finite type point of $U$.
We say $x$ is {\it versal} at $u_0$ if the morphism $\hat x$
(\ref{equation-hat-x}) is smooth, see Formal Deformation Theory, Definition
\ref{formal-defos-definition-smooth-morphism}.
\end{definition}

\noindent
This definition matches our notion of versality for formal objects of
$\mathcal{X}$.

\begin{lemma}
\label{lemma-versality-matches}
With notation as in Definition \ref{definition-versal}.
Let $R = \mathcal{O}_{U, u_0}^\wedge$.
Let $\xi$ be the formal object of $\mathcal{X}$
over $R$ associated to $x|_{\Spec(R)}$, see (\ref{equation-approximation}).
Then
$$
x\text{ is versal at }u_0
\Leftrightarrow
\xi\text{ is versal}
$$
\end{lemma}

\begin{proof}
Observe that $\mathcal{O}_{U, u_0}$ is a Noetherian local $S$-algebra
with residue field $k$. Hence $R = \mathcal{O}_{U, u_0}^\wedge$ is an object of
$\mathcal{C}_\Lambda^\wedge$, see Formal Deformation Theory, Definition
\ref{formal-defos-definition-completion-CLambda}.
Recall that $\xi$ is versal if
$\underline{\xi} : \underline{R}|_{\mathcal{C}_\Lambda} \to
\mathcal{F}_{\mathcal{X}, k, x_0}$
is smooth and $x$ is versal at $u_0$ if
$\hat x : \mathcal{F}_{(\Sch/U)_{fppf}, k, u_0}
\to \mathcal{F}_{\mathcal{X}, k, x_0}$ is smooth.
There is an identification of predeformation categories
$$
\underline{R}|_{\mathcal{C}_\Lambda}
=
\mathcal{F}_{(\Sch/U)_{fppf}, k, u_0},
$$
see Formal Deformation Theory, Remark
\ref{formal-defos-remark-formal-objects-yoneda} for notation.
Namely, given an Artinian local $S$-algebra $A$ with residue field
identified with $k$ we have
$$
\Mor_{\mathcal{C}_\Lambda^\wedge}(R, A) =
\{\varphi \in \Mor_S(\Spec(A), U) \mid \varphi|_{\Spec(k)} = u_0\}
$$
Unwinding the definitions the reader verifies that the resulting map
$$
\underline{R}|_{\mathcal{C}_\Lambda} =
\mathcal{F}_{(\Sch/U)_{fppf}, k, u_0}
\xrightarrow{\hat x}
\mathcal{F}_{\mathcal{X}, k, x_0},
$$
is equal to $\underline{\xi}$ and we see that the lemma is true.
\end{proof}

\noindent
Here is a sanity check.

\begin{lemma}
\label{lemma-versal-implies-smooth}
Let $S$ be a locally Noetherian scheme. Let $f : U \to V$
be a morphism of schemes locally of finite type over $S$.
Let $u_0 \in U$ be a finite type point. The following are equivalent
\begin{enumerate}
\item $f$ is smooth at $u_0$,
\item $f$ viewed as an object of $(\Sch/V)_{fppf}$ over $U$ is
versal at $u_0$.
\end{enumerate}
\end{lemma}

\begin{proof}
This is a restatement of More on Morphisms, Lemma
\ref{more-morphisms-lemma-lifting-along-artinian-at-point}.
\end{proof}

\noindent
It turns out that this notion is well behaved with respect to field
extensions.

\begin{lemma}
\label{lemma-versal-change-of-field}
Let $S$, $\mathcal{X}$, $U$, $x$, $u_0$ be as in
Definition \ref{definition-versal}. Let $l$ be a field and let
$u_{l, 0} : \Spec(l) \to U$ be a morphism with image $u_0$ such that
$l/k = \kappa(u_0)$ is finite. Set $x_{l, 0} = x_0|_{\Spec(l)}$.
If $\mathcal{X}$ satisfies (RS) and $x$ is versal at $u_0$, then
$$
\mathcal{F}_{(\Sch/U)_{fppf}, l, u_{l, 0}}
\longrightarrow
\mathcal{F}_{\mathcal{X}, l, x_{l, 0}}
$$
is smooth.
\end{lemma}

\begin{proof}
Note that $(\Sch/U)_{fppf}$ satisfies (RS) by
Lemma \ref{lemma-algebraic-stack-RS}.
Hence the functor of the lemma is the functor
$$
(\mathcal{F}_{(\Sch/U)_{fppf}, k , u_0})_{l/k}
\longrightarrow
(\mathcal{F}_{\mathcal{X}, k , x_0})_{l/k}
$$
associated to $\hat x$, see Lemma \ref{lemma-change-of-field}.
Hence the lemma follows from
Formal Deformation Theory, Lemma
\ref{formal-defos-lemma-change-of-fields-smooth}.
\end{proof}

\noindent
The following lemma is another sanity check. It more or less
signifies that if $x$ is versal at $u_0$ as in
Definition \ref{definition-versal},
then $x$ viewed as a morphism from $U$ to $\mathcal{X}$ is
smooth whenever we make a base change by a scheme.

\begin{lemma}
\label{lemma-base-change-versal}
Let $S$, $\mathcal{X}$, $U$, $x$, $u_0$ be as in
Definition \ref{definition-versal}. Assume
\begin{enumerate}
\item $\Delta : \mathcal{X} \to \mathcal{X} \times \mathcal{X}$
is representable by algebraic spaces,
\item $\mathcal{X}$ is limit preserving, and
\item $\mathcal{X}$ has (RS).
\end{enumerate}
Let $V$ be a scheme locally of finite type over $S$
and let $y$ be an object of $\mathcal{X}$ over $V$.
Form the $2$-fibre product
$$
\xymatrix{
\mathcal{Z} \ar[r] \ar[d] & (\Sch/U)_{fppf} \ar[d]^x \\
(\Sch/V)_{fppf} \ar[r]^y & \mathcal{X}
}
$$
Let $Z$ be the algebraic space representing $\mathcal{Z}$
and let $z_0 \in |Z|$ be a finite type point lying over $u_0$.
If $x$ is versal at $u_0$, then
the morphism $Z \to V$ is smooth at $z_0$.
\end{lemma}

\begin{proof}
Observe that $Z$ exists by Algebraic Stacks, Lemma
\ref{algebraic-lemma-representable-diagonal}.
By Lemma \ref{lemma-diagonal} we see that
$Z \to V \times_S U$ is locally of finite type.
Choose a scheme $W$, a closed point $w_0 \in W$, and
an \'etale morphism $W \to Z$ mapping $w_0$ to $z_0$, see
Morphisms of Spaces, Definition
\ref{spaces-morphisms-definition-finite-type-point}.
Then $W$ is locally of finite type over $S$ and
$w_0$ is a finite type point of $W$.
Let $l = \kappa(z_0)$. Denote $z_{l, 0}$, $v_{l, 0}$,
$u_{l, 0}$, and $x_{l, 0}$ the objects of
$\mathcal{Z}$, $(\Sch/V)_{fppf}$, $(\Sch/U)_{fppf}$,
and $\mathcal{X}$ over $\Spec(l)$ obtained by pullback to $\Spec(l) = w_0$.
Consider
$$
\xymatrix{
\mathcal{F}_{(\Sch/W)_{fppf}, l, w_0} \ar[r] &
\mathcal{F}_{\mathcal{Z}, l, z_{l, 0}} \ar[d] \ar[r] &
\mathcal{F}_{(\Sch/U)_{fppf}, l, u_{l, 0}} \ar[d] \\
& \mathcal{F}_{(\Sch/V)_{fppf}, l, v_{l, 0}} \ar[r] &
\mathcal{F}_{\mathcal{X}, l, x_{l, 0}}
}
$$
By Lemma \ref{lemma-fibre-product-deformation-categories}
the square is a fibre product of predeformation categories.
By Lemma \ref{lemma-versal-change-of-field}
we see that the right vertical arrow is smooth.
By Formal Deformation Theory, Lemma
\ref{formal-defos-lemma-smooth-properties}
the left vertical arrow is smooth.
By Lemma \ref{lemma-formally-smooth-on-deformation-categories}
we see that the left horizontal arrow is smooth.
We conclude that the map
$$
\mathcal{F}_{(\Sch/W)_{fppf}, l, w_0} \to
\mathcal{F}_{(\Sch/V)_{fppf}, l, v_{l, 0}}
$$
is smooth by Formal Deformation Theory, Lemma
\ref{formal-defos-lemma-smooth-properties}.
Thus we conclude that $W \to V$ is smooth at $w_0$ by
More on Morphisms, Lemma
\ref{more-morphisms-lemma-lifting-along-artinian-at-point}.
This exactly means that $Z \to V$ is smooth at $z_0$
and the proof is complete.
\end{proof}

\noindent
We restate the approximation result in terms of
versal objects.

\begin{lemma}
\label{lemma-approximate-versal}
Let $S$ be a locally Noetherian scheme. Let
$p : \mathcal{X} \to (\Sch/S)_{fppf}$ be a category fibred in groupoids.
Let $\xi = (R, \xi_n, f_n)$ be a formal object of $\mathcal{X}$ with
$\xi_1$ lying over $\Spec(k) \to S$ with image $s \in S$. Assume
\begin{enumerate}
\item $\xi$ is versal,
\item $\xi$ is effective,
\item $\mathcal{O}_{S, s}$ is a G-ring, and
\item $p : \mathcal{X} \to (\Sch/S)_{fppf}$ is limit preserving on objects.
\end{enumerate}
Then there exist a morphism of finite type $U \to S$, a finite type
point $u_0 \in U$ with residue field $k$, and an object $x$ of $\mathcal{X}$
over $U$ such that $x$ is versal at $u_0$ and such that
$x|_{\Spec(\mathcal{O}_{U, u_0}/\mathfrak m_{u_0}^n)} \cong \xi_n$.
\end{lemma}

\begin{proof}
Choose an object $x_R$ of $\mathcal{X}$ lying over $\Spec(R)$ whose associated
formal object is $\xi$. Let $N = 2$ and apply Lemma \ref{lemma-approximate}.
We obtain $A, \mathfrak m_A, x_A, \ldots$.
Let $\eta = (A^\wedge, \eta_n, g_n)$ be the formal object associated to
$x_A|_{\Spec(A^\wedge)}$. We have a diagram
$$
\vcenter{
\xymatrix{
& \eta \ar[d] \\
\xi \ar[r] \ar@{..>}[ru] & \xi_2 = \eta_2
}
}
\quad\text{lying over}\quad
\vcenter{
\xymatrix{
& A^\wedge \ar[d] \\
R \ar[r] \ar@{..>}[ru] & R/\mathfrak m_R^2 = A/\mathfrak m_A^2
}
}
$$
The versality of $\xi$ means exactly that we can find the
dotted arrows in the diagrams, because we can successively find
morphisms $\xi \to \eta_3$, $\xi \to \eta_4$, and so on by
Formal Deformation Theory, Remark \ref{formal-defos-remark-versal-object}.
The corresponding ring map $R \to A^\wedge$ is surjective by
Formal Deformation Theory, Lemma
\ref{formal-defos-lemma-surjective-cotangent-space}.
On the other hand, we have
$\dim_k \mathfrak m_R^n/\mathfrak m_R^{n + 1} =
\dim_k \mathfrak m_A^n/\mathfrak m_A^{n + 1}$ for all $n$ by construction.
Hence $R/\mathfrak m_R^n$ and $A/\mathfrak m_A^n$ have the same (finite)
length as $\Lambda$-modules by additivity of length and
Formal Deformation Theory, Lemma \ref{formal-defos-lemma-length}.
It follows that $R/\mathfrak m_R^n \to A/\mathfrak m_A^n$ is an isomorphism
for all $n$, hence $R \to A^\wedge$ is an isomorphism. Thus $\eta$ is
isomorphic to a versal object, hence versal itself. By
Lemma \ref{lemma-versality-matches}
we conclude that $x_A$ is versal at the point $u_0$ of
$U = \Spec(A)$ corresponding to $\mathfrak m_A$.
\end{proof}

\begin{example}
\label{example-approximate-versal-implies}
In this example we show that the local ring $\mathcal{O}_{S, s}$ has to be
a G-ring in order for the result of Lemma \ref{lemma-approximate-versal} to
be true. Namely, let $\Lambda$ be a Noetherian ring and let $\mathfrak m$
be a maximal ideal of $\Lambda$. Set $R = \Lambda_\mathfrak m^\wedge$. Let
$\Lambda \to C \to R$ be a factorization with $C$ of finite type over
$\Lambda$. Set $S = \Spec(\Lambda)$, $U = S \setminus \{\mathfrak m\}$, and
$S' = U \amalg \Spec(C)$. Consider the functor
$F : (\Sch/S)_{fppf}^{opp} \to \textit{Sets}$ defined by the rule
$$
F(T) = 
\left\{
\begin{matrix}
* & \text{if }T \to S\text{ factors through }S' \\
\emptyset & \text{else}
\end{matrix}
\right.
$$
Let $\mathcal{X} = \mathcal{S}_F$ is the category fibred in sets associated
to $F$, see Algebraic Stacks, Section \ref{algebraic-section-split}.
Then $\mathcal{X} \to (\Sch/S)_{fppf}$ is limit preserving on objects and
there exists an effective, versal formal object $\xi$ over $R$.
Hence if the conclusion of Lemma \ref{lemma-approximate-versal} holds
for $\mathcal{X}$, then there exists a finite type ring map $\Lambda \to A$
and a maximal ideal $\mathfrak m_A$ lying over $\mathfrak m$ such that
\begin{enumerate}
\item $\kappa(\mathfrak m) = \kappa(\mathfrak m_A)$,
\item $\Lambda \to A$ and $\mathfrak m_A$ satisfy condition (4) of
Algebra, Lemma \ref{algebra-lemma-smooth-test-artinian}, and
\item there exists a $\Lambda$-algebra map $C \to A$.
\end{enumerate}
Thus $\Lambda \to A$ is smooth at $\mathfrak m_A$ by the lemma cited.
Slicing $A$ we may assume that $\Lambda \to A$ is \'etale at
$\mathfrak m_A$, see for example
More on Morphisms, Lemma \ref{more-morphisms-lemma-slice-smooth}
or argue directly. Write $C = \Lambda[y_1, \ldots, y_n]/(f_1, \ldots, f_m)$.
Then $C \to R$ corresponds to a solution in $R$ of the system of equations
$f_1 = \ldots = f_m = 0$, see Smoothing Ring Maps, Section
\ref{smoothing-section-approximation-G-rings}.
Thus if the conclusion of
Lemma \ref{lemma-approximate-versal} holds for every $\mathcal{X}$ as
above, then a system of equations which has a solution in $R$ has a
solution in the henselization of $\Lambda_{\mathfrak m}$.
In other words, the approximation property holds for
$\Lambda_{\mathfrak m}^h$. This implies that $\Lambda_{\mathfrak m}^h$
is a G-ring (insert future reference here; see also discussion in
Smoothing Ring Maps, Section \ref{smoothing-section-introduction})
which in turn implies that $\Lambda_{\mathfrak m}$ is a G-ring.
\end{example}






\section{Openness of versality}
\label{section-openness-versality}

\noindent
Next, we come to openness of versality.

\begin{definition}
\label{definition-openness-versality}
Let $S$ be a locally Noetherian scheme.
\begin{enumerate}
\item Let $\mathcal{X}$ be a category
fibred in groupoids over $(\Sch/S)_{fppf}$. We say $\mathcal{X}$ satisfies
{\it openness of versality} if given a scheme $U$ locally of finite type
over $S$, an object $x$ of $\mathcal{X}$ over $U$, and a finite type point
$u_0 \in U$ such that $x$ is versal at $u_0$, then there exists an open
neighbourhood $u_0 \in U' \subset U$ such that $x$ is versal at every finite
type point of $U'$.
\item Let $f : \mathcal{X} \to \mathcal{Y}$ be a $1$-morphism of categories
fibred in groupoids over $(\Sch/S)_{fppf}$. We say $f$ satisfies
{\it openness of versality} if given a scheme $U$ locally of finite type
over $S$, an object $y$ of $\mathcal{Y}$ over $U$, openness
of versality holds for
$(\Sch/U)_{fppf} \times_\mathcal{Y} \mathcal{X}$.
\end{enumerate}
\end{definition}

\noindent
Openness of versality is often the hardest to check. The following example
shows that requiring this is necessary however.

\begin{example}
\label{example-versality}
Let $k$ be a field and set $\Lambda = k[s, t]$. Consider the functor
$F : \Lambda\text{-algebras} \longrightarrow \textit{Sets}$
defined by the rule
$$
F(A) =
\left\{
\begin{matrix}
* & \text{if there exist }f_1, \ldots, f_n \in A\text{ such that } \\
  & A = (s, t, f_1, \ldots, f_n)\text{ and } f_i s = 0\ \forall i \\
\emptyset & \text{else}
\end{matrix}
\right.
$$
Geometrically $F(A) = *$ means there exists a quasi-compact open neighbourhood
$W$ of $V(s, t) \subset \Spec(A)$ such that $s|_W = 0$.
Let $\mathcal{X} \subset (\Sch/\Spec(\Lambda))_{fppf}$ be the full
subcategory consisting of schemes $T$ which have an affine open covering
$T = \bigcup \Spec(A_j)$ with $F(A_j) = *$ for all $j$. Then $\mathcal{X}$
satisfies [0], [1], [2], [3], and [4] but not [5]. Namely, over
$U = \Spec(k[s, t]/(s))$
there exists an object $x$ which is versal at $u_0 = (s, t)$ but not
at any other point. Details omitted.
\end{example}

\noindent
Let $S$ be a locally Noetherian scheme.
Let $f : \mathcal{X} \to \mathcal{Y}$ be a $1$-morphism of categories
fibred in groupoids over $(\Sch/S)_{fppf}$. Consider the following property
\begin{equation}
\label{equation-smooth}
\begin{matrix}
\text{for all fields }k\text{ of finite type over }S
\text{ and all }x_0 \in \Ob(\mathcal{X}_{\Spec(k)})\text{ the}\\
\text{map }
\mathcal{F}_{\mathcal{X}, k, x_0} \to \mathcal{F}_{\mathcal{Y}, k, f(x_0)}
\text{ of predeformation categories is smooth}
\end{matrix}
\end{equation}
We formulate some lemmas around this concept. First we link it with
(openness of) versality.

\begin{lemma}
\label{lemma-versal-smooth}
Let $S$ be a locally Noetherian scheme. Let $\mathcal{X}$ be a category
fibred in groupoids over $(\Sch/S)_{fppf}$. Let $U$ be a scheme locally
of finite type over $S$. Let $x$ be an object of $\mathcal{X}$ over $U$.
Assume that $x$ is versal at every finite type point of $U$ and that
$\mathcal{X}$ satisfies (RS). Then $x : (\Sch/U)_{fppf} \to \mathcal{X}$
satisfies (\ref{equation-smooth}).
\end{lemma}

\begin{proof}
Let $\Spec(l) \to U$ be a morphism with $l$ of finite type over $S$.
Then the image $u_0 \in U$ is a finite type point of $U$ and
$\kappa(u_0) \subset l$ is a finite extension, see discussion in
Morphisms, Section \ref{morphisms-section-points-finite-type}.
Hence we see that
$\mathcal{F}_{(\Sch/U)_{fppf}, l, u_{l, 0}} \to
\mathcal{F}_{\mathcal{X}, l, x_{l, 0}}$
is smooth by Lemma \ref{lemma-versal-change-of-field}.
\end{proof}

\begin{lemma}
\label{lemma-composition-smooth}
Let $S$ be a locally Noetherian scheme. Let $f : \mathcal{X} \to \mathcal{Y}$
and $g : \mathcal{Y} \to \mathcal{Z}$ be composable $1$-morphisms of
categories fibred in groupoids over $(\Sch/S)_{fppf}$. If $f$ and $g$
satisfy (\ref{equation-smooth}) so does $g \circ f$.
\end{lemma}

\begin{proof}
This follows formally from Formal Deformation Theory, Lemma
\ref{formal-defos-lemma-smooth-properties}.
\end{proof}

\begin{lemma}
\label{lemma-base-change-smooth}
Let $S$ be a locally Noetherian scheme. Let $f : \mathcal{X} \to \mathcal{Y}$
and $\mathcal{Z} \to \mathcal{Y}$ be $1$-morphisms of
categories fibred in groupoids over $(\Sch/S)_{fppf}$. If $f$
satisfies (\ref{equation-smooth}) so does the projection
$\mathcal{X} \times_\mathcal{Y} \mathcal{Z} \to \mathcal{Z}$.
\end{lemma}

\begin{proof}
Follows immediately from
Lemma \ref{lemma-fibre-product-deformation-categories}
and
Formal Deformation Theory, Lemma
\ref{formal-defos-lemma-smooth-properties}.
\end{proof}

\begin{lemma}
\label{lemma-smooth-smooth}
Let $S$ be a locally Noetherian scheme. Let $f : \mathcal{X} \to \mathcal{Y}$
be a $1$-morphisms of categories fibred in groupoids over $(\Sch/S)_{fppf}$.
If $f$ is formally smooth on objects, then $f$ satisfies
(\ref{equation-smooth}). If $f$ is representable by algebraic spaces
and smooth, then $f$ satisfies (\ref{equation-smooth}).
\end{lemma}

\begin{proof}
A reformulation of Lemma \ref{lemma-formally-smooth-on-deformation-categories}.
\end{proof}

\begin{lemma}
\label{lemma-implies-smooth}
Let $S$ be a locally Noetherian scheme. Let $f : \mathcal{X} \to \mathcal{Y}$
be a $1$-morphism of categories fibred in groupoids over $(\Sch/S)_{fppf}$.
Assume
\begin{enumerate}
\item $f$ is representable by algebraic spaces,
\item $f$ satisfies (\ref{equation-smooth}),
\item $\mathcal{X} \to (\Sch/S)_{fppf}$ is limit preserving on objects, and
\item $\mathcal{Y}$ is limit preserving.
\end{enumerate}
Then $f$ is smooth.
\end{lemma}

\begin{proof}
The key ingredient of the proof is More on Morphisms, Lemma
\ref{more-morphisms-lemma-lifting-along-artinian-at-point}
which (almost) says that a morphism of schemes of finite type over $S$
satisfying (\ref{equation-smooth}) is a smooth morphism. The other
arguments of the proof are essentially bookkeeping.

\medskip\noindent
Let $V$ be a scheme over $S$ and let $y$ be an object of $\mathcal{Y}$ over
$V$. Let $Z$ be an algebraic space representing the $2$-fibre product
$\mathcal{Z} = \mathcal{X} \times_{f, \mathcal{X}, y} (\Sch/V)_{fppf}$.
We have to show that the projection morphism $Z \to V$ is smooth, see
Algebraic Stacks, Definition
\ref{algebraic-definition-relative-representable-property}.
In fact, it suffices to do this when $V$ is an affine scheme
locally of finite presentation over $S$, see
Criteria for Representability, Lemma
\ref{criteria-lemma-check-property-limit-preserving}.
Then $(\Sch/V)_{fppf}$ is limit preserving by
Lemma \ref{lemma-limit-preserving-algebraic-space}.
Hence $Z \to S$ is locally of finite presentation by
Lemmas \ref{lemma-fibre-product-limit-preserving} and
\ref{lemma-limit-preserving-algebraic-space}.
Choose a scheme $W$ and a surjective \'etale morphism $W \to Z$.
Then $W$ is locally of finite presentation over $S$.

\medskip\noindent
Since $f$ satisfies (\ref{equation-smooth}) we see that so does
$\mathcal{Z} \to (\Sch/V)_{fppf}$, see Lemma \ref{lemma-base-change-smooth}.
Next, we see that $(\Sch/W)_{fppf} \to \mathcal{Z}$ satisfies
(\ref{equation-smooth}) by Lemma \ref{lemma-smooth-smooth}.
Thus the composition
$$
(\Sch/W)_{fppf} \to \mathcal{Z} \to (\Sch/V)_{fppf}
$$
satisfies (\ref{equation-smooth}) by Lemma \ref{lemma-composition-smooth}.
More on Morphisms, Lemma
\ref{more-morphisms-lemma-lifting-along-artinian-at-point}
shows that the composition $W \to Z \to V$ is smooth at every finite type
point $w_0$ of $W$. Since the smooth locus is open we conclude
that $W \to V$ is a smooth morphism of schemes by
Morphisms, Lemma \ref{morphisms-lemma-enough-finite-type-points}.
Thus we conclude that $Z \to V$ is a smooth morphism
of algebraic spaces by definition.
\end{proof}

\noindent
The lemma below is how we will use openness of versality.

\begin{lemma}
\label{lemma-get-smooth}
Let $S$ be a locally Noetherian scheme. Let
$p : \mathcal{X} \to (\Sch/S)_{fppf}$ be a category fibred in groupoids.
Let $k$ be a finite type field over $S$ and let $x_0$ be an object of
$\mathcal{X}$ over $\Spec(k)$ with image $s \in S$. Assume
\begin{enumerate}
\item $\Delta : \mathcal{X} \to \mathcal{X} \times \mathcal{X}$ is
representable by algebraic spaces,
\item $\mathcal{X}$ satisfies axioms [1], [2], [3] (see
Section \ref{section-axioms}),
\item every formal object of $\mathcal{X}$ is effective,
\item openness of versality holds for $\mathcal{X}$, and
\item $\mathcal{O}_{S, s}$ is a G-ring.
\end{enumerate}
Then there exist a morphism of finite type $U \to S$ and an object
$x$ of $\mathcal{X}$ over $U$ such that
$$
x : (\Sch/U)_{fppf} \longrightarrow \mathcal{X}
$$
is smooth and such that there exists a finite type point $u_0 \in U$
whose residue field is $k$ and such that $x|_{u_0} \cong x_0$.
\end{lemma}

\begin{proof}
By axiom [2], Lemma \ref{lemma-deformation-category}, and
Remark \ref{remark-deformation-category-implies}
we see that $\mathcal{F}_{\mathcal{X}, k, x_0}$ satisfies (S1) and (S2).
Since also the tangent space has finite dimension by axiom [3]
we deduce from Formal Deformation Theory, Lemma
\ref{formal-defos-lemma-versal-object-existence}
that $\mathcal{F}_{\mathcal{X}, k, x_0}$ has a versal formal object $\xi$.
Assumption (3) says $\xi$ is effective. By axiom [1] and
Lemma \ref{lemma-approximate-versal}
there exists a morphism of finite type $U \to S$, an object $x$ of
$\mathcal{X}$ over $U$, and a finite type point $u_0$ of $U$ with residue
field $k$ such that $x$ is versal at $u_0$ and such that
$x|_{\Spec(k)} \cong x_0$. By openness of versality we may shrink
$U$ and assume that $x$ is versal at every finite type point of $U$.
We claim that
$$
x : (\Sch/U)_{fppf} \longrightarrow \mathcal{X}
$$
is smooth which proves the lemma. Namely, by Lemma \ref{lemma-versal-smooth}
$x$ satisfies (\ref{equation-smooth})
whereupon Lemma \ref{lemma-implies-smooth}
finishes the proof.
\end{proof}






\section{Axioms}
\label{section-axioms}

\noindent
Let $S$ be a locally Noetherian scheme. Let
$p : \mathcal{X} \to (\Sch/S)_{fppf}$ be a category fibred in groupoids.
Here are the axioms we will consider on $\mathcal{X}$.
\begin{enumerate}
\item[{[-1]}] a set theoretic condition\footnote{The condition is the
following: the supremum of all the cardinalities
$|\Ob(\mathcal{X}_{\Spec(k)})/\cong|$ and
$|\text{Arrows}(\mathcal{X}_{\Spec(k)})|$ where $k$ runs over the finite
type fields over $S$ is $\leq$ than the size of some
object of $(\Sch/S)_{fppf}$.} to be ignored by
readers who are not interested in set theoretical issues,
\item[{[0]}] $\mathcal{X}$ is a stack in groupoids for the \'etale topology,
\item[{[1]}] $\mathcal{X}$ is limit preserving,
\item[{[2]}] $\mathcal{X}$ satisfies the Rim-Schlessinger condition (RS),
\item[{[3]}] the spaces $T\mathcal{F}_{\mathcal{X}, k, x_0}$ and
$\text{Inf}(\mathcal{F}_{\mathcal{X}, k, x_0})$
are finite dimensional
for every $k$ and $x_0$, see
(\ref{equation-tangent-space}) and
(\ref{equation-infinitesimal-automorphisms}),
\item[{[4]}] the functor (\ref{equation-approximation}) is an equivalence,
\item[{[5]}] $\mathcal{X}$ and
$\Delta : \mathcal{X} \to \mathcal{X} \times \mathcal{X}$ satisfy
openness of versality.
\end{enumerate}










\section{Axioms for functors}
\label{section-axioms-functors}

\noindent
Let $S$ be a scheme. Let $F : (\Sch/S)_{fppf}^{opp} \to \textit{Sets}$ be a
functor. Denote $\mathcal{X} = \mathcal{S}_F$ the category fibred in sets
associated to $F$, see Algebraic Stacks, Section \ref{algebraic-section-split}.
In this section we provide a translation between the material above
as it applies to $\mathcal{X}$, to statements about $F$.

\medskip\noindent
Let $S$ be a locally Noetherian scheme. Let
$F : (\Sch/S)_{fppf}^{opp} \to \textit{Sets}$ be a functor. Let $k$ be
a field of finite type over $S$. Let $x_0 \in F(\Spec(k))$.
The associated predeformation category (\ref{equation-predeformation-category})
corresponds to the functor
$$
F_{k, x_0} : \mathcal{C}_\Lambda \longrightarrow \textit{Sets},
\quad
A \longmapsto \{x \in F(\Spec(A)) \mid x|_{\Spec(k)} = x_0 \}.
$$
Recall that we do not distinguish between
categories cofibred in sets over $\mathcal{C}_\Lambda$
and functor $\mathcal{C}_\Lambda \to \textit{Sets}$,
see Formal Deformation Theory, Remarks
\ref{formal-defos-remarks-cofibered-groupoids}
(\ref{formal-defos-item-convention-cofibered-sets}).
Given a transformation of functors $a : F \to G$, setting
$y_0 = a(x_0)$ we obtain a morphism
$$
F_{k, x_0} \longrightarrow G_{k, y_0}
$$
see (\ref{equation-functoriality}).
Lemma \ref{lemma-formally-smooth-on-deformation-categories} tells us that if
$a : F \to G$ is formally smooth (in the sense of
More on Morphisms of Spaces, Definition
\ref{spaces-more-morphisms-definition-formally-smooth-etale-unramified}), then
$F_{k, x_0} \longrightarrow G_{k, y_0}$ is smooth as
in Formal Deformation Theory, Remark
\ref{formal-defos-remark-compare-smooth-schlessinger}.

\medskip\noindent
Lemma \ref{lemma-pushout} says that if $Y' = Y \amalg_X X'$ in the
category of schemes over $S$ where $X \to X'$ is a thickening and
$X \to Y$ is affine, then the map
$$
F(Y \amalg_X X') \to F(Y) \times_{F(X)} F(X')
$$
is a bijection, provided that $F$ is an algebraic space.
We say a general functor $F$ satisfies the {\it Rim-Schlessinger condition}
or we say $F$ {\it satisfies (RS)} if given any
pushout $Y' = Y \amalg_X X'$ where $Y, X, X'$ are spectra of Artinian
local rings of finite type over $S$, then
$$
F(Y \amalg_X X') \to F(Y) \times_{F(X)} F(X')
$$
is a bijection. Thus every algebraic space satisfies (RS).

\medskip\noindent
Lemma \ref{lemma-deformation-category} says that
given a functor $F$ which satisfies (RS), then all $F_{k, x_0}$
are deformation functors as in
Formal Deformation Theory, Definition
\ref{formal-defos-definition-deformation-category}, i.e., they satisfy
(RS) as in
Formal Deformation Theory, Remark
\ref{formal-defos-remark-compare-schlessinger-H4}.
In particular the tangent space
$$
TF_{k, x_0} = \{x \in F(\Spec(k[\epsilon])) \mid x|_{\Spec(k)} = x_0\}
$$
has the structure of a $k$-vector space by Formal Deformation Theory,
Lemma \ref{formal-defos-lemma-tangent-space-vector-space}.

\medskip\noindent
Lemma \ref{lemma-finite-dimension} says that an algebraic space $F$
locally of finite type over $S$ gives rise to deformation functors
$F_{k, x_0}$ with finite dimensional tangent spaces $TF_{k, x_0}$.

\medskip\noindent
A {\it formal object}\footnote{This is what Artin calls a formal deformation.}
$\xi = (R, \xi_n)$ of $F$ consists of a Noetherian
complete local $S$-algebra $R$ whose residue field is of finite type
over $S$, together with elements $\xi_n \in F(\Spec(R/\mathfrak m^n))$
such that $\xi_{n + 1}|_{\Spec(R/\mathfrak m^n)} = \xi_n$. A formal
object $\xi$ defines a formal object $\xi$ of $F_{R/\mathfrak m, \xi_1}$.
We say $\xi$ is {\it versal} if and only if it is versal in the sense of
Formal Deformation Theory, Definition \ref{formal-defos-definition-versal}.
A formal object $\xi = (R, \xi_n)$ is called {\it effective}
if there exists an $x \in F(\Spec(R))$ such that
$\xi_n = x|_{\Spec(R/\mathfrak m^n)}$ for all $n \geq 1$.
Lemma \ref{lemma-effective} says that if $F$ is an algebraic space,
then every formal object is effective.

\medskip\noindent
Let $U$ be a scheme locally of finite type over $S$ and let $x \in F(U)$.
Let $u_0 \in U$ be a finite type point. We say that $x$ is versal at $u_0$
if and only if
$\xi = (\mathcal{O}_{U, u_0}^\wedge,
x|_{\Spec(\mathcal{O}_{U, u_0}/\mathfrak m_{u_0}^n)})$
is a versal formal object in the sense described above.

\medskip\noindent
Let $S$ be a locally Noetherian scheme. Let
$F : (\Sch/S)_{fppf}^{opp} \to \Sch$ be a functor.
Here are the axioms we will consider on $F$.
\begin{enumerate}
\item[{[-1]}]  a set theoretic condition\footnote{The condition is the
following: the supremum of all the cardinalities
$|F(\Spec(k))|$ where $k$ runs over the finite
type fields over $S$ is $\leq$ than the size of some
object of $(\Sch/S)_{fppf}$.} to be ignored by
readers who are not interested in set theoretical issues,
\item[{[0]}] $F$ is a sheaf for the \'etale topology,
\item[{[1]}] $F$ is limit preserving,
\item[{[2]}] $F$ satisfies the Rim-Schlessinger condition (RS),
\item[{[3]}] every tangent space $TF_{k, x_0}$ is finite dimensional,
\item[{[4]}] every formal object is effective,
\item[{[5]}] $F$ satisfies openness of versality.
\end{enumerate}
Here {\it limit preserving} is the notion defined in
Limits of Spaces, Definition
\ref{spaces-limits-definition-locally-finite-presentation} and
{\it openness of versality} means the following: Given a scheme $U$
locally of finite type over $S$, given $x \in F(U)$, and given
a finite type point $u_0 \in U$ such that $x$ is versal at $u_0$,
then there exists an open neighbourhood $u_0 \in U' \subset U$
such that $x$ is versal at every finite type point of $U'$.






\section{Algebraic spaces}
\label{section-algebraic-spaces}

\noindent
The following is our first main result on algebraic spaces.

\begin{proposition}
\label{proposition-spaces-diagonal-representable}
Let $S$ be a locally Noetherian scheme. Let
$F : (\Sch/S)_{fppf}^{opp} \to \textit{Sets}$ be a functor. Assume that
\begin{enumerate}
\item $\Delta : F \to F \times F$ is representable by algebraic spaces,
\item $F$ satisfies axioms [-1], [0], [1], [2], [3], [4], [5]
(see Section \ref{section-axioms-functors}), and
\item $\mathcal{O}_{S, s}$ is a G-ring for all finite type points $s$ of $S$.
\end{enumerate}
Then $F$ is an algebraic space.
\end{proposition}

\begin{proof}
Lemma \ref{lemma-get-smooth} applies to $F$. Using this we
choose, for every finite type field $k$ over $S$ and $x_0 \in F(\Spec(k))$,
an affine scheme $U_{k, x_0}$ of finite type over $S$ and a smooth morphism
$U_{k, x_0} \to F$ such that there exists a finite type point
$u_{k, x_0} \in U_{k, x_0}$ with residue field $k$ such that $x_0$
is the image of $u_{k, x_0}$. Then
$$
U = \coprod\nolimits_{k, x_0} U_{k, x_0} \longrightarrow F
$$
is smooth\footnote{Set theoretical remark: This coproduct is (isomorphic)
to an object of $(\Sch/S)_{fppf}$ as we have a bound on the index set
by axiom [-1], see Sets, Lemma \ref{sets-lemma-what-is-in-it}.}.
To finish the proof it suffices to show this map is surjective,
see Bootstrap, Lemma \ref{bootstrap-lemma-spaces-etale-smooth-cover}
(this is where we use axiom [0]). By Criteria for Representability, Lemma
\ref{criteria-lemma-check-property-limit-preserving}
it suffices to show that $U \times_F V \to V$ is surjective for those
$V \to F$ where $V$ is an affine scheme locally of finite presentation
over $S$. Since $U \times_F V \to V$ is smooth the image is open. Hence
it suffices to show that the image of $U \times_F V \to V$ contains all
finite type points of $V$, see
Morphisms, Lemma \ref{morphisms-lemma-enough-finite-type-points}.
Let $v_0 \in V$ be a finite type point. Then $k = \kappa(v_0)$ is
a finite type field over $S$. Denote $x_0$ the composition
$\Spec(k) \xrightarrow{v_0} V \to F$. Then
$(u_{k, x_0}, v_0) : \Spec(k) \to U \times_F V$ is a point mapping to
$v_0$ and we win.
\end{proof}

\begin{lemma}
\label{lemma-monomorphism}
Let $S$ be a locally Noetherian scheme. Let $a : F \to G$ be a transformation
of functors $(\Sch/S)_{fppf}^{opp} \to \textit{Sets}$.
Assume that
\begin{enumerate}
\item $a$ is injective,
\item $F$ satisfies axioms [0], [1], [2], [4], and [5],
\item $\mathcal{O}_{S, s}$ is a G-ring for all finite type points $s$ of $S$,
\item $G$ is an algebraic space locally of finite type over $S$,
\end{enumerate}
Then $F$ is an algebraic space.
\end{lemma}

\begin{proof}
By Lemma \ref{lemma-finite-dimension} the functor $G$ satisfies [3].
As $F \to G$ is injective, we conclude that $F$ also satisfies [3].
Moreover, as $F \to G$ is injective, we see that given schemes
$U$, $V$ and morphisms $U \to F$ and $V \to F$, then
$U \times_F V = U \times_G V$. Hence $\Delta : F \to F \times F$ is
representable (by schemes) as this holds for $G$ by assumption.
Thus Proposition \ref{proposition-spaces-diagonal-representable}
applies\footnote{The set
theoretic condition [-1] holds for $F$ as it holds for $G$. Details
omitted.}.
\end{proof}












\section{Algebraic stacks}
\label{section-algebraic-stacks}

\noindent
Proposition \ref{proposition-second-diagonal-representable} is our first
main result on algebraic stacks.

\begin{lemma}
\label{lemma-diagonal-representable}
Let $S$ be a locally Noetherian scheme. Let
$p : \mathcal{X} \to (\Sch/S)_{fppf}$ be a category fibred in groupoids.
Assume that
\begin{enumerate}
\item $\Delta : \mathcal{X} \to \mathcal{X} \times \mathcal{X}$
is representable by algebraic spaces,
\item $\mathcal{X}$ satisfies axioms [-1], [0], [1], [2], [3] (see
Section \ref{section-axioms}),
\item every formal object of $\mathcal{X}$ is effective,
\item $\mathcal{X}$ satisfies openness of versality, and
\item $\mathcal{O}_{S, s}$ is a G-ring for all finite type points $s$ of $S$.
\end{enumerate}
Then $\mathcal{X}$ is an algebraic stack.
\end{lemma}

\begin{proof}
Lemma \ref{lemma-get-smooth} applies to $\mathcal{X}$. Using this we
choose, for every finite type field $k$ over $S$ and every
isomorphism class of object $x_0 \in \Ob(\mathcal{X}_{\Spec(k)})$,
an affine scheme $U_{k, x_0}$ of finite type over $S$ and a smooth morphism
$(\Sch/U_{k, x_0})_{fppf} \to \mathcal{X}$ such that there exists a finite
type point $u_{k, x_0} \in U_{k, x_0}$ with residue field $k$ such that $x_0$
is the image of $u_{k, x_0}$. Then
$$
(\Sch/U)_{fppf} \to \mathcal{X},
\quad\text{with}\quad
U = \coprod\nolimits_{k, x_0} U_{k, x_0}
$$
is smooth\footnote{Set theoretical remark: This coproduct is (isomorphic to)
an object of $(\Sch/S)_{fppf}$ as we have a bound on the index set
by axiom [-1], see Sets, Lemma \ref{sets-lemma-what-is-in-it}.}.
To finish the proof it suffices to show this map is surjective,
see Criteria for Representability, Lemma \ref{criteria-lemma-stacks-etale}
(this is where we use axiom [0]). By Criteria for Representability, Lemma
\ref{criteria-lemma-check-property-limit-preserving}
it suffices to show that
$(\Sch/U)_{fppf} \times_\mathcal{X} (\Sch/V)_{fppf} \to (\Sch/V)_{fppf}$
is surjective for those $y : (\Sch/V)_{fppf} \to \mathcal{X}$ where
$V$ is an affine scheme locally of finite presentation
over $S$. By assumption (1) the fibre product
$(\Sch/U)_{fppf} \times_\mathcal{X} (\Sch/V)_{fppf}$ is representable
by an algebraic space $W$. Then $W \to V$ is smooth, hence the image is
open. Hence it suffices to show that the image of $W \to V$ contains all
finite type points of $V$, see
Morphisms, Lemma \ref{morphisms-lemma-enough-finite-type-points}.
Let $v_0 \in V$ be a finite type point. Then $k = \kappa(v_0)$ is
a finite type field over $S$. Denote $x_0 = y|_{\Spec(k)}$
the pullback of $y$ by $v_0$. Then $(u_{k, x_0}, v_0)$ will give
a morphism $\Spec(k) \to W$ whose composition with $W \to V$
is $v_0$ and we win.
\end{proof}

\begin{proposition}
\label{proposition-second-diagonal-representable}
Let $S$ be a locally Noetherian scheme. Let
$p : \mathcal{X} \to (\Sch/S)_{fppf}$ be a category fibred in groupoids.
Assume that
\begin{enumerate}
\item $\Delta_\Delta : \mathcal{X} \to
\mathcal{X} \times_{\mathcal{X} \times \mathcal{X}} \mathcal{X}$
is representable by algebraic spaces,
\item $\mathcal{X}$ satisfies axioms [-1], [0], [1], [2], [3], [4], and [5]
(see Section \ref{section-axioms}),
\item $\mathcal{O}_{S, s}$ is a G-ring for all finite type points $s$ of $S$.
\end{enumerate}
Then $\mathcal{X}$ is an algebraic stack.
\end{proposition}

\begin{proof}
We first prove that $\Delta : \mathcal{X} \to \mathcal{X} \times \mathcal{X}$
is representable by algebraic spaces. To do this it suffices to show
that
$$
\mathcal{Y} =
\mathcal{X} \times_{\Delta, \mathcal{X} \times \mathcal{X}, y} (\Sch/V)_{fppf}
$$
is representable by an algebraic space for any affine scheme $V$ locally
of finite presentation over $S$ and object $y$ of
$\mathcal{X} \times \mathcal{X}$ over $V$, see
Criteria for Representability, Lemma
\ref{criteria-lemma-check-representable-limit-preserving}\footnote{The
set theoretic condition in Criteria for Representability, Lemma
\ref{criteria-lemma-check-representable-limit-preserving}
will hold: the size of the algebraic space $Y$ representing $\mathcal{Y}$ is
suitably bounded. Namely, $Y \to S$ will be locally of finite type and $Y$
will satisfy axiom [-1]. Details omitted.}.
Observe that $\mathcal{Y}$ is fibred in setoids
(Stacks, Lemma \ref{stacks-lemma-isom-as-2-fibre-product})
and let $Y : (\Sch/S)_{fppf}^{opp} \to \textit{Sets}$,
$T \mapsto \Ob(\mathcal{Y}_T)/\cong$ be the functor of isomorphism
classes. We will apply
Proposition \ref{proposition-spaces-diagonal-representable}
to see that $Y$ is an algebraic space.

\medskip\noindent
Note that
$\Delta_\mathcal{Y} : \mathcal{Y} \to \mathcal{Y} \times \mathcal{Y}$
(and hence also $Y \to Y \times Y$)
is representable by algebraic spaces by condition (1) and
Criteria for Representability, Lemma \ref{criteria-lemma-second-diagonal}.
Observe that $Y$ is a sheaf for the \'etale topology by
Stacks, Lemmas \ref{stacks-lemma-stack-in-setoids-characterize} and
\ref{stacks-lemma-2-fibre-product-gives-stack-in-setoids}, i.e.,
axiom [0] holds. Also $Y$ is limit preserving by
Lemma \ref{lemma-fibre-product-limit-preserving}, i.e., we have [1].
Note that $Y$ has (RS), i.e., axiom [2] holds, by
Lemmas \ref{lemma-algebraic-stack-RS} and
\ref{lemma-fibre-product-RS}. Axiom [3] for $Y$ follows
from Lemmas \ref{lemma-finite-dimension} and
\ref{lemma-fibre-product-tangent-spaces}.
Axiom [4] follows from Lemmas \ref{lemma-effective} and
\ref{lemma-fibre-product-effective}.
Axiom [5] for $Y$ follows directly from openness of versality
for $\Delta_\mathcal{X}$ which is part of axiom [5] for $\mathcal{X}$.
Thus all the assumptions of
Proposition \ref{proposition-spaces-diagonal-representable}
are satisfied and $Y$ is an algebraic space.

\medskip\noindent
At this point it follows from Lemma \ref{lemma-diagonal-representable}
that $\mathcal{X}$ is an algebraic stack.
\end{proof}





\section{Strong Rim-Schlessinger}
\label{section-RS-star}

\noindent
In the rest of this chapter the following strictly stronger version
of the Rim-Schlessinger conditions will play an important role.

\begin{definition}
\label{definition-RS-star}
Let $S$ be a scheme. Let $\mathcal{X}$ be a category
fibred in groupoids over $(\Sch/S)_{fppf}$. We say $\mathcal{X}$
satisfies {\it condition (RS*)} if given a fibre product diagram
$$
\xymatrix{
B' \ar[r] & B \\
A' = A \times_B B' \ar[u] \ar[r] & A \ar[u]
}
$$
of $S$-algebras, with $B' \to B$ surjective with square zero kernel,
the functor of fibre categories
$$
\mathcal{X}_{\Spec(A')}
\longrightarrow
\mathcal{X}_{\Spec(A)} \times_{\mathcal{X}_{\Spec(B)}} \mathcal{X}_{\Spec(B')}
$$
is an equivalence of categories.
\end{definition}

\noindent
We make some observations:
with $A \to B \leftarrow B'$ as in Definition \ref{definition-RS-star}
\begin{enumerate}
\item we have $\Spec(A') = \Spec(A) \amalg_{\Spec(B)} \Spec(B')$
in the category of schemes, see
More on Morphisms, Lemma \ref{more-morphisms-lemma-pushout-along-thickening},
and
\item if $\mathcal{X}$ is an algebraic stack, then $\mathcal{X}$ satisfies
(RS*) by Lemma \ref{lemma-pushout}.
\end{enumerate}
If $S$ is locally Noetherian, then
\begin{enumerate}
\item[(3)] if $A$, $B$, $B'$ are of finite type over $S$ and
$B$ is finite over $A$, then $A'$ is of finite type over
$S$\footnote{If $\Spec(A)$ maps into an affine open of $S$
this follows from
More on Algebra, Lemma \ref{more-algebra-lemma-fibre-product-finite-type}.
The general case follows using
More on Algebra, Lemma \ref{more-algebra-lemma-diagram-localize}.}, and
\item[(4)] if $\mathcal{X}$ satisfies (RS*), then $\mathcal{X}$ satisfies (RS)
because (RS) covers exactly those cases of (RS*) where
$A$, $B$, $B'$ are Artinian local.
\end{enumerate}

\begin{lemma}
\label{lemma-algebraic-stack-RS-star}
Let $\mathcal{X}$ be an algebraic stack over a base $S$.
Then $\mathcal{X}$ satisfies (RS*).
\end{lemma}

\begin{proof}
This is implied by Lemma \ref{lemma-pushout}, see
remarks following Definition \ref{definition-RS-star}.
\end{proof}

\begin{lemma}
\label{lemma-fibre-product-RS-star}
Let $S$ be a scheme. Let $p : \mathcal{X} \to \mathcal{Y}$ and
$q : \mathcal{Z} \to \mathcal{Y}$ be $1$-morphisms of categories
fibred in groupoids over $(\Sch/S)_{fppf}$. If $\mathcal{X}$, $\mathcal{Y}$,
and $\mathcal{Z}$ satisfy (RS*), then so
does $\mathcal{X} \times_\mathcal{Y} \mathcal{Z}$.
\end{lemma}

\begin{proof}
The proof is exactly the same as the proof of
Lemma \ref{lemma-fibre-product-RS}.
\end{proof}






\section{Strong formal effectiveness}
\label{section-strong-formal-effectiveness}

\noindent
In this section we demonstrate how a strong version of effectiveness
of formal objects implies openness of versality.

\begin{lemma}
\label{lemma-infinite-sequence}
Let $S$ be a locally Noetherian scheme. Let $\mathcal{X}$ be a category
fibred in groupoids over $(\Sch/S)_{fppf}$ having (RS*).
Let $x$ be an object of
$\mathcal{X}$ over an affine scheme $U$ of finite type over $S$.
Let $u_n \in U$, $n \geq 1$ be pairwise distinct finite type points
such that $x$ is not versal at $u_n$ for all $n$. After replacing
$u_n$ by a subsequence, there exist morphisms
$$
x \to x_1 \to x_2 \to \ldots
\quad\text{in }\mathcal{X}\text{ lying over }\quad
U \to U_1 \to U_2 \to \ldots
$$
over $S$ such that
\begin{enumerate}
\item for each $n$ the morphism $U \to U_n$ is a first order
thickening,
\item for each $n$ we have a short exact sequence
$$
0 \to \kappa(u_n) \to \mathcal{O}_{U_n} \to \mathcal{O}_{U_{n - 1}} \to 0
$$
with $U_0 = U$ for $n = 1$,
\item for each $n$ there does {\bf not} exist a pair $(W, \alpha)$
consisting of an open neighbourhood $W \subset U_n$ of $u_n$
and a morphism $\alpha : x_n|_W \to x$
such that the composition
$$
x|_{U \cap W} \xrightarrow{\text{restriction of }x \to x_n}
x_n|_W \xrightarrow{\alpha} x
$$
is the canonical morphism $x|_{U \cap W} \to x$.
\end{enumerate}
\end{lemma}

\begin{proof}
After replacing $u_n$, $n \geq 1$ by a subsequence we may and do
assume that there are no specializations among these points, see
Properties, Lemma \ref{properties-lemma-thin-infinite-sequence}.
In particular, for every $n$ we can find an open $U' \subset U$
such that $u_n \in U'$ and $u_i \not \in U'$ for $i = 1, \ldots, n - 1$.
This means that the problem of constructing our system
decomposes into a separate problem for each $n$.
More precisely, suppose that for each $n \geq 1$ we can find
$$
x \to y_n
\quad\text{in }\mathcal{X}\text{ lying over}\quad
U \to T_n
$$
such that
\begin{enumerate}
\item the morphism $U \to T_n$ is a first order thickening,
\item we have a short exact sequence
$$
0 \to \kappa(u_n) \to \mathcal{O}_{T_n} \to \mathcal{O}_U \to 0
$$
\item for each $n$ there does {\bf not} exist a pair $(W, \alpha)$
consisting of an open neighbourhood $W \subset T_n$ of $u_n$
and a morphism $\beta : y_n|_W \to x$ such that the composition
$$
x|_{U \cap W} \xrightarrow{\text{restriction of }x \to y_n}
y_n|_W \xrightarrow{\beta} x
$$
is the canonical morphism $x|_{U \cap W} \to x$.
\end{enumerate}
Then we can define inductively
$$
U_1 = T_1, \quad
U_{n + 1} = U_n \amalg_U T_{n + 1}
$$
Setting $x_1 = y_1$ and using (RS*) we find inductively
$x_{n + 1}$ over $U_{n + 1}$ restricting to
$x_n$ over $U_n$ and $y_{n + 1}$ over $T_{n + 1}$.
Property (1) for $U \to U_n$ follows from the construction
of the pushout in More on Morphisms, Lemma
\ref{more-morphisms-lemma-pushout-along-thickening}.
Property (2) for $U_n$ similarly follows from
property (2) for $T_n$ by the construction of the pushout.
After shrinking to an open neighbourhood $U'$ of $u_n$
as discussed above, property (3) for $(U_n, x_n)$ follows from property (3)
for $(T_n, y_n)$ simply because the corresponding open subschemes
of $T_n$ and $U_n$ are isomorphic. Details omitted.

\medskip\noindent
This reduces us to the following: suppose given a single
finite type point $u \in U$ such that $x$ is not versal at $u$,
we need to construct a morphism $x \to y$ of $\mathcal{X}$ lying
over $U \to T$ satisfying properties (1), (2), and (3) formulated above.
Let $R = \mathcal{O}_{U, u}^\wedge$. Let $k = \kappa(u)$
be the residue field of $R$. Let $\xi$ be the formal object
of $\mathcal{X}$ over $R$ associated to $x$. Since $x$ is not
versal at $u$, we see that $\xi$ is not versal, see
Lemma \ref{lemma-versality-matches}. By the discussion following
Definition \ref{definition-versal-formal-object}
this means we can find
morphisms $\xi_1 \to x_A \to x_B$ of $\mathcal{X}$ lying over
closed immersions $\Spec(k) \to \Spec(A) \to \Spec(B)$
where $A, B$ are Artinian local rings with residue field $k$,
an $n \geq 1$ and a commutative diagram
$$
\vcenter{
\xymatrix{
& x_A \ar[ld] \\
\xi_n & \xi_1 \ar[u] \ar[l]
}
}
\quad\text{lying over}\quad
\vcenter{
\xymatrix{
& \Spec(A) \ar[ld] \\
\Spec(R/\mathfrak m^n) & \Spec(k) \ar[u] \ar[l]
}
}
$$
such that there does {\bf not} exist an $m \geq n$ and a commutative diagram
$$
\vcenter{
\xymatrix{
& & x_B \ar[lldd] \\
& & x_A \ar[ld] \ar[u] \\
\xi_m & \xi_n \ar[l] & \xi_1 \ar[u] \ar[l]
}
}
\quad\text{lying over}
\vcenter{
\xymatrix{
& & \Spec(B) \ar[lldd] \\
& & \Spec(A) \ar[ld] \ar[u] \\
\Spec(R/\mathfrak m^m) &
\Spec(R/\mathfrak m^n) \ar[l] &
\Spec(k) \ar[u] \ar[l]
}
}
$$
We may moreover assume that $B \to A$ is a small
extension, i.e., that the kernel $I$ of the surjection $B \to A$
is isomorphic to $k$ as an $A$-module.
This follows from Formal Deformation Theory, Remark
\ref{formal-defos-remark-versal-object}.
Then we simply define
$$
T = U \amalg_{\Spec(A)} \Spec(B)
$$
By property (RS*) we find $y$ over $T$ whose restriction to
$\Spec(B)$ is $x_B$ and whose restriction to $U$ is $x$
(this gives the arrow $x \to y$ lying over $U \to T$).
To finish the proof we verify conditions (1), (2), and (3).

\medskip\noindent
By the construction of the pushout we have a commutative diagram
$$
\xymatrix{
0 \ar[r] &
I \ar[r] &
B \ar[r] &
A \ar[r] &
0 \\
0 \ar[r] &
I \ar[r] \ar[u] &
\Gamma(T, \mathcal{O}_T) \ar[r] \ar[u] &
\Gamma(U, \mathcal{O}_U) \ar[r] \ar[u] &
0
}
$$
with exact rows. This immediately proves (1) and (2).
To finish the proof we will argue by contradiction.
Assume we have a pair $(W, \beta)$ as in (3).
Since $\Spec(B) \to T$ factors through $W$ we get the morphism
$$
x_B \to y|_W \xrightarrow{\beta} x
$$
Since $B$ is Artinian local with residue field $k = \kappa(u)$
we see that $x_B \to x$ lies over a morphism $\Spec(B) \to U$
which factors through $\Spec(\mathcal{O}_{U, u}/\mathfrak m_u^m)$
for some $m \geq n$. In other words, $x_B \to x$ factors
through $\xi_m$ giving a map $x_B \to \xi_m$.
The compatibility condition on the morphism $\alpha$
in condition (3) translates into the condition that
$$
\xymatrix{
x_B \ar[d] & x_A \ar[d] \ar[l] \\
\xi_m & \xi_n \ar[l]
}
$$
is commutative. This gives the contradiction we were looking for.
\end{proof}

\begin{remark}[Strong effectiveness]
\label{remark-strong-effectiveness}
Let $S$ be a locally Noetherian scheme.
Let $\mathcal{X}$ be a category fibred in groupoids over $(\Sch/S)_{fppf}$.
Assume we have
\begin{enumerate}
\item an affine open $\Spec(\Lambda) \subset S$,
\item an inverse system $(R_n)$ of $\Lambda$-algebras
with surjective transition maps whose kernels are locally nilpotent,
\item a system $(\xi_n)$ of objects of $\mathcal{X}$ lying
over the system $(\Spec(R_n))$.
\end{enumerate}
In this situation, set $R = \lim R_n$. We say that
$(\xi_n)$ is {\it effective} if there exists an object
$\xi$ of $\mathcal{X}$ over $\Spec(R)$ whose restriction
to $\Spec(R_n)$ gives the system $(\xi_n)$.
\end{remark}

\noindent
It is not the case that every algebraic stack $\mathcal{X}$
over $S$ satisfies a strong effectiveness axiom of the form:
every system $(\xi_n)$ as in Remark \ref{remark-strong-effectiveness}
is effective. An example is given in
Examples, Section \ref{examples-section-non-formal-effectiveness}.

\begin{lemma}
\label{lemma-SGE-implies-openness-versality}
Let $S$ be a locally Noetherian scheme. Let $\mathcal{X}$ be a category fibred
in groupoids over $(\Sch/S)_{fppf}$. Assume
\begin{enumerate}
\item $\Delta : \mathcal{X} \to \mathcal{X} \times \mathcal{X}$ is
representable by algebraic spaces,
\item $\mathcal{X}$ has (RS*),
\item $\mathcal{X}$ is limit preserving,
\item systems $(\xi_n)$ as in Remark \ref{remark-strong-effectiveness}
where $\Ker(R_m \to R_n)$ is an ideal of square zero for all $m \geq n$
are effective.
\end{enumerate}
Then $\mathcal{X}$ satisfies openness of versality.
\end{lemma}

\begin{proof}
Choose a scheme $U$ locally of finite type over $S$,
a finite type point $u_0$ of $U$, and an object $x$ of $\mathcal{X}$
over $U$ such that $x$ is versal at $u_0$. After shrinking
$U$ we may assume $U$ is affine and $U$ maps into an affine open
$\Spec(\Lambda)$ of $S$. If openness of versality does not hold,
then we get an infinite sequence of finite type points $u_n$ such
that $u_0$ is a limit point of the sequence and such that $x$ is not versal
at $u_n$ for $n \geq 1$. After passing to a subsequence we
get $x \to x_1 \to x_2 \to \ldots$ lying over
$U \to U_1 \to U_2 \to \ldots$ as in Lemma \ref{lemma-infinite-sequence}.
Write $U_n = \Spec(R_n)$ and $U = \Spec(R_0)$.
Set $R = \lim R_n$. Observe that $R \to R_0$ is surjective
with kernel an ideal of square zero. By assumption (4)
we get $\xi$ over $\Spec(R)$ whose base change to $R_n$ is $x_n$.
By assumption (3) we get that $\xi$ comes from an object $\xi'$ over
$U' = \Spec(R')$ for some finite type $\Lambda$-subalgebra
$R' \subset R$. After increasing $R'$ we may and do assume that
$R' \to R_0$ is surjective, so that $U \subset U'$ is a first order thickening.
Thus we now have
$$
x \to x_1 \to x_2 \to \ldots \to \xi'
\text{ lying over }
U \to U_1 \to U_2 \to \ldots \to U'
$$
By assumption (1) there is an algebraic space $Z$ over $S$ representing
$$
(\Sch/U)_{fppf} \times_{x, \mathcal{X}, \xi'} (\Sch/U')_{fppf}
$$
see Algebraic Stacks, Lemma \ref{algebraic-lemma-representable-diagonal}.
By construction of $2$-fibre products, a $T$-valued point of $Z$
corresponds to a triple $(a, a', \alpha)$ consisting of morphisms
$a : T \to U$, $a' : T \to U'$ and a morphism $\alpha : a^*x \to (a')^*\xi'$.
We obtain a commutative diagram
$$
\xymatrix{
U \ar[rd] \ar[rdd] \ar[rrd] \\
& Z \ar[r]_{p'} \ar[d]^p & U' \ar[d] \\
& U \ar[r] & S
}
$$
The morphism $i : U \to Z$ comes the isomorphism $x \to \xi'|_U$.
Let $z_0 = i(u_0) \in Z$. By Lemma \ref{lemma-base-change-versal}
we see that $Z \to U'$ is smooth at $z_0$. After replacing $U$ by an
affine open neighbourhood of $u_0$, replacing $U'$ by the corresponding
open, and replacing $W$ by the intersection of the inverse images
of these opens by $p$ and $p'$, we reach the situation where
$Z \to U'$ is smooth along $i(U)$. Note that this
also involves replacing $u_n$ by a subsequence, namely
by those indices such that $u_n$ is in the open. Moreover, condition
(3) of Lemma \ref{lemma-infinite-sequence}
is clearly preserved by shrinking $U$
(all of the schemes $U$, $U_n$, $U'$ have the same underlying
topological space).
Since $U \to U'$ is a first order thickening of affine schemes,
we can choose a morphism $i' : U' \to Z$
such that $p' \circ i' = \text{id}_{U'}$ and
whose restriction to $U$ is $i$
(More on Morphisms of Spaces, Lemma
\ref{spaces-more-morphisms-lemma-smooth-formally-smooth}).
Pulling back the universal morphism
$p^*x \to (p')^*\xi'$ by $i'$ we obtain a morphism
$$
\xi' \to x
$$
lying over $p \circ i' : U' \to U$ such that the composition
$$
x \to \xi' \to x
$$
is the identity. Recall that we have $x_1 \to \xi'$ lying over
the morphism $U_1 \to U'$. Composing we get a morphism
$x_1 \to x$ whose existence contradicts condition
(3) of Lemma \ref{lemma-infinite-sequence}.
This contradiction finishes the proof.
\end{proof}

\begin{remark}
\label{remark-trade-openness-versality-diagonal-with-strong-effectiveness}
There is a way to deduce openness of versality of the diagonal
of an category fibred in groupoids from a strong formal effectiveness
axiom.
Let $S$ be a locally Noetherian scheme. Let $\mathcal{X}$ be a category fibred
in groupoids over $(\Sch/S)_{fppf}$. Assume
\begin{enumerate}
\item $\Delta_\Delta : \mathcal{X} \to
\mathcal{X} \times_{\mathcal{X} \times \mathcal{X}} \mathcal{X}$
is representable by algebraic spaces,
\item $\mathcal{X}$ has (RS*),
\item $\mathcal{X}$ is limit preserving,
\item given an inverse system $(R_n)$ of $S$-algebras
as in Remark \ref{remark-strong-effectiveness}
where $\Ker(R_m \to R_n)$ is an ideal of square zero for all $m \geq n$
the functor
$$
\mathcal{X}_{\Spec(\lim R_n)} \longrightarrow
\lim_n \mathcal{X}_{\Spec(R_n)}
$$
is fully faithful.
\end{enumerate}
Then $\Delta : \mathcal{X} \to \mathcal{X} \times \mathcal{X}$
satisfies openness of versality. This follows by applying
Lemma \ref{lemma-SGE-implies-openness-versality}
to fibre products of the form
$\mathcal{X} \times_{\Delta, \mathcal{X} \times \mathcal{X}, y}
(\Sch/V)_{fppf}$ for any affine scheme $V$ locally
of finite presentation over $S$ and object $y$ of
$\mathcal{X} \times \mathcal{X}$ over $V$.
If we ever need this, we will change this remark into
a lemma and provide a detailed proof.
\end{remark}











\section{Infinitesimal deformations}
\label{section-inf}

\noindent
In this section we discuss a generalization of the notion of the
tangent space introduced in Section \ref{section-tangent-spaces}.
To do this intelligently, we borrow some notation from
Formal Deformation Theory, Sections
\ref{formal-defos-section-tangent-spaces-functors},
\ref{formal-defos-section-lifts}, and
\ref{formal-defos-section-infinitesimal-automorphisms}.

\medskip\noindent
Let $S$ be a scheme. Let $\mathcal{X}$ be a category fibred in groupoids
over $(\Sch/S)_{fppf}$. Given a homomorphism $A' \to A$ of $S$-algebras
and an object $x$ of $\mathcal{X}$ over $\Spec(A)$ we write
$\textit{Lift}(x, A')$ for the category of lifts of $x$ to $\Spec(A')$.
An object of $\textit{Lift}(x, A')$ is a morphism $x \to x'$ of $\mathcal{X}$
lying over $\Spec(A) \to \Spec(A')$ and morphisms of $\textit{Lift}(x, A')$
are defined as commutative diagrams. The set of isomorphism classes of
$\textit{Lift}(x, A')$ is denoted $\text{Lift}(x, A')$. See
Formal Deformation Theory, Definition \ref{formal-defos-definition-lifts} and
Remark \ref{formal-defos-remark-omit-arrow}.
If $A' \to A$ is surjective with locally nilpotent kernel we call an element
$x'$ of $\text{Lift}(x, A')$ a {\it (infinitesimal) deformation} of $x$.
In this case the {\it group of infinitesimal automorphisms of $x'$ over $x$}
is the kernel
$$
\text{Inf}(x'/x) =
\Ker\left(
\text{Aut}_{\mathcal{X}_{\Spec(A')}}(x') \to
\text{Aut}_{\mathcal{X}_{\Spec(A)}}(x)\right)
$$
Note that an element of $\text{Inf}(x'/x)$ is the same thing as a lift
of $\text{id}_x$ over $\Spec(A')$ for (the category fibred in sets associated
to) $\mathit{Aut}_\mathcal{X}(x')$. Compare with
Formal Deformation Theory, Definition
\ref{formal-defos-definition-relative-infinitesimal-auts} and
Formal Deformation Theory, Remark
\ref{formal-defos-remark-infaut-lifting-equalities}.

\medskip\noindent
If $M$ is an $A$-module we denote $A[M]$ the $A$-algebra whose underlying
$A$-module is $A \oplus M$ and whose multiplication is given by
$(a, m) \cdot (a', m') = (aa', am' + a'm)$. When $M = A$ this is the ring
of dual numbers over $A$, which we denote $A[\epsilon]$ as is customary.
There is an $A$-algebra map $A[M] \to A$. The pullback of $x$ to $\Spec(A[M])$
is called the {\it trivial deformation} of $x$ to $\Spec(A[M])$.

\begin{lemma}
\label{lemma-functoriality}
Let $S$ be a scheme. Let $f : \mathcal{X} \to \mathcal{Y}$ be a $1$-morphism
of categories fibred in groupoids over $(\Sch/S)_{fppf}$. Let
$$
\xymatrix{
B' \ar[r] & B \\
A' \ar[u] \ar[r] & A \ar[u]
}
$$
be a commutative diagram of $S$-algebras. Let $x$ be an object of $\mathcal{X}$
over $\Spec(A)$, let $y$ be an object of $\mathcal{Y}$ over $\Spec(B)$,
and let $\phi : f(x)|_{\Spec(B)} \to y$ be a morphism of $\mathcal{Y}$
over $\Spec(B)$. Then there is a canonical functor
$$
\textit{Lift}(x, A') \longrightarrow \textit{Lift}(y, B')
$$
of categories of lifts induced by $f$ and $\phi$. The construction is
compatible with compositions of $1$-morphisms of categories fibred in
groupoids in an obvious manner.
\end{lemma}

\begin{proof}
This lemma proves itself.
\end{proof}

\noindent
Let $S$ be a base scheme. Let $\mathcal{X}$ be a category fibred
in groupoids over $(\Sch/S)_{fppf}$. We define a category whose objects are
pairs $(x, A' \to A)$ where
\begin{enumerate}
\item $A' \to A$ is a surjection of $S$-algebras whose kernel
is an ideal of square zero,
\item $x$ is an object of $\mathcal{X}$ lying over $\Spec(A)$.
\end{enumerate}
A morphism $(y, B' \to B) \to (x, A' \to A)$ is given by a commutative
diagram
$$
\xymatrix{
B' \ar[r] & B \\
A' \ar[u] \ar[r] & A \ar[u]
}
$$
of $S$-algebras together with a morphism $x|_{\Spec(B)} \to y$ over
$\Spec(B)$. Let us call this the category of {\it deformation situations}.

\begin{lemma}
\label{lemma-properties-lift-RS-star}
Let $S$ be a scheme. Let $\mathcal{X}$ be a category
fibred in groupoids over $(\Sch/S)_{fppf}$. Assume $\mathcal{X}$ satisfies
condition (RS*). Let $A$ be an $S$-algebra and let $x$ be an object of
$\mathcal{X}$ over $\Spec(A)$.
\begin{enumerate}
\item There exists an $A$-linear functor
$\text{Inf}_x : \text{Mod}_A \to \text{Mod}_A$
such that given a deformation situation $(x, A' \to A)$ and a lift $x'$
there is an isomorphism $\text{Inf}_x(I) \to \text{Inf}(x'/x)$ where
$I = \Ker(A' \to A)$.
\item There exists an $A$-linear functor
$T_x : \text{Mod}_A \to \text{Mod}_A$
such that
\begin{enumerate}
\item given $M$ in $\text{Mod}_A$ there is a bijection
$T_x(M) \to \text{Lift}(x, A[M])$,
\item given a deformation situation $(x, A' \to A)$ there is an action
$$
T_x(I) \times \text{Lift}(x, A') \to \text{Lift}(x, A')
$$
where $I = \Ker(A' \to A)$. It is simply transitive if
$\text{Lift}(x, A') \not = \emptyset$.
\end{enumerate}
\end{enumerate}
\end{lemma}

\begin{proof}
We define $\text{Inf}_x$ as the functor
$$
\text{Mod}_A \longrightarrow \textit{Sets},\quad
M \longrightarrow
\text{Inf}(x'_M/x) = \text{Lift}(\text{id}_x, A[M])
$$
mapping $M$ to the group of infinitesimal automorphisms
of the trivial deformation $x'_M$ of $x$ to $\Spec(A[M])$
or equivalently the group of lifts of $\text{id}_x$ in
$\mathit{Aut}_\mathcal{X}(x'_M)$.
We define $T_x$ as the functor
$$
\text{Mod}_A \longrightarrow \textit{Sets},\quad
M \longrightarrow \text{Lift}(x, A[M])
$$
of isomorphism classes of infintesimal deformations of $x$ to
$\Spec(A[M])$. We apply Formal Deformation Theory, Lemma
\ref{formal-defos-lemma-linear-functor}
to $\text{Inf}_x$ and $T_x$. This lemma is applicable, since
(RS*) tells us that
$$
\textit{Lift}(x, A[M \times N]) =
\textit{Lift}(x, A[M]) \times \textit{Lift}(x, A[N])
$$
as categories (and trivial deformations match up too).

\medskip\noindent
Let $(x, A' \to A)$ be a deformation situation. Consider the ring map
$g : A' \times_A A' \to A[I]$ defined by the
rule $g(a_1, a_2) = \overline{a_1} \oplus a_2 - a_1$.
There is an isomorphism
$$
A' \times_A A' \longrightarrow A' \times_A A[I]
$$
given by $(a_1, a_2) \mapsto (a_1, g(a_1, a_2))$.
This isomorphism commutes with the projections to $A'$ on the first
factor, and hence with the projections to $A$. Thus applying (RS*)
twice we find equivalences of categories
\begin{align*}
\textit{Lift}(x, A') \times \textit{Lift}(x, A')
& =
\textit{Lift}(x, A' \times_A A') \\
& =
\textit{Lift}(x, A' \times_A A[I]) \\
& =
\textit{Lift}(x, A') \times \textit{Lift}(x, A[I])
\end{align*}
Using these maps and projection onto the last factor of the last product
we see that we obtain ``difference maps''
$$
\text{Inf}(x'/x) \times  \text{Inf}(x'/x)
\longrightarrow
\text{Inf}_x(I)
\quad\text{and}\quad
\text{Lift}(x, A') \times \text{Lift}(x, A')
\longrightarrow
T_x(I)
$$
These difference maps satisfy the transitivity rule
``$(x'_1 - x'_2) + (x'_2 - x'_3) = x'_1 - x'_3$'' because
$$
\xymatrix{
A' \times_A A' \times_A A'
\ar[rrrrr]_-{(a_1, a_2, a_3) \mapsto (g(a_1, a_2), g(a_2, a_3))}
\ar[rrrrrd]_{(a_1, a_2, a_3) \mapsto g(a_1, a_3)} & & & & &
A[I] \times_A A[I] = A[I \times I] \ar[d]^{+} \\
& & & & & A[I]
}
$$
is commutative. Inverting the string of equivalences above we obtain
an action which is free and transitive provided $\text{Inf}(x'/x)$,
resp.\ $\text{Lift}(x, A')$ is nonempty. Note that $\text{Inf}(x'/x)$
is always nonempty as it is a group.
\end{proof}

\begin{remark}[Functoriality]
\label{remark-functoriality}
Assumptions and notation as in Lemma \ref{lemma-properties-lift-RS-star}.
Suppose $A \to B$ is a ring map and $y = x|_{\Spec(B)}$.
Let $M \in \text{Mod}_A$, $N \in \text{Mod}_B$
and let $M \to N$ an $A$-linear map. Then there are canonical maps
$\text{Inf}_x(M) \to \text{Inf}_y(N)$ and
$T_x(M) \to T_y(N)$ simply because there is a pullback functor
$$
\textit{Lift}(x, A[M]) \to \textit{Lift}(y, B[N])
$$
coming from the ring map $A[M] \to B[N]$. Similarly, given a morphism of
deformation situations $(y, B' \to B) \to (x, A' \to A)$ we obtain a pullback
functor $\textit{Lift}(x, A') \to \textit{Lift}(y, B')$. Since the
construction of the action, the addition, and the scalar multiplication
on $\text{Inf}_x$ and $T_x$ use only morphisms in the categories of lifts
(see proof of
Formal Deformation Theory, Lemma
\ref{formal-defos-lemma-linear-functor})
we see that the constructions above are functorial. In other words we
obtain $A$-linear maps
$$
\text{Inf}_x(M) \to \text{Inf}_y(N)
\quad\text{and}\quad
T_x(M) \to T_y(N)
$$
such that the diagrams
$$
\vcenter{
\xymatrix{
\text{Inf}_y(J) \ar[r] & \text{Inf}(y'/y) \\
\text{Inf}_x(I) \ar[r] \ar[u] & \text{Inf}(x'/x) \ar[u]
}
}
\quad\text{and}\quad
\vcenter{
\xymatrix{
T_y(J) \times \text{Lift}(y, B') \ar[r] & \text{Lift}(y, B') \\
T_x(I) \times \text{Lift}(x, A') \ar[r] \ar[u] & \text{Lift}(x, A') \ar[u]
}
}
$$
commute. Here $I = \Ker(A' \to A)$, $J = \Ker(B' \to B)$,
$x'$ is a lift of $x$ to $A'$ (which may not always exist) and
$y' = x'|_{\Spec(B')}$.
\end{remark}

\begin{remark}[Automorphisms]
\label{remark-automorphisms}
Assumptions and notation as in Lemma \ref{lemma-properties-lift-RS-star}.
Let $x', x''$ be lifts of $x$ to $A'$. Then we have a composition
map
$$
\text{Inf}(x'/x) \times
\Mor_{\textit{Lift}(x, A')}(x', x'') \times \text{Inf}(x''/x)
\longrightarrow
\Mor_{\textit{Lift}(x, A')}(x', x'').
$$
Since $\textit{Lift}(x, A')$ is a groupoid, if
$\Mor_{\textit{Lift}(x, A')}(x', x'')$ is nonempty, then this defines
a simply transitive left action of $\text{Inf}(x'/x)$ on
$\Mor_{\textit{Lift}(x, A')}(x', x'')$ and a simply transitive
right action by $\text{Inf}(x''/x)$. Now the lemma says that
$\text{Inf}(x'/x) = \text{Inf}_x(I) = \text{Inf}(x''/x)$.
We claim that the two actions described above agree via these identifications.
Namely, either $x' \not \cong x''$ in which the claim is clear, or
$x' \cong x''$ and in that case we may assume that $x'' = x'$ in which
case the result follows from the fact that $\text{Inf}(x'/x)$ is
commutative. In particular, we obtain a well defined action
$$
\text{Inf}_x(I) \times \Mor_{\textit{Lift}(x, A')}(x', x'')
\longrightarrow
\Mor_{\textit{Lift}(x, A')}(x', x'')
$$
which is simply transitive as soon as $\Mor_{\textit{Lift}(x, A')}(x', x'')$
is nonempty.
\end{remark}

\begin{remark}
\label{remark-short-exact-sequence-thickenings}
Let $S$ be a scheme. Let $\mathcal{X}$ be a category
fibred in groupoids over $(\Sch/S)_{fppf}$. Let $A$ be an $S$-algebra. There
is a notion of a {\it short exact sequence}
$$
(x, A_1' \to A) \to (x, A_2' \to A) \to (x, A_3' \to A)
$$
of deformation situations: we ask the corresponding maps between
the kernels $I_i = \Ker(A_i' \to A)$ give a short exact sequence
$$
0 \to I_3 \to I_2 \to I_1 \to 0
$$
of $A$-modules. Note that in this case the map $A_3' \to A_1'$
factors through $A$, hence there is a canonical isomorphism
$A_1' = A[I_1]$.
\end{remark}

\begin{lemma}
\label{lemma-ses-inf-and-T}
Let $S$ be a scheme. Let $p : \mathcal{X} \to \mathcal{Y}$
and $q : \mathcal{Z} \to \mathcal{Y}$ be $1$-morphisms of categories
fibred in groupoids over $(\Sch/S)_{fppf}$. Assume $\mathcal{X}$,
$\mathcal{Y}$, $\mathcal{Z}$ satisfy (RS*).
Let $A$ be an $S$-algebra and let $w$ be an object of
$\mathcal{W} = \mathcal{X} \times_\mathcal{Y} \mathcal{Z}$ over $A$.
Denote $x, y, z$ the objects of $\mathcal{X}, \mathcal{Y}, \mathcal{Z}$
you get from $w$. For any $A$-module $M$ there is a $6$-term exact sequence
$$
\xymatrix{
0 \ar[r] &
\text{Inf}_w(M) \ar[r] &
\text{Inf}_x(M) \oplus \text{Inf}_z(M) \ar[r] &
\text{Inf}_y(M) \ar[lld] \\
 &
T_w(M) \ar[r] &
T_x(M) \oplus T_z(M) \ar[r] &
T_y(M)
}
$$
of $A$-modules.
\end{lemma}

\begin{proof}
By Lemma \ref{lemma-fibre-product-RS-star} we see that $\mathcal{W}$
satisfies (RS*) and hence $T_w(M)$ and $\text{Inf}_w(M)$ are defined.
The horizontal arrows are defined using the functoriality of
Lemma \ref{lemma-functoriality}.

\medskip\noindent
Definition of the ``boundary'' map $\delta : \text{Inf}_y(M) \to T_w(M)$.
Choose isomorphisms $p(x) \to y$ and $y \to q(z)$ such that
$w = (x, z, p(x) \to y \to q(z))$ in the description of
the $2$-fibre product of
Categories, Lemma \ref{categories-lemma-2-product-fibred-categories}
and more precisely
Categories, Lemma \ref{categories-lemma-2-product-categories-over-C}.
Let $x', y', z', w'$ denote the trivial deformation of
$x, y, z, w$ over $A[M]$. By pullback we get isomorphisms
$y' \to p(x')$ and $q(z') \to y'$. An element $\alpha \in \text{Inf}_y(M)$
is the same thing as an automorphism $\alpha : y' \to y'$
over $A[M]$ which restricts to the identity on $y$ over $A$.
Thus setting
$$
\delta(\alpha) =
(x', z', p(x') \to y' \xrightarrow{\alpha} y' \to q(z'))
$$
we obtain an object of $T_w(M)$. This is a map of $A$-modules
by Formal Deformation Theory, Lemma
\ref{formal-defos-lemma-morphism-linear-functors}.

\medskip\noindent
The rest of the proof is exactly the same as the proof of
Formal Deformation Theory, Lemma
\ref{formal-defos-lemma-deformation-categories-fiber-product-morphisms}.
\end{proof}

\begin{remark}[Compatibility with previous tangent spaces]
\label{remark-compare-deformation-spaces}
Let $S$ be a locally Noetherian scheme. Let $\mathcal{X}$ be a category fibred
in groupoids over $(\Sch/S)_{fppf}$. Assume $\mathcal{X}$ has (RS*).
Let $k$ be a field of finite type over $S$ and let $x_0$ be an object of
$\mathcal{X}$ over $\Spec(k)$. Then we have equalities of
$k$-vector spaces
$$
T\mathcal{F}_{\mathcal{X}, k, x_0} = T_{x_0}(k)
\quad\text{and}\quad
\text{Inf}(\mathcal{F}_{\mathcal{X}, k, x_0}) =
\text{Inf}_{x_0}(k)
$$
where the spaces on the left hand side of the equality signs are
given in (\ref{equation-tangent-space}) and
(\ref{equation-infinitesimal-automorphisms})
and the spaces on the right hand side are given by
Lemma \ref{lemma-properties-lift-RS-star}.
\end{remark}

\begin{remark}[Canonical element]
\label{remark-canonical-element}
Assumptions and notation as in Lemma \ref{lemma-properties-lift-RS-star}.
Choose an affine open $\Spec(\Lambda) \subset S$ such that $\Spec(A) \to S$
corresponds to a ring map $\Lambda \to A$. Consider the ring map
$$
A \longrightarrow A[\Omega_{A/\Lambda}],
\quad
a \longmapsto (a, \text{d}_{A/\Lambda}(a))
$$
Pulling back $x$ along the corresponding morphism
$\Spec(A[\Omega_{A/\Lambda}]) \to \Spec(A)$ we obtain a
deformation $x_{can}$ of $x$ over $A[\Omega_{A/\Lambda}]$. We call this
the {\it canonical element}
$$
x_{can} \in T_x(\Omega_{A/\Lambda}) = \text{Lift}(x, A[\Omega_{A/\Lambda}]).
$$
Next, assume that $\Lambda$ is Noetherian and $\Lambda \to A$
is of finite type. Let
$k = \kappa(\mathfrak p)$ be a residue field at a finite type point $u_0$
of $U = \Spec(A)$. Let $x_0 = x|_{u_0}$. By (RS*) and the fact that
$A[k] = A \times_k k[k]$ the space $T_x(k)$ is the tangent space to the
deformation functor $\mathcal{F}_{\mathcal{X}, k, x_0}$. Via
$$
T\mathcal{F}_{U, k, u_0} =
\text{Der}_\Lambda(A, k) = \Hom_A(\Omega_{A/\Lambda}, k)
$$
(see Formal Deformation Theory, Example
\ref{formal-defos-example-tangent-space-prorepresentable-functor})
and functoriality of $T_x$ the canonical element produces the map
on tangent spaces induced by the object $x$ over $U$. Namely,
$\theta \in T\mathcal{F}_{U, k, u_0}$ maps to $T_x(\theta)(x_{can})$
in $T_x(k) = T\mathcal{F}_{\mathcal{X}, k, x_0}$.
\end{remark}

\begin{remark}[Canonical automorphism]
\label{remark-canonical-isomorphism}
Let $S$ be a locally Noetherian scheme. Let $\mathcal{X}$ be a category
fibred in groupoids over $(\Sch/S)_{fppf}$. Assume $\mathcal{X}$ satisfies
condition (RS*). Let $A$ be an $S$-algebra such that
$\Spec(A) \to S$ maps into an affine open and let $x, y$ be objects of
$\mathcal{X}$ over $\Spec(A)$. Further, let $A \to B$ be a ring map and
let $\alpha : x|_{\Spec(B)} \to y|_{\Spec(B)}$ be a morphism of
$\mathcal{X}$ over $\Spec(B)$. Consider the ring map
$$
B \longrightarrow B[\Omega_{B/A}],
\quad
b \longmapsto (b, \text{d}_{B/A}(b))
$$
Pulling back $\alpha$ along the corresponding morphism
$\Spec(B[\Omega_{B/A}]) \to \Spec(B)$ we obtain a
morphism $\alpha_{can}$ between the pullbacks of $x$ and $y$ over
$B[\Omega_{B/A}]$. On the other hand, we can pullback $\alpha$
by the morphism $\Spec(B[\Omega_{B/A}]) \to \Spec(B)$ corresponding
to the injection of $B$ into the first summand of $B[\Omega_{B/A}]$.
By the discussion of Remark \ref{remark-automorphisms}
we can take the difference
$$
\varphi(x, y, \alpha) = \alpha_{can} - \alpha|_{\Spec(B[\Omega_{B/A}])} \in
\text{Inf}_{x|_{\Spec(B)}}(\Omega_{B/A}).
$$
We will call this the {\it canonical automorphism}. It depends
on all the ingredients $A$, $x$, $y$, $A \to B$ and $\alpha$.
\end{remark}





\section{Obstruction theories}
\label{section-obstruction-theory}

\noindent
In this section we describe what an obstruction theory is.
Contrary to the spaces of infinitesimal deformations and infinitesimal
automorphisms, an obstruction theory is an additional piece of data.
The formulation is motivated by the results of
Lemma \ref{lemma-properties-lift-RS-star}
and Remark \ref{remark-functoriality}.

\begin{definition}
\label{definition-obstruction-theory}
Let $S$ be a locally Noetherian base. Let $\mathcal{X}$ be a category fibred
in groupoids over $(\Sch/S)_{fppf}$. An {\it obstruction theory} is 
given by the following data
\begin{enumerate}
\item for every $S$-algebra $A$ such that $\Spec(A) \to S$
maps into an affine open and every object $x$ of $\mathcal{X}$ over
$\Spec(A)$ an $A$-linear functor
$$
\mathcal{O}_x : \text{Mod}_A \to \text{Mod}_A
$$
of {\it obstruction modules},
\item for $(x, A)$ as in (1), a ring map $A \to B$,
$M \in \text{Mod}_A$, $N \in \text{Mod}_B$, and an $A$-linear
map $M \to N$ an induced $A$-linear map $\mathcal{O}_x(M) \to \mathcal{O}_y(N)$
where $y = x|_{\Spec(B)}$, and
\item for every deformation situation $(x, A' \to A)$ an
{\it obstruction} element
$o_x(A') \in \mathcal{O}_x(I)$ where $I = \Ker(A' \to A)$.
\end{enumerate}
These data are subject to the following conditions
\begin{enumerate}
\item[(i)] the functoriality maps turn the obstruction modules into a functor
from the category of triples $(x, A, M)$ to sets,
\item[(ii)] for every morphism of deformation situations
$(y, B' \to B) \to (x, A' \to A)$ the element $o_x(A')$ maps
to $o_y(B')$, and
\item[(iii)] we have
$$
\text{Lift}(x, A') \not = \emptyset
\Leftrightarrow
o_x(A') = 0
$$
for every deformation situation $(x, A' \to A)$.
\end{enumerate}
\end{definition}

\noindent
This last condition explains the terminology. The module $\mathcal{O}_x(A')$
is called the {\it obstruction module}. The element $o_x(A')$ is the
{\it obstruction}.
Most obstruction theories have additional properties, and in order to
make them useful additional conditions are needed.
Moreover, this is just a sample definition, for example in the definition
we could consider only deformation situations of finite type over $S$.

\medskip\noindent
One of the main reasons for introducing obstruction theories is to check
openness of versality. An example of this type of result is
Lemma \ref{lemma-get-openness-obstruction-theory} below.
The initial idea to do this is due to Artin, see
the papers of Artin mentioned in the introduction. It has been taken up
for example in the work by Flenner \cite{Flenner},
Hall \cite{Hall-coherent},
Hall and Rydh \cite{rydh_axioms},
Olsson \cite{olsson_deformation},
Olsson and Starr \cite{olsson-starr}, and
Lieblich \cite{lieblich-complexes} (random order of references).
Moreover, for particular categories fibred in groupoids, often
authors develop a little bit of theory adapted to the problem at hand.
We will develop this theory later (insert future reference here).

\begin{lemma}
\label{lemma-get-openness-obstruction-theory}
\begin{reference}
This is \cite[Theorem 4.4]{Hall-coherent}
\end{reference}
Let $S$ be a locally Noetherian scheme. Let $\mathcal{X}$ be a category fibred
in groupoids over $(\Sch/S)_{fppf}$. Assume
\begin{enumerate}
\item $\Delta : \mathcal{X} \to \mathcal{X} \times \mathcal{X}$ is
representable by algebraic spaces,
\item $\mathcal{X}$ has (RS*),
\item $\mathcal{X}$ is limit preserving,
\item there exists an obstruction theory\footnote{Analyzing the proof
the reader sees that in fact it suffices to check
the functoriality (ii) of obstruction classes in
Definition \ref{definition-obstruction-theory}
for maps $(y, B' \to B) \to (x, A' \to A)$
with $B = A$ and $y = x$.},
\item for an object $x$ of $\mathcal{X}$ over $\Spec(A)$
and $A$-modules $M_n$, $n \geq 1$ we have
\begin{enumerate}
\item $T_x(\prod M_n) = \prod T_x(M_n)$,
\item $\mathcal{O}_x(\prod M_n) \to \prod \mathcal{O}_x(M_n)$
is injective.
\end{enumerate}
\end{enumerate}
Then $\mathcal{X}$ satisfies openness of versality.
\end{lemma}

\begin{proof}
We prove this by verifying condition (4) of
Lemma \ref{lemma-SGE-implies-openness-versality}.
Let $(\xi_n)$ and $(R_n)$ be as in Remark \ref{remark-strong-effectiveness}
such that $\Ker(R_m \to R_n)$ is an ideal of square zero
for all $m \geq n$. Set $A = R_1$ and $x = \xi_1$.
Denote $M_n = \Ker(R_n \to R_1)$.
Then $M_n$ is an $A$-module. Set $R = \lim R_n$.
Let
$$
\tilde R = \{(r_1, r_2, r_3 \ldots) \in \prod R_n
\text{ such that all have the same image in }A\}
$$
Then $\tilde R \to A$ is surjective with kernel $M = \prod M_n$.
There is a map $R \to \tilde R$ and a map
$\tilde R \to A[M]$, $(r_1, r_2, r_3, \ldots) \mapsto
(r_1, r_2 - r_1, r_3 - r_2, \ldots)$.
Together these give a short exact sequence
$$
(x, R \to A) \to (x, \tilde R \to A) \to (x, A[M])
$$
of deformation situations, see
Remark \ref{remark-short-exact-sequence-thickenings}.
The associated sequence of kernels
$0 \to \lim M_n \to M \to M \to 0$
is the canonical sequence computing the limit
of the system of modules $(M_n)$.

\medskip\noindent
Let $o_x(\tilde R) \in \mathcal{O}_x(M)$ be the obstruction element.
Since we have the lifts $\xi_n$ we see that $o_x(\tilde R)$
maps to zero in $\mathcal{O}_x(M_n)$. By assumption (5)(b)
we see that $o_x(\tilde R) = 0$. Choose a lift $\tilde \xi$
of $x$ to $\Spec(\tilde R)$. Let $\tilde \xi_n$ be the
restriction of $\tilde \xi$ to $\Spec(R_n)$. There exists
elements $t_n \in T_x(M_n)$ such that
$t_n \cdot \tilde \xi_n = \xi_n$ by
Lemma \ref{lemma-properties-lift-RS-star} part (2)(b).
By assumption (5)(a) we can find $t \in T_x(M)$
mapping to $t_n$ in $T_x(M_n)$. After replacing
$\tilde \xi$ by $t \cdot \tilde \xi$ we find that
$\tilde \xi$ restricts to $\xi_n$ over $\Spec(R_n)$ for all $n$.
In particular, since $\xi_{n + 1}$ restricts to $\xi_n$
over $\Spec(R_n)$, the restriction $\overline{\xi}$ of $\tilde \xi$
to $\Spec(A[M])$ has the property that it restricts to
the trivial deformation over $\Spec(A[M_n])$ for all $n$.
Hence by assumption (5)(a) we find that $\overline{\xi}$
is the trivial deformation of $x$. By axiom (RS*)
applied to $R = \tilde R \times_{A[M]} A$
this implies that $\tilde \xi$ is the pullback
of a deformation $\xi$ of $x$ over $R$. This finishes the proof.
\end{proof}

\begin{example}
\label{example-global-sections}
Let $S = \Spec(\Lambda)$ for some Noetherian ring $\Lambda$.
Let $W \to S$ be a morphism of schemes. Let $\mathcal{F}$
be a quasi-coherent $\mathcal{O}_W$-module flat over $S$.
Consider the functor
$$
F : (\Sch/S)_{fppf}^{opp} \longrightarrow \textit{Sets},
\quad
T/S \longrightarrow H^0(W_T, \mathcal{F}_T)
$$
where $W_T = T \times_S W$ is the base change and $\mathcal{F}_T$ is
the pullback of $\mathcal{F}$ to $W_T$. If $T = \Spec(A)$
we will write $W_T = W_A$, etc. Let $\mathcal{X} \to (\Sch/S)_{fppf}$
be the category fibred in groupoids associated to $F$. Then
$\mathcal{X}$ has an obstruction theory. Namely,
\begin{enumerate}
\item given $A$ over $\Lambda$ and
$x \in H^0(W_A, \mathcal{F}_A)$ we set
$\mathcal{O}_x(M) = H^1(W_A, \mathcal{F}_A \otimes_A M)$,
\item given a deformation situation $(x, A' \to A)$ we let
$o_x(A') \in \mathcal{O}_x(A)$ be the image of $x$ under the boundary map
$$
H^0(W_A, \mathcal{F}_A) \longrightarrow H^1(W_A, \mathcal{F}_A \otimes_A I)
$$
coming from the short exact sequence of modules
$$
0 \to \mathcal{F}_A \otimes_A I \to
\mathcal{F}_{A'} \to \mathcal{F}_A \to 0.
$$
\end{enumerate}
We have omitted some details, in particular the construction of the short
exact sequence above (it uses that $W_A$ and $W_{A'}$ have the same
underlying topological space) and the explanation for why flatness
of $\mathcal{F}$ over $S$ implies that the sequence above is short exact.
\end{example}

\begin{example}[Key example]
\label{example-key}
Let $S = \Spec(\Lambda)$ for some Noetherian ring $\Lambda$.
Say $\mathcal{X} = (\Sch/X)_{fppf}$ with $X = \Spec(R)$ and
$R = \Lambda[x_1, \ldots, x_n]/J$. The naive cotangent
complex $\NL_{R/\Lambda}$ is (canonically) homotopy equivalent to
$$
J/J^2
\longrightarrow
\bigoplus\nolimits_{i = 1, \ldots, n} R\text{d}x_i,
$$
see Algebra, Lemma \ref{algebra-lemma-NL-homotopy}.
Consider a deformation situation $(x, A' \to A)$. Denote $I$ the kernel of
$A' \to A$. The object $x$ corresponds to $(a_1, \ldots, a_n)$
with $a_i \in A$ such that $f(a_1, \ldots, a_n) = 0$ in $A$ for all $f \in J$.
Set
\begin{align*}
\mathcal{O}_x(A')
& =
\Hom_R(J/J^2, I)/\Hom_R(R^{\oplus n}, I) \\
& =
\Ext^1_R(\NL_{R/\Lambda}, I) \\
& =
\Ext^1_A(\NL_{R/\Lambda} \otimes_R A, I).
\end{align*}
Choose lifts $a_i' \in A'$ of $a_i$ in $A$. Then $o_x(A')$
is the class of the map $J/J^2 \to I$ defined by sending $f \in J$ to
$f(a_1', \ldots, a'_n) \in I$. We omit the verification that
$o_x(A')$ is independent of choices. It is clear that if $o_x(A') = 0$
then the map lifts. Finally, functoriality is straightforward.
Thus we obtain an obstruction theory. We observe that $o_x(A')$
can be described a bit more canonically as the composition
$$
\NL_{R/\Lambda} \to \NL_{A/\Lambda} \to \NL_{A/A'} = I[1]
$$
in $D(A)$, see Algebra, Lemma \ref{algebra-lemma-NL-surjection}
for the last identification.
\end{example}









\section{Naive obstruction theories}
\label{section-naive-obstruction-theory}

\noindent
The title of this section refers to the fact that we will use the
naive cotangent complex in this section. Let $(x, A' \to A)$
be a deformation situation for a given category fibred in groupoids over a
locally Noetherian scheme $S$. The key Example \ref{example-key}
suggests that any obstruction theory should be closely related to
maps in $D(A)$ with target the naive cotangent complex of $A$.
Working this out we find a criterion for versality in
Lemma \ref{lemma-characterize-versal} which leads to a criterion for
openness of versality in Lemma \ref{lemma-openness}. We introduce a notion of
a naive obstruction theory in
Definition \ref{definition-naive-obstruction-theory} to try to formalize
the notion a bit further.

\medskip\noindent
In the following we will use the naive cotangent complex as
defined in Algebra, Section \ref{algebra-section-netherlander}.
In particular, if $A' \to A$ is a surjection of $\Lambda$-algebras
with square zero kernel $I$, then there are maps
$$
\NL_{A'/\Lambda} \to \NL_{A/\Lambda} \to \NL_{A/A'}
$$
whose composition is homotopy equivalent to zero (see
Algebra, Remark \ref{algebra-remark-composition-homotopy-equivalent-to-zero}).
This doesn't form a distinguished triangle in general as we are using
the naive cotangent complex and not the full one.
There is a homotopy equivalence $\NL_{A/A'} \to I[1]$ (the complex
consisting of $I$ placed in degree $-1$, see
Algebra, Lemma \ref{algebra-lemma-NL-surjection}).
Finally, note that there is a canonical map
$\NL_{A/\Lambda} \to \Omega_{A/\Lambda}$.

\begin{lemma}
\label{lemma-compute-ext-into-field}
Let $A \to k$ be a ring map with $k$ a field. Let $E \in D^-(A)$.
Then $\Ext^i_A(E, k) = \Hom_k(H^{-i}(E \otimes^\mathbf{L} k), k)$.
\end{lemma}

\begin{proof}
Omitted. Hint: Replace $E$ by a bounded above complex of free $A$-modules
and compute both sides.
\end{proof}

\begin{lemma}
\label{lemma-construct-essential-surjection}
Let $\Lambda \to A \to k$ be finite type ring maps of Noetherian rings with
$k = \kappa(\mathfrak p)$ for some prime $\mathfrak p$ of $A$. Let
$\xi : E \to \NL_{A/\Lambda}$ be morphism of $D^{-}(A)$ such that
$H^{-1}(\xi \otimes^{\mathbf{L}} k)$ is not surjective.
Then there exists a surjection $A' \to A$ of $\Lambda$-algebras
such that
\begin{enumerate}
\item[(a)] $I = \Ker(A' \to A)$ has square zero and is isomorphic to $k$
as an $A$-module,
\item[(b)] $\Omega_{A'/\Lambda} \otimes k = \Omega_{A/\Lambda} \otimes k$, and
\item[(c)] $E \to \NL_{A/A'}$ is zero.
\end{enumerate}
\end{lemma}

\begin{proof}
Let $f \in A$, $f \not \in \mathfrak p$. Suppose that $A'' \to A_f$
satisfies (a), (b), (c) for the induced map
$E \otimes_A A_f \to \NL_{A_f/\Lambda}$, see
Algebra, Lemma \ref{algebra-lemma-localize-NL}.
Then we can set $A' = A'' \times_{A_f} A$ and get a solution.
Namely, it is clear that $A' \to A$ satisfies (a) because
$\Ker(A' \to A) = \Ker(A'' \to A) = I$. Pick
$f'' \in A''$ lifting $f$. Then the localization of $A'$ at
$(f'', f)$ is isomorphic to $A''$
(for example by
More on Algebra, Lemma \ref{more-algebra-lemma-diagram-localize}).
Thus (b) and (c) are clear for $A'$ too.
In this way we see that we may replace $A$ by the localization
$A_f$ (finitely many times).
In particular (after such a replacement) we may assume that $\mathfrak p$
is a maximal ideal of $A$, see
Morphisms, Lemma \ref{morphisms-lemma-point-finite-type}.

\medskip\noindent
Choose a presentation $A = \Lambda[x_1, \ldots, x_n]/J$. Then
$\NL_{A/\Lambda}$ is (canonically) homotopy equivalent to
$$
J/J^2
\longrightarrow
\bigoplus\nolimits_{i = 1, \ldots, n} A\text{d}x_i,
$$
see Algebra, Lemma \ref{algebra-lemma-NL-homotopy}. After localizing
if necessary (using Nakayama's lemma) we can choose generators
$f_1, \ldots, f_m$ of $J$ such that $f_j \otimes 1$ form a basis for
$J/J^2 \otimes_A k$. Moreover, after renumbering, we can assume that the
images of $\text{d}f_1, \ldots, \text{d}f_r$ form a
basis for the image of $J/J^2 \otimes k \to \bigoplus k\text{d}x_i$
and that $\text{d}f_{r + 1}, \ldots, \text{d}f_m$ map to zero in
$\bigoplus k\text{d}x_i$. With these choices the space
$$
H^{-1}(\NL_{A/\Lambda} \otimes^{\mathbf{L}}_A k) =
H^{-1}(\NL_{A/\Lambda} \otimes_A k)
$$
has basis $f_{r + 1} \otimes 1, \ldots, f_m \otimes 1$. Changing basis
once again we may assume that the image of $H^{-1}(\xi \otimes^{\mathbf{L}} k)$
is contained in the $k$-span of
$f_{r + 1} \otimes 1, \ldots, f_{m - 1} \otimes 1$.
Set
$$
A' = \Lambda[x_1, \ldots, x_n]/(f_1, \ldots, f_{m - 1}, \mathfrak pf_m)
$$
By construction $A' \to A$ satisfies (a). Since $\text{d}f_m$ maps
to zero in $\bigoplus k\text{d}x_i$ we see that (b) holds. Finally, by
construction the induced map $E \to \NL_{A/A'} = I[1]$ induces the zero map
$H^{-1}(E \otimes_A^\mathbf{L} k) \to I \otimes_A k$. By
Lemma \ref{lemma-compute-ext-into-field}
we see that the composition is zero.
\end{proof}

\noindent
The following lemma is our key technical result.

\begin{lemma}
\label{lemma-characterize-versal}
Let $S$ be a locally Noetherian scheme. Let $\mathcal{X}$ be a category
fibred in groupoids over $(\Sch/S)_{fppf}$ satisfying (RS*).
Let $U = \Spec(A)$ be an
affine scheme of finite type over $S$ which maps into an affine open
$\Spec(\Lambda)$. Let $x$ be an object of $\mathcal{X}$ over $U$.
Let $\xi : E \to \NL_{A/\Lambda}$ be a morphism of $D^{-}(A)$. Assume
\begin{enumerate}
\item[(i)] for every deformation situation $(x, A' \to A)$ we have:
$x$ lifts to $\Spec(A')$ if and only if
$E \to \NL_{A/\Lambda} \to \NL_{A/A'}$ is zero, and
\item[(ii)] there is an isomorphism of functors
$T_x(-) \to \Ext^0_A(E, -)$
such that $E \to \NL_{A/\Lambda} \to \Omega^1_{A/\Lambda}$
corresponds to the canonical element (see
Remark \ref{remark-canonical-element}).
\end{enumerate}
Let $u_0 \in U$ be a finite type point with residue field
$k = \kappa(u_0)$. Consider the following statements
\begin{enumerate}
\item $x$ is versal at $u_0$, and
\item $\xi : E \to \NL_{A/\Lambda}$ induces a surjection
$H^{-1}(E \otimes_A^{\mathbf{L}} k) \to
H^{-1}(\NL_{A/\Lambda} \otimes_A^{\mathbf{L}} k)$
and an injection
$H^0(E \otimes_A^{\mathbf{L}} k) \to
H^0(\NL_{A/\Lambda} \otimes_A^{\mathbf{L}} k)$.
\end{enumerate}
Then we always have (2) $\Rightarrow$ (1) and we have (1) $\Rightarrow$ (2)
if $u_0$ is a closed point.
\end{lemma}

\begin{proof}
Let $\mathfrak p = \Ker(A \to k)$ be the prime corresponding to $u_0$.

\medskip\noindent
Assume that $x$ versal at $u_0$ and that $u_0$ is a closed point of $U$.
If $H^{-1}(\xi \otimes_A^{\mathbf{L}} k)$ is not surjective, then
let $A' \to A$ be an extension with kernel $I$ as in
Lemma \ref{lemma-construct-essential-surjection}.
Because $u_0$ is a closed point, we see that $I$ is a finite $A$-module,
hence that $A'$ is a finite type $\Lambda$-algebra (this fails if
$u_0$ is not closed). In particular $A'$ is Noetherian.
By property (c) for $A'$ and (i) for $\xi$ we see that $x$ lifts to
an object $x'$ over $A'$.
Let $\mathfrak p' \subset A'$ be kernel of the surjective map to $k$.
By Artin-Rees (Algebra, Lemma \ref{algebra-lemma-Artin-Rees})
there exists an $n > 1$ such that $(\mathfrak p')^n \cap I = 0$.
Then we see that
$$
B' = A'/(\mathfrak p')^n \longrightarrow A/\mathfrak p^n = B
$$
is a small, essential extension of local Artinian rings, see
Formal Deformation Theory, Lemma
\ref{formal-defos-lemma-essential-surjection}.
On the other hand, as $x$ is versal at $u_0$ and as $x'|_{\Spec(B')}$
is a lift of $x|_{\Spec(B)}$, there exists an integer
$m \geq n$ and a map $q : A/\mathfrak p^m \to B'$
such that the composition
$A/\mathfrak p^m \to B' \to B$ is the quotient map.
Since the maximal ideal of $B'$ has $n$th power equal to zero, this
$q$ factors through $B$ which contradicts the fact that $B' \to B$ is an
essential surjection. This contradiction shows that
$H^{-1}(\xi \otimes_A^{\mathbf{L}} k)$
is surjective.

\medskip\noindent
Assume that $x$ versal at $u_0$. By Lemma \ref{lemma-compute-ext-into-field}
the map $H^0(\xi \otimes_A^{\mathbf{L}} k)$ is dual to the map
$\Ext^0_A(\NL_{A/\Lambda}, k) \to \text{Ext}^0_A(E, k)$. Note that
$$
\Ext^0_A(\NL_{A/\Lambda}, k) = \text{Der}_\Lambda(A, k)
\quad\text{and}\quad
T_x(k) = \Ext^0_A(E, k)
$$
Condition (ii) assures us the map
$\Ext^0_A(\NL_{A/\Lambda}, k) \to \text{Ext}^0_A(E, k)$
sends a tangent vector $\theta$ to $U$ at $u_0$ to the corresponding
infinitesimal deformation of $x_0$, see Remark \ref{remark-canonical-element}.
Hence if $x$ is versal, then this map is surjective, see
Formal Deformation Theory, Lemma \ref{formal-defos-lemma-versal-criterion}.
Hence $H^0(\xi \otimes_A^{\mathbf{L}} k)$ is injective.
This finishes the proof of (1) $\Rightarrow$ (2) in case $u_0$ is a
closed point.

\medskip\noindent
For the rest of the proof assume $H^{-1}(E \otimes_A^\mathbf{L} k) \to
H^{-1}(\NL_{A/\Lambda} \otimes_A^\mathbf{L} k)$
is surjective and
$H^0(E \otimes_A^\mathbf{L} k) \to
H^0(\NL_{A/\Lambda} \otimes_A^\mathbf{L} k)$
injective. Set $R = A_\mathfrak p^\wedge$ and let $\eta$ be the
formal object over $R$ associated to $x|_{\Spec(R)}$.
The map $d\underline{\eta}$ on tangent spaces is surjective
because it is identified with the dual of the injective map
$H^0(E \otimes_A^{\mathbf{L}} k) \to
H^0(\NL_{A/\Lambda} \otimes_A^{\mathbf{L}} k)$
(see previous paragraph). According to
Formal Deformation Theory, Lemma \ref{formal-defos-lemma-versal-criterion}
it suffices to prove the following:
Let $C' \to C$ be a small extension of finite type Artinian local
$\Lambda$-algebras with residue field $k$. Let $R \to C$ be a
$\Lambda$-algebra map compatible with identifications of residue fields.
Let $y = x|_{\Spec(C)}$ and let $y'$ be a lift of $y$ to $C'$.
To show: we can lift the $\Lambda$-algebra map $R \to C$ to $R \to C'$.

\medskip\noindent
Observe that it suffices to lift the $\Lambda$-algebra map $A \to C$.
Let $I = \Ker(C' \to C)$. Note that $I$ is a $1$-dimensional $k$-vector
space. The obstruction $ob$ to lifting $A \to C$ is an element of
$\Ext^1_A(\NL_{A/\Lambda}, I)$, see Example \ref{example-key}.
By Lemma \ref{lemma-compute-ext-into-field} and our assumption the map
$\xi$ induces an injection
$$
\Ext^1_A(\NL_{A/\Lambda}, I)
\longrightarrow
\Ext^1_A(E, I)
$$
By the construction of $ob$ and (i) the image of $ob$ in $\Ext^1_A(E, I)$
is the obstruction to lifting $x$ to $A \times_C C'$. By (RS*) the fact that
$y/C$ lifts to $y'/C'$ implies that $x$ lifts to $A \times_C C'$. Hence
$ob = 0$ and we are done.
\end{proof}

\noindent
The key lemma above allows us to conclude that we have openness of
versality in some cases.

\begin{lemma}
\label{lemma-openness}
Let $S$ be a locally Noetherian scheme. Let $\mathcal{X}$ be a category
fibred in groupoids over $(\Sch/S)_{fppf}$ satisfying (RS*).
Let $U = \Spec(A)$ be an affine scheme of finite type over $S$ which maps
into an affine open $\Spec(\Lambda)$. Let $x$ be an object of $\mathcal{X}$
over $U$. Let $\xi : E \to \NL_{A/\Lambda}$ be a morphism of $D^{-}(A)$.
Assume
\begin{enumerate}
\item[(i)] for every deformation situation $(x, A' \to A)$ we have:
$x$ lifts to $\Spec(A')$ if and only if
$E \to \NL_{A/\Lambda} \to \NL_{A/A'}$ is zero,
\item[(ii)] there is an isomorphism of functors
$T_x(-) \to \Ext^0_A(E, -)$
such that $E \to \NL_{A/\Lambda} \to \Omega^1_{A/\Lambda}$
corresponds to the canonical element (see
Remark \ref{remark-canonical-element}),
\item[(iii)] the cohomology groups of $E$ are finite $A$-modules.
\end{enumerate}
If $x$ is versal at a closed point $u_0 \in U$,
then there exists an open neighbourhood $u_0 \in U' \subset U$
such that $x$ is versal at every finite type point of $U'$.
\end{lemma}

\begin{proof}
Let $C$ be the cone of $\xi$ so that we have a distinguished triangle
$$
E \to \NL_{A/\Lambda} \to C \to E[1]
$$
in $D^{-}(A)$. By Lemma \ref{lemma-characterize-versal}
the assumption that $x$ is versal at $u_0$ implies that
$H^{-1}(C \otimes^\mathbf{L} k) = 0$. By
More on Algebra, Lemma \ref{more-algebra-lemma-cut-complex-in-two}
there exists an $f \in A$ not contained in the prime corresponding to
$u_0$ such that $H^{-1}(C \otimes^\mathbf{L}_A M) = 0$ for
any $A_f$-module $M$. Using
Lemma \ref{lemma-characterize-versal}
again we see that we have versality for all finite type points of
the open $D(f) \subset U$.
\end{proof}

\noindent
The technical lemmas above suggest the following definition.

\begin{definition}
\label{definition-naive-obstruction-theory}
Let $S$ be a locally Noetherian base. Let $\mathcal{X}$ be a category fibred
in groupoids over $(\Sch/S)_{fppf}$. Assume that $\mathcal{X}$
satisfies (RS*). A {\it naive obstruction theory} is 
given by the following data
\begin{enumerate}
\item
\label{item-map}
for every $S$-algebra $A$ such that $\Spec(A) \to S$
maps into an affine open $\Spec(\Lambda) \subset S$ and every object $x$
of $\mathcal{X}$ over $\Spec(A)$ we are given an object $E_x \in D^-(A)$
and a map $\xi_x : E \to \NL_{A/\Lambda}$,
\item
\label{item-inf}
given $(x, A)$ as in (\ref{item-map}) there are transformations of
functors
$$
\text{Inf}_x( - ) \to \Ext^{-1}_A(E_x, -)
\quad\text{and}\quad
T_x(-) \to \Ext^0_A(E_x, -)
$$
\item
\label{item-functoriality}
for $(x, A)$ as in (\ref{item-map}) and a ring map $A \to B$
setting $y = x|_{\Spec(B)}$ there is a functoriality map
$E_x \to E_y$ in $D(A)$.
\end{enumerate}
These data are subject to the following conditions
\begin{enumerate}
\item[(i)]
in the situation of (\ref{item-functoriality}) the diagram
$$
\xymatrix{
E_y \ar[r]_{\xi_y} & \NL_{B/\Lambda} \\
E_x \ar[u] \ar[r]^{\xi_x} & \NL_{A/\Lambda} \ar[u]
}
$$
is commutative in $D(A)$,
\item[(ii)]
given $(x, A)$ as in (\ref{item-map}) and $A \to B \to C$
setting $y = x|_{\Spec(B)}$ and $z = x|_{\Spec(C)}$ the
composition of the functoriality maps $E_x \to E_y$ and $E_y \to E_z$ is
the functoriality map $E_x \to E_z$,
\item[(iii)]
the maps of (\ref{item-inf}) are isomorphisms
compatible with the functoriality
maps and the maps of Remark \ref{remark-functoriality},
\item[(iv)]
the composition $E_x \to \NL_{A/\Lambda} \to \Omega_{A/\Lambda}$
corresponds to the canonical element of
$T_x(\Omega_{A/\Lambda}) = \Ext^0(E_x, \Omega_{A/\Lambda})$, see
Remark \ref{remark-canonical-element},
\item[(v)]
given a deformation situation $(x, A' \to A)$ with $I = \Ker(A' \to A)$
the composition $E_x \to \NL_{A/\Lambda} \to \NL_{A/A'}$ is zero in
$$
\Hom_A(E_x, \NL_{A/\Lambda}) = \Ext^0_A(E_x, \NL_{A/A'}) =
\Ext^1_A(E_x, I)
$$
if and only if $x$ lifts to $A'$.
\end{enumerate}
\end{definition}

\noindent
Thus we see in particular that we obtain an obstruction theory
as in Section \ref{section-obstruction-theory} by setting
$\mathcal{O}_x( - ) = \Ext^1_A(E_x, -)$.

\begin{lemma}
\label{lemma-naive-obstruction-theory-qis}
Let $S$ and $\mathcal{X}$ be as in
Definition \ref{definition-naive-obstruction-theory}
and let $\mathcal{X}$ be endowed with a naive obstruction theory.
Let $A \to B$ and $y \to x$ be as in (\ref{item-functoriality}).
Let $k$ be a $B$-algebra which is a field. Then the functoriality
map $E_x \to E_y$ induces bijections
$$
H^i(E_x \otimes_A^{\mathbf{L}} k) \to H^i(E_y \otimes_B^{\mathbf{L}} k)
$$
for $i = 0, 1$.
\end{lemma}

\begin{proof}
Let $z = x|_{\Spec(k)}$. Then (RS*) implies that
$$
\textit{Lift}(x, A[k]) = \textit{Lift}(z, k[k])
\quad\text{and}\quad
\textit{Lift}(y, B[k]) = \textit{Lift}(z, k[k])
$$
because $A[k] = A \times_k k[k]$ and $B[k] = B \times_k k[k]$.
Hence the properties of a naive obstruction theory imply that the
functoriality map $E_x \to E_y$ induces bijections
$\Ext^i_A(E_x, k) \to \text{Ext}^i_B(E_y, k)$
for $i = -1, 0$. By Lemma \ref{lemma-compute-ext-into-field} our maps
$H^i(E_x \otimes_A^{\mathbf{L}} k) \to H^i(E_y \otimes_B^{\mathbf{L}} k)$,
$i = 0, 1$ induce isomorphisms on dual vector spaces hence are isomorphisms.
\end{proof}

\begin{lemma}
\label{lemma-naive-obstruction-theory-gives-openness}
Let $S$ be a locally Noetherian scheme. Let
$p : \mathcal{X} \to (\Sch/S)_{fppf}^{opp}$ be a category fibred in groupoids.
Assume that $\mathcal{X}$ satisfies (RS*)
and that $\mathcal{X}$ has a naive obstruction theory.
Then openness of versality holds for $\mathcal{X}$ provided the
complexes $E_x$ of Definition \ref{definition-naive-obstruction-theory}
have finitely generated cohomology groups for pairs $(A, x)$ where
$A$ is of finite type over $S$.
\end{lemma}

\begin{proof}
Let $U$ be a scheme locally of finite type over $S$, let $x$ be an object of
$\mathcal{X}$ over $U$, and let $u_0$ be a finite type point of $U$ such that
$x$ is versal at $u_0$. We may first shrink $U$ to an affine scheme such
that $u_0$ is a closed point and such that $U \to S$ maps into an affine
open $\Spec(\Lambda)$. Say $U = \Spec(A)$. Let
$\xi_x : E_x \to \NL_{A/\Lambda}$ be the obstruction map.
At this point we may apply Lemma \ref{lemma-openness} to conclude.
\end{proof}









\section{A dual notion}
\label{section-dual}

\noindent
Let $(x, A' \to A)$ be a deformation situation for a given category
$\mathcal{X}$ fibred in groupoids over a locally Noetherian scheme $S$.
Assume $\mathcal{X}$ has an obstruction theory, see
Definition \ref{definition-obstruction-theory}. In practice
one often has a complex $K^\bullet$ of $A$-modules and isomorphisms of
functors
$$
\text{Inf}_x(-) \to H^0(K^\bullet \otimes_A^\mathbf{L} -),\quad
T_x(-) \to H^1(K^\bullet \otimes_A^\mathbf{L} -),\quad
\mathcal{O}_x(-) \to H^2(K^\bullet \otimes_A^\mathbf{L} -)
$$
In this section we formalize this a little bit and show how this leads
to a verification of openness of versality in some cases.

\begin{example}
\label{example-global-sections-dual}
Let $\Lambda, S, W, \mathcal{F}$ be as in
Example \ref{example-global-sections}.
Assume that $W \to S$ is proper and $\mathcal{F}$ coherent. By
Cohomology of Schemes, Remark
\ref{coherent-remark-explain-perfect-direct-image}
there exists a finite complex of finite projective $\Lambda$-modules
$N^\bullet$ which universally computes the cohomology of $\mathcal{F}$.
In particular the obstruction spaces from Example \ref{example-global-sections}
are $\mathcal{O}_x(M) = H^1(N^\bullet \otimes_\Lambda M)$.
Hence with $K^\bullet = N^\bullet \otimes_\Lambda A[-1]$ we see that
$\mathcal{O}_x(M) = H^2(K^\bullet \otimes_A^\mathbf{L} M)$.
\end{example}

\begin{situation}
\label{situation-dual}
Let $S$ be a locally Noetherian scheme. Let $\mathcal{X}$ be a category
fibred in groupoids over $(\Sch/S)_{fppf}$. Assume that
$\mathcal{X}$ has (RS*) so that we can speak of the functor $T_x(-)$, see
Lemma \ref{lemma-properties-lift-RS-star}.
Let $U = \Spec(A)$ be an affine scheme of finite type over $S$ which maps
into an affine open $\Spec(\Lambda)$. Let $x$ be an object of $\mathcal{X}$
over $U$. Assume we are given
\begin{enumerate}
\item a complex of $A$-modules $K^\bullet$,
\item a transformation of functors
$T_x(-) \to H^1(K^\bullet \otimes_A^\mathbf{L} -)$,
\item for every deformation situation $(x, A' \to A)$ with kernel
$I = \Ker(A' \to A)$ an element
$o_x(A') \in H^2(K^\bullet \otimes_A^\mathbf{L} I)$
\end{enumerate}
satisfying the following (minimal) conditions
\begin{enumerate}
\item[(i)] the transformation
$T_x(-) \to H^1(K^\bullet \otimes_A^\mathbf{L} -)$
is an isomorphism,
\item[(ii)] given a morphism $(x, A'' \to A) \to (x, A' \to A)$ of deformation
situations the element $o_x(A')$ maps to the element $o_x(A'')$
via the map
$H^2(K^\bullet \otimes_A^\mathbf{L} I) \to
H^2(K^\bullet \otimes_A^\mathbf{L} I')$
where $I' = \Ker(A'' \to A)$, and
\item[(iii)] $x$ lifts to an object over $\Spec(A')$ if and only if
$o_x(A') = 0$.
\end{enumerate}
It is possible to incorporate infinitesimal automorphisms as well, but
we refrain from doing so in order to get the sharpest possible result.
\end{situation}

\noindent
In Situation \ref{situation-dual} an important role will be played by
$K^\bullet \otimes_A^\mathbf{L} \NL_{A/\Lambda}$. Suppose we are given an
element $\xi \in H^1(K^\bullet \otimes_A^\mathbf{L} \NL_{A/\Lambda})$.
Then (1) for any surjection $A' \to A$ of $\Lambda$-algebras with kernel
$I$ of square zero the canonical map $\NL_{A/\Lambda} \to \NL_{A/A'} = I[1]$
sends $\xi$ to an element $\xi_{A'} \in H^2(K^\bullet \otimes_A^\mathbf{L} I)$
and (2) the map $\NL_{A/\Lambda} \to \Omega_{A/\Lambda}$ sends
$\xi$ to an element $\xi_{can}$ of
$H^1(K^\bullet \otimes_A^\mathbf{L} \Omega_{A/\Lambda})$.

\begin{lemma}
\label{lemma-dual-obstruction}
In Situation \ref{situation-dual}. Assume furthermore that
\begin{enumerate}
\item[(iv)] given a short exact sequence of deformation situations
as in Remark \ref{remark-short-exact-sequence-thickenings} and
a lift $x'_2 \in \text{Lift}(x, A_2')$ then
$o_x(A_3') \in H^2(K^\bullet \otimes_A^\mathbf{L} I_3)$
equals $\partial\theta$ where
$\theta \in H^1(K^\bullet \otimes_A^\mathbf{L} I_1)$
is the element corresponding to $x'_2|_{\Spec(A_1')}$ via
$A_1' = A[I_1]$ and the given map
$T_x(-) \to H^1(K^\bullet \otimes_A^\mathbf{L} -)$.
\end{enumerate}
In this case there exists an element
$\xi \in H^1(K^\bullet \otimes_A^\mathbf{L} \NL_{A/\Lambda})$
such that
\begin{enumerate}
\item for every deformation situation $(x, A' \to A)$ we have
$\xi_{A'} = o_x(A')$, and
\item $\xi_{can}$ matches the canonical element of
Remark \ref{remark-canonical-element} via the given transformation
$T_x(-) \to H^1(K^\bullet \otimes_A^\mathbf{L} -)$.
\end{enumerate}
\end{lemma}

\begin{proof}
Choose a $\alpha : \Lambda[x_1, \ldots, x_n] \to A$ with kernel $J$.
Write $P = \Lambda[x_1, \ldots, x_n]$. In the rest of this proof we work with
$$
\NL(\alpha) = (J/J^2 \longrightarrow \bigoplus A \text{d}x_i)
$$
which is permissible by
Algebra, Lemma \ref{algebra-lemma-NL-homotopy}
and
More on Algebra, Lemma \ref{more-algebra-lemma-derived-tor-homotopy}.
Consider the element
$o_x(P/J^2) \in H^2(K^\bullet \otimes_A^\mathbf{L} J/J^2)$ and consider
the quotient
$$
C = (P/J^2 \times \bigoplus A \text{d}x_i)/(J/J^2)
$$
where $J/J^2$ is embedded diagonally. Note that $C \to A$ is a surjection
with kernel $\bigoplus A\text{d}x_i$. Moreover there is a section
$A \to C$ to $C \to A$ given by mapping the class of $f \in P$ to the class
of $(f, \text{d}f)$ in the pushout. For later use, denote $x_C$ the
pullback of $x$ along the corresponding morphism $\Spec(C) \to \Spec(A)$.
Thus we see that $o_x(C) = 0$.
We conclude that $o_x(P/J^2)$ maps to zero in
$H^2(K^\bullet \otimes_A^\mathbf{L} \bigoplus A\text{d}x_i)$.
It follows that there exists some element
$\xi \in H^1(K^\bullet \otimes_A^\mathbf{L} \NL(\alpha))$
mapping to $o_x(P/J^2)$.

\medskip\noindent
Note that for any deformation situation $(x, A' \to A)$ there exists
a $\Lambda$-algebra map $P/J^2 \to A'$ compatible with the augmentations
to $A$. Hence the
element $\xi$ satisfies the first property of the lemma by construction
and property (ii) of Situation \ref{situation-dual}.

\medskip\noindent
Note that our choice of $\xi$ was well defined up to the choice of an
element of $H^1(K^\bullet \otimes_A^\mathbf{L} \bigoplus A\text{d}x_i)$.
We will show that after modifying $\xi$ by an element of the aforementioned
group we can arrange it so that the second assertion of the lemma is true.
Let $C' \subset C$ be the image of $P/J^2$ under the
$\Lambda$-algebra map $P/J^2 \to C$ (inclusion of first factor).
Observe that
$\Ker(C' \to A) = \Im(J/J^2 \to \bigoplus A\text{d}x_i)$.
Set $\overline{C} = A[\Omega_{A/\Lambda}]$. The map
$P/J^2 \times \bigoplus A \text{d}x_i \to \overline{C}$,
$(f, \sum f_i \text{d}x_i) \mapsto (f \bmod J, \sum f_i \text{d}x_i)$
factors through a surjective map $C \to \overline{C}$. Then
$$
(x, \overline{C} \to A) \to (x, C \to A) \to (x, C' \to A)
$$
is a short exact sequence of deformation situations. The
associated splitting $\overline{C} = A[\Omega_{A/\Lambda}]$ (from
Remark \ref{remark-short-exact-sequence-thickenings}) equals the given
splitting above. Moreover, the section $A \to C$ composed with the map
$C \to \overline{C}$
is the map $(1, \text{d}) : A \to A[\Omega_{A/\Lambda}]$ of
Remark \ref{remark-canonical-element}.
Thus $x_C$ restricts to the canonical element $x_{can}$ of
$T_x(\Omega_{A/\Lambda}) = \text{Lift}(x, A[\Omega_{A/\Lambda}])$.
By condition (iv) we conclude that $o_x(P/J^2)$ maps to $\partial x_{can}$
in
$$
H^1(K^\bullet \otimes_A^\mathbf{L} \Im(J/J^2 \to \bigoplus A\text{d}x_i))
$$
By construction $\xi$ maps to $o_x(P/J^2)$. It follows that
$x_{can}$ and $\xi_{can}$ map to the same element in the
displayed group which means (by the long exact cohomology sequence)
that they differ by an element of
$H^1(K^\bullet \otimes_A^\mathbf{L} \bigoplus A\text{d}x_i)$
as desired.
\end{proof}

\begin{lemma}
\label{lemma-dual-openness}
In Situation \ref{situation-dual} assume that (iv) of
Lemma \ref{lemma-dual-obstruction} holds and that $K^\bullet$ is a
perfect object of $D(A)$. In this case, if $x$ is versal at a closed
point $u_0 \in U$ then there exists an open neighbourhood
$u_0 \in U' \subset U$ such that $x$ is versal at every finite type
point of $U'$.
\end{lemma}

\begin{proof}
We may assume that $K^\bullet$ is a finite complex of finite projective
$A$-modules. Thus the derived tensor product with $K^\bullet$ is the
same as simply tensoring with $K^\bullet$. Let
$E^\bullet$ be the dual perfect complex to $K^\bullet$, see
More on Algebra, Lemma \ref{more-algebra-lemma-dual-perfect-complex}.
(So $E^n = \Hom_A(K^{-n}, A)$ with differentials the transpose of the
differentials of $K^\bullet$.) Let $E \in D^{-}(A)$ denote the
object represented by the complex $E^\bullet[-1]$.
Let $\xi \in H^1(\text{Tot}(K^\bullet \otimes_A \NL_{A/\Lambda}))$
be the element constructed in Lemma \ref{lemma-dual-obstruction}
and denote $\xi : E = E^\bullet[-1] \to \NL_{A/\Lambda}$ the corresponding
map (loc.cit.). We claim that the pair $(E, \xi)$ satisfies all the
assumptions of Lemma \ref{lemma-openness} which finishes the proof.

\medskip\noindent
Namely, assumption (i) of Lemma \ref{lemma-openness} follows from conclusion
(1) of Lemma \ref{lemma-dual-obstruction}
and the fact that $H^2(K^\bullet \otimes_A^\mathbf{L} -) =
\Ext^1(E, -)$ by loc.cit. Assumption (ii) of
Lemma \ref{lemma-openness} follows from conclusion (2) of
Lemma \ref{lemma-dual-obstruction}
and the fact that $H^1(K^\bullet \otimes_A^\mathbf{L} -) =
\Ext^0(E, -)$ by loc.cit. Assumption (iii) of Lemma \ref{lemma-openness}
is clear.
\end{proof}
















\begin{multicols}{2}[\section{Other chapters}]
\noindent
Preliminaries
\begin{enumerate}
\item \hyperref[introduction-section-phantom]{Introduction}
\item \hyperref[conventions-section-phantom]{Conventions}
\item \hyperref[sets-section-phantom]{Set Theory}
\item \hyperref[categories-section-phantom]{Categories}
\item \hyperref[topology-section-phantom]{Topology}
\item \hyperref[sheaves-section-phantom]{Sheaves on Spaces}
\item \hyperref[sites-section-phantom]{Sites and Sheaves}
\item \hyperref[stacks-section-phantom]{Stacks}
\item \hyperref[fields-section-phantom]{Fields}
\item \hyperref[algebra-section-phantom]{Commutative Algebra}
\item \hyperref[brauer-section-phantom]{Brauer Groups}
\item \hyperref[homology-section-phantom]{Homological Algebra}
\item \hyperref[derived-section-phantom]{Derived Categories}
\item \hyperref[simplicial-section-phantom]{Simplicial Methods}
\item \hyperref[more-algebra-section-phantom]{More on Algebra}
\item \hyperref[smoothing-section-phantom]{Smoothing Ring Maps}
\item \hyperref[modules-section-phantom]{Sheaves of Modules}
\item \hyperref[sites-modules-section-phantom]{Modules on Sites}
\item \hyperref[injectives-section-phantom]{Injectives}
\item \hyperref[cohomology-section-phantom]{Cohomology of Sheaves}
\item \hyperref[sites-cohomology-section-phantom]{Cohomology on Sites}
\item \hyperref[dga-section-phantom]{Differential Graded Algebra}
\item \hyperref[dpa-section-phantom]{Divided Power Algebra}
\item \hyperref[sdga-section-phantom]{Differential Graded Sheaves}
\item \hyperref[hypercovering-section-phantom]{Hypercoverings}
\end{enumerate}
Schemes
\begin{enumerate}
\setcounter{enumi}{25}
\item \hyperref[schemes-section-phantom]{Schemes}
\item \hyperref[constructions-section-phantom]{Constructions of Schemes}
\item \hyperref[properties-section-phantom]{Properties of Schemes}
\item \hyperref[morphisms-section-phantom]{Morphisms of Schemes}
\item \hyperref[coherent-section-phantom]{Cohomology of Schemes}
\item \hyperref[divisors-section-phantom]{Divisors}
\item \hyperref[limits-section-phantom]{Limits of Schemes}
\item \hyperref[varieties-section-phantom]{Varieties}
\item \hyperref[topologies-section-phantom]{Topologies on Schemes}
\item \hyperref[descent-section-phantom]{Descent}
\item \hyperref[perfect-section-phantom]{Derived Categories of Schemes}
\item \hyperref[more-morphisms-section-phantom]{More on Morphisms}
\item \hyperref[flat-section-phantom]{More on Flatness}
\item \hyperref[groupoids-section-phantom]{Groupoid Schemes}
\item \hyperref[more-groupoids-section-phantom]{More on Groupoid Schemes}
\item \hyperref[etale-section-phantom]{\'Etale Morphisms of Schemes}
\end{enumerate}
Topics in Scheme Theory
\begin{enumerate}
\setcounter{enumi}{41}
\item \hyperref[chow-section-phantom]{Chow Homology}
\item \hyperref[intersection-section-phantom]{Intersection Theory}
\item \hyperref[pic-section-phantom]{Picard Schemes of Curves}
\item \hyperref[weil-section-phantom]{Weil Cohomology Theories}
\item \hyperref[adequate-section-phantom]{Adequate Modules}
\item \hyperref[dualizing-section-phantom]{Dualizing Complexes}
\item \hyperref[duality-section-phantom]{Duality for Schemes}
\item \hyperref[discriminant-section-phantom]{Discriminants and Differents}
\item \hyperref[derham-section-phantom]{de Rham Cohomology}
\item \hyperref[local-cohomology-section-phantom]{Local Cohomology}
\item \hyperref[algebraization-section-phantom]{Algebraic and Formal Geometry}
\item \hyperref[curves-section-phantom]{Algebraic Curves}
\item \hyperref[resolve-section-phantom]{Resolution of Surfaces}
\item \hyperref[models-section-phantom]{Semistable Reduction}
\item \hyperref[functors-section-phantom]{Functors and Morphisms}
\item \hyperref[equiv-section-phantom]{Derived Categories of Varieties}
\item \hyperref[pione-section-phantom]{Fundamental Groups of Schemes}
\item \hyperref[etale-cohomology-section-phantom]{\'Etale Cohomology}
\item \hyperref[crystalline-section-phantom]{Crystalline Cohomology}
\item \hyperref[proetale-section-phantom]{Pro-\'etale Cohomology}
\item \hyperref[relative-cycles-section-phantom]{Relative Cycles}
\item \hyperref[more-etale-section-phantom]{More \'Etale Cohomology}
\item \hyperref[trace-section-phantom]{The Trace Formula}
\end{enumerate}
Algebraic Spaces
\begin{enumerate}
\setcounter{enumi}{64}
\item \hyperref[spaces-section-phantom]{Algebraic Spaces}
\item \hyperref[spaces-properties-section-phantom]{Properties of Algebraic Spaces}
\item \hyperref[spaces-morphisms-section-phantom]{Morphisms of Algebraic Spaces}
\item \hyperref[decent-spaces-section-phantom]{Decent Algebraic Spaces}
\item \hyperref[spaces-cohomology-section-phantom]{Cohomology of Algebraic Spaces}
\item \hyperref[spaces-limits-section-phantom]{Limits of Algebraic Spaces}
\item \hyperref[spaces-divisors-section-phantom]{Divisors on Algebraic Spaces}
\item \hyperref[spaces-over-fields-section-phantom]{Algebraic Spaces over Fields}
\item \hyperref[spaces-topologies-section-phantom]{Topologies on Algebraic Spaces}
\item \hyperref[spaces-descent-section-phantom]{Descent and Algebraic Spaces}
\item \hyperref[spaces-perfect-section-phantom]{Derived Categories of Spaces}
\item \hyperref[spaces-more-morphisms-section-phantom]{More on Morphisms of Spaces}
\item \hyperref[spaces-flat-section-phantom]{Flatness on Algebraic Spaces}
\item \hyperref[spaces-groupoids-section-phantom]{Groupoids in Algebraic Spaces}
\item \hyperref[spaces-more-groupoids-section-phantom]{More on Groupoids in Spaces}
\item \hyperref[bootstrap-section-phantom]{Bootstrap}
\item \hyperref[spaces-pushouts-section-phantom]{Pushouts of Algebraic Spaces}
\end{enumerate}
Topics in Geometry
\begin{enumerate}
\setcounter{enumi}{81}
\item \hyperref[spaces-chow-section-phantom]{Chow Groups of Spaces}
\item \hyperref[groupoids-quotients-section-phantom]{Quotients of Groupoids}
\item \hyperref[spaces-more-cohomology-section-phantom]{More on Cohomology of Spaces}
\item \hyperref[spaces-simplicial-section-phantom]{Simplicial Spaces}
\item \hyperref[spaces-duality-section-phantom]{Duality for Spaces}
\item \hyperref[formal-spaces-section-phantom]{Formal Algebraic Spaces}
\item \hyperref[restricted-section-phantom]{Algebraization of Formal Spaces}
\item \hyperref[spaces-resolve-section-phantom]{Resolution of Surfaces Revisited}
\end{enumerate}
Deformation Theory
\begin{enumerate}
\setcounter{enumi}{89}
\item \hyperref[formal-defos-section-phantom]{Formal Deformation Theory}
\item \hyperref[defos-section-phantom]{Deformation Theory}
\item \hyperref[cotangent-section-phantom]{The Cotangent Complex}
\item \hyperref[examples-defos-section-phantom]{Deformation Problems}
\end{enumerate}
Algebraic Stacks
\begin{enumerate}
\setcounter{enumi}{93}
\item \hyperref[algebraic-section-phantom]{Algebraic Stacks}
\item \hyperref[examples-stacks-section-phantom]{Examples of Stacks}
\item \hyperref[stacks-sheaves-section-phantom]{Sheaves on Algebraic Stacks}
\item \hyperref[criteria-section-phantom]{Criteria for Representability}
\item \hyperref[artin-section-phantom]{Artin's Axioms}
\item \hyperref[quot-section-phantom]{Quot and Hilbert Spaces}
\item \hyperref[stacks-properties-section-phantom]{Properties of Algebraic Stacks}
\item \hyperref[stacks-morphisms-section-phantom]{Morphisms of Algebraic Stacks}
\item \hyperref[stacks-limits-section-phantom]{Limits of Algebraic Stacks}
\item \hyperref[stacks-cohomology-section-phantom]{Cohomology of Algebraic Stacks}
\item \hyperref[stacks-perfect-section-phantom]{Derived Categories of Stacks}
\item \hyperref[stacks-introduction-section-phantom]{Introducing Algebraic Stacks}
\item \hyperref[stacks-more-morphisms-section-phantom]{More on Morphisms of Stacks}
\item \hyperref[stacks-geometry-section-phantom]{The Geometry of Stacks}
\end{enumerate}
Topics in Moduli Theory
\begin{enumerate}
\setcounter{enumi}{107}
\item \hyperref[moduli-section-phantom]{Moduli Stacks}
\item \hyperref[moduli-curves-section-phantom]{Moduli of Curves}
\end{enumerate}
Miscellany
\begin{enumerate}
\setcounter{enumi}{109}
\item \hyperref[examples-section-phantom]{Examples}
\item \hyperref[exercises-section-phantom]{Exercises}
\item \hyperref[guide-section-phantom]{Guide to Literature}
\item \hyperref[desirables-section-phantom]{Desirables}
\item \hyperref[coding-section-phantom]{Coding Style}
\item \hyperref[obsolete-section-phantom]{Obsolete}
\item \hyperref[fdl-section-phantom]{GNU Free Documentation License}
\item \hyperref[index-section-phantom]{Auto Generated Index}
\end{enumerate}
\end{multicols}


\bibliography{my}
\bibliographystyle{amsalpha}

\end{document}
